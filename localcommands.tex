%add all your local new commands to this file


% Don't do this at home. I do not like the smaller font for captions.
% I just removed loading the caption packege in langscibook.cls
%% \captionsetup{%
%% font={%
%% stretch=1%.8%
%% ,normalsize%,small%
%% },%
%% width=.8\textwidth
%% }

\makeatletter
\def\blx@maxline{77}
\makeatother


\let\citew\citet

\newcommand{\page}{}

\newcommand{\todostefan}[1]{\todo[color=orange!80]{\footnotesize #1}\xspace}
\newcommand{\todosatz}[1]{\todo[color=red!40]{\footnotesize #1}\xspace}

\newcommand{\inlinetodostefan}[1]{\todo[color=green!40,inline]{\footnotesize #1}\xspace}

\newcommand{\inlinetodoopt}[1]{\todo[color=green!40,inline]{\footnotesize #1}\xspace}
\newcommand{\inlinetodoobl}[1]{\todo[color=red!40,inline]{\footnotesize #1}\xspace}

\newcommand{\itd}[1]{\inlinetodoobl{#1}}
\newcommand{\itdobl}[1]{\inlinetodoobl{#1}}
\newcommand{\itdopt}[1]{\inlinetodoopt{#1}}

\newcommand{\itdsecond}[1]{}

\newcommand{\itddone}[1]{}
\newcommand{\LATER}[1]{}



% A. Red: Simple typos, errors in the AVMs (only a couple) to take care of on the editorial side, no need to contact the authors
% B.: Green: Wording changes which do not necessarily require authors’ approval, but are not just typos/errors
% C.: Blue: Comments to the author that they don’t have to take care of, but after all, the authors might be interested to have the comments for future revisions. 
% D.: Purple: Comments to the editors about something we need to keep in mind or do. Nothing for you

\newcommand{\colorcodingexplanation}{\todo[color=green!40,inline]{%
Explanation of colors of bubbles and text:\\
A.: Red: Things that have to be fixed/commented upon.\\
B.: Green: optional comments\\
C.: Blue: Comments to the author that they don’t have to take care of, but after all, the authors
might be interested to have the comments for future revisions.\\
Explanation of colors of text:\\
Red: newly added material (crossreferences to other chapters and other references)\\
Orange: changed material, please check\\
Blue: suggestions for deletion\\
Please also check margin notes.
}}
% D.: Purple: Comments to the editors about something we need to keep in mind or do. Nothing for you


\newcommand{\itdgreen}[1]{\todo[color=green!40,inline]{\footnotesize #1}\xspace}
\newcommand{\itdblue}[1]{\todo[color=blue!40,inline]{\footnotesize #1}\xspace}

% for editing, remove later
\usepackage{xcolor}
\newcommand{\added}[1]{{\red #1}}
\newcommand{\addedthis}{\todostefan{added this}}

\newcommand{\changed}[1]{\textcolor{orange}{#1}}
\newcommand{\deleted}[1]{\textcolor{blue}{#1}}


\newcommand{\addpages}{\todostefan{add pages}}
%\newcommand{\iaddpages}{\inlinetodoobl{add pages}}
\newcommand{\iaddpages}{\yel[add pages]{pages}\xspace}
\newcommand{\addref}{\todostefan{add reference}}
\newcommand{\inlineaddpages}{\inlinetodostefan{add pages}}
\newcommand{\addglosses}{\todostefan{add glosses}}


%\newcommand{\spacebr}{\hphantom{[}}

\newcommand{\danishep}{\jambox{(\ili{Danish})}}
\newcommand{\english}{\jambox{(\ili{English})}}
\newcommand{\german}{\jambox{(\ili{German})}}
\newcommand{\yiddish}{\jambox{(\ili{Yiddish})}}
\newcommand{\welsh}{\jambox{(\ili{Welsh})}}

% Cite and cross-reference other chapters
\newcommand{\crossrefchaptert}[2][]{\citet*[#1]{chapters/#2}, Chapter~\ref{chap-#2} of this volume} 
\newcommand{\crossrefchapterp}[2][]{(\citealp*[#1]{chapters/#2}, Chapter~\ref{chap-#2} of this volume)}
\newcommand{\crossrefchapteralt}[2][]{\citealt*[#1]{chapters/#2}, Chapter~\ref{chap-#2} of this volume}
\newcommand{\crossrefchapteralp}[2][]{\citealp*[#1]{chapters/#2}, Chapter~\ref{chap-#2} of this volume}

\newcommand{\crossrefcitet}[2][]{\citet*[#1]{chapters/#2}} 
\newcommand{\crossrefcitep}[2][]{\citep*[#1]{chapters/#2}}
\newcommand{\crossrefcitealt}[2][]{\citealt*[#1]{chapters/#2}}
\newcommand{\crossrefcitealp}[2][]{\citealp*[#1]{chapters/#2}}


% example of optional argument:
% \crossrefchapterp[for something, see:]{name}
% gives: (for something, see: Author 2018, Chapter~X of this volume)



\let\crossrefchapterw\crossrefchaptert



% Davis Koenig

\let\ig=\textsc
\let\tc=\textcolor

% evolution, Flickinger, Pollard, Wasow

\let\citeNP\citet

% Adam P

%\newcommand{\toappear}{Forthcoming}
\newcommand{\pg}[1]{p.\,#1}
\renewcommand{\implies}{\rightarrow}

\newcommand*{\rref}[1]{(\ref{#1})}
\newcommand*{\aref}[1]{(\ref{#1}a)}
\newcommand*{\bref}[1]{(\ref{#1}b)}
\newcommand*{\cref}[1]{(\ref{#1}c)}

\newcommand{\msadam}{.}
\newcommand{\morsyn}[1]{\textsc{#1}}

\newcommand{\aux}{\textsc{aux}\xspace}

\newcommand{\nom}{\morsyn{nom}}
\newcommand{\acc}{\morsyn{acc}}
\newcommand{\dat}{\morsyn{dat}}
\newcommand{\gen}{\morsyn{gen}}
\newcommand{\ins}{\morsyn{ins}}
%\newcommand{\aploc}{\morsyn{loc}}
\newcommand{\voc}{\morsyn{voc}}
\newcommand{\ill}{\morsyn{ill}}
\renewcommand{\inf}{\morsyn{inf}}
\newcommand{\passprc}{\morsyn{passp}}

%\newcommand{\Nom}{\msadam\nom}
%\newcommand{\Acc}{\msadam\acc}
%\newcommand{\Dat}{\msadam\dat}
%\newcommand{\Gen}{\msadam\gen}
\newcommand{\Ins}{\msadam\ins}
\newcommand{\Loc}{\msadam\loc}
\newcommand{\Voc}{\msadam\voc}
\newcommand{\Ill}{\msadam\ill}
\newcommand{\PassP}{\msadam\passprc}

\newcommand{\Aux}{\textsc{aux}}

\newcommand{\princ}[1]{\textnormal{\textsc{#1}}} % for constraint names
\newcommand{\notion}[1]{\emph{#1}}
\renewcommand{\path}[1]{\textnormal{\textsc{#1}}}
\newcommand{\ftype}[1]{\textit{#1}}
\newcommand{\fftype}[1]{{\scriptsize\textit{#1}}}
\newcommand{\la}{$\langle$}
\newcommand{\ra}{$\rangle$}
%\newcommand{\argst}{\path{arg-st}}
\newcommand{\phtm}[1]{\setbox0=\hbox{#1}\hspace{\wd0}}
\newcommand{\prep}[1]{\setbox0=\hbox{#1}\hspace{-1\wd0}#1}

%%%%%%%%%%%%%%%%%%%%%%%%%%%%%%%%%%%%%%%%%%%%%%%%%%%%%%%%%%%%%%%%%%%%%%%%%%%

% FROM FS.STY:

%%%
%%% Feature structures
%%%

% \fs         To print a feature structure by itself, type for example
%             \fs{case:nom \\ person:P}
%             or (better, for true italics),
%             \fs{\it case:nom \\ \it person:P}
%
% \lfs        To print the same feature structure with the category
%             label N at the top, type:
%             \lfs{N}{\it case:nom \\ \it person:P}

%    Modified 1990 Dec 5 so that features are left aligned.
\newcommand{\fs}[1]%
{\mbox{\small%
$
\!
\left[
  \!\!
  \begin{tabular}{l}
    #1
  \end{tabular}
  \!\!
\right]
\!
$}}

%     Modified 1990 Dec 5 so that features are left aligned.
%\newcommand{\lfs}[2]
%   {
%     \mbox{$
%           \!\!
%           \begin{tabular}{c}
%           \it #1
%           \\
%           \mbox{\small%
%                 $
%                 \left[
%                 \!\!
%                 \it
%                 \begin{tabular}{l}
%                 #2
%                 \end{tabular}
%                 \!\!
%                 \right]
%                 $}
%           \end{tabular}
%           \!\!
%           $}
%   }

\newcommand{\ft}[2]{\path{#1}\hspace{1ex}\ftype{#2}}
\newcommand{\fsl}[2]{\fs{{\fftype{#1}} \\ #2}}

\newcommand{\fslt}[2]
 {\fst{
       {\fftype{#1}} \\
       #2 
     }
 }

\newcommand{\fsltt}[2]
 {\fstt{
       {\fftype{#1}} \\
       #2 
     }
 }

\newcommand{\fslttt}[2]
 {\fsttt{
       {\fftype{#1}} \\
       #2 
     }
 }


% jak \ft, \fs i \fsl tylko nieco ciasniejsze

\newcommand{\ftt}[2]
% {{\sc #1}\/{\rm #2}}
 {\textsc{#1}\/{\rm #2}}

\newcommand{\fst}[1]
  {
    \mbox{\small%
          $
          \left[
          \!\!\!
%          \sc
          \begin{tabular}{l} #1
          \end{tabular}
          \!\!\!\!\!\!\!
          \right]
          $
          }
   }

%\newcommand{\fslt}[2]
% {\fst{#2\\
%       {\scriptsize\it #1}
%      }
% }


% superciasne

\newcommand{\fstt}[1]
  {
    \mbox{\small%
          $
          \left[
          \!\!\!
%          \sc
          \begin{tabular}{l} #1
          \end{tabular}
          \!\!\!\!\!\!\!\!\!\!\!
          \right]
          $
          }
   }

%\newcommand{\fsltt}[2]
% {\fstt{#2\\
%       {\scriptsize\it #1}
%      }
% }

\newcommand{\fsttt}[1]
  {
    \mbox{\small%
          $
          \left[
          \!\!\!
%          \sc
          \begin{tabular}{l} #1
          \end{tabular}
          \!\!\!\!\!\!\!\!\!\!\!\!\!\!\!\!
          \right]
          $
          }
   }



% %add all your local new commands to this file

% \newcommand{\smiley}{:)}

% you are not supposed to mess with hardcore stuff, St.Mü. 22.08.2018
%% \renewbibmacro*{index:name}[5]{%
%%   \usebibmacro{index:entry}{#1}
%%     {\iffieldundef{usera}{}{\thefield{usera}\actualoperator}\mkbibindexname{#2}{#3}{#4}{#5}}}

% % \newcommand{\noop}[1]{}



% Rui

\newcommand{\spc}[0]{\hspace{-1pt}\underline{\hspace{6pt}}\,}
\newcommand{\spcs}[0]{\hspace{-1pt}\underline{\hspace{6pt}}\,\,}
\newcommand{\bad}[1]{\leavevmode\llap{#1}}
\newcommand{\COMMENT}[1]{}


% Rui coordination
\newcommand{\subl}[1]{$_{\scriptstyle \textsc{#1}}$}



% Andy Lücking gesture.tex
\newcommand{\Pointing}{\ding{43}}
% Giotto: "Meeting of Joachim and Anne at the Golden Gate" - 1305-10 
\definecolor{GoldenGate1}{rgb}{.13,.09,.13} % Dress of woman in black
\definecolor{GoldenGate2}{rgb}{.94,.94,.91} % Bridge
\definecolor{GoldenGate3}{rgb}{.06,.09,.22} % Blue sky
\definecolor{GoldenGate4}{rgb}{.94,.91,.87} % Dress of woman with shawl
\definecolor{GoldenGate5}{rgb}{.52,.26,.26} % Joachim's robe
\definecolor{GoldenGate6}{rgb}{.65,.35,.16} % Anne's robe
\definecolor{GoldenGate7}{rgb}{.91,.84,.42} % Joachim's halo

\makeatletter
\newcommand{\@Depth}{1} % x-dimension, to front
\newcommand{\@Height}{1} % z-dimension, up
\newcommand{\@Width}{1} % y-dimension, rightwards
%\GGS{<x-start>}{<y-start>}{<z-top>}{<z-bottom>}{<Farbe>}{<x-width>}{<y-depth>}{<opacity>}
\newcommand{\GGS}[9][]{%
\coordinate (O) at (#2-1,#3-1,#5);
\coordinate (A) at (#2-1,#3-1+#7,#5);
\coordinate (B) at (#2-1,#3-1+#7,#4);
\coordinate (C) at (#2-1,#3-1,#4);
\coordinate (D) at (#2-1+#8,#3-1,#5);
\coordinate (E) at (#2-1+#8,#3-1+#7,#5);
\coordinate (F) at (#2-1+#8,#3-1+#7,#4);
\coordinate (G) at (#2-1+#8,#3-1,#4);
\draw[draw=black, fill=#6, fill opacity=#9] (D) -- (E) -- (F) -- (G) -- cycle;% Front
\draw[draw=black, fill=#6, fill opacity=#9] (C) -- (B) -- (F) -- (G) -- cycle;% Top
\draw[draw=black, fill=#6, fill opacity=#9] (A) -- (B) -- (F) -- (E) -- cycle;% Right
}
\makeatother


% pragmatics
\newcommand{\speaking}[1]{\eqparbox{name}{\textsc{\lowercase{#1}\space}}}
\newcommand{\alname}[1]{\eqparbox{name}{\textsc{\lowercase{#1}}}}
\newcommand{\HPSGTTR}{HPSG$_{\text{TTR}}$\xspace}

\newcommand{\ttrtype}[1]{\textit{#1}}
\newcommand{\avmel}{\q<\quad\q>} %% shortcut for empty lists in AVM
\newcommand{\ttrmerge}{\ensuremath{\wedge_{\textit{merge}}}}
\newcommand{\Cat}[2][0.1pt]{%
  \begin{scope}[y=#1,x=#1,yscale=-1, inner sep=0pt, outer sep=0pt]
   \path[fill=#2,line join=miter,line cap=butt,even odd rule,line width=0.8pt]
  (151.3490,307.2045) -- (264.3490,307.2045) .. controls (264.3490,291.1410) and (263.2021,287.9545) .. (236.5990,287.9545) .. controls (240.8490,275.2045) and (258.1242,244.3581) .. (267.7240,244.3581) .. controls (276.2171,244.3581) and (286.3490,244.8259) .. (286.3490,264.2045) .. controls (286.3490,286.2045) and (323.3717,321.6755) .. (332.3490,307.2045) .. controls (345.7277,285.6390) and (309.3490,292.2151) .. (309.3490,240.2046) .. controls (309.3490,169.0514) and (350.8742,179.1807) .. (350.8742,139.2046) .. controls (350.8742,119.2045) and (345.3490,116.5037) .. (345.3490,102.2045) .. controls (345.3490,83.3070) and (361.9972,84.4036) .. (358.7581,68.7349) .. controls (356.5206,57.9117) and (354.7696,49.2320) .. (353.4652,36.1439) .. controls (352.5396,26.8573) and (352.2445,16.9594) .. (342.5985,17.3574) .. controls (331.2650,17.8250) and (326.9655,37.7742) .. (309.3490,39.2045) .. controls (291.7685,40.6320) and (276.7783,24.2380) .. (269.9740,26.5795) .. controls (263.2271,28.9013) and (265.3490,47.2045) .. (269.3490,60.2045) .. controls (275.6359,80.6368) and (289.3490,107.2045) .. (264.3490,111.2045) .. controls (239.3490,115.2045) and (196.3490,119.2045) .. (165.3490,160.2046) .. controls (134.3490,201.2046) and (135.4934,249.3212) .. (123.3490,264.2045) .. controls (82.5907,314.1553) and (40.8239,293.6463) .. (40.8239,335.2045) .. controls (40.8239,353.8102) and (72.3490,367.2045) .. (77.3490,361.2045) .. controls (82.3490,355.2045) and (34.8638,337.3259) .. (87.9955,316.2045) .. controls (133.3871,298.1601) and   (137.4391,294.4766) .. (151.3490,307.2045) -- cycle;
\end{scope}%
}
%% leicht modifiziert nach Def. von Sebastian Nordhoff
\newcommand{\lueckingbox}[3]{\parbox[t]{0.7cm}{\raggedright
\strut#1}\parbox[t]{7.7cm}{\strut#2}\parbox[t]{3cm}{\raggedright\strut#3}\bigskip\\}




% KdK
\newcommand{\smiley}{:)}

\renewbibmacro*{index:name}[5]{%
  \usebibmacro{index:entry}{#1}
    {\iffieldundef{usera}{}{\thefield{usera}\actualoperator}\mkbibindexname{#2}{#3}{#4}{#5}}}

% \newcommand{\noop}[1]{}

% chngcntr.sty otherwise gives error that these are already defined
%\let\counterwithin\relax
%\let\counterwithout\relax

% the space of a left bracket for glossings
\newcommand{\LB}{\hphantom{[}}

\newcommand{\LF}{\mbox{$[\![$}}

\newcommand{\RF}{\mbox{$]\!]_F$}}

\newcommand{\RT}{\mbox{$]\!]_T$}}





% Manfred's

\newcommand{\kommentar}[1]{}

\newcommand{\bsp}[1]{\emph{#1}}
\newcommand{\bspT}[2]{\bsp{#1} `#2'}
\newcommand{\bspTL}[3]{\bsp{#1} (lit.: #2) `#3'}

\newcommand{\noidi}{§}

\newcommand{\refer}[1]{(\ref{#1})}

%\newcommand{\avmtype}[1]{\multicolumn{2}{l}{\type{#1}}}
\newcommand{\attr}[1]{\textsc{#1}}

%\newcommand{\srdefault}{\mbox{\begin{tabular}{@{}c@{}}{\large <}\\[-1.5ex]$\sqcap$\end{tabular}}}
\newcommand{\srdefault}{$\stackrel{<}{\sqcap}$}


%% \newcommand{\myappcolumn}[2]{
%% \begin{minipage}[t]{#1}#2\end{minipage}
%% }

%% \newcommand{\appc}[1]{\myappcolumn{3.7cm}{#1}}


% Jong-Bok


% clean that up and do not use \def (killing other stuff defined before)
%\if 0
%\newcommand\DEL{\textsc{del}}
%\newcommand\del{\textsc{del}}

\newcommand\conn{\textsc{conn}}
\newcommand\CONN{\textsc{conn}}
\newcommand\CONJ{\textsc{conj}}
\newcommand\LITE{\textsc{lex}}
\newcommand\lite{\textsc{lex}}
\newcommand\HON{\textsc{hon}}

%\newcommand\CAUS{\textsc{caus}}
%\newcommand\PASS{\textsc{pass}}
\newcommand\NPST{\textsc{npst}}
%\newcommand\COND{\textsc{cond}}



\newcommand\hdlite{\textsc{head-lex construction}}
\newcommand\hdlight{\textsc{head-light} Schema}
\newcommand\NFORM{\textsc{nform}}

\newcommand\RELS{\textsc{rels}}
%\newcommand\TENSE{\textsc{tense}}


%\newcommand\ARG{\textsc{arg}}
\newcommand\ARGs{\textsc{arg0}}
\newcommand\ARGa{\textsc{arg}}
\newcommand\ARGb{\textsc{arg2}}
\newcommand\TPC{\textsc{top}}
%\newcommand\PROG{\textsc{prog}}

\newcommand\LIGHT{\textsc{light}\xspace}
\newcommand\pst{\textsc{pst}}
%\newcommand\PAST{\textsc{pst}}
%\newcommand\DAT{\textsc{dat}}
%\newcommand\CONJ{\textsc{conj}}
\newcommand\nominal{\textsc{nominal}}
\newcommand\NOMINAL{\textsc{nominal}}
\newcommand\VAL{\textsc{val}}
%\newcommand\val{\textsc{val}}
\newcommand\MODE{\textsc{mode}}
\newcommand\RESTR{\textsc{restr}}
\newcommand\SIT{\textsc{sit}}
\newcommand\ARG{\textsc{arg}}
\newcommand\RELN{\textsc{rel}}
%\newcommand\REL{\textsc{rel}}
%\newcommand\RELS{\textsc{rels}}
%\newcommand\arg-st{\textsc{arg-st}}
\newcommand\xdel{\textsc{xdel}}
\newcommand\zdel{\textsc{zdel}}
\newcommand\sug{\textsc{sug}}
%\newcommand\IMP{\textsc{imp}}
%\newcommand\conn{\textsc{conn}}
%\newcommand\CONJ{\textsc{conj}}
%\newcommand\HON{\textsc{hon}}
\newcommand\BN{\textsc{bn}}
\newcommand\bn{\textsc{bn}}
\newcommand\pres{\textsc{pres}}
\newcommand\PRES{\textsc{pres}}
\newcommand\prs{\textsc{pres}}
%\newcommand\PRS{\textsc{pres}}
\newcommand\agt{\textsc{agt}}
%\newcommand\DEL{\textsc{del}}
%\newcommand\PRED{\textsc{pred}}
\newcommand\AGENT{\textsc{agent}}
\newcommand\THEME{\textsc{theme}}
%\newcommand\AUX{\textsc{aux}}
%\newcommand\THEME{\textsc{theme}}
%\newcommand\PL{\textsc{pl}}
\newcommand\SRC{\textsc{src}}
\newcommand\src{\textsc{src}}
\newcommand{\FORMjb}{\textsc{form}}
\newcommand{\formjb}{\FORM}
\newcommand\GCASE{\textsc{gcase}}
\newcommand\gcase{\textsc{gcase}}
\newcommand\SCASE{\textsc{scase}}
\newcommand\PHON{\textsc{phon}}
%\newcommand\SS{\textsc{ss}}
\newcommand\SYN{\textsc{syn}}
%\newcommand\LOC{\textsc{loc}}
\newcommand\MOD{\textsc{mod}}
\newcommand\INV{\textsc{inv}}
%\newcommand\L{\textsc{l}}
%\newcommand\CASE{\textsc{case}}
\newcommand\SPR{\textsc{spr}}
\newcommand\COMPS{\textsc{comps}}
%\newcommand\comps{\textsc{comps}}
\newcommand\SEM{\textsc{sem}}
\newcommand\CONT{\textsc{cont}}
\newcommand\SUBCAT{\textsc{subcat}}
\newcommand\CAT{\textsc{cat}}
%\newcommand\C{\textsc{c}}
%\newcommand\SUBJ{\textsc{subj}}
\newcommand\subjjb{\textsc{subj}}
%\newcommand\SLASH{\textsc{slash}}
\newcommand\LOCAL{\textsc{local}}
%\newcommand\ARG-ST{\textsc{arg-st}}
%\newcommand\AGR{\textsc{agr}}
\newcommand\PER{\textsc{per}}
%\newcommand\NUM{\textsc{num}}
%\newcommand\IND{\textsc{ind}}
\newcommand\VFORM{\textsc{vform}}
\newcommand\PFORM{\textsc{pform}}
\newcommand\decl{\textsc{decl}}
%\newcommand\loc{\textsc{loc   }}
% \newcommand\   {\textsc{  }}

%\newcommand\NEG{\textsc{neg}}
\newcommand\FRAMES{\textsc{frames}}
%\newcommand\REFL{\textsc{refl}}

\newcommand\MKG{\textsc{mkg}}

%\newcommand\BN{\textsc{bn}}
\newcommand\HD{\textsc{hd}}
\newcommand\NP{\textsc{np}}
\newcommand\PF{\textsc{pf}}
%\newcommand\PL{\textsc{pl}}
\newcommand\PP{\textsc{pp}}
%\newcommand\SS{\textsc{ss}}
\newcommand\VF{\textsc{vf}}
\newcommand\VP{\textsc{vp}}
%\newcommand\bn{\textsc{bn}}
\newcommand\cl{\textsc{cl}}
%\newcommand\pl{\textsc{pl}}
\newcommand\Wh{\ital{Wh}}
%\newcommand\ng{\textsc{neg}}
\newcommand\wh{\ital{wh}}
%\newcommand\ACC{\textsc{acc}}
%\newcommand\AGR{\textsc{agr}}
\newcommand\AGT{\textsc{agt}}
\newcommand\ARC{\textsc{arc}}
%\newcommand\ARG{\textsc{arg}}
\newcommand\ARP{\textsc{arc}}
%\newcommand\AUX{\textsc{aux}}
%\newcommand\CAT{\textsc{cat}}
%\newcommand\COP{\textsc{cop}}
%\newcommand\DAT{\textsc{dat}}
\newcommand\NEWCOMMAND{\textsc{def}}
%\newcommand\DEL{\textsc{del}}
\newcommand\DOM{\textsc{dom}}
\newcommand\DTR{\textsc{dtr}}
%\newcommand\FUT{\textsc{fut}}
\newcommand\GAP{\textsc{gap}}
%\newcommand\GEN{\textsc{gen}}
%\newcommand\HON{\textsc{hon}}
%\newcommand\IMP{\textsc{imp}}
%\newcommand\IND{\textsc{ind}}
%\newcommand\INV{\textsc{inv}}
\newcommand\LEX{\textsc{lex}}
\newcommand\Lex{\textsc{lex}}
%\newcommand\LOC{\textsc{loc}}
%\newcommand\MOD{\textsc{mod}}
\newcommand\MRK{{\nr MRK}}
%\newcommand\NEG{\textsc{neg}}
\newcommand\NEW{\textsc{new}}
%\newcommand\NOM{\textsc{nom}}
%\newcommand\NUM{\textsc{num}}
%\newcommand\PER{\textsc{per}}
%\newcommand\PST{\textsc{pst}}
\newcommand\QUE{\textsc{que}}
%\newcommand\REL{\textsc{rel}}
\newcommand\SEL{\textsc{sel}}
%\newcommand\SEM{\textsc{sem}}
%\newcommand\SIT{\textsc{arg0}}
%\newcommand\SPR{\textsc{spr}}
%\newcommand\SRC{\textsc{src}}
\newcommand\SUG{\textsc{sug}}
%\newcommand\SYN{\textsc{syn}}
%\newcommand\TPC{\textsc{top}}
%\newcommand\VAL{\textsc{val}}
%\newcommand\acc{\textsc{acc}}
%\newcommand\agt{\textsc{agt}}
\newcommand\cop{\textsc{cop}}
%\newcommand\dat{\textsc{dat}}
\newcommand\foc{\textsc{focus}}
%\newcommand\FOC{\textsc{focus}}
\newcommand\fut{\textsc{fut}}
\newcommand\hon{\textsc{hon}}
\newcommand\imp{\textsc{imp}}
\newcommand\kes{\textsc{kes}}
%\newcommand\lex{\textsc{lex}}
%\newcommand\loc{\textsc{loc}}
\newcommand\mrk{{\nr MRK}}
%\newcommand\nom{\textsc{nom}}
%\newcommand\num{\textsc{num}}
\newcommand\plu{\textsc{plu}}
\newcommand\pne{\textsc{pne}}
%\newcommand\pst{\textsc{pst}}
\newcommand\pur{\textsc{pur}}
%\newcommand\que{\textsc{que}}
%\newcommand\src{\textsc{src}}
%\newcommand\sug{\textsc{sug}}
\newcommand\tpc{\textsc{top}}
%\newcommand\utt{\textsc{utt}}
%\newcommand\val{\textsc{val}}
%% \newcommand\LITE{\textsc{lex}}
%% \newcommand\PAST{\textsc{pst}}
%% \newcommand\POSP{\textsc{pos}}
%% \newcommand\PRS{\textsc{pres}}
%% \newcommand\mod{\textsc{mod}}%
%% \newcommand\newuse{{`kes'}}
%% \newcommand\posp{\textsc{pos}}
%% \newcommand\prs{\textsc{pres}}
%% \newcommand\psp{{\it en\/}}
%% \newcommand\skes{\textsc{kes}}
%% \newcommand\CASE{\textsc{case}}
%% \newcommand\CASE{\textsc{case}}
%% \newcommand\COMP{\textsc{comp}}
%% \newcommand\CONJ{\textsc{conj}}
%% \newcommand\CONN{\textsc{conn}}
%% \newcommand\CONT{\textsc{cont}}
%% \newcommand\DECL{\textsc{decl}}
%% \newcommand\FOCUS{\textsc{focus}}
%% %\newcommand\FORM{\textsc{form}} duplicate
%% \newcommand\FREL{\textsc{frel}}
%% \newcommand\GOAL{\textsc{goal}}
\newcommand\HEAD{\textsc{head}}
%% \newcommand\INDEX{\textsc{ind}}
%% \newcommand\INST{\textsc{inst}}
%% \newcommand\MODE{\textsc{mode}}
%% \newcommand\MOOD{\textsc{mood}}
%% \newcommand\NMLZ{\textsc{nmlz}}
%% \newcommand\PHON{\textsc{phon}}
%% \newcommand\PRED{\textsc{pred}}
%% %\newcommand\PRES{\textsc{pres}}
%% \newcommand\PROM{\textsc{prom}}
%% \newcommand\RELN{\textsc{pred}}
%% \newcommand\RELS{\textsc{rels}}
%% \newcommand\STEM{\textsc{stem}}
%% \newcommand\SUBJ{\textsc{subj}}
%% \newcommand\XARG{\textsc{xarg}}
%% \newcommand\bse{{\it bse\/}}
%% \newcommand\case{\textsc{case}}
%% \newcommand\caus{\textsc{caus}}
%% \newcommand\comp{\textsc{comp}}
%% \newcommand\conj{\textsc{conj}}
%% \newcommand\conn{\textsc{conn}}
%% \newcommand\decl{\textsc{decl}}
%% \newcommand\fin{{\it fin\/}}
%% %\newcommand\form{\textsc{form}}
%% \newcommand\gend{\textsc{gend}}
%% \newcommand\inf{{\it inf\/}}
%% \newcommand\mood{\textsc{mood}}
%% \newcommand\nmlz{\textsc{nmlz}}
%% \newcommand\pass{\textsc{pass}}
%% \newcommand\past{\textsc{past}}
%% \newcommand\perf{\textsc{perf}}
%% \newcommand\pln{{\it pln\/}}
%% \newcommand\pred{\textsc{pred}}


%% %\newcommand\pres{\textsc{pres}}
%% \newcommand\proc{\textsc{proc}}
%% \newcommand\nonfin{{\it nonfin\/}}
%% \newcommand\AGENT{\textsc{agent}}
%% \newcommand\CFORM{\textsc{cform}}
%% %\newcommand\COMPS{\textsc{comps}}
%% \newcommand\COORD{\textsc{coord}}
%% \newcommand\COUNT{\textsc{count}}
%% \newcommand\EXTRA{\textsc{extra}}
%% \newcommand\GCASE{\textsc{gcase}}
%% \newcommand\GIVEN{\textsc{given}}
%% \newcommand\LOCAL{\textsc{local}}
%% \newcommand\NFORM{\textsc{nform}}
%% \newcommand\PFORM{\textsc{pform}}
%% \newcommand\SCASE{\textsc{scase}}
%% \newcommand\SLASH{\textsc{slash}}
%% \newcommand\SLASH{\textsc{slash}}
%% \newcommand\THEME{\textsc{theme}}
%% \newcommand\TOPIC{\textsc{topic}}
%% \newcommand\VFORM{\textsc{vform}}
%% \newcommand\cause{\textsc{cause}}
%% %\newcommand\comps{\textsc{comps}}
%% \newcommand\gcase{\textsc{gcase}}
%% \newcommand\itkes{{\it kes\/}}
%% \newcommand\pass{{\it pass\/}}
%% \newcommand\vform{\textsc{vform}}
%% \newcommand\CCONT{\textsc{c-cont}}
%% \newcommand\GN{\textsc{given-new}}
%% \newcommand\INFO{\textsc{info-st}}
%% \newcommand\ARG-ST{\textsc{arg-st}}
%% \newcommand\SUBCAT{\textsc{subcat}}
%% \newcommand\SYNSEM{\textsc{synsem}}
%% \newcommand\VERBAL{\textsc{verbal}}
%% \newcommand\arg-st{\textsc{arg-st}}
%% \newcommand\plain{{\it plain}\/}
%% \newcommand\propos{\textsc{propos}}
%% \newcommand\ADVERBIAL{\textsc{advl}}
%% \newcommand\HIGHLIGHT{\textsc{prom}}
%% \newcommand\NOMINAL{\textsc{nominal}}

\newenvironment{myavm}{\begingroup\avmvskip{.1ex}
  \selectfont\begin{avm}}%
{\end{avm}\endgroup\medskip}
\newcommand\pfix{\vspace{-5pt}}


\newcommand{\jbsub}[1]{\lower4pt\hbox{\small #1}}
\newcommand{\jbssub}[1]{\lower4pt\hbox{\small #1}}
\newcommand\jbtr{\underbar{\ \ \ }\ }


%\fi

% cl

\newcommand{\delphin}{\textsc{delph-in}}


% YK -- CG chapter

\newcommand{\grey}[1]{\colorbox{mycolor}{#1}}
\definecolor{mycolor}{gray}{0.8}

\newcommand{\GQU}[2]{\raisebox{1.6ex}{\ensuremath{\rotatebox{180}{\textbf{#1}}_{\scalebox{.7}{\textbf{#2}}}}}}

\newcommand{\SetInfLen}{\setpremisesend{0pt}\setpremisesspace{10pt}\setnamespace{0pt}}

\newcommand{\pt}[1]{\ensuremath{\mathsf{#1}}}
\newcommand{\ptv}[1]{\ensuremath{\textsf{\textsl{#1}}}}

\newcommand{\sv}[1]{\ensuremath{\bm{\mathcal{#1}}}}
\newcommand{\sX}{\sv{X}}
\newcommand{\sF}{\sv{F}}
\newcommand{\sG}{\sv{G}}

\newcommand{\syncat}[1]{\textrm{#1}}
\newcommand{\syncatVar}[1]{\ensuremath{\mathit{#1}}}

\newcommand{\RuleName}[1]{\textrm{#1}}

\newcommand{\SemTyp}{\textsf{Sem}}

\newcommand{\E}{\ensuremath{\bm{\epsilon}}\xspace}

\newcommand{\greeka}{\upalpha}
\newcommand{\greekb}{\upbeta}
\newcommand{\greekd}{\updelta}
\newcommand{\greekp}{\upvarphi}
\newcommand{\greekr}{\uprho}
\newcommand{\greeks}{\upsigma}
\newcommand{\greekt}{\uptau}
\newcommand{\greeko}{\upomega}
\newcommand{\greekz}{\upzeta}

\newcommand{\Lemma}{\ensuremath{\hskip.5em\vdots\hskip.5em}\noLine}
\newcommand{\LemmaAlt}{\ensuremath{\hskip.5em\vdots\hskip.5em}}

\newcommand{\I}{\iota}

\newcommand{\sem}{\ensuremath}

\newcommand{\NoSem}{%
\renewcommand{\LexEnt}[3]{##1; \syncat{##3}}
\renewcommand{\LexEntTwoLine}[3]{\renewcommand{\arraystretch}{.8}%
\begin{array}[b]{l} ##1;  \\ \syncat{##3} \end{array}}
\renewcommand{\LexEntThreeLine}[3]{\renewcommand{\arraystretch}{.8}%
\begin{array}[b]{l} ##1; \\ \syncat{##3} \end{array}}}

\newcommand{\hypml}[2]{\left[\!\!#1\!\!\right]^{#2}}

%%%%for bussproof
\def\defaultHypSeparation{\hskip0.1in}
\def\ScoreOverhang{0pt}

\newcommand{\MultiLine}[1]{\renewcommand{\arraystretch}{.8}%
\ensuremath{\begin{array}[b]{l} #1 \end{array}}}

\newcommand{\MultiLineMod}[1]{%
\ensuremath{\begin{array}[t]{l} #1 \end{array}}}

\newcommand{\hypothesis}[2]{[ #1 ]^{#2}}

\newcommand{\LexEnt}[3]{#1; \ensuremath{#2}; \syncat{#3}}

\newcommand{\LexEntTwoLine}[3]{\renewcommand{\arraystretch}{.8}%
\begin{array}[b]{l} #1; \\ \ensuremath{#2};  \syncat{#3} \end{array}}

\newcommand{\LexEntThreeLine}[3]{\renewcommand{\arraystretch}{.8}%
\begin{array}[b]{l} #1; \\ \ensuremath{#2}; \\ \syncat{#3} \end{array}}

\newcommand{\LexEntFiveLine}[5]{\renewcommand{\arraystretch}{.8}%
\begin{array}{l} #1 \\ #2; \\ \ensuremath{#3} \\ \ensuremath{#4}; \\ \syncat{#5} \end{array}}

\newcommand{\LexEntFourLine}[4]{\renewcommand{\arraystretch}{.8}%
\begin{array}{l} \pt{#1} \\ \pt{#2}; \\ \syncat{#4} \end{array}}

\newcommand{\ManySomething}{\renewcommand{\arraystretch}{.8}%
\raisebox{-3mm}{\begin{array}[b]{c} \vdots \,\,\,\,\,\, \vdots \\
\vdots \end{array}}}

\newcommand{\lemma}[1]{\renewcommand{\arraystretch}{.8}%
\begin{array}[b]{c} \vdots \\ #1 \end{array}}

\newcommand{\lemmarev}[1]{\renewcommand{\arraystretch}{.8}%
\begin{array}[b]{c} #1 \\ \vdots \end{array}}

\newcommand{\p}{\ensuremath{\upvarphi}}

% clashes with soul package
\newcommand{\yusukest}{\textbf{\textsf{st}}}

\newcommand{\shortarrow}{\xspace\hskip-1.2ex\scalebox{.5}[1]{\ensuremath{\bm{\rightarrow}}}\hskip-.5ex\xspace}

\newcommand{\SemInt}[1]{\mbox{$[\![ \textrm{#1} ]\!]$}}

\newcommand{\HypSpace}{\hskip-.8ex}
\newcommand{\RaiseHeight}{\raisebox{2.2ex}}
\newcommand{\RaiseHeightLess}{\raisebox{1ex}}

\newcommand{\ThreeColHyp}[1]{\RaiseHeight{\Bigg[}\HypSpace#1\HypSpace\RaiseHeight{\Bigg]}}
\newcommand{\TwoColHyp}[1]{\RaiseHeightLess{\Big[}\HypSpace#1\HypSpace\RaiseHeightLess{\Big]}}

\newcommand{\LemmaShort}{\ensuremath{ \ \vdots} \ \noLine}
\newcommand{\LemmaShortAlt}{\ensuremath{ \ \vdots} \ }

\newcommand{\fail}{**}
\newcommand{\vs}{\raisebox{.05em}{\ensuremath{\upharpoonright}}}
\newcommand{\DerivSize}{\small}

\def\maru#1{{\ooalign{\hfil
  \ifnum#1>999 \resizebox{.25\width}{\height}{#1}\else%
  \ifnum#1>99 \resizebox{.33\width}{\height}{#1}\else%
  \ifnum#1>9 \resizebox{.5\width}{\height}{#1}\else #1%
  \fi\fi\fi%
\/\hfil\crcr%
\raise.167ex\hbox{\mathhexbox20D}}}}

\newenvironment{samepage2}%
 {\begin{flushleft}\begin{minipage}{\linewidth}}
 {\end{minipage}\end{flushleft}}

\newcommand{\cmt}[1]{\textsl{\textbf{[#1]}}}
\newcommand{\trns}[1]{\textbf{#1}\xspace}
\newcommand{\ptfont}{}
\newcommand{\gp}{\underline{\phantom{oo}}}
\newcommand{\mgcmt}{\marginnote}

\newcommand{\term}[1]{\emph{\isi{#1}}}

\newcommand{\citeposs}[1]{\citeauthor{#1}'s \citeyearpar{#1}}

% for standalone compilations Felix: This is in the class already
%\let\thetitle\@title
%\let\theauthor\@author 
\makeatletter
\newcommand{\togglepaper}[1][0]{ 
\bibliography{../Bibliographies/stmue,../localbibliography,
collection.bib}
  %% hyphenation points for line breaks
%% Normally, automatic hyphenation in LaTeX is very good
%% If a word is mis-hyphenated, add it to this file
%%
%% add information to TeX file before \begin{document} with:
%% %% hyphenation points for line breaks
%% Normally, automatic hyphenation in LaTeX is very good
%% If a word is mis-hyphenated, add it to this file
%%
%% add information to TeX file before \begin{document} with:
%% %% hyphenation points for line breaks
%% Normally, automatic hyphenation in LaTeX is very good
%% If a word is mis-hyphenated, add it to this file
%%
%% add information to TeX file before \begin{document} with:
%% \include{localhyphenation}
\hyphenation{
A-la-hver-dzhie-va
ac-cu-sa-tive
anaph-o-ra
ana-phor
ana-phors
an-te-ced-ent
an-te-ced-ents
affri-ca-te
affri-ca-tes
ap-proach-es
Atha-bas-kan
Athe-nä-um
Be-schrei-bung
Bona-mi
Chi-che-ŵa
com-ple-ments
con-straints
Cope-sta-ke
Da-ge-stan
Dor-drecht
er-klä-ren-de
Flick-inger
Ginz-burg
Gro-ning-en
Has-pel-math
Jap-a-nese
Jon-a-than
Ka-tho-lie-ke
Ko-bon
krie-gen
Kroe-ger
Le-Sourd
moth-er
Mül-ler
Nie-mey-er
Ørs-nes
Par-a-digm
Prze-piór-kow-ski
phe-nom-e-non
re-nowned
Rie-he-mann
un-bound-ed
Ver-gleich
with-in
}

% listing within here does not have any effect for lfg.tex % 2020-05-14

% why has "erklärende" be listed here? I specified langid in bibtex item. Something is still not working with hyphenation.


% to do: check
%  Alahverdzhieva


% biblatex:

% This is a LaTeX frontend to TeX’s \hyphenation command which defines hy- phenation exceptions. The ⟨language⟩ must be a language name known to the babel/polyglossia packages. The ⟨text ⟩ is a whitespace-separated list of words. Hyphenation points are marked with a dash:

% \DefineHyphenationExceptions{american}{%
% hy-phen-ation ex-cep-tion }

\hyphenation{
A-la-hver-dzhie-va
ac-cu-sa-tive
anaph-o-ra
ana-phor
ana-phors
an-te-ced-ent
an-te-ced-ents
affri-ca-te
affri-ca-tes
ap-proach-es
Atha-bas-kan
Athe-nä-um
Be-schrei-bung
Bona-mi
Chi-che-ŵa
com-ple-ments
con-straints
Cope-sta-ke
Da-ge-stan
Dor-drecht
er-klä-ren-de
Flick-inger
Ginz-burg
Gro-ning-en
Has-pel-math
Jap-a-nese
Jon-a-than
Ka-tho-lie-ke
Ko-bon
krie-gen
Kroe-ger
Le-Sourd
moth-er
Mül-ler
Nie-mey-er
Ørs-nes
Par-a-digm
Prze-piór-kow-ski
phe-nom-e-non
re-nowned
Rie-he-mann
un-bound-ed
Ver-gleich
with-in
}

% listing within here does not have any effect for lfg.tex % 2020-05-14

% why has "erklärende" be listed here? I specified langid in bibtex item. Something is still not working with hyphenation.


% to do: check
%  Alahverdzhieva


% biblatex:

% This is a LaTeX frontend to TeX’s \hyphenation command which defines hy- phenation exceptions. The ⟨language⟩ must be a language name known to the babel/polyglossia packages. The ⟨text ⟩ is a whitespace-separated list of words. Hyphenation points are marked with a dash:

% \DefineHyphenationExceptions{american}{%
% hy-phen-ation ex-cep-tion }

\hyphenation{
A-la-hver-dzhie-va
ac-cu-sa-tive
anaph-o-ra
ana-phor
ana-phors
an-te-ced-ent
an-te-ced-ents
affri-ca-te
affri-ca-tes
ap-proach-es
Atha-bas-kan
Athe-nä-um
Be-schrei-bung
Bona-mi
Chi-che-ŵa
com-ple-ments
con-straints
Cope-sta-ke
Da-ge-stan
Dor-drecht
er-klä-ren-de
Flick-inger
Ginz-burg
Gro-ning-en
Has-pel-math
Jap-a-nese
Jon-a-than
Ka-tho-lie-ke
Ko-bon
krie-gen
Kroe-ger
Le-Sourd
moth-er
Mül-ler
Nie-mey-er
Ørs-nes
Par-a-digm
Prze-piór-kow-ski
phe-nom-e-non
re-nowned
Rie-he-mann
un-bound-ed
Ver-gleich
with-in
}

% listing within here does not have any effect for lfg.tex % 2020-05-14

% why has "erklärende" be listed here? I specified langid in bibtex item. Something is still not working with hyphenation.


% to do: check
%  Alahverdzhieva


% biblatex:

% This is a LaTeX frontend to TeX’s \hyphenation command which defines hy- phenation exceptions. The ⟨language⟩ must be a language name known to the babel/polyglossia packages. The ⟨text ⟩ is a whitespace-separated list of words. Hyphenation points are marked with a dash:

% \DefineHyphenationExceptions{american}{%
% hy-phen-ation ex-cep-tion }

  \memoizeset{
    memo filename prefix={hpsg-handbook.memo.dir/},
    % readonly
  }
  \papernote{\scriptsize\normalfont
    \@author.
    \titleTemp. 
    To appear in: 
    Stefan Müller, Anne Abeillé, Robert D. Borsley \& Jean-Pierre Koenig (eds.)
    HPSG Handbook
    Berlin: Language Science Press. [preliminary page numbering]
  }
  \pagenumbering{roman}
  \setcounter{chapter}{#1}
  \addtocounter{chapter}{-1}
}
\makeatother

\makeatletter
\newcommand{\togglepaperminimal}[1][0]{ 
  \bibliography{../Bibliographies/stmue,
                ../localbibliography,
collection.bib}
  %% hyphenation points for line breaks
%% Normally, automatic hyphenation in LaTeX is very good
%% If a word is mis-hyphenated, add it to this file
%%
%% add information to TeX file before \begin{document} with:
%% %% hyphenation points for line breaks
%% Normally, automatic hyphenation in LaTeX is very good
%% If a word is mis-hyphenated, add it to this file
%%
%% add information to TeX file before \begin{document} with:
%% %% hyphenation points for line breaks
%% Normally, automatic hyphenation in LaTeX is very good
%% If a word is mis-hyphenated, add it to this file
%%
%% add information to TeX file before \begin{document} with:
%% \include{localhyphenation}
\hyphenation{
A-la-hver-dzhie-va
ac-cu-sa-tive
anaph-o-ra
ana-phor
ana-phors
an-te-ced-ent
an-te-ced-ents
affri-ca-te
affri-ca-tes
ap-proach-es
Atha-bas-kan
Athe-nä-um
Be-schrei-bung
Bona-mi
Chi-che-ŵa
com-ple-ments
con-straints
Cope-sta-ke
Da-ge-stan
Dor-drecht
er-klä-ren-de
Flick-inger
Ginz-burg
Gro-ning-en
Has-pel-math
Jap-a-nese
Jon-a-than
Ka-tho-lie-ke
Ko-bon
krie-gen
Kroe-ger
Le-Sourd
moth-er
Mül-ler
Nie-mey-er
Ørs-nes
Par-a-digm
Prze-piór-kow-ski
phe-nom-e-non
re-nowned
Rie-he-mann
un-bound-ed
Ver-gleich
with-in
}

% listing within here does not have any effect for lfg.tex % 2020-05-14

% why has "erklärende" be listed here? I specified langid in bibtex item. Something is still not working with hyphenation.


% to do: check
%  Alahverdzhieva


% biblatex:

% This is a LaTeX frontend to TeX’s \hyphenation command which defines hy- phenation exceptions. The ⟨language⟩ must be a language name known to the babel/polyglossia packages. The ⟨text ⟩ is a whitespace-separated list of words. Hyphenation points are marked with a dash:

% \DefineHyphenationExceptions{american}{%
% hy-phen-ation ex-cep-tion }

\hyphenation{
A-la-hver-dzhie-va
ac-cu-sa-tive
anaph-o-ra
ana-phor
ana-phors
an-te-ced-ent
an-te-ced-ents
affri-ca-te
affri-ca-tes
ap-proach-es
Atha-bas-kan
Athe-nä-um
Be-schrei-bung
Bona-mi
Chi-che-ŵa
com-ple-ments
con-straints
Cope-sta-ke
Da-ge-stan
Dor-drecht
er-klä-ren-de
Flick-inger
Ginz-burg
Gro-ning-en
Has-pel-math
Jap-a-nese
Jon-a-than
Ka-tho-lie-ke
Ko-bon
krie-gen
Kroe-ger
Le-Sourd
moth-er
Mül-ler
Nie-mey-er
Ørs-nes
Par-a-digm
Prze-piór-kow-ski
phe-nom-e-non
re-nowned
Rie-he-mann
un-bound-ed
Ver-gleich
with-in
}

% listing within here does not have any effect for lfg.tex % 2020-05-14

% why has "erklärende" be listed here? I specified langid in bibtex item. Something is still not working with hyphenation.


% to do: check
%  Alahverdzhieva


% biblatex:

% This is a LaTeX frontend to TeX’s \hyphenation command which defines hy- phenation exceptions. The ⟨language⟩ must be a language name known to the babel/polyglossia packages. The ⟨text ⟩ is a whitespace-separated list of words. Hyphenation points are marked with a dash:

% \DefineHyphenationExceptions{american}{%
% hy-phen-ation ex-cep-tion }

\hyphenation{
A-la-hver-dzhie-va
ac-cu-sa-tive
anaph-o-ra
ana-phor
ana-phors
an-te-ced-ent
an-te-ced-ents
affri-ca-te
affri-ca-tes
ap-proach-es
Atha-bas-kan
Athe-nä-um
Be-schrei-bung
Bona-mi
Chi-che-ŵa
com-ple-ments
con-straints
Cope-sta-ke
Da-ge-stan
Dor-drecht
er-klä-ren-de
Flick-inger
Ginz-burg
Gro-ning-en
Has-pel-math
Jap-a-nese
Jon-a-than
Ka-tho-lie-ke
Ko-bon
krie-gen
Kroe-ger
Le-Sourd
moth-er
Mül-ler
Nie-mey-er
Ørs-nes
Par-a-digm
Prze-piór-kow-ski
phe-nom-e-non
re-nowned
Rie-he-mann
un-bound-ed
Ver-gleich
with-in
}

% listing within here does not have any effect for lfg.tex % 2020-05-14

% why has "erklärende" be listed here? I specified langid in bibtex item. Something is still not working with hyphenation.


% to do: check
%  Alahverdzhieva


% biblatex:

% This is a LaTeX frontend to TeX’s \hyphenation command which defines hy- phenation exceptions. The ⟨language⟩ must be a language name known to the babel/polyglossia packages. The ⟨text ⟩ is a whitespace-separated list of words. Hyphenation points are marked with a dash:

% \DefineHyphenationExceptions{american}{%
% hy-phen-ation ex-cep-tion }

  \memoizeset{
    memo filename prefix={hpsg-handbook.memo.dir/},
    % readonly
  }
  \papernote{\scriptsize\normalfont
    \@author.
    \@title. 
    To appear in: 
    Stefan Müller, Anne Abeillé, Robert D. Borsley \& Jean-Pierre Koenig (eds.)
    HPSG Handbook
    Berlin: Language Science Press. [preliminary page numbering]
  }
  \pagenumbering{roman}
  \setcounter{chapter}{#1}
  \addtocounter{chapter}{-1}
}
\makeatother




% In case that year is not given, but pubstate. This mainly occurs for titles that are forthcoming, in press, etc.
\renewbibmacro*{addendum+pubstate}{% Thanks to https://tex.stackexchange.com/a/154367 for the idea
  \printfield{addendum}%
  \iffieldequalstr{labeldatesource}{pubstate}{}
  {\newunit\newblock\printfield{pubstate}}
}

\DeclareLabeldate{%
    \field{date}
    \field{year}
    \field{eventdate}
    \field{origdate}
    \field{urldate}
    \field{pubstate}
    \literal{nodate}
}

%\defbibheading{diachrony-sources}{\section*{Sources}} 

% if no langid is set, it is English:
% https://tex.stackexchange.com/a/279302
\DeclareSourcemap{
  \maps[datatype=bibtex]{
    \map{
      \step[fieldset=langid, fieldvalue={english}]
    }
  }
}


% for bibliographies
% biber/biblatex could use sortname field rather than messing around this way.
\newcommand{\SortNoop}[1]{}


% Doug Ball

\newcommand{\elist}{\q<\ \ \q>}

\newcommand{\esetDB}{\q\{\ \ \q\}}


\makeatletter

\newcommand{\nolistbreak}{%

  \let\oldpar\par\def\par{\oldpar\nobreak}% Any \par issues a \nobreak

  \@nobreaktrue% Don't break with first \item

}

\makeatother


% intermediate before Frank's trees are fixed
% This will be removed!!!!!
%\newcommand{\tree}[1]{} % ignore them blody trees
%\usepackage{tree-dvips}


\newcommand{\nodeconnect}[2]{}
\newcommand{\nodetriangle}[2]{}



% Doug relative clauses
%% I've compiled out almost all my private LaTeX command, but there are some
%% I found hard to get rid of. They are defined here.
%% There are few others which defined in places in the document where they have only
%% local effect (e.g. within figures); their names all end in DA, e.g. \MotherDA
%% There are a lot of \labels -- they are all of the form \label{sec:rc-...} or
%% \label{x:rc-...} or similar, so there should be no clashes.

% Subscripts -- scriptsize italic shape lowered by .25ex 
\newcommand{\subscr}[1]{\raisebox{-.5ex}{\protect{\scriptsize{\itshape #1\/}}}}
% A boxed subscript, for avm tags in normal text
\newcommand{\subtag}[1]{\subscr{\idx{#1}}}

%% Sets and tuples: I use \setof{} to get brackets that are upright, not slanted
%\newcommand{\setof}[1]{\ensuremath{\lbrace\,\mathit{#1}\,\rbrace}}
% 11.10.2019 EP: Doug requested replacement of existing \setof definition with the following:
%\newcommand{\setof}[1]{\begin{avm}\{\textcolor{red}{#1}\}\end{avm}}
% 31.1.2019 EP: Doug requested re-replacement of the above \textcolour version with the following:
\newcommand{\setof}[1]{\begin{avm}\{#1\}\end{avm}}

\newcommand{\tuple}[1]{\ensuremath{\left\langle\,\mbox{\textit{#1}}\,\right\rangle}}

% Single pile of stuff, optional arugment is psn (e.g. t or b)
% e.g. to put a over b over c in a centered column, top aligned, do:
%   \cPile[t]{a\\b\\c} 
\newcommand{\cPile}[2][]{%
  \begingroup%
  \renewcommand{\arraystretch}{.5}\begin{tabular}[#1]{@{}c@{}}#2\end{tabular}%
  \endgroup%
}

%% for linguistic examples in running text (`linguistic citation'):
\newcommand{\lic}[1]{\textit{#1}}

%% A gap marked by an underline, raised slightly
%% Default argument indicates how long the line should be:
\newcommand{\uGap}[1][3ex]{\raisebox{.25em}{\underline{\hspace{#1}}}\xspace}

%% \TnodeDA{XP}{avmcontents} -- in a Tree, put a node label next to an AVM
\newcommand{\TnodeDA}[2]{#1~\begin{avm}{#2}\end{avm}}

%% This allows tipa stuff to be put in \emph -- we need to change to cmr first.
%% It is used in the discussion of Arabic.
\newcommand{\emphtipa}[1]{{\fontfamily{cmr}\emph{\tipaencoding #1}}} 



 
 
\definecolor{lsDOIGray}{cmyk}{0,0,0,0.45}


% morphology.tex:
% Berthold

\newcommand{\dnode}[1]{\rnode{#1}{\fbox{#1}}}
\newcommand{\tnode}[1]{\rnode{#1}{\textit{#1}}}

\newcommand{\tl}[2]{#2}

\newcommand{\rrr}[3]{%
  \psframebox[linestyle=none]{%
    \avmoptions{center}
    \begin{avm}
      \[mud & \{ #1 \}\\
      ms & \{ #2 \}\\
      mph & \<  #3 \> \]
    \end{avm}
  }
}
\newcommand{\rr}[2]{%
  \psframebox[linestyle=none]{%
    \avmoptions{center}
    \begin{avm}
      \[mud & \{ #1 \}\\
      mph & \<  #2 \> \]
    \end{avm}
  }
}
 

% Frank Richter
\newtheorem{mydef}{Definition}

\long\def\set[#1\set=#2\set]%
{%
\left\{%
\tabcolsep 1pt%
\begin{tabular}{l}%
#1%
\end{tabular}%
\left|%
\tabcolsep 1pt%
\begin{tabular}{l}%
#2%
\end{tabular}%
\right.%
\right\}%
}

\newcommand{\einruck}{\\ \hspace*{1em}}


%\newcommand{\NatNum}{\mathrm{I\hspace{-.17em}N}}
\newcommand{\NatNum}{\mathbb{N}}
\newcommand{\Aug}[1]{\widehat{#1}}
%\newcommand{\its}{\mathrm{:}}
% Felix 14.02.2020
\DeclareMathOperator{\its}{:}

\newcommand{\sequence}[1]{\langle#1\rangle}

\newcommand{\INTERPRETATION}[2]{\sequence{#1\mathsf{U}#2,#1\mathsf{S}#2,#1\mathsf{A}#2,#1\mathsf{R}#2}}
\newcommand{\Interpretation}{\INTERPRETATION{}{}}

\newcommand{\Inte}{\mathsf{I}}
\newcommand{\Unive}{\mathsf{U}}
\newcommand{\Speci}{\mathsf{S}}
\newcommand{\Atti}{\mathsf{A}}
\newcommand{\Reli}{\mathsf{R}}
\newcommand{\ReliT}{\mathsf{RT}}

\newcommand{\VarInt}{\mathsf{G}}
\newcommand{\CInt}{\mathsf{C}}
\newcommand{\Tinte}{\mathsf{T}}
\newcommand{\Dinte}{\mathsf{D}}

% this was missing from ash's stuff.

%% \def \optrulenode#1{
%%   \setbox1\hbox{$\left(\hbox{\begin{tabular}{@{\strut}c@{\strut}}#1\end{tabular}}\right)$}
%%   \raisebox{1.9ex}{\raisebox{-\ht1}{\copy1}}}



\newcommand{\pslabel}[1]{}

\newcommand{\addpagesunless}{\todostefan{add pages unless you cite the
 work as such}}

% dg.tex
% framed boxes as used in dg.tex
% original idea from stackexchange, but modified by Saso
% http://tex.stackexchange.com/questions/230300/doing-something-like-psframebox-in-tikz#230306
\tikzset{
  frbox/.style={
    rounded corners,
    draw,
    thick,
    inner sep=5pt,
    anchor=base,
  },
}

% get rid of these morewrite messages:
% https://tex.stackexchange.com/questions/419489/suppressing-messages-to-standard-output-from-package-morewrites/419494#419494
\ExplSyntaxOn
\cs_set_protected:Npn \__morewrites_shipout_ii:
  {
    \__morewrites_before_shipout:
    \__morewrites_tex_shipout:w \tex_box:D \g__morewrites_shipout_box
    \edef\tmp{\interactionmode\the\interactionmode\space}\batchmode\__morewrites_after_shipout:\tmp
  }
\ExplSyntaxOff


% This is for places where authors used bold. I replace them by \emph
% but have the information where the bold was. St. Mü. 09.05.2020
\newcommand{\textbfemph}[1]{\emph{#1}}



% Felix 09.06.2020: copy code from the third line into localcommands.tex:
% https://github.com/langsci/langscibook#defined-environments-commands-etc
% Does not work with texlive 2020, is done with sed in Makefile
%\patchcmd{\mkbibindexname}{\ifdefvoid{#3}{}{\MakeCapital{#3} }}{\ifdefvoid{#3}{}{#3 }}{}{\AtEndDocument{\typeout{mkbibindexname could not be patched.}}}



\let\textnobf\textit
% instead of "in bold" write "in italics"
\newcommand{\bolddescriptionintext}{italics\xspace}

% Berthold
\newcommand{\mathplus}{+}
% \mbox{\normalfont +}}
\newcommand{\emdash}{--\xspace}
\newcommand{\emdashUS}{--\xspace}


% Stefan to get the space remvoed infront of the : in Bargmann NPN discussion
%\DeclareMathSymbol{:}{\mathord}{operators}{"3A}
% used {:\,} instead


% for cxg.tex needed for includonly to find the counter.
\newcounter{croftyears} 




% Needed for bibtex entry for Jackendoff's xbar syntax. Without it the bar would be off in itialics.

% https://tex.stackexchange.com/questions/95014/aligning-overline-to-italics-font/95079#95079
% \newbox\usefulbox

% \makeatletter
%     \def\getslant #1{\strip@pt\fontdimen1 #1}

%     \def\skoverline #1{\mathchoice
%      {{\setbox\usefulbox=\hbox{$\m@th\displaystyle #1$}%
%         \dimen@ \getslant\the\textfont\symletters \ht\usefulbox
%         \divide\dimen@ \tw@ 
%         \kern\dimen@ 
%         \overline{\kern-\dimen@ \box\usefulbox\kern\dimen@ }\kern-\dimen@ }}
%      {{\setbox\usefulbox=\hbox{$\m@th\textstyle #1$}%
%         \dimen@ \getslant\the\textfont\symletters \ht\usefulbox
%         \divide\dimen@ \tw@ 
%         \kern\dimen@ 
%         \overline{\kern-\dimen@ \box\usefulbox\kern\dimen@ }\kern-\dimen@ }}
%      {{\setbox\usefulbox=\hbox{$\m@th\scriptstyle #1$}%
%         \dimen@ \getslant\the\scriptfont\symletters \ht\usefulbox
%         \divide\dimen@ \tw@ 
%         \kern\dimen@ 
%         \overline{\kern-\dimen@ \box\usefulbox\kern\dimen@ }\kern-\dimen@ }}
%      {{\setbox\usefulbox=\hbox{$\m@th\scriptscriptstyle #1$}%
%         \dimen@ \getslant\the\scriptscriptfont\symletters \ht\usefulbox
%         \divide\dimen@ \tw@ 
%         \kern\dimen@ 
%         \overline{\kern-\dimen@ \box\usefulbox\kern\dimen@ }\kern-\dimen@ }}%
%      {}}
%     \makeatother
