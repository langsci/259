%% -*- coding:utf-8 -*-
\documentclass[output=book
		,modfonts
		,nonflat
	        ,collection
	        ,collectionchapter
	        ,collectiontoclongg
 	        ,biblatex  
                ,babelshorthands
%                ,showindex
                ,newtxmath
                ,colorlinks, citecolor=brown 
                ,draftmode
% 	        ,coverus
		  ]{langscibook}                              
%%%%%%%%%%%%%%%%%%%%%%%%%%%%%%%%%%%%%%%%%%%%%%%%%%%%

% put all additional commands you need in the 
% following files. If you do not know what this might 
% mean, you can safely ignore this section

\title{The one and only handbook of Head-Driven Phrase Structure Grammar}  %look no further, you can change those things right here.
\subtitle{}
% \BackTitle{Change your backtitle in localmetadata.tex} % Change if BackTitle != Title
\BackBody{change blurb in localmetadata.tex
 }
%\dedication{Change dedication in localmetadata.tex}
%\typesetter{Change typesetter in localmetadata.tex}
%\proofreader{Change proofreaders in localmetadata.tex}
\author{Anne Abeillé\and Robert D. Bors­ley\and Jean-​Pierre Koenig\lastand Stefan Müller}
% \BookDOI{}%ask coordinator for DOI
\renewcommand{\lsISBNdigital}{000-0-000000-00-0}
\renewcommand{\lsISBNhardcover}{000-0-000000-00-0}
\renewcommand{\lsISBNsoftcover}{000-0-000000-00-0}
\renewcommand{\lsISBNsoftcoverus}{000-0-000000-00-0}
\renewcommand{\lsSeries}{eotms} % use lowercase acronym, e.g. sidl, eotms, tgdi
\renewcommand{\lsSeriesNumber}{99} %will be assigned when the book enters the proofreading stage
% \renewcommand{\lsURL}{http://langsci-press.org/catalog/book/000} % contact the coordinator for the right number

% add all extra packages you need to load to this file 

\usepackage{graphicx}
\usepackage{tabularx}
\usepackage{amsmath} 
\usepackage{multicol}
\usepackage{lipsum}
%%%%%%%%%%%%%%%%%%%%%%%%%%%%%%%%%%%%%%%%%%%%%%%%%%%%
%%%                                              %%%
%%%           Examples                           %%%
%%%                                              %%%
%%%%%%%%%%%%%%%%%%%%%%%%%%%%%%%%%%%%%%%%%%%%%%%%%%%%
% remove the percentage signs in the following lines
% if your book makes use of linguistic examples


\usepackage{./langsci/styles/langsci-gb4e} 
\usepackage{./langsci/styles/langsci-optional} 
\usepackage{./langsci/styles/langsci-lgr}
\usepackage{./langsci/styles/langsci-forest-setup}
\usepackage{morewrites}



% Stefan Müller's styles
\usepackage{./styles/merkmalstruktur,./styles/abbrev,./styles/makros.2e,./styles/my-xspace,./styles/article-ex,
./styles/eng-date}

\usepackage{./langsci/styles/jambox}

% Crossing out text
% uncomment when needed
%\usepackage{ulem}

\usepackage{./styles/additional-langsci-index-shortcuts}

\usepackage{./styles/avm+}

\renewcommand{\tpv}[1]{{\avmjvalfont\itshape #1}}

\regAvmFonts

\usepackage{theorem}

\newtheorem{mydefinition}{Def.}
\newtheorem{principle}{Principle}

{\theoremstyle{break}
\newtheorem{schema}{Schema}
\newtheorem{mydefinition-break}[mydefinition]{Def.}
\newtheorem{principle-break}[principle]{Principle}
}

\usepackage{subfig}

%% hyphenation points for line breaks
%% Normally, automatic hyphenation in LaTeX is very good
%% If a word is mis-hyphenated, add it to this file
%%
%% add information to TeX file before \begin{document} with:
%% %% hyphenation points for line breaks
%% Normally, automatic hyphenation in LaTeX is very good
%% If a word is mis-hyphenated, add it to this file
%%
%% add information to TeX file before \begin{document} with:
%% %% hyphenation points for line breaks
%% Normally, automatic hyphenation in LaTeX is very good
%% If a word is mis-hyphenated, add it to this file
%%
%% add information to TeX file before \begin{document} with:
%% \include{localhyphenation}
\hyphenation{
A-la-hver-dzhie-va
anaph-o-ra
affri-ca-te
affri-ca-tes
Atha-bas-kan
com-ple-ments
Da-ge-stan
Dor-drecht
er-klä-ren-de
Ginz-burg
Gro-ning-en
Jon-a-than
Ka-tho-lie-ke
Ko-bon
krie-gen
Le-Sourd
moth-er
Mül-ler
Nie-mey-er
Prze-piór-kow-ski
phe-nom-e-non
re-nowned
Rie-he-mann
un-bound-ed
}

% why has "erklärende" be listed here? I specified langid in bibtex item. Something is still not working with hyphenation.


% to do: check
%  Alahverdzhieva

\hyphenation{
A-la-hver-dzhie-va
anaph-o-ra
affri-ca-te
affri-ca-tes
Atha-bas-kan
com-ple-ments
Da-ge-stan
Dor-drecht
er-klä-ren-de
Ginz-burg
Gro-ning-en
Jon-a-than
Ka-tho-lie-ke
Ko-bon
krie-gen
Le-Sourd
moth-er
Mül-ler
Nie-mey-er
Prze-piór-kow-ski
phe-nom-e-non
re-nowned
Rie-he-mann
un-bound-ed
}

% why has "erklärende" be listed here? I specified langid in bibtex item. Something is still not working with hyphenation.


% to do: check
%  Alahverdzhieva

\hyphenation{
A-la-hver-dzhie-va
anaph-o-ra
affri-ca-te
affri-ca-tes
Atha-bas-kan
com-ple-ments
Da-ge-stan
Dor-drecht
er-klä-ren-de
Ginz-burg
Gro-ning-en
Jon-a-than
Ka-tho-lie-ke
Ko-bon
krie-gen
Le-Sourd
moth-er
Mül-ler
Nie-mey-er
Prze-piór-kow-ski
phe-nom-e-non
re-nowned
Rie-he-mann
un-bound-ed
}

% why has "erklärende" be listed here? I specified langid in bibtex item. Something is still not working with hyphenation.


% to do: check
%  Alahverdzhieva

%add all your local new commands to this file

\makeatletter
\def\blx@maxline{77}
\makeatother 

\usepackage{lipsum}

\begin{document}

\maketitle                
\frontmatter

\currentpdfbookmark{Contents}{name} % adds a PDF bookmark
%%
\mainmatter          

Basic case:
\ea
\avm{
	[\type*{sign}
	phon & \list*(phon)\\
	synsem &	[\type*{synsem}
				local &	[\type*{local}
						category &	[\type*{category}
									head & part-of-speech\\
									subj & \list*(synsem)\\
									comps & \list*(synsem)]\\
						content & \ldots\\
						context & \ldots]\\
				nonlocal \ldots]
	]
}
\avm[delimfactor=950, delimfall=5pt]{
	[\type*{sign}
	phon & \list*(phon)\\
	synsem &	[\type*{synsem}
				local &	[\type*{local}
						category &	[\type*{category}
									head & part-of-speech\\
									subj & \list*(synsem)\\
									comps & \list*(synsem)]\\
						content & \ldots\\
						context & \ldots]\\
				nonlocal \ldots]
	]
}
\avm[delimfactor=980, delimfall=8pt]{
	[\type*{sign}
	phon & \list*(phon)\\
	synsem &	[\type*{synsem}
				local &	[\type*{local}
						category &	[\type*{category}
									head & part-of-speech\\
									subj & \list*(synsem)\\
									comps & \list*(synsem)]\\
						content & \ldots\\
						context & \ldots]\\
				nonlocal \ldots]
	]
}
\z

The last example seems to be about right. The manual should explain what the delimfactor and
delimfall values do and give some examples.


Special cases:


\begin{enumerate}
\item \#49 The distance after numerated examples with AVMs is too small (order.tex)

\begin{exe}\ex\label{x:rc-79}  Lexical rule for propositional attitude predicates in \ili{French}\\
\avm{
	[ % cat & \[ head & v \]\\
	comps & < CP[slash & \{ \1 [\type*{prl}
		                    cont|index & \2 ] \} $\cup$ \3 ]  > \+ \ldots\\
        slash & \{ \1 \} $\cup$ \4 ]
}
$\mapsto$\\
\flushright
\avm{
	[slash & \{ [\type*{nprl}
		     cat & \upshape PP\subscr{de \2} ] \} $\cup$ \4 ]
}
\end{exe}%
\lipsum[1][1-4]

\item \#59, \#115: Brackets () are too tall.
\ea
\avm{
[\type*{passive-verb}
 arg-st  & \2 ( \+ < ! PP[\type{by}]$_i$ ! > )\\
 lex-dtr & [\type*{stem}
            head   & verb\\
            arg-st & < \1$_i$ > \+ \2 < NP, \ldots > ]]  
}
\z

\ea 
\avm{
\type{coord-phrase} \impl\\
	[\punk{synsem|cat}{\1} \\
	dtrs & <[synsem|loc|cat & \1 ]> \+ ne-list([synsem|loc|cat & \1 ]) ]
}\label{coordparam2}
\z

\eal
\ex
\avm{
\emph{simple-coord-phrase} \impl \\
	[dtrs & ne-list(<[coord & none]>) \+ ne-list([coord & \1 crd ]) ]
}

\ex
\avm{
\emph{omnisyndetic-coord-phrase}  \impl \\
	[dtrs & ne-list([coord & \1 crd ]) ]
}\label{omni}
\ex
\avm{
\emph{asyndetic-coord-phrase}  \impl \\
	[dtrs & ne-list([coord & none ]) ]
}
\zl



\item \#64: spacing between values is different when \verb+\punk+ is involved

\ea
\avm{
\type{nom-coord-phrase} \impl
	[\punk{synsem|loc|cont|index}{ \1} \\
	 dtrs & < [ ], \ldots, [synsem|loc|cont|index & \1 ] > ]
}
\z

\item \#65:

\ea
\avm{
  [lhs : & [cat : & \textit{command} \\
            modality : & \2 \\
            content : & \1 \\
            time : & \3 ] \\
   rhs : & [dtr1 : & [cat : & \textit{located\_command} \\
                      modality : & \6 \\
                      content : & \1!\textup{[}location\textup{]}! \\
                      time : & \7
                      ] \\
            dtr2 : & [cat : & \textit{spatial\_gesture} \\
                      content : & \5 \\
                      modality :& \9 \\
                      time :& \tag{10}
                      ]
            ] \\
   constraints : & \{overlap(\7,\tag{10}) $\lor$ follow(\7,\tag{10},\texttt{4}s) \\
                     total-time(\7,\tag{10},\3) \\ 
                     assign-modality(\6,\9,\2)
                    \}
   ]
}
\z

\item \#70: very often AVMs without alignment are needed. There can be cases where the alignment is needed
on some levels but not at all levels.
\eal
\ex \avm{
[\punk{mother|syn|val|comps}{< >} \\
head-dtr & \0	[\type*{word}
                syn|val|comps & \tag{A} ] \\
dtrs & < \0 > \+ \tag{A} nelist ]
}
\ex \avm{
[\type*{sem-obj}
index  & s \\
frames & <[\type*{eating-fr}
	   label & l\\
	   sit   & s\\
	   ingestor & i\\
	   ingestible & j]> ]
}
\ex \avm{
[\phon < I > \\
content|index & \1 [per & 1st \\
		    num & sg ] \\
context &  !$speaker(\1)$! 
]
}

\ex 

\avm{
[ head & \1\\
  head-dtr & [ head & \1 ] ]
}


\zl

Desired encoding: (with \verb+\phon+ working)
\eal
\ex \avm{
[phon < I > \\
content|index  \1 [per & 1st \\
		   num & sg ] \\
context   !$speaker(\1)$! ]
}


\ex \avm{
[\type*{sign}
phonology list!(phon\_string)! \\
synsem 	[local &	[category & category \\
						content & content \\
						qstore & set!(quantifier)! \\
						pool & set!(quantifier)! ] ] \\
retrieved  list!(quantifier)! ]
}
\zl

\item \#71 alignment and escape via !

\ea
\avm{
	[\type*{reg-bar-adj}
	\punk{phon}{\1 \+ \phonliste{ bar }} \\
	\punk{morph-b}{<[\type*{trans-verb}
			  phon \; \1 \\
			  synsem|loc  [cat|comps & < NP ![\type{acc}]!\ind{2} > \+ \3 \\
			               cont|nuc  & \4 [act & index \\
			                               und & \2] ] ] >} \\
	synsem|loc &	[cat	[head  & adj \\
				 subj  & < NP\ind{2} > \\
				 comps & \3 ] \\
			 cont|nuc  [reln    & $\diamond$ \\
				    und     & \2 \\
				    soa-arg & \4 ] ] ]
}
\z

\item \#78 distance too narrow between lines in avms (probably some setting)

\ea
\avm{
	[\type*{scope-obj}
	index & \1 \\
	restr &	\{
			[\type*{fact}
			prop|soa  &	[\type*{soa}
						nuc &	[\type*{person}
								instance & \1 ] ] ], \\
            [\type*{fact}
            prop|soa &	[\type*{soa}
            			nuc &	[\type*{speak\_to}
            					speaker & Kim\\
            					addressee & \1 ] ] ]
      		\} 
	] 
}
\z

\item \#79: The alignment of tags is to the number not to the box

\ea
\avm{\emph{head-filler-phrase} \impl\\
[slash  & \1 \\
 hd-dtr & [head & verbal \\
 	   slash & \{ \2 \} $\uplus$ \1 ] \\
 \punk{non-hd-dtrs}{ <[local &  \2 ]>} ]
}
\z
I am not sure whether I want the box to be aligned or the number.


\item \#81: The text is too high in \verb+< >+.

\ea
\avm{
[subj  & < PRO >]
}
\z


\item \#85: AVMs in type hierarchies are not at the top

\begin{forest} 
type hierarchy
[object,
    calign=midpoint, calign children={2}{3},
    [\avm{[\type*{substantive} 
           prd & boolean]}
      [\avm{[\type*{verb}
             vform & vform\\
             prd & plus]}]
      [\avm{[\type*{noun} 
             case & case]}]
    ]
    [case]
    [vform]
    [boolean
      [plus]
      [minus]
    ]
  ]
\end{forest}

\item \#92: commas and angle brackets with AVMs and line breakes, angle brackets too small
\ea
\avm{
[\punk{synsem|loc|cont|index}{[num & pl \\
	                  gend & [m & \1 \+ \2 ] \\
                          per & [me  & \3 \+ \5\\
                              you & \4 \+ \6 ] ]} \\
 dtrs &	< [synsem|loc|cont|index & [gend & [m & \1] \\
                       	            per & [me  & \3 \\
                                           you & \4] ] ],\\
	  [synsem|loc|cont|index & [gend & [m & \2] \\
                                    per & [me  & \5 \\
                                           you & \6] ] ]  >
	]
}
\z
The brackets seem to not see the comma. Apart from this they are too small (should span the whole AVMs)


\item \#93 something is needed for aligning AVM at top \verb+\atop+ does not seem to work


I want both AVMs to appear at the top right below the heading. \verb+\attop+ does not work since it puts the following AVM on the next line.
\ea
\label{entryfori}
Simplified lexical signs for  \textit{I} and \textit{she}: \\*
\avm{
[\phon < I > \\
head &	[\type*{noun}
		pers & 1st \\
		num & sg ] ] 
}\hspace{1cm}
\avm{
[\phon  < she > \\
head &	[\type*{noun}
		pers & 3rd \\
		num & sg \\
		gen & fem ] ] 
}
\z
Felix provided solution with minipage, but could there be an option for calling \verb+\avm+? And if
so, could forest enable this by default?


\item \#94: The I touches the bracket.

\ea
\avm{
[\phon < I > \\
head &	[\type*{noun}
		pers & 1st \\
		num & sg ] ] 
}
\z

Felix: This is normal in \LaTeX. Stefan: Yes, but there is italics correction and it has to be
applied here. Always.


\item \#97: Move type up? Not for types including letters like \emph{t} or \emph{l}.

\ea
% \avm[delimfactor=950, delimfall=5pt]{
% [\type*{sign}
% phonology & list!(phon\_string)!  ]
% }
\avm{
[\type*{sign}
phonology & list!(phon\_string)!  ]
}
\avm{
[\type*{head}
 vform & fin ]
}
\z

Felix: ``I don't think this is a good idea. For one, consider what would happen if two types are
placed in one line (e.g. two AVMs beside each other) and then the baselines of these types would not
be identical. This would be terrible.''

``Maybe what you really want isn't so much messing around with individual letters, but rather setting tighter options for the enclosing bracket delimiter, e.g.: \verb+\avm[delimfactor=950, delimfall=5pt] {}+.''

Interestingly the delimfactor does not seem to make a difference for simple AVMs like the ones in
(\mex{0}), but it does affect the ones at the very beginning.


\item \#99: space after \verb+\uplus+, \#106: space is different for \verb-\+- within the same avm


\ea
\avm{
\emph{wh-inter-cl}:
[cont & [params & \{ \1 \} $\uplus$ set] ]
}
\z

And this is \#106:
\ea
\avm{
[comps & \1 \+ <[loc & \4 [cat [head & verb $\lor$ prep \\
			         comps & < > ] ] ] > \+ \2  ]}
\z
There was different space after the two $\oplus$. This is fixed. But the space before and after
$\oplus$ is still different.

\ea
\avm{
[ subcat & < > $\lor$ < \type{synsem} > ]}
\z
The space before and after $\lor$ differs.


\item \#101: space is missing due to exclamation marks in avms

\ea
\avm{
	[phon & < \rm{every, dog, barks} >\\
	\ldots\ cont & [main & \tag{4a} {\normalfont \sffamily bark} ] \\
	sem &	[exc & \2 !{\normalfont \sffamily $\forall$x ($\phi$,$\psi$)}! \\
			inc & \4 !{\normalfont \sffamily bark(x)}!\\
			pts & <\2, \tag{2a}, \tag{2b}, \3, \tag{3a}, \4, \tag{4a}>]]
}
\z
The exclamation mark seems to eat the space. Felix: Hardcode space.




\item \#108: Punk causes extra newline

\ea
\avm{
[arg-st & <[arg & \upshape NP![\type{str}]! \\
            \punk{realized}{$+$} ]> ] }
\z 

\item \#109: distance betweeen lines not eaqual when normal brackets are used

\ea
\avm{
		[\type*{phrase}
		phon & \list*(phon)\\
		synsem & synsem\\
		dtrs & \list*(sign) ]
	}
\z

What I want here is the normal brackets: (). The distance between \phon and \synsem is bigger than
between \synsem and \dtrs. This is how it should look like:

\ea
\avm{
		[\type*{phrase}
		phon & \upshape !\textit{list}(\textit{phon})!\\
		synsem & synsem\\
		dtrs & \upshape !\textit{list}(\textit{sign})! ]
	}
\z
I guess the problem is due to the fact that normal brackets are not used but bigger brackets.

\end{enumerate}

\end{document} 

%%%%%%%%%%%%%%%%%%%%%%%%%%%%%%%%%%%%%%%%%%%%%%%%%%%% 
%%%                  END                         %%% 
%%%%%%%%%%%%%%%%%%%%%%%%%%%%%%%%%%%%%%%%%%%%%%%%%%%% 

%%% Local Variables:
%%% mode: latex
%%% TeX-master: t
%%% TeX-engine: xetex
%%% End:
