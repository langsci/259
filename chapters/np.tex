%% -*- coding:utf-8 -*-
\documentclass[output=paper]{langsci/langscibook} 
\author{Frank Van Eynde\affiliation{University of Leuven}}
\title{Nominal structures}

% \chapterDOI{} %will be filled in at production

\epigram{Change epigram in chapters/03.tex or remove it there }
\abstract{Change the  abstract in chapters/03.tex \lipsum[3]}
\maketitle

\begin{document}
\label{chap-np}

\section{Introduction}


We use the term `nominal' in a theory-neutral way as standing for a noun
and its phrasal projection. In this broad sense all of the bracketed 
strings in (\ref{rb}) are nominals. 

\begin{exe} 
\ex\label{rb}  [the [red [box]]] is empty
\end{exe} 
  
\noindent
For the analysis of nominals there are two main approaches in generative grammar. 
One treats the noun (N) as the head all the way through. In that analysis the 
largest bracketed string in (\ref{rb}) is an NP. 
The other makes a distinction between the nominal core, 
consisting of a noun with its complements and modifiers, if any,  
and a functional outer layer, comprising determiners, quantifiers and 
numerals. In that analysis the noun is the head of {\it red box}, 
while the determiner is the head of {\it the red box\/}, so that the category 
of the latter is DP. 

The first approach, henceforth called the {\bf NP approach}, prevailed in 
generative grammar up to and including the Government and Binding model
\citep{Chomsky81}. One of its modules, the categorial component, 
consists of phrase structure rules, such as those in (\ref{ps0}). 

\begin{exe} 
\ex\label{ps0}   
\begin{xlist}
\ex  VP ~ $\rightarrow$ ~ V ~ NP 
\ex  NP ~ $\rightarrow$ ~ Det ~ Nom
\end{xlist} 
\end{exe}

\noindent
They are required to ``meet some variety of X-bar theory'' \citep[5]{Chomsky81}. 
The original variety is that of \citet{Chomsky70}. It consists of the following
cross-categorial rule schemata:

\begin{exe} 
\ex\label{xbar} 
\begin{xlist} 
\ex\label{xbar1}   X$'$ ~ $\rightarrow$ ~ X ~  ... 
\ex\label{xbar2}   X$''$ ~ $\rightarrow$ ~ [Spec, X$'$] ~ X$'$ 
\end{xlist} 
\end{exe} 

\noindent
X$'$ stands for the combination of a head X and its complements,
where X is N, A or V, and X$''$ stands for the combination of 
X$'$ and its specifier ``where [Spec, N$'$] will be analyzed as the determiner'' 
\citep[52]{Chomsky70}. 
X-bar theory was further developed in \citet{Jackendoff77}, who added a
schema for the addition of adjuncts and who extended the range of 
X with P, the category of adpositions. Generalized Phrase Structure Grammar 
developed a monostratal version of it, exemplified by the analysis 
in Figure \ref{sis}, quoted from \citet[126]{GPSG85}. 
The top node is the double-bar category N$''$, which 
consists of the determiner and the single-bar category N$'$. 
The AP and the relative clause are adjoined to N$'$, and 
the lowest N$'$ consists of the noun and its PP complement.

\begin{figure}
\begin{center}
\footnotesize
\tree{\ntnode{Za}{N$''$},
  {\ntnode{Zd}{Det},
    {\tnode{Ze}{that}}},
  {\ntnode{Zf}{N$'$},
    {\ntnode{Zc}{AP},
      {\tnode{Zl}{very tall}}},
    {\ntnode{Zp}{N$'$},
      {\ntnode{Zg}{N$'$},
        {\ntnode{Zi}{N},
          {\tnode{Zb}{sister}}},
        {\ntnode{Zk}{PP}, 
          {\tnode{Zj}{of Leslie}}}},
      {\ntnode{Zh}{S [+R]}, 
        {\tnode{Zq}{who we met}}}}}}
\nodeconnect{Za}{Zd}
\nodeconnect{Zd}{Ze}
\nodeconnect{Za}{Zf}
\nodeconnect{Zf}{Zc}
\nodetriangle{Zc}{Zl}
\nodeconnect{Zf}{Zp}
\nodeconnect{Zp}{Zg}
\nodeconnect{Zg}{Zi}
\nodeconnect{Zi}{Zb}
\nodeconnect{Zg}{Zk}
\nodetriangle{Zk}{Zj}
\nodeconnect{Zp}{Zh}
\nodetriangle{Zh}{Zq}
\caption{\label{sis} The NP approach} 
\normalsize
\end{center}
\end{figure}
 
The second approach, henceforth called the {\bf DP approach}, results from an
extension of the range of X in (\ref{xbar}) to the functional categories. 
This was motivated by the fact that some of the phrase structure rules, 
such as (\ref{ps1}), do not fit the X-bar mould. 

\begin{exe} 
\ex\label{ps1}   S ~ $\rightarrow$ ~ NP ~ Aux ~ VP
\end{exe}   

\noindent
To repair this, the category Aux, which contained both auxiliaries and 
inflectional verbal affixes \citep{Chomsky57}, was renamed as I(nfl) and treated as the head of S. 
More specifically, I(nfl) was claimed to combine with a VP complement, yielding I$'$, 
and the latter was claimed to combine with an NP specifier (the subject), yielding I$''$
(formerly S).
For the analysis of nominals such an overhaul did not at first seem necessary, 
since the relevant PS rules did fit the X-bar mould, but it took place nonetheless, 
mainly in order to capture similarities between nominal and clausal structures. 
These are especially conspicuous in gerunds, nominalized infinitives and nominals 
with a deverbal head, and were seen as evidence for the claim that determiners have their 
own phrasal projection, just like the members of I(nfl) \citep{Abney87}. 
More specifically, members of D were claimed to take an N$''$ complement (formerly Nom), 
yielding D$'$, and the latter was claimed to have a specifier sister, as in Figure \ref{abn}.
The DP approach was also taken on board in other frameworks, 
such as Word Grammar \citep{Hudson90} and Lexical Functional Grammar \citep[99]{Bresnan00}. 

\begin{figure}
\begin{center}
\footnotesize
\tree
{\ntnode{Za}{D$''$},
  {\ntnode{Zc}{D$'$},
    {\ntnode{Zd}{D},
      {\tnode{Ze}{that}}},
    {\ntnode{Zf}{N$''$},
      {\ntnode{Zg}{N$'$},
        {\ntnode{Zi}{N},
          {\tnode{Zj}{sister}}},
        {\ntnode{Zk}{PP}, 
          {\tnode{Zq}{of Leslie}}}}}}}
\nodeconnect{Za}{Zc} 
\nodeconnect{Zc}{Zd}
\nodeconnect{Zd}{Ze}
\nodeconnect{Zc}{Zf} 
\nodeconnect{Zf}{Zg}
\nodeconnect{Zg}{Zi}
\nodeconnect{Zi}{Zj}
\nodeconnect{Zg}{Zk}
\nodetriangle{Zk}{Zq}
\caption{\label{abn} The DP approach } 
\normalsize
\end{center}
\end{figure}
     
Turning now to Head-driven Phrase Structure Grammar we find three different treatments.  
The first and oldest can be characterized as a lexicalist version of the NP approach, more 
specifically of its monostratal formulation in GPSG.  
It is first proposed in \citet{PS87} and further developed in \citet{PS94} and 
\citet{GS00}. We henceforth call it the {\bf specifier treatment}, 
after the role which it assigns to the determiner. 
The second is a lexicalist version of the DP approach.  
It is first proposed in \citet{Netter94} and further developed in \citet{Netter96},
and \citet{NerbonneMullen00}. We will call it the {\bf DP treatment}. 
The third adopts the NP approach, but neutralizes the distinction between adjuncts and specifiers, 
treating them both as functors. It is first proposed in \citet{VanEynde98a} and 
\citet{Allegranza98} and further developed in \citet{VanEynde03}, \citet{VanEynde06} 
and \citet{Allegranza06}. It is also adopted in Sign-Based Construction Grammar \citep{Sag2012}. 
We will call it the {\bf functor treatment}. This chapter presents the three treatments and 
compares them wherever this seems appropriate.  

We first focus on ordinary nominals (Section 2) and then on nominals with idiosyncratic 
properties (Section 3). For exemplification we use English and a number of other Germanic 
and Romance languages, including Dutch, German, Italian and French.  
We assume familiarity with the typed feature structure notation and with such basic notions 
as unification, inheritance and token-identity. 
    

\section{Ordinary nominals} 


We use the term `ordinary nominal' for a nominal that contains a noun, 
any number of complements and/or adjuncts and at most one determiner. 
This section shows how such nominals are analyzed in respectively the 
specifier treatment (2.1), the functor treatment (2.2) and the DP treatment (2.3).  

    
\subsection{The specifier treatment} 


The specifier treatment adopts the same distinction between heads, complements, 
specifiers and adjuncts as X-bar syntax, but the integration of these notions 
in a monostratal lexicalist framework inevitably leads to various differences. 
The presentation is mainly based on \citet{PS94} and \citet{GS00}. 
We first discuss the syntactic structure (2.1.1) and the semantic composition (2.1.2) 
of nominals, and then turn to nominals with a phrasal specifier (2.1.3). 


\subsubsection{Syntactic structure}


Continuing with the same example as in Figures 1 and 2, 
a relational noun, such as {\it sister\/}, selects a PP as its complement 
and a determiner as its specifier, as spelled out in its {\sc category} value
(\ref{n}). 

\begin{exe} 
\ex\label{n}
\begin{avm}
[{\it category\/}         \\ 
 head  & {\it noun\/}     \\
 spr   & <Det>            \\
 comps & <PP\[{\it of\/}\]> ]
\end{avm}
\end{exe}

\noindent
The combination with a matching PP, as in {\it sister of Leslie},   
is subsumed by the {\it head-comp(lement)s-phr\/} type, as defined in 
%\crossrefchaptert{basic-properties} 
Chapter 1 of this Volume, 
and yields a nominal with an empty {\sc comps} list.  
Similarly, the combination of this nominal with a matching determiner, as in {\it that sister of Leslie},    
is subsumed by the {\it head-sp(ecifie)r-phr\/} type, as defined in Chapter 1 of this Volume, 
%\crossrefchaptert{basic-properties}. 
and yields a nominal with an empty {\sc spr} list. This is spelled out in Figure \ref{les}. 

\begin{figure}
\begin{center}
\footnotesize
\tree
{\ntnode{Za}{[{\sc head} ~ \avmbox{1} {\it noun} , {\sc spr} ~ $<$ ~ $>$ , {\sc comps} ~ $<$ ~ $>$]},
  {\ntnode{Zd}{\avmbox{2}},
    {\tnode{Ze}{that}}},
  {\ntnode{Zf}{[{\sc head} ~ \avmbox{1} , {\sc spr} ~ $<$\avmbox{2}$>$ , {\sc comps} ~ $<$ ~ $>$]},
    {\ntnode{Zi}{[{\sc head} ~ \avmbox{1} , {\sc spr} ~ $<$\avmbox{2} Det$>$ , {\sc comps} ~ $<$\avmbox{3} PP$>$]},
      {\tnode{Zj}{sister}}},
    {\ntnode{Zk}{\avmbox{3}}, 
      {\tnode{Zh}{of Leslie}}}}}
\nodeconnect{Za}{Zd}
\nodeconnect{Zd}{Ze}
\nodeconnect{Za}{Zf}
\nodeconnect{Zf}{Zi}
\nodeconnect{Zi}{Zj}
\nodeconnect{Zf}{Zk}
\nodetriangle{Zk}{Zh}
\caption{\label{les} Adnominal Complements and Specifiers  }
\normalsize
\end{center}
\end{figure}

Since the noun is the head of {\it sister of Leslie\/} and since the latter is 
the head of {\it that sister of Leslie}, the Head Feature Principle implies 
that the phrase as a whole shares the {\sc head} value of the noun ($\avmbox{1}$). 
The valence features, {\sc comps} and {\sc spr}, have a double role. 
On the one hand, they register the degree of saturation of the nominal; 
in this role they supersede the bar levels of X-bar theory. 
On the other hand, they capture co-occurrence restrictions, 
such as the fact that the complement of {\it sister\/} be a PP, rather than an NP or a clause, 
and that its specifier be a singular determiner, rather than a plural one.\footnote{The 
value of the third valence feature ({\sc subj}) is invariably the empty list for nouns, 
except in the case of predicative nouns, whose {\sc subj} list contains an NP 
that is identified with the target of the predication, see \citet[409]{GS00}.}

In contrast to complements and specifiers, adjuncts are not included in valence lists, 
since their addition has no effect on the degree of saturation. At the same time, 
since the combination of an adjunct with its head is subject to 
co-occurrence restrictions, one needs a way to capture those. 
For that purpose \citet[55--57]{PS94} employs the feature {\sc mod(ified)}. 
It is part of the {\sc head} value of the substantive parts-of-speech, 
i.e. noun, verb, adjective and preposition. Its value is of type {\it synsem\/} 
in the case of modifying items and of type {\it none\/} otherwise.

\begin{exe} 
\ex   {\it substantive} : \begin{avm} 
                          [mod ~ {\it synsem\/} $\vee$ {\it none\/}]  
                          \end{avm} 
\end{exe} 

\noindent
As an example, let us take the number and gender agreement 
between nouns and adjectives in the Romance languages.\footnote{This is an 
instance of concord, as defined in 
%\crossrefchaptert{agreement}
Chapter 6 of this Volume.}  
The Italian {\it grossa\/} `big', for instance, 
is compatible with singular feminine nouns, such as {\it scatola\/} `box', 
but not with plural feminine nouns, such as {\it scatole\/} `boxes', nor
with masculine nouns, such as {\it libro\/} `book' or {\it libri\/} `books'. 
This is made explicit in its {\sc mod} value, spelled out in (\ref{rd}). 

\begin{exe} 
\ex\label{rd}
\begin{avm}
[{\it category\/}                              \\
 head [{\it adjective}                         \\
       mod|loc|cat [head [{\it noun\/}          \\
                          number ~ {\it sing\/} \\
                          gender ~ {\it fem\/}] \\
                    spr ~ <Det> ]]]
\end{avm}
\end{exe}

\noindent
The token-identity of the {\sc mod(ified)} value of the adjective
with the {\sc synsem} value of its head sister is part of the 
definition of type {\it head-adj(unct)-phr}, as defined in 
%\crossrefchaptert{basic-properties}
Chapter 1 of this Volume. 
Since the {\sc mod(ified)} feature is part of the {\sc head} value, it follows from the 
Head Feature Principle that it is shared between an adjective 
and the AP which it projects. As a consequence, the {\sc mod(ified)} value of 
{\it molto grossa\/} `very big' is shared with that of {\it grossa\/} `big'. 
 
Besides its role in modeling agreement, the {\sc mod} value is also instrumental 
in capturing constraints on linear order.  
The fact, for instance, that the AP in {\it that very tall bridge\/} appears 
after the determiner and not before, as in *{\it very tall that bridge},
is captured by the stipulation that the {\sc mod} value of the adjective 
is a nominal with a determiner on its {\sc spr} list. This blocks the 
combination with a fully saturated NP, as in 
*{\it very tall that bridge}.\footnote{This order constraint is overruled in 
the Big Mess Construction, see Section 3.2.} Also here, the {\sc mod} value of 
the adjective is shared with that of the AP, as shown in Figure \ref{lea}. 

\begin{figure}
\begin{center}
\footnotesize
\tree
{\ntnode{Za}{[{\sc head} ~ \avmbox{1} {\it noun} , {\sc spr} ~ $<$ ~ $>$]},
  {\ntnode{Zd}{\avmbox{2}},
    {\tnode{Ze}{that}}},
  {\ntnode{Zf}{[{\sc head} ~ \avmbox{1} , {\sc spr} ~ $<$\avmbox{2}$>$]},
    {\ntnode{Zl}{[{\sc head} ~ \avmbox{4}]},
      {\ntnode{Zu}{[{\sc head} ~ {\it adv\/}]},
        {\tnode{Zv}{very}}},
      {\ntnode{Zw}{[{\sc head} ~ \avmbox{4} [{\it adj\/} {\sc mod} ~ \avmbox{3}]]},
        {\tnode{Zm}{tall}}}},
    {\ntnode{Zg}{\avmbox{3} [{\sc head} ~ \avmbox{1} , {\sc spr} ~ $<$\avmbox{2} Det$>$]},
      {\tnode{Zj}{sister of Leslie}}}}}
\nodeconnect{Za}{Zd}
\nodeconnect{Zd}{Ze}
\nodeconnect{Za}{Zf}
\nodeconnect{Zf}{Zl}
\nodeconnect{Zl}{Zu}
\nodeconnect{Zu}{Zv}
\nodeconnect{Zl}{Zw}
\nodeconnect{Zw}{Zm}
\nodeconnect{Zf}{Zg}
\nodetriangle{Zg}{Zj}
\caption{\label{lea} Adnominal Modifiers}
\normalsize
\end{center}
\end{figure}
 

\subsubsection{Semantic composition}


Given the monostratal nature of {\sc hpsg}, semantic representations 
do not constitute a separate level of representation, but take the form 
of attribute value pairs that are added to the syntactic representations.   
Phrase formation and semantic composition are, hence, modeled in tandem.  
Technically, the {\sc content} feature is declared for the same type of objects 
as the {\sc category} feature. 

\begin{exe} 
\ex  {\it local} : \begin{avm} 
                   [cat ~ {\it category} \\
                    content ~ {\it semantic-object\/}]
                   \end{avm} 
\end{exe} 

\noindent 
For nominals the value of the {\sc content} feature is of type {\it scope-obj(ect)\/} 
in \citet{GS00}. 
A scope-object is an index-restriction pair in which the index stands for 
entities and in which the restriction is a set of facts which constrain the 
denotation of the index, as in the {\sc content} value of the noun {\it box\/}:    

\begin{exe} 
\ex\label{red} 
\begin{avm}
[{\it scope-obj\/}        \\
 index ~ @1 {\it index\/} \\
 restr ~ \{[{\it box\/}   \\
            arg ~ @1 ]\}]
\end{avm} 
\end{exe}

\noindent
This is comparable to the representations which are canonically used in 
Predicate Logic (PL), such as \{x$|${\it box\/}(x)\}, where x stands for 
the entities that the predicate {\it box\/} applies to. In contrast to 
PL variables, HPSG indices are sorted with respect to person, number 
and gender. This provides the means to model the type of agreement that 
is called index agreement in 
%\crossrefchaptert{agreement}
Chapter 6 of this Volume.

\begin{exe} 
\ex  {\it index\/} : \begin{avm}
                     [person ~ {\it person} \\
                      number ~ {\it number} \\
                      gender ~ {\it gender\/}] 
                     \end{avm} 
\end{exe} 

{\sc content} values of attributive adjectives are also of type {\it scope-object}. 
When combined with a noun, as in {\it red box}, the resulting representation 
is one in which the indices of the adjective and the noun are identical, as in 
(\ref{redbox}).\footnote{This is an example of intersective modification. 
The semantic contribution of other types of adjectives, such as  
{\it alleged\/} and {\it fake}, are modeled differently \citep[330--331]{PS94}.}   

\begin{exe} 
\ex\label{redbox} 
\begin{avm}
[{\it scope-obj\/}         \\
 index ~ @1                \\
 restr ~ \{ [{\it red\/}   \\
             arg ~ @1 ] ,
            [{\it box\/}   \\
             arg ~ @1 ]\}]
\end{avm}
\end{exe}

\noindent
Also this is comparable to the PL practice of representing such 
combinations with one variable to which both predicates apply, as in 
\{x$|${\it red}(x) \& {\it box}(x)\}. What triggers the index sharing is 
the {\sc mod(ified)} value of the adjective, as illustrated by the {\sc avm} of 
{\it red\/} in (\ref{reddd}) \citep[55]{PS94}.\footnote{$\avmbox{\Sigma}$ 
stands for an object of type {\it set}, as in \citet{GS00}.} 

\begin{exe} 
\ex\label{reddd}
\begin{avm}
[cat|head [{\it adjective}                               \\
           mod|loc|content [{\it scope-obj\/}            \\
                            index ~ @1                   \\
                            restr ~ $\avmbox{\Sigma}$ ]] \\
 content [index ~ @1                                     \\
          restr ~ \{[{\it red\/}                         \\
                     arg ~ @1 ]\} ~ $\bigcup$ ~ $\avmbox{\Sigma}$ ]]
\end{avm}
\end{exe}

\noindent
The adjective selects a scope-object, shares its index and adds its own 
restriction to those that are already present. The resulting {\sc content} 
value is then shared with the mother.

To model the semantic contribution of determiners, \citet{GS00} 
makes a distinction between scope-objects that contain a quantifier 
({\it quant-rel\/}), and those that do not ({\it parameter\/}). 
In terms of this distinction, the 
addition of a determiner to a nominal, as in {\it every red box}, 
triggers a shift from {\it parameter\/} to {\it quant-rel}. 
To capture this the specifier treatment employs the feature {\sc spec(ified)}. 
It is part of the {\sc head} value of the determiners, and its value is of type 
{\it semantic-obj(ect)}.\footnote{In \citet[45]{PS94} the {\sc spec(ified)}
feature was also assigned to other function words, such as complementizers, 
and its value was of type {\it synsem}.}   

\begin{exe} 
\ex   {\it determiner} : \begin{avm} [spec ~ {\it semantic-obj\/}] \end{avm}  
\end{exe} 

\noindent
In the case of {\it every}, the {\sc spec} value is an object of 
type {\it parameter}, but its own {\sc content} value is a subtype of 
{\it quant-rel} and this quantifier is put in store, to be retrieved 
at the place where its scope is determined, as illustrated by the AVM
of {\it every\/} in (\ref{every}) \citep[204]{GS00}.  

\begin{exe} 
\ex\label{every} 
\begin{avm}
[cat|head [{\it det\/}               \\
           spec [{\it parameter\/}   \\
                 index ~ @1          \\
                 restr ~ $\avmbox{\Sigma}$ ]] \\
 content ~ @2 [{\it every-rel\/}     \\
               index ~ @1            \\
               restr ~ $\avmbox{\Sigma}$ ] \\
 store ~ \{ @2 \}]
\end{avm}
\end{exe}

\noindent 
Notice that the addition of the {\sc spec} feature yields an analysis in which the determiner 
and the nominal select each other: The nominal selects 
its specifier by means of the valence feature {\sc spr} and the determiner selects the nominal 
by means of {\sc spec}.  


\subsubsection{Nominals with a phrasal specifier} 


Specifiers of nominals tend to be single words, but they can also take the form 
of a phrase. The bracketed phrase in [{\it the Queen of England's\/}] {\it sister},
for instance, is in complementary distribution with the possessive
determiner in {\it her sister\/} and has a comparable semantic contribution.   
For this reason it is treated along the same lines. More specifically, the 
possessive marker {\it 's\/} is treated as a determiner that takes an NP as its specifier, 
as shown in Figure \ref{cousin} \citep[51--54]{PS94} and \citep[193]{GS00}.\footnote{The treatment 
of the phonologically reduced {\it 's} as the head of a phrase is comparable to 
the treatment of the homophonous word in {\it he's ill\/} as the head of a VP.
Notice that the possessive {\it 's\/} is not a genitive affix, for if it were, it 
would be affixed to the head noun {\it Queen}, as in *{\it the Queen's of England sister\/}, see  
\citet[199]{SagWasow03}.}

\begin{figure}  
\begin{center}
\footnotesize
\tree
  {\ntnode{Za}{[{\sc head} ~ \avmbox{1} {\it noun} , {\sc spr} ~ $<$ ~ $>$]},
    {\ntnode{Zd}{\avmbox{2} [{\sc head} ~ \avmbox{3} {\it det} , {\sc spr} ~ $<$ ~ $>$]},
      {\ntnode{Zf}{\avmbox{4} [{\sc head} ~ {\it noun} , {\sc spr} ~ $<$ ~ $>$]},
        {\tnode{Zl}{the Queen of England}}},
      {\ntnode{Zg}{[{\sc head} ~ \avmbox{3} , {\sc spr} ~ $<$\avmbox{4}$>$]},
        {\tnode{Zi}{'s}}}},
    {\ntnode{Zk}{[{\sc head} ~ \avmbox{1} , {\sc spr} ~ $<$\avmbox{2}$>$]},
      {\tnode{Zh}{sister}}}}
\nodeconnect{Za}{Zd}
\nodeconnect{Za}{Zk}
\nodeconnect{Zd}{Zf}
\nodeconnect{Zd}{Zg}
\nodetriangle{Zf}{Zl}
\nodeconnect{Zg}{Zi}
\nodeconnect{Zk}{Zh}
\caption{\label{cousin} Phrasal Specifiers } 
\normalsize
\end{center}
\end{figure}

In this anaysis the specifier of {\it sister\/} is a DetP that is headed by {\it 's\/} 
and the latter takes the NP {\it The Queen of England\/} as its specifier.\footnote{Since the 
specifier of {\it 's\/} is an NP, it may in turn contain a specifier that is headed 
by {\it 's}, as in {\it John's uncle's car}.}
Semantically, {\it 's\/} relates the index of its specifier (the possessor) to the index
of the nominal that it selects (the possessed), as spelled out in (\ref{poss}).\footnote{The
terms `possessor' and `possessed' are meant to be understood in a broad not-too-literal sense.}     

\begin{exe} 
\ex\label{poss}
\begin{avm}
[cat [head [{\it det\/}                       \\
            spec [{\it parameter\/}           \\
                  index ~ @1                  \\
                  restr ~ $\avmbox{\Sigma}$ ]] \\
      spr ~ <[index ~ @3]> ]                  \\
 content ~ @2 [{\it the-rel\/}                \\
               index ~ @1                     \\
               restr ~ \{[{\it poss-rel\/}    \\
                          possessor ~ @3      \\
                          possessed ~ @1 ]\} ~ $\bigcup$ ~ $\avmbox{\Sigma}$ ] \\
 store ~ \{ @2 \}]  
\end{avm}
\end{exe}

\noindent
The assignment of {\it the-rel\/} as the {\sc content} value captures 
the definiteness of the resulting NP. Notice that this analysis contains a DetP, 
but in spite of that, it is not an instance of the DP approach, since the 
determiner does not head the nominal, but only its specifier. 


\subsection{The functor treatment} 


The functor treatment also adopts the NP approach, but --in contrast to the 
specifier treatment-- it does not draw a distinction between specifiers and adjuncts, 
nor between lexical (or substantive) and functional categories.\footnote{The term {\it functor} 
is also used in Categorial (Unification) Grammar, where it has 
a very broad meaning, subsuming the nonhead daughter in combinations of a 
head with a specifier or an adjunct, and the head daughter otherwise, 
see \citet{Bouma88}. This broad notion is also adopted in 
\citet{Reape94}. We adopt a more restrictive version in which functors 
are nonhead daughters which lexically select their head sister.}  
The presentation in this section is mainly based on \citet{VanEynde06} 
and \citet{Allegranza06}. We first introduce the basics (2.2.1) and then 
turn to nominals with a phrasal specifier (2.2.2) and to nominals without specifier (2.2.3).    


\subsubsection{Basics} 


The distinction between specifiers and adjuncts is usually motivated by 
the assumption that the former are obligatory and non-stackable, while the latter  
are optional and stackable. In practice, though, this distinction 
is blurred by the fact that many nominals are well-formed without specifier.
To accommodate this one can employ phonetically empty determiners, optionality in 
the {\sc spr} list or a non-branching phrase type, as in 2.1.4,   
but the functor treatment provides a more radical response and abandons 
the distinction between specifiers and adjuncts. 
Technically, this implies that the {\sc spr} feature is dropped.\footnote{Intriguingly,
Chomsky has recently argued that there is no need for the notion of specifier in 
Transformational Grammar either, see \citet[43]{Chomsky13}.}  

Besides, the functor treatment abandons the distinction between lexical and 
functional categories. The words which are commonly treated as determiners
are not treated as members of a separate functional category `Det', but are 
claimed to belong to independently needed lexical categories, such as Adjective, 
Adverb, Pronoun and Noun. The argumentation is mainly --but not only-- based on 
matters of NP-internal agreement and inflectional variation \citep{VanEynde06}. 
In Dutch, for instance, prenominal adjectives show agreement with 
the nouns they modify: They take the affix {\it -e\/} in combination with plural 
and singular non-neuter nominals, but not in combination with singular 
neuter nominals.\footnote{If the adjective is preceded by a definite determiner, 
it also takes the affix in singular neuter nominals. This phenomenon is treated 
in Section 2.2.2.} 

\begin{exe} 
\ex\label{wit} 
\gll  zwarte muren,        zwarte verf,              zwart zand \\
      black wall.{\sc pl}, black paint.{\sc sg.fem}, black sand.{\sc sg.neu} \\
\end{exe} 

\noindent
The same holds for several of the words that are claimed to be determiners, 
such as the possessive {\it ons\/} `our' and the interrogative {\it welk\/} `which'. 

\begin{exe} 
\ex\label{ons}
\gll onze ouders,         onze muur,             ons huis     \\
     our parent.{\sc pl}, our wall.{\sc sg.mas}, our house.{\sc sg.neu} \\
\ex\label{welk} 
\gll welke boeken,        welke man,              welk boek   \\
     which book.{\sc pl}, which man.{\sc sg.mas}, which book.{\sc sg.neu} \\
\end{exe} 

\noindent
By contrast, prenominal (pro)nouns do not show NP-internal agreement and  
never take the {\it -e\/} affix. This not only holds for genitive nouns, 
such as {\it Jans}, but also for several of the words that are claimed 
to be determiners, such as the interrogative {\it wiens\/} `whose' 
and the quantifying {\it wat\/} `some'. 

\begin{exe} 
\ex\label{jans}
\gll  Jans ouders,                   Jans fiets,                      Jans huis \\ 
      Jan.{\sc gen} parent.{\sc pl}, Jan.{\sc gen} bike.{\sc sg.mas}, Jan.{\sc gen} house.{\sc sg.neu} \\
\ex\label{wiens}
\gll  wiens ouders,          wiens muur,              wiens huis \\ 
      whose parent.{\sc pl}, whose wall.{\sc sg.mas}, whose house.{\sc sg.neu} \\
\ex\label{wat}
\gll  wat boeken,         wat verf,                wat zand  \\
      some book.{\sc pl}, some paint.{\sc sg.fem}, some sand.{\sc sg.neu} \\
\end{exe} 

\noindent
To model this the functor treatment assigns adjectival status 
to determiners like {\it ons\/} `our' and {\it welk\/} `which', 
and pronominal status to determiners like {\it wiens\/} `whose' and
{\it wat\/} `some'.  
This distinction is also relevant for other languages. The Italian 
possessives of the first and second person, for instance, 
show the same alternation for number and gender as the adjectives
and are subject to the same constraints on NP-internal agreement.

\begin{exe}  
\ex 
\gll  il nostro futuro, la nostra scuola, i nostri genitori, le nostre scatole \\
      the our future.{\sc sg.mas}, the our school.{\sc sg.fem}, the our parent.{\sc pl.mas}, the our box.{\sc pl.fem} \\ 
\end{exe} 

\noindent
The possessive of the third person plural, by contrast, does not show any 
inflectional variation and does not show agreement with the 
noun.\footnote{Confirming evidence for the pronominal status of {\it loro\/} is 
provided by the fact that it is also used as a personal pronoun, as in 
{\it l'ho dato a loro\/} `I gave it to them'.}

\begin{exe} 
\ex 
\gll  il loro futuro, la loro scuola, i loro genitori, le loro scatole \\   
      the their future.{\sc sg.mas}, the their school.{\sc sg.fem}, the their parent.{\sc pl.mas}, the their box.{\sc pl.fem} \\ 
\end{exe}

\noindent
An example from French is provided in \citet{Abeilleetal04}, 
which assigns adverbial status to the specifier in 
{\it beaucoup de farine\/} `much flour'.  
Technically, the elimination of the distinction between lexical and 
functional categories implies that there is no longer any need for 
drawing a distinction between the selection features {\sc mod(ified)} and {\sc spec(ified)}. 

\begin{figure}
\begin{center} 
\footnotesize
\tree   {\ntnode{Zi}{\it headed-phr},
          {\ntnode{Zv}{\it head-arg-phr}, 
            {\tnode{Zq}{head-subj-phr}},
            {\tnode{Zl}{head-comps-phr}},
            {\tnode{Zg}{...}}},
          {\ntnode{Zw}{\it head-nonarg-phr}, 
            {\tnode{Zs}{head-functor-phr}},
            {\tnode{Zu}{head-indep-phr}}}}
\nodeconnect{Zi}{Zv}
\nodeconnect{Zi}{Zw}
\nodeconnect{Zv}{Zq}
\nodeconnect{Zv}{Zl}
\nodeconnect{Zv}{Zg}
\nodeconnect{Zw}{Zs}
\nodeconnect{Zw}{Zu}
\caption{\label{typ} Hierarchy of Headed Phrases }
\normalsize
\end{center}
\end{figure}

To spell out the functor treatment we start from the 
hierarchy of headed phrases in Figure \ref{typ}. The basic distinction is
that between {\it head-argument-phr\/} and {\it head-nonargument-phr\/}. 
In the former the head daughter selects its non-head sister(s) by means of 
valence features, such as {\sc comps} and {\sc subj} (but not {\sc spr}!), 
and it is their values that register the degree of saturation of the phrase, 
as shown for {\sc comps} in Section 2.1.1.  
In head-nonargument phrases the degree of saturation is registered  
by the {\sc mark(ing)} feature. It is declared for objects of type {\it category}, 
along with the {\sc head} and valence features.\footnote{The {\sc marking} feature  
is introduced in \citet[46]{PS94} to model the combination of a complementizer 
and a clause.} Its value is shared with the head daughter in head-argument phrases
and with the non-head daughter in head-nonargument phrases, as spelled out in 
(\ref{mark1}) and (\ref{mark2}) respectively. 

\begin{exe}
\ex\label{mark1} 
\begin{avm}
[{\it head-arg-phr\/}                      \\
 synsem|loc|cat|mark ~ @1 {\it marking\/}  \\
 head-dtr|synsem|loc|cat|mark ~ @1] 
\end{avm}
\ex\label{mark2} 
\begin{avm}
[{\it head-nonarg-phr\/}                    \\
 synsem|loc|cat|mark ~ @1 {\it marking\/}   \\
 dtrs ~ <[synsem|loc|cat|mark ~ @1] ~, @2 > \\
 head-dtr ~ @2 {\it sign\/}]
\end{avm}
\end{exe}

\noindent
Besides, instead of {\sc mod} and {\sc spec}, the functor treatment employs 
one feature to model the selection of a head by its non-head sister. It is   
called {\sc sel(ect)}, and its value is an object of type {\it synsem\/} that is 
shared with the head daughter, as spelled out in (\ref{hefu}).  

\begin{exe}
\ex\label{hefu} 
\begin{avm}
[{\it head-functor-phr\/}                     \\
 dtrs ~ <[synsem|loc|cat|head|sel ~ @1]~, @2 {\it sign\/}> \\
 head-dtr ~ @2 [synsem ~ @1 {\it synsem\/}]]
\end{avm}
\end{exe} 

\noindent
This is a subtype of {\it head-nonargument-phr}. It subsumes 
the phrases in which the non-head daughter selects its head sister.
As such it contrasts with phrases of type  
{\it head-independent-phr\/}, whose defining characteristic is 
that the nonhead daughter does not select its head sister: Its {\sc sel} value 
is hence {\it none}.\footnote{This type is introduced in 
\citet[130]{VanEynde98a}. It will be used in Section 3 to deal with 
idiosyncratic nominals.}   

\begin{exe}
\ex\label{hein} 
\begin{avm}
[{\it head-independent-phr\/}                           \\
 dtrs ~ <[synsem|loc|cat|head|sel ~ {\it none\/}]~, @1> \\
 head-dtr ~ @1 {\it sign\/}]
\end{avm}
\end{exe}    

An illustration of the functor treatment is given in Figure \ref{markyy}. 
The combination of the noun with the adjective is an instance of {\it head-functor-phr}, 
in which the adjectival functor selects an unmarked nominal ($\avmbox{3}$),  
shares its {\sc marking} value ($\avmbox{5}$), and, being a non-argument, 
shares it with the mother as well. 
The combination of the resulting nominal with the demonstrative is also 
an instance of {\it head-functor-phr},
in which the demonstrative functor selects an unmarked nominal ($\avmbox{4}$), 
but --differently from the adjective-- its {\sc marking} value is of type 
{\it marked}, and this value is shared with the mother ($\avmbox{2}$).    
This accounts for the fact that adjectives can be stacked, while  
determiners cannot. It also accounts for the ill-formedness of combinations like 
*{\it long the bridge}, since attributive adjectives are not compatible with a
marked nominal.  
Notice that the demonstrative does not belong to a functional part of speech, such as Det.
Instead, it is claimed to be a pronoun. The only difference between adnominal {\it that\/} and nominal
{\it that\/} concerns the {\sc sel(ect)} value: While the former selects an unmarked singular nominal, 
the latter does not select anything.      

\begin{figure}
\begin{center}
\footnotesize
\tree
{\ntnode{Zz}{[{\sc head} ~ \avmbox{1} [{\it noun} {\sc sel} ~ {\it none\/}] , {\sc mark} ~ \avmbox{2} {\it marked\/}]},
  {\ntnode{Za}{[{\sc head} ~ [{\it pron} {\sc sel} ~ \avmbox{4}] , {\sc mark} ~ \avmbox{2}]},
    {\tnode{Zp}{that}}},
  {\ntnode{Zb}{\avmbox{4} [{\sc head} ~ \avmbox{1} , {\sc mark} ~ \avmbox{5} {\it unmarked\/}]},
    {\ntnode{Zc}{[{\sc head} ~ [{\it adj} {\sc sel} ~ \avmbox{3}] , {\sc mark} ~ \avmbox{5}]},
      {\tnode{Zq}{long}}},  
    {\ntnode{Zd}{\avmbox{3} [{\sc head} ~ \avmbox{1} , {\sc mark} ~ \avmbox{5}]},
      {\tnode{Zh}{bridge}}}}}
\nodeconnect{Zz}{Za}
\nodeconnect{Zz}{Zb}
\nodeconnect{Za}{Zp}
\nodeconnect{Zb}{Zc}
\nodeconnect{Zb}{Zd}
\nodeconnect{Zc}{Zq}
\nodeconnect{Zd}{Zh}
\caption{\label{markyy} Adnominal functors }
\normalsize
\end{center}
\end{figure}

Whether an adnominal functor is marked or unmarked is subject to cross-linguistic variation. 
The Italian possessives, for instance, are unmarked and can, hence, be preceded 
by an article, as in {\it il suo cane\/} `the his dog' and {\it un mio amico\/} `a my friend', 
while the French possessives are marked: {\it (*le) mon chien\/} `(*the) my dog' 
and {\it (*un) mon ami\/} `(*a) my friend'. 

The treatment of postnominal dependents is similar to that in the specifier treatment. 
A relational noun like {\it sister}, for instance, takes an optional PP[{\it of\/}] complement.  
Similarly, a deverbal noun like {\it description\/} takes two optional PP complements. 
If realized, as in {\it a description of the Hungarian NP by Ivan}, 
their indices are identified with those of the second and the first argument respectively. 
There is a difference, though, in the treatment of nominals 
whose first argument is realized as a possessive, as in 
{\it his description of the Hungarian NP}. In this case, the first argument is not put on the 
{\sc spr} list of the noun. Instead, the possessive selects an unmarked nominal with an optional 
PP[{\it by\/}] complement on its {\sc comps} list, and shares the index of that PP, 
as spelled out in Figure \ref{possy}. 

\begin{figure}
\begin{center}
\footnotesize
\tree
{\ntnode{Zz}{[{\sc head} ~ \avmbox{1} [{\it noun} {\sc sel} ~ {\it none\/}] , {\sc mark} ~ \avmbox{2} {\it marked\/}]},
  {\ntnode{Za}{[{\sc head} ~ [{\it pron} {\sc sel} ~ \avmbox{4}] , {\sc mark} ~ \avmbox{2}]$_{i}$},
    {\tnode{Zp}{his}}},
  {\ntnode{Zb}{\avmbox{4} [{\sc head} ~ \avmbox{1} , {\sc comps} ~ $<$(PP[{\it by\/}]$_{i}$)$>$ , {\sc mark} ~ \avmbox{5} {\it unmarked\/}]},
    {\ntnode{Zc}{[{\sc head} ~ \avmbox{1} , {\sc comps} ~ $<$ (PP[{\it by\/}]$_{i}$) , \avmbox{3}$>$ , {\sc mark} ~ \avmbox{5}]},
      {\tnode{Zq}{description}}},  
    {\ntnode{Zd}{\avmbox{3} [{\sc head} ~ {\it prep} , {\sc comps} ~ $<$ ~ $>$]},
      {\tnode{Zh}{of the Hungarian NP}}}}}
\nodeconnect{Zz}{Za}
\nodeconnect{Zz}{Zb}
\nodeconnect{Za}{Zp}
\nodeconnect{Zb}{Zc}
\nodeconnect{Zb}{Zd}
\nodeconnect{Zc}{Zq}
\nodetriangle{Zd}{Zh}
\caption{\label{possy} Deverbal Nominals }
\normalsize
\end{center}
\end{figure}


        
\subsubsection{Nominals with a phrasal functor} 


To illustrate how the functor treatment deals with phrasal specifiers we 
take the nominal {\it a hundred pages}. It has a left branching structure in
which the indefinite article selects the unmarked singular noun {\it hundred} --its plural 
counterpart is {\it hundreds}--  
and in which the resulting NP selects the unmarked plural noun 
{\it pages}, as spelled out in Figure \ref{glorie}. The  
{\sc marking} value of the article is shared with its mother ($\avmbox{1}$), 
yielding a marked NP, and since the latter is a functor it is also shared with 
the NP as a whole ($\avmbox{1}$). 
This accounts for the fact that it cannot be preceded by a determiner, 
as in *{\it those a hundred pages\/}. By contrast, if the place of the article is taken 
by a numeral, as in {\it two hundred pages}, the addition of a determiner is 
possible, as in {\it those two hundred pages}, since the numeral is unmarked. 
Notice that the {\sc head} value of the entire NP is identified with that 
of {\it pages\/}. This accounts for the fact that it is plural.  
 
\begin{figure}  
\begin{center}
\scriptsize
\tree
  {\ntnode{Za}{[{\sc head} ~ \avmbox{5} [{\it noun} {\sc sel} ~ {\it none\/}] , {\sc mark} ~ \avmbox{1}]},
    {\ntnode{Zd}{[{\sc head} ~ \avmbox{4} [{\it noun} {\sc sel} ~ \avmbox{2}] , {\sc mark} ~ \avmbox{1}]},
      {\ntnode{Zf}{[{\sc head} ~ [{\it pron} {\sc sel} ~ \avmbox{3}] , {\sc mark} ~ \avmbox{1} {\it marked\/}]},
        {\tnode{Zl}{a}}},
      {\ntnode{Zg}{\avmbox{3} [{\sc head} ~ \avmbox{4} , {\sc mark} ~ {\it unmarked\/}]},
        {\tnode{Zi}{hundred}}}},
    {\ntnode{Zk}{\avmbox{2} [{\sc head} ~ \avmbox{5} , {\sc mark} ~ {\it unmarked\/}]},
      {\tnode{Zh}{pages}}}}
\nodeconnect{Za}{Zd}
\nodeconnect{Za}{Zk}
\nodeconnect{Zd}{Zf}
\nodeconnect{Zd}{Zg}
\nodeconnect{Zf}{Zl}
\nodeconnect{Zg}{Zi}
\nodeconnect{Zk}{Zh}
\caption{\label{glorie} Phrasal functors }
\normalsize
\end{center}
\end{figure}


\subsubsection{The hierarchy of MARKING values} 


Nominals without specifier have a {\sc marking} value of type {\it unmarked}. 
Whether they need to be marked, depends on a multitude of factors and is 
largely language specific. 
Latin, for instance, does not require its nominals to contain a specifier, also 
if they are singular and count. Yet, there are cases in which an unmarked nominal 
is incomplete.    
To illustrate this let us take the attributive adjectives of Dutch again. 
As already pointed out in Section 2.2.1, they take the form without affix in  
singular neuter nominals, as in {\it zwart zand\/} `black sand'. A complication, 
though, is that they canonically take the form with the affix if the nominal is  
introduced by a definite determiner, as in {\it het zwarte zand\/} `the black sand'. 
This has consequences for the status of nominals with a singular neuter head: 
{\it zwart zand\/} and *{\it zwarte zand}, for instance, are both unmarked, 
but while the former is well-formed as it is, the latter is only 
well-formed if it is preceded by a definite determiner. 
To model this \citet{VanEynde06} differentiates between two types 
of {\it unmarked\/} nominals, as shown in Figure \ref{bare}. 

\begin{figure}
\begin{center}
\footnotesize
\tree
{\ntnode{Zz}{\it marking},
  {\ntnode{Ze}{\it unmarked},  
    {\tnode{Zf}{incomplete}},
    {\tnode{Zg}{bare}}},
  {\tnode{Zh}{\it marked}}}
\nodeconnect{Zz}{Ze}
\nodeconnect{Ze}{Zf}
\nodeconnect{Ze}{Zg}
\nodeconnect{Zz}{Zh}
\caption{\label{bare} Hierarchy of Marking values} 
\normalsize
\end{center}   
\end{figure}

Employing the more specific subtypes, the adjectives without affix which select a singular 
neuter nominal have the value {\it bare}, while their declined counterparts which select 
a singular neuter nominal have the value {\it incomplete}. 
Since this {\sc marking} value is shared with the mother, the {\sc marking} value 
of {\it zwart zand\/} is {\it bare\/}, while that of *{\it zwarte zand\/} is {\it incomplete}. 
The fact that the latter must be preceded by a definite determiner
is modeled in the {\sc sel(ect)} value of the determiner: 
While definite determiners select an unmarked nominal, which implies that 
they are compatible with both bare and incomplete nominals,
non-definite determiners select a bare nominal and are, hence, not compatible 
with an incomplete one, as in *{\it een zwarte huis\/} `a black house'. 
The {\sc marking} feature is, hence, useful to differentiate bare 
nominals from incomplete nominals.  
In a similar way, one can make finer-grained distinctions in the hierarchy of  
{\it marked} values to capture co-occurrence restrictions between determiners and 
nominals, as in the functor treatment of the Italian determiner system of 
\citet{Allegranza06}. See also the treatment of nominals with idiosyncratic properties 
in Section 3. 


\subsection{The DP treatment} 


In contrast to the specifier treatment and the functor treatment, the DP 
treatment identifies the determiner as the head of an NP. The presentation in this section   
is mainly based on \citet{Netter94}. We first sketch its main characteristics (2.3.1)
and then list some problems for it (2.3.2). 


\subsubsection{Two types of complementation} 


\begin{figure}
\begin{center}
\footnotesize
\tree
{\ntnode{Za}{[{\sc head} ~ \avmbox{1} {\it det} , {\sc comps} ~ $<$ ~ $>$]},
  {\ntnode{Zd}{[{\sc head} ~ \avmbox{1} , {\sc comps} ~ $<$\avmbox{4} NP$>$]},
    {\tnode{Ze}{that}}},
  {\ntnode{Zf}{\avmbox{4} [{\sc head} ~ \avmbox{2} {\it noun} , {\sc comps} ~ $<$ ~ $>$]},
    {\ntnode{Zi}{[{\sc head} ~ \avmbox{2} , {\sc comps} ~ $<$\avmbox{3} PP$>$]},
      {\tnode{Zj}{sister}}},
    {\ntnode{Zk}{\avmbox{3}}, 
      {\tnode{Zh}{of Leslie}}}}}
\nodeconnect{Za}{Zd}
\nodeconnect{Zd}{Ze}
\nodeconnect{Za}{Zf}
\nodeconnect{Zf}{Zi}
\nodeconnect{Zi}{Zj}
\nodeconnect{Zf}{Zk}
\nodetriangle{Zk}{Zh}
\caption{\label{net} Lexicalist DP Treatment }
\normalsize
\end{center}
\end{figure} 

As in the transformational DP treatment, nominal projections are divided in a nominal core, 
consisting of a noun and its complements and/or adjuncts, on the one hand, and an external 
functional layer, comprising the determiner, on the other hand. The former is analyzed in much the 
same way as in the specifier treatment. The relational noun   
{\it sister\/} in Figure \ref{net}, for instance, is combined with 
its PP complement in the usual way. The addition of the determiner, by contrast, is 
modeled differently. It is not the nominal which selects its specifier by means of the valence 
feature {\sc spr}; instead, it is the determiner which selects a nominal by means of 
the valence feature {\sc comps}, yielding a DP with an empty {\sc comps} list. 
To account for the fact that this DP inherits many of its properties from its nominal 
non-head daughter, Netter makes a distinction between functional 
complementation and ordinary complementation, and differentiates between major and minor 
{\sc head} features: 

\begin{exe} 
\ex  {\it part-of-speech\/} : \begin{avm}
                              [major [n ~ {\it boolean\/} \\
                                      v ~ {\it boolean\/} ] \\
                               minor [fcompl ~ {\it boolean\/}]]
                              \end{avm} 
\end{exe} 

\noindent
The {\sc major} attribute includes the boolean features N and V, where 
nouns are [+N, --V], adjectives [+N, +V], verbs [--N, +V] and prepositions [--N, --V]. 
Besides, [+N] categories also have the features {\sc case}, {\sc number} and {\sc gender}. 
Typical of functional complementation is that the functional head shares the 
{\sc major} value of its complement \citep[311--312]{Netter94}. 

\begin{exe} 
\ex\label{maj} In a lexical category of type {\it func-cat\/} the value of its {\sc major} 
      attribute is token identical with the {\sc major} value of its complement. 
\end{exe} 

\noindent
Since a determiner shares the {\sc major} value of its nominal complement and since this value is 
also shared with the DP (given the Head Feature Principle), it follows that the resulting 
DP is [+N,--V] and that its {\sc case} , {\sc number} and {\sc gender} values are 
identical to those of its nominal non-head daughter. 
This differentiates functional complementation from ordinary complementation, where 
a head and its complement do not share their {\sc major} value. The noun {\it sister\/} 
in Figure \ref{net}, for instance, does not share the part-of-speech of its PP complement. 

The {\sc minor} attribute is used to model properties which a functional head does {\em not\/} share with its 
complement. It includes {\sc fcompl}, a feature which registers whether a projection is 
functionally complete or not. Its value is positive for determiners, negative for  
singular count nouns and underspecified for plurals and mass nouns.  
Determiners take a nominal complement with a negative {\sc fcompl} value, but their 
own {\sc fcompl} value is positive and since they are the head, they share this value with 
the mother, as in Figure \ref{netter}. 
In this analysis, a nominal is complete, if it is both saturated 
(empty {\sc comps} list) and functionally complete (positive {\sc fcompl}). 

\begin{figure}
\begin{center}
\footnotesize
\tree
{\ntnode{Za}{[{\sc major} ~ \avmbox{1} [--V, +N[{\it sing\/}]] , {\sc minor$|$fcompl} ~ +]},
  {\ntnode{Zd}{[{\sc major} ~ \avmbox{1} , {\sc minor$|$fcompl} ~ +]},
    {\tnode{Ze}{that}}},
  {\ntnode{Zf}{[{\sc major} ~ \avmbox{1} , {\sc minor$|$fcompl} ~ --]},
    {\tnode{Zj}{sister of Leslie}}}}
\nodeconnect{Za}{Zd}
\nodeconnect{Zd}{Ze}
\nodeconnect{Za}{Zf}
\nodetriangle{Zf}{Zj}
\caption{\label{netter} Functional Completeness }
\normalsize
\end{center}
\end{figure} 


\subsubsection{Two problems for the DP treatment}  


A problem for the DP treatment concerns the notion of 
functional complementation, as defined in (\ref{maj}). 
If determiners share the {\sc major} value of the nominals which they select, 
then it follows that they are nominal themselves, i.e. [+N, --V].
However, while this makes sense for determiners with (pro)nominal properties,
such as the English {\it that}, the Dutch {\it wat\/} `some' and the Italian 
{\it loro\/} `their', it is rather implausible for determiners with adjectival properties,
such as the Italian {\it mio\/} `my', the Dutch {\it ons\/} `our' and the German
{\it welch\/} `which'.   
The problem also affects the associated agreement features, i.e. {\sc case}, 
{\sc number} and {\sc gender}. If a determiner 
is required to share the values of these features with its nominal complement,
as spelled out in (\ref{maj}), then we get implausible results for nominals in 
which the determiner and the noun do not show agreement.    
In the Dutch {\it wiens huizen\/} `whose houses' and 
{\it 's lands hoogste bergen\/} `the highest mountains of the country', for instance,  
the selected nominals are plural and non-genitive, and so are the entire DPs, but 
the selecting determiner (phrase) is singular and genitive. 

Another problem concerns the assumption ``that all substantive categories will 
require the complement they combine with to be both saturated and 
functionally complete'' \citep[311]{Netter94}. Complements of verbs and 
prepositions must, hence, be positively specified for {\sc fcompl}. 
This, however, is contradicted by the existence of 
prepositions which require their complement to be functionally incomplete. 
The German prepositions {\it am, im, vom, beim\/} and {\it zum}, for instance, 
which diachronically result from the incorporation of a dative form of the 
definite article ({\it dem\/}) in respectively {\it an\/} `at', 
{\it in\/} `in', {\it von\/} `of', {\it bei\/} `at' and {\it zu\/} `to',
count as syntactic atoms in {\sc hpsg}, just like 
the forms without incorporated article, but in contrast to the 
latter they require their complement to lack a determiner: While  
{\it in dem/diesem Zimmer\/} `in the/this room' and {\it im Zimmer\/} `in.the room'
are both well-formed, *{\it im dem/diesem Zimmer} `in.the the/this room' is 
not.\footnote{A similar phenomenon exists in Italian, where the prepositions 
{\it a\/} `at', {\it da\/} `from', {\it su\/} `on', {\it di\/} `of' and {\it in\/} `in' 
have alternate forms with an incorporated form of the definite article, as in  
{\it al, dal, sul, del\/} and {\it nel}, each with 
feminine and plural counterparts ({\it alla, agli, alle, dalla, dagli, dalle, ...}). 
The nominals which combine with such forms may not be introduced by a determiner:
{\it di/*della questa scatola\/} `of/*of.the this box' and {\it in/*nel ogni palazzo\/} 
`in/*in.the every palace'. The same holds m.m. for the French 
portemanteau forms {\it du\/}, {\it des\/}, {\it au\/} and {\it aux}.}   
Moreover, there are prepositions which do not contain an 
incorporated article, but nonetheless require a determinerless nominal. 
The Dutch {\it te\/} and {\it per}, for instance, are not compatible with nominals 
that contain a determiner, even if the nominal is singular and count, 
as in {\it te (*het/een) paard\/} `on horse' and {\it per (*de/een) trein\/} `by train'.   
This must be due to a lexical property of these prepositions, 
since most other prepositions require such nominals to have 
a determiner, as in {\it ze viel van *(het/een) paard\/} `she fell from *(the/a) horse' 
and {\it ze zit op *(de/een) trein naar Londen\/} `she is on *(the/a) train to London'.  
The requirement that complements of prepositions must be functionally complete is,
hence, untenable.  


\subsection{Conclusion} 


This section has presented the three main treatments of nominal structure in HPSG. 
Two adopt the NP approach and one the DP approach.
Overall, the former turn out to be more amenable to integration  
in a monostratal surface-oriented framework like HPSG than the latter. 
See also \citet{Mueller19}. Of the two NP treatments
the specifier treatment is closer to early versions of {\it X}-bar syntax and to GPSG.   
The functor treatment is closer to versions of Categorial (Unification) Grammar, and 
has also been adopted in Sign-Based Construction Grammar \citep{Sag2012}.
 

\section{Idiosyncratic nominals}


This section focusses on the analysis of nominals with idiosyncratic properties. 
Since their analysis often requires 
a relaxation of the strictly lexicalist approach of early HPSG we first introduce some basic 
notions of constructional HPSG. 

The lexicalist approach of early HPSG can be characterized as one in which the 
properties of phrases are mainly determined by properties of the constituent words 
and only to a small extent by properties of the combinatory operations as such. 
\citet[391]{PS94}, for instance, employs no more than seven types 
of combinatory operations, including those which were exemplified in Section 2.1, 
i.e. head-specifier, head-complements and head-adjunct.\footnote{The remaining four 
are head-subject, head-subject-complements, head-marker and head-filler.}   
Over time, though, this radical lexicalism gave way to an 
approach in which the properties of the combinatory operations  
play a larger role. The small inventory of highly 
abstract phrase types got replaced by a more fine-grained hierarchy  
in which the types contain more specific and --if need be-- idiosyncratic 
constraints. This development started in \citet{Sag97}, was 
elaborated in \citet{GS00}, and gained momentum afterward,
leading to what is now known as constructional HPSG. 
Characteristic of this approach is the use of a bi-dimensional hierarchy 
of phrasal signs. In such a hierarchy the phrases are not only partitioned 
in terms of {\sc headedness}, but also in terms of a second dimension, called  
{\sc clausality}, as in Figure \ref{bidim}. 

\begin{figure}
\footnotesize
\begin{center}
\tree
    {\ntnode{Za}{\it phrase},
      {\ntnode{Zh}{{\sc headedness}}, 
        {\ntnode{Zi}{\it headed-phr},
          {\tnode{Zv}{head-subj-phr}}, 
          {\tnode{Zw}{...}}}, 
        {\tnode{Zq}{non-headed-phr}}}, 
      {\ntnode{Zl}{{\sc clausality}},
        {\ntnode{Zg}{\it clause},
          {\tnode{Zs}{declarative-cl}, 
            {\tnode{Zp}{decl-head-subj-cl}}},
          {\tnode{Zu}{..}}}, 
        {\tnode{Zm}{non-clause}}}} 
\nodeconnect{Za}{Zh}
\nodeconnect{Zh}{Zi}
\nodeconnect{Zi}{Zv}
\nodeconnect{Zi}{Zw}
\nodeconnect{Zh}{Zq}
\nodeconnect{Za}{Zl}
\nodeconnect{Zl}{Zg}
\nodeconnect{Zg}{Zs}
\nodeconnect{Zg}{Zu}
\nodeconnect{Zl}{Zm}
\nodeconnect{Zv}{Zp}
\nodeconnect{Zs}{Zp}
\caption{Bi-dimensional Hierarchy of Clauses \label{bidim}}  
\normalsize
\end{center}
\end{figure}

The types in the {\sc clausality} dimension are associated with constraints,
in much the same way as the types in the {\sc headedness} dimension.  
Clauses, for instance, are required to denote an object of type {\it message\/} 
\citep[41]{GS00}.

\begin{exe}
\ex 
\begin{avm} 
[{\it clause\/} \\
 synsem|loc|content ~ {\it message\/}] 
\end{avm}
\end{exe}

\noindent
At a finer-grained level, the clauses are partitioned into 
declarative, interrogative, imperative, exclamative and relative
clauses, each with their own constraints. 
Interrogative clauses, for instance, denote a question, 
while indicative declarative clauses denote a proposition.

Exploiting the possibilities of multiple inheritance one can 
define types which inherit properties from more than one supertype. 
The type {\it decl(arative)-head-subj(ect)-cl(ause)}, for instance, inherits 
the properties of {\it head-subject-phr}, on the one hand, and 
{\it declarative-cl}, on the other hand. Besides, it may 
have properties of its own, such as the fact that its head daughter 
is a finite non-inverted verb \citep[43]{GS00}. 
This combination of multiple inheritance and specific   
constraints on maximal phrase types is also useful for the analysis of 
nominals with idiosyncratic properties, as will be shown presently. 


\subsection{The Big Mess Construction} 


In ordinary nominals determiners precede attributive adjectives. Changing the order 
yields ill-formed combinations, such as *{\it long that bridge\/} and *{\it very tall every man}. 
However, this otherwise illegitimate order is precisely what we find in 
the Big Mess Construction (BMC), a term coined by \citet{Berman74}.  

\begin{exe}
\ex\label{bime}
\begin{xlist}
\ex   It's {\it so good a bargain\/} I can't resist buying it.
\ex   {\it How serious a problem\/} is this?
\end{xlist}
\end{exe} 

\noindent
The idiosyncratic order in (\ref{bime}) is required if the nominal is introduced 
by the indefinite article, and if the preceding AP is introduced by one of a small 
set of degree markers, including {\it so, as, how, this, that\/} and {\it too}. 

A specifier treatment of the BMC is provided in \citet[201]{GS00}. It adopts  
a left branching structure, as in  [[[{\it so good\/}] {\it a\/}] {\it bargain\/}], 
in which {\it so good\/} is the specifier of the indefinite article and in which 
{\it so good a\/} is the specifier of {\it bargain}. This is comparable to the 
treatment of the possessive in [[[{\it the Queen of England\/}] {\it 's\/}] {\it sister\/}],
see Section 2.1.3.  
However, while there is evidence that {\it the Queen of England's\/} is a constituent,
since it may occur independently as in {\it this crown is the Queen of England's}, there is 
no evidence that {\it so good a\/} is a constituent. Instead, there is evidence that 
the article forms a constituent with the following noun, since it also precedes the noun 
when the AP is in postnominal position, as in {\it we never had a bargain so good as this one\/}.

\begin{figure}
\begin{center}
\tree
{\ntnode{Zd}{[{\sc head} ~ $\avmbox{3}$ {\it noun} , {\sc mark} ~ $\avmbox{1}$ {\it a\/}]},    
  {\ntnode{Ze}{[{\sc mark} ~ $\avmbox{1}$ , {\sc sel} ~ $\avmbox{2}$]},    
    {\tnode{Zl}{a}}},
  {\ntnode{Zg}{$\avmbox{2}$ [{\sc head} ~ $\avmbox{3}$ , {\sc mark} ~ {\it unmarked\/}]},
    {\tnode{Zi}{bargain}}}}
\nodeconnect{Zd}{Ze}
\nodeconnect{Ze}{Zl}
\nodeconnect{Zd}{Zg}
\nodeconnect{Zg}{Zi}
\caption{\label{aprob} The lower NP }
\end{center}
\end{figure}

\begin{figure}
\begin{center}
\tree
{\ntnode{Zd}{[{\sc head} ~ $\avmbox{3}$ {\it adj} , {\sc mark} ~ $\avmbox{1}$ {\it marked\/}]},
  {\ntnode{Ze}{[{\sc mark} ~ $\avmbox{1}$ , {\sc sel} ~ $\avmbox{2}$]},    
    {\tnode{Zl}{so}}},
  {\ntnode{Zg}{$\avmbox{2}$ [{\sc head} ~ $\avmbox{3}$ , {\sc mark} ~ {\it unmarked\/}]},
    {\tnode{Zi}{good}}}}
\nodeconnect{Zd}{Ze}
\nodeconnect{Ze}{Zl}
\nodeconnect{Zd}{Zg}
\nodeconnect{Zg}{Zi}
\caption{\label{sohow} The marked AP }
\end{center}
\end{figure}

A structure in which the article sides with the noun, as in [[{\it so good\/}] [{\it a bargain\/}]], 
is adopted in \citet{VanEynde07}, \citet{KimSells11}, \citet{KaySag12}, 
\citet{ArnoldSadler14} and \citet{VanEynde18}, all of which are functor treatments. 
The structure of the lower NP is spelled out in Figure \ref{aprob}. 
The article has a {\sc marking} value of type {\it a\/} which is a subtype of {\it marked\/} and which it
shares with the mother.\footnote{The {\sc marking} value of the article looks similar to its 
{\sc phonology} value, but it is not the same. The {\sc phonology} values of {\it a\/} and {\it an}, 
for instance, are different, but their {\sc marking} value is not.} 
The AP is also treated as an instance of the head-functor type 
in \citet{VanEynde07}, \citet{KimSells11} and \citet{VanEynde18}. 
The adverb has a {\sc marking} value of type {\it marked}, 
so that the AP is marked as well, as shown in Figure \ref{sohow}.   
In combination with the fact that the article selects an unmarked nominal, 
this accounts for the ill-formedness of (\ref{ster}). 

\begin{exe}
\ex\label{ster}
\begin{xlist}
\ex   [*] {It's a so good bargain I can't resist buying it.} 
\ex   [*] {A how serious problem is it?}   
\end{xlist}
\end{exe}

\noindent
By contrast, adverbs like {\it very\/} and {\it extremely\/} are unmarked,
so that the APs which they introduce are admissible in this position, as in (\ref{st}).  

\begin{exe}
\ex\label{st}
\begin{xlist}
\ex  This is a very serious problem. 
\ex  We struck an extremely good bargain. 
\end{xlist} 
\end{exe} 

To model the combination of the AP with the lower NP it may at first seem 
plausible to treat the AP as a functor which selects  
an NP that is introduced by the indefinite article. This, however, has 
unwanted consequences: Given that {\sc sel(ect)} is a {\sc head} feature, 
its value is shared between the AP and the adjective, so that the latter 
has the same {\sc sel(ect)} value as the AP, erroneously licensing such 
combinations as *{\it good a bargain}. To avoid this 
\citet{VanEynde18} exploits the possibilities of 
the bi-dimensional hierarchy of phrase types in Figure \ref{prot}. 

\begin{figure}
\begin{center} 
\footnotesize
\tree
    {\ntnode{Zj}{\it phrase},
      {\ntnode{Zl}{{\sc headedness}}, 
        {\ntnode{Zh}{\it headed-phr},
          {\ntnode{Zr}{\it hd-nonargument-phr},
            {\ntnode{Zx}{\it hd-functor-phr},
              {\tnode{Zp}{regular-nominal-phr}}}, 
            {\ntnode{Zt}{\it hd-independent-phr}}}}},
      {\ntnode{Zg}{{\sc clausality}},
        {\ntnode{Zk}{\it non-clause},
          {\ntnode{Zs}{\it nominal-parameter},
            {\tnode{Zn}{intersective-modification},
              {\tnode{Zo}{big-mess-phr}}}}}}}
\nodeconnect{Zj}{Zl}
\nodeconnect{Zl}{Zh}
\nodeconnect{Zh}{Zr}
\nodeconnect{Zr}{Zt}
\nodeconnect{Zr}{Zx}
\nodeconnect{Zj}{Zg}
\nodeconnect{Zg}{Zk}
\nodeconnect{Zk}{Zs}
\nodeconnect{Zs}{Zn}
\nodeconnect{Zn}{Zo}
\nodeconnect{Zt}{Zo}
\nodeconnect{Zx}{Zp}
\nodeconnect{Zn}{Zp}
\caption{ \label{prot} Bi-dimensional Hierarchy of Nominals} 
\normalsize
\end{center} 
\end{figure}

The types in the {\sc headedness} dimension are a subset of those in Figure \ref{typ}.  
The types in the {\sc clausality} dimension mainly capture semantic and 
category-specific properties, in analogy with the hierarchy of clausal phrases 
in \citet{GS00}. One of the non-clausal phrase types is {\it nominal-parameter}: 

\begin{exe}
\ex\label{param} 
\begin{avm}
[{\it nominal-parameter\/}                                             \\
 synsem|loc [cat|head ~ {\it noun\/}                                   \\
             content [{\it parameter\/}                                \\
                      index ~ @1 {\it index\/}                         \\
                      restr ~ $\avmbox{\Sigma_{1}}$ ~ $\bigcup$ ~ $\avmbox{\Sigma_{2}}$]] \\
 dtrs ~ <[synsem|loc|content|restr ~ $\avmbox{\Sigma_{1}}$ ] ~, @2>     \\
 head-dtr ~ @2 [synsem|loc|content [{\it parameter\/}                  \\
                                    index ~ @1                         \\
                                    restr ~ $\avmbox{\Sigma_{2}}$ ]]]
\end{avm}
\end{exe}

\noindent
The mother shares its index with the head daughter ($\avmbox{1}$) and 
its {\sc restr(iction)} value is the union of the {\sc restr} values 
of the daughters ($\avmbox{\Sigma_{1}}$ and $\avmbox{\Sigma_{2}}$). 
In the hierarchy of non-clausal phrases, this type contrasts amongst others with 
the quantified nominals, which have a {\sc content} value of type 
{\it quant-rel\/} \citep[203--205]{GS00}. A subtype of {\it nominal-parameter\/} is  
{\it intersective-modification}, as defined in (\ref{mononom}).  

\begin{exe}
\ex\label{mononom} 
\begin{avm}
[{\it intersective-modification\/}                \\
 synsem|loc|content|index ~ @1 {\it index\/}      \\
 dtrs ~ <[synsem|loc|content|index ~ @1 ] ~, @2 > \\
 head-dtr ~ @2 {\it sign\/} ]
\end{avm}
\end{exe}

\noindent 
This constraint requires the mother to share its index also with the 
non-head daughter. It captures the intuition that the 
noun and its non-head sister apply to the same entities, as in 
the case of {\it red box}.\footnote{Another subtype of {\it nominal-parameter\/} 
is {\it inverted-predication}, which subsumes amongst others 
the binominal noun phrase construction and certain types of apposition
(see Section 3.4).}  

The maximally specific types inherit properties of one of the types of headed phrases,
on the one hand, and of one of the non-clausal phrase types, on the other hand.  
Regular nominal phrases, for instance, such as {\it red box}, are subsumed 
by a type, called {\it regular-nominal-phr}, that inherits the 
constraints of {\it head-functor-phrase}, on the one hand, and 
{\it intersective-modification}, on the other hand.  
Another maximal type is {\it big-mess-phrase}. 
Its immediate supertype in the {\sc clausality} hierarchy is the same 
as for the regular nominal phrases, i.e. {\it intersective-modification\/}, 
but the one in the {\sc headedness} hierarchy is different: 
Being a subtype of {\it head-independent-phrase}, 
its non-head daughter does not select the head-daughter. 
Beside the inherited properties the BMC has some properties of its own.   
They are spelled out in (\ref{bigmess2}).

\begin{exe}
\ex\label{bigmess2} 
\begin{avm}
[{\it big-mess-phr\/}                                    \\     
 dtrs ~ <[{\it hd-functor-phr\/}                         \\
          synsem|loc|cat [head ~ {\it adjective\/}       \\
                          mark ~ {\it marked\/} ]] , @1> \\
 head-dtr ~ @1 [{\it regular-nominal-phr\/}              \\
                synsem|loc|cat|mark ~ {\it a\/}]]
\end{avm}
\end{exe}

\noindent
The head daughter is required to be a regular nominal phrase 
whose {\sc marking} value is {\it a}, and the non-head daughter 
is required to be an adjectival head-functor phrase
with a {\sc marking} value of type {\it marked}. 
This licenses APs which are introduced by a marked adverb, 
as in {\it so good a bargain\/} and {\it how serious a problem}, 
while it excludes unmarked APs, as in 
*{\it good a bargain\/} and *{\it very big a house}.
Iterative application is not licensed, since (\ref{bigmess2}) requires the 
head daughter to be of type {\it regular-nominal-phr}, which is incompatible with the type 
{\it big-mess-phrase}. This accounts for the fact that a big
mess phrase cannot contain another big mess phrase, as in
*{\it that splendid so good a bargain}.
A reviewer remarked that this analysis allows combinations like 
{\it so big a red expensive house}, suggesting that it should not. 
We are not sure, though, that this combination is ill-formed, 
and are anyway not inclined to exclude the presence of adjectives in 
the lower NP, since that would erroneously block the formation of the 
well-formed combinations in (\ref{shrub}).

\begin{exe} 
\ex\label{shrub} 
\begin{xlist} 
\ex  How big a new shrub from France were you thinking of buying? 
\ex  That's as beautiful a little black dress as I've ever seen.  
\end{xlist} 
\end{exe} 

\noindent
These are quoted from \citet[116]{Zwicky95} and \citet[42]{Troseth09} respectively. 


\subsection{Nominals with a verbal core} 


Ordinary nominals have a nominal core, but there are also nominals  
with a verbal core, such as gerunds and nominalized infinitives. They are 
of special interest, since they figure prominently in the argumentation 
that triggered the shift from the NP approach to the DP approach in Transformational 
Grammar. We first present a specifier treatment of the English gerund and then 
a functor treatment of the Dutch nominalized infinitive. 


\subsubsection{The English gerund : a mixed category} 


\begin{figure}
\begin{center}
\footnotesize
\tree
{\ntnode{Zz}{\it part-of-speech},
  {\ntnode{Ze}{\it noun},  
    {\tnode{Zf}{proper-noun}},
    {\tnode{Zg}{common-noun}}, 
    {\tnode{Zw}{gerund}}},
  {\ntnode{Zh}{\it relational},
    {\tnode{Zj}{verb}},      
    {\tnode{Zk}{adjective}}}}
\nodeconnect{Zz}{Ze}
\nodeconnect{Ze}{Zf}
\nodeconnect{Ze}{Zg}
\nodeconnect{Ze}{Zw}
\nodeconnect{Zz}{Zh}
\nodeconnect{Zh}{Zj}
\nodeconnect{Zh}{Zk}
\nodeconnect{Zh}{Zw}
\caption{ \label{ger} Gerunds as a Mixed Category }
\normalsize
\end{center} 
\end{figure}   

\begin{figure}
\begin{center}
\footnotesize
\tree
    {\ntnode{Zj}{\it phrase},
      {\ntnode{Zh}{{\sc headedness}}, 
        {\ntnode{Zi}{\it headed-phr},
          {\ntnode{Zo}{\it head-subj-phr},
            {\tnode{Zs}{nonfin-head-subj-cx}}}, 
          {\ntnode{Zp}{\it head-spr-phr},
            {\tnode{Zz}{noun-poss-cx}}}, 
          {\tnode{Zq}{...}}}},
     {\ntnode{Zl}{{\sc clausality}}, 
        {\ntnode{Zx}{\it clause}},
        {\ntnode{Zy}{\it non-clause}}}}
\nodeconnect{Zj}{Zh}
\nodeconnect{Zh}{Zi}
\nodeconnect{Zj}{Zl}
\nodeconnect{Zl}{Zx}
\nodeconnect{Zl}{Zy}
\nodeconnect{Zi}{Zo}
\nodeconnect{Zi}{Zp}
\nodeconnect{Zi}{Zq}
\nodeconnect{Zx}{Zs}
\nodeconnect{Zo}{Zs}
\nodeconnect{Zp}{Zz}
\nodeconnect{Zy}{Zz}
\caption{\label{bido} Bi-dimensional Hierarchy of Gerundial Phrases } 
\normalsize
\end{center}
\end{figure}

Some examples of gerunds are given in (\ref{gen}--\ref{opt}), 
quoted from \citet[1290]{Quirketal85}. 

\begin{exe} 
\ex\label{gen}  [Brown's deftly painting his daughter] is a delight to watch. 
\ex\label{acc}  I dislike [Brown painting his daughter]. 
\ex\label{opt}  Brown is well known for [painting his daughter].
\end{exe}

\noindent
The bracketed phrases have the external distribution of an NP, 
taking the subject position in (\ref{gen}), 
the complement position of a transitive verb in (\ref{acc}) and 
the complement position of a preposition in (\ref{opt}). 
The internal structure of these phrases, though, shows a mixture of nominal and verbal 
characteristics. Typically nominal is the presence of the possessive in (\ref{gen}). 
Typically verbal are the presence of the accusative subject in (\ref{acc}), of the adverbial modifier in 
(\ref{gen}) and of the NP complements in (\ref{gen}--\ref{opt}). 
To model this Rob Malouf treats the gerund as a mixed category, introducing a separate 
part-of-speech for it, which is a subtype of {\it noun}, on the one hand, and 
{\it relational}, on the other hand, see Figure \ref{ger} \citep[65]{Malouf00}. 
The distinctive properties of this mixed category are spelled out in a lexical rule 
which derives gerunds from the homophonous present participles \citep[66]{Malouf00}.\footnote{Boxed 
capitals stand for objects of type {\it list}, as in \citet{GS00}.} 

\begin{exe}
\ex 
\begin{avm} 
[head  & [{\it verb}           \\
          vform ~ {\it prp\/}] \\
 subj  & <@1 np>               \\
 comps & @A                    \\
 spr   & < ~ > ]
\end{avm} ~ $\Rightarrow$ ~ \begin{avm} 
                            [head  & {\it gerund\/} \\
                             subj  & <@1>           \\
                             comps & @A             \\
                             spr   & <@1> ] 
                            \end{avm}
\end{exe}

\noindent
The gerunds are claimed to take the same kind of complements 
as the present participles from which they are derived ($\avmbox{A}$), and
the compatibility with adverbial modifiers follows from the 
fact that adverbs typically modify objects of type {\it relational}. 
The availability of different options for realizing the subject is 
captured by the inclusion of the subject requirement of the present 
participle in both the {\sc subj} list and the {\sc spr} list of the gerund
($\avmbox{1}$). To model the two options \citet[15]{Malouf00} employs the  
bi-dimensional hierarchy of phrase types in Figure \ref{bido}. 
The combination with an accusative subject
is subsumed by {\it nonfin-head-subj-cx}, which is a subtype of 
{\it head-subject-phr} and {\it clause}. Its defining properties are 
spelled out in (\ref{acccx}) \citep[16]{Malouf00}. 

\begin{exe}
\ex\label{acccx} 
\begin{avm} 
[{\it nonfin-head-subj-cx\/}                  \\
 synsem|loc|cat|head|root ~ --                \\
 dtrs ~ <[synsem|loc|cat|head [{\it noun\/}   \\
                               case ~ {\it acc\/}]] , @1 > \\
 head-dtr ~ @1 ] 
\end{avm}
\end{exe} 

\noindent
This construction type subsumes combinations of a non-finite head with 
an accusative subject, as in (\ref{acc}). When the non-finite head is a gerund, 
the {\sc head} value of the resulting clause is {\it gerund\/} 
and since that is a subtype of {\it noun}, the clause is also a nominal phrase. 
This accounts for the fact that its external distribution is that of an NP.  
By contrast, the combination with a possessive subject
is subsumed by {\it noun-poss-cx}, which is a subtype of 
{\it head-spr-phr\/} and {\it non-clause\/} \citep[16]{Malouf00}.

\begin{exe} 
\ex\label{gencx} 
\begin{avm} 
[{\it noun-poss-cx\/}                                      \\
 synsem|loc [cat|head ~ {\it noun\/}                       \\
             content ~ {\it nom-obj\/}]                    \\
 dtrs ~ <[synsem|loc|cat|head [{\it noun\/}                \\
                               case ~ {\it gen\/}]] , @1 > \\
 head-dtr ~ @1 ] 
\end{avm}
\end{exe}
 
\noindent
This construction subsumes combinations of a nominal and a 
possessive specifier, as in {\it Brown's house}.\footnote{Malouf treats 
the English possessive as a genitive, differently from \citet{SagWasow03}.}   
It also subsumes combinations with the gerund, as in (\ref{gen}), since 
{\it gerund\/} is a subtype of {\it noun}.  

In sum, Malouf's analysis of the gerund involves the postulation of a
mixed type in the part-of-speech hierarchy, which allows the gerund to simultaneously
function as a nominal and a relational category. 


\subsubsection{The Dutch nominalized infinitive: phrasal conversion } 


The closest equivalent of the gerund in Dutch is the nominalized infinitive. 
Some examples are given in (\ref{schenk}--\ref{mobil}). 

\begin{exe} 
\ex\label{schenk} 
\gll   [geld wegschenken] maakt vrouwen gelukkig  \\
       [money donate.{\sc inf}] makes women happy \\
\trans `Donating money makes women happy' 
\ex\label{mobil}   
\gll   voor [het op diepte houden van de Vlaamse kusthavens] dient gebaggerd te worden  \\  
       for [the on depth keep.{\sc inf} of the Flemish coast.ports] needs dredged to be \\
\trans `Dredging is necessary to keep the Flemish coastal harbors accessible'  
\end{exe} 

\noindent
Also here the bracketed phrases have the external distribution of an NP, 
taking the subject position in (\ref{schenk}) and 
the complement position of a transitive verb in (\ref{mobil}). 
And also here, the internal structure shows a mix of nominal and verbal
characteristics. 
Typically nominal are the presence of the article and the postnominal PP
complement in (\ref{mobil}). 
Typically verbal are the presence of the direct object complement in (\ref{schenk})
and the predicative PP complement in (\ref{mobil}). 
To model this \citet{VanEynde19} makes a distinction between the verbal core 
and the nominal fringe of a nominalized infinitive, as in the structure of 
(\ref{mobil}), spelled out in Figure \ref{kust}.  

\begin{figure}
\begin{center}
\footnotesize
\tree
{\ntnode{Za}{[{\sc head} ~ \avmbox{1} {\it noun} , {\sc comps} ~ $<$ ~ $>$]},
  {\ntnode{Zd}{N},
    {\tnode{Ze}{het}}},
  {\ntnode{Zb}{[{\sc head} ~ \avmbox{1} , {\sc comps} ~ $<$ ~ $>$]},
    {\ntnode{Zf}{[{\sc head} ~ \avmbox{1} , {\sc comps} ~ $<$\avmbox{2} PP_{j}$>$]},
      {\ntnode{Zg}{[{\sc head} ~ \avmbox{4} {\it verb} , {\sc comps} ~ $<$NP_{j}$>$]},
        {\ntnode{Zl}{\avmbox{3}},
          {\tnode{Zm}{op diepte}}},
        {\ntnode{Zi}{[{\sc head} ~ \avmbox{4} , {\sc comps} ~ $<$NP_{j} , \avmbox{3} PP_{k}$>$]},
          {\tnode{Zj}{houden}}}}},
    {\ntnode{Zk}{\avmbox{2}}, 
      {\tnode{Zh}{van de kusthavens}}}}}
\nodeconnect{Za}{Zd}
\nodeconnect{Zd}{Ze}
\nodeconnect{Za}{Zb}
\nodeconnect{Zb}{Zf}
\nodeconnect{Zf}{Zg}
\nodeconnect{Zg}{Zl}
\nodetriangle{Zl}{Zm}
\nodeconnect{Zg}{Zi}
\nodeconnect{Zi}{Zj}
\nodeconnect{Zb}{Zk}
\nodetriangle{Zk}{Zh}
\caption{\label{kust} Dutch Nominalized Infinitive }
\normalsize
\end{center}
\end{figure}    

The infinitive is treated as unambiguously verbal at the lexical level and 
remains verbal when combined with its predicative PP complement, but then it  
is converted into a nominal projection and combined with a postnominal PP[{\it van\/}]
complement and the definite article.\footnote{Since the article is a pronominal 
determiner, it is assigned the category N in the functor treatment.}     
The conversion is modelled in terms of a non-headed phrase type, 
since it does not comply with the Head Feature Principle: 
The {\sc head} value of the mother is not shared with the daughter.  
More specifically, there is a subtype of the non-headed phrases, called {\it convert-phr}, 
whose defining characteristic is that they have a single daughter. 
In that respect they differ from coordinate phrases, which have at least two daughters.  

\begin{exe} 
\ex\label{conv} 
\begin{avm} 
[{\it convert-phr\/} \\
 dtrs ~ <X> ]
\end{avm}
\end{exe}

\noindent
The conversion which we observe in the Dutch nominalized infinitive is modeled 
in terms of a phrase type that inherits properties of {\it convert-phr\/} and 
{\it non-clause}. Its properties are spelled out in (\ref{nominf}). 

\begin{figure}
\begin{center}
\footnotesize
\tree
   {\ntnode{Zj}{\it phrase},
     {\ntnode{Zh}{{\sc headedness}}, 
        {\ntnode{Zi}{\it non-headed-phr},
          {\tnode{Zo}{coord-phr}},
          {\ntnode{Zp}{\it convert-phr},
            {\tnode{Zz}{nom-inf-phr}}}, 
          {\tnode{Zq}{...}}}},
     {\ntnode{Zl}{{\sc clausality}}, 
       {\tnode{Zx}{clause}},
       {\ntnode{Zy}{\it non-clause}}}}
\nodeconnect{Zj}{Zh}
\nodeconnect{Zh}{Zi}
\nodeconnect{Zj}{Zl}
\nodeconnect{Zl}{Zx}
\nodeconnect{Zl}{Zy}
\nodeconnect{Zi}{Zo}
\nodeconnect{Zi}{Zp}
\nodeconnect{Zi}{Zq}
\nodeconnect{Zp}{Zz}
\nodeconnect{Zy}{Zz}
\caption{\label{bido6} Nominalized Infinitives in the Bi-dimensional Hierarchy  } 
\normalsize
\end{center}
\end{figure}

\begin{exe} 
\ex\label{nominf} 
\begin{avm} 
[{\it nom-inf-phr\/}                                      \\
 synsem|cat [head [{\it noun\/}                           \\
                   number ~ {\it singular\/}              \\
                   gender ~ {\it neuter\/}]               \\
             subj ~ < ~ >                                 \\
             comps ~ <(pp_{j})> ~ $\oplus$ ~ <(pp_{i})> ~ $\oplus$ ~ @A \\ 
             marking ~ {\it unmarked\/}]                     \\           
 dtrs ~ <[synsem|cat [head|vform ~ {\it inf\/}               \\
                      subj ~ <np_{j}>                         \\
                      comps ~ <(np_{i})> ~ $\oplus$ ~ @A ]]>]
\end{avm}
\end{exe}

\noindent
The daughter is an infinitive with a non-empty {\sc subj} list and 
a possibly empty {\sc comps} list. $\avmbox{A}$  
The mother is an unmarked singular neuter nominal that inherits the 
unsaturated complement requirements of its daughter, 
albeit with the twist that NP complements become PP complements. 
The subject requirement of the infinitive is added to the {\sc comps} list 
of the nominal and becomes a PP too. It is made optional, since 
it is often left unexpressed, as in (\ref{schenk}--\ref{mobil}). If present,  
it is introduced by {\it van\/} `of' or {\it door\/} `by', as in 
(\ref{van}) and (\ref{door}) respectively. 

\begin{exe} 
\ex\label{van} 
\gll   het trage afsterven van de koraalriffen   \\
       the slow die.{\sc inf} of the coral.reefs \\
\trans `the slow dying of the coral reefs' 
\ex\label{door}
\gll   het uitschakelen van Chelsea door Real Madrid      \\
       the eliminate.{\sc inf} of Chelsea by Real Madrid  \\ 
\trans `the elimination of Chelsea by Real Madrid'  
\end{exe} 

For a full treatment that also covers the semantic type shift that 
accompanies the syntactic conversion, we refer to \citet{VanEynde19}.


\subsection{Idiosyncratic P+NP combinations} 


When a nominal combines with a preposition the result is usually a PP, but 
not always. The French {\it de\/} `of', for instance, heads a PP in 
{\it je viens de Roubaix\/} `I come from Roubaix', 
but not in {\it beaucoup de farine\/} `much flour'. 
As argued in \citet{Abeilleetal04}, there are many differences 
between these two uses of {\it de\/}, both syntactic and semantic ones.
To model them they treat the former as the ordinary head of a PP and 
the latter as a weak head. Typical of a weak head is that it shares 
nearly all properties of its complement, as spelled out in (\ref{de}).

\begin{exe} 
\ex\label{de} 
\begin{avm} 
[cat [head ~ @1                              \\
      subj ~ @A                              \\
      spr ~ @B                               \\
      comps ~ <[cat [head ~ @1               \\
                     subj ~ @A               \\
                     spr ~ @B                \\
                     comps ~ < ~ >]          \\
                     mark ~ {\it unmarked\/} \\
                content ~ @4 ]>              \\ 
      mark ~ {\it de\/}]                      \\
 content ~ @4 ] 
\end{avm}
\end{exe} 

\noindent
{\it de\/} has the same values for {\sc head}, {\sc subj}, {\sc spr} and 
{\sc content} as its complement. 
When combined with {\it farine\/} `flour', it is, hence, a noun 
that selects a specifier and that denotes a parameter.\footnote{The 
weak {\it de\/} also combines with infinitival VPs, as in {\it de sortir un peu
te ferait du bien\/} `getting out a bit would do you some good'. In that 
combination, {\it de\/} is a verb that selects a subject and that denotes 
a state-of-affairs.} The only difference between the weak head and its complement  
concerns the {\sc marking} value: {\it de\/} requires an unmarked complement, but 
its own {\sc marking} value is of type {\it de}, and this value is shared with the mother.  
Specifiers that require the presence of {\it de}, such as {\it beaucoup\/} `much/many', 
select a nominal with that {\sc marking} value, 
as shown in (\ref{coup}).\footnote{In this AVM, quoted from \citet[18]{Abeilleetal04}, 
the value of {\sc spec} is of type {\it synsem}, as in \citet{PS94}, and not of type 
{\it semantic-object}, as in \citet{GS00}.}  

\begin{exe} 
\ex\label{coup} 
\begin{avm} 
[cat|head [{\it adverb\/}                                   \\
           spec|loc [cat [head ~ {\it noun\/}               \\
                          spr ~ <X>                         \\
                          mark ~ {\it de\/}]                \\
                     content [index ~ @2                    \\
                              restr ~ $\avmbox{\Sigma}$ ]]] \\ 
 content ~ @1 [{\it beaucoup-rel\/}        \\
               index ~ @2                  \\
               restr ~ $\avmbox{\Sigma}$ ] \\
 store ~ \{ @1 \} ]
\end{avm}
\end{exe} 

\noindent
Besides, the selected nominal is required to be unsaturated for {\sc spr} and 
the quantifier is required to share the index and restrictions of its head sister. 

The weak head analysis of \citet{Abeilleetal04} fits the mold of the specifier treatment. 
The same data can be analyzed in a way that fits the mold of the functor treatment,
as shown in Figure \ref{beau}. 
In this analysis the preposition in {\it beaucoup de farine\/} `much flour'
is a functor that selects a nominal of type {\it bare\/} and that 
shares its own {\sc marking} value ({\it de\/}) with the mother. 
The functor {\it beaucoup} in its turn selects a nominal with the {\sc marking} value 
{\it de\/} and shares its own {\sc marking} value with the NP as a whole. 
In this analysis {\it de\/} does not share the part-of-speech, valence and 
meaning of its nominal sister. Instead, those properties are shared directly between
{\it farine\/} and {\it de farine}. 

\begin{figure}
\begin{center}
\footnotesize
\tree
{\ntnode{Zz}{[{\sc head} ~ \avmbox{1} {\it noun}[{\sc sel} ~ {\it none\/}] , {\sc mark} ~ \avmbox{2} {\it marked\/}]},
  {\ntnode{Za}{[{\sc head} ~ [{\it adv} {\sc sel} ~ \avmbox{4}] , {\sc mark} ~ \avmbox{2}]},
    {\tnode{Zp}{beaucoup}}},
  {\ntnode{Zb}{\avmbox{4} [{\sc head} ~ \avmbox{1} , {\sc mark} ~ \avmbox{5} {\it de\/}]},
    {\ntnode{Zc}{[{\sc head} ~ [{\it prep} {\sc sel} ~ \avmbox{3}] , {\sc mark} ~ \avmbox{5}]},
      {\tnode{Zq}{de}}},  
    {\ntnode{Zd}{\avmbox{3} [{\sc head} ~ \avmbox{1} , {\sc mark} ~ {\it bare\/}]},
      {\tnode{Zh}{farine}}}}}
\nodeconnect{Zz}{Za}
\nodeconnect{Zz}{Zb}
\nodeconnect{Za}{Zp}
\nodeconnect{Zb}{Zc}
\nodeconnect{Zb}{Zd}
\nodeconnect{Zc}{Zq}
\nodeconnect{Zd}{Zh}
\normalsize
\caption{\label{beau} A Prepositional Functor }
\end{center}
\end{figure}

While the differences between the weak head treatment and the functor treatment are small and subtle, 
they may have empirical consequences. This is shown in \citet{Maekawa15}, which offers 
an analysis of English nominals of the {\it kind/type/sort\/} variety.  
To account for the fact that the determiner shows agreement with the 
rightmost noun in {\it these sort of problems\/} and {\it those kind of pitch changes\/}
Maekawa considers the option of treating {\it of\/} and the immediately preceding noun as 
weak heads, but considers this unsatisfactory, since it has the unwanted effect of treating 
{\it sort/kind/type} as plural. As an alternative, he develops an analysis in which {\it of\/} and 
the preceding noun are functors \citep[149]{Maekawa15}. This avoids the unwanted side-effect, 
since functors do not share the {\sc head} value of their head sister. 
Further evidence is provided in \citet{Maekawa16}, which analyzes combinations of a   
singular numeral with a plural noun, as in {\it these thousand pages\/}. If 
{\it thousand} is treated as a weak head, it inadvertantly inherits the number value of
{\it pages}, i.e. plural, whereas it is in fact singular, its plural counterpart being 
{\it thousands}. This unwanted effect is avoided if {\it thousand\/} is treated as a functor.


\subsection{Other nominals with idiosyncratic properties} 


A much studied nominal with idiosyncratic traits is the binominal noun phrase 
construction (BNPC), exemplified in (\ref{climb}). 

\begin{exe}
\ex\label{climb}
\begin{xlist}
\ex  She blames it on [her nitwit of a husband]. 
\ex  She had [a skullcracker of a headache]. 
\end{xlist}
\end{exe}

\noindent
In contrast to ordinary [NP--{\it of\/}--NP] sequences, 
as in {\it an employee of a Japanese company}, where the 
first nominal is the head of the entire NP, and where the second 
nominal is part of its PP adjunct, the {\sc bnpc} shows a pattern
in which the relation between the nominals is a predicative one: 
her husband is claimed to be a nitwit, and the headache is claimed to be 
like a skullcracker. HPSG treatments of the {\sc bnpc} are provided in
\citet{KimSells14} and \citet{VanEynde18}. The latter uses 
the phrase type hierarchy in (\ref{prot}), defining the {\sc bnpc} as 
a maximal type that inherits from {\it head-independent-phr\/} and 
{\it inverted-predication}. To capture the intuition that the second
nominal is the head of the entire NP, the preposition {\it of\/} is 
not treated as the head of a PP, but as a functor that selects 
a nominal head. 

Comparable to NPs with a verbal core are NPs with an adjectival core,
such as {\it the very poor\/} and {\it the merely skeptical}. They are  
described and provided with an HPSG analysis in \citet{ArnoldSpencer2015}.
Interestingly, it employs a device for phrasal conversion that is similar to 
the one for the Dutch nominalized infinitive in Section 3.2.2. 

Idiosyncratic are also the nominals with an extracted determiner, as in 
the French (\ref{combi}) and the Dutch (\ref{gelu}). 

\begin{exe} 
\ex\label{combi}  
\gll   Combien as-tu lu [\_\_ de livres en latin] ?    \\
       how.many have-you read [\_\_ of books in Latin] \\ 
\trans `How many books have you read in Latin?' 
\ex\label{gelu}
\gll   Wat zijn dat [\_\_ voor vreemde geluiden] ? \\
       what are that [\_\_ for strange noises]     \\ 
\trans `What kind of strange noises are those?'  
\end{exe} 

\noindent
The French example is analyzed in \citet[20--21]{Abeilleetal04} and the Dutch one in 
\citet[47--50]{VanEynde04}. 

Another special case is apposition. It comes in (at least) two types, known as 
close apposition and loose apposition. Relevant examples are respectively 
{\it my brother Richard is a soldier\/} and {\it Sarajevo, the capital of Bosnia, is where WW I began}.
Both are compared and analyzed in \citet{Kim12} and \citet{Kim14}. 
\citet{VanEyndeKim16} provides an analysis of loose apposition in
Sign-Based Construction Grammar. 



\section{Conclusion} 


This chapter has provided a survey of how nominals are analyzed in HPSG. 
Over time three treatments have taken shape, i.e. the specifier treatment, 
the functor treatment and the DP treatment. They were presented and applied to ordinary 
nominals in Section 2. Nominals with idiosyncratic properties were discussed in Section 3.
Modeling them requires a relaxation of the strictly lexicalist stance of early HPSG.  
The more flexible tools of constructional HPSG have been put to use in the analysis 
of the Big Mess Construction and of nominals with a verbal core, such as gerunds and 
nominalized infinitives. For other nominals with idiosyncratic properties pointers 
have been given to relevant literature. 

In terms of the NP vs. DP debate, it is clear that the NP approach has been 
more successful in HPSG than the DP approach. This fits in with the tendency to 
avoid the postulation of functional categories with their respective projections. 
Clauses, for instance, are not analyzed as IPs either. The finer-grained 
differences between the specifier treatment and the functor treatment are 
a topic of ongoing debate. They both have their advocates and for the analysis of 
ordinary nominals there does not seem to be any evidence that decisively tilts the balance. 
For nominals with idiosyncratic properties, however, there is a clear tendency in the recent 
literature to adopt the functor treatment, usually in combination with phrasal constraints.    

\bibliography{prenom}
\bibliographystyle{unified}  

\end{document} 



For the analysis of nominals without specifier Netter 
assumes that they are headed by the noun, as in Figure \ref{losnet}.  
To resolve the underspecifation in the {\sc fcompl} feature 
he adds a type, called {\it dp}, 
to the hierarchy of {\sc synsem} values and declares it to be both saturated and functionally 
complete. Unification of this type with the {\sc synsem} value of {\it pictures of Sandy\/} 
results in a phrase with a positive {\sc fcompl} value.  

\begin{figure}
\begin{center}
\footnotesize
\tree
    {\ntnode{Zg}{[{\sc major} ~ \avmbox{1} [--V, +N[{\it plu\/}]] , {\sc minor$|$fcompl} ~ + $\vee$ --]},
      {\ntnode{Zi}{[{\sc major} ~ \avmbox{1} , {\sc minor$|$fcompl} ~  + $\vee$ --]},
        {\tnode{Zj}{pictures}}},
      {\ntnode{Zk}{PP}, 
        {\tnode{Zh}{of Sandy}}}}
\nodeconnect{Zg}{Zi}
\nodeconnect{Zi}{Zj}
\nodeconnect{Zg}{Zk}
\nodetriangle{Zk}{Zh}
\caption{\label{losnet} Nominals without Specifier}
\normalsize
\end{center}
\end{figure}


The semantic relation between the prenominal genitive and the head noun is 
modeled in the AVM of the genitive common noun {\it lands}, spelled out in (\ref{lands}). 

\begin{exe} 
\ex\label{lands}
\begin{avm}
[cat [head [{\it noun\/}                    \\           
            sel|loc [cat [head ~ {\it noun\/}   \\
                          marking ~ {\it unmarked\/}]  \\
                     content|index ~ @2 ]]      \\         
      marking ~ {\it unmarked\/} ]          \\
 content [index ~ @1                        \\
          restr ~ \{[{\it the-rel\/}       \\
                     possessor ~ @1         \\
                     possessed ~ @2 ] , 
                    [{\it land\/}           \\
                     arg ~ @1 ] \} ]]
\end{avm}
\end{exe}

\noindent
The genitive noun denotes an entity that is the identified as the possessor
($\avmbox{1}$) and selects an unmarked nominal that denotes the possessed
($\avmbox{2}$). 

\subsubsection{Nominals without specifier} 


Not all nominals have an overt specifier. 
Mass nouns and plurals, for instance, are routinely used without specifier, 
as in {\it oil is expensive\/} and {\it books are cheap}. 
To accommodate this one could employ phonetically empty determiners, 
as in Transformational Grammar, but this does not square well with the tendency 
in {\sc hpsg} to avoid the use of empty elements as much as possible. 
An alternative is to make the selection of a determiner optional, as in
{\sc spr} ~ $<$(Det)$>$, but this complicates the treatment of modifiers, 
for if the {\sc spr} list of a mass noun or a plural is allowed to be empty, 
then one has to allow the attributive adjective in {\it black oil is expensive\/} 
to combine with a nominal with an empty {\sc spr} list, and if that is allowed, 
then one also allows it to combine with a fully saturated nominal, as in the ill-formed
*{\it black the oil}.
Another alternative is proposed in \citet[191--192]{GS00}. It involves the addition
of a phrase type, called {\it bare-nominal-phrase\/}, that licenses the cancelation 
of an {\sc spr} requirement without the presence of an overt specifier, 
as in Figure \ref{los}. 

\begin{figure}
\begin{center}
\footnotesize
\tree
{\ntnode{Za}{[{\sc head} ~ \avmbox{1} {\it noun} , {\sc spr} ~ $<$ ~ $>$]},
  {\ntnode{Zf}{[{\sc head} ~ \avmbox{1} , {\sc spr} ~ $<$\avmbox{2}$>$]},
    {\ntnode{Zl}{[{\sc head} ~ [{\it adj} {\sc mod} ~ \avmbox{3}]]},
      {\tnode{Zm}{black}}},
    {\ntnode{Zg}{\avmbox{3} [{\sc head} ~ \avmbox{1} , {\sc spr} ~ $<$\avmbox{2} Det$>$]},
      {\tnode{Zj}{oil}}}}}
\nodeconnect{Za}{Zf}
\nodeconnect{Zf}{Zl}
\nodeconnect{Zl}{Zm}
\nodeconnect{Zf}{Zg}
\nodeconnect{Zg}{Zj}
\caption{\label{los} Nominals without specifier}
\normalsize
\end{center}
\end{figure}



\begin{figure}
\begin{center}
\footnotesize
\tree
{\ntnode{Za}{[{\sc head} ~ \avmbox{1} {\it noun} , {\sc spr} ~ $<$ ~ $>$]},
  {\ntnode{Zd}{\avmbox{2} [{\sc head} ~ [{\it det\/} {\sc spec} ~ \avmbox{3}]]},
    {\tnode{Ze}{every}}},
  {\ntnode{Zf}{[{\sc head} ~ \avmbox{1} , {\sc spr} ~ $<$\avmbox{2}$>$ , {\sc content} ~ \avmbox{3}]},
    {\tnode{Zj}{red box}}}}
\nodeconnect{Za}{Zd}
\nodeconnect{Zd}{Ze}
\nodeconnect{Za}{Zf}
\nodetriangle{Zf}{Zj}
\caption{\label{laa} Mutual Selection in NPs}
\normalsize
\end{center}
\end{figure}


\begin{figure}
\begin{center} 
\tree   {\ntnode{Zv}{\it headed-phrase}, 
          {\tnode{Zl}{head-complements-phr}},
          {\tnode{Zs}{head-specifier-phr}},
          {\tnode{Zu}{head-adjunct-phr}}, 
          {\tnode{Zg}{...}}}
\nodeconnect{Zv}{Zl}
\nodeconnect{Zv}{Zs}
\nodeconnect{Zv}{Zu}
\nodeconnect{Zv}{Zg}
\caption{\label{typol} Hierarchy of headed phrases}
\end{center}
\end{figure}

Starting point for the presentation is the partial hierarchy of phrase types in Figure \ref{typol}.  
Headed phrases are subject to the Head Feature Principle, spelled out in (\ref{hfp}). 

\begin{exe} 
\ex\label{hfp}  {\it headed-phr\/} ~ $\Rightarrow$ ~ \begin{avm} 
                                          [synsem|loc|cat|head ~ @1 {\it part-of-speech\/}  \\
                                           head-dtr|synsem|loc|cat|head ~ @1 ]
                                          \end{avm} 
\end{exe} 



This is also useful to deal with partitive constructions, as in (\ref{partit}),
brought to my attention by Bob Borsley. 

\begin{exe} 
\ex\label{partit}
\begin{xlist} 
\ex  All of the beer was drunk. 
\ex  All of the men were drunk. 
\end{xlist} 
\end{exe} 

\noindent
Since the finite verb shows number agreement with the rightmost noun,
the most straightforward analysis is one in which that noun is the head
of the entire NP, and in which the preposition is a functor, rather than 
the head of a PP. 


\ex
\gll   Paul n'a pas lu [de livre]. \\
       Paul did not read [of book] \\
\trans `Paul did not read any book.'  
\end{exe} 




An illustration of its use is given in Figure \ref{mark}. 

\begin{figure}
\begin{center}
\tree
{\ntnode{Zz}{[{\sc head} ~ \avmbox{4} ~ {\it noun} , {\sc mark} ~ \avmbox{2} ~ {\it marked\/}]},
  {\ntnode{Za}{[{\sc mark} ~ \avmbox{2}]},
    {\tnode{Zp}{that}}},
  {\ntnode{Zb}{\avmbox{4} [{\sc head} ~ \avmbox{4} , {\sc mark} ~ \avmbox{1} {\it unmarked\/}]}, 
    {\ntnode{Zd}{[{\sc mark} ~ \avmbox{1}]},
      {\tnode{Zh}{long}}},
    {\ntnode{Zc}{\avmbox{3} [{\sc head} ~ \avmbox{4} , {\sc mark} ~ \avmbox{1}]},
      {\tnode{Zq}{bridge}}}}}
\nodeconnect{Zz}{Za}
\nodeconnect{Zz}{Zb}
\nodeconnect{Za}{Zp}
\nodeconnect{Zb}{Zc}
\nodeconnect{Zb}{Zd}
\nodeconnect{Zc}{Zq}
\nodeconnect{Zd}{Zh}
\caption{\label{mark} Marking }
\end{center}
\end{figure}



That may be all there is, as in the subject of {\it oil is expensive}, 
but usually the noun is accompanied by one or more dependents. 
They can precede the noun, such as the article and the adjective in 
{\it the red box}, or follow the noun, such as the prepositional phrase
and the relative clause in {\it books about WW I which are out of print}. 

Such combinations are constrained by co-occurrence restrictions. 
An obvious one concerns the possibility of stacking. 
While a noun can be combined with more than one adjective or PP, 
as in {\it red wooden boxes\/} and {\it books on sale about WW II}, 
it cannot be combined with more than one article: *{\it the a box}. 
In that respect, the articles belong to a group of words, known as 
determiners (Det), which also includes
the demonstrative {\it this/that/these/those}, 
the possessive {\it my/our/your/his/her/its/their}, 
the quantifying {\it every/each/some/no/any\/} and 
the interrogative {\it which/whose}. A common property of these words 
is that they are in complementary distribution: 
*{\it the this box, the your box, ...}.  
One way to model these observations is the phrase structure rule in (\ref{ps0}).  

\begin{exe}
\ex\label{ps0}  NP ~ $\rightarrow$ ~ (Det) ~ A* ~ N ~ PP* ~ S*  ~~~~ (where * is the Kleene star) 
\end{exe} 

\noindent
This rule licenses flat structures in which the dependents are all sisters of the noun. 
It provides a blueprint of what a regular nominal is, but in order to model its structure 
{\sc hpsg} adopts a finer-grained layered approach, reminiscent of X-bar syntax. 



\subsection{X-bar syntax} 



A non-transformational version was proposed 
in Generalized Phrase Structure Grammar.



Further doubt is cast by the existence of prepositions and verbs  
which do not require their complement to contain a determiner, also 
if that complement is singular and count.  
The Dutch prepositions {\it met\/} `with' and {\it zonder\/} `without', 
for instance, are compatible with a bare singular count noun, as in 
{\it een huis met garage\/} `a house with garage' and 
{\it een huis zonder dak\/} `a house without roof'.   
The same holds for the Dutch and German copular verbs, as illustrated by 
{\it Henk wordt leraar\/} `Henk becomes a teacher' and {\it Hans ist Lehrer\/} 
`Hans is a teacher', respectively. 



This is currently the most promising avenue for an analysis of nominal structures which 
does justice both to what is regular and to what is idiosyncratic about them. 


Evidence for the assignment of a more hierarchical structure is provided by  
two classical constituency tests. The first one concerns conjoinability. 
It reveals that it is not only possible to conjoin full NPs, but also parts of NPs. 
In {\it every man above forty and woman under thirty}, for instance, 
{\it every\/} is combined with the nominal {\it man above forty and woman under thirty}. 
This suggests a binary branching structure in which the determiner is combined 
with the rest of the noun phrase (Nom), as in (\ref{ps1}), quoted from 
\citet[31--32]{SagWasow03}. 

\begin{exe}
\ex\label{ps1}  NP ~ $\rightarrow$ ~ (Det) ~ Nom
\end{exe} 

\noindent
Confirming evidence is provided by the pro-form test. While personal pronouns are 
distributionally equivalent to full NPs, suggesting that they are pro-NPs, there are
also pro-forms which are distributionally equivalent to nominals without determiner, 
such as the English {\it one\/} and its plural counterpart {\it ones\/} in
(\ref{app}) \cite{Payneetal2013}. 

\begin{exe}
\ex\label{app}  
\begin{xlist} 
\ex  John took this apple_{i} and Mary took that one_{i}. 
\ex  John took these apples_{i} and Mary took those ones_{i}. 
\end{xlist}
\end{exe}

The same tests provide evidence that the internal structure of Nom 
is binary branching too. Adjectives, for instance, are added 
recursively in (\ref{ps2}), quoted from \citet[80]{Levin17}.

\begin{exe}
\ex\label{ps2}   Nom ~ $\rightarrow$ ~ Adj ~ Nom 
\end{exe} 

\noindent
This accounts for the fact that an adjective can scope over a conjunction 
of Noms, as in (\ref{king}), and that the pro-Nom {\it one(s)\/} can just as well 
stand for a nominal without adjective as for a nominal that includes an adjective,  
as in (\ref{box}--\ref{wood}). 

\begin{exe} 
\ex\label{king} a large [wooden box or iron cupboard]
\ex\label{box}  John bought the red box_{i} and Mary bought the blue one_{i}. 
\ex\label{wood} John bought the [wooden box]_{i} with the green lid and Mary bought the one_{i} with the blue lid. 
\end{exe}

\noindent 
For PPs and clausal dependents it is common to make a distinction between 
modifiers and complements. While modifying PPs and clauses are added 
by the same kind of recursive rules as the adjectives, as in (\ref{ps4}), 
the addition of a PP complement or a clausal complement is typically modeled 
by a non-recursive rule, as in (\ref{ps3}). 

\begin{exe} 
\ex\label{ps4} 
\begin{xlist}
\ex   Nom ~ $\rightarrow$ ~ Nom ~ PP 
\ex   Nom ~ $\rightarrow$ ~ Nom ~ S
\end{xlist} 
\ex\label{ps3} 
\begin{xlist} 
\ex   Nom ~ $\rightarrow$ ~ N ~ (PP) 
\ex   Nom ~ $\rightarrow$ ~ N ~ (S) 
\end{xlist} 
\end{exe} 

\noindent
This captures the fact that the number of complements is limited by 
the argument structure of the noun.
The relational noun {\it sister}, for instance, takes at most  
one PP complement, as in {\it the sister of Leslie}, and 
the deverbal noun {\it claim\/} takes at most one clausal complement, as in 
{\it the claim that Brexit is inevitable}. 



Since the value of {\sc spec} is of type {\it synsem}, it can also be used to 
model morpho-syntactic co-occurrence restrictions, as in \citet[371--373]{PS94}, 
where it is used to model the effect of declension class ({\it weak\/} vs. 
{\it strong\/}) on the inflection of prenominal adjectives in German.  



as in the second conjunct of the
German {\it das rote Auto und das blaue\/} `the red car and the 
blue (one)', discussed and analyzed in \citet[152--170]{Netter96}. 
The phenomenon is less common in English, but it is not entirely absent 
there either, as shown by (\ref{many}), discussed in 

\begin{exe} 
\ex\label{many}
\begin{xlist}
\ex  Paul read 20 abstracts. {\it The most interesting\/} were on creoles. 
\ex  Guy did not see {\it any}, but we saw {\it many}. 
\end{xlist}
\end{exe} 




\begin{figure}
\tree
    {\ntnode{Zg}{S [{\sc mark} ~ \avmbox{1}]},
      {\ntnode{Zi}{[{\sc mark} ~ \avmbox{1} {\it marked} , {\sc spec} ~ \avmbox{2}]},
        {\tnode{Zj}{that}}},
      {\ntnode{Zk}{\avmbox{2} S [{\sc mark} ~ {\it unmarked\/}]}, 
        {\tnode{Zh}{John left}}}}
\nodeconnect{Zg}{Zi}
\nodeconnect{Zi}{Zj}
\nodeconnect{Zg}{Zk}
\nodetriangle{Zk}{Zh}
\caption{\label{compl}}
\end{figure}



\begin{frame}{Our last common ancestor} 

\begin{itemize}
\item   X^{1} ~ $\rightarrow$ ~ X^{0} ~ Complements
\item   X^{2} ~ $\rightarrow$ ~ Specifier ~ X^{1} 
\end{itemize} 

\bigskip

\begin{itemize}
\item   X^{1} ~ $\rightarrow$ ~ Modifier ~ X^{1} 
\item   X^{1} ~ $\rightarrow$ ~ X^{1} ~ Modifier
\end{itemize} 


\bigskip

\bigskip

\footnotesize
Chomsky (1970), Remarks on nominalization 

\smallskip

Jackendoff (1977), {\it X-bar Syntax: a study of phrase structure} 

\normalsize

\end{frame} 

\begin{frame}{Generalized Phrase Structure Grammar} 

\begin{center} 
\tree{\ntnode{Za}{N^{2}},
  {\ntnode{Zd}{Det},
    {\tnode{Ze}{that}}},
  {\ntnode{Zf}{N^{1}},
    {\ntnode{Zc}{AP},
      {\tnode{Zl}{very tall}}},
    {\ntnode{Zp}{N^{1}},
      {\ntnode{Zg}{N^{1}},
        {\ntnode{Zi}{N},
          {\tnode{Zb}{sister}}},
        {\ntnode{Zk}{PP}, 
          {\tnode{Zj}{of Leslie}}}},
      {\ntnode{Zh}{S [+R]}, 
        {\tnode{Zq}{who we met}}}}}}
\nodeconnect{Za}{Zd}
\nodeconnect{Zd}{Ze}
\nodeconnect{Za}{Zf}
\nodeconnect{Zf}{Zc}
\nodetriangle{Zc}{Zl}
\nodeconnect{Zf}{Zp}
\nodeconnect{Zp}{Zg}
\nodeconnect{Zg}{Zi}
\nodeconnect{Zi}{Zb}
\nodeconnect{Zg}{Zk}
\nodetriangle{Zk}{Zj}
\nodeconnect{Zp}{Zh}
\nodetriangle{Zh}{Zq}
\end{center}

\bigskip

\footnotesize
Gazdar, Klein, Pullum \& Sag (1985), {\it Generalized Phrase Structure Grammar}, 126
\normalsize

\end{frame} 

\begin{frame}{Head-driven Phrase Structure Grammar}    

\begin{itemize} 
\item  The specifier treatment 
\item  The DP treatment 
\item  An intermezzo on NP completeness
\item  The functor treatment 
\item  An idiosyncratic nominal: the nominalized infinitive 
\end{itemize} 

\end{frame} 

\begin{frame} 

\begin{center} 
\Huge 
The Specifier Treatment 
\normalsize
\end{center}

\end{frame} 

\begin{frame}{Heads, complements and specifiers}   

\begin{center}
\tree
{\ntnode{Za}{N [{\sc spr} ~ $<$ ~ $>$ , {\sc comps} ~ $<$ ~ $>$]},
  {\ntnode{Zd}{\avmbox{1}},
    {\tnode{Ze}{that}}},
  {\ntnode{Zf}{N [{\sc spr} ~ $<$\avmbox{1}$>$ , {\sc comps} ~ $<$ ~ $>$]},
    {\ntnode{Zl}{Adj},
      {\tnode{Zm}{tall}}},
    {\ntnode{Zg}{N [{\sc spr} ~ $<$\avmbox{1}$>$ , {\sc comps} ~ $<$ ~ $>$]},
      {\ntnode{Zi}{N [{\sc spr} ~ $<$\avmbox{1} Det$>$ , {\sc comps} ~ $<$\avmbox{2} PP$>$]},
        {\tnode{Zj}{sister}}},
      {\ntnode{Zk}{\avmbox{2}}, 
        {\tnode{Zh}{of Leslie}}}}}}
\nodeconnect{Za}{Zd}
\nodeconnect{Zd}{Ze}
\nodeconnect{Za}{Zf}
\nodeconnect{Zf}{Zl}
\nodeconnect{Zl}{Zm}
\nodeconnect{Zf}{Zg}
\nodeconnect{Zg}{Zi}
\nodeconnect{Zi}{Zj}
\nodeconnect{Zg}{Zk}
\nodetriangle{Zk}{Zh}
\end{center}

\bigskip

The values of the valence features register the degree of saturation; they also model co-occurrence constraints: 
*{\it those sister of Leslie\/} and *{\it that sister until Leslie}

\bigskip

\footnotesize
Pollard \& Sag (1994), {\it Head-driven Phrase Structure Grammar}. 
\normalsize

\end{frame} 

\begin{frame}{Typed feature structures} 

\tree
{\ntnode{Zz}{\it sign},
  {\ntnode{Ze}{\it lexical-sign},  
    {\tnode{Zf}{lexeme}},
    {\tnode{Zg}{word}}},
  {\ntnode{Zh}{\it phrase}, 
    {\tnode{Zi}{headed-phrase}}, 
    {\tnode{Zj}{nonheaded-phrase}}}}
\nodeconnect{Zz}{Ze}
\nodeconnect{Ze}{Zf}
\nodeconnect{Ze}{Zg}
\nodeconnect{Zz}{Zh}
\nodeconnect{Zh}{Zi}
\nodeconnect{Zh}{Zj}

\bigskip

{\it sign\/} : \begin{avm}
               \[phonology & {\it list\/}({\it speech-sound\/}) \\
                 synsem    & {\it synsem\/}\] 
               \end{avm}

\bigskip

{\it phrase\/} : \begin{avm} \[daughters ~ {\it nelist\/}({\it sign\/})\] \end{avm}

\bigskip

{\it headed-phr\/} : \begin{avm} \[head-daughter ~ {\it sign\/}\] \end{avm}

\end{frame}   

\begin{frame}   

{\it synsem\/} : \begin{avm}
                 \[category & {\it category\/}    \\
                   content  & {\it semantic-object}\] 
                 \end{avm}

\bigskip

\bigskip

{\it category\/} : \begin{avm}
                   \[head  & {\it part-of-speech\/}       \\
                     spr   & {\it list\/}({\it synsem\/}) \\
                     comps & {\it list\/}({\it synsem\/}) \\
                     marking & {\it marking\/}\]
                    \end{avm}

\end{frame} 

\begin{frame}{Parts of speech}

\tree
{\ntnode{Zz}{\it part-of-speech},
  {\ntnode{Zh}{\it substantive}, 
    {\tnode{Zj}{noun}},
    {\tnode{Zk}{verb}},
    {\tnode{Zl}{adj}}, 
    {\tnode{Zm}{prep}}}, 
  {\ntnode{Zi}{\it functional},
    {\tnode{Zp}{det}},
    {\tnode{Zq}{...}}}}  
\nodeconnect{Zz}{Zh}
\nodeconnect{Zz}{Zi}
\nodeconnect{Zi}{Zp}
\nodeconnect{Zi}{Zq}
\nodeconnect{Zh}{Zl}
\nodeconnect{Zh}{Zj}
\nodeconnect{Zh}{Zk}
\nodeconnect{Zh}{Zm}

\bigskip

{\it noun\/} : \begin{avm}
               \[case   & {\it case\/}   \\
                 number & {\it number\/} \\
                 gender & {\it gender\/} \]
                \end{avm} 

\end{frame}   

\begin{frame}{Headed phrases} 

\tree
{\ntnode{Zz}{\it sign},
  {\ntnode{Zh}{\it phrase}, 
    {\ntnode{Zi}{\it headed-phr}, 
      {\tnode{Zj}{head-spr-phr}},
      {\tnode{Zk}{head-comp-phr}},
      {\tnode{Zl}{head-adj-phr}}}}}
\nodeconnect{Zz}{Zh}
\nodeconnect{Zh}{Zi}
\nodeconnect{Zi}{Zj}
\nodeconnect{Zi}{Zk}
\nodeconnect{Zi}{Zl}

\bigskip

{\it headed-phr\/} : \begin{avm}
                     \[synsem\|cat\|head ~ $\avmbox{1}$ {\it part-of-speech\/} \\
                       hd-dtr\|synsem\|cat\|head ~ $\avmbox{1}$ \] 
                     \end{avm}

\end{frame} 

\begin{frame} 

\small

{\it hd-spr-phr\/} : \begin{avm}
                     \[synsem\|cat\|spr ~ \< ~ \> \\
                       hd-dtr ~ $\avmbox{2}$ \[synsem\|cat\|spr ~ \<$\avmbox{1}$ {\it synsem\/}\>\] \\
                       dtrs ~ \< \[synsem ~ $\avmbox{1}$\]~, $\avmbox{2}$ {\it sign\/} \>\]
                     \end{avm} 

\bigskip

\bigskip

{\it hd-comp-phr\/} : \begin{avm}
                      \[synsem\|cat\|comps ~ $\avmbox{A}$ \\
                        hd-dtr ~ $\avmbox{2}$ \[synsem\|cat\|comps ~ $\avmbox{A}$ ~ $\oplus$ ~ \<$\avmbox{1}$ {\it synsem\/}\>\] \\
                        dtrs ~ \< $\avmbox{2}$ {\it sign\/}~, \[synsem ~ $\avmbox{1}$\] \>\]
                      \end{avm} 

\normalsize

\end{frame}   

\begin{frame}{Heads and modifiers}   

\begin{center}
\tree
{\ntnode{Za}{N [{\sc spr} ~ $<$ ~ $>$]},
  {\ntnode{Zd}{\avmbox{1}},
    {\tnode{Ze}{that}}},
  {\ntnode{Zf}{N [{\sc spr} ~ $<$\avmbox{1}$>$]},
    {\ntnode{Zl}{A [{\sc mod} ~ \avmbox{3}]},
      {\ntnode{Zz}{Adv},
        {\tnode{Zy}{very}}},
      {\ntnode{Zx}{Adj [{\sc mod} ~ \avmbox{3}]},
        {\tnode{Zm}{tall}}}},
    {\ntnode{Zg}{\avmbox{3} N [{\sc spr} ~ $<$\avmbox{1} Det$>$]},
      {\tnode{Zj}{sister of Leslie}}}}}
\nodeconnect{Za}{Zd}
\nodeconnect{Zd}{Ze}
\nodeconnect{Za}{Zf}
\nodeconnect{Zf}{Zl}
\nodeconnect{Zl}{Zz}
\nodeconnect{Zz}{Zy}
\nodeconnect{Zl}{Zx}
\nodeconnect{Zx}{Zm}
\nodeconnect{Zf}{Zg}
\nodetriangle{Zg}{Zj}
\end{center}

\bigskip

{\it substantive\/} : \begin{avm} \[mod(ified) ~ {\it synsem-or-none\/}\] \end{avm} 

\bigskip

The {\sc mod} value models co-occurrence constraints: 
\begin{itemize} 
\item *{\it tall that sister of Leslie\/}
\item {\it un cane pericoloso\/} `a dangerous dog' vs. *{\it una gatta pericoloso\/} `a dangerous cat' and *{\it dei cani pericoloso\/} `dangerous dogs' 
\end{itemize}

\end{frame} 

\begin{frame} 

\small

{\it head-adj-phr\/} : \begin{avm}
                       \[hd-dtr ~ $\avmbox{2}$ \[synsem ~ $\avmbox{1}$ {\it synsem\/}\] \\
                         dtrs ~ \< \[synsem\|cat\|head\|mod ~ $\avmbox{1}$\] , $\avmbox{2}$ {\it sign\/} \>\]
                       \end{avm} 

\normalsize 

\bigskip

\bigskip

The {\sc mod} value is shared between the adjective and the AP: {\it un cane molto pericoloso\/} `a very dangerous dog' 
vs. *{\it una gatta molto pericoloso\/} `a very dangerous cat'  

\bigskip

The {\sc mod} value is also used to model semantic composition. 

\end{frame}

\begin{frame}{Semantic composition}  

\begin{exe} 
\ex  $\lambda$ x {\it box\/} (x)
\end{exe} 

\bigskip

{\it scope-object\/} : \begin{avm}
                       \[index ~ {\it index\/} \\
                         restr ~ {\it set\/} ({\it fact\/}) \] 
                      \end{avm} 
\bigskip

\bigskip

\begin{avm}
\[index ~ $\avmbox{1}$                   \\
  restr ~ \{\[reln & {\it box\/}          \\
              inst & $\avmbox{1}$ \]\}\]
\end{avm} 

\end{frame}

\begin{frame} 

\begin{exe}
\ex  $\lambda$ x {\it red} (x) \& {\it box} (x)
\end{exe}

\bigskip

\begin{avm}
\[index ~ $\avmbox{1}$              \\
  restr ~ \{ \[reln & {\it red\/}   \\
               inst & $\avmbox{1}$ \] ,
             \[reln & {\it box\/}   \\
               inst & $\avmbox{1}$ \]\}\]
\end{avm}

\end{frame} 

\begin{frame} 

\small
\begin{avm}
\[cat\|head \[{\it adj}                                \\
              mod\|content \[index & $\avmbox{1}$      \\
                             restr & $\avmbox{2}$ \]\] \\
 content \[index ~ $\avmbox{1}$                         \\
           restr ~ \{ \[reln & {\it red\/}              \\
                        inst & $\avmbox{1}$ \]\} ~ $\bigcup$ ~ $\avmbox{2}$ \]\]
\end{avm}
\normalsize

\end{frame} 

\begin{frame}{Generalized quantifiers} 

\small
\begin{avm}
\[cat\|head \[{\it det\/}                            \\
              spec\|content \[index & $\avmbox{1}$  \\
                              restr & $\Sigma$ \]\] \\
  content ~ $\avmbox{2}$ \[{\it every-rel\/}         \\
                           index & $\avmbox{1}$      \\
                           restr & $\Sigma$ \]       \\
  store ~ \{ $\avmbox{2}$ \} \]
\end{avm}
\normalsize

\bigskip

{\it functional\/} : \begin{avm} \[spec ~ {\it synsem\/} \] \end{avm}

\end{frame} 

\begin{frame}{Mutual selection}  

\begin{center}
\tree
{\ntnode{Za}{N [{\sc spr} ~ $<$ ~ $>$]},
  {\ntnode{Zd}{\avmbox{1} [{\sc spec} ~ \avmbox{4}]},
    {\tnode{Ze}{that}}},
  {\ntnode{Zf}{\avmbox{4} N [{\sc spr} ~ $<$\avmbox{1}$>$]},
    {\ntnode{Zl}{A [{\sc mod} ~ \avmbox{3}]},
      {\tnode{Zm}{tall}}},
    {\ntnode{Zg}{\avmbox{3} N [{\sc spr} ~ $<$\avmbox{1} Det$>$]},
      {\tnode{Zj}{sister of Leslie}}}}}
\nodeconnect{Za}{Zd}
\nodeconnect{Zd}{Ze}
\nodeconnect{Za}{Zf}
\nodeconnect{Zf}{Zl}
\nodeconnect{Zl}{Zm}
\nodeconnect{Zf}{Zg}
\nodetriangle{Zg}{Zj}
\end{center}

\bigskip

\bigskip

The {\sc spec} value also models co-occurrence constraints: {\it der gute Wein\/} `the good wine' 
and {\it ein guter Wein\/} `a good wine' vs. *{\it der guter Wein\/} and *{\it ein gute Wein\/}

\end{frame} 

\begin{frame}{Nominals without determiner}  

\begin{center}
\tree
{\ntnode{Za}{N [{\sc spr} ~ $<$ ~ $>$ , {\sc comps} ~ $<$ ~ $>$]},
  {\ntnode{Zf}{N [{\sc spr} ~ $<$\avmbox{1}$>$ , {\sc comps} ~ $<$ ~ $>$]},
    {\ntnode{Zl}{A [{\sc mod} ~ \avmbox{3}]},
      {\tnode{Zm}{beautiful}}},
    {\ntnode{Zg}{\avmbox{3} N [{\sc spr} ~ $<$\avmbox{1}$>$ , {\sc comps} ~ $<$ ~ $>$]},
      {\ntnode{Zi}{N [{\sc spr} ~ $<$\avmbox{1}$>$ , {\sc comps} ~ $<$\avmbox{2} PP$>$]},
        {\tnode{Zj}{pictures}}},
      {\ntnode{Zk}{\avmbox{2}}, 
        {\tnode{Zh}{of Sandy}}}}}}
\nodeconnect{Za}{Zf}
\nodeconnect{Zf}{Zl}
\nodeconnect{Zl}{Zm}
\nodeconnect{Zf}{Zg}
\nodeconnect{Zg}{Zi}
\nodeconnect{Zi}{Zj}
\nodeconnect{Zg}{Zk}
\nodetriangle{Zk}{Zh}
\end{center}

\bigskip

\footnotesize
Ginzburg \& Sag (2000), {\it Interrogative Investigations}  
\normalsize

\end{frame} 

\begin{frame}{A bi-dimensional hierarchy of phrases} 

\begin{center}
\tree
    {\ntnode{Zj}{\it phrase},
      {\ntnode{Zl}{{\sc clausality}}, 
        {\tnode{Zx}{clause}},
        {\ntnode{Zy}{\it non-clause}}}, 
      {\ntnode{Zh}{{\sc headedness}}, 
        {\ntnode{Zi}{\it headed-phr},
          {\ntnode{Zp}{\it head-only-phr},
            {\tnode{Zz}{bare-nom-phr}}}, 
          {\tnode{Zo}{head-spr-phr}},
          {\tnode{Zq}{...}}}}}
\nodeconnect{Zj}{Zh}
\nodeconnect{Zh}{Zi}
\nodeconnect{Zj}{Zl}
\nodeconnect{Zl}{Zx}
\nodeconnect{Zl}{Zy}
\nodeconnect{Zi}{Zo}
\nodeconnect{Zi}{Zp}
\nodeconnect{Zi}{Zq}
\nodeconnect{Zp}{Zz}
\nodeconnect{Zy}{Zz}
\end{center}

\end{frame} 

\begin{frame}{Bare nominals}  

\small
{\it bare-nom-phr\/} : \begin{avm} 
                       \[synsem\|cat\|spr ~ \< ~ \>       \\
                         hd-dtr\|synsem\|cat \[head & {\it noun\/} \\
                                               spr  & \<Det \[{\sc wh} ~ \{ ~ \}\]\>\]\] 
                       \end{avm}
\normalsize

\bigskip

Intended to apply to plurals and singular mass nouns.   

\end{frame} 

\begin{frame} 

\begin{center} 
\Huge 
The DP Treatment 
\normalsize
\end{center} 

\end{frame} 

\begin{frame}{X-bar for functional categories} 

\begin{center}
\tree{\ntnode{Za}{D^{2}},
  {\ntnode{Zc}{D^{1}},
    {\ntnode{Zd}{D},
      {\tnode{Ze}{that}}},
    {\ntnode{Zf}{N^{2}},
      {\ntnode{Zg}{N^{1}},
        {\ntnode{Zi}{N},
          {\tnode{Zj}{sister}}},
        {\ntnode{Zk}{PP}, 
          {\tnode{Zq}{of Leslie}}}}}}}
\nodeconnect{Za}{Zc} 
\nodeconnect{Zc}{Zd}
\nodeconnect{Zd}{Ze}
\nodeconnect{Zc}{Zf} 
\nodeconnect{Zf}{Zg}
\nodeconnect{Zg}{Zi}
\nodeconnect{Zi}{Zj}
\nodeconnect{Zg}{Zk}
\nodetriangle{Zk}{Zq}
\end{center}

\bigskip

\footnotesize
Abney (1987), {\it The English noun phrase in its sentential aspects}
\normalsize

\end{frame} 

\begin{frame} 

\begin{center}
\tree
{\ntnode{Za}{D [{\sc comps} ~ $<$ ~ $>$]},
  {\ntnode{Zd}{D [{\sc comps} ~ $<$\avmbox{1} NP$>$]},
    {\tnode{Ze}{that}}},
  {\ntnode{Zf}{\avmbox{1} N [{\sc comps} ~ $<$ ~ $>$]},
    {\ntnode{Zi}{N [{\sc comps} ~ $<$\avmbox{2} PP$>$]},
      {\tnode{Zj}{sister}}},
    {\ntnode{Zk}{\avmbox{2}}, 
      {\tnode{Zh}{of Leslie}}}}}
\nodeconnect{Za}{Zd}
\nodeconnect{Zd}{Ze}
\nodeconnect{Za}{Zf}
\nodeconnect{Zf}{Zi}
\nodeconnect{Zi}{Zj}
\nodeconnect{Zf}{Zk}
\nodetriangle{Zk}{Zh}
\end{center} 

\bigskip

\footnotesize
Netter (1994), Towards a theory of functional heads: German nominal phrases

Netter (1996), {\it Functional categories in an} {\sc hpsg} {\it for German}.  
\normalsize

\end{frame} 

\begin{frame}{Functional complementation} 

{\it part-of-speech\/} : \begin{avm} 
                         \[major \[N & +/--      \\
                                   V & +/-- \]   \\
                           minor \[fcompl & +/-- \\
                                   spec   & +/-- \]\]
                         \end{avm}
\bigskip

\bigskip

Functional Complementation: In a lexical category of type {\it func-cat\/} the value of its {\sc major} 
attribute is token identical with the {\sc major} value of its complement.  

\bigskip

Consequence: the determiner is [+N , --V] and has the same case, number and gender values as its {\sc np} complement

\end{frame}

\begin{frame}{Functional completeness} 

Functional Completeness Constraint: Every maximal projection is marked as functionally complete in its {\sc minor} feature.  

\bigskip

\begin{center}
\scriptsize
\tree
{\ntnode{Za}{D [{\sc comps} ~ $<$ ~ $>$ , {\sc fcompl} ~ +]},
  {\ntnode{Zd}{D [{\sc comps} ~ $<$\avmbox{1} NP$>$ , {\sc fcompl} ~ +]},
    {\tnode{Ze}{that}}},
  {\ntnode{Zf}{\avmbox{1} N [{\sc comps} ~ $<$ ~ $>$ , {\sc fcompl} ~ --]},
    {\ntnode{Zi}{N [{\sc comps} ~ $<$\avmbox{2}$>$ , {\sc fcompl} ~ --]},
      {\tnode{Zj}{sister}}},
    {\ntnode{Zk}{\avmbox{2}}, 
      {\tnode{Zh}{of Leslie}}}}}
\nodeconnect{Za}{Zd}
\nodeconnect{Zd}{Ze}
\nodeconnect{Za}{Zf}
\nodeconnect{Zf}{Zi}
\nodeconnect{Zi}{Zj}
\nodeconnect{Zf}{Zk}
\nodetriangle{Zk}{Zh}
\normalsize
\end{center} 

\end{frame} 

\begin{frame}{Nominals without determiner} 

\tree
    {\ntnode{Zg}{N [{\sc comps} ~ $<$ ~ $>$ , {\sc fcompl} ~ + $\vee$ --]},
      {\ntnode{Zi}{N [{\sc comps} ~ $<$\avmbox{2} PP$>$ , {\sc fcompl} ~ + $\vee$ --]},
        {\tnode{Zj}{pictures}}},
      {\ntnode{Zk}{\avmbox{2}}, 
        {\tnode{Zh}{of Sandy}}}}
\nodeconnect{Zg}{Zi}
\nodeconnect{Zi}{Zj}
\nodeconnect{Zg}{Zk}
\nodetriangle{Zk}{Zh}

\bigskip

\bigskip

No empty determiner or non-branching phrase type, but unification with 
a type, called {\it dp}, in the hierarchy of {\it synsem\/} objects.  

\end{frame} 

\begin{frame}{Problems with functional complementation}  

The determiner is [+N , --V], just like its {\sc np} complement. 

\medskip

BUT: in many languages, determiners are adjectival rather than nominal. 

\begin{exe} 
\ex 
\begin{xlist} 
\ex  questa scatola rossa  ~~~ `this box.{\sc sg.fem} red' 
\ex  queste scatole rosse  ~~~ `these box.{\sc pl.fem} red'
\ex  questo libro nero ~~~~~~~ `this book.{\sc sg.mas} black' 
\ex  questi libri neri ~~~~~~~ `these book.{\sc pl.mas} black' 
\end{xlist} 
\end{exe} 

\bigskip

The German {\it jeder\/} `each', {\it dieser\/} `this' and {\it welcher\/} `which' have the same 
inflectional paradigm as the adjectives. 

\end{frame} 

\begin{frame}{Problems with functional complementation}  

The determiner has the same case, number and gender values as its {\sc np} complement.

\bigskip

BUT: many languages have determiners that do not show agreement in case, number and gender 
with the nominal. 

\begin{exe} 
\ex 
\begin{xlist} 
\ex  wiens huis ?  ~~~~~ `who.{\sc gen.sg} house'   
\ex  wiens huizen ?  ~~~ `who.{\sc gen.sg} house.{\sc pl}' 
\end{xlist} 
\ex
\begin{xlist} 
\ex  
\gll   's lands glorie \\
       the.{\sc gen.sg} country.{\sc gen.sg} glory \\
\trans `the country's glory' 
\ex 
\gll   's werelds hoogste bergen  \\
       the.{\sc gen.sg} world.{\sc gen.sg} highest mountain.{\sc pl} \\ 
\trans `the world's highest mountains' 
\end{xlist} 
\end{exe} 

\end{frame} 

\begin{frame} 

\begin{center}
\Huge 
NP completeness
\normalsize
\end{center}

\end{frame} 

\begin{frame}

``we assume that all substantive categories will require the complement 
they combine with to be both saturated and functionally complete.'' 
Netter (1994), p. 311.     

\bigskip

\bigskip

German prepositions with incorporated dative article: {\it am, im, vom, beim, zum\/} and {\it zur}

\begin{exe} 
\ex
\begin{xlist} 
\ex  in dem Zimmer ~~~~~ `in the room' 
\ex  im Zimmer ~~~~~~~~ `in.the room'
\ex  * im dem Zimmer ~~~ `in.the the room'
\end{xlist} 
\end{exe} 

\end{frame} 

\begin{frame}
 
Italian prepositions with incorporated definite article: {\it al, dal, sul, del\/} and {\it nel\/}, each with 
feminine and plural counterparts ({\it allo, alla, agli, alle ...})

\begin{exe} 
\ex
\begin{xlist} 
\ex   in questa stanza  ~~~~~ `in this room' 
\ex   nella stanza ~~~~~~~~~~ `in.the room'
\ex   * nella questa stanza ~~~ `in.the this room'
\end{xlist} 
\end{exe} 

\bigskip

French prepositions with incorporated definite article: {\it du, des\/} and {\it au, aux\/} 

\bigskip

Dutch prepositions {\it te\/} and {\it per}

\begin{exe} 
\ex
\begin{xlist} 
\ex  te (*het/een) paard ~~~  `on horse' 
\ex  per (*de/een) trein ~~~  `by train'
\end{xlist} 
\end{exe}

\end{frame} 

\begin{frame} 

Dutch prepositions {\it met\/} `with' and {\it zonder\/} `without' 

\begin{exe} 
\ex
\begin{xlist} 
\ex  een huis met (een) garage ~~~  `a house with (a) garage'
\ex  huizen zonder (een) dak ~~~ `houses without (a) roof' 
\end{xlist} 
\end{exe} 

\bigskip

Dutch and German copular verbs

\bigskip

\begin{exe} 
\ex
\begin{xlist} 
\ex  Henk wordt leraar  ~~~ `Henk becomes a teacher' 
\ex  Hans ist Lehrer  ~~~~~~  `Hans is a teacher' 
\end{xlist} 
\end{exe} 

\end{frame} 

\begin{frame} 

There is no context-independent norm for determining whether a nominal is fit for use in context. 
Instead, different external selectors require or allow different degrees 
of saturation. 

\bigskip

In this respect, the combination of a determiner with a nominal 
is comparable to the combination of a complementizer with a clause. 

\begin{exe} 
\ex
\begin{xlist} 
\ex  They claim (that) everything is ok.  
\ex  Do not believe the claim that everything is ok. 
\ex  That everything is ok surprises me.   
\end{xlist} 
\end{exe} 

\end{frame} 

\begin{frame}{Marking} 

\begin{center}
\tree
    {\ntnode{Zg}{S [{\sc marking} ~ \avmbox{1}]},
      {\ntnode{Zi}{C [{\sc marking} ~ \avmbox{1} {\it marked\/} , {\sc spec} ~ \avmbox{2}]},
        {\tnode{Zj}{that}}},
      {\ntnode{Zk}{\avmbox{2} S [{\sc marking} ~ {\it unmarked\/}]}, 
        {\tnode{Zh}{John left}}}}
\nodeconnect{Zg}{Zi}
\nodeconnect{Zi}{Zj}
\nodeconnect{Zg}{Zk}
\nodetriangle{Zk}{Zh}
\end{center}

\bigskip

{\it hd-mark-phr\/} : \begin{avm}
                      \[synsem\|cat\|marking ~ $\avmbox{1}$ {\it marking\/}  \\
                        hd-dtr ~ $\avmbox{2}$ {\it sign\/}  \\
                        dtrs ~ \< \[synsem\|cat\|marking ~ $\avmbox{1}$\]~, $\avmbox{2}$\>\]
                      \end{avm} 

\bigskip

\footnotesize
Pollard \& Sag (1994), {\it Head-driven Phrase Structure Grammar}. 
\normalsize

\end{frame} 

\begin{frame} 

\begin{center} 
\Huge 
The Functor Treatment
\normalsize
\end{center} 

\end{frame} 

\begin{frame} 

The degree of saturation is not registered by valence features, nor by the boolean 
{\sc fcompl} feature, but by the {\sc marking} feature.  

\bigskip

Elimination of the distinction between substantive and functional categories. 

\bigskip 

Determiners do not belong to a separate part of speech, but are --categorially speaking--
a heterogeneous lot. The Dutch ones include adjectives, pronouns and common nouns. 

\end{frame}

\end{frame}

\begin{frame}{The functor treatment}  

{\it hd-func-phr} : \begin{avm}
                    \[synsem\|cat\|marking ~ $\avmbox{2}$ {\it marking\/}            \\
                      hd-dtr ~ $\avmbox{3}$ \[synsem ~ $\avmbox{1}$ {\it synsem\/}\] \\
                      dtrs ~ \< \[synsem\|cat \[head\|select ~ $\avmbox{1}$          \\
                                                marking ~ $\avmbox{2}$ \]\] , $\avmbox{3}$ \>\]
                     \end{avm}

\bigskip

{\sc select} replaces both {\sc mod} and {\sc spec} 

\bigskip

No {\sc spr} feature

\bigskip

\footnotesize
Van Eynde (2006), NP-internal agreement and the structure of the noun phrase

Allegranza (2007), {\it The signs of determination. Constraint-based modelling across languages}.  
\normalsize

\end{frame} 

\begin{frame}{An example} 

\begin{center}
\tree
{\ntnode{Zz}{N [{\sc mark} ~ \avmbox{2} {\it marked\/}]},
  {\ntnode{Za}{N [{\sc mark} ~ \avmbox{2} , {\sc sel} ~ \avmbox{4}]},
    {\tnode{Zp}{that}}},
  {\ntnode{Zb}{\avmbox{4} N [{\sc mark} ~ \avmbox{1} {\em unmarked\/}]},
    {\ntnode{Zc}{A [{\sc mark} ~ \avmbox{1} , {\sc sel} ~ \avmbox{3}]},
      {\tnode{Zq}{long}}},  
    {\ntnode{Zd}{\avmbox{3} N [{\sc mark} ~ \avmbox{1}]},
      {\tnode{Zh}{bridge}}}}}
\nodeconnect{Zz}{Za}
\nodeconnect{Zz}{Zb}
\nodeconnect{Za}{Zp}
\nodeconnect{Zb}{Zc}
\nodeconnect{Zb}{Zd}
\nodeconnect{Zc}{Zq}
\nodeconnect{Zd}{Zh}
\end{center}

\end{frame} 

\begin{frame}{Phrasal determiners} 

\begin{center} 
\tree
  {\ntnode{Za}{N [{\sc mark} ~ \avmbox{1}]}, 
    {\ntnode{Zd}{N [{\sc mark} ~ \avmbox{1}]},
      {\ntnode{Zf}{[{\sc mark} ~ \avmbox{1} {\it marked\/}]},
        {\tnode{Zl}{'s}}},
      {\ntnode{Zg}{N [{\sc mark} ~ {\it unmarked\/}]},
        {\tnode{Zi}{lands}}}},
    {\ntnode{Zk}{N [{\sc mark} ~ {\it unmarked\/}]},
      {\tnode{Zh}{glorie}}}}
\nodeconnect{Za}{Zd}
\nodeconnect{Za}{Zk}
\nodeconnect{Zd}{Zf}
\nodeconnect{Zd}{Zg}
\nodeconnect{Zf}{Zl}
\nodeconnect{Zg}{Zi}
\nodeconnect{Zk}{Zh}
\normalsize
\end{center}

\end{frame} 

\begin{frame}{Nominals without determiner} 

\tree
    {\ntnode{Zg}{N [{\sc comps} ~ $<$ ~ $>$ , {\sc mark} ~ \avmbox{1} {\em unmarked\/}]},
      {\ntnode{Zi}{N [{\sc comps} ~ $<$\avmbox{2} PP$>$ , {\sc mark} ~ \avmbox{1} ]},
        {\tnode{Zj}{pictures}}},
      {\ntnode{Zk}{\avmbox{2}}, 
        {\tnode{Zh}{of Sandy}}}}
\nodeconnect{Zg}{Zi}
\nodeconnect{Zi}{Zj}
\nodeconnect{Zg}{Zk}
\nodetriangle{Zk}{Zh}

\bigskip
No need for an empty determiner or a non-branching phrase type. Also if the nominal is singular and count.  

\end{frame} 

\begin{frame}{Bare vs. incomplete} 

\begin{exe} 
\ex
\begin{xlist} 
\ex  warm brood ~~~~~~ `warm bread.{\sc sg.neu}'
\ex  een warm brood ~~~~~  `a warm bread.{\sc sg.neu}'
\ex  welk warm brood ~~~  `which warm bread.{\sc sg.neu}'
\ex  het warme brood ~~~~~ `the warm.{\sc dcl} bread.{\sc sg.neu}'
\ex  dat warme brood ~~~~~ `that warm.{\sc dcl} bread.{\sc sg.neu}'
\end{xlist} 
\end{exe}  

\bigskip

\noindent
*{\it warme brood\/} `warm.{\sc dcl} bread' is inherently incomplete, in contrast to {\it warm brood\/} 
which is a bare singular. 

\bigskip

\begin{center}
\tree
{\ntnode{Zz}{\it marking},
  {\ntnode{Ze}{\it unmarked},  
    {\tnode{Zf}{incomplete}},
    {\tnode{Zg}{bare}}},
  {\tnode{Zh}{\it marked}}}
\nodeconnect{Zz}{Ze}
\nodeconnect{Ze}{Zf}
\nodeconnect{Ze}{Zg}
\nodeconnect{Zz}{Zh}
\end{center}   

\end{frame}

\begin{frame} 

\begin{center} 
\Huge 
The Nominalized Infinitive
\normalsize
\end{center}

\end{frame} 

\begin{frame}{{\sc np} distribution} 

\end{frame} 

\begin{frame}{Internal structure: Nominal}  

The infinitive shares properties with singular neuter nouns

\begin{exe} 
\ex  
\gll   het trage afsterven van de koraalriffen             \\
       the slow.{\sc dcl} die.{\sc inf} of the coral.reefs \\
\trans `the slow dying of the coral reefs'
\ex 
\gll   de gevolgen van [mijn handelen] op langere termijn    \\
       the consequences of [my act.{\sc inf}] on longer term \\
\trans `the long term consequences of my acting'  
\end{exe} 

\end{frame}

\begin{frame}{Internal structure: Verbal}  

\begin{exe} 
\small
\ex 
\gll   het {\it goed investeren\/} van een patrimonium  \\
       the {good invest.{\sc inf}} of a patrimonium     \\
\trans `investing well of a patrimonium' 
\ex 
\gll   het {\it toegankelijk maken\/} van onderwijsvoorzieningen voor volwassenen  \\
       the {accessible make.{\sc inf}} of education.provisions for adults          \\
\trans `making educational facilities for adults accessible' 
\ex 
\gll   het {\it op diepte houden\/} van de Vlaamse kusthavens      \\
       the {up depth hold.{\sc inf}} of the Flemish coast.harbours \\
\trans `the up depth hold of the Flemish coast.harbours'
\end{exe} 

\end{frame} 

\begin{frame}

\begin{exe} 
\ex 
\gll   het {\it niet kunnen toeschrijven\/} van huisartsen aan de praktijk waarin ... \\
       the {not can assign.{\sc inf}} of house.medics to the service where.in ...     \\
\trans `the impossibility to assign general practitioners to the service in which ...  ' 
\normalsize 
\end{exe}

\end{frame} 

\begin{frame}{Phrasal Coherence}      


\end{frame} 


\begin{frame}

\begin{center} 
\Huge
Conclusion
\normalsize
\end{center} 

\end{frame} 

\begin{frame}

\begin{itemize} 
\item  There is no context-independent norm for determining whether a nominal is fit for use in context. 
Instead, different external selectors require or allow different degrees of saturation. 

\item  Determiners do not belong to a separate part of speech, but are --categorially speaking--
a heterogeneous lot: some are adjectives, others pronouns or sth. else.  

\item  The {\sc hpsg} style functor treatment models this in a straightforward manner.

\item  The treatment is extensible to nominals with idiosyncratic properties.    
\end{itemize}

\end{frame} 

\begin{frame}{Treatments of other idiosyncratic nominals}  

\begin{itemize} 
\item  Verbal and nominal gerunds (Malouf 2000) 
\item  Loose and close apposition (Kim 2012, Kim 2014, Van Eynde \& Kim 2016)
\item  Binominal {\sc np} Construction (Kim \& Sells 2014, Van Eynde 2018)
\item  Big Mess Construction (Van Eynde 2007, Kim \& Sells 2011, Kay \& Sag 2012, Arnold \& Sadler 2014, Van Eynde 2018) 
\end{itemize} 

\end{frame} 

\begin{frame}

\begin{center} 
\Huge
Thank you!
\normalsize
\end{center} 

\end{frame} 

\end{document} 


\begin{frame}

\begin{exe}
\small
\ex 
\gll   voor [[het {\it modelleren\/} en het {\it simuleren\/}] van waterkwaliteit] \\ 
       for [[the model.{\sc inf} and the simulate.{\sc inf}] of water.quality] \\ 
\trans `for modelling and simulating the quality of the water'
\normalsize
\end{exe}  

\end{frame} 



  

Of the 660 instances in the {\sc dpc} treebank\footnote{{\sc dpc} is short for 
``Dutch Parallel Corpus'', one of the five subparts of LASSY Small
\cite{VanNoord13}. The {\sc dpc} treebank consists of 11,716 syntactically annotated 
sentences, comprising a total of 193,029 words. It contains 38,294 {\sc np}s. 
660 of those (1.72 \%) have an infinitival head. All examples  
are taken from that set. The nominalized infinitive is bracketed, if there are also other 
constituents. The verbal core is italicized.}  
88 (13.33 \%) are subjects, as in (\ref{conf}), and 328 (49.70 \%) are objects of a verb or 
preposition, as in (\ref{bj}). 159 (24.09 \%) are conjuncts, often in combination with 
an ordinary {\sc np}, as in (\ref{mob}).\footnote{The remaining 85 hits concern uses as 
predicative complement and modifier.} 

\noindent
The internal structure shows a mixture of nominal and verbal characteristics. 
The {\bf verbal characteristics} include, aside from the verbal morphology:   

\begin{itemize} 
\item the formation of a cluster if the infinitival head is a clustering verb, 
such as the modal in {\it kunnen toeschrijven\/} `can attribute' in 
(\ref{niet});\footnote{Other such examples from the {\sc dpc} treebank are 
{\it laten sterven\/} `let die' and {\it leren kennen\/} `learn (to) know'.}
\item the realization of complements of the verb to the left of the
infinitive, such as the noun in (\ref{conf}--\ref{bj}), the predicative 
adjective in (\ref{toeg}) and the {\sc pp} in (\ref{prec});
the position of the {\sc pp} complement is noteworthy, since {\sc pp}s invariably
follow the noun in a nominal; 
\item the presence of adverbs which do not combine with nominals, such as 
{\it opnieuw\/} `again', {\it terug\/} `again' and {\it niet\/} `not', as in (\ref{niet}); 
\item the presence of non-declined adjectives in positions where 
declension would be required in a nominal, such as {\it goed\/} in (\ref{plo}); in 
{\it het goede doel\/} `the good.{\sc dcl} purpose' the adjective must be declined.  
\end{itemize} 


\noindent
The {\bf nominal characteristics} include: 

\begin{itemize} 
\item the presence of a determiner, usually the neuter definite article {\it het\/},  
as in (\ref{niet}--\ref{plo}); 
\item the presence of a declined adjective, as in (\ref{trage}); 
\item the realization of {\sc np} and {\sc pp} complements of the verb as 
{\sc pp}s which follow the infinitive; these can be direct objects, such 
as the {\sc pp}[{\it van\/}] in (\ref{niet}--\ref{plo}), or
indirect objects, such as the {\sc pp}[{\it aan\/}] in (\ref{bj}) and (\ref{niet}). 
\end{itemize} 

\noindent
The subject of the verb is often unexpressed. If realized, it takes the 
form of a postnominal {\sc pp}, introduced by {\it door\/} `by' or 
{\it van\/} `of', as in (\ref{trage}), or of a 
possessive pronoun/genitive proper noun, as in (\ref{hand}). 


The nominalized infinitive conforms to the {\bf Phrasal Coherence Hypothesis} \cite[4]{Bresnan97}. 
It has a verbal core and the shift to a nominal projection is irreversible. 
The verbal core minimally consists of an infinitive ---possibly an infinitival cluster, 
as in (\ref{niet})--- and may be extended with complements and modifiers 
that precede the infinitive, as in (\ref{conf}--\ref{bj}) and (\ref{toeg}--\ref{plo}), 
in conformity with the general constraint that non-finite {\sc vp}s are head final in Dutch,
modulo clustering verbs.
When converted into a nominal, the infinitive may be combined with determiners 
and declined adjectives to the left and other dependents to the right, usually {\sc pp}s. 


\section{Phrasal conversion } 


This section provides a novel analysis of the nominalized infinitive. It is based 
on phrasal conversion. The starting pont is the assumption that  
the nominalized infinitive has the same properties as the 
ordinary infinitive at the lexical level. 
It is combined with its complements and/or modifiers, if any, in the usual way. 

The conversion to nominal status is modelled by the phrasal constraint in Figure \ref{pc}. 
It concerns a non-headed phrase type, since the value of the {\sc head} 
feature is not shared between mother and daughter.  
The daughter is an infinitive ---lexical or phrasal--- with a non-empty 
{\sc subj} list and a possibly empty {\sc comps} list. 
The mother is a nominal, which implies that it has a {\sc case} feature 
and morpho-syntactic {\sc number} and {\sc gender} features. 
The former's value is underspecified; the latter are declared to be 
resp. {\it singular\/} and {\it neuter}. Its {\sc subj} list is empty, as 
usual for nominals. The {\sc subj} requirement of the daughter is 
added to the mother's {\sc comps} list, keeping the index, but changing its category 
from {\sc np} to {\sc pp} and making it optional. 
The unsaturated {\sc comps} requirements of the daughter, if any, 
remain on the {\sc comps} list of the nominal; if nominal, 
they are required to contain a preposition.  
The {\sc marking} value of the nominal is {\it bare}, a subtype of {\it unmarked}. 
The morpho-syntactic conversion is accompanied by a semantic shift. 
In terms of the inventory of {\sc content} values in \citet{GS00}, 
this is a shift from a state-of-affairs to a parameter whose index is 
identified with the situation that restricts the denotation of the infinitive.

The resulting nominal combines with its adjectival modifier(s) and determiner, if any,  
in the usual way. If combined with a declined adjective, as 
in (\ref{trage}), the {\sc marking} value is changed to {\it incomplete}, another 
subtype of {\it unmarked}. To yield a well-formed nominal it must then be combined with 
a definite determiner, as modeled in \citet{VanEynde06}. 
The addition of a determiner yields a marked {\sc np}. Postnominal 
dependents can be added at will. (\ref{ware}) suggests that the postnominal 
dependents are added after the prenominal ones.


\noindent
The resulting bare or marked nominal can then be used in positions where {\sc np}s 
commonly occur. 


\section{Comparison with the mixed category treatment}   


Just like the nominalized infinitive, the English gerund shows a mixture of nominal and 
verbal characteristics. To model it Malouf introduces a separate part-of-speech.
As shown in Figure \ref{mc}, it is a subtype of {\it noun}, on the one hand, and of 
{\it relational}, on the other hand \cite[65]{Malouf00}. 
In this analysis the gerund belongs to a mixed category and shares this mixed status 
with its phrasal projection (GerP). Phrasal coherence is enforced by default 
constraint inheritance in a four-dimensional hierarchy of words 
\cite[128--140]{Malouf00}. 

This differs from our analysis, in which the nominalized infinitive is unambiguously verbal 
before conversion and unambiguously nominal after conversion. The advantages of 
the phrasal conversion treatment are that (1) there is no need for a new part-of-speech, 
(2) there are no special devices needed to enforce phrasal coherence, (3) there is no need 
for any changes to the rest of the grammar: the infinitive 
combines with its verbal dependents in the same way as ordinary infinitives, and it combines 
with its nominal dependents in the same way as common nouns.   

\footnotesize

\bibliography{prenom}
\bibliographystyle{clin}  


\begin{figure} 
\small
\begin{center} 
\tree
{\ntnode{Zz}{\it head},
  {\ntnode{Ze}{\it noun},  
    {\tnode{Zf}{proper-noun}},
    {\tnode{Zg}{common-noun}}, 
    {\tnode{Zw}{gerund}}},
  {\ntnode{Zh}{\it relational},
    {\tnode{Zj}{verb}},      
    {\tnode{Zk}{adjective}}}}
\nodeconnect{Zz}{Ze}
\nodeconnect{Ze}{Zf}
\nodeconnect{Ze}{Zg}
\nodeconnect{Ze}{Zw}
\nodeconnect{Zz}{Zh}
\nodeconnect{Zh}{Zj}
\nodeconnect{Zh}{Zk}
\nodeconnect{Zh}{Zw}
\caption{Mixed category -- Malouf (2000), page 65}\label{mc} 
\end{center} 
\end{figure} 

\end{document} 


\bigskip

\begin{figure} 
\footnotesize
\begin{center} 
\begin{avm} 
[head [{\it verb}           \\
       vform ~ {\it prp\/}] \\
 valence [subj  & <@1 np>               \\
          comps & @2                    \\
          spr   & < ~ > ]]
\end{avm} ~ $\Rightarrow$ ~ \begin{avm} 
                            [head ~ {\it gerund\/} \\
                             valence [subj  & <@1 np>        \\
                                      comps & @2             \\
                                      spr   & <@1 np> ]] 
                            \end{avm}
\caption{Lexical rule (Malouf, 2000)}\label{lr} 
\end{center}
\end{figure} 



\cite{ANS97}. 

\begin{exe} 
\ex\label{nom}  //node[@cat=``np'' and node[@rel=``hd'' and @wvorm=``inf'']]  
\ex\label{obj}  //node[@cat=``np'' and @rel=``obj1'' and node[@rel=``hd'' and @wvorm=``inf'']]  
\ex\label{det}  //node[@cat=``np'' and node[@rel=``det''] and node[@rel=``hd'' and @wvorm=``inf'']]  
\end{exe} 
 
(\ref{nom}) retrieves the nominalized infinitives. 
(\ref{obj}) retrieves the nominalized infinitives in object position. 
(\ref{det}) retrieves the nominalized infinitives with a determiner.  



\subsection{A hierarchy of non-clausal phrases} 


While the {\sc headedness} types are defined in cross-categorial 
terms and mainly concern syntactic properties, the 
{\sc clausality} types concern category-specific and semantic 
properties. Taking a cue from the treatment of 
clausal phrase types in \citet{GS00}, we develop a partial 
hierarchy for the nominal phrases. As a starting point we take  
the hierarchy of semantic objects in (\ref{sem}), 
an abbreviated version of \citet[386]{GS00}. 

\begin{exe}
\ex\label{sem}
\scriptsize
\tree
{\ntnode{Zt}{\it semantic-object}, 
  {\ntnode{Zd}{\it message}, 
    {\tnode{Za}{proposition}},
    {\tnode{Zg}{fact}},
    {\tnode{Zh}{...}}},
  {\tnode{Ze}{state-of-affairs}},
  {\ntnode{Zr}{\it scope-object},
    {\tnode{Zb}{parameter}},
    {\tnode{Zc}{quant-rel}}},
  {\tnode{Zf}{relation}},
  {\tnode{Zp}{index}}}
\nodeconnect{Zt}{Zd}
\nodeconnect{Zd}{Za}
\nodeconnect{Zd}{Zg}
\nodeconnect{Zd}{Zh}
\nodeconnect{Zt}{Ze}
\nodeconnect{Zt}{Zr}
\nodeconnect{Zr}{Zb}
\nodeconnect{Zr}{Zc}
\nodeconnect{Zt}{Zf}
\nodeconnect{Zt}{Zp}
\normalsize
\end{exe}

\noindent
Some of these types have already been mentioned in section 2, 
such as {\it message, proposition, state-of-affairs\/} and 
{\it scope-object}. For our purpose, it is mainly the latter 
that matters, since it is the prototypical {\sc content} value 
of a nominal. It consists of an index and a set of 
facts which impose restrictions on the index \citep[122]{GS00}.  

\begin{exe}
\ex\label{sco} {\it scope-object\/} : \begin{avm} 
                           [index ~ {\it index\/}           \\
                            restriction ~ {\it set\/} \({\it fact\/}\)]    
                           \end{avm}
\end{exe}

\noindent
The common noun {\it box}, for instance, denotes the set of those 
entities for which it is a fact that they are a box. The type is also 
relevant for the adjectives. {\it Red}, for instance, denotes the set 
of those entities for which it is a fact that they are red.

At a finer-grained level \citet[135--136]{GS00} differentiates 
between quantified and non-quantified scope-objects. 
The former are of type {\it quant-rel}. They include {\sc np}s which are 
introduced by a quantifying determiner, such as {\it every, each\/} or
{\it some}, and to quantifying pronouns, such as {\it everybody\/} 
and {\it someone}. The nominals which lack quantificational force are 
of type {\it parameter}. They include bare nominals, proper nouns, 
various types of pronouns and {\sc np}s which are introduced by a 
non-quantifying determiner. It is not always immediately clear whether 
a nominal is quantified or not. \citet{GS00}, for instance, 
devotes a whole chapter to the issue of whether interrogative 
{\it wh\/}-phrases, such as {\it who, what\/} and {\it which table},  
are quantified, concluding that they are not (chapter 4). 
For our purpose, it is the nominal phrases which denote a parameter that 
matter. They are subsumed by a subtype of the non-clausal phrases, which we 
call {\it nominal-parameter}. Its properties are spelled out in (\ref{param}).   

\subsection{Multiple inheritance} 




\section{The Big Mess Construction} 


Some examples of the {\sc bmc} are given in (\ref{bime}), repeated in (\ref{bime5}).

\begin{exe}
\ex\label{bime5}
\begin{xlist}
\ex   It's {\it so good a bargain\/} I can't resist buying it.
\ex   {\it How serious a problem\/} is this?
\end{xlist}
\end{exe} 

\noindent
The italicized phrases are {\sc np}s.

Taking stock, the {\sc bmc} is an {\sc np} that consists of 
a marked {\sc ap} and a marked {\sc np}, and the internal structure 
of these two constituents is unexceptional: They  
are both instances of the head-functor type of combination. 
What makes the {\sc bmc} exceptional is the   
combination of the pre-determiner {\sc ap} and the indefinite {\sc np}.  
It will be discussed and analysed in section 4.1. 
The variant of the {\sc bmc} with {\it of\/} is treated in section 4.2. 
A comparison with an alternative treatment is provided in section 4.3 
and a summary in section 4.4.   


\subsection{The canonical BMC}  


To model the combination of the {\sc ap} and the indefinite {\sc np}  
we add a type to the hierarchy of phrases, called {\it big-mess-phrase}, 
which is a subtype of {\it head-independent-phrase}, on the one hand, and  
{\it restrictive-modification}, on the other hand.


\subsubsection{Inherited properties} 


Being a subtype of {\it head-nonargument-phrase}, the {\sc bmc} has a 
head daughter with which it shares its {\sc head} value ($\avmbox{1}$) 
and a non-head daughter with which it shares its {\sc marking} value 
($\avmbox{2}$). These are in turn shared with the head daughter 
of the {\sc np} and the functor daughter of the {\sc ap}, respectively.  

\begin{exe}
\ex\label{ver}
\footnotesize
\tree
  {\ntnode{Za}{[{\sc head} ~ $\avmbox{1}$ {\it noun\/} , {\sc marking} ~ $\avmbox{2}$ {\it marked\/}]},
    {\ntnode{Zd}{[{\sc head} ~ $\avmbox{3}$ {\it adj\/} , {\sc marking} ~ $\avmbox{2}$]},
      {\ntnode{Ze}{[{\sc marking} ~ $\avmbox{2}$]},    
        {\tnode{Zl}{how}}},
      {\ntnode{Zg}{[{\sc head} ~ $\avmbox{3}$]},
        {\tnode{Zi}{serious}}}},
    {\ntnode{Zk}{[{\sc head} ~ $\avmbox{1}$ , {\sc marking} ~ $\avmbox{4}$ {\it marked\/}]},
      {\ntnode{Zh}{[{\sc marking} ~ $\avmbox{4}$]},    
        {\tnode{Zm}{a}}},
      {\ntnode{Zn}{[{\sc head} ~ $\avmbox{1}$]},
        {\tnode{Zo}{problem}}}}}
\nodeconnect{Za}{Zd}
\nodeconnect{Zd}{Ze}
\nodeconnect{Ze}{Zl}
\nodeconnect{Zd}{Zg}
\nodeconnect{Zg}{Zi}
\nodeconnect{Za}{Zk}
\nodeconnect{Zk}{Zh}
\nodeconnect{Zh}{Zm}
\nodeconnect{Zk}{Zn}
\nodeconnect{Zn}{Zo}
\normalsize
\end{exe}


\bigskip
 
Being a subtype of {\it restrictive-modification}, 
the {\sc bmc} denotes a parameter in which the daughters 
share their index. In this respect, it resembles regular nominals, 
such as {\it red box}. Evidence for treating the {\sc bmc} along 
the same lines is provided by the fact that some of 
the {\sc ap}s with a degree marker are both used 
in the idiosyncratic pre-determiner position and in the canonical
post-determiner position. This is illustrated for {\sc ap}s with 
{\it more, less\/} and {\it enough\/} in 
(\ref{serious}--\ref{enou}).

\noindent
As a consequence, since the instances with the post-determiner order are 
an instance of restrictive modification, and since there 
is no difference in truth-conditional meaning with the instances of 
the pre-determiner order, it is only natural to treat the latter as instances 
of restrictive modification as well. 
Notice, incidentally, that this provides further 
evidence for the {\sc drt} style treatment of the indefinite article as lacking
quantificational force, for if the article were treated as an existential 
quantifier, the {\sc bmc} would not comply with the constraints on 
{\it restrictive-modification}, which in turn would necessitate an
otherwise unmotivated differentiation between the (a) and (b) examples in 
(\ref{serious}--\ref{enou}). 


\subsubsection{Inherent properties}


Beside the properties which it inherits from its supertypes, the {\sc bmc}
has some properties of its own. They are spelled out in (\ref{bigmess2}).  

\subsection{The variant with {\it of\/}} 


Some examples of the variant with {\it of\/} are given in (\ref{of77}).

\begin{exe}
\ex\label{of77}
\begin{xlist}
\ex   He took {\it so big of a piece\/} that he couldn't finish it.
\ex   It was a judgment question as to {\it how big of a risk\/} it was.   
\end{xlist}
\end{exe}

\noindent
This construction is mainly used in American English, as pointed out 
in \citet[136]{Bolinger72}, \citet[113--114]{Zwicky95}, 
\citet[125--126]{Kennedy00}, and \citet[348]{Fillmoreetal12}. 
It provides an extra challenge, for if we adopt the canonical analysis 
of a [{\sc p}--{\sc np}] sequence, we get a {\sc pp} that is combined 
with an {\sc ap} to yield an {\sc np}! An analysis along these lines 
is actually proposed in \citet[117]{Zwicky95} and 
\citet[357]{KimSells11}, but it is not the one which we adopt. 

Instead, we assume that the [{\it of\/}--{\sc np}] sequence is headed 
by the {\sc np}. The preposition is, hence, not treated as the complement 
selecting head of a {\sc pp} but as a head selecting functor, more 
specifically as a functor which selects an {\sc np} with the {\sc dtype} value 
{\it a}. Its own {\sc dtype} value captures the fact that we 
are dealing with a use of {\it of\/} in which it is necessarily 
followed by the indefinite article.
We call it {\it ofa\/} for mnemonic reasons. It is a subtype of 
{\it indefinite}, just like {\it a}.
The {\sc synsem} value in (\ref{of4}) spells this out in formal terms.

\begin{exe}
\ex\label{of4}
\footnotesize
\begin{avm} 
[category [head [{\it preposition\/}                \\
                 select|category|marking|dtype ~ {\it a\/}] \\
           marking|dtype ~ {\it ofa\/}]                \\
 content|restriction ~~ \{  \}]
\end{avm}
\normalsize
\end{exe}

\noindent
The assignment of the empty set as its {\sc restr} value
captures the semantic vacuity of the preposition. 

As a consequence, the combination of the preposition with the {\sc np} 
can be treated as an instance of the {\it regular-nominal\/} type: 
In terms of the {\sc headedness} dimension it is an instance of
{\it head-functor-phrase\/} and in terms of the {\sc clausality} dimension
it is an instance of {\it restrictive-modification}, albeit with 
the proviso that the preposition does not contribute any restrictions. 

The resulting nominal matches nearly all of the constraints on the head 
daughter of the big-mess phrase, as spelled out in (\ref{bigmess2}). 
The only modification we need is to make the {\sc dtype} value 
of the head daughter in (\ref{bigmess2}) disjunctive, 
allowing both {\it a\/} and {\it ofa}. 
This suffices to license combinations as in (\ref{bigof}). 

\begin{exe}
\ex\label{bigof}
\footnotesize
\tree
    {\ntnode{Zh}{[{\sc hd} ~ $\avmbox{3}$ {\it noun} , {\sc marking} ~ $\avmbox{2}$]},
      {\ntnode{Zm}{[{\sc marking} ~ $\avmbox{2}$ {\it marked\/}]},
        {\tnode{Zr}{how big}}},  
      {\ntnode{Zg}{[{\sc hd} ~ $\avmbox{3}$ , {\sc dtype} ~ $\avmbox{1}$ {\it ofa\/}]},
        {\ntnode{Zx}{[{\sc hd$|$sel} ~ $\avmbox{4}$ , {\sc dtype} ~ $\avmbox{1}$]},
          {\tnode{Zy}{of}}},
        {\ntnode{Zz}{$\avmbox{4}$ [{\sc hd} ~ $\avmbox{3}$ , {\sc dtype} ~ {\it a\/}]},
          {\tnode{Zi}{a risk}}}}}
\nodeconnect{Zh}{Zm}
\nodetriangle{Zm}{Zr}
\nodeconnect{Zh}{Zg}
\nodeconnect{Zg}{Zx}
\nodeconnect{Zx}{Zy}
\nodeconnect{Zg}{Zz}
\nodetriangle{Zz}{Zi}
\normalsize
\end{exe}

Independent evidence for the functor treatment is provided by the 
stranding test. In languages which allow adposition stranding, such 
as English and Dutch, adpositions may be stranded if they head 
a {\sc pp}, as in (\ref{afr1}). 

\begin{exe}
\ex\label{afr1}  
\begin{xlist} 
\ex  Everybody wants a copy of that book. 
\ex  That is a book that everybody wants a copy of \_\_.    
\end{xlist} 
\end{exe}  

\noindent
This is just a regular instance of complement extraction. 
The preposition in the {\sc bmc}, by contrast, cannot be stranded.

\begin{exe} 
\ex\label{afr2}
\begin{xlist} 
\ex    I never saw so disgusting of a movie. 
\ex    [*] {That is a movie that I never saw so disgusting of \_\_.}
\end{xlist} 
\end{exe}

\noindent
This meshes well with its functor status, since heads cannot be 
freely extracted. 

Further evidence is provided by the fact that the functor treatment 
also applies to the use of {\it of\/} in the Binominal {\sc np} Construction.  
Moreover, it has been argued to apply to certain uses of the 
Dutch prepositions {\it om, te, van\/} and {\it voor\/} in 
\citet{VanEynde04}. Also for certain uses of the French 
prepositions {\it \`a\/} and {\it de\/} it has been argued that 
they are markers, rather than heads of {\sc pp}s
\citep{AbeilleGodard2000}.\footnote{Markers share many properties of 
functors. The main difference is that markers are required to belong 
to a functional closed class category, whereas functors can belong to 
any part of speech.}  


\subsection{Comparison with another treatment} 


Another monostratal analysis of the {\sc bmc} is provided in \citet{KaySag12}. 
It is cast in the notation of Sign-Based Construction Grammar, which is very
similar to that of constructivist {\sc hpsg}. 
It employs the head-functor type of combination to model both the lower and the 
higher {\sc np} of the {\sc bmc}, and resorts to a special 
type of construction to model the pre-determiner {\sc ap}. 
It is called the {\it complex pre-determiner construction\/} and 
its main idiosyncracy is that it     
triggers a change in the {\sc select} value of the adjective:
While an adjective such as {\it long\/} selects an unmarked nominal, 
the combination {\it how long\/} selects a marked nominal that is 
introduced by {\it a\/} \citep[238]{KaySag12}. 
In support of this choice \citet{KaySag12} points out  
that it generalizes to other indefinite {\sc np}s with a pre-determiner, 
such as {\it what a mess}, {\it such a beauty\/} and {\it many a day}. 
Assuming that {\it what, such\/} and {\it many\/} are functors 
that select an indefinite {\sc np} and that have some other {\sc marking} 
value themselves, it is tempting to analyze the {\sc bmc} along the same lines. 

While we agree that it is never good to miss a generalization,
we are not convinced that the {\sc bmc} is so similar to
the combinations with {\it what, such\/} and {\it many\/} that a 
uniform treatment is called for. In fact, there are at least three 
differences between them.    
First, in the combinations with {\it what, such\/} and {\it many\/} 
the adjectives (if any) follow rather than precede the determiner. 

\begin{exe}
\ex
\begin{xlist}
\ex   What a big mess it was!
\ex   It was such a big mess!
\ex   She spent many a boring day at the office.
\end{xlist}
\ex
\begin{xlist}
\ex   [*] {What big a mess it was!}
\ex   [*] {It was such big a mess!}
\ex   [*] {She spent many boring a day at the office.}
\end{xlist}
\end{exe}

\noindent
Second, {\it what\/} and {\it such\/} are not only compatible with 
{\sc np}s that are introduced by the indefinite article, 
but also with unmarked nominals.  

\begin{exe}
\ex 
\begin{xlist}
\ex   What promise she had shown! 
\ex   What fools they are!
\end{xlist}
\ex
\begin{xlist}
\ex   Would you yourself follow such advice?  
\ex   What is one to make of such statements? 
\end{xlist}
\end{exe}

\noindent
Third, the pre-determiners may not be separated from the {\sc np} 
by {\it of\/} : {\it what (*of) a mess}, {\it such (*of) a beauty\/} 
and {\it many (*of) a splendid day at the beach}.  
These differences make the adoption of a uniform treatment less appealing. 

Conversely, the shift of the idiosyncracy to the 
internal structure of the {\sc ap} leads to a  
loss of generalization in the combination of the 
degree marker with the adjective, for while the
addition of the degree marker changes the {\sc select} 
value of the adjective in the {\sc bmc}, it does not trigger 
such a change when the {\sc ap} is used in other positions,
such as the postnominal and predicative ones in (\ref{bimepr2}).  

\begin{exe}
\ex\label{bimepr2}
\begin{xlist}
\ex   Never before had we seen a bridge that long.
\ex   This beer is so good !
\end{xlist}
\end{exe}

\noindent
Our treatment is more general in this respect, since it treats 
the internal structure of the {\sc ap} in a uniform way, no matter 
whether it is in predicative, postnominal or pre-determiner position.  

On balance then, we think that a treatment which locates the idiosyncracy 
of the {\sc bmc} in the combination of the pre-determiner {\sc ap} with 
the {\sc np} is preferable to one which locates it in the internal 
structure of the {\sc ap}. 


\subsection{Summing up}


This section has provided an analysis of the Big Mess Construction, 
both the canonical one and the variant with {\it of}.  
It is based on the assumption that the internal structure of the 
pre-determiner {\sc ap} and the indefinite {\sc np} are 
unexceptional and that the idiosyncracy resides in their combination. 
This combination is modeled in terms of a type, called {\it big-mess-phrase},
that inherits properties of {\it head-independent-phrase}, on the one hand, 
and {\it restrictive-modification}, on the other hand. The result of 
unifying the inherited constraints with the inherent ones is spelled out 
in (\ref{bigs2}). 

\begin{exe}
\ex\label{bigs2} 
\footnotesize
\begin{avm}
[{\it big-mess-phrase\/}                                         \\
 synsem [category [head ~ @1 {\it noun\/}                        \\
                   subject ~ @A                                  \\
                   comps ~ @B                                    \\
                   marking ~ @2 {\it marked\/}]                  \\
         content [{\it parameter\/}                              \\
                  index ~ @3 {\it index\/}                       \\
                  restriction ~ $\Sigma_{1}$ ~ $\bigcup$ ~ $\Sigma_{2}$]] \\
 dtrs ~ <[{\it head-functor-phrase\/}                            \\
          synsem [category [head [{\it adjective\/}              \\
                                  select ~ {\it none\/}]         \\ 
                            marking ~ @2 ]                          \\
                  content [index ~ @3                          \\
                           restriction ~ $\Sigma_{1}$ ]]] ~, @4>      \\
 head-dtr ~ @4 [{\it regular-nominal\/}                     \\
                synsem [category [head ~ @1                      \\
                                  subject ~ @A                      \\
                                  comps ~ @B                     \\
                                  marking [{\it marked\/}           \\
                                           dtype ~ {\it a\/} \($\vee$ {\it ofa\/}\)]] \\                   
                        content [{\it parameter\/}             \\
                                 index ~ @3                    \\
                                 restriction ~ $\Sigma_{2}$ ]]]]
\end{avm}
\normalsize
\end{exe}

Comparing this to the description of the regular nominals in section 3.3
the main difference concerns the proportion of inherited and inherent 
constraints. While the properties of the regular nominals are 
all inherited from its supertypes, many of the properties of 
the {\sc bmc} concern inherent constraints.


\section{The Binominal Noun Phrase Construction}


Some examples of the {\sc bnpc} are given in (\ref{climb}), repeated in (\ref{climb2}). 

\begin{exe}
\ex\label{climb2}
\begin{xlist}
\ex  She blames it on {\it her nitwit of a husband}. 
\ex\label{skull}  She had {\it a skullcracker of a headache}. 
\end{xlist}
\end{exe}

\noindent
For its analysis we subscribe to the view, argued for at length in
\citet{Aarts98} and \citet[85--108]{Keizer07}, that the 
rightmost {\sc np} is both the syntactic and semantic head: 
What she is claimed to have in (\ref{skull}),
for instance, is a headache, rather than a skullcracker. 
A challenge for that view is that it is not 
in sync with what the surface structure of the combination suggests: 
In an [{\sc n}--{\it of\/}--{\sc np}] sequence, it is the 
leftmost nominal that is canonically identified as the head. 
To solve the mismatch \citet{Aarts98} and 
\citet{Keizer07} treat the first nominal as part 
of a Modifier Phrase ({\sc mp}) that also includes {\it of\/} and the 
indefinite article, as in (\ref{aarts}). 

\begin{exe}
\ex\label{aarts}
%\footnotesize
\tree
{\ntnode{Zx}{NP},
  {\ntnode{Zy}{Det},
    {\tnode{Zz}{her}}},
  {\ntnode{Za}{N'},
    {\ntnode{Zd}{MP},
      {\tnode{Zl}{nitwit of a}}},
    {\ntnode{Zb}{N'},
    {\ntnode{Zk}{N},
      {\tnode{Zh}{husband}}}}}}
\nodeconnect{Zx}{Zy}
\nodeconnect{Zy}{Zz}
\nodeconnect{Zx}{Za}
\nodeconnect{Za}{Zd}
\nodetriangle{Zd}{Zl}
\nodeconnect{Za}{Zb}
\nodeconnect{Zb}{Zk}
\nodeconnect{Zk}{Zh}
%\normalsize
\end{exe}

\noindent
To motivate this the authors point out that the preposition 
and the article can be incorporated in the noun, 
as in {\it a helluva job}, and that the semantic contribution of 
the {\sc mp} can be paraphrased by a prenominal {\sc ap},
as in {\it a hellish job}. 

A problem with these arguments is that they do not generalize. 
Incorporation, for instance, is possible with {\it hell}, but not with 
other nouns: Combinations like *{\it her nitwituva husband\/} and 
*{\it a skullcrackeruva headache\/} do not exist. 
Similarly, the paraphrase in terms of a prenominal adjective is 
possible with nouns like {\it hell\/} and {\it fool}, but 
not with nouns like {\it nitwit\/} and {\it skullcracker}, 
at least if one expects a morphological relation between the noun 
and the adjective. 

Besides, and more importantly, it is not clear that incorporation
is a good criterion for constituency. The fact, for instance, that 
the infinitival {\it to\/} can be incorporated in certain verbs, 
as in {\it wanna\/} and {\it gonna}, then becomes 
an argument for assigning a left branching structure to 
[{\sc v}--{\it to\/}--{\sc vp}] sequences, as in  
{\it [[like to] swim]\/}, contrary to the wide-spread and independently
motivated practice of assigning it a right branching structure, 
as in {\it [like [to swim]]}.  
Similarly, the fact that the definite article is incorporated 
in certain prepositions, as in the German {\it am, zum\/} 
and {\it zur} and the French {\it du\/} and {\it au}, 
then becomes an argument for assigning a left branching 
structure to [{\sc p}--{\sc det\/}--{\sc nom}] sequences, 
as in {\it [[auf dem] Tisch]\/} `on the table' and 
{\it [[\`a la] maison]\/} `at home', contrary to the 
wide-spread and independently motivated practice of assigning 
them a right branching structure, as in {\it [auf [dem Tisch]]} and 
{\it [\`a [la maison]]\/}.    
Similar questions apply to the relevance of paraphrase relations for 
determining constituency. The fact, for instance, that 
the Dutch transitive verb {\it beluisteren\/} is nearly 
synonymous with {\it luisteren naar\/} `listen to' then becomes an argument
for treating the latter as a constituent in {\it [[luisteren naar] Mozarts laatste opera]\/} 
`listen to Mozart's last opera', contrary to the wide-spread practice 
of treating the preposition as part of the {\sc pp} complement of the verb, as in 
{\it [luisteren [naar Mozart's laatste opera]]}.\footnote{A {\it JL\/}
reviewer points out that {\it beluisteren\/} and {\it luisteren naar\/}
are not interchangeable in all contexts. That is true, but notice that  
the same remark applies to {\it hellish\/} and {\it hell of a}.}
    
In sum, the arguments in favor of (\ref{aarts}) are less than convincing. 
Besides, there are some arguments against it. \citet[14]{KimSells14}, 
for instance, points out that the treatment of the [{\sc n}--{\it of a\/}] 
sequence as a constituent complicates the expression of the constraint 
that the rightmost nominal must be a count noun, since the article is not 
a sister of that nominal. 

\bigskip

In order to avoid the problems with (\ref{aarts}) without giving up
the intuition that the rightmost nominal is the head, we adopt a 
structure which is isomorphic to the canonical structure of 
an [{\sc n}--{\it of\/}--{\sc np}] sequence, as in (\ref{napol}), 
but with the stipulation that the preposition is a functor, rather 
than the head of a {\sc pp}.    

\begin{exe}
\ex\label{napol}
%\footnotesize
\tree
{\ntnode{Zx}{NP},
  {\ntnode{Zy}{Det},
    {\tnode{Zz}{her}}},
  {\ntnode{Za}{N'},
    {\ntnode{Zd}{N},
      {\tnode{Zl}{nitwit}}},
    {\ntnode{Zw}{NP},
      {\ntnode{Ze}{P},
        {\tnode{Zm}{of}}},
      {\ntnode{Zk}{NP},
        {\ntnode{Zr}{Det},
          {\tnode{Zs}{a}}},
        {\ntnode{Zh}{N},
          {\tnode{Zi}{husband}}}}}}}
\nodeconnect{Zx}{Zy}
\nodeconnect{Zy}{Zz}
\nodeconnect{Zx}{Za}
\nodeconnect{Za}{Zd}
\nodeconnect{Zd}{Zl}
\nodeconnect{Za}{Zw}
\nodeconnect{Zw}{Ze}
\nodeconnect{Ze}{Zm}
\nodeconnect{Zw}{Zk}
\nodeconnect{Zk}{Zr}
\nodeconnect{Zr}{Zs}
\nodeconnect{Zk}{Zh}
\nodeconnect{Zh}{Zi}
%\normalsize
\end{exe}

\noindent
In fact, the properties of the preposition in the {\sc bnpc}  
are the same as those of the preposition in the {\sc bmc}, as 
spelled out in (\ref{of4}).\footnote{The similarity between the 
{\it of\/} of the {\sc bmc} and the {\it of\/} of the {\sc bnpc} 
is also pointed out in \citet[126]{Kennedy00}.}
Corroborating evidence for the functor treatment is provided by 
the contrast in (\ref{skul}).  

\begin{exe}
\ex\label{skul}
\begin{xlist}
\ex\label{sku7}  That is a book that everybody wants a copy of \_\_.    
\ex\label{sku8}  [*] {He is a husband that she blamed it on her nitwit of \_\_ !} 
\end{xlist}
\end{exe} 

\noindent
(\ref{sku7}) is well-formed since the stranded preposition is the head of a
{\sc pp}, but ({\ref{sku8}) is not, since the stranded preposition is a functor.  

Given its functor status, the combination of the preposition with its 
{\sc np} sister is a straightforward instance of the {\it regular-nominal\/} 
type. What is truly idiosyncratic about the {\sc bnpc} is the combination of 
this [{\it of\/}--{\sc np}] sequence with the first nominal,  
as in {\it [nitwit [of a husband]]}. Its properties are discussed and analyzed in 
section 5.1. A variant of the {\sc bnpc} with a bare plural,   
as in {\it jewels of villages\/}, is treated in section 5.2. 
A comparison with another monostratal treatment is provided in 
section 5.3 and a summary in section 5.4. 


\subsection{The canonical BNPC} 


To model the combination of the first nominal with the 
[{\it of\/}--{\sc np}] sequence we add a type to the hierarchy of phrases, 
called {\it binominal-np}, which is a subtype of {\it head-independent-phrase}, 
on the one hand, and {\it inverted-predication}, on the other hand.  

\begin{exe}
\ex\label{protor}
%\footnotesize
\tree
    {\ntnode{Zj}{\it phrase},
      {\ntnode{Zl}{{\sc headedness}}, 
        {\ntnode{Zh}{\it headed-phrase},
          {\ntnode{Zq}{\it hd-nonargument-phrase},
            {\ntnode{Zr}{\it hd-independent-phrase}}}}},
      {\ntnode{Zg}{{\sc clausality}},
        {\ntnode{Zm}{\it non-clause},
          {\ntnode{Zk}{\it nominal-parameter},
            {\ntnode{Zn}{\it inverted-predication},
              {\tnode{Zo}{binominal-np}}}}}}}
\nodeconnect{Zj}{Zl}
\nodeconnect{Zl}{Zh}
\nodeconnect{Zh}{Zq}
\nodeconnect{Zq}{Zr}
\nodeconnect{Zj}{Zg}
\nodeconnect{Zg}{Zm}
\nodeconnect{Zm}{Zk}
\nodeconnect{Zk}{Zn}
\nodeconnect{Zn}{Zo}
\nodeconnect{Zr}{Zo}
%\normalsize
\end{exe}

\noindent
Just as in the treatment of the {\sc bmc}, we differentiate the 
properties which the {\sc bnpc} inherits from its supertypes (5.1.1) 
from its inherent properties (5.1.2). 

 
\subsubsection{Inherited properties} 


Sharing the assumption of \citet{Aarts98} and \citet{Keizer07} 
that the first nominal is a prenominal adjunct,  
we treat the {\sc bnpc} as a {\it head-nonargument-phrase\/} in which the 
first daughter is the non-head daughter.  
Since it shares its {\sc marking} value with the mother, 
the addition of the first nominal has the effect of 
turning a marked {\sc np} into an unmarked one, as shown in (\ref{giant}).
  
\begin{exe}
\ex\label{giant}
\footnotesize
\tree
  {\ntnode{Zc}{[{\sc head} ~ $\avmbox{1}$ {\it noun} , {\sc marking} ~ $\avmbox{2}$ {\it unmarked\/}]},
    {\ntnode{Zr}{[{\sc head} ~ {\it noun} , {\sc marking} ~ $\avmbox{2}$]},
      {\tnode{Zs}{nitwit}}},  
    {\ntnode{Zb}{[{\sc head} ~ $\avmbox{1}$ , {\sc dtype} ~ {\it ofa\/}]},
      {\tnode{Zv}{of a husband}}}}
\nodeconnect{Zc}{Zr}
\nodeconnect{Zr}{Zs}
\nodeconnect{Zc}{Zb}
\nodetriangle{Zb}{Zv}
\normalsize
\end{exe}

\noindent
This accounts for the fact that the resulting combination is 
compatible with a determiner, as in {\it her nitwit of a husband}. 

Just like in the {\sc bmc}, the nonhead daughter is not a functor.  
We, hence, assume that it does not lexically select its sister.
Assuming otherwise would lead to the rather far-fetched claim 
that nouns like {\it nitwit, hell\/} and {\it skullcracker\/} 
lexically select an {\sc np} sister that is introduced by {\it of a\/}. 
In fact, since they only combine with such {\sc np}s when they 
are part of the {\sc bnpc}, the proper place to model this is 
in the definition of the {\sc bnpc} itself, rather than in the lexical 
entries of the nouns.  

\bigskip

Turning to the {\sc clausality} dimension we assume that 
the relation between the two nominals is not one of 
restrictive modification, but of predication, 
as suggested by the paraphrases in (\ref{nitwit2}). 

\begin{exe}
\ex\label{nitwit2}
\begin{xlist}
\ex  her nitwit of a husband $\rightarrow$ her husband is a nitwit
\ex  that skullcracker of a headache $\rightarrow$ that headache is like a skullcracker  
\end{xlist}
\end{exe}

\noindent
We call it more specifically {\it inverted\/} predication, since the 
order of the predicate and the predicand in the {\sc bnpc} differs from 
the canonical order in the paraphrase. This captures the same intuition 
as the one that underlies the transformational analysis of 
\citet{DenDikken98} and \citet{Bennisco98}, where the {\sc bnpc} 
is treated as a result of predicate inversion. It is also in line with  
the observation in \citet{KimSells14} that `N1 and N2 are 
in a reverse subject-predicate relation' (o.c., 6).

To model this we build on the {\sc hpsg} treatment of predicative constructions 
in \citet{VanEynde15}. In that treatment predication is 
a relation of type {\it attribute-rel\/} between a theme denoting 
nominal and an attribute denoting predicate.\footnote{This is an
alternative for the small clause treatment of predication in
transformational grammar. Arguments in favor of the relational 
approach are provided in \citet{VanEynde15}.}  
In ordinary predicative constructions, this relation is expressed 
by the verb, usually the copula, but in the {\sc bnpc} there is no 
verb: The theme denoting nominal and the predicate are part of one 
complex {\sc np}. Rather than postulating an empty verb or some newly 
minted functional category, we include the relevant relation in the 
definition of the construction itself. More specifically, we add
a type to the hierarchy of non-clausal phrases, called 
{\it inverted-predication}, which is a subtype of 
{\it nominal-parameter\/} and 
whose inherent properties are spelled out in (\ref{inver5}). 

\begin{exe}
\ex\label{inver5} {\bf Inverted Predication}                \\
\footnotesize
{\it inv-pred\/} ~ $\Rightarrow$ ~ 
\begin{avm}
[synsem|content|restr ~ $\Sigma$ ~ $\bigcup$ ~ \{[{\it fact\/} \\
                               ...|nucl [{\it attribute-rel\/} \\
                                         theme ~ @1         \\
                                         attrib ~ @2 ]]\}   \\
 dtrs ~ <[synsem|content|index ~ @2 {\it index\/}] ~, @3>      \\
 head-dtr ~ @3 [synsem|content|index ~ @1 {\it index\/}]]
\end{avm}
\normalsize
\end{exe}

\noindent
The predicative relation between the daughters is made explicit 
in the {\sc restr} value of the phrase, which contains 
a fact whose nucleus is a relation of type {\it attribute-rel\/} 
in which the {\sc theme} role is assigned to the head daughter 
($\avmbox{1}$) and the {\sc attribute} role to the non-head daughter
($\avmbox{2}$). 
The head daughter does not share its index with its sister. 
In this respect, it differs from 
phrases of type {\it restrictive-modification}, in which the  
daughters share their index. 
This has some empirical bite, since indices in {\sc hpsg} are 
used to model agreement. More specifically, the objects of type 
{\it index\/} are declared to have person, number and gender features.  

\begin{exe}
\ex {\it index\/} : \begin{avm}
                    [person ~ {\it person\/} \\
                     number ~ {\it number\/} \\
                     gender ~ {\it gender\/}]
                    \end{avm}
\end{exe}

\noindent
Signs which share their index are, hence, required to have the same 
person, number and gender. This type of agreement obtains a.o. between 
an anaphoric pronoun and its antecedent, and is distinct from 
morpho-syntactic concord, as argued in \citet{PS94}, 
\citet{Kathol99} and \citet{WechslerZlatic03}.  

The fact that the daughters in (\ref{inver5}) do not 
share their index, therefore, implies that they do not necessarily show 
agreement for person, number and gender. This may at first 
seem undesirable since the first nominal is canonically required to  
show number agreement with the second one, as illustrated by the 
ill-formedness of *{\it her nitwits of a husband}. 
Yet, a requirement of index sharing would be too strong, since 
mismatches are not necessarily ill-formed, as illustrated by  
{\it those prejudiced fools of a jury}, quoted from \citet[101]{Keizer07}. 
Unsurprisingly, the situation resembles the agreement between 
predicate nominals and subjects in copular constructions. 
A recent corpus-based study of that agreement is provided in 
\citet{VanEyndecs16}. It reports that there is 
an agreement effect (with 89.48 \% matches vs. 10.52 \% mismatches in a 
one million word sample), but it also shows that index sharing 
is not appropriate to model that effect. As an alternative, 
it develops an analysis which involves a distinction 
between collective and distributive interpretations. 
The details of that treatment need not detain us here. 
What matters in this context is that index sharing is not the 
appropriate way to handle the agreement effect, and that the assignment of 
different indices to the nominals in the {\sc bnpc} is, hence, justified. 

As compared to the {\it restrictive-modification\/} type, 
which subsumes a wide variety of nominal phrases,  
the {\it inverted-predication\/} type has a smaller range, 
but we expect it to include more members than the {\sc bnpc} alone. 
Another plausible candidate is the appositive {\it of\/}-construction, 
as exemplified by {\it the city of Rome}, where the relation 
between {\it city\/} and {\it Rome\/} can be treated as an instance of 
inverted predication: Rome is claimed to be a city.     


\subsubsection{Inherent properties} 


Besides the properties which it inherits from its supertypes, 
the {\sc bnpc} has some properties of its own. They are spelled 
out in (\ref{binominal}). 

\begin{exe}
\ex\label{binominal} {\bf Binominal Noun Phrase}        \\
\footnotesize
{\it binominal-np\/} ~ $\Rightarrow$ ~    
\begin{avm}
[dtrs ~ <[synsem|category [head ~ {\it noun\/}               \\
                           marking ~ {\it unmarked\/}]] ~, @1>  \\
 head-dtr ~ @1 [{\it regular-nominal\/}                 \\
                synsem|category|marking [{\it marked\/}         \\
                                         dtype ~ {\it ofa\/}]]] 
\end{avm}
\normalsize
\end{exe}

\noindent
The head daughter is required to be a marked regular nominal with 
the {\sc dtype} value {\it ofa}. 
The non-head daughter is required to be an unmarked nominal. 
It may be a single word, as in (\ref{climb2}), or a phrase, 
as in (\ref{destr}). 

\begin{exe}
\ex\label{destr} 
\begin{xlist} 
\ex  Her [absolute nitwit] of a husband is in trouble again. 
\ex  They ousted that [destroyer of education] of a minister. 
\end{xlist}
\end{exe}

\noindent
The former's structure is spelled out in (\ref{chop}). 

\begin{exe}
\ex\label{chop}
\footnotesize
\tree
  {\ntnode{Zb}{[{\sc head} ~ $\avmbox{1}$ {\it noun\/} , {\sc marking} ~ $\avmbox{2}$ {\it unmarked\/}]},
    {\ntnode{Zc}{[{\sc head} ~ $\avmbox{4}$ {\it noun\/} , {\sc marking} ~ $\avmbox{2}$]},
      {\ntnode{Zm}{[{\sc marking} ~ $\avmbox{2}$]},
        {\tnode{Zn}{absolute}}},
      {\ntnode{Zr}{[{\sc head} ~ $\avmbox{4}$ , {\sc marking} ~ {\it unmarked\/}]},
        {\tnode{Zs}{nitwit}}}},  
    {\ntnode{Za}{[{\sc head} ~ $\avmbox{1}$ , {\sc dtype} ~ {\it ofa\/}]},
      {\tnode{Zf}{of a husband}}}}
\nodeconnect{Zb}{Zc}
\nodeconnect{Zc}{Zm}
\nodeconnect{Zm}{Zn}
\nodeconnect{Zc}{Zr}
\nodeconnect{Zr}{Zs}
\nodeconnect{Zb}{Za}
\nodetriangle{Za}{Zf}
\normalsize
\end{exe}

\noindent
Since {\it absolute nitwit\/} is a head-functor phrase, 
its {\sc marking} value is shared with the adjective ($\avmbox{2}$}), 
and since its combination with {\it of a husband\/} is a binominal 
{\sc np} and, hence, a subtype of {\it head-nonargument-phrase}, 
that {\sc marking} value is also shared with the mother, yielding 
an unmarked nominal, which is headed by the rightmost {\sc np} 
($\avmbox{1}$). The resulting nominal can in turn be combined 
with a determiner, as in {\it her absolute nitwit of a husband}. 
Since this combination is a regular nominal, iterative application 
is not excluded. It is, hence, not impossible for a binominal 
{\sc np} to contain another binominal {\sc np}. A relevant example 
is the following quote from the preface of {\it Moby-Dick\/}. 

\begin{exe} 
\ex  this mere painstaking burrower and grubworm of a poor devil of a Sub-Sub appears to have gone through the long Vaticans and 
     street-stalls of the earth  ~~~  (Herman Melville, {\it Moby-Dick}, p. xvii) 
\end{exe}  

\noindent
The lower {\sc bnpc} {\it poor devil of a Sub-Sub\/} is contained in the 
higher {\sc bnpc} {\it painstaking burrower and grubworm of a poor devil 
of a Sub-Sub}. Semantically, the Sub-Sub is claimed to be a poor devil, as well 
as a painstaking burrower and grubworm. 


\subsection{The variant with a bare plural} 


The variant with a bare plural is exemplified by the following 
fragments from the British National Corpus, quoted in \citet[5]{KimSells14}.

\begin{exe}
\ex\label{jewels} 
\begin{xlist} 
\ex  It also has {\it jewels of villages\/} like West Burton and Askrigg ...
\ex  There was a shadowy vagueness about the rest with {\it its hulks of desks\/}
     and clutter of baskets and papers.  
\end{xlist}
\end{exe}

\noindent
This variant is not accepted by all speakers: 
\citet[85]{Foolen04} claims that English does not allow it,  
\citet[1285]{Quirk85} calls it marginal, and   
\citet[212]{Napoli89} claims that it is acceptable in
British English but not in American English. 

To model the grammar of speakers who use this variant we 
add a second lexical entry for the functor {\it of}, 
in which it selects an unmarked (bare) plural and in which it has the 
{\sc marking} value {\it ofbpl}, another subtype of {\it indefinite}. 
Its {\sc synsem} value is spelled out in (\ref{of6}). 

\begin{exe}
\ex\label{of6}
\footnotesize
\begin{avm}
[category [head [{\it preposition\/}                              \\
                 select [category|marking ~ {\it unmarked\/}      \\
                         content|index|number ~ {\it plural\/}]]  \\
           marking [{\it marked\/}                                \\
                    dtype ~ {\it ofbpl\/}]]                       \\
 content|restriction  ~~ \{ \}]
\end{avm}
\normalsize
\end{exe}

\noindent
Besides, we slightly modify the definition of the {\sc bnpc} in 
(\ref{binominal}), allowing the {\sc dtype} value of the 
head daughter to be either {\it ofa\/} or {\it ofbpl}. 


\subsection{Comparison with another treatment} 


There are many analyses to which our treatment could be compared, 
but we will limit the discussion here to one that is  
sufficiently similar to our treatment to make a comparison 
feasible and fruiful, i.e. the one of \citet{KimSells14}. 
It is cast in the notation of Sign-Based Construction Grammar, 
just like the analysis of the {\sc bmc} in \citet{KaySag12}. 
The intuition that underlies the analysis is basically the same as 
that of \citet{Aarts98}, \citet{Keizer07} and our treatment: 
`we assume that the second element {\sc np2/n2} functions as 
the syntactic as well as semantic head while the first one 
serves as the modifier' \citep[29]{KimSells14}. The way in which 
this is made explicit is rather different, though. For a start, the 
{\sc bnpc} is assigned a flat structure, as in (\ref{kim}). 

\begin{exe}
\ex\label{kim}
%\footnotesize
\tree
{\ntnode{Zx}{NP},
  {\ntnode{Zy}{Det},
    {\tnode{Zz}{her}}},
  {\ntnode{Za}{N'_{i}},
    {\ntnode{Zd}{N'},
      {\tnode{Zl}{nitwit}}},
    {\ntnode{Ze}{P},
      {\tnode{Zm}{of}}},
    {\ntnode{Zk}{NP_{i}},
      {\tnode{Zh}{a husband}}}}}
\nodeconnect{Zx}{Zy}
\nodeconnect{Zy}{Zz}
\nodeconnect{Zx}{Za}
\nodeconnect{Za}{Zd}
\nodeconnect{Zd}{Zl}
\nodeconnect{Za}{Ze}
\nodeconnect{Ze}{Zm}
\nodeconnect{Za}{Zk}
\nodetriangle{Zk}{Zh}
%\normalsize
\end{exe}

\noindent
Second, the construction which licenses this flat structure 
is claimed to be headed and non-headed at the same time. 
It is headed in the sense that the first nominal 
denotes a property of the entities ({\it i\/}) 
that are denoted by the second nominal. 
And it is non-headed in the sense that `the {\sc bnp} has 
sequences of nominals fulfilling the same grammatical function, 
neither of which is syntactically dependent on the other' (o.c., 22). 
To make sense of this somewhat paradoxical situation the authors 
introduce a type, called {\it head-mod-juxtaposition}, which is a subtype of 
{\it coordination}, on the one hand, and {\it head-mod-cx},
on the other hand, as in (\ref{juxt}).

\begin{exe}
\ex\label{juxt}
\footnotesize
\tree
      {\ntnode{Zl}{{\sc headedness}}, 
        {\ntnode{Zg}{\it non-headed},
          {\tnode{Zm}{coordination},
            {\ntnode{Zn}{\it head-mod-juxtaposition},
              {\tnode{Zo}{bnp-cx}},
              {\tnode{Zp}{correlative}},
              {\tnode{Zs}{one-more}},
              {\tnode{Zx}{...}}}}},
        {\ntnode{Zh}{\it headed},
          {\ntnode{Zr}{\it head-mod-cx}},
          {\tnode{Zy}{...}}}}
\nodeconnect{Zl}{Zh}
\nodeconnect{Zh}{Zr}
\nodeconnect{Zh}{Zy}
\nodeconnect{Zl}{Zg}
\nodeconnect{Zg}{Zm}
\nodeconnect{Zm}{Zn}
\nodeconnect{Zn}{Zo}
\nodeconnect{Zn}{Zp}
\nodeconnect{Zn}{Zs}
\nodeconnect{Zn}{Zx}
\nodeconnect{Zr}{Zn}
\normalsize
\end{exe}

\noindent
Being a subtype of {\it head-mod-juxtaposition\/}, the {\sc bnpc}  
`inherits only some syntactic properties from coordination and 
some semantic properties from subordination' (o.c., 30).\footnote{Besides
the {\sc bnpc}, the {\it head-mod-juxtaposition\/} type 
subsumes the correlative construction, as exemplified by 
{\it the less I do, the better I feel}, and the 
{\it one more\/} construction, as exemplified by 
{\it one more can of beer and I am leaving\/} \citep[28]{KimSells14}.} 
This begs the question of which constraints are inherited from 
which of the supertypes, but on that point the authors 
remain vague: `we leave it open for future research how
to filter these partial properties correctly, and ensure their inheritance
in the {\sc bnp}' (o.c, 30). In this respect, the treatment
that we propose is more detailed: Both the {\sc bnpc} itself 
and its supertypes are defined explicitly and the inheritance is 
strictly monotonic, in the sense that subtypes inherit all properties 
of their supertypes. 
 

\subsection{Summing up} 


This section has provided an analysis of the Binominal Noun Phrase 
Construction, both the canonical one and the variant with the 
bare plural. Its properties are modeled in terms of a phrase type 
that inherits the constraints of {\it head-independent-phrase}, on 
the one hand, and {\it inverted-predication}, on the other hand. 
The result of unifying the inherited constraints with the inherent 
ones is spelled out in (\ref{bnp2}). 

Comparing this {\sc avm} to that of the Big Mess Construction
we observe both similarities and differences. 
The main similarity concerns the fact that they
are both subtypes of {\it head-independent-phrase}. 
Another similarity concerns the functor status of the preposition {\it of}. 
The main difference concerns their place in the {\sc clausality} hierarchy: 
While the {\sc bmc} is an instance of restrictive modification, 
just like the regular nominals, the {\sc bnpc} is an instance of 
inverted predication, a less common type of combination. 
It appears then that the {\sc bnpc} has less in common with the 
regular nominals than the {\sc bmc}. In that sense 
it shows a higher degree of idiosyncracy. 

\begin{exe}
\ex\label{bnp2} 
\footnotesize
\begin{avm}
[{\it binominal-np\/}                                       \\
 synsem [category [head ~ @1 {\it noun\/}                        \\
                   subject ~ @A                                     \\
                   comps ~ @B                                    \\
                   marking ~ @2 {\it unmarked\/}]                   \\
         content [{\it parameter\/}                            \\
                  index ~ @3 {\it index\/}                     \\
                  restr ~ $\Sigma_{1}$ ~ $\bigcup$ ~ $\Sigma_{2}$ ~ 
                          $\bigcup$ ~ \{[{\it fact\/}          \\
                                  ...|nucl [{\it attribute-rel\/} \\
                                            theme ~ @3         \\
                                            attrib ~ @4 ]]\}]] \\
 dtrs ~ <[synsem [category [head [{\it noun\/}                   \\
                                  select ~ {\it none\/}]         \\ 
                            marking ~ @2 ]                          \\
                  content [index ~ @3                          \\
                           restriction ~ $\Sigma_{1}$ ]]] ~, @5>      \\
 head-dtr ~ @5 [{\it regular-nominal\/}                     \\
                synsem [category [head ~ @1                      \\
                                  subject ~ @A                      \\
                                  comps ~ @B                     \\
                                  marking [{\it marked\/}           \\
                                           dtype ~ {\it ofa\/} \($\vee$ {\it ofbpl\/}\)]] \\                   
                        content [{\it parameter\/}             \\
                                 index ~ @4                    \\
                                 restriction ~ $\Sigma_{2}$ ]]]]
\end{avm}
\normalsize
\end{exe}


\section{Conclusion}


This paper has provided a sign-based treatment of both regular nominals
and two types of idiosyncratic nominals, i.e. the Big Mess Construction 
and the Binominal {\sc np} Construction. Special attention has been paid to the 
interaction of the regular and the exceptional in the idiosyncratic nominals. 
To model that interaction we have exploited the potential of 
constructivist {\sc hpsg}, as laid out in \citet{GS00}. 
The core of the analysis is a bi-dimensional hierarchy of 
phrase types in which the properties of the nominals 
are partly inherited from their supertypes and partly spelled out 
in construction-specific constraints. 
(\ref{protorp}) presents the part of the hierarchy that we have focused on. 

\begin{exe}
\ex\label{protorp}
\footnotesize
\tree
    {\ntnode{Zj}{\it phrase},
      {\ntnode{Zl}{{\sc headedness}}, 
        {\ntnode{Zh}{\it headed-phrase},
          {\ntnode{Zq}{\it hd-nonargument-phrase}, 
            {\ntnode{Zr}{\it hd-functor-phrase},
              {\tnode{Zx}{regular-nominal}}},
            {\ntnode{Zt}{\it hd-indep-phrase}}}}},
      {\ntnode{Zg}{{\sc clausality}},
        {\ntnode{Zm}{\it non-clause},
          {\ntnode{Zk}{\it nominal-parameter},
            {\ntnode{Za}{\it restrictive-mod},
              {\tnode{Zb}{big-mess-phrase}}},   
            {\ntnode{Zn}{\it inverted-pred},
              {\tnode{Zo}{binominal-np}}}}}}}
\nodeconnect{Zj}{Zl}
\nodeconnect{Zl}{Zh}
\nodeconnect{Zh}{Zq}
\nodeconnect{Zq}{Zr}
\nodeconnect{Zq}{Zt}
\nodeconnect{Zj}{Zg}
\nodeconnect{Zg}{Zm}
\nodeconnect{Zm}{Zk}
\nodeconnect{Zk}{Za}
\nodeconnect{Zk}{Zn}
\nodeconnect{Zn}{Zo}
\nodeconnect{Zt}{Zo}
\nodeconnect{Za}{Zb}
\nodeconnect{Zt}{Zb}
\nodeconnect{Zr}{Zx}
\nodeconnect{Za}{Zx}
\normalsize
\end{exe}

\bigskip

The resulting treatment is one `in which the particular and the 
general are knit together seamlessly' \citep{KayFillmore99}. 
Besides, it chimes well with one of the tenets of Construction 
Grammar that `a linguistic structure is motivated to the 
extent that it is related to other structures in the language' 
\citep{Taylor04}. This relatedness is made fully explicit 
in the phrase type hierarchy and in the constraints which 
are associated with the various types. 

In future work we want to explore the potential of this approach 
for dealing with other types of complex nominals with idiosyncratic 
properties, such as appositions with and without {\it of\/} 
({\it the city of Berlin\/} vs. {\it the poet Burns\/}),  
and partitive and pseudo-partitive {\sc np}s. This, we believe, 
will ultimately yield a fine-grained hierarchy of nominal phrase types 
which charts their mutual similarities and differences in 
a fully explicit manner and which makes it possible to measure 
the degree of idiosyncracy of their formation.  


\bibliographystyle{unified} 
\bibliography{prenom}
\end{document}




















\section{Introduction}


Nominal structures minimally contain a noun. 
That may be all there is, as in the subject of {\it oil is expensive}, 
but usually the noun is accompanied by one or more dependents. 
They can precede the noun, such as the article and the adjective in 
{\it the red box}, or follow the noun, such as the prepositional phrase
and the relative clause in {\it books about WW I which are out of print}. 

Such combinations are constrained by co-occurrence restrictions. 
An obvious one concerns the possibility of stacking. 
While a noun can be combined with more than one adjective or {\sc pp}, 
as in {\it red wooden boxes\/} and {\it books on sale about WW II}, 
it cannot be combined with more than one article: *{\it the a box}. 
In that respect, the articles belong to a group of words, known as 
determiners (Det), that also includes the demonstrative in {\it this box\/} 
and the interrogative in {\it which car}. 
The defining property of this group of words is that they 
are in complementary distribution with the articles: 
*{\it the this/which box\/} and *{\it this/which a box}.
One way to model these observations is the rewrite rule in (\ref{ps0}).  

\begin{exe}
\ex\label{ps0}  NP ~ $\rightarrow$ ~ (Det) ~ A* ~ N ~ PP* ~ S*  ~~~~ (where * is Kleene star) 
\end{exe} 

\noindent
This rule licenses flat structures in which the dependents are all sisters of the noun. 

Applying some classical constituency tests, though, it turns out that there is
evidence for a more hierarchical structure. 
The conjunction test, for instance, reveals that it is not only possible to 
conjoin full NPs, but also parts of NPs.
In {\it every man above forty and woman under thirty}, for instance,  
{\it every\/} is combined with the nominal {\it man above forty and woman under thirty}. 
This suggests a binary branching structure in which the determiner is combined 
with the rest of the noun phrase, as in (\ref{ps1}), quoted from \citeasnoun[31--32]{SagWasow03}. 

\begin{exe}
\ex\label{ps1}  NP ~ $\rightarrow$ ~ (Det) ~ Nom
\end{exe} 

\noindent
Confirming evidence is provided by the pro-form replacement test. While personal pronouns  
are distributionally equivalent to full NPs, suggesting that they are pro-NPs, there are
also pro-forms which are distributionally equivalent to nominals without determiner, 
i.e. pro-Noms, such as the English {\it one\/} and its plural counterpart {\it ones\/} in (\ref{app}).  

\begin{exe}
\ex\label{app}  
\begin{xlist} 
\ex  John took this apple_{i} and Mary took that one_{i}. 
\ex  John took these apples_{i} and Mary took those ones_{i}. 
\end{xlist}
\end{exe}

The same tests provide evidence that the addition of adjectives and {\sc pp}s
had better be modeled in terms of binary branching rules too, as in (\ref{ps2}), 
quoted from \citeasnoun[80]{Levin17}.

\begin{exe}
\ex\label{ps2} 
\begin{xlist} 
\ex   Nom ~ $\rightarrow$ ~ Adj ~ Nom 
\ex   Nom ~ $\rightarrow$ ~ Nom ~ PP 
\end{xlist} 
\end{exe} 

\noindent
This accounts for the fact that an adjective or {\sc pp} can scope over a conjunction 
of Noms, as in (\ref{king}), and that it can be followed, c.q. preceded, by the pro-Nom 
{\it one}, as in (\ref{wood}). 

\begin{exe} 
\ex\label{king} 
\begin{xlist} 
\ex  the [former [French presidents and Spanish kings]]
\ex  the [[wooden tables and leather chairs] in this room]
\end{xlist}
\ex\label{wood}
\begin{xlist} 
\ex  John bought the [old [box with the green lid]_{i}] and Mary bought the [new [one]_{i}]. 
\ex  John bought the [[wooden box]_{i} with the green lid] and Mary bought the [[one]_{i} with the blue lid]. 
\end{xlist}
\end{exe}

\noindent
Since the rules in (\ref{ps2}) are recursive, they allow stacking.  
This is justified for {\sc pp} modifiers, but not for {\sc pp} complements, 
since their number is limited by the argument structure of the noun.
The relational noun {\it sister}, for instance, takes at most  
one {\sc pp} complement, as in {\it the sister of Leslie}. 
Modeling this requires a non-recursive rule, as in (\ref{ps3}).

\begin{exe} 
\ex\label{ps3}    Nom ~ $\rightarrow$ ~ N ~ (PP) 
\end{exe} 

The distinction is also relevant for clausal dependents. 
While clausal modifiers, such as relative clauses, can be stacked, 
clausal complements cannot. The deverbal noun {\it claim},
for instance, takes at most one clausal complement, as in 
{\it the claim that Brexit is inevitable}. To model this 
one can use the same kinds of rules as for the {\sc pp} dependents, 
i.e. a recursive one for the modifiers anda non-recursive one for the complements. 

\begin{exe}
\ex\label{ps4} 
\begin{xlist}
\ex   Nom ~ $\rightarrow$ ~ Nom ~ S
\ex   Nom ~ $\rightarrow$ ~ N ~ (S) 
\end{xlist} 
\end{exe} 

The resulting set of rewrite rules fits the mold of X-bar syntax, as 
developed in 
Characteristic of X-bar syntax is that it does not employ category specific 
rewrite rules, but cross-categorial rule schemata.  
For any category X, where X initially ranged over the substantive categories 
V, N, A and P, there is a phrasal projection X^{i}, for 0 $\leq$ i $\leq$ 2,
which conforms to the rule schemata in (\ref{xbar}).     

\noindent
A double-bar category X^{2} consists of a specifier and the single-bar 
category X^{1}, and X^{1} consists of the lexical category X^{0} and its 
complements. Modifiers are recursively adjoined to X^{1}.\footnote{In some 
versions of X-bar syntax modifiers can also be adjoined to X^{2}. There 
are also versions with more than two bar levels.} 
In transformational grammar the X-bar schemata are constraints on
the rules which generate deep structures, but they also got adopted in 
non-transformational frameworks and applied to surface structures. 
An example is the analysis of the noun phrase in (\ref{sis}), quoted from 

, the standard reference for Generalized Phrase Structure Grammar ({\sc gpsg}).

\noindent
The noun phrase is analyzed as a projection N^{i}, in which the 
double-bar category N^{2} (NP) consists of the determiner and the 
single-bar category N^{1} (Nom), in which the AP and the relative clause are 
adjoined to N^{1}, and in which the lowest N^{1} consists of the 
noun (N) and its {\sc pp} complement.  

For a presentation of the {\sc hpsg} treatment of nominal structures 
we make a distinction between those that fit the mold of rule (\ref{ps0}), 
henceforth called the regular nominals, and those that do not. 
They are discussed in sections 2 and 3 respectively.\footnote{Relative clauses 
will not be discussed. They are the topic of a separate chapter in this volume.}  


\section{Regular nominals} 


The oldest {\sc hpsg} analysis of nominal structures can be characterized 
as a lexicalist reformulation of the {\sc gpsg} analysis. 
It is first proposed in

We call it the specifier treatment, 
after the role which it assigns to the determiner. 
As such, it contrasts with the {\sc dp} treatment, in which the determiner is 
considered to be the head of the nominal, as proposed in 
\citeasnoun{Netter94} and \citeasnoun{Netter96}, and with  
the functor treatment, in which the distinction between specifiers and 
modifiers is eliminated, as proposed in \citeasnoun{VanEynde98a} and 
\citeasnoun{Allegranza98}, and further developed in \citeasnoun{VanEynde03}, 
\citeasnoun{VanEynde06} and \citeasnoun{Allegranza07}.\footnote{The term {\it functor\/} 
is also used in categorial (unification) grammar, where it stands for the nonhead daughter in
combinations of a head with a specifier or a modifier, and for the head
daughter otherwise, see \cite{Bouma88}. This broader notion of `functor' 
is also used in \cite{Reape94}.}  



\subsection{The specifier treatment} 


For the presentation of the specifier treatment we first focus on the  
nominals with a determiner and then on those without. 


\subsubsection{A lexicalist NP treatment} 


The first proposal to model the noun phrase in {\sc hpsg} adopts the same 
kind of structure as in {\sc gpsg}, but dispenses with the bar levels. 
The role of registering the degree of saturation is taken over by the 
valence feature {\sc subcat}. In the case of a relational noun, such as 
{\it sister}, the {\sc subcat} value contains a determiner and a {\sc pp}. 
When the noun is combined with a matching {\sc pp}, the {\sc pp}-requirement 
is canceled from the {\sc subcat} list, and when it is combined with a determiner, 
the Det-requirement is canceled too, as illustrated in (\ref{les}) \cite[139--143]{PS87} and 
\cite[47--57]{PS94}.\footnote{The members of the {\sc subcat} list 
in (\ref{les}) are ordered as in \citeasnoun{PS94}, which is the mirror image of 
the order in \citeasnoun{PS87}.} 

\noindent
In this analysis a fully saturated nominal is one with an empty {\sc subcat} list. 
Since the members of a {\sc subcat} list are objects of type {\it synsem}, i.e. 
bundles of syntactic and semantic features, they are not only used to 
register the degree of saturation, but also to capture more specific co-occurrence 
restrictions, such as number agreement between noun and determiner (*{\it that sisters\/})
and constraints on the category of the complement ({\sc pp} vs. {\sc s}). 
 
In contrast to the determiner and the complements, the modifiers are 
not part of the {\sc subcat} value. This is justified in the sense 
that their addition has no effect on the degree of saturation of the 
nominal, but it also implies that one needs another device to   
model the co-occurrence restrictions between modifiers and their heads. 
The fact, for instance, that modifiers of nominals can be adjectival but not 
adverbial has to be captured in some other way.
For this purpose early {\sc hpsg} explored two approaches. 

The first one is based on the assumption that a head selects its modifiers. 
To model this \citeasnoun[161--168]{PS87} employs a set valued feature, 
called {\sc adjuncts}, which specifies for some given word with what kind of 
modifiers it combines. The {\sc adjuncts} value of the noun {\it sister\/} in 
(\ref{les}), for instance, contains an {\sc ap}. Since {\sc adjuncts} is a 
{\sc head} feature, this value is shared with the phrase {\it sister of Leslie}, 
and matched with the category of the modifier. 

The second approach is based on the assumption that a modifier selects 
its head. To model this \citeasnoun[47--57]{PS94} employs  
a feature called {\sc mod(ified)}. It is part of the 
{\sc head} value of the substantive parts-of-speech, i.e. noun, verb, adjective
and preposition. Its value is of type {\it synsem\/} in the case of 
modifying items and of type {\it none\/} otherwise. 
In the case of the adjective {\it tall\/} in (\ref{les}), the    
{\sc mod} value consists of a nominal that has not yet been 
combined with a determiner ([{\sc head} ~ {\it noun\/}, {\sc subcat} ~ $<$Det$>$]). 
This value is shared with the {\sc ap} {\it very tall},  
as required by the Head Feature Principle, and matched with 
the {\sc synsem} value of its head sister, as illustrated in (\ref{lea}). 

\noindent
The constraint on the {\sc subcat} value of the selected nominal 
captures the fact that the {\sc ap} cannot be combined with a fully saturated 
nominal, as in *{\it very tall that sister of Leslie}. 
The incompatibility with adverbial modifiers is accounted for in 
the {\sc mod} value of the adverbs: They select a head that is not nominal.   

The {\sc mod} feature is also used to model {\sc np}-internal agreement. 
For languages in which the adjectives show number and gender agreement with the 
nominal, such as Italian and French, the {\sc mod} value of the adjective captures 
the relevant co-occurrence restrictions. 
The {\sc mod} value of the Italian {\it rossa\/} `red', for instance, 
contains the restriction that the modifed nominal be singular and feminine. 
This accounts for the fact that it is compatible with 
{\it scatola\/} `box', but not with the plural {\it scatole\/} `boxes',
nor with the masculine {\it libro\/} `book' or {\it libri\/} `books'.

\bigskip

Parallel with the syntactic build-up is the semantic composition. 
This is modeled in terms of the {\sc content} feature. For nominal 
phrases the relevant {\sc content} value is called {\it nominal-object\/} 
in \citeasnoun{PS94} and {\it scope-object\/} in \citeasnoun{GS00}. 
The latter also makes a distinction between scope-objects 
that contain a quantifier ({\it quant-rel\/}), and those that do not 
({\it parameter\/}). A scope-object is an index-restriction pair in which 
the index stands for entities and in which the restriction is a set of facts 
which constrain the denotation of the index, as in (\ref{red}).    

\noindent
The adjective selects a nominal, shares its index and adds its own 
restriction to those that are already present in the nominal.\footnote{{\sc n'} 
stands for {\sc n}-bar, i.e. a nominal with one element on its {\sc subcat} list, 
i.e. the determiner.} The resulting {\sc content} value is then shared with the mother. 

The addition of a quantifying determiner, such as {\it every}, triggers a 
type shift from {\it parameter\/} to {\it quant-rel}. This is modeled in terms of 
another selection feature, called {\sc spec(ified)}. It is assigned to the functional 
parts-of-speech, i.e. determiner and marker, and its value is of type {\it synsem}. 
In the case of {\it every}, for instance, the {\sc spec} value contains a nominal 
whose {\sc content} value is an index-restriction pair. This is integrated in the 
{\sc content} value of the determiner, and the resulting quantifier is 
put in store, to be retrieved at the place where its scope is determined, 
as illustrated by the {\sc avm} of {\it every\/} in 
\citeasnoun[204]{GS00}.\footnote{{\it every-rel\/} is a subtype of {\it quant-rel}.}  


\noindent
Just like the {\sc mod} feature, the {\sc spec} feature also plays a role 
in the treatment of {\sc np}-internal agreement. \citeasnoun[371--373]{PS94}, 
for instance, uses it to model the effect of declension class ({\it weak, strong, mixed\/}) 
on the inflection of prenominal adjectives in German.  


\subsubsection{Nominals without determiner} 


A complication for the specifier treatment concerns the existence of bare plurals 
and mass nouns. In contrast to singular count nouns, which can be said to 
require a determiner in languages like English, mass nouns and plurals may 
combine with a determiner, but they do not need to. 

One way to accommodate this is to make the Det requirement in their 
{\sc subcat} list optional. This, however, complicates the treatment of the modifiers, 
for if the {\sc subcat} list of the mass noun in {\it water is a scarce commodity\/} 
is empty, then one has to allow the adjective in {\it fresh water\/} to combine 
with a nominal with an empty {\sc subcat} list, and in that case one also 
licenses the ill-formed combination *{\it fresh the water}.

An alternative is ``to employ phonetically empty determiners in bare plural 
and mass NPs.'' \cite[90]{PS94} This preempts the need to tinker with the 
treatment of the modifiers, but it does not square well with the tendency 
in {\sc hpsg} to avoid the use of empty elements. 

Another alternative is the use of a non-branching rule, as proposed 
in \citeasnoun[191--192]{GS00}. Such a rule licences a fully saturated nominal 
with one daughter in which the specifier is missing, 
as in (\ref{los}).\footnote{In this analysis the {\sc subcat} feature is 
replaced by the more specific features {\sc subj(ect)}, {\sc sp(ecifie)r} 
and {\sc comp(lement)s}. The use of the more specific valence features 
is advocated in chapter 9 of \citeasnoun{PS94}, following a proposal in
\citeasnoun{Borsley87}, and has become a matter of common practice since. 
The {\sc subj} feature is not mentioned in (\ref{los}),  
since its value is the empty list throughout.}   

\noindent
Innocent as this move may seem, it has far reaching consequences, since 
it involves a departure from the purely lexicalist approach of early {\sc hpsg}. 
Characteristic of the purely lexicalist approach is that the properties of a 
phrase are fully determined by the properties of the words which it contains, 
on the one hand, and by a small set of cross-categorial schemata, on the other hand. 
In less radical variants of lexicalism, phrases may have properties which 
derive from more specific constraints on phrase formation. Non-branching rules
are an example of this.  
 
To pave the way for an analysis of phenomena which resist a purely lexicalist 
treatment \citeasnoun{Sag97} introduced the idea of classifying phrases in terms of  
a bi-dimensional hierarchy.  
In such a hierarchy, the cross-categorial schemata are rephrased as constraints on 
types of phrases in one dimension, called {\sc headedness}, and 
complemented by possibly category-specific constraints on types of phrases 
in another dimension, called {\sc clausality}. Since the dimensions are 
orthogonal to one another, it is possible to define phrasal types by multiple inheritance,
in the same way as this is done for the lexical types, see the chapter on the lexicon in this volume. 
The proposal was originally proposed for the analysis of relative clauses, 
but got extended to all types of clauses in \citeasnoun{GS00}. 
The latter also contains a small inventory of 
non-clausal phrase types, including the one in (\ref{bidi}). 


\noindent
The {\sc headedness} hierarchy is extended with a new type, 
called {\it head-only-phrase}, which subsumes headed phrases 
which have only got a head daughter. One of its subtypes is 
{\it bare-nominal-phrase}. It inherits the properties of 
its supertypes and has some properties of its own, spelled out 
in (\ref{bare}), quoted from \citeasnoun[191]{GS00}.\footnote{We 
have taken the liberty of converting the rewrite 
rule format of \citeasnoun{GS00} into the {\sc avm} format.}  

\noindent
This type is category-specific and contains the extra requirement that the
unexpressed determiner have an empty {\sc wh} set.\footnote{{\sc wh} registers the presence
of {\it wh}-words in a sign. It is used to model pied piping effects.} Strange enough,  
the constraint does not include a restriction to plurals and mass nouns.  
That such a restriction is intended, though, is clear from the discussion of bare plurals 
and mass nouns in \citeasnoun[265--266]{SagWasow03}, where the formulation of the 
constraint is left as an exercise for the reader. 

Summing up, the specifier treatment of bare plurals and mass nouns involves a 
departure from the purely lexicalist stance of early {\sc hpsg}. 
This can take various forms, but the one that has stuck and that will also play a 
pivotal role in the treatment of idiosyncratic nominals in section 3 is based on  
the use of a bi-dimensional hierarchy of phrases, in which the maximal types 
may contain detailed category-specific constraints, either by inheritance or 
by declaration. 


\subsection{The DP treatment} 


The departure from pure lexicalism did not appeal to Klaus Netter. 
He proposed an alternative, aiming for 
``the theoretical rigor we can derive from forbidding unary rules 
for category transformation or unmotivated empty terminals.'' \cite[321]{Netter94} 
This alternative takes some inspiration from a development in transformational grammar: 
While the X-bar principles were initially meant for  
verbs, nouns and adjectives (V, N, A) and somewhat later for 
prepositions (P), it became common practice during the eighties to 
apply them to all categories, including the functional ones, 
such as complementizer (C) and determiner (D). 
In this style of analysis, determiners are lexical heads (D) which take a 
nominal projection as their complement and which yield a DP,  
as in (\ref{abn}) \cite{Abney87}.\footnote{Modifiers are recursively adjoined to N^{1} or N^{2}.} 

\noindent
This style of analysis is widely known as the {\sc dp} treatment and was taken on board 
in a number of non-transformational frameworks, such as Word Grammar \cite{Hudson90} 
and Lexical Functional Grammar \cite[99]{Bresnan00}. Klaus Netter took it as the starting point for 
an {\sc hpsg} analysis of the noun phrase, described in \cite{Netter94} and \cite{Netter96}. 




\subsection{The Binominal Noun Phrase Construction}
 

This paper explores the interaction of regularity and idiosyncracy in the 
formation of nominals. For the sake of concreteness it first focusses on a 
highly regular type of combination, as exemplified by {\it red box}, and
then on two combinations which show some degree of idiosyncracy.   

The {\sc bmc} and the {\sc bnpc} are of course not the only nominals with 
idiosyncratic properties, nor are they particularly similar in their
idiosyncracies, but it is precisely the diversity which makes them 
a good test case for the more general purpose of investigating  
how the regular and the idiosyncratic interact in the formation of  
nominals. Focussing on only one or two very similar constructions would be
less revealing. 
For the investigation we adopt an approach that is close in spirit to 
one of the tenets of Construction Grammar, as formulated in 
\citet{KayFillmore99}:

\begin{quote} 
To know what is idiomatic about a phrase one has to 
know what is nongeneral and to identify 
something as nongeneral one has to be able to identify the general... 
The picture that emerges from the consideration of special constructions
... is of a grammar in which the particular and the general are knit 
together seamlessly.
\end{quote}

To make this emerging picture sharp we adopt the framework of 
constructivist Head-driven Phrase Structure Grammar ({\sc hpsg}), 
as pioneered in \citet{Sag97} and developed more fully in 
\citet{GS00}. This is arguably the framework that has been most 
hospitable to ideas from Construction Grammar, providing them with 
formal underpinnings and an explicitly defined notation. 
Characteristic of constructivist {\sc hpsg} is 
the idea that phrases are organized in terms of a hierarchy of types 
in which the lower (more specific) types are a blend of properties 
that are inherited from the higher (less specific) types, on the one hand, 
and of type-specific (inherent) properties, on the other hand.   
\citet{GS00} provides a detailed application of this approach to 
a range of clausal types. This paper extends it to nominal types. 

The paper starts with a presentation of constructivist {\sc hpsg} 
in section 2. It then provides an analysis of the regular nominals in 
section 3, of the Big Mess Construction in section 4 and of the 
Binominal Noun Phrase Construction in section 5. 
The conclusions are in section 6. 


\section{Basic notions of constructivist {\sc hpsg}} 


In {\sc hpsg} the basic unit of linguistic analysis is the sign,
understood in the Saussurean way as a unit of form 
({\it signifiant\/}) and meaning ({\it signifi\'e\/}).  
Employing the notation of typed feature structures, this is 
expressed in terms of a {\bf feature declaration}. 

\begin{exe}
\ex\label{sign} 
{\it sign\/} : \begin{avm} 
               [form   ~ {\it list\/} \({\it grapheme\/}\) \\
                synsem ~ {\it synsem\/}]
               \end{avm}
\end{exe}

\noindent
The value of the {\sc form} feature is a list of 
graphemes,\footnote{Beside or instead of {\sc form}, one also finds 
{\sc phonology}, which takes a list of phonemes as its value.}   
and the value of the {\sc synsem} feature is an object of type 
{\it synsem}, which in turn is declared to have a feature for modeling 
syntactic properties, called {\sc category}, and a feature 
for modeling semantic properties, called {\sc content}. 

\begin{exe}
\ex {\it synsem\/} : \begin{avm}
                     [category ~ {\it category\/}      \\
                      content ~ {\it semantic-object\/}]
                     \end{avm}
\end{exe}

Signs are organized in a {\bf hierarchy}. The basic dichotomy is that 
between lexical signs and phrases. 

\begin{exe}
\ex\label{phrase}
\tree
{\ntnode{Zt}{\it sign}, 
  {\tnode{Zd}{lexical-sign}},
  {\tnode{Zc}{phrase}}}
\nodeconnect{Zt}{Zd}
\nodeconnect{Zt}{Zc}
\end{exe}

\noindent
Since types inherit the properties of their supertypes, 
the objects of type {\it lexical-sign\/} and {\it phrase\/} 
have a {\sc form} and a {\sc synsem} feature as well. 
Besides, phrases have a {\sc daughters} feature, 
whose value is a list of signs. 

\begin{exe}
\ex {\it phrase\/} : \begin{avm} 
                     [dtrs ~~ {\it list\/} \({\it sign\/}\)]
                     \end{avm}
\end{exe}

\noindent
The phrase {\it Kim laughed}, for instance, has two daughters, and 
since those daughters are objects of type {\it sign}, they each have 
a {\sc form} and a {\sc synsem} feature, as shown in 
(\ref{metkim}).\footnote{The members of a list are separated by commas. 
The empty list is written as $<$ $>$. The values of the {\sc cat(egory)} 
and {\sc cont(ent)} features in (\ref{metkim}) are abbreviations of more 
complex feature structures, to be spelled out as we go along.}
 
\begin{exe}
\ex\label{metkim} 
\footnotesize
\begin{avm} 
[{\it phrase\/}                                   \\
 form ~ <{\rm Kim , laughed}>                     \\
 synsem [cat ~ {\it clause}                    \\
         cont ~ {\it proposition\/}]            \\
 dtrs ~ <[form ~ <{\rm Kim}>                      \\
          synsem [cat  ~ {\it proper-noun\/}      \\
                  cont ~ {\it scope-object\/}]] ~,   
         [form ~ <{\rm laughed}>                  \\
          synsem [cat  ~ {\it intr-verb\/}        \\
                  cont ~ {\it state-of-affairs\/}]]>]
\end{avm}
\normalsize
\end{exe}

\noindent
This feature structure makes explicit that the phrase 
{\it Kim laughed\/} is a clause that denotes a proposition, 
and that its daughters are a proper noun that denotes a scope-object
and an intransitive verb that denotes a state-of-affairs. 

To model the formation of phrases in general terms 
{\sc hpsg} does not employ rewrite rules and monadic categories, 
such as {\sc np} and {\sc vp}. Instead, it analyzes categories in 
terms of features and it models their combination in terms of 
cross-categorial constraints on phrase types. More specifically, 
syntactic categories are objects of type {\it category\/} 
and these are declared to have a {\sc head\/} feature, the valence features
{\sc subject} and {\sc comp(lement)s} and a {\sc marking} feature.  
   
\begin{exe}
\ex\label{cate}
{\it category\/} : \begin{avm}
                   [head  ~ {\it part-of-speech\/}          \\
                    subject  ~ {\it list\/}~\({\it synsem\/}\) \\
                    comps ~ {\it list\/}~\({\it synsem\/}\) \\
                    marking  ~ {\it marking\/}]
                   \end{avm}
\end{exe}

\noindent
The value of the {\sc head} feature is a part-of-speech label, such 
as {\it verb\/} or {\it noun}. The {\sc subj} value spells out 
whether and --if so-- what kind of subject the category selects, and 
the {\sc comps} value does the same for the selection of complements.
An intransitive verb, for instance, selects an {\sc np} as its 
subject and does not select any complements.  
The role of the {\sc marking} feature will be described in section 3.1.

The combinatorics is modeled in terms of constraints on phrase types. 
They are organized in a hierarchy, of which (\ref{he}) provides a first glimpse. 

\begin{exe}
\ex\label{he} 
\tree
   {\ntnode{Zj}{\it phrase},
     {\ntnode{Zi}{\it headed-phrase},
       {\tnode{Zg}{head-subject-phrase}},
       {\tnode{Zm}{...}}},  
     {\tnode{Zq}{non-headed-phrase}}}
\nodeconnect{Zj}{Zi}
\nodeconnect{Zj}{Zq}
\nodeconnect{Zi}{Zg}
\nodeconnect{Zi}{Zm}
\end{exe}

\noindent
The phrases are partitioned into those that have a head daughter, 
and those that do not. The former have an extra feature, 
called {\sc head-daughter}, whose value is of type 
{\it sign}.\footnote{The non-headed phrases include a.o. coordinate phrases, 
such as {\it Gilbert and George}. They lack the {\sc head-dtr} feature.} 

\begin{exe}
\ex {\it headed-phrase\/} : \begin{avm} 
                            [head-dtr ~~ {\it sign\/}]
                            \end{avm} 
\end{exe}

\noindent
A defining property of the headed phrases is that their part-of-speech 
is identified with the one of their head daughter. 
This is expressed in terms of the {\bf implicational constraint} in 
(\ref{hp}).\footnote{(\ref{hp}) employs the abbreviatory path notation: 
[{\sc synsem$|$category$|$head} ~ {\it part-of-speech\/}] is short for 
[{\sc synsem} [{\sc category} [{\sc head} ~ {\it part-of-speech\/}]]].} 

\begin{exe}
\ex\label{hp} {\bf Head Feature Principle}   \\
\footnotesize
{\it headed-phrase\/} ~ $\Rightarrow$ ~ 
\begin{avm}
[synsem|category|head ~~ @1 {\it part-of-speech\/} \\
 head-dtr|synsem|category|head ~~ @1 ]
\end{avm}
\normalsize
\end{exe}

\noindent
The boxed integer ($\avmbox{1}$) is a typed variable. 
Recurrence within the same representation expresses token-identity, 
in much the same way as the recurrence of a variable in a 
Predicate Logic formula. In this case, the part-of-speech is required to 
be the same for the mother and its head daughter. This captures the 
intuition that phrasal categories are projections of lexical 
categories.\footnote{This is reminiscent of {\sc x}-bar theory, 
but notice that the {\sc hpsg} constraint applies to surface-oriented 
structures, not to the more abstract deep structures of transformational 
grammar.} 

Of the various types of headed phrases, there is one that models 
the combination of a head and its subject. It is defined in (\ref{hsp}). 

\noindent
Phrases of this type have two daughters. The second one is the 
head daughter ($\avmbox{1}$), and requires the other daughter  
to be an {\sc np} ($\avmbox{2}$). 
The {\sc subj} value of the resulting phrase is the empty list,  
since that phrase does not require a subject anymore. 
Application to {\it Kim laughed\/} yields (\ref{metk}).\footnote{This
is a partial representation. Not included are the {\sc comps}, 
{\sc marking} and {\sc content} values.}   
 
\begin{exe}
\ex\label{metk} 
\footnotesize
\begin{avm} 
[{\it head-subject-phrase\/}                  \\
 form ~ <{\rm Kim , laughed}>          \\
 synsem|category [head ~ @3                 \\
                  subject ~ < ~ > ]                          \\
 dtrs ~ <[form ~ <{\rm Kim}>                         \\
          synsem ~ @2 [category [head ~ {\it noun\/}      \\
                                 subject ~ < ~ > ]]] ~, @1 > \\   
 head-dtr ~ @1 [form ~ <{\rm laughed}>               \\
                synsem|category [head ~ @3 {\it verb\/}   \\
                                 subject ~ <@2> ]]]            
\end{avm}
\normalsize
\end{exe}

\noindent
Being an object of type {\it head-subject-phrase}, the  
second daughter is identified as the head daughter ($\avmbox{1}$),  
the first daughter's {\sc synsem} value is token-identical to 
the {\sc subject} value of the head daughter ($\avmbox{2}$) and 
the {\sc subject} value of the combination is the empty list. 
Moreover, since {\it head-subject-phrase\/} is a subtype of {\it headed-phrase}, 
it is subsumed by the Head Feature Principle, which implies 
that it shares its {\sc head} value ($\avmbox{3}$) with the head daughter. 
For expository purposes, representations of phrasal signs are often 
converted into trees whose nodes are enriched with features, 
separated by commas, as in (\ref{bri}). 

\begin{exe}
\ex\label{bri}
\footnotesize 
\tree
{\ntnode{Zz}{[{\sc head} ~ $\avmbox{3}$ , {\sc subject} ~ $<$ $>$]},
  {\ntnode{Za}{$\avmbox{2}$ [{\sc head} ~ {\it noun} , {\sc subject} ~ $<$ $>$]},
    {\tnode{Zp}{Kim}}},
  {\ntnode{Zb}{[{\sc head} ~ $\avmbox{3}$  {\it verb} , {\sc subject} ~ $<\avmbox{2}>$]},
    {\tnode{Zq}{laughed}}}}  
\nodeconnect{Zz}{Za}
\nodeconnect{Zz}{Zb}
\nodeconnect{Za}{Zp}
\nodeconnect{Zb}{Zq}
\normalsize
\end{exe}

\noindent
It is a style of notation that we will also use in this paper. 

\bigskip

The analysis presented so far is typical of the strictly lexicalist approach
of early {\sc hpsg}, in which properties of phrases are mainly determined by 
properties of the constituent words and only to a small extent 
by properties of the combinatory operations as such. 
\citet[391]{PS94}, for instance, employs only seven types 
of combinatory operations, including the one for combining a head 
with its subject. Over time, though, the radical lexicalism gave way to an 
approach in which the properties of the combinatory operations  
play a larger role. The small inventory of highly 
abstract phrase types got replaced by a more fine-grained hierarchy  
in which the types contain more specific and ---if need be--- idiosyncratic 
constraints. This development started in \citet{Sag97}, was 
elaborated in \citet{GS00}, and gained momentum afterward,
leading to what is now known as constructivist {\sc hpsg}. 
Characteristic of this approach is the use of a 
{\bf bi-dimensional hierarchy} of phrasal signs. 
In such a hierarchy the phrases are not only partitioned in terms of 
{\sc headedness}, but also in terms of a second dimension, called  
{\sc clausality}, as in (\ref{bidim}). 

\begin{exe}
\ex\label{bidim} 
\footnotesize
\tree
    {\ntnode{Za}{\it phrase},
      {\ntnode{Zh}{{\sc headedness}}, 
        {\ntnode{Zi}{\it headed-phrase},
          {\tnode{Zv}{head-subject-phrase}}, 
          {\tnode{Zw}{...}}}, 
        {\tnode{Zq}{...}}}, 
      {\ntnode{Zl}{{\sc clausality}},
        {\ntnode{Zg}{\it clause},
          {\tnode{Zs}{declarative-clause}}, 
          {\tnode{Zu}{..}}}, 
        {\tnode{Zm}{non-clause}}}}
\nodeconnect{Za}{Zh}
\nodeconnect{Zh}{Zi}
\nodeconnect{Zi}{Zv}
\nodeconnect{Zi}{Zw}
\nodeconnect{Zh}{Zq}
\nodeconnect{Za}{Zl}
\nodeconnect{Zl}{Zg}
\nodeconnect{Zg}{Zs}
\nodeconnect{Zg}{Zu}
\nodeconnect{Zl}{Zm}
\normalsize
\end{exe}

\noindent
At the upper level the {\sc clausality} hierarchy 
differentiates clauses from non-clauses. 
Just like the types in the {\sc headedness} hierarchy, 
they are associated with implicational constraints. 
Clauses, for instance, are required to denote an object 
of type {\it message\/} \citep[41]{GS00}.

\begin{exe}
\ex {\bf Clause}   \\ 
\footnotesize
{\it clause\/} ~ $\Rightarrow$ ~ \begin{avm} 
                                 [synsem|content ~~ {\it message\/}] 
                                 \end{avm}
\normalsize
\end{exe}

\noindent
At a more fine-grained level, the clauses are partitioned into 
declarative, interrogative, imperative and exclamative clauses, 
each with their own constraints. 
Interrogative clauses, for instance, denote a question, 
while indicative declarative clauses denote a proposition.\footnote{The 
structure and the role of the semantic objects 
is spelled out in more detail in section 3.2.} 

Since the {\sc headedness} and {\sc clausality} dimensions are mutually independent,
it is possible to define types whose properties are partly inherited from 
a type in the first dimension and partly from a type in the second dimension. 
This is called {\bf multiple inheritance}. It is illustrated in (\ref{bidi}). 

The type {\it declarative-head-subject-clause\/} inherits 
the properties of {\it head-subject-phrase}, on the one hand, and 
{\it declarative-clause}, on the other hand. Besides, it may 
have properties of its own, such as the fact that its head daughter 
is a finite non-inverted verb \citep[43]{GS00}. 
It is this combination of type-specific 
implicational constraints and multiple inheritance 
that we will use to capture the subtle interaction of the 
regular and the idiosyncratic in the formation of nominal phrases.  


\section{Regular nominals} 


This section provides a bi-dimensional treatment of regular nominals. 
We first situate them in the {\sc headedness} hierarchy (3.1), 
and then in the {\sc clausality} hierarchy (3.2).   
For the former we build on the functor treatment of the noun phrase, 
as pioneered in \citet{VanEynde98a} and \citet{Allegranza98} 
and further developed in \citet{VanEynde06} and  
\citet{Allegranza07}.\footnote{The functor treatment is 
taken on board in Sign-Based Construction Grammar, see \citet{Sag2012} 
and \citet{KaySag12}.} For the latter we are breaking new ground. 
The combination of the two dimensions is modeled in terms of multiple 
inheritance (3.3). 


\subsection{A hierarchy of headed phrases} 


In a first step we extend the hierarchy of headed phrases that was introduced 
in section 2. 

\noindent
To motivate why the propagation of the {\sc marking} value is different 
in head-argument combinations, we take a prenominal adjective with a complement. 
Since that combination is rare (if not impossible) in English, 
we choose a German example: {\it mir unbekannte Frauen\/} `women unknown to me'.  

\begin{exe}
\ex\label{frau} 
\footnotesize
\tree
  {\ntnode{Za}{[{\sc head} ~ $\avmbox{1}$ {\it noun\/} , {\sc mark} ~ $\avmbox{2}$ {\it unmarked\/}]},
    {\ntnode{Zd}{[{\sc head} ~ $\avmbox{3}$ {\it adj\/} , {\sc comps} ~ $<$ $>$ , {\sc mark} ~ $\avmbox{2}$]},
      {\ntnode{Ze}{$\avmbox{4}$ [{\sc mark} ~ {\it marked\/}]},    
        {\tnode{Zl}{mir}}},
      {\ntnode{Zg}{[{\sc head} ~ $\avmbox{3}$ , {\sc comps} ~ $<\avmbox{4}>$ , {\sc mark} ~ $\avmbox{2}$]},
        {\tnode{Zi}{unbekannte}}}},
    {\ntnode{Zk}{[{\sc head} ~ $\avmbox{1}$]},
      {\tnode{Zh}{Frauen}}}}
\nodeconnect{Za}{Zd}
\nodeconnect{Zd}{Ze}
\nodeconnect{Za}{Zk}
\nodeconnect{Ze}{Zl}
\nodeconnect{Zd}{Zg}
\nodeconnect{Zg}{Zi}
\nodeconnect{Zk}{Zh}
\normalsize
\end{exe}

\noindent
The nominal shares its {\sc head} value with the noun {\it Frauen\/} 
($\avmbox{1}$) and its {\sc marking} value with the {\sc ap} 
{\it mir unbekannte\/} ($\avmbox{2}$). 
The latter in turn shares its {\sc head\/} value with the adjective 
($\avmbox{3}$), and since {\it mir\/} is an argument of the adjective, 
the {\sc ap} also shares its {\sc marking} value
with the adjective ($\avmbox{2}$), as required by (\ref{mark2}). 
If we had not differentiated the head-argument from the head-nonargument 
phrases, the inherently marked pronoun {\it mir\/} would share 
its {\sc marking} value with the {\sc ap} and hence with the nominal, 
yielding a marked {\sc np}, which is undesirable since it erroneously 
predicts that the nominal cannot be preceded by a determiner.   


\subsubsection{Functors and independents} 


Functors are non-arguments which lexically select their head sister. 
To model this we use the feature {\sc select}. 
It is assigned to objects of type {\it part-of-speech}, which 
implies that it forms part of the {\sc head} values. Its 
value is of type {\it synsem\/} or {\it none}.\footnote{The 
{\sc select} feature replaces the {\sc mod(ified)} and 
{\sc spec(ified)} features of \citet{PS94}. 
The reasons for including it in the {\sc head} values and for 
allowing the value {\it none\/} are given further down.}

\begin{exe}
\ex  {\it part-of-speech\/} : \begin{avm} 
                              [select ~~ {\it synsem-or-none\/}]
                              \end{avm}
\end{exe}

\noindent
The match of the selection requirements with the properties of the 
head sister is modeled in the constraint on head-functor 
phrases.\footnote{As compared to \citet{PS94}, 
the constraint in (\ref{hefu}) replaces the Spec Principle, as well as  
the constraint that the {\sc mod} value of an adjunct be token-identical 
to the {\sc synsem} value of its head sister.}


In combination with the {\sc marking} device, the {\sc select} feature 
can be used to model co-occurrence restrictions. 
Determiners, for instance, can be said to select an unmarked nominal 
as their head sister, thus accounting for the well-formedness of 
{\it the (long) bridges}, as well as for the ill-formedness of 
*{\it some the bridges\/} and 
*{\it the those bridges}.\footnote{A {\it JL} reviewer remarks that 
this is a mere description of the facts and
suggests that a semantic motivation should be added. We are
skeptical about this, since the distinction between marked 
and unmarked nominals shows a fair degree of arbitrariness. 
Notice, for instance, that nominals with a possessive determiner 
are marked in English but unmarked in Italian, 
{\it (*the) your friend\/} vs. {\it *(il) tuo amico}. 
Similarly, nominals with a demonstrative determiner are 
marked in English but unmarked in modern Greek. It is in fact
for this reason that the {\sc marking} feature is part of the syntactic 
feature bundle. Semantically motivated distinctions are made in the 
{\sc content} values.}  
It can also be used to model agreement relations. 
The plural feminine forms of Italian adjectives, for instance, 
can be said to select a plural feminine nominal, 
thus accounting for the well-formedness of 
{\it strade pericolose\/} `dangerous streets', as well as for the 
ill-formedness of *{\it strada pericolose\/} and 
*{\it cani pericolose\/} `dangerous dogs', in which the plural
feminine adjective is combined with respectively a 
singular feminine noun and a plural masculine noun.    

The reason why the {\sc select} feature is included in the 
{\sc head} value of the non-head daughter is that the selection 
properties of a phrasal functor are shared with those of its 
head daughter: the Italian {\sc ap} {\it molto pericolose\/} 
`very dangerous', for instance, requires a plural feminine nominal, 
because {\it pericolose\/} requires a plural feminine nominal. 

To illustrate the interaction between marking and selection we take 
again {\it the long bridges}.  

\begin{exe}
\ex\label{bridd} 
\footnotesize
\tree
{\ntnode{Zz}{[{\sc hd} ~ $\avmbox{1}$ {\it noun} , {\sc mark} ~ $\avmbox{3}$ {\it marked\/}]},
  {\ntnode{Za}{[{\sc hd$|$sel} ~ $\avmbox{4}$ , {\sc mark} ~ $\avmbox{3}$]},
    {\tnode{Zp}{the}}},
  {\ntnode{Zb}{$\avmbox{4}$ [{\sc hd} ~ $\avmbox{1}$ , {\sc mark} ~ $\avmbox{2}$ {\it unmarked\/}]},
    {\ntnode{Zc}{[{\sc hd$|$sel} ~ $\avmbox{5}$ , {\sc mark} ~ $\avmbox{2}$]},
      {\tnode{Zq}{long}}},  
    {\ntnode{Zg}{$\avmbox{5}$ [{\sc hd} ~ $\avmbox{1}$ , {\sc mark} ~ {\it unm\/}]},
      {\tnode{Zh}{bridges}}}}}
\nodeconnect{Zz}{Za}
\nodeconnect{Zz}{Zb}
\nodeconnect{Za}{Zp}
\nodeconnect{Zb}{Zc}
\nodeconnect{Zc}{Zq}
\nodeconnect{Zb}{Zg}
\nodeconnect{Zg}{Zh}
\normalsize
\end{exe}

\noindent
The adjective selects an unmarked nominal ($\avmbox{5}$) and shares its 
{\sc marking} value ($\avmbox{2}$) with the mother, yielding the unmarked 
nominal {\it long bridges}. The determiner then selects an unmarked nominal 
($\avmbox{4}$) and shares its {\sc marking} value ($\avmbox{3}$) with the phrase,
yielding the marked nominal {\it those long bridges}. 
Given the selection requirements of prenominal adjectives and the determiners, 
the latter is not compatible with another adjective or determiner. 
This correctly blocks the formation of such ill-formed combinations as 
*{\it wooden the (long) bridges\/} and *{\it our the (long) bridges}. 

\bigskip

Signs which do not select their head sister have the {\sc select} 
value {\it none}. This value is assigned to the non-head daughter
in {\it head-complement\/} phrases. The {\sc np} complement of the 
verb in {\it built a long bridge}, for instance, does not select 
its head sister. Instead, it is the verb that selects the {\sc np}.  
The latter's {\sc select} value is, hence, {\it none}, and given 
the Head Feature Principle this value is shared with the noun {\it bridge}. 

It will play a role in our treatment of the {\sc bmc} and the {\sc bnpc}. 


\subsubsection{Summing up} 


Noun phrases are projections of nouns in which prenominal adjectives and 
determiners are functors: They lexically select their head sister, 
as spelled out in the constraint on head-functor phrases (\ref{hefu}), 
and leave their mark on the combination, as spelled out 
in the constraint on head-nonargument phrases (\ref{mark1}). 
More elaborate analyses along these lines with applications to 
other languages and with special attention to {\sc np}-internal
agreement are presented in \citet{VanEynde06} and 
\citet{Allegranza07}. It should be added, though, that 
the present analysis differs from the earlier treatments 
in one important respect: We do not treat the selection 
and the marking as pertaining to the same phrase type, but 
separate them, associating the marking with all head-nonargument phrases,
and the selection with a subtype of them, i.e. the head-functor phrases. 
The relevance of this differentiation will become clear in sections 4 and 5. 

 
\subsection{A hierarchy of non-clausal phrases} 


While the {\sc headedness} types are defined in cross-categorial 
terms and mainly concern syntactic properties, the 
{\sc clausality} types concern category-specific and semantic 
properties. Taking a cue from the treatment of 
clausal phrase types in \citet{GS00}, we develop a partial 
hierarchy for the nominal phrases. As a starting point we take  
the hierarchy of semantic objects in (\ref{sem}), 
an abbreviated version of \citet[386]{GS00}. 

\begin{exe}
\ex\label{sem}
\scriptsize
\tree
{\ntnode{Zt}{\it semantic-object}, 
  {\ntnode{Zd}{\it message}, 
    {\tnode{Za}{proposition}},
    {\tnode{Zg}{fact}},
    {\tnode{Zh}{...}}},
  {\tnode{Ze}{state-of-affairs}},
  {\ntnode{Zr}{\it scope-object},
    {\tnode{Zb}{parameter}},
    {\tnode{Zc}{quant-rel}}},
  {\tnode{Zf}{relation}},
  {\tnode{Zp}{index}}}
\nodeconnect{Zt}{Zd}
\nodeconnect{Zd}{Za}
\nodeconnect{Zd}{Zg}
\nodeconnect{Zd}{Zh}
\nodeconnect{Zt}{Ze}
\nodeconnect{Zt}{Zr}
\nodeconnect{Zr}{Zb}
\nodeconnect{Zr}{Zc}
\nodeconnect{Zt}{Zf}
\nodeconnect{Zt}{Zp}
\normalsize
\end{exe}

\noindent
Some of these types have already been mentioned in section 2, 
such as {\it message, proposition, state-of-affairs\/} and 
{\it scope-object}. For our purpose, it is mainly the latter 
that matters, since it is the prototypical {\sc content} value 
of a nominal. It consists of an index and a set of 
facts which impose restrictions on the index \citep[122]{GS00}.  

\begin{exe}
\ex\label{sco} {\it scope-object\/} : \begin{avm} 
                           [index ~ {\it index\/}           \\
                            restriction ~ {\it set\/} \({\it fact\/}\)]    
                           \end{avm}
\end{exe}

\noindent
The common noun {\it box}, for instance, denotes the set of those 
entities for which it is a fact that they are a box. The type is also 
relevant for the adjectives. {\it Red}, for instance, denotes the set 
of those entities for which it is a fact that they are red.

At a finer-grained level \citet[135--136]{GS00} differentiates 
between quantified and non-quantified scope-objects. 
The former are of type {\it quant-rel}. They include {\sc np}s which are 
introduced by a quantifying determiner, such as {\it every, each\/} or
{\it some}, and to quantifying pronouns, such as {\it everybody\/} 
and {\it someone}. The nominals which lack quantificational force are 
of type {\it parameter}. They include bare nominals, proper nouns, 
various types of pronouns and {\sc np}s which are introduced by a 
non-quantifying determiner. It is not always immediately clear whether 
a nominal is quantified or not. \citet{GS00}, for instance, 
devotes a whole chapter to the issue of whether interrogative 
{\it wh\/}-phrases, such as {\it who, what\/} and {\it which table},  
are quantified, concluding that they are not (chapter 4). 
For our purpose, it is the nominal phrases which denote a parameter that 
matter. They are subsumed by a subtype of the non-clausal phrases, which we 
call {\it nominal-parameter}. Its properties are spelled out in (\ref{param}).   

\subsection{Multiple inheritance} 




\section{The Big Mess Construction} 


Some examples of the {\sc bmc} are given in (\ref{bime}), repeated in (\ref{bime5}).

\begin{exe}
\ex\label{bime5}
\begin{xlist}
\ex   It's {\it so good a bargain\/} I can't resist buying it.
\ex   {\it How serious a problem\/} is this?
\end{xlist}
\end{exe} 

\noindent
The italicized phrases are {\sc np}s.

Taking stock, the {\sc bmc} is an {\sc np} that consists of 
a marked {\sc ap} and a marked {\sc np}, and the internal structure 
of these two constituents is unexceptional: They  
are both instances of the head-functor type of combination. 
What makes the {\sc bmc} exceptional is the   
combination of the pre-determiner {\sc ap} and the indefinite {\sc np}.  
It will be discussed and analysed in section 4.1. 
The variant of the {\sc bmc} with {\it of\/} is treated in section 4.2. 
A comparison with an alternative treatment is provided in section 4.3 
and a summary in section 4.4.   


\subsection{The canonical BMC}  


To model the combination of the {\sc ap} and the indefinite {\sc np}  
we add a type to the hierarchy of phrases, called {\it big-mess-phrase}, 
which is a subtype of {\it head-independent-phrase}, on the one hand, and  
{\it restrictive-modification}, on the other hand.


\subsubsection{Inherited properties} 


Being a subtype of {\it head-nonargument-phrase}, the {\sc bmc} has a 
head daughter with which it shares its {\sc head} value ($\avmbox{1}$) 
and a non-head daughter with which it shares its {\sc marking} value 
($\avmbox{2}$). These are in turn shared with the head daughter 
of the {\sc np} and the functor daughter of the {\sc ap}, respectively.  

\begin{exe}
\ex\label{ver}
\footnotesize
\tree
  {\ntnode{Za}{[{\sc head} ~ $\avmbox{1}$ {\it noun\/} , {\sc marking} ~ $\avmbox{2}$ {\it marked\/}]},
    {\ntnode{Zd}{[{\sc head} ~ $\avmbox{3}$ {\it adj\/} , {\sc marking} ~ $\avmbox{2}$]},
      {\ntnode{Ze}{[{\sc marking} ~ $\avmbox{2}$]},    
        {\tnode{Zl}{how}}},
      {\ntnode{Zg}{[{\sc head} ~ $\avmbox{3}$]},
        {\tnode{Zi}{serious}}}},
    {\ntnode{Zk}{[{\sc head} ~ $\avmbox{1}$ , {\sc marking} ~ $\avmbox{4}$ {\it marked\/}]},
      {\ntnode{Zh}{[{\sc marking} ~ $\avmbox{4}$]},    
        {\tnode{Zm}{a}}},
      {\ntnode{Zn}{[{\sc head} ~ $\avmbox{1}$]},
        {\tnode{Zo}{problem}}}}}
\nodeconnect{Za}{Zd}
\nodeconnect{Zd}{Ze}
\nodeconnect{Ze}{Zl}
\nodeconnect{Zd}{Zg}
\nodeconnect{Zg}{Zi}
\nodeconnect{Za}{Zk}
\nodeconnect{Zk}{Zh}
\nodeconnect{Zh}{Zm}
\nodeconnect{Zk}{Zn}
\nodeconnect{Zn}{Zo}
\normalsize
\end{exe}


\bigskip
 
Being a subtype of {\it restrictive-modification}, 
the {\sc bmc} denotes a parameter in which the daughters 
share their index. In this respect, it resembles regular nominals, 
such as {\it red box}. Evidence for treating the {\sc bmc} along 
the same lines is provided by the fact that some of 
the {\sc ap}s with a degree marker are both used 
in the idiosyncratic pre-determiner position and in the canonical
post-determiner position. This is illustrated for {\sc ap}s with 
{\it more, less\/} and {\it enough\/} in 
(\ref{serious}--\ref{enou}).

\noindent
As a consequence, since the instances with the post-determiner order are 
an instance of restrictive modification, and since there 
is no difference in truth-conditional meaning with the instances of 
the pre-determiner order, it is only natural to treat the latter as instances 
of restrictive modification as well. 
Notice, incidentally, that this provides further 
evidence for the {\sc drt} style treatment of the indefinite article as lacking
quantificational force, for if the article were treated as an existential 
quantifier, the {\sc bmc} would not comply with the constraints on 
{\it restrictive-modification}, which in turn would necessitate an
otherwise unmotivated differentiation between the (a) and (b) examples in 
(\ref{serious}--\ref{enou}). 


\subsubsection{Inherent properties}


Beside the properties which it inherits from its supertypes, the {\sc bmc}
has some properties of its own. They are spelled out in (\ref{bigmess2}).  

\subsection{The variant with {\it of\/}} 


Some examples of the variant with {\it of\/} are given in (\ref{of77}).

\begin{exe}
\ex\label{of77}
\begin{xlist}
\ex   He took {\it so big of a piece\/} that he couldn't finish it.
\ex   It was a judgment question as to {\it how big of a risk\/} it was.   
\end{xlist}
\end{exe}

\noindent
This construction is mainly used in American English, as pointed out 
in \citet[136]{Bolinger72}, \citet[113--114]{Zwicky95}, 
\citet[125--126]{Kennedy00}, and \citet[348]{Fillmoreetal12}. 
It provides an extra challenge, for if we adopt the canonical analysis 
of a [{\sc p}--{\sc np}] sequence, we get a {\sc pp} that is combined 
with an {\sc ap} to yield an {\sc np}! An analysis along these lines 
is actually proposed in \citet[117]{Zwicky95} and 
\citet[357]{KimSells11}, but it is not the one which we adopt. 

Instead, we assume that the [{\it of\/}--{\sc np}] sequence is headed 
by the {\sc np}. The preposition is, hence, not treated as the complement 
selecting head of a {\sc pp} but as a head selecting functor, more 
specifically as a functor which selects an {\sc np} with the {\sc dtype} value 
{\it a}. Its own {\sc dtype} value captures the fact that we 
are dealing with a use of {\it of\/} in which it is necessarily 
followed by the indefinite article.
We call it {\it ofa\/} for mnemonic reasons. It is a subtype of 
{\it indefinite}, just like {\it a}.
The {\sc synsem} value in (\ref{of4}) spells this out in formal terms.

\begin{exe}
\ex\label{of4}
\footnotesize
\begin{avm} 
[category [head [{\it preposition\/}                \\
                 select|category|marking|dtype ~ {\it a\/}] \\
           marking|dtype ~ {\it ofa\/}]                \\
 content|restriction ~~ \{  \}]
\end{avm}
\normalsize
\end{exe}

\noindent
The assignment of the empty set as its {\sc restr} value
captures the semantic vacuity of the preposition. 

As a consequence, the combination of the preposition with the {\sc np} 
can be treated as an instance of the {\it regular-nominal\/} type: 
In terms of the {\sc headedness} dimension it is an instance of
{\it head-functor-phrase\/} and in terms of the {\sc clausality} dimension
it is an instance of {\it restrictive-modification}, albeit with 
the proviso that the preposition does not contribute any restrictions. 

The resulting nominal matches nearly all of the constraints on the head 
daughter of the big-mess phrase, as spelled out in (\ref{bigmess2}). 
The only modification we need is to make the {\sc dtype} value 
of the head daughter in (\ref{bigmess2}) disjunctive, 
allowing both {\it a\/} and {\it ofa}. 
This suffices to license combinations as in (\ref{bigof}). 

\begin{exe}
\ex\label{bigof}
\footnotesize
\tree
    {\ntnode{Zh}{[{\sc hd} ~ $\avmbox{3}$ {\it noun} , {\sc marking} ~ $\avmbox{2}$]},
      {\ntnode{Zm}{[{\sc marking} ~ $\avmbox{2}$ {\it marked\/}]},
        {\tnode{Zr}{how big}}},  
      {\ntnode{Zg}{[{\sc hd} ~ $\avmbox{3}$ , {\sc dtype} ~ $\avmbox{1}$ {\it ofa\/}]},
        {\ntnode{Zx}{[{\sc hd$|$sel} ~ $\avmbox{4}$ , {\sc dtype} ~ $\avmbox{1}$]},
          {\tnode{Zy}{of}}},
        {\ntnode{Zz}{$\avmbox{4}$ [{\sc hd} ~ $\avmbox{3}$ , {\sc dtype} ~ {\it a\/}]},
          {\tnode{Zi}{a risk}}}}}
\nodeconnect{Zh}{Zm}
\nodetriangle{Zm}{Zr}
\nodeconnect{Zh}{Zg}
\nodeconnect{Zg}{Zx}
\nodeconnect{Zx}{Zy}
\nodeconnect{Zg}{Zz}
\nodetriangle{Zz}{Zi}
\normalsize
\end{exe}

Independent evidence for the functor treatment is provided by the 
stranding test. In languages which allow adposition stranding, such 
as English and Dutch, adpositions may be stranded if they head 
a {\sc pp}, as in (\ref{afr1}). 

\begin{exe}
\ex\label{afr1}  
\begin{xlist} 
\ex  Everybody wants a copy of that book. 
\ex  That is a book that everybody wants a copy of \_\_.    
\end{xlist} 
\end{exe}  

\noindent
This is just a regular instance of complement extraction. 
The preposition in the {\sc bmc}, by contrast, cannot be stranded.

\begin{exe} 
\ex\label{afr2}
\begin{xlist} 
\ex    I never saw so disgusting of a movie. 
\ex    [*] {That is a movie that I never saw so disgusting of \_\_.}
\end{xlist} 
\end{exe}

\noindent
This meshes well with its functor status, since heads cannot be 
freely extracted. 

Further evidence is provided by the fact that the functor treatment 
also applies to the use of {\it of\/} in the Binominal {\sc np} Construction.  
Moreover, it has been argued to apply to certain uses of the 
Dutch prepositions {\it om, te, van\/} and {\it voor\/} in 
\citet{VanEynde04}. Also for certain uses of the French 
prepositions {\it \`a\/} and {\it de\/} it has been argued that 
they are markers, rather than heads of {\sc pp}s
\citep{AbeilleGodard2000}.\footnote{Markers share many properties of 
functors. The main difference is that markers are required to belong 
to a functional closed class category, whereas functors can belong to 
any part of speech.}  


\subsection{Comparison with another treatment} 


Another monostratal analysis of the {\sc bmc} is provided in \citet{KaySag12}. 
It is cast in the notation of Sign-Based Construction Grammar, which is very
similar to that of constructivist {\sc hpsg}. 
It employs the head-functor type of combination to model both the lower and the 
higher {\sc np} of the {\sc bmc}, and resorts to a special 
type of construction to model the pre-determiner {\sc ap}. 
It is called the {\it complex pre-determiner construction\/} and 
its main idiosyncracy is that it     
triggers a change in the {\sc select} value of the adjective:
While an adjective such as {\it long\/} selects an unmarked nominal, 
the combination {\it how long\/} selects a marked nominal that is 
introduced by {\it a\/} \citep[238]{KaySag12}. 
In support of this choice \citet{KaySag12} points out  
that it generalizes to other indefinite {\sc np}s with a pre-determiner, 
such as {\it what a mess}, {\it such a beauty\/} and {\it many a day}. 
Assuming that {\it what, such\/} and {\it many\/} are functors 
that select an indefinite {\sc np} and that have some other {\sc marking} 
value themselves, it is tempting to analyze the {\sc bmc} along the same lines. 

While we agree that it is never good to miss a generalization,
we are not convinced that the {\sc bmc} is so similar to
the combinations with {\it what, such\/} and {\it many\/} that a 
uniform treatment is called for. In fact, there are at least three 
differences between them.    
First, in the combinations with {\it what, such\/} and {\it many\/} 
the adjectives (if any) follow rather than precede the determiner. 

\begin{exe}
\ex
\begin{xlist}
\ex   What a big mess it was!
\ex   It was such a big mess!
\ex   She spent many a boring day at the office.
\end{xlist}
\ex
\begin{xlist}
\ex   [*] {What big a mess it was!}
\ex   [*] {It was such big a mess!}
\ex   [*] {She spent many boring a day at the office.}
\end{xlist}
\end{exe}

\noindent
Second, {\it what\/} and {\it such\/} are not only compatible with 
{\sc np}s that are introduced by the indefinite article, 
but also with unmarked nominals.  

\begin{exe}
\ex 
\begin{xlist}
\ex   What promise she had shown! 
\ex   What fools they are!
\end{xlist}
\ex
\begin{xlist}
\ex   Would you yourself follow such advice?  
\ex   What is one to make of such statements? 
\end{xlist}
\end{exe}

\noindent
Third, the pre-determiners may not be separated from the {\sc np} 
by {\it of\/} : {\it what (*of) a mess}, {\it such (*of) a beauty\/} 
and {\it many (*of) a splendid day at the beach}.  
These differences make the adoption of a uniform treatment less appealing. 

Conversely, the shift of the idiosyncracy to the 
internal structure of the {\sc ap} leads to a  
loss of generalization in the combination of the 
degree marker with the adjective, for while the
addition of the degree marker changes the {\sc select} 
value of the adjective in the {\sc bmc}, it does not trigger 
such a change when the {\sc ap} is used in other positions,
such as the postnominal and predicative ones in (\ref{bimepr2}).  

\begin{exe}
\ex\label{bimepr2}
\begin{xlist}
\ex   Never before had we seen a bridge that long.
\ex   This beer is so good !
\end{xlist}
\end{exe}

\noindent
Our treatment is more general in this respect, since it treats 
the internal structure of the {\sc ap} in a uniform way, no matter 
whether it is in predicative, postnominal or pre-determiner position.  

On balance then, we think that a treatment which locates the idiosyncracy 
of the {\sc bmc} in the combination of the pre-determiner {\sc ap} with 
the {\sc np} is preferable to one which locates it in the internal 
structure of the {\sc ap}. 


\subsection{Summing up}


This section has provided an analysis of the Big Mess Construction, 
both the canonical one and the variant with {\it of}.  
It is based on the assumption that the internal structure of the 
pre-determiner {\sc ap} and the indefinite {\sc np} are 
unexceptional and that the idiosyncracy resides in their combination. 
This combination is modeled in terms of a type, called {\it big-mess-phrase},
that inherits properties of {\it head-independent-phrase}, on the one hand, 
and {\it restrictive-modification}, on the other hand. The result of 
unifying the inherited constraints with the inherent ones is spelled out 
in (\ref{bigs2}). 

\begin{exe}
\ex\label{bigs2} 
\footnotesize
\begin{avm}
[{\it big-mess-phrase\/}                                         \\
 synsem [category [head ~ @1 {\it noun\/}                        \\
                   subject ~ @A                                  \\
                   comps ~ @B                                    \\
                   marking ~ @2 {\it marked\/}]                  \\
         content [{\it parameter\/}                              \\
                  index ~ @3 {\it index\/}                       \\
                  restriction ~ $\Sigma_{1}$ ~ $\bigcup$ ~ $\Sigma_{2}$]] \\
 dtrs ~ <[{\it head-functor-phrase\/}                            \\
          synsem [category [head [{\it adjective\/}              \\
                                  select ~ {\it none\/}]         \\ 
                            marking ~ @2 ]                          \\
                  content [index ~ @3                          \\
                           restriction ~ $\Sigma_{1}$ ]]] ~, @4>      \\
 head-dtr ~ @4 [{\it regular-nominal\/}                     \\
                synsem [category [head ~ @1                      \\
                                  subject ~ @A                      \\
                                  comps ~ @B                     \\
                                  marking [{\it marked\/}           \\
                                           dtype ~ {\it a\/} \($\vee$ {\it ofa\/}\)]] \\                   
                        content [{\it parameter\/}             \\
                                 index ~ @3                    \\
                                 restriction ~ $\Sigma_{2}$ ]]]]
\end{avm}
\normalsize
\end{exe}

Comparing this to the description of the regular nominals in section 3.3
the main difference concerns the proportion of inherited and inherent 
constraints. While the properties of the regular nominals are 
all inherited from its supertypes, many of the properties of 
the {\sc bmc} concern inherent constraints.


\section{The Binominal Noun Phrase Construction}


Some examples of the {\sc bnpc} are given in (\ref{climb}), repeated in (\ref{climb2}). 

\begin{exe}
\ex\label{climb2}
\begin{xlist}
\ex  She blames it on {\it her nitwit of a husband}. 
\ex\label{skull}  She had {\it a skullcracker of a headache}. 
\end{xlist}
\end{exe}

\noindent
For its analysis we subscribe to the view, argued for at length in
\citet{Aarts98} and \citet[85--108]{Keizer07}, that the 
rightmost {\sc np} is both the syntactic and semantic head: 
What she is claimed to have in (\ref{skull}),
for instance, is a headache, rather than a skullcracker. 
A challenge for that view is that it is not 
in sync with what the surface structure of the combination suggests: 
In an [{\sc n}--{\it of\/}--{\sc np}] sequence, it is the 
leftmost nominal that is canonically identified as the head. 
To solve the mismatch \citet{Aarts98} and 
\citet{Keizer07} treat the first nominal as part 
of a Modifier Phrase ({\sc mp}) that also includes {\it of\/} and the 
indefinite article, as in (\ref{aarts}). 

\begin{exe}
\ex\label{aarts}
%\footnotesize
\tree
{\ntnode{Zx}{NP},
  {\ntnode{Zy}{Det},
    {\tnode{Zz}{her}}},
  {\ntnode{Za}{N'},
    {\ntnode{Zd}{MP},
      {\tnode{Zl}{nitwit of a}}},
    {\ntnode{Zb}{N'},
    {\ntnode{Zk}{N},
      {\tnode{Zh}{husband}}}}}}
\nodeconnect{Zx}{Zy}
\nodeconnect{Zy}{Zz}
\nodeconnect{Zx}{Za}
\nodeconnect{Za}{Zd}
\nodetriangle{Zd}{Zl}
\nodeconnect{Za}{Zb}
\nodeconnect{Zb}{Zk}
\nodeconnect{Zk}{Zh}
%\normalsize
\end{exe}

\noindent
To motivate this the authors point out that the preposition 
and the article can be incorporated in the noun, 
as in {\it a helluva job}, and that the semantic contribution of 
the {\sc mp} can be paraphrased by a prenominal {\sc ap},
as in {\it a hellish job}. 

A problem with these arguments is that they do not generalize. 
Incorporation, for instance, is possible with {\it hell}, but not with 
other nouns: Combinations like *{\it her nitwituva husband\/} and 
*{\it a skullcrackeruva headache\/} do not exist. 
Similarly, the paraphrase in terms of a prenominal adjective is 
possible with nouns like {\it hell\/} and {\it fool}, but 
not with nouns like {\it nitwit\/} and {\it skullcracker}, 
at least if one expects a morphological relation between the noun 
and the adjective. 

Besides, and more importantly, it is not clear that incorporation
is a good criterion for constituency. The fact, for instance, that 
the infinitival {\it to\/} can be incorporated in certain verbs, 
as in {\it wanna\/} and {\it gonna}, then becomes 
an argument for assigning a left branching structure to 
[{\sc v}--{\it to\/}--{\sc vp}] sequences, as in  
{\it [[like to] swim]\/}, contrary to the wide-spread and independently
motivated practice of assigning it a right branching structure, 
as in {\it [like [to swim]]}.  
Similarly, the fact that the definite article is incorporated 
in certain prepositions, as in the German {\it am, zum\/} 
and {\it zur} and the French {\it du\/} and {\it au}, 
then becomes an argument for assigning a left branching 
structure to [{\sc p}--{\sc det\/}--{\sc nom}] sequences, 
as in {\it [[auf dem] Tisch]\/} `on the table' and 
{\it [[\`a la] maison]\/} `at home', contrary to the 
wide-spread and independently motivated practice of assigning 
them a right branching structure, as in {\it [auf [dem Tisch]]} and 
{\it [\`a [la maison]]\/}.    
Similar questions apply to the relevance of paraphrase relations for 
determining constituency. The fact, for instance, that 
the Dutch transitive verb {\it beluisteren\/} is nearly 
synonymous with {\it luisteren naar\/} `listen to' then becomes an argument
for treating the latter as a constituent in {\it [[luisteren naar] Mozarts laatste opera]\/} 
`listen to Mozart's last opera', contrary to the wide-spread practice 
of treating the preposition as part of the {\sc pp} complement of the verb, as in 
{\it [luisteren [naar Mozart's laatste opera]]}.\footnote{A {\it JL\/}
reviewer points out that {\it beluisteren\/} and {\it luisteren naar\/}
are not interchangeable in all contexts. That is true, but notice that  
the same remark applies to {\it hellish\/} and {\it hell of a}.}
    
In sum, the arguments in favor of (\ref{aarts}) are less than convincing. 
Besides, there are some arguments against it. \citet[14]{KimSells14}, 
for instance, points out that the treatment of the [{\sc n}--{\it of a\/}] 
sequence as a constituent complicates the expression of the constraint 
that the rightmost nominal must be a count noun, since the article is not 
a sister of that nominal. 

\bigskip

In order to avoid the problems with (\ref{aarts}) without giving up
the intuition that the rightmost nominal is the head, we adopt a 
structure which is isomorphic to the canonical structure of 
an [{\sc n}--{\it of\/}--{\sc np}] sequence, as in (\ref{napol}), 
but with the stipulation that the preposition is a functor, rather 
than the head of a {\sc pp}.    

\begin{exe}
\ex\label{napol}
%\footnotesize
\tree
{\ntnode{Zx}{NP},
  {\ntnode{Zy}{Det},
    {\tnode{Zz}{her}}},
  {\ntnode{Za}{N'},
    {\ntnode{Zd}{N},
      {\tnode{Zl}{nitwit}}},
    {\ntnode{Zw}{NP},
      {\ntnode{Ze}{P},
        {\tnode{Zm}{of}}},
      {\ntnode{Zk}{NP},
        {\ntnode{Zr}{Det},
          {\tnode{Zs}{a}}},
        {\ntnode{Zh}{N},
          {\tnode{Zi}{husband}}}}}}}
\nodeconnect{Zx}{Zy}
\nodeconnect{Zy}{Zz}
\nodeconnect{Zx}{Za}
\nodeconnect{Za}{Zd}
\nodeconnect{Zd}{Zl}
\nodeconnect{Za}{Zw}
\nodeconnect{Zw}{Ze}
\nodeconnect{Ze}{Zm}
\nodeconnect{Zw}{Zk}
\nodeconnect{Zk}{Zr}
\nodeconnect{Zr}{Zs}
\nodeconnect{Zk}{Zh}
\nodeconnect{Zh}{Zi}
%\normalsize
\end{exe}

\noindent
In fact, the properties of the preposition in the {\sc bnpc}  
are the same as those of the preposition in the {\sc bmc}, as 
spelled out in (\ref{of4}).\footnote{The similarity between the 
{\it of\/} of the {\sc bmc} and the {\it of\/} of the {\sc bnpc} 
is also pointed out in \citet[126]{Kennedy00}.}
Corroborating evidence for the functor treatment is provided by 
the contrast in (\ref{skul}).  

\begin{exe}
\ex\label{skul}
\begin{xlist}
\ex\label{sku7}  That is a book that everybody wants a copy of \_\_.    
\ex\label{sku8}  [*] {He is a husband that she blamed it on her nitwit of \_\_ !} 
\end{xlist}
\end{exe} 

\noindent
(\ref{sku7}) is well-formed since the stranded preposition is the head of a
{\sc pp}, but ({\ref{sku8}) is not, since the stranded preposition is a functor.  

Given its functor status, the combination of the preposition with its 
{\sc np} sister is a straightforward instance of the {\it regular-nominal\/} 
type. What is truly idiosyncratic about the {\sc bnpc} is the combination of 
this [{\it of\/}--{\sc np}] sequence with the first nominal,  
as in {\it [nitwit [of a husband]]}. Its properties are discussed and analyzed in 
section 5.1. A variant of the {\sc bnpc} with a bare plural,   
as in {\it jewels of villages\/}, is treated in section 5.2. 
A comparison with another monostratal treatment is provided in 
section 5.3 and a summary in section 5.4. 


\subsection{The canonical BNPC} 


To model the combination of the first nominal with the 
[{\it of\/}--{\sc np}] sequence we add a type to the hierarchy of phrases, 
called {\it binominal-np}, which is a subtype of {\it head-independent-phrase}, 
on the one hand, and {\it inverted-predication}, on the other hand.  

\begin{exe}
\ex\label{protor}
%\footnotesize
\tree
    {\ntnode{Zj}{\it phrase},
      {\ntnode{Zl}{{\sc headedness}}, 
        {\ntnode{Zh}{\it headed-phrase},
          {\ntnode{Zq}{\it hd-nonargument-phrase},
            {\ntnode{Zr}{\it hd-independent-phrase}}}}},
      {\ntnode{Zg}{{\sc clausality}},
        {\ntnode{Zm}{\it non-clause},
          {\ntnode{Zk}{\it nominal-parameter},
            {\ntnode{Zn}{\it inverted-predication},
              {\tnode{Zo}{binominal-np}}}}}}}
\nodeconnect{Zj}{Zl}
\nodeconnect{Zl}{Zh}
\nodeconnect{Zh}{Zq}
\nodeconnect{Zq}{Zr}
\nodeconnect{Zj}{Zg}
\nodeconnect{Zg}{Zm}
\nodeconnect{Zm}{Zk}
\nodeconnect{Zk}{Zn}
\nodeconnect{Zn}{Zo}
\nodeconnect{Zr}{Zo}
%\normalsize
\end{exe}

\noindent
Just as in the treatment of the {\sc bmc}, we differentiate the 
properties which the {\sc bnpc} inherits from its supertypes (5.1.1) 
from its inherent properties (5.1.2). 

 
\subsubsection{Inherited properties} 


Sharing the assumption of \citet{Aarts98} and \citet{Keizer07} 
that the first nominal is a prenominal adjunct,  
we treat the {\sc bnpc} as a {\it head-nonargument-phrase\/} in which the 
first daughter is the non-head daughter.  
Since it shares its {\sc marking} value with the mother, 
the addition of the first nominal has the effect of 
turning a marked {\sc np} into an unmarked one, as shown in (\ref{giant}).
  
\begin{exe}
\ex\label{giant}
\footnotesize
\tree
  {\ntnode{Zc}{[{\sc head} ~ $\avmbox{1}$ {\it noun} , {\sc marking} ~ $\avmbox{2}$ {\it unmarked\/}]},
    {\ntnode{Zr}{[{\sc head} ~ {\it noun} , {\sc marking} ~ $\avmbox{2}$]},
      {\tnode{Zs}{nitwit}}},  
    {\ntnode{Zb}{[{\sc head} ~ $\avmbox{1}$ , {\sc dtype} ~ {\it ofa\/}]},
      {\tnode{Zv}{of a husband}}}}
\nodeconnect{Zc}{Zr}
\nodeconnect{Zr}{Zs}
\nodeconnect{Zc}{Zb}
\nodetriangle{Zb}{Zv}
\normalsize
\end{exe}

\noindent
This accounts for the fact that the resulting combination is 
compatible with a determiner, as in {\it her nitwit of a husband}. 

Just like in the {\sc bmc}, the nonhead daughter is not a functor.  
We, hence, assume that it does not lexically select its sister.
Assuming otherwise would lead to the rather far-fetched claim 
that nouns like {\it nitwit, hell\/} and {\it skullcracker\/} 
lexically select an {\sc np} sister that is introduced by {\it of a\/}. 
In fact, since they only combine with such {\sc np}s when they 
are part of the {\sc bnpc}, the proper place to model this is 
in the definition of the {\sc bnpc} itself, rather than in the lexical 
entries of the nouns.  

\bigskip

Turning to the {\sc clausality} dimension we assume that 
the relation between the two nominals is not one of 
restrictive modification, but of predication, 
as suggested by the paraphrases in (\ref{nitwit2}). 

\begin{exe}
\ex\label{nitwit2}
\begin{xlist}
\ex  her nitwit of a husband $\rightarrow$ her husband is a nitwit
\ex  that skullcracker of a headache $\rightarrow$ that headache is like a skullcracker  
\end{xlist}
\end{exe}

\noindent
We call it more specifically {\it inverted\/} predication, since the 
order of the predicate and the predicand in the {\sc bnpc} differs from 
the canonical order in the paraphrase. This captures the same intuition 
as the one that underlies the transformational analysis of 
\citet{DenDikken98} and \citet{Bennisco98}, where the {\sc bnpc} 
is treated as a result of predicate inversion. It is also in line with  
the observation in \citet{KimSells14} that `N1 and N2 are 
in a reverse subject-predicate relation' (o.c., 6).

To model this we build on the {\sc hpsg} treatment of predicative constructions 
in \citet{VanEynde15}. In that treatment predication is 
a relation of type {\it attribute-rel\/} between a theme denoting 
nominal and an attribute denoting predicate.\footnote{This is an
alternative for the small clause treatment of predication in
transformational grammar. Arguments in favor of the relational 
approach are provided in \citet{VanEynde15}.}  
In ordinary predicative constructions, this relation is expressed 
by the verb, usually the copula, but in the {\sc bnpc} there is no 
verb: The theme denoting nominal and the predicate are part of one 
complex {\sc np}. Rather than postulating an empty verb or some newly 
minted functional category, we include the relevant relation in the 
definition of the construction itself. More specifically, we add
a type to the hierarchy of non-clausal phrases, called 
{\it inverted-predication}, which is a subtype of 
{\it nominal-parameter\/} and 
whose inherent properties are spelled out in (\ref{inver5}). 

\begin{exe}
\ex\label{inver5} {\bf Inverted Predication}                \\
\footnotesize
{\it inv-pred\/} ~ $\Rightarrow$ ~ 
\begin{avm}
[synsem|content|restr ~ $\Sigma$ ~ $\bigcup$ ~ \{[{\it fact\/} \\
                               ...|nucl [{\it attribute-rel\/} \\
                                         theme ~ @1         \\
                                         attrib ~ @2 ]]\}   \\
 dtrs ~ <[synsem|content|index ~ @2 {\it index\/}] ~, @3>      \\
 head-dtr ~ @3 [synsem|content|index ~ @1 {\it index\/}]]
\end{avm}
\normalsize
\end{exe}

\noindent
The predicative relation between the daughters is made explicit 
in the {\sc restr} value of the phrase, which contains 
a fact whose nucleus is a relation of type {\it attribute-rel\/} 
in which the {\sc theme} role is assigned to the head daughter 
($\avmbox{1}$) and the {\sc attribute} role to the non-head daughter
($\avmbox{2}$). 
The head daughter does not share its index with its sister. 
In this respect, it differs from 
phrases of type {\it restrictive-modification}, in which the  
daughters share their index. 
This has some empirical bite, since indices in {\sc hpsg} are 
used to model agreement. More specifically, the objects of type 
{\it index\/} are declared to have person, number and gender features.  

\begin{exe}
\ex {\it index\/} : \begin{avm}
                    [person ~ {\it person\/} \\
                     number ~ {\it number\/} \\
                     gender ~ {\it gender\/}]
                    \end{avm}
\end{exe}

\noindent
Signs which share their index are, hence, required to have the same 
person, number and gender. This type of agreement obtains a.o. between 
an anaphoric pronoun and its antecedent, and is distinct from 
morpho-syntactic concord, as argued in \citet{PS94}, 
\citet{Kathol99} and \citet{WechslerZlatic03}.  

The fact that the daughters in (\ref{inver5}) do not 
share their index, therefore, implies that they do not necessarily show 
agreement for person, number and gender. This may at first 
seem undesirable since the first nominal is canonically required to  
show number agreement with the second one, as illustrated by the 
ill-formedness of *{\it her nitwits of a husband}. 
Yet, a requirement of index sharing would be too strong, since 
mismatches are not necessarily ill-formed, as illustrated by  
{\it those prejudiced fools of a jury}, quoted from \citet[101]{Keizer07}. 
Unsurprisingly, the situation resembles the agreement between 
predicate nominals and subjects in copular constructions. 
A recent corpus-based study of that agreement is provided in 
\citet{VanEyndecs16}. It reports that there is 
an agreement effect (with 89.48 \% matches vs. 10.52 \% mismatches in a 
one million word sample), but it also shows that index sharing 
is not appropriate to model that effect. As an alternative, 
it develops an analysis which involves a distinction 
between collective and distributive interpretations. 
The details of that treatment need not detain us here. 
What matters in this context is that index sharing is not the 
appropriate way to handle the agreement effect, and that the assignment of 
different indices to the nominals in the {\sc bnpc} is, hence, justified. 

As compared to the {\it restrictive-modification\/} type, 
which subsumes a wide variety of nominal phrases,  
the {\it inverted-predication\/} type has a smaller range, 
but we expect it to include more members than the {\sc bnpc} alone. 
Another plausible candidate is the appositive {\it of\/}-construction, 
as exemplified by {\it the city of Rome}, where the relation 
between {\it city\/} and {\it Rome\/} can be treated as an instance of 
inverted predication: Rome is claimed to be a city.     


\subsubsection{Inherent properties} 


Besides the properties which it inherits from its supertypes, 
the {\sc bnpc} has some properties of its own. They are spelled 
out in (\ref{binominal}). 

\begin{exe}
\ex\label{binominal} {\bf Binominal Noun Phrase}        \\
\footnotesize
{\it binominal-np\/} ~ $\Rightarrow$ ~    
\begin{avm}
[dtrs ~ <[synsem|category [head ~ {\it noun\/}               \\
                           marking ~ {\it unmarked\/}]] ~, @1>  \\
 head-dtr ~ @1 [{\it regular-nominal\/}                 \\
                synsem|category|marking [{\it marked\/}         \\
                                         dtype ~ {\it ofa\/}]]] 
\end{avm}
\normalsize
\end{exe}

\noindent
The head daughter is required to be a marked regular nominal with 
the {\sc dtype} value {\it ofa}. 
The non-head daughter is required to be an unmarked nominal. 
It may be a single word, as in (\ref{climb2}), or a phrase, 
as in (\ref{destr}). 

\begin{exe}
\ex\label{destr} 
\begin{xlist} 
\ex  Her [absolute nitwit] of a husband is in trouble again. 
\ex  They ousted that [destroyer of education] of a minister. 
\end{xlist}
\end{exe}

\noindent
The former's structure is spelled out in (\ref{chop}). 

\begin{exe}
\ex\label{chop}
\footnotesize
\tree
  {\ntnode{Zb}{[{\sc head} ~ $\avmbox{1}$ {\it noun\/} , {\sc marking} ~ $\avmbox{2}$ {\it unmarked\/}]},
    {\ntnode{Zc}{[{\sc head} ~ $\avmbox{4}$ {\it noun\/} , {\sc marking} ~ $\avmbox{2}$]},
      {\ntnode{Zm}{[{\sc marking} ~ $\avmbox{2}$]},
        {\tnode{Zn}{absolute}}},
      {\ntnode{Zr}{[{\sc head} ~ $\avmbox{4}$ , {\sc marking} ~ {\it unmarked\/}]},
        {\tnode{Zs}{nitwit}}}},  
    {\ntnode{Za}{[{\sc head} ~ $\avmbox{1}$ , {\sc dtype} ~ {\it ofa\/}]},
      {\tnode{Zf}{of a husband}}}}
\nodeconnect{Zb}{Zc}
\nodeconnect{Zc}{Zm}
\nodeconnect{Zm}{Zn}
\nodeconnect{Zc}{Zr}
\nodeconnect{Zr}{Zs}
\nodeconnect{Zb}{Za}
\nodetriangle{Za}{Zf}
\normalsize
\end{exe}

\noindent
Since {\it absolute nitwit\/} is a head-functor phrase, 
its {\sc marking} value is shared with the adjective ($\avmbox{2}$}), 
and since its combination with {\it of a husband\/} is a binominal 
{\sc np} and, hence, a subtype of {\it head-nonargument-phrase}, 
that {\sc marking} value is also shared with the mother, yielding 
an unmarked nominal, which is headed by the rightmost {\sc np} 
($\avmbox{1}$). The resulting nominal can in turn be combined 
with a determiner, as in {\it her absolute nitwit of a husband}. 
Since this combination is a regular nominal, iterative application 
is not excluded. It is, hence, not impossible for a binominal 
{\sc np} to contain another binominal {\sc np}. A relevant example 
is the following quote from the preface of {\it Moby-Dick\/}. 

\begin{exe} 
\ex  this mere painstaking burrower and grubworm of a poor devil of a Sub-Sub appears to have gone through the long Vaticans and 
     street-stalls of the earth  ~~~  (Herman Melville, {\it Moby-Dick}, p. xvii) 
\end{exe}  

\noindent
The lower {\sc bnpc} {\it poor devil of a Sub-Sub\/} is contained in the 
higher {\sc bnpc} {\it painstaking burrower and grubworm of a poor devil 
of a Sub-Sub}. Semantically, the Sub-Sub is claimed to be a poor devil, as well 
as a painstaking burrower and grubworm. 


\subsection{The variant with a bare plural} 


The variant with a bare plural is exemplified by the following 
fragments from the British National Corpus, quoted in \citet[5]{KimSells14}.

\begin{exe}
\ex\label{jewels} 
\begin{xlist} 
\ex  It also has {\it jewels of villages\/} like West Burton and Askrigg ...
\ex  There was a shadowy vagueness about the rest with {\it its hulks of desks\/}
     and clutter of baskets and papers.  
\end{xlist}
\end{exe}

\noindent
This variant is not accepted by all speakers: 
\citet[85]{Foolen04} claims that English does not allow it,  
\citet[1285]{Quirk85} calls it marginal, and   
\citet[212]{Napoli89} claims that it is acceptable in
British English but not in American English. 

To model the grammar of speakers who use this variant we 
add a second lexical entry for the functor {\it of}, 
in which it selects an unmarked (bare) plural and in which it has the 
{\sc marking} value {\it ofbpl}, another subtype of {\it indefinite}. 
Its {\sc synsem} value is spelled out in (\ref{of6}). 

\begin{exe}
\ex\label{of6}
\footnotesize
\begin{avm}
[category [head [{\it preposition\/}                              \\
                 select [category|marking ~ {\it unmarked\/}      \\
                         content|index|number ~ {\it plural\/}]]  \\
           marking [{\it marked\/}                                \\
                    dtype ~ {\it ofbpl\/}]]                       \\
 content|restriction  ~~ \{ \}]
\end{avm}
\normalsize
\end{exe}

\noindent
Besides, we slightly modify the definition of the {\sc bnpc} in 
(\ref{binominal}), allowing the {\sc dtype} value of the 
head daughter to be either {\it ofa\/} or {\it ofbpl}. 


\subsection{Comparison with another treatment} 


There are many analyses to which our treatment could be compared, 
but we will limit the discussion here to one that is  
sufficiently similar to our treatment to make a comparison 
feasible and fruiful, i.e. the one of \citet{KimSells14}. 
It is cast in the notation of Sign-Based Construction Grammar, 
just like the analysis of the {\sc bmc} in \citet{KaySag12}. 
The intuition that underlies the analysis is basically the same as 
that of \citet{Aarts98}, \citet{Keizer07} and our treatment: 
`we assume that the second element {\sc np2/n2} functions as 
the syntactic as well as semantic head while the first one 
serves as the modifier' \citep[29]{KimSells14}. The way in which 
this is made explicit is rather different, though. For a start, the 
{\sc bnpc} is assigned a flat structure, as in (\ref{kim}). 

\begin{exe}
\ex\label{kim}
%\footnotesize
\tree
{\ntnode{Zx}{NP},
  {\ntnode{Zy}{Det},
    {\tnode{Zz}{her}}},
  {\ntnode{Za}{N'_{i}},
    {\ntnode{Zd}{N'},
      {\tnode{Zl}{nitwit}}},
    {\ntnode{Ze}{P},
      {\tnode{Zm}{of}}},
    {\ntnode{Zk}{NP_{i}},
      {\tnode{Zh}{a husband}}}}}
\nodeconnect{Zx}{Zy}
\nodeconnect{Zy}{Zz}
\nodeconnect{Zx}{Za}
\nodeconnect{Za}{Zd}
\nodeconnect{Zd}{Zl}
\nodeconnect{Za}{Ze}
\nodeconnect{Ze}{Zm}
\nodeconnect{Za}{Zk}
\nodetriangle{Zk}{Zh}
%\normalsize
\end{exe}

\noindent
Second, the construction which licenses this flat structure 
is claimed to be headed and non-headed at the same time. 
It is headed in the sense that the first nominal 
denotes a property of the entities ({\it i\/}) 
that are denoted by the second nominal. 
And it is non-headed in the sense that `the {\sc bnp} has 
sequences of nominals fulfilling the same grammatical function, 
neither of which is syntactically dependent on the other' (o.c., 22). 
To make sense of this somewhat paradoxical situation the authors 
introduce a type, called {\it head-mod-juxtaposition}, which is a subtype of 
{\it coordination}, on the one hand, and {\it head-mod-cx},
on the other hand, as in (\ref{juxt}).

\begin{exe}
\ex\label{juxt}
\footnotesize
\tree
      {\ntnode{Zl}{{\sc headedness}}, 
        {\ntnode{Zg}{\it non-headed},
          {\tnode{Zm}{coordination},
            {\ntnode{Zn}{\it head-mod-juxtaposition},
              {\tnode{Zo}{bnp-cx}},
              {\tnode{Zp}{correlative}},
              {\tnode{Zs}{one-more}},
              {\tnode{Zx}{...}}}}},
        {\ntnode{Zh}{\it headed},
          {\ntnode{Zr}{\it head-mod-cx}},
          {\tnode{Zy}{...}}}}
\nodeconnect{Zl}{Zh}
\nodeconnect{Zh}{Zr}
\nodeconnect{Zh}{Zy}
\nodeconnect{Zl}{Zg}
\nodeconnect{Zg}{Zm}
\nodeconnect{Zm}{Zn}
\nodeconnect{Zn}{Zo}
\nodeconnect{Zn}{Zp}
\nodeconnect{Zn}{Zs}
\nodeconnect{Zn}{Zx}
\nodeconnect{Zr}{Zn}
\normalsize
\end{exe}

\noindent
Being a subtype of {\it head-mod-juxtaposition\/}, the {\sc bnpc}  
`inherits only some syntactic properties from coordination and 
some semantic properties from subordination' (o.c., 30).\footnote{Besides
the {\sc bnpc}, the {\it head-mod-juxtaposition\/} type 
subsumes the correlative construction, as exemplified by 
{\it the less I do, the better I feel}, and the 
{\it one more\/} construction, as exemplified by 
{\it one more can of beer and I am leaving\/} \citep[28]{KimSells14}.} 
This begs the question of which constraints are inherited from 
which of the supertypes, but on that point the authors 
remain vague: `we leave it open for future research how
to filter these partial properties correctly, and ensure their inheritance
in the {\sc bnp}' (o.c, 30). In this respect, the treatment
that we propose is more detailed: Both the {\sc bnpc} itself 
and its supertypes are defined explicitly and the inheritance is 
strictly monotonic, in the sense that subtypes inherit all properties 
of their supertypes. 
 

\subsection{Summing up} 


This section has provided an analysis of the Binominal Noun Phrase 
Construction, both the canonical one and the variant with the 
bare plural. Its properties are modeled in terms of a phrase type 
that inherits the constraints of {\it head-independent-phrase}, on 
the one hand, and {\it inverted-predication}, on the other hand. 
The result of unifying the inherited constraints with the inherent 
ones is spelled out in (\ref{bnp2}). 

Comparing this {\sc avm} to that of the Big Mess Construction
we observe both similarities and differences. 
The main similarity concerns the fact that they
are both subtypes of {\it head-independent-phrase}. 
Another similarity concerns the functor status of the preposition {\it of}. 
The main difference concerns their place in the {\sc clausality} hierarchy: 
While the {\sc bmc} is an instance of restrictive modification, 
just like the regular nominals, the {\sc bnpc} is an instance of 
inverted predication, a less common type of combination. 
It appears then that the {\sc bnpc} has less in common with the 
regular nominals than the {\sc bmc}. In that sense 
it shows a higher degree of idiosyncracy. 

\begin{exe}
\ex\label{bnp2} 
\footnotesize
\begin{avm}
[{\it binominal-np\/}                                       \\
 synsem [category [head ~ @1 {\it noun\/}                        \\
                   subject ~ @A                                     \\
                   comps ~ @B                                    \\
                   marking ~ @2 {\it unmarked\/}]                   \\
         content [{\it parameter\/}                            \\
                  index ~ @3 {\it index\/}                     \\
                  restr ~ $\Sigma_{1}$ ~ $\bigcup$ ~ $\Sigma_{2}$ ~ 
                          $\bigcup$ ~ \{[{\it fact\/}          \\
                                  ...|nucl [{\it attribute-rel\/} \\
                                            theme ~ @3         \\
                                            attrib ~ @4 ]]\}]] \\
 dtrs ~ <[synsem [category [head [{\it noun\/}                   \\
                                  select ~ {\it none\/}]         \\ 
                            marking ~ @2 ]                          \\
                  content [index ~ @3                          \\
                           restriction ~ $\Sigma_{1}$ ]]] ~, @5>      \\
 head-dtr ~ @5 [{\it regular-nominal\/}                     \\
                synsem [category [head ~ @1                      \\
                                  subject ~ @A                      \\
                                  comps ~ @B                     \\
                                  marking [{\it marked\/}           \\
                                           dtype ~ {\it ofa\/} \($\vee$ {\it ofbpl\/}\)]] \\                   
                        content [{\it parameter\/}             \\
                                 index ~ @4                    \\
                                 restriction ~ $\Sigma_{2}$ ]]]]
\end{avm}
\normalsize
\end{exe}


\section{Conclusion}


This paper has provided a sign-based treatment of both regular nominals
and two types of idiosyncratic nominals, i.e. the Big Mess Construction 
and the Binominal {\sc np} Construction. Special attention has been paid to the 
interaction of the regular and the exceptional in the idiosyncratic nominals. 
To model that interaction we have exploited the potential of 
constructivist {\sc hpsg}, as laid out in \citet{GS00}. 
The core of the analysis is a bi-dimensional hierarchy of 
phrase types in which the properties of the nominals 
are partly inherited from their supertypes and partly spelled out 
in construction-specific constraints. 
(\ref{protorp}) presents the part of the hierarchy that we have focused on. 

\begin{exe}
\ex\label{protorp}
\footnotesize
\tree
    {\ntnode{Zj}{\it phrase},
      {\ntnode{Zl}{{\sc headedness}}, 
        {\ntnode{Zh}{\it headed-phrase},
          {\ntnode{Zq}{\it hd-nonargument-phrase}, 
            {\ntnode{Zr}{\it hd-functor-phrase},
              {\tnode{Zx}{regular-nominal}}},
            {\ntnode{Zt}{\it hd-indep-phrase}}}}},
      {\ntnode{Zg}{{\sc clausality}},
        {\ntnode{Zm}{\it non-clause},
          {\ntnode{Zk}{\it nominal-parameter},
            {\ntnode{Za}{\it restrictive-mod},
              {\tnode{Zb}{big-mess-phrase}}},   
            {\ntnode{Zn}{\it inverted-pred},
              {\tnode{Zo}{binominal-np}}}}}}}
\nodeconnect{Zj}{Zl}
\nodeconnect{Zl}{Zh}
\nodeconnect{Zh}{Zq}
\nodeconnect{Zq}{Zr}
\nodeconnect{Zq}{Zt}
\nodeconnect{Zj}{Zg}
\nodeconnect{Zg}{Zm}
\nodeconnect{Zm}{Zk}
\nodeconnect{Zk}{Za}
\nodeconnect{Zk}{Zn}
\nodeconnect{Zn}{Zo}
\nodeconnect{Zt}{Zo}
\nodeconnect{Za}{Zb}
\nodeconnect{Zt}{Zb}
\nodeconnect{Zr}{Zx}
\nodeconnect{Za}{Zx}
\normalsize
\end{exe}

\bigskip

The resulting treatment is one `in which the particular and the 
general are knit together seamlessly' \citep{KayFillmore99}. 
Besides, it chimes well with one of the tenets of Construction 
Grammar that `a linguistic structure is motivated to the 
extent that it is related to other structures in the language' 
\citep{Taylor04}. This relatedness is made fully explicit 
in the phrase type hierarchy and in the constraints which 
are associated with the various types. 

In future work we want to explore the potential of this approach 
for dealing with other types of complex nominals with idiosyncratic 
properties, such as appositions with and without {\it of\/} 
({\it the city of Berlin\/} vs. {\it the poet Burns\/}),  
and partitive and pseudo-partitive {\sc np}s. This, we believe, 
will ultimately yield a fine-grained hierarchy of nominal phrase types 
which charts their mutual similarities and differences in 
a fully explicit manner and which makes it possible to measure 
the degree of idiosyncracy of their formation.  

 
\section*{Abbreviations}
\section*{Acknowledgements}

\printbibliography[heading=subbibliography,notkeyword=this] 
\end{document}
