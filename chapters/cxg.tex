\documentclass[output=paper,biblatex,babelshorthands,newtxmath,draftmode,colorlinks,citecolor=brown]{langscibook}
\ChapterDOI{10.5281/zenodo.5599882}

\IfFileExists{../localcommands.tex}{%hack to check whether this is being compiled as part of a collection or standalone
  \usepackage{../nomemoize}
  % add all extra packages you need to load to this file 

% the ISBN assigned to the digital edition
\usepackage[ISBN=9783961102556]{ean13isbn} 

\usepackage{graphicx}
\usepackage{tabularx}
\usepackage{amsmath} 

%\usepackage{tipa}      % Davis Koenig
\usepackage{xunicode} % Provide tipa macros (BC)

\usepackage{multicol}

% Berthold morphology
\usepackage{relsize}
%\usepackage{./styles/rtrees-bc} % forbidden forest 08.12.2019

% provides logo priniting commands
\usepackage{langsci-basic}

\usepackage{langsci-optional} 
% used to be in this package
\providecommand{\citegen}{}
\renewcommand{\citegen}[2][]{\citeauthor{#2}'s (\citeyear*[#1]{#2})}
\providecommand{\lsptoprule}{}
\renewcommand{\lsptoprule}{\midrule\toprule}
\providecommand{\lspbottomrule}{}
\renewcommand{\lspbottomrule}{\bottomrule\midrule}
\providecommand{\largerpage}{}
\renewcommand{\largerpage}[1][1]{\enlargethispage{#1\baselineskip}}

\usepackage{./styles/biblatex-series-number-checks}


\usepackage{langsci-lgr}

\newcommand{\MAS}{\textsc{m}\xspace} % \M is taken by somebody

%\usepackage{./styles/forest/forest}
\usepackage{langsci-forest-setup}

% is loaded in main etc.
% \usepackage{nomemoize} 
% \memoizeset{
%   memo filename prefix={chapters/hpsg-handbook.memo.dir/},
%   register=\todo{O{}+m},
%   prevent=\todo,
% }

\usepackage{tikz-cd}

\usepackage{./styles/tikz-grid}
\usetikzlibrary{shadows}


% removed with texlive 2020 06.05.2020
% %\usepackage{pgfplots} % for data/theory figure in minimalism.tex
% % fix some issue with Mod https://tex.stackexchange.com/a/330076
% \makeatletter
% \let\pgfmathModX=\pgfmathMod@
% \usepackage{pgfplots}%
% \let\pgfmathMod@=\pgfmathModX
% \makeatother

\usepackage{subcaption}

% Stefan Müller's styles
\usepackage{./styles/merkmalstruktur,./styles/makros.2020,./styles/my-xspace,./styles/article-ex,
./styles/eng-date}

\usepackage{varioref}
\newcommand\refORregion[2]{%
 \vrefpagenum\firstnum{#1}%
 \vrefpagenum\secondnum{#2}%
\ifthenelse{\equal\firstnum\secondnum}%
{\pageref{#1}}%
{\pageref{#1}--\pageref{#2}}%
}

% I am sick of fiddeling arround with babel. I want these shorthands also to work in commands I
% define. St.Mü. 13.08.2020
% e.g. with \iwithini
\usepackage{german}
\selectlanguage{USenglish}

\usepackage{./styles/abbrev}


% Has to be loaded late since otherwise footnotes will not work

%%%%%%%%%%%%%%%%%%%%%%%%%%%%%%%%%%%%%%%%%%%%%%%%%%%%
%%%                                              %%%
%%%           Examples                           %%%
%%%                                              %%%
%%%%%%%%%%%%%%%%%%%%%%%%%%%%%%%%%%%%%%%%%%%%%%%%%%%%
% remove the percentage signs in the following lines
% if your book makes use of linguistic examples
\usepackage{langsci-gb4e} 



% This introduces labels which makes hyperlinks work so that proofreading is easier.
%\makeatletter
%\newcommand{\mex}[1]{\ref{ex-\the\c@chapter-\the\numexpr\c@equation+#1}\relax}
%\newcommand{\eaautolabel}{\label{ex-\the\c@chapter-\the\numexpr\c@equation+1}}
%\makeatother

%\let\oldea\ea
%\def\ea{\oldea\eaautolabel}

%\let\oldeal\eal
%\def\eal{\oldeal\eaautolabel}


% Crossing out text
% uncomment when needed
%\usepackage{ulem}

\usepackage{./styles/additional-langsci-index-shortcuts}

% this is the completely redone avm package
\usepackage{langsci-avm}
\avmsetup{columnsep=.3ex,style=narrow}

\avmdefinecommand{phon}[phon]
  {
    attributes  = \itshape%,
%    delimfactor = 900,
%    delimfall   = 10pt
}

\avmdefinecommand{form}[form]
  {
    attributes  = \itshape%,
%    delimfactor = 900,
%    delimfall   = 10pt
}

% \set was already taken
\avmdefinecommand{avmset}[set]{ attributes=\itshape } % define a new \set command
\avmdefinecommand{list}[list]{ attributes=\itshape } % define a new \list command
   % Note: the label "list" will be output in whatever font is currently active.

% \avm{
% 	[subj  & \1 \\
% 	comps & \2 \- \list*(gap-ss) \\ % Produce a \list
% 	deps  & < \1 > \+ \2
% 	]
% }


\avmdefinecommand{nelist}[ne-list]{ attributes=\itshape } % define a new \nelist command
   % Note: the label "ne-list" will be output in whatever font is currently active.



% https://github.com/langsci/langsci-avm/issues/33#issuecomment-671201576
%\avmsetup{extraskip=0pt}

% if you have to use both langsci-avm and avm
% \usepackage{langsci-avm} % Load pkg with meaning A of conflicting cmd
% \let\lavm\avm % Send the conflicting command to an alternative
% \let\avm\undefined % Send the conflicting cmd to be \undefined
% \usepackage{avm} % Load pkg with meaning B for conf. cmd 

%\let\asort\type*

% remove this, once we really do without avm
%\usepackage{./styles/avm+}

% copied over from avm+.sty
% some relation operators:
%\newcommand{\append}[0]{\ensuremath{\oplus\hspace{.24em}}}
%\newcommand{\shuffle}[0]{\ensuremath{\bigcirc\hspace{.24em}}}

\newcommand{\append}[0]{\ensuremath{\oplus}\xspace}
\newcommand{\shuffle}[0]{\ensuremath{\bigcirc}\xspace}


% command to fontify relations in avms 
\newcommand{\rel}[1]{\texttt{#1}}
%\def\relfont{\slshape}%
%\def\relfont{\ttdefault}%


\let\idx\ibox
\let\avmbox\ibox

% command to fontify attributes in ordinary text
%\newcommand{\attrib}[1]{\textsc{#1}}


% some relation operators:
%\newcommand{\append}[0]{\ensuremath{\oplus\hspace{.24em}}}
%\newcommand{\shuffle}[0]{\ensuremath{\bigcirc\hspace{.24em}}}

\def\relfont{\slshape}%
%
% command to fontify relations in avms 
%\newcommand{\rel}[1]{{\relfont #1}}



% \renewcommand{\tpv}[1]{{\avmjvalfont\itshape #1}}

% % no small caps please
% \renewcommand{\phonshape}[0]{\normalfont\itshape}

% \regAvmFonts

\usepackage{theorem}

\newtheorem{mydefinition}{Def.}
\newtheorem{principle}{Principle}

{\theoremstyle{break}
%\newtheorem{schema}{Schema}
\newtheorem{mydefinition-break}[mydefinition]{Def.}
\newtheorem{principle-break}[principle]{Principle}
}


%% \newcommand{schema}[2]{
%% \begin{minipage}{\textwidth}
%% {\textbf{Schema~\theschema}}]\hspace{.5em}\textbf{(#1)}\\
%% #2
%% \end{minipage}}


% This avoids linebreaks in the Schema
\newcounter{schemacounter}
\makeatletter
\newenvironment{schema}[1][]
  {%
   \refstepcounter{schemacounter}%
   \par\bigskip\noindent
   \minipage{\linewidth}%
   \textbf{Schema~\theschemacounter\hspace{.5em} \ifx&#1&\else(#1)\fi}\par
  }{\endminipage\par\bigskip\@endparenv}%
\makeatother

%\usepackage{subfig}





% Davis Koenig Lexikon

\usepackage{tikz-qtree,tikz-qtree-compat} % Davis Koenig remove

\usepackage{shadow}



\usepackage[english]{isodate} % Andy Lücking
\usepackage[autostyle]{csquotes} % Andy
%\usepackage[autolanguage]{numprint}

%\defaultfontfeatures{
%    Path = /usr/local/texlive/2017/texmf-dist/fonts/opentype/public/fontawesome/ }

%% https://tex.stackexchange.com/a/316948/18561
%\defaultfontfeatures{Extension = .otf}% adds .otf to end of path when font loaded without ext parameter e.g. \newfontfamily{\FA}{FontAwesome} > \newfontfamily{\FA}{FontAwesome.otf}
%\usepackage{fontawesome} % Andy Lücking
\usepackage{pifont} % Andy Lücking -> hand

\usetikzlibrary{decorations.pathreplacing} % Andy Lücking
\usetikzlibrary{matrix} % Andy 
\usetikzlibrary{positioning} % Andy
\usepackage{tikz-3dplot} % Andy

% pragmatics
\usepackage{eqparbox} % Andy
\usepackage{enumitem} % Andy
\usepackage{longtable} % Andy
\usepackage{tabu} % Andy              needs to be loaded before hyperref as of texlive 2020

% tabu-fix
% to make "spread 0pt" work
% -----------------------------
\RequirePackage{etoolbox}
\makeatletter
\patchcmd
	\tabu@startpboxmeasure
	{\bgroup\begin{varwidth}}%
	{\bgroup
	 \iftabu@spread\color@begingroup\fi\begin{varwidth}}%
	{}{}
\def\@tabarray{\m@th\def\tabu@currentgrouptype
    {\currentgrouptype}\@ifnextchar[\@array{\@array[c]}}
%
%%% \pdfelapsedtime bug 2019-12-15
\patchcmd
	\tabu@message@etime
	{\the\pdfelapsedtime}%
	{\pdfelapsedtime}%
	{}{}
%
%
\makeatother
% -----------------------------


% Manfred's packages

%\usepackage{shadow}

\usepackage{tabularx}
\newcolumntype{L}[1]{>{\raggedright\arraybackslash}p{#1}} % linksbündig mit Breitenangabe


% Jong-Bok

%\usepackage{xytree}

\newcommand{\xytree}[2][dummy]{Let's do the tree!}

% seems evil, get rid of it
% defines \ex is incompatible with gb4e
%\usepackage{lingmacros}

% taken from lingmacros:
\makeatletter
% \evnup is used to line up the enumsentence number and an entry along
% the top.  It can take an argument to improve lining up.
\def\evnup{\@ifnextchar[{\@evnup}{\@evnup[0pt]}}

\def\@evnup[#1]#2{\setbox1=\hbox{#2}%
\dimen1=\ht1 \advance\dimen1 by -.5\baselineskip%
\advance\dimen1 by -#1%
\leavevmode\lower\dimen1\box1}
\makeatother


% YK -- CG chapter

%\usepackage{xspace}
\usepackage{bm}
\usepackage{ebproof}


% Antonio Branco, remove this
\usepackage{epsfig}

% now unicode
%\usepackage{alphabeta}





\usepackage{pst-node}


% fmr: additional packages
%\usepackage{amsthm}


% Ash and Steve: LFG
\usepackage{./styles/lfg/dalrymple}

\RequirePackage{graphics}
%\RequirePackage{./styles/lfg/trees}
%% \RequirePackage{avm}
%% \avmoptions{active}
%% \avmfont{\sc}
%% \avmvalfont{\sc}
\RequirePackage{./styles/lfg/lfgmacrosash}

\usepackage{./styles/lfg/glue}

%%%%%%%%%%%%%%%%%%%%%%%%%%%%%%
%% Markup
%%%%%%%%%%%%%%%%%%%%%%%%%%%%%%
\usepackage[normalem]{ulem} % For thinks like strikethrough, using \sout

% \newcommand{\high}[1]{\textbf{#1}} % highlighted text
\newcommand{\high}[1]{\textit{#1}} % highlighted text
%\newcommand{\term}[1]{\textit{#1}\/} % technical term
\newcommand{\qterm}[1]{``{#1}''} % technical term, quotes
%\newcommand{\trns}[1]{\strut `#1'} % translation in glossed example
\newcommand{\trnss}[1]{\strut \phantom{\sqz{}} `#1'} % translation in ungrammatical glossed example
\newcommand{\ttrns}[1]{(`#1')} % an in-text translation of a word
\newcommand{\LFGfeat}[1]{\mbox{\textsc{\MakeLowercase{#1}}}}     % feature name
%\newcommand{\val}[1]{\mbox{\textsc{\MakeLowercase{#1}}}}    % f-structure value
\newcommand{\featt}[1]{\mbox{\textsc{\MakeLowercase{#1}}}}     % feature name
\newcommand{\vall}[1]{\mbox{\textsc{\textup{\MakeLowercase{#1}}}}}    % f-structure value
\newcommand{\mg}[1]{\mbox{\textsc{\MakeLowercase{#1}}}}    % morphological gloss
%\newcommand{\word}[1]{\textit{#1}}       % mention of word
\providecommand{\kstar}[1]{{#1}\ensuremath{^*}}
\providecommand{\kplus}[1]{{#1}\ensuremath{^+}}
\newcommand{\template}[1]{@\textsc{\MakeLowercase{#1}}}
\newcommand{\templaten}[1]{\textsc{\MakeLowercase{#1}}}
\newcommand{\templatenn}[1]{\MakeUppercase{#1}}
\newcommand{\tempeq}{\ensuremath{=}}
\newcommand{\predval}[1]{\ensuremath{\langle}\textsc{#1}\ensuremath{\rangle}}
\newcommand{\predvall}[1]{{\rm `#1'}}
\newcommand{\lfgfst}[1]{\ensuremath{#1\,}}
\newcommand{\scare}[1]{``#1''} % scare quotes
\newcommand{\bracket}[1]{\ensuremath{\left\langle\mathit{#1}\right\rangle}}
\newcommand{\sectionw}[1][]{Section#1} % section word: for cap/non-cap
\newcommand{\tablew}[1][]{Table#1} % table word: for cap/non-cap
\newcommand{\lfgglue}{LFG+Glue}
\newcommand{\hpsgglue}{HPSG+Glue}
\newcommand{\gs}{GS}
%\newcommand{\func}[1]{\ensuremath{\mathbf{#1}}}
\newcommand{\func}[1]{\textbf{#1}}
\renewcommand{\glue}{Glue}
%\newcommand{\exr}[1]{(\ref{ex:#1}}
\newcommand{\exra}[1]{(\ref{ex:#1})}


%%%%%%%%%%%%%%%%%%%%%%%%%%%%%%
% Notation
%\newcommand{\xbar}[1]{$_{\mbox{\textsc{#1}$^{\raisebox{1ex}{}}$}}$}
\newcommand{\xprime}[2][]{\textup{\mbox{{#2}\ensuremath{^\prime_{\hspace*{-.0em}\mbox{\footnotesize\ensuremath{\mathit{#1}}}}}}}}
\providecommand{\xzero}[2][]{#2\ensuremath{^0_{\mbox{\footnotesize\ensuremath{\mathit{#1}}}}}}



\let\leftangle\langle
\let\rightangle\rangle

%\newcommand{\pslabel}[1]{}

% remove when finished
\usepackage{proofread}
  %add all your local new commands to this file

% The orchid-id is specified and then extracted by scripts for zenodo.
\newcommand{\orcid}[1]{} 

% do not show the chapter number. It is redundant, since most references to figures are within the
% same chapter.
\renewcommand{\thefigure}{\arabic{figure}}


% Don't do this at home. I do not like the smaller font for captions.
% I just removed loading the caption packege in langscibook.cls
%% \captionsetup{%
%% font={%
%% stretch=1%.8%
%% ,normalsize%,small%
%% },%
%% width=.8\textwidth
%% }

\makeatletter
\def\blx@maxline{77}
\makeatother


\let\citew\citet

\newcommand{\page}{}

\newcommand{\todostefan}[1]{\todo[color=orange!80]{\footnotesize #1}\xspace}
\newcommand{\todosatz}[1]{\todo[color=red!40]{\footnotesize #1}\xspace}

\newcommand{\inlinetodostefan}[1]{\todo[color=green!40,inline]{\footnotesize #1}\xspace}

\newcommand{\inlinetodoopt}[1]{\todo[color=green!40,inline]{\footnotesize #1}\xspace}
\newcommand{\inlinetodoobl}[1]{\todo[color=red!40,inline]{\footnotesize #1}\xspace}

\newcommand{\itd}[1]{\inlinetodoobl{#1}}
\newcommand{\itdobl}[1]{\inlinetodoobl{#1}}
\newcommand{\itdopt}[1]{\inlinetodoopt{#1}}

\newcommand{\itdsecond}[1]{}

\newcommand{\itddone}[1]{}
%\let\itddone\itdopt
\newcommand{\LATER}[1]{}



% A. Red: Simple typos, errors in the AVMs (only a couple) to take care of on the editorial side, no need to contact the authors
% B.: Green: Wording changes which do not necessarily require authors’ approval, but are not just typos/errors
% C.: Blue: Comments to the author that they don’t have to take care of, but after all, the authors might be interested to have the comments for future revisions. 
% D.: Purple: Comments to the editors about something we need to keep in mind or do. Nothing for you

\newcommand{\colorcodingexplanation}{\todo[color=green!40,inline]{%
Explanation of colors of bubbles and text:\\
A.: Red: Things that have to be fixed/commented upon.\\
B.: Green: optional comments\\
C.: Blue: Comments to the author that they don’t have to take care of, but after all, the authors
might be interested to have the comments for future revisions.\\
Explanation of colors of text:\\
Red: newly added material (crossreferences to other chapters and other references)\\
Orange: changed material, please check\\
Blue: suggestions for deletion\\
Please also check margin notes.
}}
% D.: Purple: Comments to the editors about something we need to keep in mind or do. Nothing for you


\newcommand{\itdgreen}[1]{\todo[color=green!40,inline]{\footnotesize #1}\xspace}
\newcommand{\itdblue}[1]{\todo[color=blue!40,inline]{\footnotesize #1}\xspace}

% for editing, remove later
\usepackage{xcolor}
\newcommand{\added}[1]{{\red #1}}
\newcommand{\addedthis}{\todostefan{added this}}

\newcommand{\changed}[1]{\textcolor{orange}{#1}}
\newcommand{\deleted}[1]{\textcolor{blue}{#1}}


% \newcommand{\addpages}{\todostefan{add pages}}
% %\newcommand{\iaddpages}{\inlinetodoobl{add pages}}
% \newcommand{\iaddpages}{\yel[add pages]{pages}\xspace}
% \newcommand{\addref}{\todostefan{add reference}}
% \newcommand{\inlineaddpages}{\inlinetodostefan{add pages}}
% \newcommand{\addglosses}{\todostefan{add glosses}}

\newcommand{\addpages}{\xspace}%np
\newcommand{\iaddpages}{\xspace}%islands und understudied languages
\newcommand{\addref}{\xspace}
\newcommand{\inlineaddpages}{\xspace}
% not used \newcommand{\addglosses}{}


%\newcommand{\spacebr}{\hphantom{[}}

\newcommand{\danishep}{\jambox{(\ili{Danish})}}
\newcommand{\english}{\jambox{(\ili{English})}}
\newcommand{\german}{\jambox{(\ili{German})}}
\newcommand{\yiddish}{\jambox{(\ili{Yiddish})}}
\newcommand{\welsh}{\jambox{(\ili{Welsh})}}

% Cite and cross-reference other chapters
\newcommand{\crossrefchaptert}[2][]{\citet*[#1]{chapters/#2}, Chapter~\ref{chap-#2} of this volume} 
\newcommand{\crossrefchapterp}[2][]{(\citealp*[#1]{chapters/#2}, Chapter~\ref{chap-#2} of this volume)}
\newcommand{\crossrefchapteralt}[2][]{\citealt*[#1]{chapters/#2}, Chapter~\ref{chap-#2} of this volume}
\newcommand{\crossrefchapteralp}[2][]{\citealp*[#1]{chapters/#2}, Chapter~\ref{chap-#2} of this volume}

\newcommand{\crossrefcitet}[2][]{\citet*[#1]{chapters/#2}} 
\newcommand{\crossrefcitep}[2][]{\citep*[#1]{chapters/#2}}
\newcommand{\crossrefcitealt}[2][]{\citealt*[#1]{chapters/#2}}
\newcommand{\crossrefcitealp}[2][]{\citealp*[#1]{chapters/#2}}


% example of optional argument:
% \crossrefchapterp[for something, see:]{name}
% gives: (for something, see: Author 2018, Chapter~X of this volume)



\let\crossrefchapterw\crossrefchaptert



% Davis Koenig

\let\ig=\textsc
\let\tc=\textcolor

% evolution, Flickinger, Pollard, Wasow

\let\citeNP\citet

% Adam P

%\newcommand{\toappear}{Forthcoming}
\newcommand{\pg}[1]{p.\,#1}
\renewcommand{\implies}{\rightarrow}

\newcommand*{\rref}[1]{(\ref{#1})}
\newcommand*{\aref}[1]{(\ref{#1}a)}
\newcommand*{\bref}[1]{(\ref{#1}b)}
\newcommand*{\cref}[1]{(\ref{#1}c)}

\newcommand{\msadam}{.}
\newcommand{\morsyn}[1]{\textsc{#1}}

\newcommand{\aux}{\textsc{aux}\xspace}

\newcommand{\nom}{\morsyn{nom}}
\newcommand{\acc}{\morsyn{acc}}
\newcommand{\dat}{\morsyn{dat}}
\newcommand{\gen}{\morsyn{gen}}
\newcommand{\ins}{\morsyn{ins}}
%\newcommand{\aploc}{\morsyn{loc}}
\newcommand{\voc}{\morsyn{voc}}
\newcommand{\ill}{\morsyn{ill}}
\renewcommand{\inf}{\morsyn{inf}}
\newcommand{\passprc}{\morsyn{passp}}

%\newcommand{\Nom}{\msadam\nom}
%\newcommand{\Acc}{\msadam\acc}
%\newcommand{\Dat}{\msadam\dat}
%\newcommand{\Gen}{\msadam\gen}
\newcommand{\Ins}{\msadam\ins}
\newcommand{\Loc}{\msadam\loc}
\newcommand{\Voc}{\msadam\voc}
\newcommand{\Ill}{\msadam\ill}
\newcommand{\PassP}{\msadam\passprc}

\newcommand{\Aux}{\textsc{aux}}

%\newcommand{\princ}[1]{\textnormal{\textsc{#1}}} % for constraint names
\newcommand{\princ}[1]{\textnormal{#1}} % for constraint names
\newcommand{\notion}[1]{\emph{#1}}
\renewcommand{\path}[1]{\textnormal{\textsc{#1}}}
\newcommand{\ftype}[1]{\textit{#1}}
\newcommand{\fftype}[1]{{\scriptsize\textit{#1}}}
\newcommand{\la}{$\langle$}
\newcommand{\ra}{$\rangle$}
%\newcommand{\argst}{\path{arg-st}}
\newcommand{\phtm}[1]{\setbox0=\hbox{#1}\hspace{\wd0}}
\newcommand{\prep}[1]{\setbox0=\hbox{#1}\hspace{-1\wd0}#1}


% Rui

\newcommand{\spc}[0]{\hspace{-1pt}\underline{\hspace{6pt}}\,}
\newcommand{\spcs}[0]{\hspace{-1pt}\underline{\hspace{6pt}}\,\,}
\newcommand{\bad}[1]{\leavevmode\llap{#1}}
\newcommand{\COMMENT}[1]{}


% Rui coordination
\newcommand{\subl}[1]{$_{\scriptstyle \textsc{#1}}$}



% Andy Lücking gesture.tex
\newcommand{\Pointing}{\ding{43}}
% Giotto: "Meeting of Joachim and Anne at the Golden Gate" - 1305-10 
\definecolor{GoldenGate1}{rgb}{.13,.09,.13} % Dress of woman in black
\definecolor{GoldenGate2}{rgb}{.94,.94,.91} % Bridge
\definecolor{GoldenGate3}{rgb}{.06,.09,.22} % Blue sky
\definecolor{GoldenGate4}{rgb}{.94,.91,.87} % Dress of woman with shawl
\definecolor{GoldenGate5}{rgb}{.52,.26,.26} % Joachim's robe
\definecolor{GoldenGate6}{rgb}{.65,.35,.16} % Anne's robe
\definecolor{GoldenGate7}{rgb}{.91,.84,.42} % Joachim's halo

\makeatletter
\newcommand{\@Depth}{1} % x-dimension, to front
\newcommand{\@Height}{1} % z-dimension, up
\newcommand{\@Width}{1} % y-dimension, rightwards
%\GGS{<x-start>}{<y-start>}{<z-top>}{<z-bottom>}{<Farbe>}{<x-width>}{<y-depth>}{<opacity>}
\newcommand{\GGS}[9][]{%
\coordinate (O) at (#2-1,#3-1,#5);
\coordinate (A) at (#2-1,#3-1+#7,#5);
\coordinate (B) at (#2-1,#3-1+#7,#4);
\coordinate (C) at (#2-1,#3-1,#4);
\coordinate (D) at (#2-1+#8,#3-1,#5);
\coordinate (E) at (#2-1+#8,#3-1+#7,#5);
\coordinate (F) at (#2-1+#8,#3-1+#7,#4);
\coordinate (G) at (#2-1+#8,#3-1,#4);
\draw[draw=black, fill=#6, fill opacity=#9] (D) -- (E) -- (F) -- (G) -- cycle;% Front
\draw[draw=black, fill=#6, fill opacity=#9] (C) -- (B) -- (F) -- (G) -- cycle;% Top
\draw[draw=black, fill=#6, fill opacity=#9] (A) -- (B) -- (F) -- (E) -- cycle;% Right
}
\makeatother


% pragmatics
\newcommand{\speaking}[1]{\eqparbox{name}{\textsc{\lowercase{#1}\space}}}
\newcommand{\alname}[1]{\eqparbox{name}{\textsc{\lowercase{#1}}}}
\newcommand{\HPSGTTR}{HPSG$_{\text{TTR}}$\xspace}

\newcommand{\ttrtype}[1]{\textit{#1}}
\newcommand{\avmel}{\q<\quad\q>} %% shortcut for empty lists in AVM
\newcommand{\ttrmerge}{\ensuremath{\wedge_{\textit{merge}}}}
\newcommand{\Cat}[2][0.1pt]{%
  \begin{scope}[y=#1,x=#1,yscale=-1, inner sep=0pt, outer sep=0pt]
   \path[fill=#2,line join=miter,line cap=butt,even odd rule,line width=0.8pt]
  (151.3490,307.2045) -- (264.3490,307.2045) .. controls (264.3490,291.1410) and (263.2021,287.9545) .. (236.5990,287.9545) .. controls (240.8490,275.2045) and (258.1242,244.3581) .. (267.7240,244.3581) .. controls (276.2171,244.3581) and (286.3490,244.8259) .. (286.3490,264.2045) .. controls (286.3490,286.2045) and (323.3717,321.6755) .. (332.3490,307.2045) .. controls (345.7277,285.6390) and (309.3490,292.2151) .. (309.3490,240.2046) .. controls (309.3490,169.0514) and (350.8742,179.1807) .. (350.8742,139.2046) .. controls (350.8742,119.2045) and (345.3490,116.5037) .. (345.3490,102.2045) .. controls (345.3490,83.3070) and (361.9972,84.4036) .. (358.7581,68.7349) .. controls (356.5206,57.9117) and (354.7696,49.2320) .. (353.4652,36.1439) .. controls (352.5396,26.8573) and (352.2445,16.9594) .. (342.5985,17.3574) .. controls (331.2650,17.8250) and (326.9655,37.7742) .. (309.3490,39.2045) .. controls (291.7685,40.6320) and (276.7783,24.2380) .. (269.9740,26.5795) .. controls (263.2271,28.9013) and (265.3490,47.2045) .. (269.3490,60.2045) .. controls (275.6359,80.6368) and (289.3490,107.2045) .. (264.3490,111.2045) .. controls (239.3490,115.2045) and (196.3490,119.2045) .. (165.3490,160.2046) .. controls (134.3490,201.2046) and (135.4934,249.3212) .. (123.3490,264.2045) .. controls (82.5907,314.1553) and (40.8239,293.6463) .. (40.8239,335.2045) .. controls (40.8239,353.8102) and (72.3490,367.2045) .. (77.3490,361.2045) .. controls (82.3490,355.2045) and (34.8638,337.3259) .. (87.9955,316.2045) .. controls (133.3871,298.1601) and   (137.4391,294.4766) .. (151.3490,307.2045) -- cycle;
\end{scope}%
}
%% leicht modifiziert nach Def. von Sebastian Nordhoff:
% \newcommand{\lueckingbox}[3]{\parbox[t][][t]{0.7cm}{\raggedright
%     \strut#1}\parbox[t][][t]{7.7cm}{\strut#2}\parbox[t][][t]{3cm}{\raggedright\strut#3}\bigskip\\}
\newcommand{\lueckingbox}[3]{\parbox[t][][t]{0.7cm}{\raggedright
    \strut\vspace*{-\baselineskip}\newline#1}\parbox[t][][t]{7.7cm}{\strut\vspace*{-\baselineskip}\newline#2}\parbox[t][][t]{3cm}{\raggedright\strut\vspace*{-\baselineskip}\newline#3}\bigskip\\}




% KdK
\newcommand{\smiley}{:)}

\renewbibmacro*{index:name}[5]{%
  \usebibmacro{index:entry}{#1}
    {\iffieldundef{usera}{}{\thefield{usera}\actualoperator}\mkbibindexname{#2}{#3}{#4}{#5}}}

% \newcommand{\noop}[1]{}

% chngcntr.sty otherwise gives error that these are already defined
%\let\counterwithin\relax
%\let\counterwithout\relax

% the space of a left bracket for glossings
\newcommand{\LB}{\hphantom{[}}

\newcommand{\LF}{\mbox{$[\![$}}

\newcommand{\RF}{\mbox{$]\!]_F$}}

\newcommand{\RT}{\mbox{$]\!]_T$}}





% Manfred's

\newcommand{\kommentar}[1]{}

\newcommand{\bsp}[1]{\emph{#1}}
\newcommand{\bspT}[2]{\bsp{#1} `#2'}
\newcommand{\bspTL}[3]{\bsp{#1} (lit.: #2) `#3'}

\newcommand{\noidi}{§}

\newcommand{\refer}[1]{(\ref{#1})}

%\newcommand{\avmtype}[1]{\multicolumn{2}{l}{\type{#1}}}
\newcommand{\attr}[1]{\textsc{#1}}

%\newcommand{\srdefault}{\mbox{\begin{tabular}{@{}c@{}}{\large <}\\[-1.5ex]$\sqcap$\end{tabular}}}
\newcommand{\srdefault}{$\stackrel{<}{\sqcap}$}


%% \newcommand{\myappcolumn}[2]{
%% \begin{minipage}[t]{#1}#2\end{minipage}
%% }

%% \newcommand{\appc}[1]{\myappcolumn{3.7cm}{#1}}


% Jong-Bok


% clean that up and do not use \def (killing other stuff defined before)
%\if 0
%\newcommand\DEL{\textsc{del}}
%\newcommand\del{\textsc{del}}

\newcommand\conn{\textsc{conn}}
\newcommand\CONN{\textsc{conn}}
\newcommand\CONJ{\textsc{conj}}
\newcommand\LITE{\textsc{lex}}
\newcommand\lite{\textsc{lex}}
\newcommand\HON{\textsc{hon}}

%\newcommand\CAUS{\textsc{caus}}
%\newcommand\PASS{\textsc{pass}}
\newcommand\NPST{\textsc{npst}}
%\newcommand\COND{\textsc{cond}}



\newcommand\hdlite{\textsc{head-lex construction}}
\newcommand\hdlight{\textsc{head-light} Schema}
\newcommand\NFORM{\textsc{nform}}

\newcommand\RELS{\textsc{rels}}
%\newcommand\TENSE{\textsc{tense}}


%\newcommand\ARG{\textsc{arg}}
\newcommand\ARGs{\textsc{arg0}}
\newcommand\ARGa{\textsc{arg}}
\newcommand\ARGb{\textsc{arg2}}
\newcommand\TPC{\textsc{top}}
%\newcommand\PROG{\textsc{prog}}

\newcommand\LIGHT{\textsc{light}\xspace}
\newcommand\pst{\textsc{pst}}
%\newcommand\PAST{\textsc{pst}}
%\newcommand\DAT{\textsc{dat}}
%\newcommand\CONJ{\textsc{conj}}
\newcommand\nominal{\textsc{nominal}}
\newcommand\NOMINAL{\textsc{nominal}}
\newcommand\VAL{\textsc{val}}
%\newcommand\val{\textsc{val}}
\newcommand\MODE{\textsc{mode}}
\newcommand\RESTR{\textsc{restr}}
\newcommand\SIT{\textsc{sit}}
\newcommand\ARG{\textsc{arg}}
\newcommand\RELN{\textsc{rel}}
%\newcommand\REL{\textsc{rel}}
%\newcommand\RELS{\textsc{rels}}
%\newcommand\arg-st{\textsc{arg-st}}
\newcommand\xdel{\textsc{xdel}}
\newcommand\zdel{\textsc{zdel}}
\newcommand\sug{\textsc{sug}}
%\newcommand\IMP{\textsc{imp}}
%\newcommand\conn{\textsc{conn}}
%\newcommand\CONJ{\textsc{conj}}
%\newcommand\HON{\textsc{hon}}
\newcommand\BN{\textsc{bn}}
\newcommand\bn{\textsc{bn}}
\newcommand\pres{\textsc{pres}}
\newcommand\PRES{\textsc{pres}}
\newcommand\prs{\textsc{pres}}
%\newcommand\PRS{\textsc{pres}}
\newcommand\agt{\textsc{agt}}
%\newcommand\DEL{\textsc{del}}
%\newcommand\PRED{\textsc{pred}}
\newcommand\AGENT{\textsc{agent}}
\newcommand\THEME{\textsc{theme}}
%\newcommand\AUX{\textsc{aux}}
%\newcommand\THEME{\textsc{theme}}
%\newcommand\PL{\textsc{pl}}
\newcommand\SRC{\textsc{src}}
\newcommand\src{\textsc{src}}
\newcommand{\FORMjb}{\textsc{form}}
\newcommand{\formjb}{\FORM}
\newcommand\GCASE{\textsc{gcase}}
\newcommand\gcase{\textsc{gcase}}
\newcommand\SCASE{\textsc{scase}}
\newcommand\PHON{\textsc{phon}}
%\newcommand\SS{\textsc{ss}}
\newcommand\SYN{\textsc{syn}}
%\newcommand\LOC{\textsc{loc}}
\newcommand\MOD{\textsc{mod}}
\newcommand\INV{\textsc{inv}}
%\newcommand\L{\textsc{l}}
%\newcommand\CASE{\textsc{case}}
\newcommand\SPR{\textsc{spr}}
\newcommand\COMPS{\textsc{comps}}
%\newcommand\comps{\textsc{comps}}
\newcommand\SEM{\textsc{sem}}
\newcommand\CONT{\textsc{cont}}
\newcommand\SUBCAT{\textsc{subcat}}
\newcommand\CAT{\textsc{cat}}
%\newcommand\C{\textsc{c}}
%\newcommand\SUBJ{\textsc{subj}}
\newcommand\subjjb{\textsc{subj}}
%\newcommand\SLASH{\textsc{slash}}
\newcommand\LOCAL{\textsc{local}}
%\newcommand\ARG-ST{\textsc{arg-st}}
%\newcommand\AGR{\textsc{agr}}
\newcommand\PER{\textsc{per}}
%\newcommand\NUM{\textsc{num}}
%\newcommand\IND{\textsc{ind}}
\newcommand\VFORM{\textsc{vform}}
\newcommand\PFORM{\textsc{pform}}
\newcommand\decl{\textsc{decl}}
%\newcommand\loc{\textsc{loc   }}
% \newcommand\   {\textsc{  }}

%\newcommand\NEG{\textsc{neg}}
\newcommand\FRAMES{\textsc{frames}}
%\newcommand\REFL{\textsc{refl}}

\newcommand\MKG{\textsc{mkg}}

%\newcommand\BN{\textsc{bn}}
\newcommand\HD{\textsc{hd}}
\newcommand\NP{\textsc{np}}
\newcommand\PF{\textsc{pf}}
%\newcommand\PL{\textsc{pl}}
\newcommand\PP{\textsc{pp}}
%\newcommand\SS{\textsc{ss}}
\newcommand\VF{\textsc{vf}}
\newcommand\VP{\textsc{vp}}
%\newcommand\bn{\textsc{bn}}
\newcommand\cl{\textsc{cl}}
%\newcommand\pl{\textsc{pl}}
\newcommand\Wh{\ital{Wh}}
%\newcommand\ng{\textsc{neg}}
\newcommand\wh{\ital{wh}}
%\newcommand\ACC{\textsc{acc}}
%\newcommand\AGR{\textsc{agr}}
\newcommand\AGT{\textsc{agt}}
\newcommand\ARC{\textsc{arc}}
%\newcommand\ARG{\textsc{arg}}
\newcommand\ARP{\textsc{arc}}
%\newcommand\AUX{\textsc{aux}}
%\newcommand\CAT{\textsc{cat}}
%\newcommand\COP{\textsc{cop}}
%\newcommand\DAT{\textsc{dat}}
\newcommand\NEWCOMMAND{\textsc{def}}
%\newcommand\DEL{\textsc{del}}
\newcommand\DOM{\textsc{dom}}
\newcommand\DTR{\textsc{dtr}}
%\newcommand\FUT{\textsc{fut}}
\newcommand\GAP{\textsc{gap}}
%\newcommand\GEN{\textsc{gen}}
%\newcommand\HON{\textsc{hon}}
%\newcommand\IMP{\textsc{imp}}
%\newcommand\IND{\textsc{ind}}
%\newcommand\INV{\textsc{inv}}
\newcommand\LEX{\textsc{lex}}
\newcommand\Lex{\textsc{lex}}
%\newcommand\LOC{\textsc{loc}}
%\newcommand\MOD{\textsc{mod}}
\newcommand\MRK{{\nr MRK}}
%\newcommand\NEG{\textsc{neg}}
\newcommand\NEW{\textsc{new}}
%\newcommand\NOM{\textsc{nom}}
%\newcommand\NUM{\textsc{num}}
%\newcommand\PER{\textsc{per}}
%\newcommand\PST{\textsc{pst}}
\newcommand\QUE{\textsc{que}}
%\newcommand\REL{\textsc{rel}}
\newcommand\SEL{\textsc{sel}}
%\newcommand\SEM{\textsc{sem}}
%\newcommand\SIT{\textsc{arg0}}
%\newcommand\SPR{\textsc{spr}}
%\newcommand\SRC{\textsc{src}}
\newcommand\SUG{\textsc{sug}}
%\newcommand\SYN{\textsc{syn}}
%\newcommand\TPC{\textsc{top}}
%\newcommand\VAL{\textsc{val}}
%\newcommand\acc{\textsc{acc}}
%\newcommand\agt{\textsc{agt}}
\newcommand\cop{\textsc{cop}}
%\newcommand\dat{\textsc{dat}}
\newcommand\foc{\textsc{focus}}
%\newcommand\FOC{\textsc{focus}}
\newcommand\fut{\textsc{fut}}
\newcommand\hon{\textsc{hon}}
\newcommand\imp{\textsc{imp}}
\newcommand\kes{\textsc{kes}}
%\newcommand\lex{\textsc{lex}}
%\newcommand\loc{\textsc{loc}}
\newcommand\mrk{{\nr MRK}}
%\newcommand\nom{\textsc{nom}}
%\newcommand\num{\textsc{num}}
\newcommand\plu{\textsc{plu}}
\newcommand\pne{\textsc{pne}}
%\newcommand\pst{\textsc{pst}}
\newcommand\pur{\textsc{pur}}
%\newcommand\que{\textsc{que}}
%\newcommand\src{\textsc{src}}
%\newcommand\sug{\textsc{sug}}
\newcommand\tpc{\textsc{top}}
%\newcommand\utt{\textsc{utt}}
%\newcommand\val{\textsc{val}}
%% \newcommand\LITE{\textsc{lex}}
%% \newcommand\PAST{\textsc{pst}}
%% \newcommand\POSP{\textsc{pos}}
%% \newcommand\PRS{\textsc{pres}}
%% \newcommand\mod{\textsc{mod}}%
%% \newcommand\newuse{{`kes'}}
%% \newcommand\posp{\textsc{pos}}
%% \newcommand\prs{\textsc{pres}}
%% \newcommand\psp{{\it en\/}}
%% \newcommand\skes{\textsc{kes}}
%% \newcommand\CASE{\textsc{case}}
%% \newcommand\CASE{\textsc{case}}
%% \newcommand\COMP{\textsc{comp}}
%% \newcommand\CONJ{\textsc{conj}}
%% \newcommand\CONN{\textsc{conn}}
%% \newcommand\CONT{\textsc{cont}}
%% \newcommand\DECL{\textsc{decl}}
%% \newcommand\FOCUS{\textsc{focus}}
%% %\newcommand\FORM{\textsc{form}} duplicate
%% \newcommand\FREL{\textsc{frel}}
%% \newcommand\GOAL{\textsc{goal}}
\newcommand\HEAD{\textsc{head}}
%% \newcommand\INDEX{\textsc{ind}}
%% \newcommand\INST{\textsc{inst}}
%% \newcommand\MODE{\textsc{mode}}
%% \newcommand\MOOD{\textsc{mood}}
%% \newcommand\NMLZ{\textsc{nmlz}}
%% \newcommand\PHON{\textsc{phon}}
%% \newcommand\PRED{\textsc{pred}}
%% %\newcommand\PRES{\textsc{pres}}
%% \newcommand\PROM{\textsc{prom}}
%% \newcommand\RELN{\textsc{pred}}
%% \newcommand\RELS{\textsc{rels}}
%% \newcommand\STEM{\textsc{stem}}
%% \newcommand\SUBJ{\textsc{subj}}
%% \newcommand\XARG{\textsc{xarg}}
%% \newcommand\bse{{\it bse\/}}
%% \newcommand\case{\textsc{case}}
%% \newcommand\caus{\textsc{caus}}
%% \newcommand\comp{\textsc{comp}}
%% \newcommand\conj{\textsc{conj}}
%% \newcommand\conn{\textsc{conn}}
%% \newcommand\decl{\textsc{decl}}
%% \newcommand\fin{{\it fin\/}}
%% %\newcommand\form{\textsc{form}}
%% \newcommand\gend{\textsc{gend}}
%% \newcommand\inf{{\it inf\/}}
%% \newcommand\mood{\textsc{mood}}
%% \newcommand\nmlz{\textsc{nmlz}}
%% \newcommand\pass{\textsc{pass}}
%% \newcommand\past{\textsc{past}}
%% \newcommand\perf{\textsc{perf}}
%% \newcommand\pln{{\it pln\/}}
%% \newcommand\pred{\textsc{pred}}


%% %\newcommand\pres{\textsc{pres}}
%% \newcommand\proc{\textsc{proc}}
%% \newcommand\nonfin{{\it nonfin\/}}
%% \newcommand\AGENT{\textsc{agent}}
%% \newcommand\CFORM{\textsc{cform}}
%% %\newcommand\COMPS{\textsc{comps}}
%% \newcommand\COORD{\textsc{coord}}
%% \newcommand\COUNT{\textsc{count}}
%% \newcommand\EXTRA{\textsc{extra}}
%% \newcommand\GCASE{\textsc{gcase}}
%% \newcommand\GIVEN{\textsc{given}}
%% \newcommand\LOCAL{\textsc{local}}
%% \newcommand\NFORM{\textsc{nform}}
%% \newcommand\PFORM{\textsc{pform}}
%% \newcommand\SCASE{\textsc{scase}}
%% \newcommand\SLASH{\textsc{slash}}
%% \newcommand\SLASH{\textsc{slash}}
%% \newcommand\THEME{\textsc{theme}}
%% \newcommand\TOPIC{\textsc{topic}}
%% \newcommand\VFORM{\textsc{vform}}
%% \newcommand\cause{\textsc{cause}}
%% %\newcommand\comps{\textsc{comps}}
%% \newcommand\gcase{\textsc{gcase}}
%% \newcommand\itkes{{\it kes\/}}
%% \newcommand\pass{{\it pass\/}}
%% \newcommand\vform{\textsc{vform}}
%% \newcommand\CCONT{\textsc{c-cont}}
%% \newcommand\GN{\textsc{given-new}}
%% \newcommand\INFO{\textsc{info-st}}
%% \newcommand\ARG-ST{\textsc{arg-st}}
%% \newcommand\SUBCAT{\textsc{subcat}}
%% \newcommand\SYNSEM{\textsc{synsem}}
%% \newcommand\VERBAL{\textsc{verbal}}
%% \newcommand\arg-st{\textsc{arg-st}}
%% \newcommand\plain{{\it plain}\/}
%% \newcommand\propos{\textsc{propos}}
%% \newcommand\ADVERBIAL{\textsc{advl}}
%% \newcommand\HIGHLIGHT{\textsc{prom}}
%% \newcommand\NOMINAL{\textsc{nominal}}

\newenvironment{myavm}{\begingroup\avmvskip{.1ex}
  \selectfont\begin{avm}}%
{\end{avm}\endgroup\medskip}
\newcommand\pfix{\vspace{-5pt}}


\newcommand{\jbsub}[1]{\lower4pt\hbox{\small #1}}
\newcommand{\jbssub}[1]{\lower4pt\hbox{\small #1}}
\newcommand\jbtr{\underbar{\ \ \ }\ }


%\fi

% cl

\newcommand{\delphin}{\textsc{delph-in}}


% YK -- CG chapter

\newcommand{\grey}[1]{\colorbox{mycolor}{#1}}
\definecolor{mycolor}{gray}{0.8}

\newcommand{\GQU}[2]{\raisebox{1.6ex}{\ensuremath{\rotatebox{180}{\textbf{#1}}_{\scalebox{.7}{\textbf{#2}}}}}}

\newcommand{\SetInfLen}{\setpremisesend{0pt}\setpremisesspace{10pt}\setnamespace{0pt}}

\newcommand{\pt}[1]{\ensuremath{\mathsf{#1}}}
\newcommand{\ptv}[1]{\ensuremath{\textsf{\textsl{#1}}}}

\newcommand{\sv}[1]{\ensuremath{\bm{\mathcal{#1}}}}
\newcommand{\sX}{\sv{X}}
\newcommand{\sF}{\sv{F}}
\newcommand{\sG}{\sv{G}}

\newcommand{\syncat}[1]{\textrm{#1}}
\newcommand{\syncatVar}[1]{\ensuremath{\mathit{#1}}}

\newcommand{\RuleName}[1]{\textrm{#1}}

\newcommand{\SemTyp}{\textsf{Sem}}

\newcommand{\E}{\ensuremath{\bm{\epsilon}}\xspace}

\newcommand{\greeka}{\upalpha}
\newcommand{\greekb}{\upbeta}
\newcommand{\greekd}{\updelta}
\newcommand{\greekp}{\upvarphi}
\newcommand{\greekr}{\uprho}
\newcommand{\greeks}{\upsigma}
\newcommand{\greekt}{\uptau}
\newcommand{\greeko}{\upomega}
\newcommand{\greekz}{\upzeta}

\newcommand{\Lemma}{\ensuremath{\hskip.5em\vdots\hskip.5em}\noLine}
\newcommand{\LemmaAlt}{\ensuremath{\hskip.5em\vdots\hskip.5em}}

\newcommand{\I}{\iota}

\newcommand{\sem}{\ensuremath}

\newcommand{\NoSem}{%
\renewcommand{\LexEnt}[3]{##1; \syncat{##3}}
\renewcommand{\LexEntTwoLine}[3]{\renewcommand{\arraystretch}{.8}%
\begin{array}[b]{l} ##1;  \\ \syncat{##3} \end{array}}
\renewcommand{\LexEntThreeLine}[3]{\renewcommand{\arraystretch}{.8}%
\begin{array}[b]{l} ##1; \\ \syncat{##3} \end{array}}}

\newcommand{\hypml}[2]{\left[\!\!#1\!\!\right]^{#2}}

%%%%for bussproof
\def\defaultHypSeparation{\hskip0.1in}
\def\ScoreOverhang{0pt}

\newcommand{\MultiLine}[1]{\renewcommand{\arraystretch}{.8}%
\ensuremath{\begin{array}[b]{l} #1 \end{array}}}

\newcommand{\MultiLineMod}[1]{%
\ensuremath{\begin{array}[t]{l} #1 \end{array}}}

\newcommand{\hypothesis}[2]{[ #1 ]^{#2}}

\newcommand{\LexEnt}[3]{#1; \ensuremath{#2}; \syncat{#3}}

\newcommand{\LexEntTwoLine}[3]{\renewcommand{\arraystretch}{.8}%
\begin{array}[b]{l} #1; \\ \ensuremath{#2};  \syncat{#3} \end{array}}

\newcommand{\LexEntThreeLine}[3]{\renewcommand{\arraystretch}{.8}%
\begin{array}[b]{l} #1; \\ \ensuremath{#2}; \\ \syncat{#3} \end{array}}

\newcommand{\LexEntFiveLine}[5]{\renewcommand{\arraystretch}{.8}%
\begin{array}{l} #1 \\ #2; \\ \ensuremath{#3} \\ \ensuremath{#4}; \\ \syncat{#5} \end{array}}

\newcommand{\LexEntFourLine}[4]{\renewcommand{\arraystretch}{.8}%
\begin{array}{l} \pt{#1} \\ \pt{#2}; \\ \syncat{#4} \end{array}}

\newcommand{\ManySomething}{\renewcommand{\arraystretch}{.8}%
\raisebox{-3mm}{\begin{array}[b]{c} \vdots \,\,\,\,\,\, \vdots \\
\vdots \end{array}}}

\newcommand{\lemma}[1]{\renewcommand{\arraystretch}{.8}%
\begin{array}[b]{c} \vdots \\ #1 \end{array}}

\newcommand{\lemmarev}[1]{\renewcommand{\arraystretch}{.8}%
\begin{array}[b]{c} #1 \\ \vdots \end{array}}

\newcommand{\p}{\ensuremath{\upvarphi}}

% clashes with soul package
\newcommand{\yusukest}{\textbf{\textsf{st}}}

\newcommand{\shortarrow}{\xspace\hskip-1.2ex\scalebox{.5}[1]{\ensuremath{\bm{\rightarrow}}}\hskip-.5ex\xspace}

\newcommand{\SemInt}[1]{\mbox{$[\![ \textrm{#1} ]\!]$}}

\newcommand{\HypSpace}{\hskip-.8ex}
\newcommand{\RaiseHeight}{\raisebox{2.2ex}}
\newcommand{\RaiseHeightLess}{\raisebox{1ex}}

\newcommand{\ThreeColHyp}[1]{\RaiseHeight{\Bigg[}\HypSpace#1\HypSpace\RaiseHeight{\Bigg]}}
\newcommand{\TwoColHyp}[1]{\RaiseHeightLess{\Big[}\HypSpace#1\HypSpace\RaiseHeightLess{\Big]}}

\newcommand{\LemmaShort}{\ensuremath{ \ \vdots} \ \noLine}
\newcommand{\LemmaShortAlt}{\ensuremath{ \ \vdots} \ }

\newcommand{\fail}{**}
\newcommand{\vs}{\raisebox{.05em}{\ensuremath{\upharpoonright}}}
\newcommand{\DerivSize}{\small}

% This is not needed, we just take unicode symbols
% The result of the code below came out wrong anyway.
% St. Mü. 10.06.2021
%
% \def\maru#1{{\ooalign{\hfil
%   \ifnum#1>999 \resizebox{.25\width}{\height}{#1}\else%
%   \ifnum#1>99 \resizebox{.33\width}{\height}{#1}\else%
%   \ifnum#1>9 \resizebox{.5\width}{\height}{#1}\else #1%
%   \fi\fi\fi%
% \/\hfil\crcr%
% \raise.167ex\hbox{\mathhexbox20D}}}}

\newenvironment{samepage2}%
 {\begin{flushleft}\begin{minipage}{\linewidth}}
 {\end{minipage}\end{flushleft}}

\newcommand{\cmt}[1]{\textsl{\textbf{[#1]}}}
\newcommand{\trns}[1]{\textbf{#1}\xspace}
\newcommand{\ptfont}{}
\newcommand{\gp}{\underline{\phantom{oo}}}
\newcommand{\mgcmt}{\marginnote}

\newcommand{\term}[1]{\emph{\isi{#1}}}

\newcommand{\citeposs}[1]{\citeauthor{#1}'s \citeyearpar{#1}}

% for standalone compilations Felix: This is in the class already
%\let\thetitle\@title
%\let\theauthor\@author 
\makeatletter
\newcommand{\togglepaper}[1][0]{ 
\bibliography{../Bibliographies/stmue,../localbibliography,
collection.bib}
  %% hyphenation points for line breaks
%% Normally, automatic hyphenation in LaTeX is very good
%% If a word is mis-hyphenated, add it to this file
%%
%% add information to TeX file before \begin{document} with:
%% %% hyphenation points for line breaks
%% Normally, automatic hyphenation in LaTeX is very good
%% If a word is mis-hyphenated, add it to this file
%%
%% add information to TeX file before \begin{document} with:
%% \include{localhyphenation}
\hyphenation{
A-la-hver-dzhie-va
ac-cu-sa-tive
anaph-o-ra
ana-phor
ana-phors
an-te-ced-ent
an-te-ced-ents
affri-ca-te
affri-ca-tes
ap-proach-es
Atha-bas-kan
Athe-nä-um
Be-schrei-bung
Bona-mi
Chi-che-ŵa
com-ple-ments
con-straints
Cope-sta-ke
Da-ge-stan
Dor-drecht
er-klä-ren-de
Flick-inger
Ginz-burg
Gro-ning-en
Has-pel-math
Jap-a-nese
Jon-a-than
Ka-tho-lie-ke
Ko-bon
krie-gen
Kroe-ger
Le-Sourd
moth-er
Mül-ler
Nie-mey-er
Ørs-nes
Par-a-digm
Prze-piór-kow-ski
phe-nom-e-non
re-nowned
Rie-he-mann
un-bound-ed
Ver-gleich
with-in
}

% listing within here does not have any effect for lfg.tex % 2020-05-14

% why has "erklärende" be listed here? I specified langid in bibtex item. Something is still not working with hyphenation.


% to do: check
%  Alahverdzhieva


% biblatex:

% This is a LaTeX frontend to TeX’s \hyphenation command which defines hy- phenation exceptions. The ⟨language⟩ must be a language name known to the babel/polyglossia packages. The ⟨text ⟩ is a whitespace-separated list of words. Hyphenation points are marked with a dash:

% \DefineHyphenationExceptions{american}{%
% hy-phen-ation ex-cep-tion }

\hyphenation{
A-la-hver-dzhie-va
ac-cu-sa-tive
anaph-o-ra
ana-phor
ana-phors
an-te-ced-ent
an-te-ced-ents
affri-ca-te
affri-ca-tes
ap-proach-es
Atha-bas-kan
Athe-nä-um
Be-schrei-bung
Bona-mi
Chi-che-ŵa
com-ple-ments
con-straints
Cope-sta-ke
Da-ge-stan
Dor-drecht
er-klä-ren-de
Flick-inger
Ginz-burg
Gro-ning-en
Has-pel-math
Jap-a-nese
Jon-a-than
Ka-tho-lie-ke
Ko-bon
krie-gen
Kroe-ger
Le-Sourd
moth-er
Mül-ler
Nie-mey-er
Ørs-nes
Par-a-digm
Prze-piór-kow-ski
phe-nom-e-non
re-nowned
Rie-he-mann
un-bound-ed
Ver-gleich
with-in
}

% listing within here does not have any effect for lfg.tex % 2020-05-14

% why has "erklärende" be listed here? I specified langid in bibtex item. Something is still not working with hyphenation.


% to do: check
%  Alahverdzhieva


% biblatex:

% This is a LaTeX frontend to TeX’s \hyphenation command which defines hy- phenation exceptions. The ⟨language⟩ must be a language name known to the babel/polyglossia packages. The ⟨text ⟩ is a whitespace-separated list of words. Hyphenation points are marked with a dash:

% \DefineHyphenationExceptions{american}{%
% hy-phen-ation ex-cep-tion }

  \memoizeset{
    memo filename prefix={hpsg-handbook.memo.dir/},
    % readonly
  }
  \papernote{\scriptsize\normalfont
    \@author.
    \titleTemp. 
    To appear in: 
    Stefan Müller, Anne Abeillé, Robert D. Borsley \& Jean-Pierre Koenig (eds.)
    HPSG Handbook
    Berlin: Language Science Press. [preliminary page numbering]
  }
  \pagenumbering{roman}
  \setcounter{chapter}{#1}
  \addtocounter{chapter}{-1}
}
\makeatother

\makeatletter
\newcommand{\togglepaperminimal}[1][0]{ 
  \bibliography{../Bibliographies/stmue,
                ../localbibliography,
collection.bib}
  %% hyphenation points for line breaks
%% Normally, automatic hyphenation in LaTeX is very good
%% If a word is mis-hyphenated, add it to this file
%%
%% add information to TeX file before \begin{document} with:
%% %% hyphenation points for line breaks
%% Normally, automatic hyphenation in LaTeX is very good
%% If a word is mis-hyphenated, add it to this file
%%
%% add information to TeX file before \begin{document} with:
%% \include{localhyphenation}
\hyphenation{
A-la-hver-dzhie-va
ac-cu-sa-tive
anaph-o-ra
ana-phor
ana-phors
an-te-ced-ent
an-te-ced-ents
affri-ca-te
affri-ca-tes
ap-proach-es
Atha-bas-kan
Athe-nä-um
Be-schrei-bung
Bona-mi
Chi-che-ŵa
com-ple-ments
con-straints
Cope-sta-ke
Da-ge-stan
Dor-drecht
er-klä-ren-de
Flick-inger
Ginz-burg
Gro-ning-en
Has-pel-math
Jap-a-nese
Jon-a-than
Ka-tho-lie-ke
Ko-bon
krie-gen
Kroe-ger
Le-Sourd
moth-er
Mül-ler
Nie-mey-er
Ørs-nes
Par-a-digm
Prze-piór-kow-ski
phe-nom-e-non
re-nowned
Rie-he-mann
un-bound-ed
Ver-gleich
with-in
}

% listing within here does not have any effect for lfg.tex % 2020-05-14

% why has "erklärende" be listed here? I specified langid in bibtex item. Something is still not working with hyphenation.


% to do: check
%  Alahverdzhieva


% biblatex:

% This is a LaTeX frontend to TeX’s \hyphenation command which defines hy- phenation exceptions. The ⟨language⟩ must be a language name known to the babel/polyglossia packages. The ⟨text ⟩ is a whitespace-separated list of words. Hyphenation points are marked with a dash:

% \DefineHyphenationExceptions{american}{%
% hy-phen-ation ex-cep-tion }

\hyphenation{
A-la-hver-dzhie-va
ac-cu-sa-tive
anaph-o-ra
ana-phor
ana-phors
an-te-ced-ent
an-te-ced-ents
affri-ca-te
affri-ca-tes
ap-proach-es
Atha-bas-kan
Athe-nä-um
Be-schrei-bung
Bona-mi
Chi-che-ŵa
com-ple-ments
con-straints
Cope-sta-ke
Da-ge-stan
Dor-drecht
er-klä-ren-de
Flick-inger
Ginz-burg
Gro-ning-en
Has-pel-math
Jap-a-nese
Jon-a-than
Ka-tho-lie-ke
Ko-bon
krie-gen
Kroe-ger
Le-Sourd
moth-er
Mül-ler
Nie-mey-er
Ørs-nes
Par-a-digm
Prze-piór-kow-ski
phe-nom-e-non
re-nowned
Rie-he-mann
un-bound-ed
Ver-gleich
with-in
}

% listing within here does not have any effect for lfg.tex % 2020-05-14

% why has "erklärende" be listed here? I specified langid in bibtex item. Something is still not working with hyphenation.


% to do: check
%  Alahverdzhieva


% biblatex:

% This is a LaTeX frontend to TeX’s \hyphenation command which defines hy- phenation exceptions. The ⟨language⟩ must be a language name known to the babel/polyglossia packages. The ⟨text ⟩ is a whitespace-separated list of words. Hyphenation points are marked with a dash:

% \DefineHyphenationExceptions{american}{%
% hy-phen-ation ex-cep-tion }

  \memoizeset{
    memo filename prefix={hpsg-handbook.memo.dir/},
    % readonly
  }
  \papernote{\scriptsize\normalfont
    \@author.
    \@title. 
    To appear in: 
    Stefan Müller, Anne Abeillé, Robert D. Borsley \& Jean-Pierre Koenig (eds.)
    HPSG Handbook
    Berlin: Language Science Press. [preliminary page numbering]
  }
  \pagenumbering{roman}
  \setcounter{chapter}{#1}
  \addtocounter{chapter}{-1}
}
\makeatother




% In case that year is not given, but pubstate. This mainly occurs for titles that are forthcoming, in press, etc.
\renewbibmacro*{addendum+pubstate}{% Thanks to https://tex.stackexchange.com/a/154367 for the idea
  \printfield{addendum}%
  \iffieldequalstr{labeldatesource}{pubstate}{}
  {\newunit\newblock\printfield{pubstate}}
}

\DeclareLabeldate{%
    \field{date}
    \field{year}
    \field{eventdate}
    \field{origdate}
    \field{urldate}
    \field{pubstate}
    \literal{nodate}
}

%\defbibheading{diachrony-sources}{\section*{Sources}} 

% if no langid is set, it is English:
% https://tex.stackexchange.com/a/279302
\DeclareSourcemap{
  \maps[datatype=bibtex]{
    \map{
      \step[fieldset=langid, fieldvalue={english}]
    }
  }
}


% for bibliographies
% biber/biblatex could use sortname field rather than messing around this way.
\newcommand{\SortNoop}[1]{}


% Doug Ball

\newcommand{\elist}{\q<\ \ \q>}

\newcommand{\esetDB}{\q\{\ \ \q\}}


\makeatletter

\newcommand{\nolistbreak}{%

  \let\oldpar\par\def\par{\oldpar\nobreak}% Any \par issues a \nobreak

  \@nobreaktrue% Don't break with first \item

}

\makeatother


% intermediate before Frank's trees are fixed
% This will be removed!!!!!
%\newcommand{\tree}[1]{} % ignore them blody trees
%\usepackage{tree-dvips}


\newcommand{\nodeconnect}[2]{}
\newcommand{\nodetriangle}[2]{}



% Doug relative clauses
%% I've compiled out almost all my private LaTeX command, but there are some
%% I found hard to get rid of. They are defined here.
%% There are few others which defined in places in the document where they have only
%% local effect (e.g. within figures); their names all end in DA, e.g. \MotherDA
%% There are a lot of \labels -- they are all of the form \label{sec:rc-...} or
%% \label{x:rc-...} or similar, so there should be no clashes.

% Subscripts -- scriptsize italic shape lowered by .25ex 
\newcommand{\subscr}[1]{\raisebox{-.5ex}{\protect{\scriptsize{\itshape #1\/}}}}
% A boxed subscript, for avm tags in normal text
\newcommand{\subtag}[1]{\subscr{\idx{#1}}}

%% Sets and tuples: I use \setof{} to get brackets that are upright, not slanted
%\newcommand{\setof}[1]{\ensuremath{\lbrace\,\mathit{#1}\,\rbrace}}
% 11.10.2019 EP: Doug requested replacement of existing \setof definition with the following:
%\newcommand{\setof}[1]{\begin{avm}\{\textcolor{red}{#1}\}\end{avm}}
% 31.1.2019 EP: Doug requested re-replacement of the above \textcolour version with the following:
\newcommand{\setof}[1]{\begin{avm}\{#1\}\end{avm}}

\newcommand{\tuple}[1]{\ensuremath{\left\langle\,\mbox{\textit{#1}}\,\right\rangle}}

% Single pile of stuff, optional arugment is psn (e.g. t or b)
% e.g. to put a over b over c in a centered column, top aligned, do:
%   \cPile[t]{a\\b\\c} 
\newcommand{\cPile}[2][]{%
  \begingroup%
  \renewcommand{\arraystretch}{.5}\begin{tabular}[#1]{@{}c@{}}#2\end{tabular}%
  \endgroup%
}

%% for linguistic examples in running text (`linguistic citation'):
\newcommand{\lic}[1]{\textit{#1}}

%% A gap marked by an underline, raised slightly
%% Default argument indicates how long the line should be:
\newcommand{\uGap}[1][3ex]{\raisebox{.25em}{\underline{\hspace{#1}}}\xspace}

%% \TnodeDA{XP}{avmcontents} -- in a Tree, put a node label next to an AVM
\newcommand{\TnodeDA}[2]{#1~\begin{avm}{#2}\end{avm}}

%% This allows tipa stuff to be put in \emph -- we need to change to cmr first.
%% It is used in the discussion of Arabic.
\newcommand{\emphtipa}[1]{{\fontfamily{cmr}\emph{\tipaencoding #1}}} 



 
 
\definecolor{lsDOIGray}{cmyk}{0,0,0,0.45}


% morphology.tex:
% Berthold

\newcommand{\dnode}[1]{\rnode{#1}{\fbox{#1}}}
\newcommand{\tnode}[1]{\rnode{#1}{\textit{#1}}}

\newcommand{\tl}[2]{#2}

\newcommand{\rrr}[3]{%
  \psframebox[linestyle=none]{%
    \avmoptions{center}
    \begin{avm}
      \[mud & \{ #1 \}\\
      ms & \{ #2 \}\\
      mph & \<  #3 \> \]
    \end{avm}
  }
}
\newcommand{\rr}[2]{%
  \psframebox[linestyle=none]{%
    \avmoptions{center}
    \begin{avm}
      \[mud & \{ #1 \}\\
      mph & \<  #2 \> \]
    \end{avm}
  }
}
 

% Frank Richter
\newtheorem{mydef}{Definition}

\long\def\set[#1\set=#2\set]%
{%
\left\{%
\tabcolsep 1pt%
\begin{tabular}{l}%
#1%
\end{tabular}%
\left|%
\tabcolsep 1pt%
\begin{tabular}{l}%
#2%
\end{tabular}%
\right.%
\right\}%
}

\newcommand{\einruck}{\\ \hspace*{1em}}


%\newcommand{\NatNum}{\mathrm{I\hspace{-.17em}N}}
\newcommand{\NatNum}{\mathbb{N}}
\newcommand{\Aug}[1]{\widehat{#1}}
%\newcommand{\its}{\mathrm{:}}
% Felix 14.02.2020
\DeclareMathOperator{\its}{:}

\newcommand{\sequence}[1]{\langle#1\rangle}

\newcommand{\INTERPRETATION}[2]{\sequence{#1\mathsf{U}#2,#1\mathsf{S}#2,#1\mathsf{A}#2,#1\mathsf{R}#2}}
\newcommand{\Interpretation}{\INTERPRETATION{}{}}

\newcommand{\Inte}{\mathsf{I}}
\newcommand{\Unive}{\mathsf{U}}
\newcommand{\Speci}{\mathsf{S}}
\newcommand{\Atti}{\mathsf{A}}
\newcommand{\Reli}{\mathsf{R}}
\newcommand{\ReliT}{\mathsf{RT}}

\newcommand{\VarInt}{\mathsf{G}}
\newcommand{\CInt}{\mathsf{C}}
\newcommand{\Tinte}{\mathsf{T}}
\newcommand{\Dinte}{\mathsf{D}}

% this was missing from ash's stuff.

%% \def \optrulenode#1{
%%   \setbox1\hbox{$\left(\hbox{\begin{tabular}{@{\strut}c@{\strut}}#1\end{tabular}}\right)$}
%%   \raisebox{1.9ex}{\raisebox{-\ht1}{\copy1}}}



\newcommand{\pslabel}[1]{}

\newcommand{\addpagesunless}{\todostefan{add pages unless you cite the
 work as such}}

% dg.tex
% framed boxes as used in dg.tex
% original idea from stackexchange, but modified by Saso
% http://tex.stackexchange.com/questions/230300/doing-something-like-psframebox-in-tikz#230306
\tikzset{
  frbox/.style={
    rounded corners,
    draw,
    thick,
    inner sep=5pt,
    anchor=base,
  },
}

% get rid of these morewrite messages:
% https://tex.stackexchange.com/questions/419489/suppressing-messages-to-standard-output-from-package-morewrites/419494#419494
\ExplSyntaxOn
\cs_set_protected:Npn \__morewrites_shipout_ii:
  {
    \__morewrites_before_shipout:
    \__morewrites_tex_shipout:w \tex_box:D \g__morewrites_shipout_box
    \edef\tmp{\interactionmode\the\interactionmode\space}\batchmode\__morewrites_after_shipout:\tmp
  }
\ExplSyntaxOff


% This is for places where authors used bold. I replace them by \emph
% but have the information where the bold was. St. Mü. 09.05.2020
\newcommand{\textbfemph}[1]{\emph{#1}}



% Felix 09.06.2020: copy code from the third line into localcommands.tex:
% https://github.com/langsci/langscibook#defined-environments-commands-etc
% Does not work with texlive 2020, is done with sed in Makefile
%\patchcmd{\mkbibindexname}{\ifdefvoid{#3}{}{\MakeCapital{#3} }}{\ifdefvoid{#3}{}{#3 }}{}{\AtEndDocument{\typeout{mkbibindexname could not be patched.}}}



\let\textnobf\textit
% instead of "in bold" write "in italics"
\newcommand{\bolddescriptionintext}{italics\xspace}

% Berthold
\newcommand{\mathplus}{+}
% \mbox{\normalfont +}}
\newcommand{\emdash}{--\xspace}
\newcommand{\emdashUS}{--\xspace}


% Stefan to get the space remvoed infront of the : in Bargmann NPN discussion
%\DeclareMathSymbol{:}{\mathord}{operators}{"3A}
% used {:\,} instead


% for cxg.tex needed for includonly to find the counter.
\newcounter{croftyears} 




% Needed for bibtex entry for Jackendoff's xbar syntax. Without it the bar would be off in itialics.

% https://tex.stackexchange.com/questions/95014/aligning-overline-to-italics-font/95079#95079
% \newbox\usefulbox

% \makeatletter
%     \def\getslant #1{\strip@pt\fontdimen1 #1}

%     \def\skoverline #1{\mathchoice
%      {{\setbox\usefulbox=\hbox{$\m@th\displaystyle #1$}%
%         \dimen@ \getslant\the\textfont\symletters \ht\usefulbox
%         \divide\dimen@ \tw@ 
%         \kern\dimen@ 
%         \overline{\kern-\dimen@ \box\usefulbox\kern\dimen@ }\kern-\dimen@ }}
%      {{\setbox\usefulbox=\hbox{$\m@th\textstyle #1$}%
%         \dimen@ \getslant\the\textfont\symletters \ht\usefulbox
%         \divide\dimen@ \tw@ 
%         \kern\dimen@ 
%         \overline{\kern-\dimen@ \box\usefulbox\kern\dimen@ }\kern-\dimen@ }}
%      {{\setbox\usefulbox=\hbox{$\m@th\scriptstyle #1$}%
%         \dimen@ \getslant\the\scriptfont\symletters \ht\usefulbox
%         \divide\dimen@ \tw@ 
%         \kern\dimen@ 
%         \overline{\kern-\dimen@ \box\usefulbox\kern\dimen@ }\kern-\dimen@ }}
%      {{\setbox\usefulbox=\hbox{$\m@th\scriptscriptstyle #1$}%
%         \dimen@ \getslant\the\scriptscriptfont\symletters \ht\usefulbox
%         \divide\dimen@ \tw@ 
%         \kern\dimen@ 
%         \overline{\kern-\dimen@ \box\usefulbox\kern\dimen@ }\kern-\dimen@ }}%
%      {}}
%     \makeatother




\newcommand{\acknowledgmentsEN}{Acknowledgements}
\newcommand{\acknowledgmentsUS}{Acknowledgments}

% to put two examples next to eachother
%\newcommand{\shortbox}[3][-.7]{
%    \parbox[t]{.4\textwidth}{
%      \vspace{#1\baselineskip} #2\strut~~ #3}%
%}

\newcommand{\twomulticolexamples}[2]{
\begin{tabular}[t]{@{}l@{~~}l@{\hspace{1em}}l@{~~}l@{}}
a. & \parbox[t]{.4\textwidth}{#1} & b. & \parbox[t]{.4\textwidth}{#2}\\
\end{tabular}
}




% This does a linebreak for \gll for long sentences leaving space for the language at the right
% margin.
% St.Mü. 17.06.2021
\newcommand{\longexampleandlanguage}[2]{%
\begin{tabularx}{\linewidth}[t]{@{}X@{}p{\widthof{(#2)}}@{}}%
\begin{minipage}[t]{\linewidth}%
#1%
\end{minipage} & (\ili{#2})%
\end{tabularx}}



\renewcommand{\indexccg}{\is{Categorial Grammar (CG)!Combinatorial \textasciitilde{} (CCG)}\xspace}
\newcommand{\indexccgstart}{\is{Categorial Grammar (CG)!Combinatorial \textasciitilde{} (CCG)|(}\xspace}
\newcommand{\indexccgend}{\is{Categorial Grammar (CG)!Combinatorial \textasciitilde{} (CCG)|)}\xspace}
\renewcommand{\indexmp}{\is{Minimalism}\xspace}


\newcommand{\gisu}{Giuseppe Varaschin\xspace}

\newcommand{\NPi}{NP$\mkern-1mu_i$\xspace}
\newcommand{\NPj}{NP$\mkern-1.5mu_j$\xspace}
  %% -*- coding:utf-8 -*-

%%%%%%%%%%%%%%%%%%%%%%%%%%%%%%%%%%%%%%%%%%%%%%%%%%%%%%%%%%%%
%
% gb4e

% fixes problem with to much vertical space between \zl and \eal due to the \nopagebreak
% command.
\makeatletter
\def\@exe[#1]{\ifnum \@xnumdepth >0%
                 \if@xrec\@exrecwarn\fi%
                 \if@noftnote\@exrecwarn\fi%
                 \@xnumdepth0\@listdepth0\@xrectrue%
                 \save@counters%
              \fi%
                 \advance\@xnumdepth \@ne \@@xsi%
                 \if@noftnote%
                        \begin{list}{(\thexnumi)}%
                        {\usecounter{xnumi}\@subex{#1}{\@gblabelsep}{0em}%
                        \setcounter{xnumi}{\value{equation}}}
% this is commented out here since it causes additional space between \zl and \eal 06.06.2020
%                        \nopagebreak}%
                 \else%
                        \begin{list}{(\roman{xnumi})}%
                        {\usecounter{xnumi}\@subex{(iiv)}{\@gblabelsep}{\footexindent}%
                        \setcounter{xnumi}{\value{fnx}}}%
                 \fi}
\makeatother

% the texlive 2020 langsci-gb4e adds a newline after \eas, the texlive 2017 version was OK.
% \makeatletter
% \def\eas{\ifnum\@xnumdepth=0\begin{exe}[(34)]\else\begin{xlist}[iv.]\fi\ex\begin{tabular}[t]{@{}p{.98\linewidth}@{}}}
% \makeatother



%%%%%%%%%%%%%%%%%%%%%%%%%%%%%%%%%%%%%%%%%%%%%%%%%%%%%%%%%%
%
% biblatex

% biblatex sets the option autolang=hyphens
%
% This disables language shorthands. To avoid this, the hyphens code can be redefined
%
% https://tex.stackexchange.com/a/548047/18561

\makeatletter
\def\hyphenrules#1{%
  \edef\bbl@tempf{#1}%
  \bbl@fixname\bbl@tempf
  \bbl@iflanguage\bbl@tempf{%
    \expandafter\bbl@patterns\expandafter{\bbl@tempf}%
    \expandafter\ifx\csname\bbl@tempf hyphenmins\endcsname\relax
      \set@hyphenmins\tw@\thr@@\relax
    \else
      \expandafter\expandafter\expandafter\set@hyphenmins
      \csname\bbl@tempf hyphenmins\endcsname\relax
    \fi}}
\makeatother


% the package defined \attop in a way that produced a box that has textwidth
%
\def\attop#1{\leavevmode\begin{minipage}[t]{.995\linewidth}\strut\vskip-\baselineskip\begin{minipage}[t]{.995\linewidth}#1\end{minipage}\end{minipage}}


%%%%%%%%%%%%%%%%%%%%%%%%%%%%%%%%%%%%%%%%%%%%%%%%%%%%%%%%%%%%%%%%%%%%


% Don't do this at home. I do not like the smaller font for captions.
% This does not work. Throw out package caption in langscibook
% \captionsetup{%
% font={%
% stretch=1%.8%
% ,normalsize%,small%
% },%
% width=\textwidth%.8\textwidth
% }
% \setcaphanging


  \togglepaper[34]
}{}

\author{Stefan Müller\orcid{0000-0003-4413-5313}\affiliation{Humboldt-Universität zu Berlin}}
\title{HPSG and Construction Grammar}


\abstract{This chapter discusses the main tenets of Construction Grammar (CxG) and shows that HPSG adheres to them. The discussion includes surface orientation, language acquisition without UG, and inheritance networks and shows how HPSG (and other frameworks) are positioned along these dimensions. Formal variants of CxG will be briefly discussed and their relation to HPSG will be pointed out. It is argued that lexical representations of valence are more appropriate than phrasal approaches, which are assumed in most variants of CxG. Other areas of grammar seem to require headless phrasal constructions (\eg the NPN construction and certain extraction constructions) and it is shown how HPSG handles these. Derivational morphology is discussed as a further example of an early constructionist analysis in HPSG.}


\begin{document}
\maketitle
\label{firstpage-cxg}\label{chap-cxg}
\indexcxgstart


%% \inlinetodostefan{
%% BobB: No doubt a ‘what you see is what you get’ is an important feature of the various forms of Construction Grammar, but it seems to me that the insistence that a few very general combinatorial mechanisms are not sufficient is more important and I would be inclined to emphasize this.}

\noindent
This chapter deals with Construction Grammar (CxG) and its relation to Head-Driven Phrase Structure
Grammar (HPSG). The short version of the message is: HPSG is a Construction Grammar.\footnote{%
  This does not mean that HPSG is not a lot of other things at the same time. For instance, it is
  also a Generative Grammar in the sense of \citet[\page 4]{Chomsky65a}, that is, it is explicit and
  formalized. HPSG is also very similar to Categorial Grammar (\citealt{MuellerUnifying};
  \crossrefchapteralt{cg}). Somewhat ironically, Head-Driven Phrase Structure Grammar is not entirely
  head-driven anymore (see Section~\ref{sec-npn}), nor is it a phrase structure grammar \crossrefchapterp{formal-background}.
}
It had constructional properties right from the beginning\label{page-HPSG-always-was-constructional} and over the years -- due to influence by
Construction Grammarians like Fillmore and Kay -- certain aspects were adapted, making it possible to better
capture generalizations over phrasal patterns. In what follows I will first say what Construction
Grammars are (Section~\ref{sec-cxg}), and I will explain why HPSG as developed in \citew{ps,ps2} was a
Construction Grammar and how it was changed to become even more Constructive
(Section~\ref{sec-inheritance}). Section~\ref{sec-valence} deals with so"=called argument structure
constructions, which are usually dealt with by assuming phrasal constructions in CxG, and explains
why this is problematic and why lexical approaches are more appropriate. Section~\ref{sec-cxg-morphology} explains
Construction Morphology, Section~\ref{sec-phrasal} shows how cases that should be treated phrasally
can be handled in HPSG, and Section~\ref{sec-summary} sums up the chapter.

\section{What is Construction Grammar?}
\label{sec-cxg}

%\inlinetodostefan{Add Fillmore's insistance that there are grammatical constructions in the sense
%  Chomsky rejected. Mention What is X doing Y and way-construction.}

Construction Grammar was developed as a theory that can account for non-regular phenomena as observed
in many idioms \citep*{FKoC88a}. It clearly set itself apart from theories like Government \& Binding\indexgb
\citep{Chomsky81a}, which assumes very abstract schemata for the combination of lexical items (\xbar
rules). The argument was that grammatical constructions are needed to capture irregular phenomena
and their interaction with more regular ones. In contrast, \citet[\page 7]{Chomsky81a} considered
rules for passive or relative clauses as epiphenomenal; everything was supposed to follow from
general principles.\footnote{
  The passive in GB is assumed to follow from suppression of case assignment and the Case Filter,
  which triggers movement of the object to SpecIP. The important part of the analysis is the combination of the verb stem
  with the passive morphology. This is where suppression of case assignment takes place. This
  morphological part of the analysis corresponds to the Passive Construction in theories like HPSG
  and SBCG: a lexical rule (\citealt[\page 215]{ps}; \citealt{Mueller2003e}; \citealt{MOe2013a}; \crossrefchapteralt[Section~\ref{arg-st:sec-passives}]{arg-st}). So in a sense there is a Passive Construction in GB as well.%
} According to Chomsky, grammars consisted of a set of general combinatorial rules
and some principles. The Minimalist Program \citep{Chomsky95a-u} is even more radical, since only two combinatorial rules
are left (External and Internal Merge). Various forms of CxG object to this view and state that
several very specific phrasal constructions are needed in order to account for language in its
entirety and full complexity. Phenomena for which this is true will be discussed in
Section~\ref{sec-phrasal}. However, the case is not as clear in general, since one of the authors of
\citet*{FKoC88a} codeveloped a head"=driven, lexical theory of idioms that is entirely compatible with the abstract rules of
Minimalism \citep*{Sag2007a,KSF2015a,KM2019a}. This theory will be discussed in
Section~\ref{sec-locality}. Of course, the more recent lexical theory of idioms is a constructional
theory as well. So the first question to answer in a chapter like this is: what is a construction in
the sense of Construction Grammar? What is Construction Grammar? While it is
relatively clear what a construction is, the answer to the question regarding Construction Grammar
is less straight-forward (see also \citealt[\page 35]{Fillmore88a} on this). Section~\ref{sec-def-construction} provides the definition for the term
\emph{construction} and Section~\ref{sec-tenets} states the tenets of CxG and discusses to what
extent the main frameworks currently on the market adhere to them.

\subsection{What is a construction}
\label{sec-def-construction}

\citet*{FKoC88a}\is{comparative correlative|(} discuss sentences like (\mex{1}) and notice that they pose puzzles for standard
accounts of syntax and the syntax/semantics interface.
\eal
\ex The more carefully you do your work, the easier it will get.
\ex I wouldn't pay five dollars for it, let alone ten dollars.
\zl
The \emph{the} \suffix{er} \emph{the} \suffix{er} Construction is remarkable, since it combines aspects of normal syntax
(clause structure and extraction) with idiosyncratic aspects like the special use of
\emph{the}. In (\mex{0}a) the adverb phrase \emph{more carefully} does not appear to the left of
\emph{work} but is fronted and \emph{the} appears without a noun. The second clause in (\mex{0}a) is structured
in a parallel way. There have to be two of these \emph{the} clauses to form the respective
construction. \citet*{FKoC88a} extensively discuss the properties of \emph{let alone}, which are
interesting for syntactic reasons (the fragments following \emph{let alone}) and for semantic and information
structural reasons. I will not repeat the discussion here but refer the reader to the
paper.\footnote{
  For an analysis of comparative correlative constructions as in (\mex{0}a) in HPSG, see
  \crossrefchaptert[Section~\ref{coord:sec-comparative-correlatives}]{coordination} and the papers
  cited there.
}\is{comparative correlative|)}

In later papers, examples like (\mex{1}) were discussed:
\eal
\ex What is this scratch doing on the table? \hfill\citep[\page 3]{KF99a}
\ex Frank dug his way out of prison. \hfill\citep[\page 199]{Goldberg95a}
\zl
Again, the semantics of the complete sentences is not in an obvious relation to the material
involved. The question in (\mex{0}a) is not about a scratch's actions, but rather the question is why
there is a scratch. Similarly, (\mex{0}b) is special in that there is a directional PP that does not
normally go together with verbs like \emph{dug}. It is licensed by \emph{way} in combination with a
possessive pronoun.

\citet{FKoC88a}, \citet{Goldberg95a}, \citet{KF99a} and Construction Grammarians in general argue
that the notion of ``construction'' is needed for adequate models of grammar, that is, for models of
grammar that are capable of analyzing the examples above. \citet[\page 501]{FKoC88a} define
\term{construction} as follows:

\begin{quote}\label{def-construction-fillmore}
Constructions on our view are much like the nuclear family (mother plus daughters) subtrees
admitted by phrase structure rules, \textsc{except} that (1) constructions need not be limited to a mother
and her daughters, but may span wider ranges of the sentential tree; (2) constructions may specify,
not only syntactic, but also lexical, semantic, and pragmatic information; (3) lexical items,
being mentionable in syntactic constructions, may be viewed, in many cases at least, as
constructions themselves; and (4) constructions may be idiomatic in the sense that a large
construction may specify a semantics (and/""or pragmatics) that is distinct from what might be
calculated from the associated semantics of the set of smaller constructions that could be used to
build the same morphosyntactic object. \citep[\page 501]{FKoC88a}
\end{quote}
A similar definition can be found in Goldberg's work. \citet[\page 5]{Goldberg2006a} defines \isi{construction} as follows:
\begin{quote}
Any linguistic pattern is recognized as a construction as long as some aspect of its form or
function is not strictly predictable from its component parts or from other constructions recognized
to exist. In addition, patterns are stored as constructions even if they are fully predictable as
long as they occur with sufficient frequency. \citep[\page 5]{Goldberg2006a}
\end{quote}
The difference between this definition and earlier definitions by her and others is that patterns
that are stored because of their frequencies are included. This addition is motivated by
psycholinguistic findings that show that forms may be stored even though they are fully regular and
predictable \parencites{Bybee95a-u}[\page 228]{PJ2005a}.
%% Bob:
%% > p,12, Goldberg quote. I assume not all forms of Construction
%% > Grammar would assume that ‘patterns are stored as constructions even if
%% > they are fully predictable as long as they occur with sufficient
%% > frequency.’ If this is right, perhaps it is worth saying.
%

\largerpage
Goldberg provides Table~\ref{tab-constructions} as examples of constructions.
%\inlinetodostefan{BB: I understand her view is that constructions must have a distinctive meaning. I
%  believe Jackendoff argues against this, and Newmeyer and I did here: \citew{BN2009a}}
\begin{table}[b]
\oneline{%
\begin{tabular}{ll}\lsptoprule
Word                    & e.g., \emph{tentacle}, \emph{gangster}, \emph{the} \\
Word (partially filled) & e.g., \emph{post}-N, V-\emph{ing}\\
Complex word            & e.g., \emph{textbook}, \emph{drive-in}\\
Idiom (filled)          & e.g., \emph{like a bat out of hell}\\
Idiom (partially filled) & e.g., \emph{believe} <one’s> \emph{ears/eyes}\\
Covariational Conditional\is{comparative correlative} &  The Xer the Yer\\
                          & (e.g., \emph{The more you watch the less you know})\\
Ditransitive              &  Subj V Obj1 Obj2\\
                          & (e.g., \emph{She gave him a kiss};
                           \emph{He fixed her some fish tacos.})\\
Passive                   & Subj aux VPpp ( PPby )\\
% Subj kann bei Kontrollkonstruktionen entfallen
                          & (e.g., \emph{The cell phone tower was struck by lightning.})\\\lspbottomrule
\end{tabular}}
\caption{\label{tab-constructions}Examples of constructions, varying in size and complexity
  according to \citet[\page 94]{Goldberg2009b-u}}
\end{table}
In addition to such constructions with a clear syntax-semantics or syntax-function relation,
\pagebreak\citet[\page 453]{Goldberg2013a} assumes a rather abstract VP construction specifying ``statistical
constraints on the ordering of postverbal complements, dependent on weight and information
structure''.

If one just looks at Goldberg's definition of construction, all theories currently on the market could be
regarded as Construction Grammars. As Peter Staudacher\aimention{Peter Staudacher} pointed out in the discussion after a talk by
Knud Lambrecht\aimention{Knud Lambrecht} in May 2006 in Potsdam, lexical items are form-meaning pairs and the rules of
phrase structure grammars come with specific semantic components as well, even if it is just
\isi{functional application}.
%% \footnote{
%% For example, the semantics of \emph{all children} is computed from the meaning contribution of
%% \emph{all} and \emph{children}. Which term is applied to which has to be determined by the grammar
%% rules somehow. In a rule for NPs combining a determiner with a noun the nominal semantics has to be
%% inserted into the determiner semantics:
%% \eal
%% \ex
%% \ex
%% \zl
%% }
So, \cg, GB-style\indexgb theories paired with semantics \citep{HK98a-u}, \gpsg, \treeag,
\lfg, HPSG, and even \minimalism would be Construction Grammars. If one looks at the
examples of constructions in Table~\ref{tab-constructions}, things change a bit. Idioms are generally
not the focus of work in \isi{Mainstream Generative Grammar (MGG)}.\footnote{
  The term \emph{Mainstream Generative Grammar} is used to refer to work in Transformational
  Grammar, for example Government \& Binding \citep{Chomsky81a} and Minimalism
  \citep{Chomsky95a-u}. Some authors working in Construction Grammar see themselves in the tradition of
  Generative Grammar in a wider sense, see for example \citew*[\page 501]{FKoC88a} and \citew[\page 36]{Fillmore88a}.%
} MGG is usually concerned with explorations of the so-called Core\is{core grammar} Grammar as opposed to the Periphery\is{periphery}, to which the
idioms are assigned.
%\todostefan{Anne: \citew{CvCHHLT2018a-ed}}
The Core Grammar is the part of the grammar that is supposed to be acquired with
help of innate domain specific knowledge, something whose existence Construction Grammar
denies. But if one takes \citet*{HCF2002a} seriously and assumes that only the ability to form complex
linguistic objects out of less complex linguistic objects (\isi{Merge}) is part of this innate knowledge,
then the core/periphery distinction does not have much content and after all, Minimalists could adopt
a version of Sag's local, selection-based analysis of idioms\is{idiom}
\citep*{Sag2007a,KSF2015a,KM2019a} and in fact, some did: \citew{Everaert2010a-u} and G.\
\citew[\page 21]{GMueller2011a}.\footnote{
  See also \crossrefchapterw[Section~\ref{idioms:recent-lexical-approaches}]{idioms} on lexical approaches to idioms.
}
However, as is discussed in the next subsection, there are other aspects that really set Construction Grammar apart from
MGG.

\subsection{Basic tenets of Construction Grammar}
\label{sec-tenets}

%\largerpage
\enlargethispage{7pt}
\citet[\page 219]{Goldberg2003b} names the following tenets as core assumptions standardly made in CxG:
\begin{description}
\item[Tenet 1] All levels of description are understood to involve pairings of form with semantic or discourse function, including morphemes or words, idioms, partially lexically filled and fully abstract phrasal patterns. (See Table~\ref{tab-constructions}.)
\item[Tenet 2] An emphasis is placed on subtle aspects of the way we conceive of events and states of
affairs.
\item[Tenet 3] A ``what you see is what you get'' approach to syntactic form is adopted: no underlying levels
  of syntax or any phonologically empty elements are posited.
\item[Tenet 4] Constructions are understood to be learned on the basis of the input and general cognitive mechanisms (they are constructed), and are expected to vary cross-linguistically.
\item[Tenet 5] Cross-linguistic generalizations are explained by appeal to general cognitive constraints together with the functions of the constructions involved.
\item[Tenet 6] Language-specific generalizations across constructions are captured via inheritance networks much like those that have long been posited to capture our non-linguistic knowledge.
\item[Tenet 7] The totality of our knowledge of language is captured by a network of constructions: a ``constructicon''.
\end{description}

I already commented on Tenet~1 above. Tenet~2 concerns semantics and the syntax-semantics interface,
which are part of most HPSG analyses. In what follows I want to look in more detail at the other
tenets. Something that is not mentioned in Goldberg's tenets but is part of the definition of
construction by \citet[\page 501]{FKoC88a} is the non-locality of constructions. I will comment on
this in a separate subsection.

\subsubsection{Surface orientation and empty elements}

Tenet~3 requires a surface-oriented approach. Underlying levels and phonologically empty elements
are ruled out. This excludes derivational models of transformational syntax assuming an underlying
structure (the so-called \isi{D-structure}) and some derived structure or more recent derivational variants of Minimalism. There was a time
where representational models of Government \& Binding\indexgb (GB, \citealt{Chomsky81a}) did not assume a D-structure but just one structure with
traces (Koster \citeyear[\page ]{Koster78b-u}; \citeyear[\page 235]{Koster87a-u};
%\citealp[\page 66, Fußnote~4]{Bierwisch83a};
\citealp{KT91a}; \citealp[Section~1.4]{Haider93a};
\citealp[\page 14]{Frey93a}; \citealp[\page 87--88, 177--178]{Lohnstein93a-u}; \citealp[\page
  38]{FC94a}; \citealp[\page 58]{Veenstra98a}). Some of these analyses are rather similar to HPSG
analyses as they are assumed today \citep{Kiss95a,BvN98a,Meurers2000b,Mueller2005c,MuellerGS,MuellerGermanic}. Chomsky's Minimalist work \citep{Chomsky95a-u} assumes a derivational model and comes
with a rhetoric of building structure in a bottom-up way and sending complete Phases to the
interfaces for pronunciation and interpretation. This is incompatible with Tenet~3, but in principle,
Minimalist approaches are very similar to Categorial Grammar, so there could be representational
approaches adhering to Tenet~3.\footnote{%
  There is a variant of \isi{Minimalist Grammars} \citep{Stabler2010b}, namely \isi{Top-down Phase-based Minimalist Grammar
(TPMG)} as developed by \citet{Chesi2004a-u,Chesi2007a-u} and \citet{BC2006a-u,BC2012a-u}. There is
  no movement in TPMG. Rather, \emph{wh}-phrases are linked to their ``in situ'' positions with the
aid of a short-term memory  buffer that functions like a stack. See
also \citet{Hunter2010a-u,Hunter2019a-u} for a related account where the
information about the presence of a \emph{wh}-phrase is percolated in the
syntax tree, like in GPSG\indexgpsg/HPSG. For a general comparison of Minimalist grammars and HPSG, see
\citew[Section~2.3]{MuellerUnifying} and \citew[177--180]{MuellerGT-Eng4}, which includes the discussion of a more
recent variant suggested by \citet{Torr2019a-u}.
}

A comment on empty elements is in order: all articles introducing Construction Grammar state that
CxG does not assume empty elements. Most of the alternative theories do use empty elements: see
\citet{KoenigE99a-u} on Categorial Grammar, \citet*[\page 143]{GKPS85a} on GPSG, \citet[\page
  67]{Bresnan2001a} on LFG, \citew{Bender2000a} and \citew*[\page 464]{SWB2003a} on HPSG/Sign-Based
Construction Grammar. There are results from the 60s that show that phrase structure grammars
containing empty elements can be translated into grammars that do not contain empty elements
\citep*[\page 153, Lemma~4.1]{BHPS61a} and sure enough there are versions of GPSG \citep[\page
  76--77]{Uszkoreit87a}, LFG \citep{KZ89a,DKK2001a-u}, and HPSG \parencites{BMS2001a}[\page 508]{Sag2010b} that do not use empty elements. Grammars with empty elements often are more compact than
those without empty elements and express generalizations more directly. See for example
\citet{Bender2000a} for copulaless sentences in African American Vernacular
English\il{English!African American Vernacular} and \citet{Mueller2004e} on nounless NPs in
\ili{German}. The argument against empty elements usually refers to language \isi{acquisition}: it
is argued that empty elements cannot be learned since they are not detectable in the input. However,
if the empty elements alternate with visible material, it can be argued that what is learned is the
fact that a certain element can be left out. What is true, though, is that things like empty
expletives cannot be learned since these empty elements are neither visible nor do they contribute to
meaning. Their only purpose in grammars is to keep uniformity. For example, \citet{Grewendorf93}
working in GB suggests an analysis of the passive in \ili{German} that is parallel to the movement-based analysis of \ili{English}
passives \citep[\page 124]{Chomsky81a}. In order to account for the fact that the subject does not move to
initial position in \ili{German}, he suggests an empty expletive pronoun that takes the subject position and that is connected to the original
non-moved subject. Such elements cannot be acquired without innate knowledge about the IP/VP system and
constraints about the obligatory presence of subjects. The CxG criticism is justified here.

A frequent argumentation for empty elements in MGG is based on the fact that there are overt
realizations of an element in other languages (\eg object agreement\is{agreement!object} in \ili{Basque} and \isi{focus} markers in
\ili{Gungbe}). But since there is no language-internal evidence for these empty elements, they cannot be
learned and one would have to assume that they are innate. This kind of empty element is rightly
rejected (by proponents of CxG and others).

Summing up, it can be said that all grammars can be turned into grammars without empty elements and
hence fulfill Tenet~3. It was argued that the reason for assuming Tenet~3 (problems in language
acquisition) should be reconsidered and that a weaker form of Tenet~3 should be assumed: empty
elements are forbidden unless there is language-internal evidence for them. This revised version of
Tenet~3 would allow one to count empty element versions of CG, GPSG, LFG, and HPSG among Construction Grammars.



\subsubsection{Language acquisition without the assumption of UG}

Tenets~4 and~5\is{acquisition|(} are basically what everybody should assume in MGG if \citet*{HCF2002a} are taken seriously. Of
course, this is not what is done in large parts of the field. The most extreme variant is
\citet{CR2010a}, who assume at least 400 functional heads (p.\,57) being part of \isi{Universal Grammar
(UG)} and being present in the grammars of all languages, although sometimes
invisibly (p.\,55). Such assumptions beg the question why the genera of \ili{Bantu} languages should be part of our
genome\is{genetics} and how they got there.
Researchers working on language \isi{acquisition} realized that the \isi{Principles \& Parameters}
approach \citep{Meisel95a} makes wrong predictions. They now talk about \isi{Micro-Cues} instead of
parameters \citep{Westergaard2014a} and these Micro-Cues are just features that can be
learned. However, Westergaard still assumes that the features are determined by UG, a dubious
assumption seen from a CxG perspective (and from the perspective of \citeauthor*{HCF2002a} and
genetics in general; \citealt{Bishop2002a}).

Note that even those versions of Minimalism that do not follow the Rizzi-style Cartographic
approaches are far from being minimalist in their assumptions. Some distinguish between strong
and weak features, some assume enumerations\is{numeration} of lexical items from which a particular derivation
draws its input, and some assume that all movement has to be feature"=driven. Still others assume that
derivations work in so-called Phases\is{Phase} and that a Phase, once completed, is ``shipped to the
interfaces''. Construction of Phases is bottom-up, which is incompatible with psycholinguistic
results (see also \citealt[Section~\ref{sec-minimalism-processing}]{chapters/minimalism}, Chapter~\ref{chap-minimalism} in this
volume). None of these assumptions is a natural one to make from a
language acquisition point of view. Most of these assumptions do not have any empirical motivation; the
only motivation usually given is that they result in ``restrictive theories''. But if there is no
motivation for them, this means that the respective architectural assumptions have to be part of our
innate domain-specific knowledge, which is implausible according to \citet*{HCF2002a}.

As research in computational linguistics shows, our input is rich
enough to form classes, to determine the part of speech of lexical items, and even to infer syntactic
structure thought to be underdetermined by the input. For instance, \citet{Bod2009a} shows that the
classical auxiliary inversion examples that Chomsky still uses in his \isi{Poverty of the Stimulus}
arguments (\citealp[\page 29--33]{Chomsky71a-u}; \citealp*{BPYC2011a}) can also be learned from language input available to children. See also
\citew{FPG2006a,FPAG2007a} on input-based language acquisition.

HPSG does not make any assumptions about complicated mechanisms like feature"=driven movement and so
on.
%\todostefan{Bob: You could say that HPSG is feature"=driven. VALENCE features
%are central to determining what structures are grammatical.}
HPSG states properties of linguistic objects like part of speech, case, gender, etc., and
states relations between features like agreement and government. In this respect it is like
other Construction Grammars and hence experimental results regarding theories of language
acquisition can be carried over to HPSG. See also \crossrefchaptert[Section~\ref{sec-acquisition-minimalism}]{minimalism} on language
acquisition.\is{acquisition|)}


\subsubsection{Inheritance networks}
\label{sec-inheritance}

This leaves us with Tenets~6 and~7, that is, \emph{inheritance networks} and the
constructicon. Inheritance is something that is used in the classification of knowledge. For example,
the word \emph{animal} is very general and refers to entities with certain properties. There are
subtypes of this kind of entity: \emph{mammal} and further subtypes like \emph{mouse}. In inheritance
hierarchies, the knowledge of superconcepts is not restated at the subconcepts but instead, the superconcept
is referred to. This is like Wikipedia: the Wikipedia entry of \emph{mouse} states that mice
are mammals without listing all the information that comes with the concept of mammal. Such
inheritance hierarchies can be used in linguistics as well. They can be used to classify roots,
words, and phrases. An example of such a hierarchy used for the classification of adjectives and
adjectival derivation is discussed in Section~\ref{sec-cxg-morphology}. See also
\crossrefchaptert[Section~4]{lexicon} on inheritance in the lexicon.

MGG does not make reference to inheritance hierarchies. HPSG\label{cxg:page-HPSG-inheritance} did
this right from the beginning in 1985 \citep*{FPW85a} for lexical items and since 1995 also for
phrasal constructions \citep{Sag97a}. LFG rejected the use of types, but used macros in computer
implementations. The macros were abbreviatory devices specific to the implementation and did not have any theoretical importance.
This changed in 2004, when macros were suggested in theoretical work \citep*{DKK2004a}. And
although any connection to constructionist work is vehemently denied by some of the authors, recent
work in LFG has a decidedly constructional flavor \citep*{ADT2008a,AGT2014a}.\footnote{ See
  \citew[\page 516]{Toivonen2013a} for an explicit reference to construction-specific phrase
  structure rules in the sense of Construction Grammar. See \citew{MuellerLFGphrasal} for a
  discussion of phrasal LFG approaches.} LFG differs from frameworks like HPSG,
though, in assuming a separate level of c-structure. c-structure rules are basically context-free
phrase structure rules, and they are not modeled by feature value pairs (but there is a
model-theoretic formalization; \citealt[\page 12]{Kaplan95a}). This means that it is not possible to
capture a generalization regarding lexical items, lexical rules, and phrasal schemata, or any
two-element subset of these three kinds of objects. While HPSG describes all of these elements with
the same inventory and hence can use common supertypes in the description of all three, this is not
possible in LFG \citep[Section~23.1]{MuellerGT-Eng2}.\footnote{ One could use templates
  \citep{DKK2004a,ADT2013a} to specify properties of lexical items and of mother nodes in
  c-structure rules, but usually c-structure rules specify the syntactic categories of mothers and
  daughters, so this information has a special status within the c-structure rules.  } For example,
\citet{Hoehle97a} argued that complementizers and finite verbs in initial position in \ili{German}
form a natural class. HPSG can capture this since complementizers (lexical elements) and finite
verbs in initial position (results of lexical rule applications or a phrasal schema, see \crossrefchapteralt[Section~\ref{sec-head-movement}]{order}) can have a
common supertype.  TAG is also using inheritance in the Meta Grammar \citep{LK2017a}.

\largerpage
Since HPSG's lexical entries, lexical rules, and phrasal schemata are all described by typed feature
descriptions, one could call the set of these descriptions the \isi{constructicon}. Therefore,
Tenet~7 is also adhered to.

\subsubsection{Non-locality}
\label{sec-non-locality}

\citet*[\page 501]{FKoC88a} stated in their definition of constructions that constructions may
involve more than mothers and immediate daughters (see p.\,\pageref{def-construction-fillmore}
below).\footnote{%
  This subsection is based on a much more thorough discussion of locality and SBCG in
  \citew[Section~10.6.2.1.1 and Section~18.2]{MuellerGT-Eng1}.
} That is, daughters of daughters can be specified
as well. A straightforward example of such a specification is given in Figure~\ref{fig-take-into-account-TAG}, which shows the
  TAG analysis of the idiom \emph{take into account} following \citet[\page 7]{AS89a}.
\begin{figure}
\centering
\begin{forest}
tag
[S
	[NP$\downarrow$]
	[VP
		[V
			[takes]]
		[NP$\downarrow$]
		[PP\sub{NA}
			[P
				[into]]
			[NP\sub{NA}
				[N\sub{NA}
					[account]]]]]]
\end{forest}
\caption{\label{fig-take-into-account-TAG}TAG tree for \emph{take into account} by \citet[\page 7]{AS89a}}
\end{figure}%
The fixed parts of the idiom are just stated in the tree. NP$\downarrow$ stands for an open slot
into which an NP has to be inserted. The subscript NA says that adjunction to the respectively
marked nodes is forbidden. Theories like Constructional HPSG can state such complex tree structures
like TAG can. Dominance relationships are modeled by feature structures in HPSG and it is possible
to have a description that corresponds to Figure~\ref{fig-take-into-account-TAG}. The NP slots would
just be left underspecified and can be filled in models that are total (see
\citealt{Richter2007a} and \crossrefchapteralt{formal-background} for formal foundations of HPSG).

\largerpage\enlargethispage{3pt}
It does not come without some irony that the theoretical approach that was developed out of Berkeley
Construction Grammar and Constructional HPSG, namely Sign-Based Construction Grammar
\citep*{SBK2012a,Sag2012a}, is strongly local: it is made rather difficult to access daughters of
daughters \citep{Sag2007a}. So, if one would stick to the early definition, this would rule out SBCG
as a Construction Grammar. Fortunately, this is not justified. First, there are ways to establish
nonlocal selection (see Section~\ref{sec-locality}) and second, there are ways to analyze idioms
locally. \citet{Sag2007a}, \citet*{KSF2015a}, and \citet{KM2019a} develop a theory of idioms that is
entirely based on local selection.\footnote{
  Of course this theory is also compatible with any other variant of HPSG. As
  \crossrefchaptert[\page \pageref{evolution:page-sbcg-vs-lingo}]{evolution} point out, it was part of the
  grammar fragment that has been developed at the CSLI by Dan Flickinger \citep*{FCS2000a,Flickinger2000a,Flickinger2011a-u} years
  before the SBCG manifesto was published.
} For example, for \emph{take into account}, one can state that
\emph{take} selects two NPs and a PP with the fixed lexical material \emph{into} and
\emph{account}. The right form of the PP is enforced by means of the feature \textsc{lexical
  identifier} (\textsc{lid}\isfeat{lid}). A special word \emph{into} with the \textsc{lid} value \type{into} is
specified as selecting a special word \emph{account}. What is done in TAG via direct specification
is done in SBCG via a series of local selections of specialized lexical items. The interesting (intermediate)
conclusion is: if SBCG can account for idioms via local selection, then theories like Categorial
Grammar and Minimalism can do so as well. So, they cannot be excluded from Construction Grammars on
the basis of arguments concerning idioms and non-locality of selection.

However, there may be cases of idioms that cannot be handled via local selection.
For example, \citew{RS2009a} discuss the following idiom:
\ea
\label{mich-tritt-ein-Pferd}
\gll glauben, X\_Acc tritt ein Pferd\\
     believe  X     kicks a horse\\
\glt `be utterly surprised'
\z
\largerpage
The X"=constituent has to be a pronoun that refers to the subject of the matrix clause. If this is not the case, the sentence becomes ungrammatical or loses its idiomatic meaning.
\eal
\ex[]{
\gll Ich glaube, mich      /  \# dich tritt ein Pferd.\\
     I   believe me.\acc{} {} {} you.\acc{} kicks a horse\\
}
\ex[]{
\gll Jonas glaubt, ihn  tritt ein Pferd.\footnotemark\\
     Jonas believes him kicks a horse\\
\footnotetext{%
  \url{http://www.machandel-verlag.de/readerview/der-katzenschatz.html}, 2021-01-29.
}
\glt `Jonas is utterly surprised.'
}
\ex[\#]{
\gll Jonas glaubt, dich  tritt ein Pferd.\\
     Jonas believes you kicks a horse\\
\glt `Jonas believes that a horse kicks you.'
}
\zl
\citet[\page 313]{RS2009a} argue that the idiomatic reading
is only available if the accusative pronoun is fronted and the embedded clause is V2\is{word order!V2}. The examples
in (\mex{1}) do not have the idiomatic reading:
\eal
\ex
\gll Ich glaube, dass mich ein Pferd tritt.\\
     I believe   that me   a horse   kicks\\
\glt `I believe that a horse kicks me.'
\ex
\gll Ich glaube, ein Pferd tritt mich.\\
     I believe   a horse   kicks me\\
\glt `I believe that a horse kicks me.'
\zl
They develop an analysis with a partly fixed configuration and some open slots, similar in spirit to
the TAG analysis in Figure~\ref{fig-take-into-account-TAG}. However, their restrictions on \emph{Pferd} clauses are too strict since there are
variants of the idiom that do not have the accusative pronoun in the \vf:
\ea
%Und von wegen, die dort lebende Menschen hätte keine Bedürfnisse wie Strassen, schulen,
%Krankenhäuser,
\gll ich glaub es tritt mich ein Pferd wenn ich einen derartigen Unsinn lese.\footnotemark\\
     I believe \expl{} kicks me a horse when I a such nonsense read\\
\footnotetext{%
  \url{http://www.welt.de/wirtschaft/article116297208/Die-verlogene-Kritik-an-den-Steuerparadiesen.html},
  commentary section, 2018-02-20.
}
\glt `I am utterly surprised when I read such nonsense.'
\z
So it might be the case that the organization of the embedded clause can be stated clause-internally,
and hence it is an open question whether there are idioms that make nonlocal Constructions necessary.


What is not an open empirical question, though, is whether humans store chunks with complex internal
structure or not. It is clear that we do, and much Construction Grammar literature emphasizes
this. Constructional HPSG can represent such chunks directly in the theory, but SBCG cannot, since
linguistic signs do not have daughters. So here, Constructional HPSG and TAG are the theories that
can represent complex chunks of linguistic material with its internal structure, while other
theories like GB, Minimalism, CG, LFG, SBCG, and DG cannot.

\subsubsection{Summary}

\largerpage
If all these points are taken together, it is clear that most variants of MGG are not Construction
Grammars. However, CxG had considerable influence on other frameworks so that there are
constructionist variants of LFG, HPSG, and TAG. HPSG in the version of \citet{Sag97a} (also called
Constructional HPSG) and the HPSG dialect Sign-Based Construction Grammar are Construction Grammars
that follow all the tenets mentioned above.

%\citet{Goldberg95a,Goldberg2006a,Michaelis2012a}
%% Historical aspects:
%% \begin{itemize}
%% \item non-locality \citep{FKoC88a}
%% \item type inheritance \citet{KF99a,Sag97a}
%% \end{itemize}

\subsection{Variants of Construction Grammar}

The previous section discussed the tenets of CxG and to what degree other frameworks adhere to them. This
section deals with frameworks that have Construction Grammar explicitly in their name or are usually
grouped among Construction Grammars:
\begin{sloppypar}
\begin{itemize}
\item Berkeley Construction Grammar\is{Construction Grammar (CxG)!Berkeley} \citep{Fillmore85a,Fillmore88a,KF99a,FriedHSK}
\item Cognitive Construction Grammar\is{Construction Grammar (CxG)!Cognitive} \citep{Lakoff87a-u,Goldberg95a,Goldberg2006a}
\item \isi{Cognitive Grammar} \citep{Langacker87a-u,Langacker2000a,Langacker2008a-u,Dabrowska2004a}
\item Radical Construction Grammar\is{Construction Grammar (CxG)!Radical} \citep{Croft2001a}
\item Embodied Construction Grammar\is{Construction Grammar (CxG)!Embodied} \citep{BC2005a}
\item Fluid Construction Grammar\is{Construction Grammar (CxG)!Fluid} \citep{SDB2006a-u,SteelsFluid-ed}
\item Sign-Based Construction Grammar\is{Construction Grammar (CxG)!Sign-Based} \citep{Sag2010b,Sag2012a}
\end{itemize}
\end{sloppypar}
\largerpage[2]%\enlargethispage{5pt}
\noindent
Berkeley Construction Grammar, Embodied Construction Grammar, Fluid Construction Grammar, and
Sign-Based Construction Grammar are the ones that are more formal. All of these variants use feature
value pairs and are constraint-based. They are sometimes also referred to as unification-based
approaches. Berkeley Construction Grammar\is{Construction Grammar (CxG)!Berkeley|(} never had a consistent formalization. The variant of
\isi{unification} assumed by \citet{KF99a} was formally inconsistent \citep[Section~2.4]{Mueller2006d} and the computation of \isi{construction-like
object}s (CLOs) suggested  by \citet{Kay2002a} did not work either \citep[Section~3]{Mueller2006d}. Berkeley Construction
Grammar was dropped by the authors, who joined forces with Ivan Sag and Laura Michaelis\aimention{Laura Michaelis} and eventually
came up with an HPSG variant named \sbcg\is{Construction Grammar (CxG)!Sign-Based|(} \citep{Sag2012a}. The differences between
Constructional HPSG \citep{Sag97a} and SBCG are to some extent cosmetic: semantic relations got the
suffix \type{-fr} for \emph{frame} (\type{like-rel} became \type{like-fr}), phrases were called constructions (\type{hd-subj-ph} became
\type{subj-head-cxt}), and lexical rules were called \emph{derivational constructions}.\footnote{
This renaming trick was so successful that it even confused some of the co-editors of the volume about
SBCG \citep{BS2012a-ed}. See for example \citew{Boas2014a} and the reply in \citew{MWArgStReply}.
}
While this renaming would not have changed anything in terms of expressiveness of theories, there
was another change that was not motivated by any of the tenets of Construction Grammar but rather by
the wish to get a more restrictive theory: \citet*{SWB2003a} and \citet{Sag2007a} changed the feature geometry of phrasal
signs in such a way that signs do not contain daughters. The information about mother-daughter relations
is contained in lexical rules and phrasal schemata (constructions) only. The phrasal schemata are
more like GPSG immediate dominance schemata (phrase structure rules without constraints on the order of the daughters) in licensing a mother node when certain daughters are present,
but without the daughters being represented as part of the AVM that stands for the mother node, as
was common in HPSG from 1985 till \citew*{SWB2003a}.\footnote{%
  The two approaches will be discussed in more detail in Section~\ref{sec-constructional-hpsg} and Section~\ref{cxg:sec-sbcg}.
}
This differs quite dramatically from what was done in Berkeley Construction
Grammar, since BCxG explicitly favored a non-local approach \parencites[\page 37]{Fillmore88a}[\page
  501]{FKoC88a}. Arguments were not canceled but
passed up to the mother node. Adjuncts were passed up as well, so that the complete internal
structure of an expression is available at the top-most node \citep[\page 9]{KF99a}. The advantage of BCxG and
Constructional HPSG \citep{Sag97a} is that complex expressions (\eg idioms and other more transparent expressions
with high frequency) can be stored as chunks containing the internal structure. This is not possible
with SBCG, since phrasal signs never contain internal structures. For a detailed discussion of \sbcg
see Section~\ref{cxg:sec-sbcg} and \citew[Section~10.6.2]{MuellerGT-Eng1}.\is{Construction Grammar (CxG)!Berkeley|)}\is{Construction Grammar (CxG)!Sign-Based|)}

Embodied\is{Construction Grammar (CxG)!Embodied} Construction Grammar \citep{BC2005a} uses typed feature descriptions for the description of
linguistic objects and allows for discontinuous constituents. As
argued by \citet[Section~10.6.3]{MuellerGT-Eng1}, it is a notational variant of Reape-style HPSG
(\citealt{Reape94a}; see also \crossrefchapteralt[Section~\ref{sec-domains}]{order} for discontinuous constituents in HPSG).

\largerpage[2]%\enlargethispage{5pt}
Fluid\is{Construction Grammar (CxG)!Fluid} Construction Grammar is also rather similar to HPSG. An important difference is that FCG attaches
weights to constraints, something that is usually not done in HPSG. But in principle, there is
nothing that forbids adding weights to HPSG as well, and in fact it has been done \citep{Brew95a,BC99a,MT2008a-u}, and it should be
done to a larger extent \citep{Miller2013a}. \Citet{vanTrijp2013a} tried to show that
Fluid Construction Grammar is fundamentally different from SBCG, but I think he failed in every
single respect. See \citew{MuellerFCG} for a detailed discussion, which cannot be repeated here for
space reasons.

In what follows I will compare Constructional HPSG (as assumed in this volume) with SBCG.

%% \section{HPSG as a Construction Grammar}
%% \label{sec-hpsg-as-cxg}

%% \begin{itemize}
%% \item form-meaning pairs
%% \item type hierarchies
%% \item surface oriented
%% \end{itemize}

%\subsection{Constructional HPSG and formal variants of Construction Grammar}


%\subsubsection{Berkeley Construction Grammar}

\subsubsection{Constructional HPSG}
\label{sec-constructional-hpsg}

As is discussed in other chapters in more detail
\parencites[Section~\ref{sec-essentials}]{chapters/formal-background}[Section~\ref{prop:sec-elements}]{chapters/properties},
HPSG uses feature value pairs to model linguistic objects. One important tool is structure sharing. For example, determiner, adjective, and noun agree
with respect to certain features in languages like \ili{German}. The identity of properties is modeled by
identity of feature values and this identity is established by identifying the values in descriptions. Now,
it is obvious that certain features are always shared simultaneously. In order to facilitate the statement
of respective constraints, feature value pairs are put into groups. This is why HPSG feature
descriptions are very complex. Information about syntax and semantics is represented under
\textsc{syntax-semantics} (\synsem), information about syntax under \textsc{category} (\cat), and
information that is projected along the head path of a projection is represented under \head. All
feature structures have to have a type. The type may be omitted in the description, but there has to
be one in the model. Types are organized in hierarchies. They are written in italics. (\mex{1})
shows an example lexical item for the word \emph{ate}:\footnote{
% JP: strange non-local should be mentioned
  The first `\ldots' stands for the feature \local, which is irrelevant in the present discussion. It plays
  a role in the treatment of nonlocal dependencies \crossrefchapterp{udc}.
}\footnote{
  To keep things simple, I omitted the feature \argst here. \argst stands for argument
  structure. The value of \argst is a list containing all arguments, that is, the elements of \spr
  and \comps are also contained in the \argst. Linking constraints are formulated with respect to
  the argument structure list. See \crossrefchaptert{arg-st} for a discussion of linking. The way
  arguments are linked to the valence features \spr and \comps is language- or language-class-specific.
  See Chapter~\ref{chap-arg-st} and also \crossrefchapterw[Section~\ref{sec-svo-sov}]{order}.
}
%\largerpage

\ea
\label{le-ate}
Lexical item for the word \emph{ate}:\\
\scalebox{.9}{%
\avm{
[\type*{word}
\punk{phonology}{<\type{ate}>} \\
syntax-semantics \ldots &	[\type*{local}
							category & [\type*{category}
										head & [\type*{verb}
												vform & fin] \\
										spr & <NP![\type{nom}]$_{\1}$!> \\
										comps & <NP![\type{acc}]$_{\2}$!> ]\\
							content & \ldots	[\type*{eat}
												actor & \1 \\
												undergoer & \2 ] ] ]
}}
\z

\largerpage[2]
\noindent
The information about part of speech and finiteness is bundled under \head. The selection of a
subject is represented under \spr (sometimes the feature \subj is used for subjects) and the
non-subject arguments are represented as part of a list under \comps. The semantic indices \ibox{1}
and \ibox{2} are linked to thematic roles in the semantic representation (for more on linking, see \crossrefchapteralt{arg-st}).

Dominance structures can also be represented with feature value pairs. While \citet{ps} and
\citet{ps2} had a \textsc{daughters} feature and then certain phrasal types constraining the
daughters within the \textsc{daughters} feature, \citet{Sag97a} represented the daughters and
constraints upon them at the top level of the sign.\footnote{ The top level is the outermost
  level. So in (\mex{0}), \textsc{phonology} and \textsc{syntax-semantics} are at the top level.}
This move made it possible to have subtypes of the type \type{phrase}, \eg
\type{filler-head-phrase}, \type{specifier-head-phrase}, and
\type{head-complement-phrase}. Generalizations over these types can now be captured within the type
hierarchy together with other types for linguistic objects like lexical items and lexical
rules (see Section~\ref{sec-inheritance}). (\mex{1}) shows an implicational constraint on the type \type{head-complement-phrase}:\footnote{ The schema
  in (\mex{1}) licenses flat structures. See \crossrefchapterw[\page \pageref{hcs-binary}]{order}
  for binary branching structures.  } \ea
\label{schema-bin-prel}
%~\\[-8pt]
Head-Complement Schema adapted from \citew[\page 479]{Sag97a}:
\emph{head-complement-phrase}\istype{head-complement-phrase} \impl\\
\scalebox{.9}{%
\avm{
[synsem|loc|cat|comps \eliste \\
head-dtr|synsem|loc|cat|comps < \1, \ldots, \tag{n} > & \\
non-head-dtrs < [ \synsem \1 ], \ldots, [ \synsem \tag{n} ] > ]
}}
\z
The constraint says that feature structures of type \type{head-complement-phrase} have to have a
\synsemv with an empty \compsl, a \textsc{head-dtr} feature, and a list-valued \textsc{non-head-dtrs} feature. The list has
to contain elements whose \synsem values are identical to respective elements of the \compsl
of the head daughter (\,\ibox{1}, \ldots, \ibox{n}\,).

\largerpage[2]
Dominance schemata (corresponding to grammar rules in phrase structure grammars) refer to such
phrasal types. (\mex{1}) shows how the lexical item in (\ref{le-ate}) can be used in a
head-complement configuration:

\eas
\label{ex-ate-a-pizza-sag97}%
Analysis of \emph{ate a pizza} in Constructional HPSG:\\
\scalebox{.9}{%
\avm{
[\type*{head-complement-phrase}
\phon < ate, a, pizza > \\
synsem|loc &	[cat &	[head & \1 \\
                            spr & \2 \\
                            comps & < >] \\
                 cont & \ldots ] \\ %\2\\
head-dtr & [\type*{word}
           \phon < ate > \\
           synsem|loc	[cat &	[head  & \1 [\type*{verb}
                                            vform & fin] \\
                                spr & \2 < NP![\type{nom}]! > \\
                                comps & < \3 NP![\type{acc}]! > ] \\
                         cont & \ldots
% 									\2 [\type*{essen}
%                                       agens & \1 \\
%                                       thema & \3 ] \\
                        ]
			] \\
non-head-dtrs & <[\phon < a, pizza > \\
                 synsem & \3 ]> ]
}}
\zs

\noindent
The description in the \compsl of the head is identified with the \synsemv of the non-head daughter
\iboxb{3}. The information about the missing specifier is represented at the mother node
\iboxb{2}. Head information is also shared between head daughter and mother node. The respective
structure sharings are enforced by principles: the Subcategorization Principle\is{principle!Subcategorization} or, in more recent
versions of HPSG, the Valence Principle\is{principle!Valence} makes sure that all valents of the head daughter that are not
realized in a certain configuration are still present at the mother node. The Head Feature Principle
ensures that the head information of a head daughter in headed structures is identical to the head
information on the mother node, that is, \head features are shared.

This is a very brief sketch of Constructional HPSG and is by no means intended to be a full-blown
introduction to HPSG, but it provides a description of properties that can be used to compare
Constructional HPSG to Sign-Based Construction Grammar in the next subsection.


\subsubsection{Sign-Based Construction Grammar}
\label{cxg:sec-sbcg}\label{sec-sbcg}

\is{Construction Grammar (CxG)!Sign-Based|(}
Having discussed some aspects of Constructional HPSG, I now turn to SBCG. SBCG is an HPSG variant,
so it shares most properties of HPSG but there are some interesting properties that are discussed in
this section. Locality constraints are discussed in the next subsection, and changes in feature geometry
in the subsections to follow. Subsection~\ref{sec-frame-semantics-SBCG} discusses Frame Semantics.

\subsubsubsection{Locality constraints}
\label{sec-locality}

As mentioned in Section~\ref{sec-non-locality}, SBCG assumes a strong version of locality: phrasal
signs do not have daughters. This is due to the fact that phrasal schemata (= phrasal constructions)
are defined as in (\mex{1}):
\eas
Head"=Complement Construction following \citet[481]{SWB2003a}:\\
\emph{head"=comp"=cx}\istype{head-comp-cx} \impl\\
\avm{
[\punk{mother|syn|val|comps}{< >} \\
head-dtr & \0	[\type*{word}
                syn|val|comps & \tag{A} ] \\
\punk{dtrs}{< \0 > \+ \tag{A} nelist} ]
}
\zs
Rather than specifying syntactic and semantic properties of the complete linguistic object at the
top level (as earlier versions of HPSG did), these properties are specified as properties under \textsc{mother}. Hence a construction
licenses a sign (a phrase or a complex word), but the sign does not include daughters. The daughters live at the level of the
construction only. While earlier versions of HPSG licensed signs directly, SBCG needs a statement
saying that all objects under \textsc{mother} are objects licensed by the grammar \citep*[\page
  478]{SWB2003a}:\footnote{%
A less formal version of this constraint is given as the Sign Principle\is{principle!Sign} by
\citet[\page 105]{Sag2012a}: ``Every sign must be listemically or constructionally licensed, where: a
sign is listemically licensed only if it satisfies some listeme, and a sign is constructionally
licensed if it is the mother of some well-formed construct.''
}
\ea
\label{meta-construction-statemnet}
$\Phi$ is a Well"=Formed Structure according to a grammar $G$ if and only if:
\begin{enumerate}
\item there is a construction $C$ in $G$, and
\item there is a feature structure $I$ that is an instantiation of $C$, such that
      $\Phi$ is the value of the \textsc{mother} feature of $I$.
\end{enumerate}
\z
The idea behind this change in feature geometry is that heads cannot select for daughters of their valents and hence
the formal setting is more restrictive and hence reducing computational complexity of the formalism (Ivan Sag,
p.c. 2011). However, this restriction can be circumvented by just structure sharing an element of
the daughters list with some value within \mother. The \textsc{xarg} feature making one argument
available at the top level of a projection \citep{BF99a} is such a feature. So, at the formal level, the \motherf alone does not
result in restrictions on complexity. One would have to forbid such structure sharings in addition,
but then one could keep \mother out of the business and state the restriction for earlier variants
of HPSG \citep[Section~10.6.2.1.3]{MuellerGT-Eng2}.

Note that analyses like the one of the \isi{Big Mess Construction} by \citet[\page
841]{VanEynde2018a-u}, also discussed in \crossrefchapterw[\page \pageref{bigmess2}]{np}, cannot be
directly transferred to SBCG since in the analysis of Van Eynde, this construction specifies the
phrasal type of its daughters, something that is excluded by design in SBCG: all \textsc{mother} values of
phrasal constructions are of type \type{phrase} and this type does not have any subtypes
\citep[\page 98]{Sag2012a}. Daughters in syntactic constructions are of type \type{word} or
\type{phrase}. So, it is impossible to require a daughter to be of type
\type{regular-nominal-phrase} as in the analysis of \citeauthor{VanEynde2018a-u}. In order to
capture the Big Mess Construction in SBCG, one would have to specify the properties of the daughters
with respect to their features rather than specifying the types of the daughters, that is, one has
to explicitly provide the features that are characteristic for feature structures of type
\type{regular-nominal-phrase} in Van Eynde's analysis rather than just naming the type. See
\citet{KS2012a-u} and \citet{KS2011a-u} for analyses of the Big Mess Construction in SBCG.

\subsubsubsection{\spr and \comps vs.\ \textsc{valence}}

\citet*{SWB2003a} differentiated between specifiers and complements, but this distinction was given up
in later work on SBCG. \citet{Sag2012a} has just one valence list that includes both subjects and
non-subjects. This is a return to the valence representations of \citet{ps}. An argument for this
was never given, despite arguments for a separation of valence information by \citet{Borsley87a}. With one single valence feature, a VP would be an unsaturated projection and generalizations
concerning phrases cannot be captured. For example, a generalization concerning extraposition (in
\ili{German}) is that maximal projections (that is projections with an empty \compsl) can be
extraposed \citep[Section~13.1.2]{Mueller99a}. It is impossible to state this generalization in SBCG
in a straightforward way \citep[Section~10.6.2.3]{MuellerGT-Eng2}.

\subsubsubsection{The Head Feature Principle}

There\isfeat{head|(}\is{principle!Head Feature|(} have been some other developments as well. \citet{Sag2012a} got rid of the Head Feature
Principle and stated identity of information explicitly within constructions. Structure sharing is
not stated with boxed numbers but with capital letters instead. An exclamation mark can be used to
specify information that is not shared \citep[\page 125]{Sag2012a}. While the use of letters instead
of numbers is just a presentational variant, the exclamation mark is a non-trivial
extension. (\mex{1}) provides an example: the constraints on the type \type{pred-hd"=comp"=cxt}:
\eas
Predicational Head"=Complement Construction following \citet[\page 152]{Sag2012a}:\\
\emph{pred-hd"=comp"=cxt}\istype{head-comp-cx} \impl\\
\avm{
[\punk{mother|syn X !}{[val & < Y >]} \\
head-dtr & \upshape Z: [\type*{word}
               			syn & \upshape X: [val & < Y > \+ \upshape L] ] \\
dtrs & \upshape < Z > \+ L:\type{nelist} ]
}
\zs
The X stands for all syntactic properties of the head daughter. These are identified with the
value of \textsc{syn} of the mother with the exception of the \textsc{val} value, which is specified
to be a list with the element Y. It is interesting to note that the !-notation is not without
problems: \citet[\page 145]{Sag2012a} states that the version of SBCG that he presents is ``purely
monotonic (non-default)'', but if the \textsc{syn} value of the mother is not identical due to overwriting of \textsc{val}, it is unclear how the type of \textsc{syn} can be constrained. ! can be
understood as explicitly sharing all features that are not mentioned after the !. Note, though, that
the type has to be shared as well. This is not trivial, since structure sharing cannot be applied
here, since structure sharing the type would also identify all features belonging to the respective
value. So one would need a relation that singles out a type of a structure and identifies this type
with the value of another structure. Note also that information from features behind the ! can make
the type of the complete structure more specific. Does this affect the shared structure (\eg
\textsc{head-dtr|syn} in (\mex{0}))? What if the
type of the complete structure is incompatible with the features in this structure? What seems to be a harmless notational device in fact involves some
non-trivial machinery in the background. Keeping the Head Feature Principle makes this additional machinery
unnecessary.
% There is only VAL in SBCG.
%On the other hand it has to be stated which valence information has to be passed up and
%which is not.
\isfeat{head|)}\is{principle!Head Feature|)}

\subsubsubsection{Feature geometry and the \textsc{form} feature}

The phrasal sign for \emph{ate a pizza} in Constructional HPSG was given in
(\ref{ex-ate-a-pizza-sag97}). (\mex{1}) is the Predicational Head Complement Construction with
daughters and mother filled in.
\ea
\label{feat-geom-sag2012}
\avm{
[\type*{pred-hd-comp-cxt}
mother &	[\type*{phrase}
			form & <\type{ate, a, pizza}> \\
			syn &	[cat & \1\\
					val & < NP![\type{nom}]! > ] \\
			sem & \ldots ] \\
head-dtr & \2	[\type*{word}
				form & <\type{ate}> \\
				syn &	[cat & \1	[\type*{verb}
									vf & fin] \\
						val & < NP![\type{nom}]!, \3 NP![\type{acc}]! > ]] \\
dtrs & < \2 , \3 > ]
}
\z
As was explained in the previous subsection, Constructional HPSG groups all selectable information under
\synsem and then differentiates into \cat and \cont. SBCG goes back to \citew{ps} and uses \textsc{syn} and
\textsc{sem}. The idea behind \synsem was to exclude the selection of phonological information and
daughters \citep[\page 23]{ps2}. Since daughters are outside of the definition of \type{synsem}, they cannot be accessed from
within valence lists. Now, SBCG pushes this idea one step further and also restricts the access
to daughters in phrasal schemata (constructions in SBCG terminology): since signs do not have
daughters, constructions may not refer to the daughters of their parts. But obviously signs need to
have a form part, since signs are per definition form-meaning pairs. It follows that the form part of
signs is selectable in SBCG. This will be discussed in more detail in the following
subsection. Subsection~\ref{sec-local-feature-sbcg} discusses the omission of the \localf.


\subsubsubsection{Selection of \phon and \textsc{form} values}

The feature geometry of Constructional HPSG has the \phonv outside of \synsem. Therefore verbs can
select for syntactic and semantic properties of their arguments but not for their phonology. For
example, they can require that an object has accusative case but not that it starts with a
vowel. SBCG allows for the selection of phonological information (the feature is called \form here)
and one example of such a selection is the indefinite article in \ili{English}, which has to be either \emph{a} or
\emph{an} depending on whether the noun or nominal projection it is combined with starts with a
vowel or not (Flickinger, Mail to the HPSG mailing list, 01.03.2016):
\eal
\ex an institute
\ex a  house
\zl
%\addlines
The distinction can be modeled by assuming a selection feature for determiners.\footnote{%
  In the 1994 version of HPSG there is mutual selection between the determiner and the noun. The noun selects
  the determiner via \spr and the determiner selects the noun via a feature called
  \textsc{specified}\isfeat{specified} \citep[\page 45--54]{ps2}.
%This feature is similar to the \modf, which was explained in Section~\ref{sec-adjuncts-hpsg}.
} An alternative would be, of course, to capture all phonological\is{phonology} phenomena by formulating constraints on phonology at the
phrasal level (see \citealp{BK94b}, \citealp{Hoehle99a-u}, and \citealp{Walther99a-u} for phonology in HPSG).

Note also that the treatment of raising in SBCG admits nonlocal selection of phonology
values, since the analysis of raising in SBCG assumes that the element on the valence list of the
embedded verb is identical to an element in the \argstl of the matrix verb \citep[\page
  159]{Sag2012a}. Hence, both verbs in (\mex{1}) can see the phonology of the subject:
\ea
Kim can eat apples.
\z
In principle, there could be languages in which the form of the downstairs verb depends on the
presence of an initial consonant in the phonology of the subject. \ili{English} allows for long chains of
raising verbs and one could imagine languages in which all the verbs on the way are sensitive to the
phonology of the subject. Such languages probably do not exist.

Now, is this a problem? Not really, but if one develops a general setup in a
way to exclude everything that is not attested in the languages of the world (as for instance the
selection of arguments of arguments of arguments), then it is a problem that heads can see the
phonology of elements that are far away.

%\addlines
There are two possible conclusions for practitioners of Sign-Based Construction Grammar: either the \motherf could be given up,
since one agrees that theories that do not make wrong predictions are sufficiently constrained and
one does not have to explicitly state what cannot occur in languages, or one would have to address
the problem with nonlocally selected phonology values and therefore assume a \synsem or \localf that
bundles information that is relevant in raising and does not include the
phonology.
%% \footnote{%
%%   If \synsem is reintroduced, the elements in the valence lists could be of type
%%   \type{synsem}. Information about phonology would not be part of the description of the selected elements. This would not solve the problem
%%   of partial verb phrase fronting though, since the \lexf is selected for (hence part of the
%%   information under \synsem) but not shared with the filler. One would need a \localf in addition to \synsem. See Section~\ref{sec-sbcg-local-feature}.
%% }
%This supports the arguments I made on \mother and \local in the previous subsections.
In the latter case, the feature geometry of SBCG would get more complicated. This additional
complication is further evidence against \mother, adding to the argument I made about \mother in Subsection~\ref{sec-locality}.

\subsubsubsection{The \local feature and information shared in nonlocal dependencies}
\label{sec-local-feature-sbcg}

Similarly, elements of the \argstl contain information about \form. In nonlocal dependencies, this
information is shared in the \textsc{gap} list (\slasch set or list in other versions of
HPSG) and is available all the way to the filler \citep[Section~10]{Sag2012a}. In other versions of HPSG, only \local information is shared and elements in valence lists
do not have a \phonf. If the sign that is contained in the \textsc{gap} list were identified with
the filler, the information about phonological properties of the filler would be available at the
extraction site and SBCG could be used to model languages in which the phonology of
a filler is relevant for a head from which it is extracted. So for instance, \emph{likes} could see the
phonology of \emph{bagels} in (\mex{1}):
\ea
Bagels, I think that Peter likes.
\z
It would be possible to state constraints saying that the filler has to contain a vowel or two
vowels or that it ends with a consonant. In addition, all elements on the extraction path
(\emph{that} and \emph{think}) can see the phonology of the filler as well. While there are
languages that mark the extraction path (\citealt[\page 4--5]{BMS2001a}; \crossrefchapteralt[\pageref{page-start-extraction-path-effects}--\pageref{page-end-extraction-path-effects}]{udc}), I doubt that there are languages that have phonological
effects over unbounded dependencies. This problem can be and has been solved by assuming that the
filler is not shared with the information in the \textsc{gap} list, but parts of the filler are
shared with parts in the \textsc{gap} list: \citet[\page 166]{Sag2012a} assumes that \textsc{syn},
\textsc{sem}, and \textsc{store} information is identified individually. Originally, the feature
geometry of HPSG was motivated by the wish to structure share information. Everything within \local
was shared between filler and extraction site. This kind of motivation is given up in SBCG.

Note, also, that not sharing the complete filler with the gap means that the \form
value of the element in the \argstl at the extraction site is not constrained. Without any
constraints, the theory would be compatible with infinitely many models, since the \form value could be
anything. For example, the \form value of an extracted adjective could be \phonliste{ Donald Duck }
or \phonliste{ Dunald Dock } or any arbitrary chaotic sequence of letters/""phonemes. To exclude this, one can stipulate the \form values of extracted
elements to be the empty list, but this leaves one with the unintuitive situation that the element
in \textsc{gap} has an empty \form list while the corresponding filler has a different, filled one.

See also \crossrefchapterw[Section~\ref{udc:sec-SBCG}]{udc} for a comparison of the treatment of unbounded dependencies in
Constructional HPSG and SBCG.

\subsubsubsection{Frame Semantics}
\label{sec-frame-semantics-SBCG}

Another difference between SBCG and other variants of HPSG is the use of Frame Semantics\is{Frame Semantics|(}
\citep{Fillmore82a-u,Fillmore85b-u}. The actual representations in SBCG are based on \mrs (Minimal
Recursion Semantics, \citealt{CFPS2005a}, see also \crossrefchapteralt{semantics}) and
the change seems rather cosmetic (relations have the suffix \type{-fr} for frame rather than
\type{-rel} for relation and the feature is called \textsc{frames}\isfeat{frames} rather than
\textsc{relations}\isfeat{rels}), but there is one crucial difference: the labels of semantic roles
are more specific than what is usually used in other variants of HPSG.\footnote{%
  \citet[\page 95]{ps} and \citet{ps2} use role labels like \textsc{kisser} and \textsc{kissee} that are predicate-specific.
  Generalizations over these feature names are impossible within the standard formal setting of
  HPSG \parencites[Section~8.5.3]{ps2}[\page 24, Fn.~1]{Mueller99a}[Section~4.2.1]{Davis2001a-u}.%
}
\citet[\page 89]{Sag2012a} provides the following representation for the meaning contribution of the
verb \emph{eat}:
%% JP: drop
%% \footnote{
%%   Note that the role label \textsc{ingestible} seems to be inappropriate for negated sentences like
%%   (i):
%%   \ea
%%   Peter did not eat the stone.
%%   \zlast
%% }
\ea
\avm{
[\type*{sem-obj}
\punk{index}{s} \\
frames & <[\type*{eating-fr}
	   label & l\\
	   sit   & s\\
	   ingestor & i\\
	   ingestible & j]> ]
}
\z
While some generalizations over verbs of a certain type can be captured with role labels like
\textsc{ingestor} and \textsc{ingestible}, this is limited to verbs of ingestion. More general role
labels like \textsc{agent} and \textsc{patient} (or \textsc{proto-agent} and \pagebreak\textsc{proto-patient},
\citealt{Dowty91a}, or \textsc{actor} and \textsc{undergoer} \citealt{VanValin99a-u}) allow for more generalizations of broader classes of verbs (see
\citealt{DK2000b-u}, \citealt[Section~4.2.1]{Davis2001a-u}, and \crossrefchapteralt{arg-st}).
\is{Frame Semantics|)}
\is{Construction Grammar (CxG)!Sign-Based|)}
%% \inlinetodostefan{
%% Rui: Head feature principle, LID from Berkeley CxG
%% }

%\subsubsection{Embodied Construction Grammar}

\subsubsection{Summary}

This section enumerated various flavors of Construction Grammars and briefly discussed the more
formal variants. It was noted that the formal underpinnings are rather similar in many cases. What
is different, though, is the kind of approach taken towards the representation of valence and argument
structure constructions. Constructional HPSG and SBCG differ from other Construction Grammars in
taking a strongly lexicalist stance (\citealt[Section~10.4.3]{SW2011a};
\crossrefchapteralt[Section~\ref{processing:sec-lexicalism}]{processing}): argument structure is
encoded lexically. A ditransitive verb is a ditransitive verb since it selects for three NP
arguments. This selection is encoded in valence features of lexical items. It is not assumed that
phrasal configurations can license additional arguments as it is in basically all other variants of
Construction Grammar. The next section discusses phrasal CxG approaches in more
detail. Section~\ref{sec-phrasal-patterns} then discusses patterns that should be analyzed phrasally
and which are problematic for entirely head-driven (or rather functor-driven) theories like
Categorial\indexcg Grammar, Dependency\indexdg Grammar, and \isi{Minimalism}.



\section{Valence vs.\ phrasal patterns}
\label{sec-valence}\label{sec-valence-vs-phrasal-patterns}\label{cxg:sec-valence-vs-phrasal-patterns}

\enlargethispage{3pt}
Much\is{acquisition|(} work in Construction Grammar starts from the observation that
children acquire patterns and, in later acquisition stages, abstract from these patterns to schemata
containing open slots to be filled by variable material, for example subjects and objects
\citep{Tomasello2003a}. The conclusion that is drawn from this is that language should be
described with reference to phrasal patterns. Most Construction Grammar variants assume a phrasal
approach to argument structure constructions \citep{Goldberg95a,Goldberg2006a,GJ2004a}, with
Constructional HPSG \citep{Sag97a}, \citegen{Boas2003a} work, and SBCG \citep*{SBK2012a,Sag2012a} being the
three exceptions. So, for examples like the resultative construction in (\mex{1}),
\citet[Chapter~8]{Goldberg95a} assumes that there is a phrasal construction [Subj [V Obj
Obl]]\footnote{
Goldberg does not state the resultative construction, but the Caused-Motion Construction, which is
syntactically parallel to the Resultative Construction, is specified this way on p.\,152. She
describes the syntax of resultative constructions on p.\,192.
} into which
material is inserted and which contributes the resultative semantics as a whole.

\ea
She fished the pond empty.
\z
HPSG follows the lexical approach and assumes that \stem{fish} is inserted into a lexical
construction (lexical rule), which licenses the combination with other parts of the resultative
construction \citep[Section~5.2]{Mueller2002b}.

I argued in several publications that the language acquisition facts can be explained in lexical
models as well \parencites[Section~6.3]{MuellerPersian}[Section~9]{MWArgSt}. While a pattern-based approach claims that (\mex{1}) is
analyzed by inserting \emph{Kim}, \emph{loves}, and \emph{Sandy} into a phrasal schema stating that
NP[nom] verb NP[acc] or subject verb object are possible sequences in \ili{English}, a lexical approach
would state that there is a verb \emph{loves} selecting for an NP[nom] and an NP[acc] (or for a
subject and an object).
\ea
Kim loves Sandy.
\z
Since objects follow the verb in \ili{English} (modulo extraction) and subjects precede the verb, the same
sequence is licensed in the lexical approach. The lexical approach does not have any problems
accounting for patterns in which the sequence of subject, verb, and object is discontinuous. For
example, an adverb may intervene between subject and verb:
\ea
Kim really loves Sandy.
\z
In a lexical approach it is assumed that verb and object may form a unit (a VP). The adverb attaches
to this VP and the resulting VP is combined with the subject. The phrasal approach has to assume
either that adverbs are part of phrasal schemata licensing cases like (\mex{0}) (see
\citealt[Section~6.3.2]{Uszkoreit87a} for such a proposal in a \gpsg approach to \ili{German}) or that the phrasal
construction may license discontinuous\is{constituent!discontinuous} patterns. \citet[\page 170]{BC2005a} follow the latter approach and
assume that subject and verb may be discontinuous but verb and object(s) have to be adjacent. While
this accounts for adverbs like the one in (\mex{0}), it does not solve the general problem, since
there are other examples showing that verb and object(s) may appear discontinuously as well:
\ea
Mary tossed me a juice and Peter a water.
\z
Even though \emph{tossed} and \emph{Peter a water} are discontinuous in (\mex{0}), they are an
instance of the ditransitive construction. The conclusion is that what has to be acquired is not a phrasal pattern but rather the fact that
there are dependencies\is{dependency} between certain elements in phrases (see also
\citealt{Behrens2009a} for a similar view from a language acquisition perspective). I return to ditransitive
constructions in Section~\ref{sec-coercion}.%
\is{acquisition|)}

\begin{sloppypar}
I discussed several phrasal approaches to \isi{argument structure} and showed where they fail
\citep{Mueller2006d,Mueller2006c,Mueller2007d,MuellerPersian,MWArgSt,MWArgStReply,MuellerLFGphrasal}. Of course,
the discussion cannot be reproduced here, but I want to repeat four points showing that lexical valence
representation is necessary and that effects that are the highlight of phrasal approaches can be
achieved in lexical proposals as well. The first two are problems that were around in GPSG times and basically
were solved by abandoning the framework and adopting a new framework which was a fusion of GPSG and
Categorial Grammar: HPSG.\footnote{
  For further criticism of GPSG see \citew{Jacobson87b}. A detailed discussion of reasons for
  abandoning GPSG can be found in \citew[Section~5.5]{MuellerGT-Eng1}.
}
\end{sloppypar}

\subsection{Derivational morphology and valence}

The first argument \citep[Section~5.5.1]{MuellerGT-Eng1} is that certain patterns in derivational\is{morphology!derivational|(} morphology refer to valence. For
example, the \bard productively applies to transitive verbs only, that is, to verbs that govern an accusative.
\eal
\ex[]{
\gll unterstützbar\\
     supportable\\
}
\ex[*]{
\gll helfbar\\
     helpable\\
}
\ex[*]{
\gll schlafbar\\
     sleepable\\
}
\zl
\begin{sloppypar}
\noindent
Note that the \bard is like the passive in that it suppresses the subject and promotes the accusative object: the
accusative object is the element adjectives derived with the \bard predicate over. There is no argument
realized with the adjective \emph{unterstützbaren} `supportable' attaching to \emph{Arbeitsprozessen} `work.""processes' in \emph{unterstützbaren
  Arbeitsprozessen}.\footnote{
Adjectives realize their arguments preverbally in \ili{German}:
\ea
\gll der [seiner Frau treue] Mann\\
     the \spacebr{}his wife faithful man\\
\glt `the man who is faithful to his wife'
\z
\emph{unterstützbaren} `supportable' does not take an argument; it is a complete adjectival projection like \emph{seiner Frau treue}.
}  Hence one could not claim that the stem enters a phrasal construction with arguments and
\suffix{bar} attaches to this phrase. It follows that information about valence has to be present in
stems.
\end{sloppypar}

Note also that the resultative construction interacts with the \bard. (\mex{1}) shows an example of the resultative
construction in \ili{German} in which the accusative object is introduced by the construction: it is the subject of
\emph{leer} `empty' but not a semantic argument of the verb \emph{fischt} `fishes'.
\ea
\gll Sie fischt den Teich leer.\\
     she fishes the pond empty\\
\z
So even though the accusative object is not a semantic argument of the verb, the \bard is possible
and an adjective like \emph{leerfischbar} `empty.fishable' meaning `can be fished empty' can be derived. This is explained by lexical analyses of
the \bard and the resultative construction, since if one assumes that there is a lexical item for
the verb \stem{fisch} selecting an accusative object and a result predicate, then this item may function as
the input for the \bard. See Section~\ref{sec-cxg-morphology} for further discussion of \bard and
 \citew{Verspoor97a}, \citew{Wechsler97a}, \citew{WN2001a}, and \citew[Chapter~5]{Mueller2002b} for
 lexical analyses of the resultative construction in the framework of HPSG.%
\is{morphology!derivational|)} 

\subsection{Partial verb phrase fronting}
\label{cxg:sec-partial-verb-phrase-fronting}

The second argument concerns partial verb phrase fronting\is{partial verb phrase fronting|(} \citep[Section~5.5.2]{MuellerGT-Eng1}.
(\mex{1}) gives some examples: in (\mex{1}a) the bare verb is fronted and its arguments are realized
to the right of the finite verb in the so-called middle field, in (\mex{1}b) one of the objects is fronted together with the verb, and in
(\mex{1}c) both objects are fronted with the verb.
\eal
\ex
\gll Erzählen wird er seiner Tochter ein Märchen können.\\
     tell will he his daughter a fairy.tale can\\
\ex
\gll Ein Märchen erzählen wird er seiner Tochter können.\\
     a fairy.tale tell will he his daughter can\\
\ex
\gll Seiner Tochter ein Märchen erzählen wird er können.\\
     his daughter a fairy.tale tell will he can\\
\glt `He will be able to tell his daughter a fairy tale.'
\zl
The problem with sentences such as those in (\mex{0}) is that the valence requirements of the verb
\emph{erzählen} `to tell' are realized in various positions in the sentence. For fronted
constituents, one requires a rule which allows a ditransitive to be realized without its arguments
or with one or two objects. This basically destroys the idea of a fixed phrasal configuration for
the ditransitive construction and points again in the direction of dependencies\is{dependency}.

Furthermore, it has to be ensured that the arguments that are
missing in the prefield are realized in the remainder of the clause. It is not legitimate to omit
obligatory arguments or realize arguments with other properties like a different case, as the
examples in (\mex{1}) show:
\eal
\ex[]{
\gll Verschlungen hat er        ihn        nicht.\\
     devoured     has he.\NOM{} him.\ACC{} not\\
\glt `He did not devour it/him.'
}
\ex[*]{
\gll Verschlungen hat er        nicht.\\
     devoured     has he.\NOM{} not\\
}
\ex[*]{
\gll Verschlungen hat er        ihm        nicht.\\
     devoured     has he.\NOM{} him.\DAT{} not\\
}
\zl
The obvious generalization is that the fronted and unfronted arguments must add up to the total
set of arguments selected by the verb. This is scarcely possible with the rule-based
representation of valence in GPSG \citep{Nerbonne86a,Johnson86a}. In theories such as Categorial
Grammar\indexcg, it is possible to formulate elegant analyses of (\mex{0})
\citep{Geach70a}. \citet{Nerbonne86a} and \citet{Johnson86a} both suggest analyses for sentences
such as (\mex{0}) in the framework of GPSG which
ultimately amount to changing the representation of valence information in the direction of
Categorial Grammar.
With a switch to CG-like valence representations in HPSG, the phenomenon of partial verb phrase
fronting found elegant solutions
\parencites[Section~4]{HoehleSpuren}{Mueller96a}{Meurers99a}.\is{partial verb phrase fronting|)}
%\todostefan{reference to UDC chapter?}
% no, it is not in there.


\subsection{Coercion}
\label{sec-coercion}

An important observation in constructionist work is that, in certain cases, verbs can be used in
constructions that differ from the constructions they are normally used in. For example, verbs that are usually
used with one or two arguments may be used in the ditransitive construction:
\eal
\ex She smiled.
\ex She smiled herself an upgrade.\footnote{
Douglas Adams. 1979. \emph{The Hitchhiker’s Guide to the Galaxy}, Harmony Books. Quoted from
\citew[\page 220]{Goldberg2003b}.
}
\ex He baked a cake.
\ex He baked her a cake.
\zl
The usual explanation for sentences like (\mex{0}b) and (\mex{0}d) is that there is a phrasal
pattern with three arguments into which intransitive and strictly transitive verbs may enter. It is
assumed that the phrasal patterns are associated with a certain meaning
\citep{Goldberg96a,GJ2004a}. For example, the benefactive meaning of (\mex{0}d) is contributed by
the phrasal pattern (\citealt[Section~6]{Goldberg96a}; \citealt*[\page 81]{AGT2014a}).

The insight that a verb is used in the ditransitive pattern and thereby contributes a certain
meaning is of course also captured in lexical approaches. \citet[Section~5]{BC99a} suggested a lexical
rule-based analysis mapping a transitive version of verbs like \emph{bake} onto a ditransitive one
and adding the benefactive semantics. This is parallel to the phrasal approach in that it says:
three-place \emph{bake} behaves like other three-place verbs (\eg \emph{give}) in taking three
arguments and by doing so, it comes with a certain meaning (see \citealt{MuellerLFGphrasal} for a
lexical rule-based analysis of the benefactive constructions that works for both \ili{English} and \ili{German},
despite the surface differences of the respective languages). The lexical rule is a form-meaning pair
and hence a construction.
%\newcounter{croftyears} is defined in localcommands since we need this for includeonly to work.
\setcounter{croftyears}{\year-2003}
As Croft put it \thecroftyears{} years ago: lexical rule vs.\ phrasal schema is a false
dichotomy \citep{Croft2003a}. But see \citew{MuellerLFGphrasal,Mueller2006d,MuellerUnifying} and \citew{MWArgSt} for
differences between the approaches.

\citet{BC99a} paired their lexical rules with probabilities to be able to explain differences in
productivity. This corresponds to the association strength that \citet[\page 141]{vanTrijp2011a}
used in Fluid Construction Grammar to relate lexical items to phrasal constructions of various kinds.

\subsection{Non-predictability of valence}

\largerpage
The last subsection discussed phrasal models of coercion that assume that verbs can be inserted
into constructions that are compatible with the semantic contribution of the verb. \citet[Section~7.4]{MWArgSt}
pointed out that this is not sufficiently constrained. Müller \& Wechsler discussed the examples in
(\mex{1}), among others:

\eal\label{depends-on-ex}
\ex John depends on Mary.  (\emph{counts, relies,} etc.)
\ex John trusts (*on) Mary.
\zl
While \emph{depends} can be combined with a \emph{on}-PP, this is impossible for \emph{trusts}. Also
the form of the preposition of prepositional objects is not always predictable from semantic
properties of the verb. So there has to be a way to state that certain verbs go together with
certain kinds of arguments and others do not. A lexical specification of valence information is the
most direct way to do this. Phrasal approaches sometimes assume other means to establish connections
between lexical items and phrasal constructions. For instance, \citet[\page 50]{Goldberg95a} assumes
that verbs are ``conventionally associated with constructions''. The more technical work in Fluid
CxG\is{Construction Grammar (CxG)!Fluid} assumes that every lexical item is connected to various phrasal constructions via coapplication
links \citep[\page 141]{vanTrijp2011a}. This is very similar to Lexicalized Tree Adjoining
Grammar\is{Tree Adjoining Grammar (TAG)}
(LTAG; \citealt*{SAJ88a-u}), where a rich syntactic structure is associated to a lexical anchor. So,
phrasal approaches that link syntactic structure to lexical items are actually lexical
approaches as well. As in \gpsg, they include means to ensure that lexical items enter into correct
constructions. In GPSG, this was taken care of by a number. I already discussed the GPSG shortcomings
in previous subsections.

Concluding this section, it can be said that there has to be a connection between lexical items and
their arguments and that a lexical representation of argument structure is the best way to establish
such a relation.

%\section{HPSG as a Construction Grammar}
%\label{sec-hpsg-as-cxg}

\section{Construction Morphology}
\label{sec-cxg-morphology}\label{cxg:sec-cxg-morphology}

\largerpage\enlargethispage{5pt}
The\is{morphology!derivational|(} first publications in \isi{Construction Morphology} were the
master's thesis of \citet{Riehemann93a}, which later appeared as \citet{Riehemann98a}, and Koenig's
1994 WCCFL paper and thesis
\citep{KJ95a-u,Koenig94a-u,Koenig99a}. Riehemann called her framework \emph{Type-Based Derivational
  Morphology}, since it was written before influential work like \citew{Goldberg95a}
appeared and before the term \emph{Construction Morphology} \citep{Booij2005a} was used. Riehemann did a careful corpus study on adjective
derivations with the suffix \suffix{bar} `-able'. She noticed that there is a productive pattern
that can be analyzed by a lexical rule relating a verbal stem to the adjective suffixed with
\suffix{bar}.\footnote{
  She did not call her rule a lexical rule, but the difference between her template and the
  formalization of lexical rules by \citet[\page 26]{Mueller2002b} is the naming of the feature \textsc{morph-b}
  vs.\ \textsc{lex-dtr}. \citet[Section~8.2.3]{CB92a}, \citet[Section~2]{BC99a}, and \citet[\page 176]{Meurers2001a} use a representation with
  \textsc{in} and \textsc{out} features that actually corresponds to the
  \textsc{mother}/\textsc{dtrs} format of SBCG. See Section~\ref{sec-locality}.
} The productive pattern applies to verbs governing an accusative as in (\mex{1}a) but
is incompatible with verbs taking a dative as in (\mex{1}b):
\eal
\ex[]{
\gll unterstützbar\\
     supportable\\
}
\ex[*]{
\gll helfbar\\
     helpable\\
}
\ex[*]{
\gll schlafbar\\
     sleepable\\
}
\zl
Intransitive verbs are also excluded, as (\mex{0}c) shows. Riehemann suggests a schema like the one in
(\mex{1}):

\ea
\label{ex-schema-bar}
%\begin{tabular}[t]{@{}l@{}}
Schema for productive adjective derivations with the suffix \suffix{bar} in \ili{German} adapted from
\citet[\page 68]{Riehemann98a}:\\*
%\end{tabular}
% This is the original:
% \avm{
% [\type*{reg-bar-adj}
% phonology & \1 $+$ bar\\
% morph-b &	<[\type*{trans-verb}
% 			phon \1\\
% 			synsem|loc &	[cat|val|comps & < NP![\type{acc}]!:\2 > \+ \3 \\
% 							cont|nuc \4	[act & \\
% 										und & \2] ] ]>\\
% synsem|local	[category &	[head & adj\\
% 							valence &	[subj & < NP:\2 >\\
% 										comps & \3 ] ] \\
% 				content|nucleus	[reln & $\Diamond$\\
% 								arg1 & \2\\
% 								arg2 & \4 ] ] ]
% }
\avm{
	[\type*{reg-bar-adj}
	\punk{phon}{\1 \+ \phonliste{ bar }} \\
	\punk{morph-b}{<[\type*{trans-verb}
			  phon \; \1 \\
			  synsem|loc  [cat|comps & < NP[\type{acc}]\ind{2} > \+ \3 \\
			               cont|nuc  & \4 [act & index \\
			                               und & \2] ] ] >} \\
	synsem|loc &	[cat	[head  & adj \\
				 subj  & < NP\ind{2} > \\
				 comps & \3 ] \\
			 cont|nuc  [reln    & $\diamond$ \\
				    und     & \2 \\
				    soa-arg & \4 ] ] ]
}
\z
\largerpage
\textsc{morph-b} is a list that contains a description of a transitive verb (something that governs
an accusative object which is linked to the undergoer role \iboxb{2} and has an actor).\footnote{
  Note that the specification of the type \type{trans-verb} in the list under \textsc{morph-b} is
  redundant, since it is stated that there has to be an accusative object and that there is an actor
  and an undergoer in the semantics. Depending on further properties of the grammar, the
  specification of the type is actually wrong: productively derived particle verbs may be input to
  the \bard, and these are not a subtype of \type{trans-verb}, since the respective particle verb rule
  derives both transitive (\emph{anlachen} `laugh at somebody')  and intransitive verbs
  (\emph{loslachen} `start to laugh') \citep[\page 296]{Mueller2003a}. \emph{Anlachen} does not have an
  undergoer in the semantic representation suggested by \citet{Stiebels96a}. See
  \citet[\page 308]{Mueller2003a} for a version of the \bard schema that is compatible with particle verb
  formations as input.

  The original formulation of Riehemann shares the \contv of the semantics of the accusative NP
  with the subject of the adjective and the value of the \textsc{undergoer} feature. I adapted the
  rule here to just share the index, since values of \textsc{actor} and \textsc{undergoer} features
  are of type \type{index}. Jean-Pierre Koenig pointed out to me that sharing of the whole content
  of the accusative object and the subject of the adjective is necessary, since otherwise the \contv
  of the accusative object would be unrestricted and -- according to the formal basics of HPSG --
  could vary in infinitely many ways. Such an explicit sharing of semantics is not necessary in
  \citeauthor{Mueller2003a}'s approach, since he distinguishes between structural and lexical case
  \crossrefchapterp[Section~\ref{sec:case:str}]{case} and this makes it possible to structure share the complete description of
  the accusative object with the subject of the adjective.
}
The phonology of this element \iboxb{1} is combined with the suffix \suffix{bar} and forms the
phonology of the complete lexical item. The resulting object is of category \type{adj} and the
index of the accusative object of the input verb \iboxb{2} is identified with the one of the subject
of the resulting adjective and with the value of the \textsc{undergoer} feature in the semantic
representations of the adjective. The semantics of the
input verb \iboxb{4} is embedded under a modal operator in the semantics of the adjective.

While the description of the \bard given so far captures
the situation quite well, there are niches and isolated items that are exceptions. According to
\citet[\page 5]{Riehemann98a}, this was the case for 7\% of the adjectives she looked at in her
corpus study. Examples are verbs ending in \suffix{ig} like \emph{entschuldigen} `to excuse'. The
\suffix{ig} is dropped in the derivation:
\ea
\gll entschuldbar\\
     excuseable\\
\z
Other cases are lexicalized forms like \emph{essbar} `safely edible', which have a special
lexicalized meaning. Exceptions of the accusative requirement are verbs selecting a dative
(\mex{1}a), a prepositional object (\mex{1}b), reflexive verbs (\mex{1}c), and even intransitive, mono-valent verbs (\mex{1}d):
\eal
\ex
\gll unentrinnbar\\
     inescapable\\
\ex
\gll verfügbar\\
     available\\
\ex
\gll regenerierbar\\
     regenerable\\
\ex
\gll brennbar\\
     inflammable\\
\zl

\largerpage
To capture generalizations about productive, semi-productive and fixed patterns/items, Riehemann
suggests a type hierarchy, parts of which are provided in
Figure~\ref{fig-bar-Riehemann}.
%\todostefan{why doesn't the figure start at the left?}
\begin{figure}
\centerfit{%
%\scalebox{0.95}{%
\begin{forest}
type hierarchy
[bar-adj
  [trans-bar-adj
    [reg-bar-adj]
    [essbar]
    [\ldots]]
  [dative-bar-adj
    [unentrinnbar]
    [\ldots]]
  [prep-bar-adj
    [verfügbar]]
  [intr-bar-adj
    [brennbar]
    [\ldots]]]
\end{forest}}
\caption{\label{fig-bar-Riehemann}Parts of the type hierarchy for \bard adapted from \citew[\page 15]{Riehemann98a}}
\end{figure}
The type \type{bar-adj} stands for all \suffix{bar} adjectives and comes with the constraints that
apply to all of them. One subtype of this general type is \type{trans-bar-adj}, which subsumes all
adjectives that are derived from transitive verbs. This includes all regularly derived \baradjs,
which are of the type \type{reg-bar-adj} but also \emph{essbar} `edible' and \emph{sichtbar} `visible'.

As this recapitulation of Riehemann's proposal shows, the analysis is a typical CxG analysis:
V-\emph{bar} is a partially filled word (see Goldberg's examples in
Table~\ref{tab-constructions}). The schema in (\ref{ex-schema-bar}) is a form-meaning
pair. Exceptions and subregularities are represented in an inheritance network.
\is{morphology!derivational|(}

\section{Phrasal patterns}
\label{sec-phrasal-patterns}
\label{sec-phrasal}

Section~\ref{sec-valence} discussed the claim that constructions in the sense of CxG have to be phrasal. I
showed that this is not true and that in fact lexical approaches to valence have to be preferred
under the assumptions usually made in non-transformational theories. However, there are other areas
of grammar that give exclusively head-driven approaches like \cg, \minimalism, and \dg a hard
time. In what follows I discuss the NPN construction and various forms of filler gap constructions.

\subsection{The NPN Construction}
\label{sec-npn}

\citet{Matsuyama2004a} and \citet{Jackendoff2008a} discuss the \isi{NPN Construction}, examples
of
%\todostefan{Anne: Compare analyses more precisely: inheritance?}
which are provided in (\mex{1}):
\eal
\ex Student after student left the room.
\ex
\label{cxg:ex-npn-iteration}
Day after day after day went by, but I never found the courage to talk to
her. \citep{Bargmann2015a}
\zl
The properties of the NPN construction (with \emph{after})  are summarized by Barg\-mann \citeyearpar{Bargmann2015a}
in a concise way and I will repeat his examples and summarization below to motivate his analysis in (\ref{ex-npn-bragmann}).

The examples in (\mex{0}) show that the N-after-N Construction has the \emph{distribution of NPs}.
As (\mex{1}) shows, the construction is \emph{partially lexically fixed}: \emph{after} cannot be
replaced by any other word \citep[\page 73]{Matsuyama2004a}.
\ea
Alex asked me question \{ after / * following / * succeeding \} question.
\z

\largerpage
\noindent
The construction is \emph{partially lexically flexible}: the choice of Ns is free, except for the fact that
the Ns must be identical (\mex{1}a), the Ns must be count nouns (\mex{1}b), Ns must be in the
singular (\mex{1}c), and the Ns must be bare (\mex{1}d).

\eal
\settowidth\jamwidth{(Ns have determiners)}
\ex[*]{
bus after car \jambox{(N1 $\neq$ N2)}
}
\ex[*]{
water after water \jambox{(Ns = mass nouns)}
}
\ex[*]{
books after books \jambox{(Ns = plurals)}
}
\ex[*]{
a day after a day \jambox{(Ns have determiners)}
}
\zl

\noindent
The construction is \emph{syntactically fixed}: N-after-N cannot be split by syntactic operations as the
contrast in (\mex{1}) shows \citep{Matsuyama2004a}:
\eal
\ex[]{
Man after man passed by.
}
\ex[*]{
Man passed by after man.
}
\zl
If extraposition of the \emph{after}-N constituent were possible, (\mex{0}b) with an extraposed\is{extraposition}
\emph{after man} should be fine but it is not, so NPN seems to be a fixed configuration.

There is a syntax-semantics mismatch:
while N-after-N is syntactically singular, as (\mex{1}) shows, it is plural semantically, as
(\mex{2}) shows:
\ea
Study after study \{ reveals / *reveal \} the dangers of lightly trafficked streets.
\z\vspace{-\baselineskip}
\eal
\ex John ate \{ apple after apple / apples / *an apple \} for an hour.
\ex John ate \{ *apple after apple / *apples / an apple \} in an hour.
\zl
Furthermore there is an aspect of semantic sequentiality: N-after-N conveys a temporal or spatial
sequence: as \citet{Bargmann2015a} states, the meaning of (\mex{1}a) is something like (\mex{1}b).
\eal
\ex Man after man passed by.
\ex First one man passed by, then another(, then another(, then another(, then  \ldots{} ))).
\zl
The Ns in the construction do not refer to one individual each; rather, they contribute to a holistic meaning.

\largerpage[2]
The NPN construction allows adjectives to be combined with the nouns, but this is restricted.
N1 can only be preceded by an adjective if N2 is preceded by the same adjective:
\eal
\ex[]{
bad day after bad day (N1 and N2 are preceded by the same adjective.)
}
\ex[*]{
bad day after awful day (N1 and N2 are preceded by different adjectives.)
}
\ex[*]{
bad day after day (Only N1 is preceded by an adjective.)
}
\ex[]{
day after bad day (Only N2 is preceded by an adjective.)
}
\zl

\noindent
Finally, \emph{after} N may be iterated to emphasize the fact that there are several referents of N, as the example in (\ref{cxg:ex-npn-iteration}) shows.


This empirical description is covered by the following phrasal construction, which is adapted from
\citew{Bargmann2015a}:\footnote{%
Jackendoff and Bargmann assume that the result of combining N, P, and N is an NP. However this is
potentially problematic, as Matsuyama's example in (\ref{ex-hearty-chear-after-chear}) shows \citep[\page 71]{Matsuyama2004a}:
\ea
\label{ex-hearty-chear-after-chear}
All ranks joined in hearty cheer after cheer for every member of the royal family \ldots
\z
As Matsuyama points out, the reading of such examples is like the reading of \emph{old men and women}
in which \emph{old} scopes over both \emph{men} and \emph{women}. This is accounted for in
structures like the one indicated in (\ref{ex-hearty-cheer-after-cheer}):
\ea
\label{ex-hearty-cheer-after-cheer}
hearty [cheer after cheer]
\z
Since adjectives attach to \nbars and not to NPs, this means that NPN constructions should be
\nbars. Of course (\ref{ex-hearty-cheer-after-cheer}) cannot be combined with a determiner, so one would have to assume that
NPN constructions select for a determiner that has to be dropped obligatorily. Determiners are also
dropped in noun phrases with mass nouns with a certain reading.
}
\ea
\label{ex-npn-bragmann}%
NPN Construction\is{schema!NPN} as formalized by \citew{Bargmann2015a}:\\
%\begin{sideways}
	\oneline{%
\avm{
[\phon < \ldots\ N \ldots, after, \ldots\ N \ldots > \\
\punk{ss|loc|cat}{	[head &	[\type*{noun}
					count & $-$ \\
                    agr & 3rdsing ] \\
             val  &	[spr & < > \\
                    comps & < > ] ]	} \\
sr & !$\lambda P.\exists X.|X| >1~\&~\forall x \in X{:\,}N'(x)~\&~\exists R^{order} \subseteq X^{2}~\&~P(x)$! \\
dtrs &	<	[\phon < \ldots\ N \ldots > \\
       		 ss|l|c &	[head &	[\type*{noun}
								count & $+$ \\
								agr & 3rdsing ] \\
						val  &	[spr & < Det > \\
								comps & < > ] ] \\
        	sr & !$\ldots~\lambda x.N'(x)~\ldots$! ]
        ,
		(	[\phon <after> \\
          	\ldots\ head & prep \\
            sr & $\exists R^{order} \subseteq X^{2}$ ]
            ,
			[\phon < \ldots\ N \ldots > \\
        	ss|l|c &	[head &	[\type*{noun}
								count & $+$ \\
                            	agr & 3rdsing ] \\
                		val &	[spr & < Det > \\
                            	comps & < > ] ] \\
        	sr & !$\ldots~\lambda x.N'(x)~\ldots$! ]
        )\textsuperscript{+}%todo this sould be placed a lot higher!!
        >
]
}
	}
%\end{sideways}
\z
There is a list of daughters consisting of a first daughter and an arbitrarily long list of
\emph{after} N pairs. The `+'\is{+} means that there has to be at least one \emph{after} N pair. The
nominal daughters select for a determiner via \spr, so they can be either bare nouns or nouns
modified by adjectives. The semantic representation, non-standardly represented as the value of
\textsc{sr}, says that there have to be several objects in a set X ($\exists X.|X| >1$) and for all of them, the meaning
of the \nbar has to hold ($\forall x \in X{:\,}N'(x)$). Furthermore there is an order between the elements of X as stated by $\exists R^{order} \subseteq X^{2}$.


%% A further piece of data that is interesting was also provided by Matsuyama: The nouns of the NPN
%% construction may take complements. (\mex{1}) is one of his examples \citep[\page 59]{Matsuyama2004a}:
%% \ea
%% \ldots{} and he kept asking question after question about the world that lay away down the river,
%% with all its perils and marvels
%% \z
%% silly question after silly question about the ...
%% This can be fixed easily in Bargmann's NPN construction by just assuming that

From looking at this construction, it is clear that it cannot be accounted for by standard \xbar
rules. Even without requiring \xbar syntactic rules, there seems to be no way to capture these
constructions in head-based approaches like \minimalism, \cg, or \dg.\footnote{%
  \crossrefchaptert[\page \pageref{dg:page-npn-construction}]{dg} provides an analysis of the NPN Construction in the framework of \isi{Word
    Grammar}. Since Word Grammar is a Dependency Grammar, this seems to falsify my claim, but it does not since Word Grammar is more powerful
  than usual Dependency Grammars. Hudson uses a network with some extra syntactic primitives that
  allow him to account for loops.
} For simple NPN constructions,
one could claim that \emph{after} is the head. \emph{After} would be categorized as a third-person singular
mass noun and select for two \nbar{}s. It would (non-compositionally) contribute the semantics stated above. But it is unclear how the general schema with arbitrarily
many repetitions of \emph{after} N could be accounted for. If one assumes that \emph{day after day}
forms a constituent, then the first \emph{after} in (\mex{1}) would have to combine an N with an NPN sequence.
\ea
day after [day [after day]]
\z
This means that we would have to assume two different items for \emph{after}: one for the
combination of \nbar{}s and another one for the combination of \nbar with NPN combinations. Note
that an analysis of the type in (\mex{0}) would have to project information about the \nbar{}s contained
in the NPN construction, since this information has to be matched with the single \nbar at the
beginning. In any case, a lexical analysis would require several highly idiosyncratic lexical items
(prepositions projecting nominal information and selecting items they usually do not select).
It is clear that a reduplication account of the NPN construction as suggested by
G.\ \citet{GMueller2011a} does not work, since patterns with several repetitions of PN as in
(\mex{0}) cannot be accounted for as reduplication. G.\ Müller (p.\,241) stated that reduplication works
for word-size elements only (in \ili{German}) and hence his account does not extend to the \ili{English}
examples given above. (\mex{1}) shows an attested \ili{German} example containing adjectives, which means
that G.\,Müller's approach is not appropriate for \ili{German} either.
\ea
\label{ex-schwarzen-stein}
\gll Die beiden tauchten nämlich geradewegs wieder aus dem heimischen Legoland auf, wo sie im
Wohnzimmer, schwarzen Stein um schwarzen Stein, vermeintliche Schusswaffen nachgebaut
hatten.\footnotemark\\
     the two    surfaced namely straightaway again   from the home Legoland \particle{} where they
     in.the living.room black brick after black brick alleged firearms recreated had\\%\german
\footnotetext{
  Attested example from the newspaper taz, 05.09.2018, p.\,20
}
\glt `The two surfaced straightaway from their home Legoland where they had recreated alleged
firearms black brick after black brick.'
\z

\noindent
\citet[\page 240]{Travis2003a} suggested a syntactic approach to the NPN construction. The trees she
provides are broken and contain symbols like Spec, so the details of the analysis are unclear, but
she assumes that the preposition is of category Q and Q heads are special reduplication heads. An
element from inside of the complement of Q is moved to SpecQP. The analysis begs several questions: why can incomplete constituents
move to SpecQP? How is the external distribution of NPN constructions accounted for? Are they QPs?
Where can QPs appear? Why do some NPN constructions behave like NPs? How is the meaning of this construction accounted
for? If it is assigned to a special Q, the question is: how are examples like
(\ref{cxg:ex-npn-iteration}) accounted for? Are two Q heads assumed? And if so, what is their
semantic contribution?
% Travis allows for copying of substructures, so she may copy a part of the reduplication. But why?
% Furthermore, \citet{Lu2021a} made the interesting observation that Travis'
% analysis can account for occurrences of 2, 4, and 8 Ns but not of 3, 5, 6, or 7 Ns since only existing
% phrases can be duplicated.

%% \subsection{\emph{To hell with this government!}}

%% \citet{Jacobs2008a}

%% \citet[Section~21.10.1]{MuellerGT-Eng1}

This subsection showed how a special phrasal pattern can be analyzed within HPSG. The next section
will discuss filler-gap constructions, which were analyzed as instances of a single schema by
\citet[\page 164]{ps2} but which were later reconsidered and analyzed as a family of subconstructions by
\citew{Sag97a,Sag2010b}.

\subsection{Specialized sub-constructions}

HPSG took over the treatment of nonlocal dependencies from GPSG (\citealt{Gazdar81a}; see also
\crossrefchapteralt{evolution} on the history of HPSG and \crossrefchapteralt{udc} on
unbounded dependencies). \citet[Chapters~4 and~5]{ps2} had
an analysis of topicalization constructions like (\mex{1}) and an analysis of relative
clauses. However, more careful examination revealed that more fine-grained distinctions have to be
made. \citet[\page 491]{Sag2010b} looked at the following examples:
\settowidth\jamwidth{(topicalized clause)}
\eal
\ex {}[My bagels,] she likes.                           \jambox{(topicalized clause)}
\ex {}[\emph{What} books] do they like?                 \jambox{(\emph{wh}-interrogative)}
\ex (the person) [\emph{who} (\emph{se} book)] they like \jambox{(\emph{wh}-relative)}
\ex {}[\emph{What a} play] he wrote!                    \jambox{(\emph{wh}-exclamative)}
\ex {}[\emph{the more} books] they read \ldots          \jambox{(\emph{the}-clause)}
\zl
As Sag shows, the fronted element is specific to the construction at hand:
\eal
\ex[*]{
[\emph{Which} bagels] / [\emph{Who}], she likes. \jambox{(topicalized clause)}
}
\ex[*]{
[\emph{What a} book] do they like? \jambox{(\emph{wh}-interrogative)}
}
\ex[\%]{
the thing [[\emph{what}] they like] \jambox{(\emph{wh}-relative)}
}
\ex[*]{
[\emph{Which} bagels] / [\emph{What}] she likes! \jambox{(\emph{wh}-exclamative)}
}
\ex[*]{
[\emph{which} books] they read, the more they learn. \jambox{(\emph{the}-clause)}
}
\zl
%\begin{sloppypar}
\noindent
A topicalized clause should not contain a \emph{wh}-item (\mex{0}a), a \emph{wh}-interrogative should not
contain a \emph{what a} sequence appropriate for \emph{wh}-exclamatives (\mex{0}b), and so on.
%\end{sloppypar}

Furthermore, some of these constructions allow non-finite clauses and others do not:
\eal
\ex[*]{
Bagels, (for us) to like. \jambox{(topicalized clause)}
}
\ex[*]{
It's amazing [what a dunce (for them) to talk to]. (\emph{wh}-exclamative)
}
\ex[*]{
The harder (for them) to come, the harder (for them) to fall.\\  \jambox{(\emph{the}-clause)}
}
\ex[]{
I know how much time (* for them) to take.  \jambox{(\emph{wh}-interrogative)}
}
\ex[]{
The time in which (*for them) to finish.  \jambox{(\emph{wh}-relative)}
}
\zl
So there are differences as far as fillers and sentences from which something is extracted
are concerned. Sag discussed further differences like inversion/""non-inversion in the clauses out of
which something is extracted. I do not repeat the full discussion here but refer the reader to the
original paper.

In principle, there are several ways to model the phenomena. One could assume empty
heads as \citet[Chapter~5]{ps2} suggested for the treatment of relative clauses. Or one could assume empty
heads as they are assumed in Minimalism: certain so-called operators have features that have to be
checked and cause items with the respective properties to move \citep[\page 330--331]{Adger2003a}. \citet{Borsley2006a} discussed potential
analyses of relative clauses involving empty heads and showed that one would need a large number of
such empty heads, and since there is no theory of the lexicon in Minimalism, generalizations are
missed (see also \crossrefchapteralt[Section~\ref{minimalism-sec-empty-elements-for-relative-clauses}]{minimalism}). The alternative suggested by
\citet{Sag2010b} is to assume a general Filler-Head Schema of the kind assumed in \citew[\page 164]{ps2} and
then define more specific sub-constructions. To take an example, the \emph{wh}-exclamative is a filler-head
structure, so it inherits everything from the more general construction, but in addition, it
specifies that the filler daughter must contain a \emph{what a} part and states the semantics that is
contributed by the exclamative construction.


\section{Summary}
\label{sec-summary}

This paper summarized the properties of Construction Grammar, or rather Construction Grammars, and
showed that HPSG can be seen as a Construction Grammar, since it fulfills all the tenets assumed in
CxG: it is surface-based, grammatical constraints pair form and function/meaning, the grammars do
not rely on innate domain-specific knowledge, and the grammatical knowledge is represented in
inheritance hierarchies. This sets HPSG and CxG apart from other generative theories that either
assume innate language-specific knowledge (Minimalism, \eg
\citealt{Chomsky2013a,Kayne94a-u,CR2010a}) or do not assume inheritance hierarchies for all linguistic levels (\eg LFG).

I showed why lexical analyses of argument structure should be
preferred over phrasal ones and that there are other areas in grammar where phrasal analyses are
superior to lexical ones. I showed that they can be covered in HPSG, while they are problematic for
proposals assuming that all structures have to have a head.
\indexcxgend

%% \appendix
%% \section{Appendix: Sign-Based Construction Grammar}
%% \label{appendix-cxg}

%\section*{Abbreviations}
\section*{\acknowledgmentsUS}

I thank Anne Abeillé, Bob Borsley, Rui Chaves, and Jean-Pierre Koenig for comments on earlier
versions of this chapter and for discussion in general. I thank Frank Richter for discussion of
formal properties of SBCG. Thanks go to Frank Van Eynde for discussion of the Big Mess Construction
in relation to SBCG.


{\sloppy
\printbibliography[heading=subbibliography,notkeyword=this]
}
\end{document}

\if0

Stefan


Your outline made me think about what exactly construction grammar is. It seems to me it’s not a simple matter. It might be seen as any approach which rejects the idea emphasized by Chomskyans that traditional constructions are epiphenomena. Constructions appear to be just epiphenomena in any approach in which the syntax is just a few general combinatorial mechanisms, so not just P&P and Minimalism but also categorial grammar and early HPSG with its few ID schemata. This might mean that construction grammar is any framework which has more than just a few general combinatorial mechanisms. That would include HPSG since Sag (1997), but also GPSG with its numerous ID rules and classical TG with its numerous PS rules. But I assume most people wouldn’t see either GPSG and classical TG as forms of Construction Grammar.



This might suggest that form-meaning pairs is what’s key for construction grammar. That would include early GPSG with its pairs of syntactic and semantic rules, but not, I think, later GPSG with its few general semantic principles. It would also, I think, not include HPSG, since an HPSG phrase type may or may not have any semantic properties.


I’m inclined to see the most important feature of construction grammar as complex hierarchies of phrase types, allowing the grammar to capture very broad general facts, very specific facts, and everything in between. This includes HPSG since Sag (1997), but not early HPSG, and not P&P and Minimalism, categorial grammar, GPSG, or classical TG. However, I don’t know whether all frameworks calling themselves construction grammar have complex hierarchies of phrase types. If not, I don’t know how I would define construction grammar.


best


Bob



JP:

 This is a nice outline. I will make some high level remarks in the context of Bob’s comments. It’s clear that a decision was made to focus on that part of ConsGr that is closest to HPSG. It certainly makes sense for the Handbook, but of course the center of gravity of ConsGr is not that part which was closest to HPSG around 2012. So, I wonder a little more history of what ConsGram was is needed, particularly work by Fillmore and Kay (I assume Goldberg will be discussed in section 3), which presumably should be cited. Two things in particular might be stressed:

1) Typing of constructions was not part of the original ConsGr if memory serves me right,
2) Constructions that are not of depth 1 and thus resolute non-locality as a possibility was parts and parcels of ConsGram.

Finally, assigning meaning to phrase structure combinations (and meaning that can be quite more detailed than a standard type-logical meaning, see the Let alone paper, is a big thing of ConsGram and something that clearly can be done in HPSG (see Ginzburg and Sag), but is not always embraced. So, I remember Frank telling me LRS typically does not go for that, although he is quick to acknowledge nothing prevents it.



As for comments, I suggest you discuss SBCG, and in particular, cite
Boas & Sag 2012. I also suggest discussing Lee-Goldman 2011, a SBCG
dissertation from Berkeley available at
https://escholarship.org/uc/item/02m6b3hx

Finally, note that Adele Goldberg has now a new book  "Explain me
this: Creativity, Competition, and the Partial Productivity of
argument structure constructions" Princeton University, which you may
want to check too.



BB: 04.05.19
A few suggestions for possible revisions.



It still seems to me that the most important feature of Construction Grammar is a rejection of the view that grammar is just a few very general combinatorial mechanisms. The Chomskyan mainstream has assumed this view since the early 80s, and it seems to me that both categorial grammar and early HPSG subscribe to it. As I understand it, you think Minimalism is not a form of Construction Grammar because of its abstractness. No doubt a ‘what you see is what you get’ is an important feature of the various forms of Construction Grammar, but it seems to me that the insistence that a few very general combinatorial mechanisms are not sufficient is more important and I would be inclined to emphasize this.



I quite like Goldberg’s definition of a construction on p.839, and I think you might make a bit more of it. On this definition it follows that VP and PP are probably not constructions because their properties are predictable from the head-complement construction and the verbs and prepositions that they contain. On the other hand, wh-interrogative and wh-relative are constructions because they have properties which are not predictable either from more general constructions like head-filler-phrase or the lexical items that they contain. I think it’s important to say something about what is not a construction as well as about what is.



I also think it would be good to have the material in section 4 before the material in section 2. I think you should first show that grammar cannot be just a few very general combinatorial mechanisms, and then present the argument against the over-use of constructions.




Some more minor points:



p.837, fn.1. Is HPSG not a phrase structure grammar because it doesn’t have phrase structure rules? If so, GPSG wasn’t a phrase structure grammar either except in its earliest incarnations.




p.841, Tenet 1. Not all frameworks have morphemes.



p.841, bottom. Tenet 3 also excludes Postal’s Metagraph Grammar although it’s not a derivational model.






p.849, near the bottom. You refer to GPSG phrase structure rules. Shouldn’t it be like GPSG immediate dominance rules.






\fi



%      <!-- Local IspellDict: en_US-w_accents -->
