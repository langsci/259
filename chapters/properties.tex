\documentclass[output=paper
	        ,collection
	        ,collectionchapter
 	        ,biblatex
                ,babelshorthands
                ,newtxmath
                ,draftmode
                ,colorlinks, citecolor=brown
]{langscibook}

\IfFileExists{../localcommands.tex}{%hack to check whether this is being compiled as part of a collection or standalone
  % add all extra packages you need to load to this file 

\usepackage{graphicx}
\usepackage{tabularx}
\usepackage{amsmath} 
\usepackage{multicol}
\usepackage{lipsum}
%%%%%%%%%%%%%%%%%%%%%%%%%%%%%%%%%%%%%%%%%%%%%%%%%%%%
%%%                                              %%%
%%%           Examples                           %%%
%%%                                              %%%
%%%%%%%%%%%%%%%%%%%%%%%%%%%%%%%%%%%%%%%%%%%%%%%%%%%%
% remove the percentage signs in the following lines
% if your book makes use of linguistic examples


\usepackage{./langsci/styles/langsci-gb4e} 
\usepackage{./langsci/styles/langsci-optional} 
\usepackage{./langsci/styles/langsci-lgr}
\usepackage{./langsci/styles/langsci-forest-setup}
\usepackage{morewrites}



% Stefan Müller's styles
\usepackage{./styles/merkmalstruktur,./styles/abbrev,./styles/makros.2e,./styles/my-xspace,./styles/article-ex,
./styles/eng-date}

\usepackage{./langsci/styles/jambox}

% Crossing out text
% uncomment when needed
%\usepackage{ulem}

\usepackage{./styles/additional-langsci-index-shortcuts}

\usepackage{./styles/avm+}

\renewcommand{\tpv}[1]{{\avmjvalfont\itshape #1}}

\regAvmFonts

\usepackage{theorem}

\newtheorem{mydefinition}{Def.}
\newtheorem{principle}{Principle}

{\theoremstyle{break}
\newtheorem{schema}{Schema}
\newtheorem{mydefinition-break}[mydefinition]{Def.}
\newtheorem{principle-break}[principle]{Principle}
}

\usepackage{subfig}

  %add all your local new commands to this file

\makeatletter
\def\blx@maxline{77}
\makeatother
  %% -*- coding:utf-8 -*-

%%%%%%%%%%%%%%%%%%%%%%%%%%%%%%%%%%%%%%%%%%%%%%%%%%%%%%%%%%%%
%
% gb4e

% fixes problem with to much vertical space between \zl and \eal due to the \nopagebreak
% command.
\makeatletter
\def\@exe[#1]{\ifnum \@xnumdepth >0%
                 \if@xrec\@exrecwarn\fi%
                 \if@noftnote\@exrecwarn\fi%
                 \@xnumdepth0\@listdepth0\@xrectrue%
                 \save@counters%
              \fi%
                 \advance\@xnumdepth \@ne \@@xsi%
                 \if@noftnote%
                        \begin{list}{(\thexnumi)}%
                        {\usecounter{xnumi}\@subex{#1}{\@gblabelsep}{0em}%
                        \setcounter{xnumi}{\value{equation}}}
% this is commented out here since it causes additional space between \zl and \eal 06.06.2020
%                        \nopagebreak}%
                 \else%
                        \begin{list}{(\roman{xnumi})}%
                        {\usecounter{xnumi}\@subex{(iiv)}{\@gblabelsep}{\footexindent}%
                        \setcounter{xnumi}{\value{fnx}}}%
                 \fi}
\makeatother

% the texlive 2020 langsci-gb4e adds a newline after \eas, the texlive 2017 version was OK.
\makeatletter
\def\eas{\ifnum\@xnumdepth=0\begin{exe}[(34)]\else\begin{xlist}[iv.]\fi\ex\begin{tabular}[t]{@{}b{.99\linewidth}@{}}}
\makeatother


%%%%%%%%%%%%%%%%%%%%%%%%%%%%%%%%%%%%%%%%%%%%%%%%%%%%%%%%%%
%
% biblatex

% biblatex sets the option autolang=hyphens
%
% This disables language shorthands. To avoid this, the hyphens code can be redefined
%
% https://tex.stackexchange.com/a/548047/18561

\makeatletter
\def\hyphenrules#1{%
  \edef\bbl@tempf{#1}%
  \bbl@fixname\bbl@tempf
  \bbl@iflanguage\bbl@tempf{%
    \expandafter\bbl@patterns\expandafter{\bbl@tempf}%
    \expandafter\ifx\csname\bbl@tempf hyphenmins\endcsname\relax
      \set@hyphenmins\tw@\thr@@\relax
    \else
      \expandafter\expandafter\expandafter\set@hyphenmins
      \csname\bbl@tempf hyphenmins\endcsname\relax
    \fi}}
\makeatother


% the package defined \attop in a way that produced a box that has textwidth
%
\def\attop#1{\leavevmode\begin{minipage}[t]{.995\linewidth}\strut\vskip-\baselineskip\begin{minipage}[t]{.995\linewidth}#1\end{minipage}\end{minipage}}


%%%%%%%%%%%%%%%%%%%%%%%%%%%%%%%%%%%%%%%%%%%%%%%%%%%%%%%%%%%%%%%%%%%%


% Don't do this at home. I do not like the smaller font for captions.
% This does not work. Throw out package caption in langscibook
% \captionsetup{%
% font={%
% stretch=1%.8%
% ,normalsize%,small%
% },%
% width=\textwidth%.8\textwidth
% }
% \setcaphanging


  \togglepaper[1]
}{}

\title{Basic properties and elements} 
\author{%
 Anne Abeillé\affiliation{Université Paris Diderot}%
 \lastand Robert D. Borsley\affiliation{University of Essex}%
}
% \chapterDOI{} %will be filled in at production

%\epigram{Change epigram in chapters/01.tex or remove it there }

\abstract{Head-driven Phrase Structure Grammar (HPSG) is a declarative and monostratal version of generative grammar, in which linguistic expressions have a single relatively simple constituent structure. It seeks to develop detailed formal analyses using a system of types, features, and constraints. Constraints on types of \emph{lexical-sign} are central to the lexicon of a language and constraints on types of \emph{phrase} are at the heart of the syntax, and both lexical and phrasal types include semantic and phonological information. Different versions of the framework have been developed, including versions in which constituent order is a reflection not of constituent structure but of a separate system of order domains and the Sign-Based Construction Grammar version, which make a fundamental distinction between signs of various kinds and the constructions which license them.%
%
\footnote{We are grateful to Stefan Müller, Jean-Pierre Koenig, and Frank Richter for many helpful comments on earlier versions of this chapter. We alone are responsible for what appears here.}
}


%%%%%%%%%%%%%%%%%%%%%%%%%%%%%%%%%%%%%%%%%%%%%%%%%%%%%%%%%%%%
\begin{document}
\maketitle
\label{chapter-basic-properties}\label{chap-properties}

\inlinetodostefan{footnote from the title had to be moved here bc it created compilation errors}
\inlinetodostefan{Fix the label for page-hfp \label{page-hfp}}


%%%%%%%%%%%%%%%%%%%%
%%%%%%%%%%%%%%%%%%%%
\section{Introduction} 
Head-driven Phrase Structure Grammar (HPSG) dates back to early 1985 when Carl Pollard presented his Lectures on HPSG. It was often seen in the early days as a revised version of the earlier Generalized Phrase Structure Grammar (GPSG) framework \citep{GKPS85a}, but it was also influenced by Categorial Grammar, and, as \citet[1]{ps} emphasized, by other frameworks like Lexical-Functional Grammar (LFG) \citep{Bresnan82a-ed}, as well. Naturally it has changed in various ways over the decades. This is discussed in much more detail in the next chapter (Flickinger et al., this volume, chapter~\ref{chap-evolution}), but it makes sense here to distinguish three versions of HPSG. Firstly, there is what might be called early HPSG, the framework presented in \citet{ps} and \citet{ps2}%
%
\footnote{As discussed in Richter, this volume, chapter~\ref{chap-formal-background}, the approaches that are developed in these two books have rather different formal foundations. However, they propose broadly similar syntactic analyses, and for this reason it seems reasonable to group them together as early HPSG.}.
%
This has most of the properties of more recent versions but only exploits the analytic potential of type hierarchies to a limited degree \citep{Flickinger87,FPW85a}. Next there is what is sometimes called Constructional HPSG, the framework adopted in \citet{Sag97a,GSag2000a-u}, and much other work. Unlike earlier work this uses a rich hierarchy of phrase-types. This is why it is called constructional%
%
\footnote{As discussed below, HPSG has always assumed a rich hierarchy of lexical types. One might argue, therefore, that it has always been constructional.}.
%
Finally, in the 2000s, Sag developed a version of HPSG called \emph{Sign-Based Construction Grammar} (SBCG) \citep{Sag2012a}. The fact that this approach has a new name suggests that it is very different from earlier work, but probably most researchers in HPSG would see it as a version of HPSG, and it was identified as such in \citet[486]{Sag2010b}. Its central feature is the special status it assigns to constructions. In earlier work they are just types of sign, but for SBSG signs and constructions are quite different objects. In spite of this difference, most analyses in Constructional HPSG could probably be translated into SBCG and vice versa. In this chapter we will concentrate on the ideas of Constructional HPSG, which is probably the version of the framework that has been most widely assumed. We will comment briefly on SBCG in the final section.

\inlinetodostefan{Add sec refs}
The chapter is organized as follows. In section~2, we set out the properties that characterize the approach and the assumptions it makes about the nature of linguistic analyses and the conduct of linguistic research. Then, in section~3, we consider the main elements of HPSG analyses: types, features, and constraints. In section~4 we look more closely at the HPSG approach to the lexicon, and in section~5, we outline the basics of the HPSG approach to syntax. In section~6, we look at some further syntactic structures, and in section~7, we consider some further topics, including SBCG. Finally, in section~8, we summarize the chapter.


%%%%%%%%%%%%%%%%%%%%
%%%%%%%%%%%%%%%%%%%%
\section{Properties}
Perhaps the first thing to say about HPSG is that it is a form of generative grammar in the sense of \citet{Chomsky57a}. This means that it seeks to develop precise and explicit analyses of grammatical phenomena. But unlike many versions of generative grammar, it is a declarative or constraint-based approach to grammar, belonging to what \citet{PullumScholz2001} call ‘Model Theoretic syntax’. As such, it assumes that a linguistic analysis involves a set of constraints to which linguistic objects must conform, and that a linguistic object is well-formed if and only if it conforms to all relevant constraints.%
%
\footnote{In most HPSG work, all constraints are equal. Hence, there is no possibility as there is in Optimality Theory of violating one if it is the only way to satisfy another more important one \citep{Malouf2003a}. However, see \citealt{MK2000a-cr,OFTM2004a-u} for an HPSG parser with probabilities or weighted constraints.}
%
This includes linguistic objects of all kinds – words, phrases, phonological segments, etc. There are no procedures constructing representations such as the phrase structure and transformational rules of classical transformational grammar or the Merge and Agree operations of Minimalism. Of course, speakers and hearers do construct representations and must have procedures that enable them to do so, but this is a matter of performance, and there is no need to think that the knowledge that is used in performance has a procedural character. Rather, the fact that it used in both production and comprehension (and other activities, e.g. translation) suggests that it should be neutral between the two and hence declarative. For further discussion of the issues, see e.g. \citet{PullumScholz2001,Postal2003a} and \citet{SW2011a,SW2015a}.

HPSG is also a monostratal approach, which assumes that linguistic expressions have a single constituent structure. This makes it quite different from transformational grammar, in which an expression can have a number of constituent structures. It means among other things that there is no possibility of saying that an expression occupies one position at one level of structure and another at another. Hence, HPSG has nothing like the movement processes of transformational grammar. The relations that are attributed to movement in transformational work are captured by constraints that require certain features to have the same value. For example, as discussed in section~4\todo{add sec ref}, a raising sentence is one with a verb which has the same value for the feature \textsc{subj(ect)} as its complement and hence combines with whatever kind of subject its complement requires.

HPSG is sometimes described as a concrete approach to syntax. This description refers not only to the fact that it assumes a single constituent structure but also to the fact that this structure is relatively simple, especially compared with the structures that are postulated within Minimalism. Unlike Minimalism, HPSG does not assume that all branching is binary. This inevitably leads to simpler flatter structures. Also unlike Minimalism, it makes limited use of phonologically empty elements. For example, it is not assumed, as in Minimalism, that because some clauses contain a complementizer they all do, an empty one if not an overt one. Similarly, it is not assumed that because some languages like English have determiners, they all do, overt or covert. It is also not generally assumed that null subject sentences, such as (\ref{ex:prop1b}) from Polish, have a phonologically empty subject in their constituent structure. Thus, the constituent structure of the two following sentences are quite different, even if their semantics are similar:

\ea\label{ex:prop1}
	\ea\label{ex:prop1a}
	I read a book.
	\ex\label{ex:prop1b}
	\gll Czytałem książkę.\\
	read\textsc{.pst.1sg} book\textsc{.acc}\\
	\glt ‘I read a book.’
	\z
\z

It is also assumed in much HPSG work that there are no phonologically empty elements in the constituent structure of an unbounded dependency construction such as the following:

\ea\label{ex:prop2}
What did you say?
\z

On this view, the verb \emph{say} in (\ref{ex:prop2}) does not have an empty complement. There is, however, some debate here (\citealp{SF94a,Mueller2004e}; Borsley \& Crysmann, this volume, chapter~\ref{sec:UDC:MoreOnGaps}).

{\color{red}
A further important feature of HPSG is a rejection of the Chomskyan idea that grammatical phenomena can be divided into a core, which merits serious investigation, and a periphery, which can be safely ignored.%
%
\footnote{This is not to deny that some constructions are more canonical and more frequent in use than others and that this may be important in various ways.}
%
This means that it is not only concerned with such ‘core’ phenomena as \emph{wh}-interrogatives, relative clauses, and passives but also with more ‘peripheral’ phenomena such as the following:
}

\ea\label{ex:prop3}
	\ea\label{ex:prop3a}
	It’s amazing the people you see here.
	\ex\label{ex:prop3b}
	The more I read, the more I understand.
	\ex\label{ex:prop3c}
	Chris lied his way into the meeting.
	\z
\z

These exemplify the nominal extraposition construction \citep{MichaelisLambrecht1996}, the comparative correlative construction \citep{Abeille2006a,AB2008a-u,Borsley2011a-u}, and the \emph{X’s Way} construction \citep[7.4]{KF99a,Sag2012a}. As we will see, HPSG is an approach which is able to accommodate broad linguistic generalizations and highly idiosyncratic facts and everything in between.%
%
\footnote{Idioms have also been an important focus of research in HPSG. See e.g. \citealt{RS2009a,KM2019a}, and Sailer, this volume, chapter~\ref{chap-idioms}.}
%

Another notable feature of the framework since the earliest work is a concern with semantics as well as syntax. More generally, it does not try to reduce either semantics or morphology to syntax (see Crysmann, this volume, chapter~\ref{chap-morphology}, Koenig and Richter, this volume, chapter~\ref{chap-semantics}). We will comment further on this in the following sections.

We turn now to some assumptions which are more about the conduct of linguistic research than the nature of linguistic analyses. Firstly, HPSG emphasizes the importance of firm empirical foundations and of detailed formal analyses of the kind advocated by Chomsky in \emph{Syntactic Structures} \citep{Chomsky57a}. Whereas transformational work typically offers sketches of analyses which might be fleshed out one day, HPSG commonly provides detailed analyses which can be set out in an appendix. A notable example is \citet{GSag2000a-u}, which sets out its analysis of English interrogatives in a 50 page appendix. Arguably, one can only be fully confident that a complex analysis works if it is incorporated into a computer implementation. Hence, computer implementations of HPSG analyses are also quite common (see e.g. \citealp{Babel,MuellerCoreGram,Copestake2002a,BDFPS2010a-u,Bender2016}, and Bender and Emerson, this volume, chapter~\ref{chap-cl}).


%\avm{[ \type*{A type spanning a line}
%attr & [\type{type}] ]}


%\avm{
%	[ ctxt & [ max-qud \\
%		sal-utt & \{ [ cat \\
%			cont <ind & i> ] \}
%			]
%	]
%}


Another property of the framework is a rejection of abstract analyses with tenuous links to the observable data. As we noted above, phonologically empty elements are only assumed if there is compelling evidence for them.%
%
\footnote{There may be compelling evidence for some empty elements in some languages. Thus, \citet[section~8]{Borsley2009a-u}Borsley argues that Welsh has phonologically empty pronouns. For general discussion of empty elements, see \citet[chapter~19.2]{MuellerGT-Eng1-linked}.}
%
Similarly, overt elements are only assumed to have properties for which there is clear evidence. For example, words are only assumed to have case or agreement features if there is some concrete morphological evidence for them, as in Polish, illustrated in (\ref{ex:prop1b}). This feature of HPSG stems largely from considerations about acquisition (\citealt[chapter~19]{MuellerGT-Eng1-linked}; Ginzburg this volume, Borsley and Müller, this volume, chapter~\ref{sec-acquisition-minimalism}). Every element or property which is postulated for which there is no clear evidence in the data increases the complexity of the acquisition task and hence necessitates more complex innate machinery. This suggests that such elements and properties should be avoided as much as possible. It has important implications both for the analysis of individual languages and for how differences between languages are viewed.
	
A related property of the framework is a rejection of the idea that it is reasonable to assume that a language has some element or property if some other languages do. Many languages have case and many languages have agreement, but for HPSG it does not follow that they all do. As \citet[25]{MuellerCoreGram} puts it, “Grammars should be motivated on a language-specific basis.” Does this mean that other languages are irrelevant when one investigates a specific language? Clearly not. As Müller also puts it, “In situations where more than one analysis would be compatible with a given dataset for language X, the evidence from language Y with similar constructs is most welcome and can be used as evidence in favour of one of the two analyses for language X.” (\citeyear[43]{MuellerCoreGram})


%%%%%%%%%%%%%%%%%%%%
%%%%%%%%%%%%%%%%%%%%
\section{Elements}

For HPSG, a linguistic analysis is a system of types (or sorts), features, and constraints. Types provide a complex classification of linguistic objects, features identify their basic properties, and constraints impose further restrictions. In this section, we will explain these three elements. We note at the outset that HPSG distinguishes between the linguistic objects (lexemes, words phrases, etc.) and descriptions of such objects. Linguistic objects must have all relevant properties of their description and cannot be underspecified in any way. Descriptions in contrast can be underspecified and in fact always are.

There are many different kinds of types, but particularly important is the type sign and its various subtypes. For \citet[19]{GSag2000a-u}, this type has the subtypes lexical-sign and phrase, and lexical-sign has the subtypes lexeme and word. (Types are written in lower case italics.) Thus, we have the following type hierarchy in Figure \ref{fig:prop1}.


\begin{figure}[h!]
\begin{forest}
[\emph{sign}
	[\emph{lexical-sign}
		[\emph{lexeme}]
		[\emph{word}]
	]
	[\emph{phrase}]
]
\end{forest}

\caption{A hierarchy of types of signs}\label{fig:prop1}
\end{figure}


\emph{Lexeme}, \emph{word} and \emph{phrase} have a complex system of subtypes. The type \emph{lexical-sign}, its subtypes, and the constraints on them are central to the lexicon of a language, while the type \emph{phrase}, its subtypes, and the constraints on them are at the heart of the syntax. In both cases, complex hierarchies mean that the framework is able to deal with broad, general facts, very idiosyncratic facts, and everything in between. We will say more about this below.

Signs are obviously complex objects with (at least) phonological, syntactic and semantic properties. Hence, the type \emph{sign} must have features that encode these properties. For much work in HPSG, phonological properties are encoded as the value of a feature \textsc{phon(ology)}, whose value is a list of objects of type \emph{phon}, while syntactic and semantic properties are grouped together as the value of a feature \textsc{synsem}, whose value is an object of type \emph{synsem}. (Feature or attributes are written in small caps.) A type has certain features associated with it, and each feature has a value of some kind. A bundle of features can be represented by an attribute"=value"=matrix (AVM) with the type name at the top on the left hand side and the features below followed by their values. Thus, signs can be described as follows:

%\ea\label{ex:prop4}
%\avm{
%[\type{sign}\\
%phon & \normalfont{list(}phon\normalfont{)}\\
%synsem & synsem]
%}
%\z


The descriptions of specific signs will obviously have specific values for the two features. For example, we might have the following simplified AVM for the phrase \emph{the cat}:

%\ea\label{ex:prop5}
%\avm{
%	[\type{phrase}\\
%	phon & <the,cat>\\
%	synsem & NP]
%\z


Here, following a widespread practice, we use standard orthography instead of real \emph{phon} objects%
%
\footnote{See Bird \& Klein 1999, Höhle 2018%
	%
	\inlinetodostefan{both not given in references, r they not supposed to appear there? What section r the others referring to?}
	%
	and de Kuthy (this volume, section~7), Abeillé and Chaves (this volume, section~7).}%
%
, and we use the traditional label NP as an abbreviation for the relevant \emph{synsem} object. We will say more about \emph{synsem} objects shortly. First, however, we must say something about phrases.



















%%%%%%%%%%%%%%%%%%%%%%%%%%%%%%%%%%%%%%%%%%%%%%%%%%%%%%%%%%%%
%%%%%%%%%%%%%%%%%%%%%%%%%%%%%%%%%%%%%%%%%%%%%%%%%%%%%%%%%%%%

\begin{table}
\caption{Frequencies of word classes}
\label{tab:1:frequencies}
 \begin{tabular}{lllll} % add l for every additional column or remove as necessary
  \lsptoprule
            & nouns & verbs & adjectives & adverbs\\ %table header
  \midrule
  absolute  &   12 &    34  &    23     & 13\\
  relative  &   3.1 &   8.9 &    5.7    & 3.2\\
  \lspbottomrule
 \end{tabular}
\end{table}

\citep{Chomsky57a}.

\citet{Meier2017}
\ea\label{ex:1:descartes}
\langinfo{Latin}{}{personal knowledge}\\
\gll cogit-o ergo sum \\
think-1{\SG}.{\PRS}.{\IND} hence exist.1{\SG}.{\PRS}.{\IND}\\
\glt `I think therefore I am'
\z


\is{prolegomena}
Sed cursus \footnote{eros condimentum mi consectetur, ac consectetur} sapien pulvinar. Sed consequat, magna\footnote{eu scelerisque laoreet, ante erat tristique justo, nec cursus eros diam eu nisl. Vestibulum non arcu tellus}. Nunc dignissim tristique massa ut gravida. Nullam auctor orci gravida tellus egestas, vitae pharetra nisl porttitor. Pellentesque turpis nulla, venenatis id porttitor non, volutpat ut leo. Etiam hendrerit scelerisque luctus. Nam sed egestas est. Suspendisse potenti. Nunc vestibulum nec odio non laoreet. Proin lacinia nulla lectus, eu vehicula erat vehicula sed. 

\section*{Abbreviations}
\begin{tabularx}{.45\textwidth}{lX}
\textsc{cop} & copula\\ 
\textsc{fv} & final vowel\\
\end{tabularx}
\begin{tabularx}{.45\textwidth}{lX}
\textsc{neg} & negation\\ 
\textsc{sm} & subject marker\\
\end{tabularx}


\section*{Acknowledgements}

{\sloppy
\printbibliography[heading=subbibliography,notkeyword=this]
}
\end{document}
