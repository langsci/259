%% -*- coding:utf-8 -*-






\begin{document}
% \chapterDOI{} %will be filled in at production

%\epigram{Change epigram in chapters/03.tex or remove it there }

\abstract{Coordination is a central topic in theoretical linguistics. Following GPSG, which provided the first formal analysis of unlike coordination, HPSG has developed detailed analyses of different coordination constructions in a variety of unrelated languages. Central to the HPSG analyses are two main ideas: (i) coordination structures are non-headed phrases, and (ii) coordinate daughters display some kind of parallelism, which is captured by feature sharing. From these ideas, specific properties can be derived, regarding extraction and agreement, for instance. Many HPSG analyses also agree that coordination is a cover term for a wide variety of different constructions which can be viewed as different subtypes of coordinate phrases, and which can be cross-classified with other subtypes of the grammar (nominal or not, with ellipsis or not, etc.). We present the description of various coordination phenomena and show that HPSG can account for their subtle properties, while integrating them into the general organization of the grammar.}

\maketitle
\label{chap-coordination}


%\if0

\section{Introduction} 

In this chapter we refer to expressions like \emph{and}, \emph{either},  \emph{or}, \emph{but}, 
\emph{let alone}, etc.\ as \emph{coordinators}\is{coordinator} and the phrases that a coordinator can combine with as  \emph{coordinands}\is{coordinand}.
Thus,  in ``A or B'', both A and B are coordinands and \emph{or} is the coordinator. 
A great deal of research has been dedicated to the topic of coordination structures in the last  70 years, spanning a multitude of different approaches in many different theoretical frameworks.  With regard to the linguistic problems, research questions abound. In the realm of syntax there is much debate concerning the role of coordination lexemes, the existence of null coordinators, the syntactic relationship between coordinands, the peculiar extraction phenomena that certain coordination structures exhibit, the necessary properties that allow two different structures to be coordinated, the relation between coordination structures and comparative and subordination structures, peculiar ellipsis phenomena that can optionally occur, the various patterns of agreement that obtain in nominal coordination structures, the distribution and syntactic realization of the lexemes \emph{either} and \emph{or}, etc. In the realm of semantics, the issues are no less complex, and the debate no less lively. There are many questions pertaining to how exactly the meaning of coordination structures is construed. 

Among the first attempts to offer a precise formalization of the syntax and semantics of coordination was the seminal work of \citet{gazdarc}. Other seminal work soon followed, including the demonstration that phrase structure grammar offered a way to model filler-gap dependencies and certain island constraints \citep{gazdar}. In particular, Gazdar's account showed how long-distance dependencies involving multiple gaps linked to the same filler phrase could be modeled straightforwardly, something that mainstream movement-based models still struggle with to this day. Finally, there were also 
 in-depth examinations of a number of complex empirical phenomena in  \citet{gazd1982}, which  proved highly influential in the genesis of Generalized Phrase Structure Grammar, and later, of HPSG. Coordination thus has a special place in the history of HPSG, and still figures  in many theoretical arguments within Generative Grammar,  given the extremely challenging phenomena it poses for linguistic theory. 
Nevertheless, there is no clear consensus, even within HPSG, about how to analyze coordination. For example, in some accounts the coordinator
expression is a weak head, whereas in others it is a marker. Coordinate structures are binary branching in some accounts, but not so in others. Finally, in  some accounts, non-constituent coordination involves some form of deletion, but in others, no deletion operation is assumed.  
In this chapter we survey the empirical arguments and formal accounts of coordination, with special focus on its morphosyntax.

\section{Headedness}
\label{coord:sec-headedness}

The head of a construction is traditionally defined as the constituent which determines the syntactic distribution and the meaning of the whole, and it is also often the case that a dependent can be omitted, fronted, or extraposed while the head cannot be \citep{zwicky85}. In coordination constructions, something very different occurs. First, the syntactic category and the distribution of a coordinate phrase is collectively determined by the coordinands, not by one particular coordinand nor by the coordination particle. Thus, an S coordination yields an S, a VP coordination yields a VP, and so on, for virtually all categories.\footnote{The exceptions include coordinator expressions themselves, e.g.\ *~\emph{You ordered a coffee and or or a tea?} This oddness may be due to the coordinands being of the wrong semantic type. See Section~\ref{lexcoord} for more on lexical coordination.}
This is perhaps clearer in cases like (\ref{c0}), where
expressions such as \emph{simultaneously}, \emph{both}, and
\emph{together} can be used to show that the entire bracketed string
is interpreted as a complex unit denoting a plurality.


\eal
\label{c0}
\ex{} [[Tom sang]\subl{s} and [Mia danced]\subl{s}]\subl{s} simultaneously.

\ex{} Often [[Kim goes to the beach]\subl{s} and [Sue goes to the city]\subl{s}]\subl{s}.

\ex{} Sue [[read the instructions]\subl{vp} and [dried her hair]\subl{vp}]\subl{vp} in twenty seconds.

\ex{} You can't simultaneously [[drive a car]\subl{vp} and [talk on the phone]\subl{vp}]\subl{vp}.

\ex{} Simultaneously [[shocked]\subl{vp} and [saddened]\subl{vp}]\subl{vp}, Robin decided to go home.

\ex Robin is both [[tall]\subl{a} and [thin]\subl{a}]\subl{a}.

\ex{} [[Tom]\subl{np} and [Mia]\subl{np}]\subl{np} agreed to jump into the water together.
\zl


Generally, a coordinate structure has the same grammatical function and category as the coordinands:
given a number of coordinands of category X, the distribution of the coordinate constituent that is
obtained is again the same as of an X constituent, what \citet[\page 752]{pullumzwicky} refer to as \isi{Wasow's Generalization}.
In particular, this is what allows coordination to apply recursively:

\eal
\ex {}[[Tom and Mary]\subl{np} or [Mia and Sue]\subl{np}]\subl{np} got married.
\ex I can either [[sing and dance]\subl{vp} or [sing and play the guitar]\subl{vp}]\subl{vp}.
\ex Either [[John went to Paris and Kim went to Brussels]\subl{s} or [none of them ever left home]\subl{s}]\subl{s}.
\zl

Another piece of evidence in favor of a non"=headed analysis comes from the fact that there is no typological correlation between the position of the coordinator and the head directionality \citep{zwart}.
%\itd{Stefan: Give example?} 
For example, in Zwart's  survey of 136 languages where half are verb-final and half
verb-initial,  verb-final languages overwhelmingly employ coordinator-initial strategies.
%\itd{Stefan: The coordination is in the middle, isn't it?}
% Rui: by coordination initial it is
%meant "A [& B]" and as coordination final it is meant "A [B &]", but I
%think that is clear from the prose... and for those readers that don't
%find it clear, they can see the details in the cited paper (Zwart
%2005). I don't think it is possible to foresee (and avoid) all
%possible misinterpretations that a paragraph can have.
In particular, 119 of these languages have exclusively coordinator-initial, 12 exhibit both coordinator-initial
and coordinator-final strategies, and only 4 have exclusively coordinator-final structures. 


Finally, coordination is also special in that the relationship between coordinands is unlike
adjunction \citep[\page 156--160]{levinepostal}.
% looked for coordination and conjunct
Whereas adjuncts can in principle be displaced, coordinands do not have any mobility, as (\ref{c2}) illustrates.

\eal
\label{c2}
\ex[] {Because/Since Jane likes music, Tom learned to play the piano.}
\ex[*] {And Jane likes music, Tom learned to play the piano.}
\zl


\noindent
Thus, no coordinand can usually be said to be a dependent. For example,  reversing the order of the coordinands in (\ref{c1}) causes no major change in meaning. Neither daughter can be said to be the head because no subordination dependency is established between coordinands.

\eal
\label{c1}
\ex Sam ordered a burger and Robin ordered a pizza.
\ex Robin ordered a pizza and Sam ordered a burger.
\zl

\noindent
To be sure, there are certain coordination structures like those in
(\ref{ex-robin-jumped-on-a-horse}) which do not have such symmetric 
interpretations \citep{goldsmith,lakoff86,levinprince86}.
Regardless, such constructions retain many of the properties that characterize coordinate structures, and therefore are likely to be
coordinate just the same \citep[Chapter~5]{kehler}.

\eal
\label{ex-robin-jumped-on-a-horse}
\ex Robin jumped on a horse and rode into the sunset.
\ex Robin rode into the sunset and jumped on a horse.
\zl
%the b example could be marked as ? or #

For these reasons, HPSG adopts a rather traditional non"=headed analysis of coordination, an approach  going back
to \citet[195]{bloom} and  \citet[Section~4.2]{ross67}, and later adopted in many other frameworks
such as \citet[Section~3.1]{pesetsky}, \citet[\page 407]{gazdarc}, and \citet[1275]{rodney}, among many others. 
See \citet{borsley94,Borsley2005a}
% ok: Borsley2005a, Borsley94a:243 mentiones non-headed 
and 
\citet[Chapter~2]{chavesthesis} for more discussion about previous claims in the literature that coordination structures are headed.
Finally, we note that the HPSG account is in agreement with \citet[196]{chom65}, who argued against postulating complex syntactic representations without direct empirical evidence:\footnote{In more recent times, Chomskyan theorizing has assumed that all structures should be binary branching purely on conceptual economy grounds; see \citet{Johnson:Lappin:99} for criticism.}

\begin{quote}
It has sometimes been claimed that the traditional coordinate structures are necessarily
right-recursive \citep{Yngve60a-u} or left-recursive (Harman, 1963, p.\,613, rule 3i). These
conclusions seem to me equally unacceptable. Thus to assume (with Harman) that the phrase ``a tall,
young, handsome, intelligent man'' has the structure [[[[tall young] handsome] intelligent] man]
seems to me no more justifiable than to assume that it has the structure [tall [young [handsome
[intelligent man]]]]. In fact, there is no grammatical motivation for any internal structure
[\ldots]. The burden of proof rests on one who claims additional structure beyond
this. \citep[196--197]{chom65} 
\end{quote}

\noindent
As we shall see, the empirical evidence suggests that
the simplest and most parsimonious structure for coordination is neither left- nor right-recursive.


\section{On the syntax of coordinate structures}


There is a wide range of coordination strategies in the languages of the world \citep{haspelmath}. In some languages, no coordinand is accompanied by any coordinator (syndenton coordination, as in \emph{We came, we saw, we conquered}), or one of the coordinands is accompanied by a coordinator (monosyndenton coordination,  as in \emph{We came, we saw, and we conquered}). Other strategies involve marking multiple coordinands with a coordinator (polysyndenton coordination;
\emph{We came, and we saw, and we conquered}), or all coordinands (omnisyndenton coordination;
\emph{Either you come or you go}).
All of these are schematically depicted in (\ref{types}); see
 \citet{Drellishak:Bender:05} for more discussion about how to accommodate such typological patterns in a computational HPSG platform.

\eal
\label{types}
\settowidth\jamwidth{(monosyndenton)}
\ex A, B, C                                      \jambox{(asyndenton)}
\ex A, B \emph{coord} C                          \jambox{(monosyndenton)}
\ex A \emph{coord} B \emph{coord} C              \jambox{(polysyndenton)}
\ex \emph{coord} A \emph{coord} B \emph{coord} C \jambox{(omnisyndenton)}
\zl


\noindent
 Finally, a single coordination strategy often serves to coordinate all types of constituent phrases, but in many languages, different coordination strategies only cover a subset of the types of phrases in the language. For example, in
\ili{Japanese} the clitic \emph{to} is used for nominal coordination
and \emph{te} is used for other coordinations.

In what follows, we start by focusing on monosyndenton coordination. There are three possible
structures one can assign to such coordinations, as Figure~\ref{f1} illustrates. The binary
branching approach (left) goes back to \citet[\page 456]{yngve}, and is used in HPSG work such as 
\citet[\page 200--205]{pollardsag}, \citet{Yatabe:03}, \citet{berthold03}, \citet{Beavers},
\citet{Drellishak:Bender:05}, \citet{chavesthesis}, and \citet{chavesextr}, among others.
% This is OK, since the papers are about coordination explicitely.
The flat branching approach (center) has also been  assumed in HPSG
\citep{Abeille:05,Abeille06,Mouret:05,Mouret:06,Bilbiie:17}, and the totally flat approach (right)
much less frequently  \citep{sagwasowbender,Sag:03}.\footnote{See \citet{Borsley2005a} for criticism
  of ConjP and of the binary branching analysis of coordinate structures with three
  coordinands. ConjP is also discussed in \crossrefchapterw[Section~\ref{sec-coordination-minimalism}]{minimalism}.}

\begin{figure}
\hfill
\begin{forest}
[X, baseline
 [X] 
 [X 
  [X] 
  [X 
   [Coord]  
   [X] ] ] ]
\end{forest}
\hfill
\begin{forest}
[X,baseline 
  [X]
  [X]
  [X 
    [Coord]
    [X] ] ]
\end{forest}
\hfill
\begin{forest}
[X,baseline 
  [X]
  [X]
  [Coord]
  [X] ]
\end{forest} 
\hfill\mbox{}
\caption{Three possible headless analyses of coordination}\label{f1}
\end{figure}


The binary branching analysis requires two different rules, informally depicted in (\ref{bin}), and a special feature to prevent the coordinator from recursively applying to the last coordinand, e.g.\ *\emph{Robin and and and Kim}. Otherwise, the two rules are unremarkable and are handled by the grammar like any other immediate dominance schema. See, for example, \citet{Beavers}
% paper OK
for a formalization.

%\inlinetodoobl{JP: This does not work for more than two X (at least as is)}
\eal
\label{bin}
\ex X$_{crd+}$ $\rightarrow$ \, \emph{Coord} \, X$_{crd-}$
 
\ex
\label{ex-x-cord-minus-cord-plus}
X $\rightarrow$ \, X$_{crd-}$  \,\, X$_{crd+}$
\zl
\itdsecond{Stefan: Doesn't (\mex{0}) also permit \emph{and Peter and Mary}?}

\noindent
\citet[Chapter~6]{Kayne:94} and  \citet[Chapter~3]{johann} argue that coordination follows X-bar theory and that the coordinator is the head of the construction; see \crossrefchaptert[Section~\ref{sec-coordination-minimalism}]{minimalism}. But in HPSG, even though one of the coordinands (or more) may combine with a coordinator, this subconstituent is not the head of the construction, which is considered as unheaded.
The two analyses are contrasted in Figure~\ref{f10}.

% \begin{figure}
% \hfill
%     \Tree[.ConjP NP1 [.Conj$'$  Coord NP2 ] ]
% \hfill
%     \Tree[.NP NP1 [.NP  Coord NP2 ] ]
% \hfill\mbox{}
% \caption{Binary-branching analyses of coordination, headed and non"=headed}\label{f10}
% \end{figure}
\begin{figure}
\hfill
\begin{forest}
[ConjP 
  [NP1] 
  [Conj$'$  
    [Conj]
    [NP2] ] ]
\end{forest}
\hfill
\begin{forest}
[NP 
  [NP1]
  [NP  
    [Coord]
    [NP2] ] ]
\end{forest}
\hfill\mbox{}
\caption{Binary-branching analyses of coordination, headed and non"=headed}\label{f10}
\end{figure}


Similarly, the flat branching analysis where the coordinator and the coordinand attach to each other  requires two  
rules as well (where $n \geq 1$):

\eal
\label{ok}
\ex\label{coordination:rule-coord-X} 
X$_{crd+}$ $\rightarrow$ \, \emph{Coord} \, X$_{crd-}$
 
\ex\label{coordination:rule-several-x} 
X $\rightarrow$ \, X$^1_{crd-}$  \,\, \ldots{} \,\, X$^n_{crd-}$ \,\, X$_{crd+}$
\zl

\noindent
However, the flat analysis requires only one rule, and no
special features at all, as (\ref{flat}) illustrates. 

\ea
\label{flat}
X  $\rightarrow$ X$^1$ \ldots{} X$^n$ \emph{Coord} \, X$_{n+1}$
\z

That said, there are some reasons for assuming that the coordinator does in fact combine with the
coordinand, as in (\ref{coordination:rule-coord-X}). First, in some languages of the world, the
coordinator is a bound morpheme instead of a free morpheme. For example, verbs are coordinated by
adding one of a set of suffixes to one of the coordinands in \ili{Abelam} (Papua New Guinea),
usually the first one in a coordination of two items.
Similarly, in \ili{Kanuri} (Nilo-Saharan), verb phrases are coordinated by marking the first verb
with a conjunctive form affix, and in languages like \ili{Telugu} (Dravidian), the coordination of
proper names is marked by the lengthening of their final vowels \citep[\page
111]{Drellishak:Bender:05}. This last example is illustrated in (\ref{telugu}), quoted from
\citew[\page 111]{Drellishak:Bender:05}.

\ea
\label{telugu}
\gll kamalaa wimalaa poDugu \\ 
     Kamala Vimala tall\\
\glt `Kamala and Vimala are tall.'
\z



Second, as \citet[165]{ross67} originally noted, the natural intonation break occurs before the
coordination lexeme, rather than between the coordinator and the coordinand, so that a  prosodic
constituent is formed. Although prosodic phrasing is not generally believed to always align with
syntactic phrasing, the fact that the coordinator prosodifies with the  coordinand suggests that the
former forms a unit with the latter. 

 Aspects of the phrase structure rule in (\ref{coordination:rule-several-x}) can be formalized in HPSG as
 shown in  (\ref{coordparam2}),  using parametric lists  \citep[\page 396, fn.\,2]{pollardsag} to enforce
 that all coordinands structure-share the morphosyntactic information. The type \type{ne-list} (\textit{non-empty-list}) corresponds
 to a list that has at least one member, and when used parametrically as in (\ref{coordparam2}), it additionally requires that
 every member of the list bear the features
 \avm{[synsem|loc|cat & \1]}.

\ea 
\avm{
\type{coord-phrase} \impl\\
	[\punk{synsem|cat}{\1} \\
	dtrs & <[synsem|loc|cat & \1 ]> \+ ne-list([synsem|loc|cat & \1 ]) ]
}\label{coordparam2}
\z

\noindent
The constraint forcing all daughters to be of the same category is excessive, as we shall see below,
and this will have to undergo a revision. Later in the chapter, we will see further proposals. For
now, we are focusing on standard coordinations. 

In order to  account for the fact that different kinds of coordination strategies are possible,
\citet[\page 260]{Mouret:06} and \citet[\page 205]{Bilbiie:17} define three subtypes of
\type{coord-phrase}, assuming a lexical feature \textsc{coord} to distinguish between   coordination
types:\footnote{Mouret's and Bı̂lbı̂ie's formulations are slightly different in that the relevant
  feature is instead called \textsc{conj}, and a slightly different type hierarchy is assumed, with
  negative constraints like  \textsc{conj} $\not=$ \type{nil} being employed instead of
  \textsc{coord} \type{crd}. The current formulation   avoids negative constraints, though nothing
  much hinges on this. Similar liberty is taken in subsequent constraints, for exposition purposes.

Strictly speaking tags that appear only once in a structure are illegitimate, since tags are about
sharing values. The purpose of the tags in (\mex{1}a) and (\mex{1}b) is to ensure that all members
in the list have the same \textsc{coord} value. A more precise way would add a constraint to
(\mex{1}a) and (\mex{1}b) saying that \ibox{1} = $\top$, $\top$ (top) being the most general type in the type
hierarchy. While this does not really add restrictive constraints on \ibox{1}, it makes sure that
all list members of the second list get the same \textsc{coord} value, since all elements of the
lists are [\textsc{coord} \ibox{1}] and since they are all shared with the \ibox{1} mentioned in
\ibox{1} = $\top$.}
 

\eal
\ex
\avm{
\emph{simple-coord-phrase} \impl \\
	[dtrs & ne-list(<[coord & none]>) \+ ne-list([coord & \1 crd ]) ]
}

\ex
\avm{
\emph{omnisyndetic-coord-phrase}  \impl \\
	[dtrs & ne-list([coord & \1 crd ]) ]
}\label{omni}
\ex
\avm{
\emph{asyndetic-coord-phrase}  \impl \\
	[dtrs & ne-list([coord & none ]) ]
}
\zl
% Stefan: How does this work with structure sharing.
% Rui: you could dummy bind [1] in 12a,b by adding a conjunction to (12a, b) stating [1] = T

\noindent
Here, we assume that the value of \textsc{coord} must be typed as \type{coord},
and that the latter has various sub-types as shown in Figure~\ref{fig:mlabelc}.
% \begin{figure}
%     \centering
%     \Tree[.\type{coord} \type{none} [.\type{crd} \type{and} \type{or} \type{but} {\ldots{}} ] ]
%     \caption{Coordinator sub-types}\label{fig:mlabelc}
% \end{figure}
\begin{figure}
    \centering
\begin{forest}
type hierarchy
[coord 
   [none]
   [crd 
     [and]
     [or]
     [but]
     [\ldots] ] ]
\end{forest}
    \caption{Coordinator sub-types}\label{fig:mlabelc}
\end{figure}
Thus, simple (monosyndenton and polysyndenton) coordinations are those where all but the first
coordinand are allowed to combine with a coordinator, omnisyndenton coordinations are those where
all coordinands have combined with a coordinator, and likewise, asyndenton coordinations are those
where none of the coordinands have combined with a coordinator. 


We turn to  the  analysis of coordinators. 
In other words, what exactly are words like \emph{and}, \emph{or}, 
and others, and how do they combine with coordinands?

\subsection{The status of coordinator expressions}


In HPSG, coordinators are sometimes analyzed as markers
\parencites[Section~4.1]{Beavers}[Section~4.1]{Drellishak:Bender:05}. In such a view, the
coordinator's lexical entry does not select any arguments, since it has none. In
(\ref{le-coord-lexeme-marker}), we show the lexical entry for the conjunction, using current HPSG
feature geometry. Note that the \textsc{mrkg} (marking) value of the coordinator is the same as the
coordinand's, which makes this marker a bit unusual in that it is transparent. Thus, if \emph{and}
coordinates S nodes that are \textsc{mrkg} \type{that} (i.e.\ CPs in the analysis of
  \citealt[Section~1.6]{ps2}), then the result will be an S that is also \textsc{mrkg} \type{that},
and so on, for any given value of \textsc{mrkg}.\footnote{The semantics and pragmatics of
  coordination  is a particularly complex topic which we cannot do justice to here, especially when
  it comes to interactions with other phenomena such as quantifier scope and collective,
  distributive, and reciprocal readings. See \crossrefchapterw{semantics} for more discussion and in
  particular \citet[Section~6.7]{mrs},  \citet{jfast}, Chaves
  (\citeyear[Chapters~4--6]{chavesthesis}; \citeyear[Section~5.3]{chavesextr};
  \citeyear{chavessubjexp}; \citeyear{Chaves:09}), and \citet[Chapters~4--5]{sangheepark} for HPSG
  work that specifically focuses on the semantics of coordination.} 

\ea
\avm{
 %\type{coordinator} \impl
[\type*{coord-lexeme}
 \phon < \type{and} > \\
 \punk{synsem|loc|cat}{[head & [\type*{coord}
                           sel & [loc|cat & [coord & none \\
                                             mrkg & \1 ] ] ] \\
                   coord  & and \\
             	   mrkg & \1 ]} ] 
}\label{le-coord-lexeme-marker}
\z




\noindent
This sign imposes constraints on the head sign it combines with via the feature \textsc{sel}(\textsc{ection}), the same feature that allows other markers and 
adjuncts in general to combine with their
hosts. The syntactic construction that allows such elements with their selected heads is the Head-Functor Construction in (\ref{rulem}).
Since the second daughter is the head, the value of the mother's \textsc{head} feature will have to be the same as the head daughter's, as per the
\isi{Head Feature Principle}.\footnote{The Head Feature Principle \citep[\page 34]{pollardsag} states that the value of
the mother's \textsc{head} feature is identical to that of the head daughter's \textsc{head}
feature. See also \crossrefchaptert[\page \pageref{page-hfp}]{properties}.}

\eas
\label{rulem}\label{head-functor-construction}
\avm{
\emph{head-functor-phrase} \impl

	[synsem|loc|cat [ subj  & \1 \\ 
                          comps & \2 \\
			  coord & \3 \\
			  mrkg  & \4 ] \\
	hd-dtr  \5 \\
	dtrs  <	[synsem|l|cat [ sel   & \6 \\
				coord & \3 \\ 
				mrkg  & \4 ] ],
		 \5[synsem & \6[l|cat & [subj  & \1 \\ 
                                         comps & \2 ] ] ] > ]
}
\zs



\noindent
Thus, the coordinator projects an NP when combined with an NP, an AP when combined with an AP, etc., as Figure~\ref{coordphr} illustrates.

\begin{figure}
\hfill
\begin{forest}
[{NP[\textsc{coord} \type{and}]}	
  [{C[\textsc{coord} \type{and}]} [and] ] 
  [N [Mary] ] ]
\end{forest}
\hfill
\begin{forest}
[{AP[\textsc{coord} \type{or}]}  
  [{C[\textsc{coord} \type{or}]} [or] ]
  [AP [tall] ] ]
\end{forest}
\hfill\mbox{}
\caption{Coordinate marking constructions}\label{coordphr}
%\itdopt{Stefan: Shouldn't the coordination be the same in both trees to make them minimal pairs? \Eg both \emph{and}?}
\end{figure}

\begin{sloppypar}
An alternative HPSG account that yields almost the same representation through different means is adopted by \citet{Abeille:03,Abeille:05}, \citet{Mouret:07}, \citet{Bilbiie:17}, and others. This approach
takes coordinators to be \emph{weak heads}, i.e.\ heads which inherit most of their syntactic properties from their complement,
like argument-marking prepositions do. Thus, the coordinator combines with coordinands via the same headed constructions that license non-coordinate structures.
It  preserves the \textsc{mrkg} feature when coordinands are themselves marked. The coordinator takes the adjacent coordinand as a complement. This captures its being first in head-initial languages like \ili{English}, and its final position in head-final languages like \ili{Japanese}.
\end{sloppypar}

\eal
\settowidth\jamwidth{(\ili{Japanese})}
\ex Lee [and Kim]\jambox{(\ili{English})}
% St. Mü. 15.07.2020 removed the final bracket from "Lee [and Kim]]"
\ex 
\gll Lee=to Kim\\
     Lee=and Kim\\\jambox{(\ili{Japanese})}
\glt `Lee and Kim'
\zl

\noindent
Since it is a weak head, it inherits most of  its syntactic features (\textsc{head}, \textsc{mrkg}) from its complement, and adds its own  \textsc{coord} feature. The lexical entry for the coordinator \emph{and} is shown in (\ref{lexcoordentry}).

\ea 
\avm{
[\type*{coord-lexeme}
 \phon  < and > \\
 \punk{synsem|loc|cat}{[ head  & \1\\
                    coord & and\\  
                    subj  & \2\\
                    comps & < [ loc|cat [head  & \1\\
                                         subj  & \2\\
                                         comps & \3\\
                                         mrkg  & \4 ] ] > \+ \3\\
                    mrkg  & \4 ]} ]
}\label{lexcoordentry}
\z

\noindent
The weak head analysis is illustrated in
Figure~\ref{coordphr2}. Here, the category of the coordinator, the coordinand, and the mother node are the same, because the coordinator's head value is lexically required
to be structure-shared with the head value of the coordinand it combines with (which is its first complement; see Section~\ref{lexcoord} on lexical coordination to see why the coordinator may inherit some complements expected by the coordinand).


\begin{figure}
\hfill
\begin{forest}
[{NP[\textsc{coord} \emph{and}]}	
  [{N$[$\textsc{coord} \emph{and}]}  [and] ] 
  [NP [Mary] ] ]
\end{forest}
\hfill
\begin{forest}
[{AP[\textsc{coord} \emph{or}]}  
  [{A[\textsc{coord} \emph{or}]}   [or] ]
  [AP [tall] ] ]
\end{forest}
\hfill\mbox{}
\caption{Coordinate weak-head constructions}\label{coordphr2}
%\itdopt{Stefan: \emph{and} instead of \emph{or}?}
\end{figure}



Before moving on, we note that the weak head analysis of coordinators makes certain problematic
predictions that the marker analysis in (\ref{le-coord-lexeme-marker}) does not make. Since
coordinands are selected as arguments in the former approach,  additional assumptions need to be
made in order to prevent the  extraction of coordinands as in (\ref{cc}). If coordinands are
arguments and hence listed in valence lists like \comps and \textsc{arg-st}, then they are
expected to be extractable (see \crossrefchapteralt[Section~\ref{sec:UDC:Middle}]{udc} and
\crossrefchapteralt[\page
\pageref{page-hpsg-traceless-account-arg-st-extraction-conjuncts}]{islands}). 

%, and subject to the o-command constraint that governs anaphora (see \crossrefchaptert{binding}).

\ea[*]{
\label{cc}
Which boy did you compare Robin and \trace?\\
 (cf.\ with \emph{which boy did you compare Robin with \trace?})
}
\z

%\begin{exe}
%\ex
%\begin{xlista}
%\ex  We invited [[Betsy's$_i$ mother] and [her$_i$]] to the ceremony.\\
%\citep{gazd1982}

%\ex I wish to inform [[him$_i$] and [all of Dr. Phil$_i$'s viewers] that real counseling
%sessions take place behind closed doors, not in TV.\\
%\citep{chavesthesis}

%\end{xlista}
%\end{exe}\label{bt}

\noindent
For this reason, the members of \textsc{arg-st} of the coordinator are typed as \type{canonical} by \citet[\page 17]{Abeille:03} to prevent their extraction, analogously to how prepositions in most languages must prevent their complements from being extracted, unlike \ili{English} and a few other languages.
See \citet[Section~3.2]{Abeille06} for a weak head analysis of certain \ili{French} prepositions.




\subsection{Correlative coordination}\label{correlphr}

Having discussed monosyndenton coordination structures, we now move on to cases where
multiple interdependent coordinators are present, such as correlative \emph{either \ldots{} or \ldots{}},
\emph{neither \ldots{} nor \ldots{}}, 
and \emph{both \ldots{} and \ldots{}}. See \citet{hof} for an account  in HPSG. Given the linearization flexibility of the first coordinator, it can be analyzed in \ili{English} as an adverbial rather than as a true coordinator:

\eal
\ex  Either Fred bought a cooking book or he bought a gardening magazine.
\ex  Fred either bought a cooking book or he bought a gardening magazine.
\ex  Fred can either buy a cooking book or he can buy a gardening magazine.
\zl
\itdsecond{Stefan: Just out of curiosity: What about \emph{Fred either bought a cooking book and he
    bought \ldots}. So the problem is that if \emph{either} is just an adjunct it has to be ensured
  somehow that there is an \emph{or}. How is this done? No need to change anything in the paper.}

\eal
\ex John will read both the introduction and the conclusion.
\ex John will both read the introduction and the conclusion.
\zl



\noindent
In \ili{French}, as in other \ili{Romance} languages, the coordinator itself can be reduplicated, and it is
obligatory for some coordinators (\emph{soit} `or' in \ili{French}) \parencites{Mouret:05}[\page 205--206]{Bilbiie:17}:

\eal
\ex[] {
\gll Jean lira et l'introduction et la conclusion.\\
     Jean read.\textsc{fut} and the.introduction and the conclusion\\
\glt `Jean read both the introduction and the conclusion.'
}
\ex[*] {
\gll Jean et lira l'introduction et la conclusion.\\
     Jean and read.\textsc{fut} the.introduction and the conclusion\\
}
\ex[] {
\gll Jean  lira soit l'introduction soit la conclusion.\\
    Jean read.\textsc{fut} or the.introduction or the conclusion\\
}
\ex[*] {
\gll Jean  lira l'introduction soit la conclusion.\\
    Jean read.\textsc{fut} the.introduction or the conclusion\\
}
\zl

\noindent
Thus, there are  different structures for different types of correlative, as Figure~\ref{f2} illustrates. The one on the left is for correlatives that exhibit adverbial properties and the one on the right is for correlatives that do not.
See \citet[\page 33--36]{Bilbiie:08} for arguments that both types are attested in \ili{Romanian}.



% \begin{figure}
% \tikzexternaldisable

%     \hfill
%     \Tree[.X [.{Adv} both ]  [.X X [.X [.{Coord} and ]  X ] ] ]
% \hfill
%     \Tree[.X [.X [.{Coord} et ]  X ] [.X [.{Coord} et ]  X ] ]
% \hfill\mbox{}
% \caption{Two possible structures for correlative coordination}\label{f2}
% \end{figure}
\begin{figure}
% here is a problem with externalization, trees are not aligned to the baseline 02.05.2020
%\tikzexternaldisable
    \hfill
\begin{forest}
%baseline
[X [Adv [both] ]  
   [X  
     [X] 
     [X 
       [Coord [and] ]  
       [X] ] ] ]
\end{forest}
\hfill
\begin{forest}
%baseline
[X 
  [X 
    [Coord [et] ]  
    [X] ] 
  [X [Coord [et] ]  
     [X] ] ]
\end{forest}
\hfill\mbox{}
\caption{Two possible structures for correlative coordination}\label{f2}
\itdsecond{Stefan: How is the correspondence between \emph{both} and \emph{and} established? Is the
  \textsc{coord} value projected? (\ref{ex-x-cord-minus-cord-plus}) says it is not.}
\end{figure}


The correlative coordinate structure on the right is covered by (\ref{omni}), since it requires the \coord feature to be the same for all coordinands. 
%See  \S\ref{correlphr} for more discussion of correlative constructions in a broader context, beyond coordinate correlatives.


\subsection{Comparative correlatives}
\label{coord:sec-comparative-correlatives}


When\is{comparative correlative|(} there is no overt coordinator, it is not always clear whether a binary clause construction is coordinate or not. Comparative correlatives such as (\ref{cc0}) have been analyzed as coordinate by \citet{culijack} for \ili{English} (in syntax, though not in semantics) and as universally subordinate  by \citet{dikken}. 
% refs ok

\ea
The more I read, the more I understand. \label{cc0}
\z

On the semantic side, the interpretation is something like: `if I read more, I understand
more'. \citet{Abeille:06} and \citet{Abeille:Borsley:08} propose that they are  coordinate in some languages 
 and subordinate in others. In \ili{English}, one can add the adverb \emph{then}, whereas in \ili{French}, one can add the coordinand \emph{et} (`and'). In \ili{English}, the first clause can also be used as a standard adjunct (\ref{adjcc}).
 
\eal
\label{adjcc}
\ex The more I read, (then) the more I understand.
\ex 
\gll Plus je lis (et) plus je comprends.\\
     more I read \hphantom{(}and more I understand\\\hfill(French)
\glt `If I read more, I understand more.'
\ex I understand more, the more I read.
\zl


As shown by \citet[549--550]{culijack}, the second clause shows matrix clause properties, not the first one:

\eal
\ex[]	{The more we eat, the angrier you get, don't you?}
\ex[*] {The more we eat, the angrier you get, don't we?}
\zl

Syntactic parallelism seems to be stricter in \ili{French}; for example, clitic inversion or extraction
must take place out of both clauses at the same time \citep[\page 1152]{Abeille:Borsley:08}:

\eal
\ex 
\gll Paul a     peu  de temps: aussi plus  vite commencera-t-il,  plus   vite  aura-t-il  fini.\\
     Paul has little of time so more fast start.\textsc{fut}-he more fast  \textsc{aux}.\textsc{fut}-he finish.\textsc{ptcp} \\\hfill{(\ili{French})}
\glt `Paul has little time left: so the faster he starts, the faster he will finish.'
\ex 
\gll C'   est un livre  que      plus   tu    lis, plus  tu    appr\'{e}cies. \\
     this is    a  book \textsc{comp} more you read.2\textsc{sg}  more you appreciate.2\textsc{sg} \\
\glt `This is a book that the more you read the more you like.'
\zl

In \ili{Spanish}, comparative correlatives come in two varieties as the following examples by \citet[\page 7]{Abeille:Borsley:Espinal:06} show: one that can be analyzed as subordinate as in (\ref{ex-spanish-a}), and one that can be analyzed as coordinate, as in (\ref{ex-spanish-b}).

\eal
\label{spanishab}
\ex 
\label{ex-spanish-a}
\gll Cuanto   m\'{a}s leo,     (tanto)        m\'{a}s entiendo. \\
     how.much more    read.1\textsc{sg} \hphantom{(}that.much more understand.1\textsc{sg} \\\hfill{(\ili{Spanish})}
\glt `The more I read, the more I understand.'
\ex 
\label{ex-spanish-b}
\gll	M\'{a}s leo        (y) m\'{a}s entiendo.\\
	more read.1\textsc{sg} \hphantom{(}and more understand.1\textsc{sg} \\
\glt `The more I read, the more I understand.'\\ 
\zl

Be they coordinate or subordinate, comparative correlatives are special kinds of construction: they
are binary, with a fixed order (the meaning changes if the order is reversed as in
(\ref{ex-the-more-a})). The internal structure of each clause is also special. In \ili{English}, it
must start with \emph{the} and a comparative phrase, as the oddness of (\ref{ex-the-more-b}) shows,
and may involve a long distance dependency (\ref{ex-the-more-c}). Each clause must be finite and
allow for copula omission, as shown in (\ref{ex-the-more-d}).

\eal
\label{intell}
\ex[]{
\label{ex-the-more-a}
The more I understand, the more I read.}
\ex[*]{
\label{ex-the-more-b}
I understand (the) more, I read (the) more.}
\ex[]{
\label{ex-the-more-c}
The more I manage to read, the more I start to understand.}
\ex[]{
\label{ex-the-more-d}
The more intelligent the students, the better the marks.}
\zl

These \emph{the}-clauses  are a special subtype of finite clause, starting with a comparative
phrase. \citet[\page 19]{Abeille:Borsley:Espinal:06} and \citet[\page 14]{Borsley:11} 
define a \textsc{correl} feature which is a \textsc{left edge} feature (see the \textsc{edge}
feature in \citealt{Bonami:2004} for \ili{French} \isi{liaison}). Assuming a degree word \textit{the}, which can only appear as a specifier of
a comparative word, \citet[\page 13]{Borsley:11}  defines the \textit{the}-clause as a subtype of \type{head-filler-phrase} with [\textsc{correl} \type{the}]; see also \citet[\page 527]{fgsag08}.

Comparative correlatives belong to a more general class of (binary) correlative constructions, including \emph{as \ldots{} so \ldots{}},
and \emph{if \ldots{} then  \ldots{}} constructions  
\parencites[Section~3.2]{Borsley:04}[\page 17--18]{Borsley:11}.\footnote{This does not handle \ili{Hindi} type correlatives, which differ in that  only the first clause is introduced by a correlative word, and the first clause is mobile and optional; see \citet[228]{pollardsag} for an analysis.}
Correlative constructions can be defined as follows, 
where \type{correl-construction} is a sub-type of 
\type{declarative-clause} and the feature \textsc{correl} introduces a \type{correl} type
hierarchy analogous to that of \type{coord} in Figure~\ref{fig:mlabelc} above.
The construction in (\ref{correlphrr}) thus states that all correlative
constructions have in common the fact that both daughters are marked by a special expression. 

\ea
\label{correlphrr}
\type{correl-phrase} \impl\\
\oneline{%
\avm{
[synsem|loc|cat|correl  \type{none}\\
 dtrs < [synsem|loc|cat|correl \type{corr-mrk} ],~~~\vspace{1mm}\\
        [synsem|loc|cat|correl \type{corr-mrk} ] > ]
}}
\z

Naturally, \type{correl-construction} has various sub-types, each imposing particular patterns of correlative marking, including coordinate correlatives. More specifically,  this family of constructions  comes in two varieties: asymmetric (for the subordinate ones, like \ili{English} comparative correlatives), and symmetric (for coordinate ones, like \ili{French} comparative correlatives). The symmetric subtype inherits from \type{clausal-coordination-phrase}, while the asymmetric one inherits from the \type{head-adjunct-phrase}, as seen in Figure~\ref{figcorr}.

\begin{figure}
\centering
{\small 
\begin{forest}
[\type{construction}
  [\type{causality}
    [{\ldots{}}]
    [\type{declar-clause}
      [{\ldots{}}] 
      [\type{correl-phrase}, name=correl
        [{\ldots{}}]
        [\type{symmetric-correl-phrase}, name = sym ] ] ] ]
  [\type{headedness}, 
    [\type{non-headed-phrase} 
        [{\ldots{}}]
        [\type{coord-phrase}, name=coord]  ]
    [\type{headed-phrase}
        [{\ldots{}}]
        [\type{head-adj-phrase}, name=adj
          [{\ldots{}}]
          [\type{asymmetric-correl-phrase}, name = asym ] ] ]    
        ] ] 
\draw  (coord.south) --(sym.north)
       (correl.south) --(asym.north);
\end{forest}}

\caption{Type hierarchy for correlative constructions}\label{figcorr}
\end{figure}


Thus,  asymmetric \ili{English} comparative correlatives  can be defined as
in (\ref{ecc}), where \type{the} is a sub-type of \type{corr-mrk} (i.e.\ is a correlative marker).

\ea
\label{ecc}
\type{asymmetric-correl-phrase} \impl\\ %\type{assymetric-correl-cx} \& 
\oneline{%
\avm{
[hd-dtr \1\\
 dtrs < [synsem|loc|cat|correl & the ],
                   \1[ synsem|loc|cat|correl & the ] > ]
}
}
\z

\noindent
Similarly,  symmetric \ili{French} comparative correlatives can be  defined as in (\ref{ccf}), where
both clauses are coordinated (the second one may be introduced by \emph{et} or without a conjunction) and
introduced by a comparative correlative marker (\emph{plus} `more', \emph{moins} `less',
\emph{mieux} `better'). 

\ea
\label{ccf}
\type{symmetric-correl-phrase} \impl\\ %\type{symmetric-correl-cx} \& 
\oneline{%
\avm{
[dtrs < [synsem|loc|cat|correl & compar ], 
        [synsem|loc|cat [ coord  & \type{none} $\vee$ \type{et}\\
                          correl & \type{compar} ] ] > ]
}}
\z

A more complete analysis would take into account the semantics as well \citep[Section~5.5]{fgsag08}. From a syntactic point of view, HPSG seems to be in a good position to handle both the general properties and the idiosyncrasy of the comparative correlative construction, as well as its crosslinguistic variation. 
For an analysis of a number of \ili{Arabic} correlative constructions see \citet{Alqurashi:Borsley:14}.
See also  \citet{Borsley:11} for a comparison with a tentative Minimalist\is{Minimalism} analysis.%
\is{comparative correlative|)}

\section{Phrasal coordination and feature resolution}

\subsection{Feature sharing between coordinands}

The coordination construction in (\ref{coordparam2}) requires the value of \textsc{cat} to be structure-shared across the coordinands and the mother node. Given the large number of features within \textsc{cat}, such a constraint makes a series of predictions and mispredictions.
For example, this entails that all valence constraints are identical. Thus, in VP coordination, all nodes have an empty \textsc{comps} list and share exactly the same singleton \textsc{subj} list, as illustrated in Figure~\ref{valenceif}. Thus, nothing needs to be said from the semantic composition side: the verbs will have to share exactly the same referent for their subject. The same goes for any other combination of categories of whatever part of speech.

\begin{figure}
\begin{forest}
sm edges
[%
\avm{
	S [subj & < > \\
	comps & < > ]
}
	[\ibox{1}NP
		[Sam, roof]
	]
	[%
	\avm{
		VP [subj  & < \1 > \\
		    comps & < > ]
	} 
    	[%
    	\avm{
    		VP [subj  & < \1 > \\
                    comps & < > ]
		}
			[ate cheese pizza, roof]
		]
    	[%
    	\avm{
			VP [subj  & < \1 > \\
			    comps & < > ]
		}
			[and drank soda, roof]
		]
	]
]
\end{forest}
\caption{Valence identity in  coordination}\label{valenceif}
%\itd{Stefan: Actually the \comps lists are not identical in the figure. I am not even sure that the \subj values are, since it is the \cat values that are identified and it only follows that \subj has the same value.}
\end{figure}

All the unsaturated valence arguments become one and the same for all coordinands, and it becomes impossible to have daughters with different subcategorization information. For example, if one daughter requires a complement while the other does not,
\textsc{cat} identity  is impossible. This correctly rules out  a coordination of  VP and V categories
like the one in (\ref{vpbad1}), or S and VP as in (\ref{vpbad2}):

\eal
\ex[*]{Fred [read a book]\sub{\comps $\langle \rangle$}  and [opened]\sub{\comps $\langle$ NP $\rangle$}.}\label{vpbad1}
\ex[*]{Fred [she has a hat]\sub{\subj $\langle \rangle$} and [smiled]\sub{\subj $\langle$ NP $\rangle$}.}\label{vpbad2}
\zl

\noindent
But there is other information in \textsc{cat} besides valence. For example, the head feature
\textsc{vform} encodes the verb form, and the coordination of inconsistent \textsc{vform} values is
ruled out as ungrammatical as seen in (\ref{vform}), while consistent values of \textsc{vform} are
accepted as illustrated by (\ref{vform2}).\footnote{%
That said, some cases are more acceptable, such as (i):
\ea
I expect [to be there]\sub{\vform \type{inf}} and [that you will be there too]\sub{\vform \type{fin}}.
\z
See Section~\ref{unlikessec} for more discussion about such cases.}


\eal
\label{vform}
\ex[*]{Tom [whistled]\sub{\vform \type{fin}}      and [singing]\sub{\vform \type{prp}}.}
\ex[*]{Sue [buy something]\sub{\vform \type{inf}} and [came home]\sub{\vform \type{fin}}.}
\zl



\eal
\label{vform2}
\ex Tom [is married]\sub{\vform \type{fin}} and [just bought  a house]\sub{\vform \type{fin}}.
\ex Sue [buys groceries here]\sub{\vform \type{fin}} and [could be interested in working with us]\sub{\vform \type{fin}}.
\ex Dan [protested for two years]\sub{\vform \type{fin}} and [will keep protesting]\sub{\vform \type{fin}}.
\zl

Yet another feature that resides in the \textsc{cat} value of verbal expressions is the head feature
\textsc{inv}, which indicates whether a given verbal expression is invertable or not. Hence,
inverted structures cannot be coordinated with non-inverted ones: 


\eal
\ex[] {[Sue has sung in public]\sub{\inv $-$} and [Kim has tap-danced]\sub{\inv $-$}.}
\ex[*] {[Sue has sung in public]\sub{\inv $-$} and [has Kim tap-danced]\sub{\inv $+$}.}
\zl

\eal
\ex[] {[Elvis is alive]\sub{\inv $-$} and [there was a CIA conspiracy]\sub{\inv $-$}.}
\ex[*] {[Elvis is alive]\sub{\inv $-$} and [was there a CIA conspiracy]\sub{\inv $+$}.}
\zl

\noindent
But if the inverted clause precedes the non-inverted one, then such coordinations become somewhat
more acceptable. In fact, \citet[1332--1333]{rodney} note attested cases like (\ref{tax}).

\ea
Did you make your own contributions to a complying superannuation fund and
your assessable income is less than \$31,000?\label{tax}\label{ex-did-you-make-and-your-asessable-income-is}
\z

\noindent
A similar problem arises for the feature \textsc{aux}, which distinguishes auxiliary verbal expressions from those that
are not auxiliary:

\eal
\label{aux}
\ex {}[I stayed home]\sub{\aux $-$} but [Fred could have gone fishing]\sub{\aux $+$}.
\ex {}[Tom went to NY yesterday]\sub{\aux $-$} and [he will return next Tuesday]\sub{\aux $+$}.
\ex Fred [sang well]\sub{\aux $-$} and [will keep on singing]\sub{\aux $+$}.
\zl

\noindent
However, this problem vanishes in the account of the \ili{English} Auxiliary System detailed in \citet{SagEtAl20}, since in that analysis,
the feature \textsc{aux} does not indicate whether the verb is auxiliary or not. Rather, the value of \textsc{aux} for auxiliary verbs is resolved by the construction in which the verb is used. Since all the constructions in (\ref{aux}) are canonical VPs (i.e.\ non-inverted), then 
all the coordinands in (\ref{aux}) are specified as \textsc{aux--} in
the \citet{SagEtAl20} analysis.




Similarly, argument-marking PPs cannot be coordinated with modifying PPs simply because the former are specified with different  \textsc{pform} and \textsc{select} values. This explains the contrast
in (\ref{pp}). The first PP is the complement that \emph{rely}
selects but the second is a modifier. Thus, they have different \textsc{cat} values 
and cannot be coordinated.


\eal
\label{pp}
\ex[] {Kim relied on Mia on Sunday.}
\ex[*] {Kim relied on Mia and on Sunday.}
\zl

\noindent
Consequently, it is in general not possible to coordinate argument marking PPs headed by different prepositions, simply because they bear
different \textsc{pform} values, as shown in (\ref{pp2}).

\eal
\label{pp2}
\ex[*] {Kim depends  [[on Sandy]\sub{\pform \type{on}}
                       or [to Fred]\sub{\pform \type{to}}].}

\ex[*] {Kim is afraid  [[of Sandy]\sub{\pform \type{of}}
                       and [to Fred]\sub{\pform \type{to}}].}
\zl

Similarly, adjectives that are specified as \textsc{pred}$+$ cannot be
coordinated with  \textsc{pred}$-$ adjectives, without stipulation:

\eal
\ex[*]{I became [former]\sub{\prd$-$} and [happy]\sub{\prd$+$}.}
\ex[*]{He is  [happy]\sub{\prd$+$} and [Fred]\sub{\prd$-$}.}
\ex[*]{[Mere]\sub{\prd$-$} and [happy]\sub{\prd$+$}, Fred rode on into the sunset.}
\zl


\noindent
Since case information is also part of \textsc{cat},  the theory
predicts that coordinands must be consistent, which is borne out by the
facts, as the unacceptability of  (\ref{ex-case-match-coordination}) shows.\footnote{There are nonetheless collocational cases where the distribution of pronouns defies this pattern, due to presumably prescriptive forces \citep{grano}. See also \citet[105, 107]{binomial} for a broader multifactorial study of binomial expressions in which syllable length and  frequency have a major effect in predicting nominal coordinand order, among other things.}
 Many other examples of \textsc{cat} mismatches exist, but the  list above suffices to
illustrate the breadth of predictions that follow from the feature geometry of \textsc{cat} and the constraints imposed by
the coordination construction.

\eal
\label{ex-case-match-coordination}
\ex[*] {I saw [her$_{acc}$ and he$_{nom}$].}
\ex[*] {He likes [she$_{nom}$ and me$_{acc}$].}
\zl


% (Boolean, non-Boolean, etc.  agreement resolution, 1st conjunct agreement)


Mispredictions also exist. We already discussed the example in (\ref{ex-did-you-make-and-your-asessable-income-is}), concerning the feature \textsc{inv}, but there are others. For example, requiring that the \slashv of the coordinands be the same readily predicts \isi{Coordinate Structure Constraint} effects like 
(\ref{cs1}), but it incorrectly rules out asymmetric coordination violation cases like (\ref{assym}). 
See \citet{goldsmith}, \citet{lakoff86}, \citet{levinprince86}, and \citet{kehler} for more examples and discussion.


\eal
\label{cs1}
\ex[] {[To him]\sub{\ibox{1}PP}  [[Fred gave a football \trace]$_{\textup{\textsc{slash}} \langle \ibox{1} \rangle}$ and
[Kim gave a book \trace]$_{\textup{\textsc{slash}} \langle \ibox{1} \rangle}$].}

\ex[*] {[To him]\sub{\ibox{1}PP} [[Fred gave a football \trace]$_{\textup{\textsc{slash}} \langle \ibox{1} \rangle}$ and
[Kim gave me a book]$_{\textup{\textsc{slash}} \langle \, \rangle}$].}

\ex[*] {[To him]\sub{\ibox{1}PP} [[Fred gave a football to me]$_{\textup{\textsc{slash}} \langle \, \rangle}$ and
[Kim gave a book \trace]$_{\textup{\textsc{slash}} \langle \ibox{1} \rangle}$].}
\zl

%% \inlinetodostefan{The gap is bound off within the relative clause. In PS94 and in
%%   \citet[\page 456]{Sag97a}. Please fix this. (\mex{1}b) is not ruled out because of non-matching
%%   \slashvs but because the relative clause is internally not well-formed: a head-filler phrase needs
%% a gap and \emph{that every kid wants it} does not have a gap and hence the relative pronoun cannot
%% be integrated. I guess you should just remove the example in (\mex{1}).}

%% \begin{exe}
%% \ex \begin{xlista}
%% \ex[] {It offers something$_{i}$  [that every kid wants \trace]$_{\textup{\textsc{slash}} \langle \textup{\textsc{np}}_i \rangle}$ and
%% [that every parent tries to help their child to achieve \trace]$_{\textup{\textsc{slash}} \langle \textup{\textsc{np}}_i \rangle}$.}

%% \ex[*] {It offers something$_i$ [that every kid wants \trace]$_{\textup{\textsc{slash}} \langle 
%% \textup{\textsc{np}}_i \rangle}$ and
%% [that every parent tries to help their child to achieve it]$_{\textup{\textsc{slash}} \langle \, \rangle}$} 

%% \ex[*] {It offers something$_i$  [that every kid wants it]$_{\textup{\textsc{slash}} \langle \, \rangle}$ and
%% [that every parent tries to help their child to achieve \trace]$_{\textup{\textsc{slash}} \langle
%% \textup{\textsc{np}}_i \rangle}$}
%% \end{xlista}\label{cs2}
%% \end{exe}


\eal
\label{assym}
\ex {}[Who]\sub{\ibox{1}NP} did Sam [pick up the phone$_{\textup{\textsc{slash}} \langle \, \rangle}$ and call \trace$_{\textup{\textsc{slash}} \langle \ibox{1} \rangle}$]?
\ex What was the maximum amount$_i$ that
I can [contribute \trace$_{\textup{\textsc{slash}} \langle NP_i \rangle}$ and still get a tax deduction$_{\textup{\textsc{slash}} \langle \, \rangle}$]?
\zl


\citet{chavesextr} argues that there are no independent grounds to assume that asymmetric
coordination is anything other than coordination, and therefore the coordination construction must
not impose \textsc{slash} identity across coordinands (\gap identity in his version of the
theory). Rather, the Coordinate Structure Constraint and its asymmetric exceptions are best analyzed
as pragmatic in nature, as \citet[Chapter~5]{kehler} argues.  See
\crossrefchapterw[Section~\ref{sec:UDC:MoreOnGaps}]{udc} for more discussion.  In practice, this
means that the coordination construction should impose identity of some of the features in
\textsc{cat}, though not all, despite the fact that one of the prime motivations for \textsc{cat}
was coordination phenomena.

Like in the case of locally specified valents, the category of the extracted phrase is also
structure-shared in coordination. Hence, case mismatches like (\ref{gapcase}) are correctly ruled
out.


\ea[*]{
\label{gapcase}
{}[Him]\sub{NP[\type{acc}]},  [all the critics like to praise \trace]$_{\textup{\textsc{slash}} \langle \textup{\scriptsize NP}[\type{acc}] \rangle}$
but [I think \trace would probably not be present at the awards]$_{\textup{\textsc{slash}} \langle \textup{\scriptsize NP}[\type{nom}] \rangle}$.}
\z

\noindent
There\label{coordination:page-case-syncretism-start} are, however, cases where the case of the
ATB-extracted phrase can be syncretic \citep{anderson83}. This is illustrated in (\ref{syn}) using
examples by \citet[205]{levineetal} and \citet[75]{goodall87}, respectively.

\eal
\label{syn}
\ex{Robin is someone who$_i$ even [good friends of \trace$_i$] believe \trace$_i$ should be closely watched.}

\ex{We went to see a movie  which$_i$ [the critics praised \trace$_i$ but that Fred said \trace$_i$ would probably be too violent for my taste].}

\zl

The feature \textsc{case} is responsible for identifying the case of nominal expressions.  Pronouns
like \emph{him} are specified as \emph{acc}(\emph{usative}), and pronouns like \emph{I} are
\emph{nom}(\emph{inative}), and expressions like \emph{who} or \emph{Robin} are left underspecified
for case.  According to \citet[207]{levineetal}, the case system of \ili{English} involves the
hierarchy in Figure~\ref{qwsa}.


\begin{figure}
\centering

\begin{forest}
type hierarchy
[scase 
   [snom
      [nom]
        [nom\_acc ]] 
   [sacc
      [,identify=!P]
      [acc ]]]
\end{forest}


\caption{Type hierarchy of (structural) case assignments}\label{qwsa}
\end{figure}


%\pagebreak
Finite verbs assign structural nominative (\type{snom}) to their subjects and structural accusative
(\type{sacc}) to their objects. Most nouns and some pronouns like \emph{who} and \emph{what} are
underspecified for case, and thus typed as \emph{scase}, which makes them consistent with both
nominative and accusative positions. Hence, \emph{a movie} can simultaneously be required to be
consistent with \emph{snom} and \emph{sacc} by resolving into the syncretic type \emph{nom\_acc},
which is a subtype of both \emph{snom} and \emph{sacc}. Pronouns like \emph{him} and \emph{her} are
specified as \type{acc} and therefore are not compatible with the \type{nom\_acc} type. The same
goes for \type{nom} pronouns like \emph{he} and \emph{she}, etc.  Hence, the problem of case
syncretism is easily solved.  See Section~\ref{unlikessec} for more discussion about the related
phenomenon of coordination of unlike categories.\label{coordination:page-case-syncretism-end}


\subsection{Coordination and agreement}
\label{coordination:sec-agreement}

Another thorny issue for syntactic theory and coordination structures concerns agreement. According to 
\citet[Section~2.4.2]{pollardsag}, agreement information is introduced by the \textsc{index} feature in semantics, not morphosyntax. Hence, different expressions
with inconsistent person, gender, and number specifications are free to combine. But \citet[Chapter~2]{wechsler} have also argued that there should be a distinct feature called \textsc{concord}, which is morphosyntactic in nature (see \crossrefchapteralt[Section~\ref{agreement-sec-index-concord}]{agreement}). The motivation for this move is that there are languages, like \ili{Serbo-Croatian}, 
which display hybrid agreement:

\ea
\gll Ta dobra deca su do\v{s}-l-a.\footnotemark\\
         that.\SG.\F{} good.\SG.\F{} children \textsc{aux}.3\PL{} come-\textsc{ptcp}-\N.\PL\\\hfill(Serbo-Croatian)
\footnotetext{
\citew[51]{wechsler}
}
\glt `Those good children came.'
\z

\noindent
The collective noun \emph{deca} `children' triggers feminine singular (morphosyntactic) agreement on
NP-internal items, in this case the determiner \emph{ta} `that' and the adjective \emph{dobra}
`good'.  There are  HPSG
analyses that argue that what appears to be Closest Conjunct Agreement (see
Section~\ref{coordination:sec-cca} below) is in fact agreement with
the whole coordinate NP, which has additional features inherited from the first and last
coordinands. \citet[Section~5]{Villavicencio:Sadler:ea:05} propose two additional features:
\textsc{lagr} (for the left-most coordinand) and \textsc{ragr} (for the right-most coordinand) for
determiner and (attributive) adjective agreement in \ili{Romance}, which involves the
\textsc{concord} feature.  Semantic agreement  on the other hand, is seen in the
verb \emph{su}, which is inflected for third person plural, in agreement with the semantic
properties of the subject \emph{deca}. The two kinds of agreement are also visible in \ili{English}:

\eal
\ex This/*These committee made a decision.
\ex The committee have/has made a decision.
\zl


\noindent
The resolution of agreement information in coordination is not a trivial matter of matching the conjunct's agreement information. There are usually complex constraints involved in determining what the agreement of the mother node is, given that of the coordinands. We turn to this problem below.






\subsection{Agreement strategies with coordinate phrases}
\label{coordination:sec-agreement-with-coordinate-phrases}

In case of coordinands with conflicting agreement values, various resolution strategies are observed 
crosslinguistically. For example, a coordination with a first person is first person, and a coordination with second person (and no first person) is second person:

\eal
\ex Paul and I like ourselves / * themselves.
\ex Paul and you like yourselves / * themselves.
\zl

In gender-marking languages, coordination with conflicting gender values is often resolved to 
masculine, at least for animates \citep[186]{Corbet91}. This is illustrated in (\ref{pt}) for \ili{Portuguese} taken from or based on examples by \citet*{Villavicencio:Sadler:ea:05}.
\eal
\label{pt}
\ex 
\gll o homem e a mulher modernos\footnotemark\\
     the.\textsc{m.sg} man.\textsc{m.sg} and the.\textsc{f.sg} woman.\textsc{f.sg} modern.\textsc{m.pl} \\
\footnotetext{
\citew[433]{Villavicencio:Sadler:ea:05}}
\glt `the modern man and woman'
\ex 
\gll morbidez e morte prematuras\footnotemark\\
     morbidity.\textsc{f.sg} and death.\textsc{f.sg} premature.\textsc{f.pl}\\
\footnotetext{
See \citew[434]{Villavicencio:Sadler:ea:05} for similar examples.}
\glt `premature morbidity and death'\\
\zl


\citet[\page 281]{Sag:03} proposes that first person is a subtype of second person, which is itself a subtype of third person. This way, person resolution in coordination amounts to type unification. Addressing gender resolution, \citet{Aguila:Crysmann:18} propose a list-based encoding of person and gender values, and list concatenation as a combining operation, as shown in (\ref{aguila}). For gender, they propose a \textsc{m}(\textsc{asculine}) feature that has an empty list value for feminine words, and a non-empty list value for masculine words.  The coordination of a masculine noun (\emph{chevaux} `horses') with a feminine noun (\emph{\^{a}nesses} `female donkey')  yields a masculine NP with a non-empty list value for \textsc{m}. Only the coordination of two feminine nouns yields a feminine NP with an empty list value \textsc{m}.
 
\eas
\label{aguila}
\type{nom-coord-phrase} \impl\\
\avm{
[\punk{synsem|loc|cont|index}{[num & pl \\
	                  gend & [m & \1 \+ \2 ] \\
                          per & [me  & \3 \+ \5\\
                              you & \4 \+ \6 ] ]} \\
 dtrs &	< [synsem|loc|cont|index & [gend & [m & \1] \\
                       	            per & [me  & \3 \\
                                           you & \4] ] ],\\
	  [synsem|loc|cont|index & [gend & [m & \2] \\
                                    per & [me  & \5 \\
                                           you & \6] ] ]  >
	]
}
\zs
 
 \noindent
For person agreement, they use two list valued features \textsc{me} and \textsc{you}. A first person has a non-empty \textsc{me} list, second person has an empty \textsc{me} list and a non-empty \textsc{you} list, and third person has both empty lists.  Thus, coordinating a first with a third person yields a \textsc{me} feature with a non-empty list, and a \textsc{you} feature with a non-empty list, hence a first person phrase. Coordinating a third person with a second person yields a non-empty \textsc{you} list  and an empty \textsc{me} list, hence a second person phrase. This enables person and gender resolution by list concatenation over
coordinands. 

\subsubsection{Closest Conjunct Agreement}
\label{coordination:sec-cca}

As observed by \citet[186]{Corbet91}, many languages, including \ili{Romance}, \ili{Celtic},
\ili{Semitic}, and \ili{Bantu} languages, also have another strategy: partial agreement with only
one coordinand, the one closest to the target, called \emph{Closest Conjunct Agreement}
(CCA)\is{agreement!closest conjunct}. In the following examples, again from \ili{Portuguese} and
taken from \citew{Villavicencio:Sadler:ea:05}, the determiner and prenominal adjective agree with
the first noun (\ref{fo}a) and the postnominal adjective with the last noun (\ref{fo}b).

\eal
\label{fo}
\ex 
\gll suas pr\'{o}prias rea\c{c}\~{o}es ou julgamentos\footnotemark\\
     his.\textsc{f.pl} own.\textsc{f.pl} reactions.\textsc{f.pl} or judgements.\textsc{m.pl} \\
\footnotetext{
\citew[435]{Villavicencio:Sadler:ea:05}  
}
\glt `his own reactions or judgements'
\ex 
\gll Esta canc\~{a}o anima os cora\c{c}\~{o}es e mentes brasileiras.\footnotemark\\
     this.\textsc{f.sg}  song.\textsc{f.sg} animates the.\textsc{m.pl} hearts.\textsc{m.pl} and minds.\textsc{f.pl} Brazilian.\textsc{f.pl} \\
\footnotetext{
\citew[437]{Villavicencio:Sadler:ea:05} 
}
\glt `This song animates Brazilian hearts and minds.'
\zl

For \ili{French} determiners and attributive adjectives, \citet{An:Abeille:17} and \citet{Abeille:An:Shiraishi:18} show on the basis of corpus data and experiments that number agreement may also obey CCA. As far as gender is concerned, prenominal adjectives always obey CCA, while postnominal ones do so half of the time (in contemporary \ili{French}). In (\ref{ft}a), the determiner can be singular (CCA) or plural (resolution), while in (\ref{ft}b), CCA (feminine Det) is obligatory. In (\ref{ft}c), the postnominal adjective can be masculine (resolution) or feminine (CCA), with the same meaning.

%\inlinetodostefan{no source for (\ref{ft}c) answer: no need to put a source for every example). State sources in main text or provide source below (\ref{ft}c).}
\eal
\label{ft}
\ex  
\gll votre / vos nom et pr\'{e}nom\footnotemark\\
     your.\SG{} {} you.\PL{} name.\textsc{m.sg} and first.name.\textsc{m.sg} \\
\footnotetext{
 \citew[\page 34]{An:Abeille:17}
}
\glt `your name and first name'

\ex 
\gll certaines          / *   certains collectivités et organismes publics\footnotemark\\
     certain.\F.\PL{} {} {} certain.\MAS.\PL{} collectivity.\F.\PL{} and organism.\MAS.\PL{} public.\MAS.\PL{} \\
\footnotetext{
\citew[\page 17]{Abeille:An:Shiraishi:18}
}
\glt `certain public collectivities and organisms' 

\ex 
\gll des d\'{e}partements et r\'{e}gions importants / importantes\\
     some department.\textsc{m.pl} and region.\textsc{f.pl} important.\textsc{m.pl} {} important.\textsc{f.pl}\\
\glt `some important departments and regions'
\zl


As proposed by \citet[Chapter~2]{wechsler}, HPSG distinguishes two agreement features: \textsc{concord} is used for
morphosyntactic agreement and \textsc{index} is used for semantic agreement (see
\crossrefchapteralt[Section~\ref{agreement-sec-index-concord}]{agreement}). \citet{Moosally} proposes an account
of single coordinand predicate-argument agreement in \ili{Ndebele}, which she analyses as  \textsc{index} agreement. She has  a version of the following 
constraint that shares the \textsc{index} value of the (nominal) coordinate mother with that of the
last coordinand (p.\,389):

\ea
\avm{
\type{nom-coord-phrase} \impl
	[\punk{synsem|loc|cont|index}{ \1} \\
	 dtrs & < [ ], \ldots, [synsem|loc|cont|index & \1 ] > ]
}
\z


But in other languages, such as \ili{Welsh}, there is evidence that the \textsc{index} of the coordinate
structure is resolved, even though predicate-argument agreement is controlled by the closest coordinand: 

\ea 
\gll Dw              i a   Gwenllian              heb     gael ein                      talu.\footnotemark\\
     be.1\textsc{sg} I and Gwenllian.3\textsc{sg} without get  \textsc{cl}.1\textsc{pl} pay \\
\footnotetext{
 \citew[\page 90]{Sadler2003a-u}
}
\glt  `Gwenllian and I have not been paid.'
\z

\noindent
This is why \citet{Borsley:2009} proposes that CCA is superficial in \ili{Welsh} and uses linearization domains\is{linearization}\label{page-linearization-domains-in-coordination-one}\footnote{%
  Order domains were introduced into HPSG by \citet{Reape94a}; for more on order domains see
  \crossrefchaptert[Section~\ref{sec-domains}]{order}.
} to handle partial agreement between the initial verb and the first coordinand, which are not sisters.
The hypothesis  was that verb-subject agreement involves order domains and coordinate structures are
not represented in order domains. This allows what looks like agreement with a closest coordinand to
be just that. See also \crossrefchapterw[Section~\ref{agreement:sec-superficial-agreement}]{agreement}. The alternative developed by \citet{Villavicencio:Sadler:ea:05} assumes that
coordinate structures have features reflecting the agreement properties of their first and last
coordinands, to which agreement constraints may refer. As mentioned above,
\citet{Villavicencio:Sadler:ea:05} use three features: \textsc{concord}, \textsc{lagr} (for the
left-most coordinand), and \textsc{ragr} (for the right-most coordinand). 

\ea
\avm{
\type{nom-coord-phrase} \impl\\
	[\punk{synsem|loc|cat|head}{[lagr & \1 \\
                         		ragr & \2 ]} \\
	dtrs &	<	[synsem|l|cat|head|lagr & \1 ], \ldots, [synsem|l|cat|head|ragr & \2]	> ]
}
\z
\ea
\avm{
\emph{noun} \impl    
	[lagr & \1 \\
  	ragr & \1 \\
  	concord & \1 ]
}  
\z

Nouns have the same value for  \textsc{concord}, \textsc{lagr}, and \textsc{ragr}, and 
determiner and (attributive) adjective agreement in \ili{Romance}  involves the  \textsc{concord} feature.
Attributive adjectives constrain the agreement features of the noun they modify (via the \textsc{mod} or \textsc{sel} feature). One may distinguish two types for prenominal and postnominal adjectives, by the binary \textsc{lex} $\pm$ feature \citep{Sadler:Arnold:94} or by the \textsc{weight} light/non-light feature \citep{Abeille:Godard:99}. In this perspective, each has its agreement pattern, which we simplify as follows, using `$\vee$' to express a disjunction of feature values:

\ea
\avm{
\type{prenominal-adj} \impl\\ 
[\punk{concord}{\1} \\
 sel &	[lagr & \1 ] ]
}
\z
\ea
\avm{
\type{postnominal-adj} \impl\\  
[\punk{concord}{\1 $\lor$ \2} \\
 sel & [concord & \1 \\
        ragr & \2 ] ]
}
\z

\noindent
In the absence of coordination, these constraints apply vacuously, since \textsc{concord}, \textsc{lagr}, and \textsc{ragr} all share the same values. 


\section{Lexical coordination}\label{lexcoord}\label{coordination:sec-lexical-coordination}

%(including coordination of word parts, Abeill\'{e} 2006, Chaves 2008)

While coordinands have often been assumed to be phrasal (see for example \citealt[Section~6.2]{Kayne:94} and \citealt[Section~5.2]{bruening}, among others), \citet{Abeille:06} gives several arguments in favor of lexical coordination.
In some contexts, words (or phrases with a premodifier) are allowed, but not full phrases. In \ili{English}, this is the case with prenominal adjectives and postverbal particles. See \citet[Section~4]{Abeille:06} for similar examples with various categories in different languages. Most \ili{English} attributive adjectives are prenominal unless they have a complement. Although  adjectival phrases with complements are not licit in prenominal position,  it is possible to have complex adjectival expressions if they are coordinate.

\eal
\ex[] {a tall / proud man}
\ex[*] {a [taller than you] man}
\ex[*] {a [proud of his work] man}
\ex[] {a [big and tall] man}
\zl

As observed by \citet[\page 176--177]{hpsg1}, a particle may project a phrase after the nominal
complement (\mex{1}a), but not before (\mex{1}b); but coordination is possible, at least for some
speakers, as the example in (\mex{1}c) from \citew[\page 23]{Abeille2006a} shows.
\eal
\ex Paul turned the radio [(completely) off].
\ex Paul turned [(*completely) off] the radio.
\ex Paul was turning [on and off] the radio all the time.
\zl

While phrasal coordination can conjoin unlike categories (see below), this is not the case with lexical coordination:

\eal
\ex[] {Paul is  [head of the school] [and proud of it].}
\ex[\#] {Paul is [head and proud] of the school.}
\zl

Semantically, lexical coordination is more constrained than phrasal coordination. With \textit{and},
two lexical verbs that share a preverbal \isi{clitic} in \ili{French} must share the same verbal root, and in
\ili{Spanish}, they must refer to the same event \citep{Bosque:86}.
%\todostefan{Stefan: Could not find paper. Please add page numbers. answer:I dont have it}

\eal
\ex[]{
\gll Je te  dis  et  redis  que  tu  as   tort. \\
     I  you tell and retell that you have wrong\\\hfill(\ili{French})
\glt `I'm telling you again and again that you are wrong.'
}
\ex[\#]{ 
\gll Je te dis et promets que tu as tort.\\
     I you tell and promise that you have wrong\\
\glt `I'm telling and promising you that you are wrong.'
}
\ex[]{
\gll Lo compro y vendio en una sola operacion.\\
     it buy.1\textsc{sg} and sell.1\textsc{sg} in a single operation\\\hfill(\ili{Spanish})
\glt `I buy and sell it in one single operation.'
}
\ex[*]{
\gll Lo compro hoy y vendio ma\~{n}ana.\\
     it buy.1\textsc{sg} today and sell.1\textsc{sg} tomorrow \\
\glt `I buy it today and sell it tomorrow.'
}
\zl

Some apparent cases of lexical coordination may be analyzed as Right-Node Raising
\citep{Beavers}. These cases differ semantically and prosodically from Right-Node Raising, however: with typical Right-Node
Raising, the two coordinands must stand in contrast to one another, and do not have to refer to the
same event. With Right-Node Raising, there is usually a prosodic boundary at the ellipsis site (see
\citealt[843--844]{chavesrnr} and \crossrefchapteralt[Section~\ref{ellipsis:sec-rnr}]{ellipsis}). In \ili{French}, the first coordinand cannot end with a clitic article or with a weak preposition as in (\ref{fr1}b,c), quoted from \citep[\page 14]{Abeille:06}.

\eal
\label{fr1}
\ex[*]{ \gll Paul cherche le, et Marie conna{\^{i}t} la responsable.\\
 Paul searches the.\textsc{m.sg} and Marie knows the.\textsc{f.sg} responsible\\
\glt `Paul looks for the and Marie knows the one in charge.'
}
\ex[*]{  \gll Paul parle de, et Marie discute avec Woody Allen.\\
 Paul speaks of and Marie talks with Woody Allen\\
\glt `Paul speaks of and Marie talks with Woody Allen.'}
\zl




No such boundary occurs before the coordinator in lexical coordination. Thus, in \ili{French},  clitic
articles or weak prepositions with a shared argument can be conjoined \citep[\page 14]{Abeille:06}:


\eal
\judgewidth{??}
\ex[]{
\gll Paul cherche   le                ou la                responsable.\\
     Paul looks.for the.\textsc{m.sg} or the.\textsc{m.sg} responsible\\
\glt `Paul is looking for the man or woman in charge.'
}
\ex[]{
\gll un film de et avec Woody Allen\\
     a film by and with Woody Allen\\
}
\zl

\noindent
The functor analysis of coordinands in (\ref{le-coord-lexeme-marker}) is compatible with lexical
coordination, since the head-functor phrase in (\ref{head-functor-construction}) has the same
valence features as the head. The weak head analysis in (\ref{lexcoordentry}) is also compatible,
since the coordinator inherits the complements expected by the coordinand (this is done by
concatenation of \comps lists as it is for complex predicates; see
\crossrefchapteralt[Section~\ref{complex-predicates-sec-argument-attraction}]{complex-predicates}).


%\ea
%\begin{avm}
%\type{ lex-coord} \impl
 %  \[synsem|loc|cat  \[head &  \@{0}\\
  %                     weight &  \@{3}light\]\\
  %  arg-st  \@{1} $\oplus$ 
  %  \< \[synsem|loc|cat \[head & \@{0}\\
  %                        weight & \@{3}\\
   %                       arg-st \@{1} $\oplus$ \@{2} \]\]\> $\oplus$ 
  %  \@{2}\]\end{avm}
%\z
                                                    
The construct resulting from the coordination of lexical elements has hybrid properties: as a syntactic construct, it must be a phrase, but it also behaves as a word. Coordinate verbs behave as lexical heads; coordinate adjectives may occur in positions ruled out for phrases. To overcome this apparent paradox, \citet[Section~5.1]{Abeille:06} analyses it as an instance of a ``light'' phrase, following the \textsc{weight} account of \citet{Abeille:Godard:2000, Abeille:Godard:2004}. Light elements can be words or phrases, and can have restricted mobility (see \crossrefchapteralt{order}). For example, prenominal modifiers can be constrained to be [\textsc{weight} \emph{light}]. In this theory, light phrases can be coordinate phrases or head-adjunct phrases, provided all their daughters are light. Figure~\ref{light} illustrates this, assuming a functor analysis.

% \begin{figure}
% \tikzexternaldisable
%     \hfill
% \Tree[.{VP\begin{avm}
%         \[spr & \< NP$_x$ \>\\
%          comps & \<NP$_y$\>\\
%          weight & light \]
%         \end{avm}} [.{V\begin{avm}
%         \[spr & \< NP$_x$ \>\\
%          comps & \<NP$_y$\> \]
%         \end{avm}} {likes} ] 
% [.{VP\begin{avm}
%         \[spr & \< NP$_x$ \>\\
%          comps & \<NP$_y$\>\\
%          weight & light \]
%         \end{avm}}  [.Coord {and} ]  [.{V\begin{avm}
%         \[spr & \< NP$_x$ \>\\
%          comps & \<NP$_y$\> \]
%         \end{avm}} {approves } ]	 ] ]
% \hfill
% \Tree[.AP$[$\emph{light}$]$  [.A {big} ]  
% [.{AP$[$\emph{light}$]$} [.Coord {and} ] [.A$[$\emph{light}$]$  {tall} ]  ] ]
% \hfill\mbox{}
%     \caption{Examples of lexical coordination}
%     \label{light}
% \end{figure}
\begin{figure}
%    \hfill
\begin{forest}
sm edges
[\avm{
        V$'$[subj   & < \1 >\\
             comps  & < \2 >\\
             weight & light ]
        } 
  [\avm{
        V[subj & < \1 >\\
         comps & < \2 > ]
        } [likes] ] 
[\avm{
        V$'$[subj   & < \1 >\\
             comps  & < \2 >\\
             weight & light ]
        }  [Coord [and] ]  
  [\avm{
        V[subj  & < \1 NP$_x$ >\\
          comps & < \2 NP$_y$ > ]
        } [approves] ]	 ] ]
\end{forest}
\hfill
\begin{forest}
sm edges
[{A$'$[\emph{light}]} 
  [A [big] ]  
  [{A$'$[\emph{light}]} 
    [Coord [and] ] 
    [{A[\emph{light}]}  [tall] ]  ] ]
\end{forest}
%\hfill\mbox{}
    \caption{Examples of lexical coordination}
    \label{light}
\end{figure}


\section{Coordination of unlike categories}\label{unlikessec}\label{coordination:sec-unlikes}

The categories of coordinands are required to be the same per the the coordination construction in (\ref{coordparam2}).
But this requirement is excessive, as illustrated by  the coordinations in (\ref{unlk1}) from
\citet[\page 580]{bayer} and % Republican & proud of it
\citet[\page 1327]{rodney}, % four month trip to Africa
among  others; see \citet[\page 169--170]{Chaves2013b-u}.
 Such data raise the problem of determining what the part of speech and the categorial status of the coordinate phrase should be.

\eal
\label{unlk1}
\ex Kim is [alone]\sub{AP} and [without money]\sub{PP}.
\ex Pat is [a Republican]\sub{NP} and [proud of it]\sub{AP}.
\ex Jack is [a good cook]\sub{NP} and [always improving]\sub{VP}.
\ex What I would love is [a trip to Fiji]\sub{NP} and [to win \$10,000]\sub{VP}.
\ex That was [a rude remark]\sub{NP} and [in very bad taste]\sub{PP}.
\ex Chimpanzees hunt [frequently]\sub{AdvP} and [with an unusual degree of\newline success]\sub{PP}.
\ex I'm planning [a four-month trip to Africa]\sub{NP} and [to return to York after\-wards]\sub{VP}.
\zl

%\begin{exe}
%\ex \begin{xlista}
%\ex Kim is  [alone and without money].\\
% \hfill [AP \& PP]
%\ex  Pat is [a Republican and proud of it]. \\
% \hfill [NP \& AP]
%
%\ex  Jack is [a good cook and always improving].\\ \hfill [NP \& VP]
%
%\ex What I would love is [a trip to Fiji and to win \$10,000].\\
%\hfill [NP \& VP]
%
%\ex  That was [a rude remark and in very bad taste]. \\
%\hfill [NP \& PP]
%
%\ex Chimpanzees hunt [frequently and with an unusual degree of success].\\
%\hfill [AdvP \& PP]
%
%\ex I'm  planning [a four-month trip to Africa and  to return to York afterwards].\\
%\hfill [NP \& VP]
% \end{xlista}\label{unlk1}
%\end{exe}


\noindent
Building on observations from \citet[417]{jacobson}, \citet{Sag:03} and others pointed out that the
features of the mother are not simply the intersection of the features of the coordinands.  For
example, verbs like \emph{remain} are compatible with both AP and NP complements, whereas
\emph{grew} is only compatible with APs.  This is shown in (\ref{republican}).  Crucially,
however, the information associated with the phrase \emph{wealthy and a Republican} somehow allows
\emph{grew} to detect the presence of the nominal, as (\ref{show2}a) illustrates, even when the
verbs are coordinated, as in (\ref{show2}b--d).


\eal
\label{republican}
\ex  Kim remained/grew wealthy.
\ex  Kim remained/*grew a Republican.
\zl



\eal
\label{show2}
\ex[] {Kim remained/*grew [wealthy and a Republican].}
\ex[] {Kim grew and remained wealthy.}
\ex[*] {Kim grew and remained a Republican.}
\ex[*] {Kim grew and remained [wealthy and a Republican].}
\zl


A number of influential accounts in \isi{Type-Logical Grammar}
\citep{morrill90,morrill94,bayer}
% St. Mü.
%\inlinetodoobl{Stefan: I have all papers, but cannot find the places searching for disjunction.}
% Morrill:90 page 10 in downloaded PDF
% Morrill:94 has it on page 166, but I am not sure this is the right one.
use disjunction introduction, one of the rules of inference from propositional calculus, in order to
deal with coordination of unlikes phenomena. Disjunction introduction allows one to infer $P \vee Q$
from $P$, and if one assumes that categories like NP, PP, and so on can also be disjunctive, the
grammar allows an expression of type `NP' to lead a double life as an `NP $\vee$ PP' expression, or
the type `AP' to be taken as an `AP $\vee$ PP $\vee$ NP', and so on. This kind of approach has been
adapted to HPSG; see, for example, \citet{Daniels02} and \citet{Yatabe:04}.  Related work, such as \citet{Sag:03}, aims to
achieve the same result using type-underspecification. Other, more
exploratory work views coordination of unlike categories as the result of parts of speech being
gradient and epiphenomenal rather than hard-coded into the type signature
\citep{Chaves2013b-u}. Finally, \citet{berthold0}, \citet{yatabe}, \citet{Beavers}, and
\citet{chaves06} argue that coordination of unlikes can be explained by a deletion operation that
omits the left periphery of non-initial coordinands, illustrated in (\ref{unlk}).

\eal
\label{unlk}
\ex Tom gave a book to Mary, and \sout{gave} a magazine to Sue.\footnote{\citew[\page 171]{Chaves2013b-u}}

\ex He drinks coffee with milk at breakfast and \sout{drinks coffee with} cream in the evening.\footnote{\citew[\page 214]{hudson84}}

\ex There was one fatality yesterday, and \sout{there were} two others on the day
before.\footnote{\citew[339]{chavesthesis}}

\ex I see the music as both going backward and \sout{going} forward.\footnote{
\url{https://www.hdtracks.com/music/artist/view/?id=2418}; accessed 2020-04-01.}
\zl

\noindent
In such a view, the examples in (\ref{unlk1}) are verbal coordinations where the verb (or the verb
and subject) has been deleted (e.g.\ \emph{Kim is alone and \sout{is} without money}).  The problem
is that left-periphery ellipsis cannot fully explain coordination of unlikes phenomena. For example,
there is no elliptical analysis of data like (\ref{baaad}). \citet{levine11} offers arguments
against the coercion account of \citet{chaves06} and against the existence of left-periphery
ellipsis. See \cite{yatabe12} for a reply.

\eal
\label{baaad}
\ex Simultaneously shocked and in awe, Fred couldn't believe his eyes.\footnote{\citew[\page 172]{Chaves2013b-u}}
\ex Both tired and in a foul mood, Bob packed his gear and headed North.\footnote{\citew[\page 112]{chaves06}}
\ex Both poor and a Republican, \emph{no one} can possibly be.\footnote{\citew[\page 172]{Chaves2013b-u}}
\ex Dead drunk and yet in complete control of the situation, \emph{no one} can be.\footnote{\citew[\page 142]{levine11}}
\zl



\noindent
Further problems for an ellipsis account of coordination of unlikes phenomena are posed by the
position of the correlative coordinators \emph{both}, \emph{either}, and \emph{neither} in
(\ref{baaad2z}).

\eal
\label{baaad2z}
\ex Isn't this both illegal and a safety hazard?
\ex It's both odd and in very poor taste to have a fake wedding.
\ex Who's neither tired nor in a hurry?
\ex Isn't she either drunk or on medication?
\zl



\noindent
 If (\ref{baaad2z}a) is an elliptical coordination
like \emph{isn't this both illegal and \sout{isn't this} a safety
hazard}, then the location of \emph{both} is unexpected. Instead of
occurring before the first coordinand, it is realized inside the first
coordinand. Crucially, the non-elided counterparts are not
grammatical, e.g.\ *\emph{isn't this both illegal and isn't this a
safety hazard?} The same issue is raised by (\ref{baaad2z}b,c). In
an elliptical account, one would have to stipulate
that \emph{both} can only float in the presence of ellipsis, which
is unmotivated.
Finally, see \citet{Mouret:07} for  an extensive discussion in favor of a non-elliptical analysis of unlike coordination, based on correlative coordination.
In sum, left-periphery ellipsis does not 
offer a complete account of coordination of unlikes, and underspecification
accounts are more promising.




\section{Non-constituent coordination}
\label{sec-non-constituent-coordination}

The fact that not all coordination of unlike categories can be reduced to deletion  does not entail that
deletion is impossible, or that no phenomena involve deletion.
We refer the reader to \crossrefchaptert{ellipsis} for more discussion about ellipsis.

Consider, for example, the non-constituent coordinations in (\ref{lpecx}). 

\eal
\label{lpecx}
\ex 
\label{ex-tom-gave-a-book-to-mary-and-a-magazine-to-sue}
Tom gave a book to Mary, and a magazine to Sue.\\
(\isi{Argument Cluster Coordination})

\ex Tom loves -- and Mary absolutely hates -- spinach dip.\\
(\isi{Right-Node Raising})

\ex Tom knows how to cook pizza, and Fred -- spaghetti.\\
(\isi{Gapping})

\zl

Some authors regard Argument Cluster Coordination as elliptical \citep{yatabe01,Crysmann:04,Beavers}; others
regard such phenomena as non-elliptical sequences \citep{Mouret:06}.
In the former approach,  phonological material in the left periphery of the non-initial coordinand that is identical to
phonological material in the left periphery of the initial coordinand is allowed to be absent in the mother node.
This can be achieved by adding the constraints in (\ref{lpec}) to the coordination construction, here shown in the binary-branching format for perspicuity.
Here, {\it coord} is an abbreviation of the phonologies of coordinators, like
{\it and}, {\it or}, etc.

\ea
\label{lpec}
\type{coord-phrase} \impl\\
\oneline{%
\avm{
[phon & \1 \+ \2 \+ \3 \+ \4 \\
 dtrs &	< [phon & \1 \+ \2 ne-list \\
	   \punk{synsem|loc|cat|coord}{none}],
	  [phon & \3 <(\type{coord})> \+ \1 \+ \4 ne-list \\
   	   \punk{synsem|loc|cat|coord}{crd}]	> ]
}
}
\z

\noindent
If \ibox{1} is resolved as the empty list then no ellipsis occurs, but if \ibox{1} is non-empty then ellipsis occurs, as illustrated in Figure~\ref{lpe}. 
Some accounts, like  \citet{yatabe01}, \citet{Crysmann:04}, \citet{Beavers}, and \citet{chaveslp} operate on
linearization domain\label{page-linearization-domains-in-coordination-two} elements instead of directly on \textsc{phon}.  
See \crossrefchaptert[Section~\ref{sec-domains}]{order} for more discussion about linearization theory.


\begin{figure}
    \centering
%\oneline{%    
%\begin{forest}
%[{VP[ \phon  \ibox{1} $\oplus$ \ibox{2} $\oplus$ \ibox{0} $\oplus$ \ibox{3} ]} 
%          [{VP[ \phon \ibox{1} \phonliste{ give } $\oplus$ \ibox{2} \phonliste{ a, book, to, Mary } ]}]
%         [{VP[ \phon \ibox{0} \phonliste{ and } $\oplus$ \ibox{1} \phonliste{ give } $\oplus$ \ibox{3} \phonliste{ a, magazine, to, Sue } ]}
%           [Coord]
%           [VP] ] ]
%\end{forest}
%}

\oneline{%    
\begin{forest}
[\avm{VP [phon & \1 \+ \2 \+ \3 \+ \4]} 
	[\avm{VP [phon & \1 < \type{give} > \+ \2 < \type{a, book, to, Mary} > ]}]
	[\avm{VP [phon & \3 < \type{and} > \+ \1 < \type{give} > \+ \4 < \type{a, magazine, to, Sue} > ]}
	[Coord]
	[VP] ] ]
\end{forest}
}
    \caption{Analysis of \emph{give a book to Mary and} \sout{\emph{give}} \emph{a magazine to Sue}}\label{lpe}
\end{figure}

This approach is motivated by the existence of ambiguity in 
sentences like (\ref{treesa}); see  \citet{Beavers} and \citet{chaves06} for more examples and discussion. Because (\ref{treesa}a) involves a one-time predicate, the ellipsis must include the subject phrase, otherwise
the interpretation is such that the same two trees were cut down twice. In contrast,  (\ref{treesa}b) does not involve a one-time
predicate, and thus it is possible for the ellipsis to simply
involve the verb.


\eal
\label{treesa}
\ex Two trees were cut down by Robin in July and by Alex in September.\\
(Two trees were cut down by Robin in July and \sout{two trees were cut down} by Alex in September.

\ex Two trees were photographed by Robin in July and by Alex in September.\\
(Two trees were photographed by Robin in July and \sout{photographed} by Alex in September)
\zl



In the non-elliptical analysis of such data, the missing material is recovered from the preceding
coordinand. For example, \citet[\page 263]{Mouret:06} proposes a rule along the lines of
(\ref{lines}).  Here, a new head feature \textsc{cluster} is introduced, which takes as its value
the list of \textsc{synsem} values of the daughters.

\ea
\label{lines}
\label{schema-ac-cx}
\type{argument-cluster-phrase} \impl\\ 
\avm{ [head|cluster < \1, \ldots{}, \tag{n} >\\
       dtrs < [ synsem \1 ], \ldots{}, [synsem  \tag{n} ] > ]
}
\z

\noindent
Mouret defines argument clusters as instances of the underspecified non"=headed construction
\type{argument-cluster-phrase} with one daughter or more. The construction is valence
saturated and clusters can be coordinated with one another.  He also postulates a lexical rule
allowing (for example) a ditransitive verb to take a coordination of clusters as complement (this
rule will also allow clusters for complements and adjuncts, assuming the latter are included in the
\textsc{comps} list):


\ea
\label{lr-cluster-selection}
\avm{
[comps < [loc|cat \1], \ldots{}, [loc|cat \tag{n}] > ]
}  $\mapsto$ \\
\flushright\avm{
[comps  < [coord $+$ \\
           head|cluster <[loc|cat \1], \ldots{}, [loc|cat \tag{n}]> ] >]
}
\z
Figure~\ref{fig-give-a-book-to-mary-and-a-mag-to-sue} shows the analysis of the VP in (\ref{ex-tom-gave-a-book-to-mary-and-a-magazine-to-sue}).
\begin{figure}
\begin{forest}
sm edges
[VP
  [{V[\comps \sliste{ \ibox{1} }]} [give]]
  [{XP\ibox{1}[\coord+, \textsc{cluster} \sliste{ NP, PP }]}
    [{XP[\textsc{cluster} \sliste{ NP, PP } ]}
      [NP [a book, roof]]
      [PP [to Mary, roof]]]
    [{XP[\textsc{cluster} \ibox{2}]}
      [X  [and]]
      [{XP[\textsc{cluster} \ibox{2} \sliste{ NP, PP }]}
        [NP [a magazine, roof]]
        [PP [to Sue, roof]]]]]]
\end{forest}
\caption{\citegen{Mouret:06} analysis of Argument Cluster Coordination}\label{fig-give-a-book-to-mary-and-a-mag-to-sue}
\itdsecond{Stefan: I think I asked this before but it is not in the text: How are the NP, PP related? Is there structure sharing? Where does it come from?}
\end{figure}
The respective NPs and PPs form a cluster that is licensed by (\ref{schema-ac-cx}). The phrases
\emph{a book to Mary} and \emph{a magazine to Sue} are coordinated and the respective
\textsc{cluster} values matched (see \citealt[\page 263]{Mouret:06} for details on this matching). The lexical item for \emph{give} is licensed by the lexical rule in (\ref{lr-cluster-selection}). This version of \emph{give} selects the cluster coordination rather than selecting the NP and PP directly.


This approach is motivated by non-clausal coordinators (\textit{as well as} and its \ili{French} equivalent \textit{ainsi que}), which are possible in Argument Cluster Coordination, but cannot conjoin tensed VPs:

\eal
\ex[] {John gave a book to Mary as well as a magazine to Sue.}
\ex[*] {John gave a book to Mary as well as gave a magazine to Sue.}
\ex[] {
\gll Paul offrira           un disque à  Marie ainsi~qu'  un livre à Jean.\footnotemark\\ 
     Paul offer.\FUT.3\SG{} a  record to Marie as.well.as a  book  to Jean\\
\footnotetext{
\citew[\page 253]{Mouret:06}
}
\glt `Paul will offer a record to Mary as well as a  book to Jean.'\\
}
\ex[*] {
\gll Paul offrira 	    un disque à  Marie ainsi~qu'  offrira 	    un livre à Jean.\\
     Paul offer.\FUT.3\SG{} a  record to Marie as.well.as offer.\FUT.3\SG{} a  book  to Jean\\
\glt `Paul will offer a record to Marie as well as will offer a book to Jean.'\\
}
\zl

Another argument is the placement of correlative coordinators: the first coordinator in
(\ref{ex:coord-76}a) must be postverbal; this shows that Argument Cluster Coordination does not
include the first verb. The examples below are from \citet[254]{Mouret:06}.

\eal
\label{ex:coord-76}
\ex[] {
\gll Jean a donné et un livre à Marie et un magazine à Sue.\\
     Jean has given  and a book to Marie and a magazine to Sue\\
\glt `Jean has given both a book to Marie and a magazine to Sue.'
}

\ex[] {
\gll Paul compte      offrir et  un disque à  Marie et  un livre à Jean.\\
     Paul plan.3\SG{} offer  and a  record to Marie and a  book  to Jean\\
\glt `Paul is planning to offer both a record to Marie and a book to Jean.'\\
}
\ex[*] {
\gll Paul compte et offrir un disque à Marie et un livre à Jean.\\
     Jean is.planning and to.offer a record to Marie and a book to Jean\\}
\zl

\noindent
Another argument is negation placement, which is a case of constituent negation
\citep[253]{Mouret:06}: 

\eal
\ex[] {
\gll Paul offrira           un disque à Marie  et  (non) pas un livre à Jean.\\
     Paul offer.\FUT.3\SG{} a  record to Marie and \hphantom{(}not not a  book  to Jean\\
\glt `Paul will offer a record to Marie and not a book to Jean.'
}
\ex[] {Paul gave a record to Mary and not a book to Bill. }
\ex[*] {Paul gave a record to Mary and not gave a book to Bill.}
% is it true in \ili{English}? John gave not a book to Sue but a record to Bob
\zl


A syntactic\label{coordination:page-rnr-start} and non-elliptical account of Right-Node Raising is harder to maintain given that this phenomenon does not seem to be sensitive to  syntactic structure, as (\ref{rnrex1}) shows. See 
\citet{bresnan74}, % OK complete paper
\citet[299]{wexlercull},  \citet[45]{grosu81}, \citet[\page 98--101]{mccawley}, and \citet[382, fn.\,30]{sab}
for more data and discussion.\footnote{%
  Steedman (\citeyear[\page 542]{steedman85}; \citeyear[\page 256]{gapsteed}; \citeyear[\page 17]{steedmanbook}) and
  \citet[183--184]{dowty88} claim that Right-Node Raising is syntactically bounded. See \citet[95]{phil} and \citet[\page 841]{Chaves2014a-u} for rebuttals.}
In the examples that follow, small capital letters indicate prosodic focus and material shared between both coordinands is delineated by square brackets.


\eal
\label{rnrex1}
\ex  I know a man who \textsc{sells} and you know a person who \textsc{buys}
                     [pictures of Elvis Presley].

\ex John wonders when Bob Dylan
\textsc{wrote} and Mary wants to know when
  he
\textsc{recorded} [his great song about the death of Emmet Till].
 
 \ex Politicians \textsc{win when they defend} and \textsc{lose when they attack}
[the right of a woman to an abortion].

\ex Lucy \textsc{claimed} that -- but \textsc{couldn't say}
exactly when --  $[$the strike would take place$]$.
 
 \ex I found a box \textsc{in} which and Andrea found a blanket \textsc{under}
which [a cat could sleep peacefully for hours without being
noticed].
\zl

Another source of evidence against syntactic and non-elliptical accounts of Right-Node Raising is that this phenomenon can involve lexical structure,
as the examples in (\ref{rnrex2}) by \citet[1325, fn.\ 44]{rodney} and \citet{chaveslp,chavesrnr} illustrate:

\eal
\label{rnrex2}
\ex Please list all publications of which you were the \textsc{sole} or
\textsc{co}-[author].\footnote{\citew[1325, fn.\ 44]{rodney}}
 
\ex  It is neither \textsc{un}- nor \textsc{overly} [patriotic] to tread that path.\footnote{\citew[\page 267]{chaveslp}} 
\ex The \textsc{ex-} or \textsc{current} [smokers] had a higher blood pressure.\footnote{\citew[\page 267]{chaveslp}} 

\ex The \textsc{neuro}- and \textsc{cognitive} [sciences] are
presently in a state of rapid development
[\ldots]\footnote{\url{https://opinionator.blogs.nytimes.com/2011/12/25/the-future-of-moral-machines/};
  accessed 2021-01-19.}

\ex Are you talking about \textsc{a new}  or about \textsc{an ex}-[boyfriend]?\footnote{\citew[867]{chavesrnr}}

\zl


%\begin{sloppypar}
Elliptical accounts of Right-Node Raising are proposed by \citet{Beavers},
\citet{Yatabe:04}, \citet{chavesrnr}, and others. The rule in (\ref{rnrcx}) illustrates the account adopted by 
 \citet[874]{chavesrnr}  and \citet*[\page 19]{aoi}
  in simplified format.\footnote{See \citet{chavesrnr} for more details about how ``cumulative'' Right-Node Raising is modeled by this rule, i.e.\
 cases like \emph{Mia donated -- and Fred spent --} (\emph{a total of}) \emph{\$10,000} (between them).
}
In a nutshell, the \textsc{m(orpho-)p(honology)} feature introduces two list-valued features, namely \textsc{phon}(\textsc{ology}) and \textsc{l(exical-)id(entifier)}. The former encodes phonological content, including phonological phrasing information,  whereas the latter is used to individuate lexical items semantically (i.e.\  the value
of \textsc{lid} is a list of semantic frames that canonically specify the meaning of a lexeme).
%\end{sloppypar}
 
\eas
\label{rnrcx}
\type{right-peripheral-ellipsis-phrase} \impl\\
\avm{
	[mp & \tag{$L_1$} \+ \tag{$R_1$} \+ \tag{$R_2$} \+ \tag{$R_3$}\\
	synsem & \1 \\
	dtrs &	<[mp & \tag{$L_1$} \+ \tag{$L_2$}	<[phon & \tag{$p_1$}\\
											lid & \1 ]
											,\ldots{},
											[phon & \tag{$p_n$}\\
											lid & \tag{n} ]> \+ \\
%			\hspace{0.7cm}
 	          ~ & \tag{$R_1$} \+ \tag{$R_2$}	<[phon & \tag{$p_1$}\\
 										lid & \1 ]
 										,\ldots{},
 										[phon & \tag{$p_n$}\\
 										lid & \tag{n} ]> \+ \tag{$R_3$}\\
			synsem & \1 ]> ]
}
\zs

\noindent
By requiring \textsc{phon} identity, this rule ensures that Right-Node Raising only targets strings
that are phonologically independent and have the same surface form, ruling out the ungrammatical
examples in (\ref{badp1}).  The assumption here is that the value of \textsc{phon} is not simply a
list of phonemes, but rather a structured list containing intonational phrases, phonological
phrases, prosodic words, syllables, and segments.


Stressed pronouns, affixes that correspond to independent prosodic words, and compound parts can be Right-Node Raised because  they are  independent prosodic units in their local domains.
See \citet{swingle} for more discussion. 
% \inlinetodostefan{Isn't there any other reference? This is such a frequently discussed topic. Is
%   there no publication since 1995 discussing this? Google scholar lists and others cite it as 1993
%   not 1995, but it is not available anymore. Anne: Selkirk? Hartmann and Fery?}

\eal
\label{badp1}
\ex[]  {He tried \textsc{to persuade}  but he couldn't \textsc{convince} [THEM] / *[them].}
\ex[*] {I think that \textsc{I'd} and I know that \textsc{Pat'll} [buy  those portraits of Elvis].}
\ex[*] {They've always \textsc{wanted} a -- and so I've \textsc{given them} a --  [coffee grinder].}
\ex[*] {I bought  \textsc{every red} and Jo liked \textsc{some blue} [t-shirt].}
\zl


\noindent
By requiring  \textsc{lid} identity, the rule prevents homophonous strings that have fundamentally different semantics from being Right-Node Raised, as in (\ref{badp2}). In such cases, oddness arises, because in general the same phrase cannot simultaneously have  two meanings, except in puns  \citep[316]{zaenenkart}. 

\eal
\label{badp2}
% \ex[*]{Randy \textsc{saw} and Rene has \textsc{been} [flying planes].\\
% \citep{booij:85}\todostefan{This is not in this paper.}
% }
\ex[*] {John \textsc{will} and Sandy \textsc{built the} [drive].\footnote{\citew[\page 936]{Milward94}}
}
%\ex[*] {Mary \textsc{fed} and Tom \textsc{enjoyed} [the lamb].\\
%(adapted from \citet[64]{buitelaar1998corelex})}\todostefan{Stefan: This is wrong. The page does not
%contain examples}
\ex[*] {Robin \textsc{swung}  and Leslie \textsc{tamed} [an unusual bat].\footnote{\citew[156]{levhubook}}}%OK
\ex[*] {We need new \textsc{black}- and \textsc{floor}[boards].\footnote{adapted from \citew[\page 371]{artstein5}}}
\ex[*] {I caught \textsc{butter}- and \textsc{fire}[flies].\footnote{\citew[\page 274]{chaveslp}}}
\ex[*] {There stood a \textsc{one-} and \textsc{well}-[armed man].\footnote{\citew[\page 869]{chavesrnr}}}
\zl

\noindent
At the same time, \textsc{lid} identity does not go as far as requiring co-referentiality of the shared material. This is  as intended, given ambiguous examples like
\emph{Chris \textsc{likes} and Bill \textsc{loves}  $[$his bike$]$}.
The\label{coord:page-rnr-I-phi-start} account of Right-Node Raising is illustrated below. Here, \emph{I} corresponds to an intonational phrase,
and  $\phi$ to a phonological phrase.
Note that this is a unary-branching rule, which means that it can in principle apply to any phrasal node, including non-coordinate cases of Right-Node Raising:


\begin{figure}
    \centering

\oneline{%
\begin{forest}
[%
\avm{
	S [\type*{phrase}
	mp &	<[phon & <[$I$ & [$\phi$ & /\textipa{kIm lAIks}/] ]> \\
			lid & \ldots ]
			,
			[phon & <[$I$ &	[$\phi$ & /\textipa{\ae nd mij@ heIts}/] ]>\\
			lid & \ldots ]
			,
			[phon & <[$I$ &	[$\phi$ &  /\textipa{beIg@lz}/] ]> \\
			lid & \ldots ]>  ]
}
	[%
	\avm{
		S [\type*{phrase}
		mp & <	[phon & <[$I$ &	[$\phi$ & /\textipa{kIm lAIks}/] ]> \\
				lid & \ldots ]
				,
				[phon & <[$I$ &	[$\phi$ & /\textipa{beIg@lz}/] ]> \\
				lid & \ldots ]
				,\\
				[phon & <[$I$ &	[$\phi$ & /\textipa{\ae nd mij@ heIts}/] ]>\\
				lid & \ldots ],
				[phon & <[$I$ &	[$\phi$ & /\textipa{beIg@lz}/ ] ]> \\
				lid & \ldots ]> ]
	}
%    
%%     \scalebox{1}{
%%     {\begin{avm}
%% S\[\type{phrase}\\
%% mp \<
%% \[phon \[$I$\\
%%     \[$\phi$\\
%%        /\textipa{kIm} \textipa{lAIks}/\]\]\\
%%        lid \ldots{}\],
%%     \[phon \[$I$\\
%%      \[$\phi$\\
%%             /\textipa{\ae nd} \textipa{mij@} \textipa{heIts}/\]\]\\
%%             lid \ldots{}\],
%%          \[phon \[$I$\\ \[$\phi$\\  /\textipa{beIg@lz}/\]\]\\
%%          lid \ldots{} \]\>  \]\end{avm}}}
%         
%%       $|$   
%      
%%     \scalebox{1}{
%% \begin{avm}
%% S\[\type{phrase}\\
%% mp
%% \<  \[phon \[$I$\\
%%     \[$\phi$\\
%%        /\textipa{kIm} \textipa{lAIks}/\] \]\\
%%       lid ..\],
%%         \[phon    \[$I$\\ \[$\phi$\\  /\textipa{beIg@lz}/\]\]\\
%%               lid \ldots{}\],\\
%%      \[phon \[$I$\\
%%      \[$\phi$\\
%%             /\textipa{\ae nd} \textipa{mij@} \textipa{heIts}/\]\]\\
%%           lid \ldots{}\],
%%          \[phon \[$I$\\ \[$\phi$\\  /\textipa{beIg@lz}/ \]\]\\
%%               lid \ldots{}\]   \>  \]\end{avm}}
%               
[{Kim \textsc{likes} bagels and Mia \textsc{hates} bagels.},roof ]]]
\end{forest}}
               
    \caption{Analysis of \emph{Kim likes, and Mia hates, bagels.}}\label{rnrt}
\end{figure}


\eal
\ex  It's interesting to compare the people who \textsc{like} with the people
       who \textsc{dislike} [the power of the big unions].\footnote{\citew[550]{hudson}}%OK

 \ex Anyone  who \textsc{meets} really comes to \textsc{like} [our sales people].\footnote{adapted from \citew[\page 267]{williams}}\label{will}

\ex   Spies who learn \textsc{when} can be more valuable than those
able to learn \textsc{where} [major troop movements are going to occur].\footnote{\citew[\page 101]{postal94}}

\ex Politicians who have fought \textsc{for} may well snub those
 who have fought \textsc{against} [chimpanzee rights]. \footnote{\citew[\page 104]{postal94}}

\ex Those who voted \textsc{against} far outnumbered those who
voted  \textsc{for} [my father's motion].\footnote{\citew[1344]{rodney}}


\ex If there are people who \textsc{oppose} then maybe there are also some
  people who actually \textsc{support}  [the hiring of unqualified
  workers].\footnote{\citew[\page 840]{chavesrnr}}

\zl




In the example in Figure~\ref{rnrt}, the sub-list \ibox{R\subscr{3}} in (\ref{rnrcx}) is resolved as the empty list, but this need not be so. When the final sublist is not resolved as the empty list, we obtain discontinuous Right-Node Raising cases like (\ref{disrnr}), due to  \citet[238--240]{Whitman:09}
and \citet[868]{chavesrnr}, where the Right-Node Raised expression is followed by extra material. 
\eal
\label{disrnr}
\ex The blast \textsc{upended} and \underline{\textsc{nearly sliced}} [an armored Chevrolet Suburban] \underline{in half}.

\ex During the War of 1982, American troops \textsc{occupied}  and \underline{\textsc{burned}} [the town] \underline{to the ground}.

\ex Please move from the exit rows if you are \textsc{unwilling} or \underline{\textsc{unable}}
 [to perform the necessary actions] \underline{without injury}.

\ex The troops that \textsc{occupied} ended up \underline{\textsc{burning}} [the town] \underline{to the ground}.
\zl\label{coord:page-rnr-I-phi-end}\label{coordination:page-rnr-end}


Finally, let us now turn our attention to Gapping, as in 
\emph{Robin likes Sam and Tim \trace Sue}.
There are elliptical accounts of Gapping  \citep{chaves06} as well as direct-interpretation accounts where the missing material is recovered from the preceding linguistic context  \citep{Mouret:06,Abeille:Blbie:Mouret:14,sangheepark}; see \crossrefchaptert[Section~\ref{ellipsis:sec-gapping}]{ellipsis}. The latter is illustrated in Figure~\ref{gfig}, in simplified format. Basically, the Question Under Discussion (QUD, \citealp{roberts96}) of the first clause is $\lambda y.\lambda x. \exists e(like(x,y))$ which is information that is shared across the clausal daughters as \ibox{1}.
This allows the second coordinand to combine the two NPs with the verbal semantics, and recover the propositional meaning.

%\fi


\begin{figure}
\centering
\oneline{
\begin{forest}
%sm edges
[%
\avm{
	S [qud & \1 \\
	content & \{ $\exists e'$!(\type{like}(\type{robin,sam}))! $\land \exists e$!(\type{like}(\type{tim,sue}))! \} ]
}
	[%
	\avm{
		S [qud & \1 \\
		content & \{ \2 $\exists e'$!(\type{like}(\type{robin,sam}))! \}]
	}]
	[%
	\avm{
		S [qud & \1 \\
		content & \{ \2 $\land \exists e$!(\type{like}(\type{tim,sue}))! \}]
	}
		[Conj [and] ]
		[%
		\avm{
			S [qud & \1 \{$\lambda y. \lambda x. \exists e$!(\type{like}(\type{x,y}))! \} \\
			content & \{$\exists e$!(\type{like}(\type{tim,sue}))! \} ]
		}
  			[\avm{NP [content & \{$tim$ \}]}]
			[\avm{NP [content & \{$sue$ \}]}] ] ] ]
\end{forest} 
     \caption{Analysis of \emph{Robin likes Sam and Tim -- Sue} (abbreviated)}\label{gfig}
}
\end{figure}


Like Right-Node Raising, Gapping is not restricted to coordinate structures as \citegen[\page 30--31]{sangheepark} attested 
examples in (\ref{gnc}) illustrate, contrary to widespread assumption.   Thus, the Gapping rule
proposed by \citet[\page 125]{sangheepark} that allows a gapped clause to follow a non-gapped clause is not
specific to coordination.

\eal
\label{gnc}
\ex Robin speaks \ili{French} better than Leslie \trace \ili{German}.
\ex My purpose here is not to resolve the crucial disagreement between two prominent theoreticians in a way that one would be declared true while the other one \trace false.
\ex The keynote of their relationship was set when Victoria, already a reigning queen,
had to propose to Albert, rather than he \trace to her.
\ex The public remembers all that and usually recognizes us before we \trace them.
% is in the main text now \citep[\page 31]{sangheepark}
\zl




\section{Conclusion}

Coordination is a pervasive phenomenon in all natural languages. Despite intensive research in the
last 70 years, its empirical properties continue to challenge most linguistic theories: the
coordination lexemes play a crucial role but do not behave like usual syntactic heads, the
coordinands do not need to be identical but display some parallelism relations and can be unlimited
in number, some non-constituent sequences can be coordinated, peculiar ellipsis phenomena can
optionally occur, etc. We have shown how HPSG offers precise detailed analyses of various coordinate
constructions for a wide variety of languages, factoring out the common properties shared by other
constructions and the properties specific to coordination.

Central to the HPSG analyses are two main ideas: (i) coordination structures are non-headed phrases
and come with different subtypes, and (ii) the parallelism between coordinate daughters is captured
by feature sharing. From these ideas, specific properties can be derived, regarding extraction and
agreement, for instance. Nevertheless, there is no clear consensus about some remaining issues. In
some accounts, the coordinator is a weak head, whereas in others it is a marker. Coordinate
structures are binary branching in some accounts but not so in others. Agreement is always local
(with the whole coordinate phrase) in some approaches, whereas locality is abandoned by others
% Borsley uses domains
to account for Closest Conjunct Agreement. Finally, in some accounts, non-constituent coordination
involves some form of deletion, but in others no deletion operation is assumed.

 
%\section*{Abbreviations}
%
%\begin{tabularx}{.99\textwidth}{@{}lX}
%\textsc{ppart} & past participle\\
%\end{tabularx}


\section*{Acknowledgements}

We are thankful to Bob Borsley, Jean-Pierre Koenig, Stefan Müller, and other reviewers for comments and suggestions on earlier drafts. 
As usual, all errors and omissions are our own.


{\sloppy
\printbibliography[heading=subbibliography,notkeyword=this] 
}
\end{document}


%      <!-- Local IspellDict: en_US-w_accents -->
