%% -*- coding:utf-8 -*-
\section*{Foreword of the second edition}

\largerpage
The second edition comes with a lot of small improvements: the index has been improved, typos have
been fixed, reference were added, and ORCIDs were added to authors and are displayed on the title pages of the papers now.

%1 properties
The type in the example (\ref{ex:prop38})
on p.\,\pageref{ex:prop38} was changed from \type{phrase} to \type{example-type}. As noted by
Philipp Trapp in 2022, the presence of the feature \textsc{head-daughter} would entail that the type
of the AVM is \type{headed-phrase} and since the type implication in (\ref{ex:prop38}) applies to all
structures of type \type{phrase} this would mean that all linguistics objects of type \type{phrase} have to be
of type \type{headed-phrase}, which would result in contradictions for all subtypes of \emph{phrase}
that are not of type \type{headed-phrase}.

%2 evolution
The LILOG system is now mentioned in the chapter about the evolution of HPSG (Section~\ref{sec-LILOG}).

%4 lexicon
The argument realization principle in (\ref{wd-bouma}) on p.\,\pageref{wd-bouma} was fixed. It
contained too many brackets in the specification of the \depsl. Footnote~\ref{fn-subject-ARP} was
added to explain how constraints on the length of the \subjl can be enforced.

%6 agreement
% 04.09.2004
Example (\ref{ex-gender-agreement-historical-development}) on
p.\,\pageref{ex-gender-agreement-historical-development} was modified so that the respective
examples for historical stages of development of gender agreement are more similar.

%7 case
The example (\ref{ex:syn:gfr}) on p.\,\pageref{ex:syn:gfr} was changed to another example excluding the possibility that we
are dealing with an instance of non-matching free relative clauses.

%9 arg-st
Footnote~\ref{fn-mapsto-lexical-rules} was added on p.\,\pageref{fn-mapsto-lexical-rules} to explain
the lexical rule notation using $\mapsto$ that was employed later in the chapter. A footnote
mentioning the mapping of non-canonical objects to valence features was added (p.\,\pageref{fn-non-cannonical-arg-st}). 
%The type \type{ppro} was explained.

%10 order
The relation \texttt{synsems2signs} was explained by adding (\ref{ex-schema-hc-flat-synsem-sign}),
which is an expansion of (\ref{schema-hc-flat}) on p.\,\pageref{schema-hc-flat}. The relation that
is used in the chapter on complex predicates has the same name now (see
(\ref{CP-ex-head-complements-phrase}) on p.\,\pageref{CP-ex-head-complements-phrase}). The
footnote~\ref{fn-order-lexical-Uszkoreit} was added. It discusses a lexical account of constituent
order assuming a separate lexical item for each ordering variant.

%14 relative clauses
Relative pronouns are NPs. The respective representation in (\ref{x:rc-18}) on p.\,\pageref{x:rc-18}
was fixed. Valence features were added to the lexical item in (\ref{x:rc-17}).

%15 islands
% 2024-09-07
%A reference was added in Chapter~\ref{chap-islands}.

%16 coordination
\emph{Mary} is of category NP rather than N in Figure~\ref{coordphr} on p.\,\pageref{coordphr}. 

%17 idioms
There was an NP to many in the \compsl in the lexical item in (\ref{le-idiomatic-spill}) on
p.\,\pageref{le-idiomatic-spill}. 
% 2024-09-07
References to recent publications were added to Chapter~\ref{chap-idioms}.

%24 Processing
%2024-09-10
A note on psycholinguistic arguments for a lexical analysis of complex predicates was added to
Section~\ref{processing:sec-lexicalism} of Chapter~\ref{chap-processing}. 

%Diff-lists are now explained A bit of explanation and a reference was added to

% order: 04.01.22
% Gray -> gray
% der Frau -> dem Kind
% added \ref{ex-schema-hc-flat-synsem-sign}

% complex-predicates 04.01.22
% unified synsems2signs. The relation has the same name now in order.tex and complex-predicates.tex

% relative-clauses.tex 05.01.22
% \trace -> \trace{}
% glosses aligned in {x:rc-129}
% added language tag
% fixed index entry for Bavarian German


% 18.01.22 added language info for German examples

% 25.01.22 Footnote~\ref{fn-hf-schema} was missing. % in udc

% 03.02.22 Idioms: NP in (8) too much, REL bad feature name, ref to Krenn&Erbach added

% 08.02.22 Information structure: added page numbers for Bildhauer & Cook 2010
%          fixed layout issue with Head-Dislocation Schema for Catalan
% 09.02.22 Added sentence about diff-list and reference to copestake2002.
%
% 14.02.22 Added glosses to helfen in chapter on processing
%
% 30.03.22 Figure 4, Mary is NP not N
%
% 26.10.22 (38a) used to be phrase => but since the constraint referred to HD-DTR this would cause a
% conflict for unheaded phrases. Noticed by student Philipp Trapp.
% The left-hand daughter in (38b) must be SYNSEM X, noted by St.Mü.

% 01.11.22
% Daughter in head-filler-phrase must have SYNSEM|LOCAL instead of LOCAL. St. Mü.
%
% 22.11.22
% added comma in 14b in lexicon.tex
%
% 01.12.2022 index entries for \ominus
%
% 10.01.2023 Added comma in np.tex
%
% 17.01.2023 unified spelling of reduced-verb and basic-verb, added dot to example in
% complex-predicates-include.tex
%
% 24.01.2023  removed space udc.tex, added Section to reference of Ross67 regarding ATB
% added crossref to island chapter.
%
% 2023-08-23 Kim Sells appeared in 2015 not in 2014, we missed this despite the check.
%
% 2023-09-26 The LILOG system is now mentioned in evolution.tex
%
% 2023-11-21 SYNSEM|LOCAL in MORPH in (34) in lexicon.tex
%
% 2023-12-12 Fixed NP for relative pronouns rather than N'.
% added SPR and COMPS for lexical item for relative pronoun
% Changed PP [3] into [LOC [3]] in figure in relative clause chapter
% Fixed PP[4], which should have been [3] in footnote in relclause chapter.
%
% 31.01.2024 added Abbreviations for case.tex since illative is not in the Leipzig Glossing Rules.

The book was used at the LSA Linguistic Institute 2023 at the University of Massachusetts Amherst by Tony Davis and in various seminars at the Humboldt
Universität zu Berlin by Stefan Müller. We want to thank everybody who commented on the book.

~\medskip

\noindent
Berlin, Paris, Bangor, Buffalo, \today\hfill Stefan Müller, Anne Abeillé, Robert D. Borsley \& Jean-​Pierre Koenig


%      <!-- Local IspellDict: en_US-w_accents -->
