\documentclass[output=paper,biblatex,babelshorthands,newtxmath,draftmode,colorlinks,citecolor=brown]{langscibook}
\ChapterDOI{10.5281/zenodo.5599842}

\IfFileExists{../localcommands.tex}{%hack to check whether this is being compiled as part of a collection or standalone
  \usepackage{../nomemoize}
  % add all extra packages you need to load to this file 

% the ISBN assigned to the digital edition
\usepackage[ISBN=9783961102556]{ean13isbn} 

\usepackage{graphicx}
\usepackage{tabularx}
\usepackage{amsmath} 

%\usepackage{tipa}      % Davis Koenig
\usepackage{xunicode} % Provide tipa macros (BC)

\usepackage{multicol}

% Berthold morphology
\usepackage{relsize}
%\usepackage{./styles/rtrees-bc} % forbidden forest 08.12.2019

% provides logo priniting commands
\usepackage{langsci-basic}

\usepackage{langsci-optional} 
% used to be in this package
\providecommand{\citegen}{}
\renewcommand{\citegen}[2][]{\citeauthor{#2}'s (\citeyear*[#1]{#2})}
\providecommand{\lsptoprule}{}
\renewcommand{\lsptoprule}{\midrule\toprule}
\providecommand{\lspbottomrule}{}
\renewcommand{\lspbottomrule}{\bottomrule\midrule}
\providecommand{\largerpage}{}
\renewcommand{\largerpage}[1][1]{\enlargethispage{#1\baselineskip}}

\usepackage{./styles/biblatex-series-number-checks}


\usepackage{langsci-lgr}

\newcommand{\MAS}{\textsc{m}\xspace} % \M is taken by somebody

%\usepackage{./styles/forest/forest}
\usepackage{langsci-forest-setup}

% is loaded in main etc.
% \usepackage{nomemoize} 
% \memoizeset{
%   memo filename prefix={chapters/hpsg-handbook.memo.dir/},
%   register=\todo{O{}+m},
%   prevent=\todo,
% }

\usepackage{tikz-cd}

\usepackage{./styles/tikz-grid}
\usetikzlibrary{shadows}


% removed with texlive 2020 06.05.2020
% %\usepackage{pgfplots} % for data/theory figure in minimalism.tex
% % fix some issue with Mod https://tex.stackexchange.com/a/330076
% \makeatletter
% \let\pgfmathModX=\pgfmathMod@
% \usepackage{pgfplots}%
% \let\pgfmathMod@=\pgfmathModX
% \makeatother

\usepackage{subcaption}

% Stefan Müller's styles
\usepackage{./styles/merkmalstruktur,./styles/makros.2020,./styles/my-xspace,./styles/article-ex,
./styles/eng-date}

\usepackage{varioref}
\newcommand\refORregion[2]{%
 \vrefpagenum\firstnum{#1}%
 \vrefpagenum\secondnum{#2}%
\ifthenelse{\equal\firstnum\secondnum}%
{\pageref{#1}}%
{\pageref{#1}--\pageref{#2}}%
}

% I am sick of fiddeling arround with babel. I want these shorthands also to work in commands I
% define. St.Mü. 13.08.2020
% e.g. with \iwithini
\usepackage{german}
\selectlanguage{USenglish}

\usepackage{./styles/abbrev}


% Has to be loaded late since otherwise footnotes will not work

%%%%%%%%%%%%%%%%%%%%%%%%%%%%%%%%%%%%%%%%%%%%%%%%%%%%
%%%                                              %%%
%%%           Examples                           %%%
%%%                                              %%%
%%%%%%%%%%%%%%%%%%%%%%%%%%%%%%%%%%%%%%%%%%%%%%%%%%%%
% remove the percentage signs in the following lines
% if your book makes use of linguistic examples
\usepackage{langsci-gb4e} 



% This introduces labels which makes hyperlinks work so that proofreading is easier.
%\makeatletter
%\newcommand{\mex}[1]{\ref{ex-\the\c@chapter-\the\numexpr\c@equation+#1}\relax}
%\newcommand{\eaautolabel}{\label{ex-\the\c@chapter-\the\numexpr\c@equation+1}}
%\makeatother

%\let\oldea\ea
%\def\ea{\oldea\eaautolabel}

%\let\oldeal\eal
%\def\eal{\oldeal\eaautolabel}


% Crossing out text
% uncomment when needed
%\usepackage{ulem}

\usepackage{./styles/additional-langsci-index-shortcuts}

% this is the completely redone avm package
\usepackage{langsci-avm}
\avmsetup{columnsep=.3ex,style=narrow}

\avmdefinecommand{phon}[phon]
  {
    attributes  = \itshape%,
%    delimfactor = 900,
%    delimfall   = 10pt
}

\avmdefinecommand{form}[form]
  {
    attributes  = \itshape%,
%    delimfactor = 900,
%    delimfall   = 10pt
}

% \set was already taken
\avmdefinecommand{avmset}[set]{ attributes=\itshape } % define a new \set command
\avmdefinecommand{list}[list]{ attributes=\itshape } % define a new \list command
   % Note: the label "list" will be output in whatever font is currently active.

% \avm{
% 	[subj  & \1 \\
% 	comps & \2 \- \list*(gap-ss) \\ % Produce a \list
% 	deps  & < \1 > \+ \2
% 	]
% }


\avmdefinecommand{nelist}[ne-list]{ attributes=\itshape } % define a new \nelist command
   % Note: the label "ne-list" will be output in whatever font is currently active.



% https://github.com/langsci/langsci-avm/issues/33#issuecomment-671201576
%\avmsetup{extraskip=0pt}

% if you have to use both langsci-avm and avm
% \usepackage{langsci-avm} % Load pkg with meaning A of conflicting cmd
% \let\lavm\avm % Send the conflicting command to an alternative
% \let\avm\undefined % Send the conflicting cmd to be \undefined
% \usepackage{avm} % Load pkg with meaning B for conf. cmd 

%\let\asort\type*

% remove this, once we really do without avm
%\usepackage{./styles/avm+}

% copied over from avm+.sty
% some relation operators:
%\newcommand{\append}[0]{\ensuremath{\oplus\hspace{.24em}}}
%\newcommand{\shuffle}[0]{\ensuremath{\bigcirc\hspace{.24em}}}

\newcommand{\append}[0]{\ensuremath{\oplus}\xspace}
\newcommand{\shuffle}[0]{\ensuremath{\bigcirc}\xspace}


% command to fontify relations in avms 
\newcommand{\rel}[1]{\texttt{#1}}
%\def\relfont{\slshape}%
%\def\relfont{\ttdefault}%


\let\idx\ibox
\let\avmbox\ibox

% command to fontify attributes in ordinary text
%\newcommand{\attrib}[1]{\textsc{#1}}


% some relation operators:
%\newcommand{\append}[0]{\ensuremath{\oplus\hspace{.24em}}}
%\newcommand{\shuffle}[0]{\ensuremath{\bigcirc\hspace{.24em}}}

\def\relfont{\slshape}%
%
% command to fontify relations in avms 
%\newcommand{\rel}[1]{{\relfont #1}}



% \renewcommand{\tpv}[1]{{\avmjvalfont\itshape #1}}

% % no small caps please
% \renewcommand{\phonshape}[0]{\normalfont\itshape}

% \regAvmFonts

\usepackage{theorem}

\newtheorem{mydefinition}{Def.}
\newtheorem{principle}{Principle}

{\theoremstyle{break}
%\newtheorem{schema}{Schema}
\newtheorem{mydefinition-break}[mydefinition]{Def.}
\newtheorem{principle-break}[principle]{Principle}
}


%% \newcommand{schema}[2]{
%% \begin{minipage}{\textwidth}
%% {\textbf{Schema~\theschema}}]\hspace{.5em}\textbf{(#1)}\\
%% #2
%% \end{minipage}}


% This avoids linebreaks in the Schema
\newcounter{schemacounter}
\makeatletter
\newenvironment{schema}[1][]
  {%
   \refstepcounter{schemacounter}%
   \par\bigskip\noindent
   \minipage{\linewidth}%
   \textbf{Schema~\theschemacounter\hspace{.5em} \ifx&#1&\else(#1)\fi}\par
  }{\endminipage\par\bigskip\@endparenv}%
\makeatother

%\usepackage{subfig}





% Davis Koenig Lexikon

\usepackage{tikz-qtree,tikz-qtree-compat} % Davis Koenig remove

\usepackage{shadow}



\usepackage[english]{isodate} % Andy Lücking
\usepackage[autostyle]{csquotes} % Andy
%\usepackage[autolanguage]{numprint}

%\defaultfontfeatures{
%    Path = /usr/local/texlive/2017/texmf-dist/fonts/opentype/public/fontawesome/ }

%% https://tex.stackexchange.com/a/316948/18561
%\defaultfontfeatures{Extension = .otf}% adds .otf to end of path when font loaded without ext parameter e.g. \newfontfamily{\FA}{FontAwesome} > \newfontfamily{\FA}{FontAwesome.otf}
%\usepackage{fontawesome} % Andy Lücking
\usepackage{pifont} % Andy Lücking -> hand

\usetikzlibrary{decorations.pathreplacing} % Andy Lücking
\usetikzlibrary{matrix} % Andy 
\usetikzlibrary{positioning} % Andy
\usepackage{tikz-3dplot} % Andy

% pragmatics
\usepackage{eqparbox} % Andy
\usepackage{enumitem} % Andy
\usepackage{longtable} % Andy
\usepackage{tabu} % Andy              needs to be loaded before hyperref as of texlive 2020

% tabu-fix
% to make "spread 0pt" work
% -----------------------------
\RequirePackage{etoolbox}
\makeatletter
\patchcmd
	\tabu@startpboxmeasure
	{\bgroup\begin{varwidth}}%
	{\bgroup
	 \iftabu@spread\color@begingroup\fi\begin{varwidth}}%
	{}{}
\def\@tabarray{\m@th\def\tabu@currentgrouptype
    {\currentgrouptype}\@ifnextchar[\@array{\@array[c]}}
%
%%% \pdfelapsedtime bug 2019-12-15
\patchcmd
	\tabu@message@etime
	{\the\pdfelapsedtime}%
	{\pdfelapsedtime}%
	{}{}
%
%
\makeatother
% -----------------------------


% Manfred's packages

%\usepackage{shadow}

\usepackage{tabularx}
\newcolumntype{L}[1]{>{\raggedright\arraybackslash}p{#1}} % linksbündig mit Breitenangabe


% Jong-Bok

%\usepackage{xytree}

\newcommand{\xytree}[2][dummy]{Let's do the tree!}

% seems evil, get rid of it
% defines \ex is incompatible with gb4e
%\usepackage{lingmacros}

% taken from lingmacros:
\makeatletter
% \evnup is used to line up the enumsentence number and an entry along
% the top.  It can take an argument to improve lining up.
\def\evnup{\@ifnextchar[{\@evnup}{\@evnup[0pt]}}

\def\@evnup[#1]#2{\setbox1=\hbox{#2}%
\dimen1=\ht1 \advance\dimen1 by -.5\baselineskip%
\advance\dimen1 by -#1%
\leavevmode\lower\dimen1\box1}
\makeatother


% YK -- CG chapter

%\usepackage{xspace}
\usepackage{bm}
\usepackage{ebproof}


% Antonio Branco, remove this
\usepackage{epsfig}

% now unicode
%\usepackage{alphabeta}





\usepackage{pst-node}


% fmr: additional packages
%\usepackage{amsthm}


% Ash and Steve: LFG
\usepackage{./styles/lfg/dalrymple}

\RequirePackage{graphics}
%\RequirePackage{./styles/lfg/trees}
%% \RequirePackage{avm}
%% \avmoptions{active}
%% \avmfont{\sc}
%% \avmvalfont{\sc}
\RequirePackage{./styles/lfg/lfgmacrosash}

\usepackage{./styles/lfg/glue}

%%%%%%%%%%%%%%%%%%%%%%%%%%%%%%
%% Markup
%%%%%%%%%%%%%%%%%%%%%%%%%%%%%%
\usepackage[normalem]{ulem} % For thinks like strikethrough, using \sout

% \newcommand{\high}[1]{\textbf{#1}} % highlighted text
\newcommand{\high}[1]{\textit{#1}} % highlighted text
%\newcommand{\term}[1]{\textit{#1}\/} % technical term
\newcommand{\qterm}[1]{``{#1}''} % technical term, quotes
%\newcommand{\trns}[1]{\strut `#1'} % translation in glossed example
\newcommand{\trnss}[1]{\strut \phantom{\sqz{}} `#1'} % translation in ungrammatical glossed example
\newcommand{\ttrns}[1]{(`#1')} % an in-text translation of a word
\newcommand{\LFGfeat}[1]{\mbox{\textsc{\MakeLowercase{#1}}}}     % feature name
%\newcommand{\val}[1]{\mbox{\textsc{\MakeLowercase{#1}}}}    % f-structure value
\newcommand{\featt}[1]{\mbox{\textsc{\MakeLowercase{#1}}}}     % feature name
\newcommand{\vall}[1]{\mbox{\textsc{\textup{\MakeLowercase{#1}}}}}    % f-structure value
\newcommand{\mg}[1]{\mbox{\textsc{\MakeLowercase{#1}}}}    % morphological gloss
%\newcommand{\word}[1]{\textit{#1}}       % mention of word
\providecommand{\kstar}[1]{{#1}\ensuremath{^*}}
\providecommand{\kplus}[1]{{#1}\ensuremath{^+}}
\newcommand{\template}[1]{@\textsc{\MakeLowercase{#1}}}
\newcommand{\templaten}[1]{\textsc{\MakeLowercase{#1}}}
\newcommand{\templatenn}[1]{\MakeUppercase{#1}}
\newcommand{\tempeq}{\ensuremath{=}}
\newcommand{\predval}[1]{\ensuremath{\langle}\textsc{#1}\ensuremath{\rangle}}
\newcommand{\predvall}[1]{{\rm `#1'}}
\newcommand{\lfgfst}[1]{\ensuremath{#1\,}}
\newcommand{\scare}[1]{``#1''} % scare quotes
\newcommand{\bracket}[1]{\ensuremath{\left\langle\mathit{#1}\right\rangle}}
\newcommand{\sectionw}[1][]{Section#1} % section word: for cap/non-cap
\newcommand{\tablew}[1][]{Table#1} % table word: for cap/non-cap
\newcommand{\lfgglue}{LFG+Glue}
\newcommand{\hpsgglue}{HPSG+Glue}
\newcommand{\gs}{GS}
%\newcommand{\func}[1]{\ensuremath{\mathbf{#1}}}
\newcommand{\func}[1]{\textbf{#1}}
\renewcommand{\glue}{Glue}
%\newcommand{\exr}[1]{(\ref{ex:#1}}
\newcommand{\exra}[1]{(\ref{ex:#1})}


%%%%%%%%%%%%%%%%%%%%%%%%%%%%%%
% Notation
%\newcommand{\xbar}[1]{$_{\mbox{\textsc{#1}$^{\raisebox{1ex}{}}$}}$}
\newcommand{\xprime}[2][]{\textup{\mbox{{#2}\ensuremath{^\prime_{\hspace*{-.0em}\mbox{\footnotesize\ensuremath{\mathit{#1}}}}}}}}
\providecommand{\xzero}[2][]{#2\ensuremath{^0_{\mbox{\footnotesize\ensuremath{\mathit{#1}}}}}}



\let\leftangle\langle
\let\rightangle\rangle

%\newcommand{\pslabel}[1]{}

% remove when finished
\usepackage{proofread}
  %add all your local new commands to this file

% The orchid-id is specified and then extracted by scripts for zenodo.
\newcommand{\orcid}[1]{} 

% do not show the chapter number. It is redundant, since most references to figures are within the
% same chapter.
\renewcommand{\thefigure}{\arabic{figure}}


% Don't do this at home. I do not like the smaller font for captions.
% I just removed loading the caption packege in langscibook.cls
%% \captionsetup{%
%% font={%
%% stretch=1%.8%
%% ,normalsize%,small%
%% },%
%% width=.8\textwidth
%% }

\makeatletter
\def\blx@maxline{77}
\makeatother


\let\citew\citet

\newcommand{\page}{}

\newcommand{\todostefan}[1]{\todo[color=orange!80]{\footnotesize #1}\xspace}
\newcommand{\todosatz}[1]{\todo[color=red!40]{\footnotesize #1}\xspace}

\newcommand{\inlinetodostefan}[1]{\todo[color=green!40,inline]{\footnotesize #1}\xspace}

\newcommand{\inlinetodoopt}[1]{\todo[color=green!40,inline]{\footnotesize #1}\xspace}
\newcommand{\inlinetodoobl}[1]{\todo[color=red!40,inline]{\footnotesize #1}\xspace}

\newcommand{\itd}[1]{\inlinetodoobl{#1}}
\newcommand{\itdobl}[1]{\inlinetodoobl{#1}}
\newcommand{\itdopt}[1]{\inlinetodoopt{#1}}

\newcommand{\itdsecond}[1]{}

\newcommand{\itddone}[1]{}
%\let\itddone\itdopt
\newcommand{\LATER}[1]{}



% A. Red: Simple typos, errors in the AVMs (only a couple) to take care of on the editorial side, no need to contact the authors
% B.: Green: Wording changes which do not necessarily require authors’ approval, but are not just typos/errors
% C.: Blue: Comments to the author that they don’t have to take care of, but after all, the authors might be interested to have the comments for future revisions. 
% D.: Purple: Comments to the editors about something we need to keep in mind or do. Nothing for you

\newcommand{\colorcodingexplanation}{\todo[color=green!40,inline]{%
Explanation of colors of bubbles and text:\\
A.: Red: Things that have to be fixed/commented upon.\\
B.: Green: optional comments\\
C.: Blue: Comments to the author that they don’t have to take care of, but after all, the authors
might be interested to have the comments for future revisions.\\
Explanation of colors of text:\\
Red: newly added material (crossreferences to other chapters and other references)\\
Orange: changed material, please check\\
Blue: suggestions for deletion\\
Please also check margin notes.
}}
% D.: Purple: Comments to the editors about something we need to keep in mind or do. Nothing for you


\newcommand{\itdgreen}[1]{\todo[color=green!40,inline]{\footnotesize #1}\xspace}
\newcommand{\itdblue}[1]{\todo[color=blue!40,inline]{\footnotesize #1}\xspace}

% for editing, remove later
\usepackage{xcolor}
\newcommand{\added}[1]{{\red #1}}
\newcommand{\addedthis}{\todostefan{added this}}

\newcommand{\changed}[1]{\textcolor{orange}{#1}}
\newcommand{\deleted}[1]{\textcolor{blue}{#1}}


% \newcommand{\addpages}{\todostefan{add pages}}
% %\newcommand{\iaddpages}{\inlinetodoobl{add pages}}
% \newcommand{\iaddpages}{\yel[add pages]{pages}\xspace}
% \newcommand{\addref}{\todostefan{add reference}}
% \newcommand{\inlineaddpages}{\inlinetodostefan{add pages}}
% \newcommand{\addglosses}{\todostefan{add glosses}}

\newcommand{\addpages}{\xspace}%np
\newcommand{\iaddpages}{\xspace}%islands und understudied languages
\newcommand{\addref}{\xspace}
\newcommand{\inlineaddpages}{\xspace}
% not used \newcommand{\addglosses}{}


%\newcommand{\spacebr}{\hphantom{[}}

\newcommand{\danishep}{\jambox{(\ili{Danish})}}
\newcommand{\english}{\jambox{(\ili{English})}}
\newcommand{\german}{\jambox{(\ili{German})}}
\newcommand{\yiddish}{\jambox{(\ili{Yiddish})}}
\newcommand{\welsh}{\jambox{(\ili{Welsh})}}

% Cite and cross-reference other chapters
\newcommand{\crossrefchaptert}[2][]{\citet*[#1]{chapters/#2}, Chapter~\ref{chap-#2} of this volume} 
\newcommand{\crossrefchapterp}[2][]{(\citealp*[#1]{chapters/#2}, Chapter~\ref{chap-#2} of this volume)}
\newcommand{\crossrefchapteralt}[2][]{\citealt*[#1]{chapters/#2}, Chapter~\ref{chap-#2} of this volume}
\newcommand{\crossrefchapteralp}[2][]{\citealp*[#1]{chapters/#2}, Chapter~\ref{chap-#2} of this volume}

\newcommand{\crossrefcitet}[2][]{\citet*[#1]{chapters/#2}} 
\newcommand{\crossrefcitep}[2][]{\citep*[#1]{chapters/#2}}
\newcommand{\crossrefcitealt}[2][]{\citealt*[#1]{chapters/#2}}
\newcommand{\crossrefcitealp}[2][]{\citealp*[#1]{chapters/#2}}


% example of optional argument:
% \crossrefchapterp[for something, see:]{name}
% gives: (for something, see: Author 2018, Chapter~X of this volume)



\let\crossrefchapterw\crossrefchaptert



% Davis Koenig

\let\ig=\textsc
\let\tc=\textcolor

% evolution, Flickinger, Pollard, Wasow

\let\citeNP\citet

% Adam P

%\newcommand{\toappear}{Forthcoming}
\newcommand{\pg}[1]{p.\,#1}
\renewcommand{\implies}{\rightarrow}

\newcommand*{\rref}[1]{(\ref{#1})}
\newcommand*{\aref}[1]{(\ref{#1}a)}
\newcommand*{\bref}[1]{(\ref{#1}b)}
\newcommand*{\cref}[1]{(\ref{#1}c)}

\newcommand{\msadam}{.}
\newcommand{\morsyn}[1]{\textsc{#1}}

\newcommand{\aux}{\textsc{aux}\xspace}

\newcommand{\nom}{\morsyn{nom}}
\newcommand{\acc}{\morsyn{acc}}
\newcommand{\dat}{\morsyn{dat}}
\newcommand{\gen}{\morsyn{gen}}
\newcommand{\ins}{\morsyn{ins}}
%\newcommand{\aploc}{\morsyn{loc}}
\newcommand{\voc}{\morsyn{voc}}
\newcommand{\ill}{\morsyn{ill}}
\renewcommand{\inf}{\morsyn{inf}}
\newcommand{\passprc}{\morsyn{passp}}

%\newcommand{\Nom}{\msadam\nom}
%\newcommand{\Acc}{\msadam\acc}
%\newcommand{\Dat}{\msadam\dat}
%\newcommand{\Gen}{\msadam\gen}
\newcommand{\Ins}{\msadam\ins}
\newcommand{\Loc}{\msadam\loc}
\newcommand{\Voc}{\msadam\voc}
\newcommand{\Ill}{\msadam\ill}
\newcommand{\PassP}{\msadam\passprc}

\newcommand{\Aux}{\textsc{aux}}

%\newcommand{\princ}[1]{\textnormal{\textsc{#1}}} % for constraint names
\newcommand{\princ}[1]{\textnormal{#1}} % for constraint names
\newcommand{\notion}[1]{\emph{#1}}
\renewcommand{\path}[1]{\textnormal{\textsc{#1}}}
\newcommand{\ftype}[1]{\textit{#1}}
\newcommand{\fftype}[1]{{\scriptsize\textit{#1}}}
\newcommand{\la}{$\langle$}
\newcommand{\ra}{$\rangle$}
%\newcommand{\argst}{\path{arg-st}}
\newcommand{\phtm}[1]{\setbox0=\hbox{#1}\hspace{\wd0}}
\newcommand{\prep}[1]{\setbox0=\hbox{#1}\hspace{-1\wd0}#1}


% Rui

\newcommand{\spc}[0]{\hspace{-1pt}\underline{\hspace{6pt}}\,}
\newcommand{\spcs}[0]{\hspace{-1pt}\underline{\hspace{6pt}}\,\,}
\newcommand{\bad}[1]{\leavevmode\llap{#1}}
\newcommand{\COMMENT}[1]{}


% Rui coordination
\newcommand{\subl}[1]{$_{\scriptstyle \textsc{#1}}$}



% Andy Lücking gesture.tex
\newcommand{\Pointing}{\ding{43}}
% Giotto: "Meeting of Joachim and Anne at the Golden Gate" - 1305-10 
\definecolor{GoldenGate1}{rgb}{.13,.09,.13} % Dress of woman in black
\definecolor{GoldenGate2}{rgb}{.94,.94,.91} % Bridge
\definecolor{GoldenGate3}{rgb}{.06,.09,.22} % Blue sky
\definecolor{GoldenGate4}{rgb}{.94,.91,.87} % Dress of woman with shawl
\definecolor{GoldenGate5}{rgb}{.52,.26,.26} % Joachim's robe
\definecolor{GoldenGate6}{rgb}{.65,.35,.16} % Anne's robe
\definecolor{GoldenGate7}{rgb}{.91,.84,.42} % Joachim's halo

\makeatletter
\newcommand{\@Depth}{1} % x-dimension, to front
\newcommand{\@Height}{1} % z-dimension, up
\newcommand{\@Width}{1} % y-dimension, rightwards
%\GGS{<x-start>}{<y-start>}{<z-top>}{<z-bottom>}{<Farbe>}{<x-width>}{<y-depth>}{<opacity>}
\newcommand{\GGS}[9][]{%
\coordinate (O) at (#2-1,#3-1,#5);
\coordinate (A) at (#2-1,#3-1+#7,#5);
\coordinate (B) at (#2-1,#3-1+#7,#4);
\coordinate (C) at (#2-1,#3-1,#4);
\coordinate (D) at (#2-1+#8,#3-1,#5);
\coordinate (E) at (#2-1+#8,#3-1+#7,#5);
\coordinate (F) at (#2-1+#8,#3-1+#7,#4);
\coordinate (G) at (#2-1+#8,#3-1,#4);
\draw[draw=black, fill=#6, fill opacity=#9] (D) -- (E) -- (F) -- (G) -- cycle;% Front
\draw[draw=black, fill=#6, fill opacity=#9] (C) -- (B) -- (F) -- (G) -- cycle;% Top
\draw[draw=black, fill=#6, fill opacity=#9] (A) -- (B) -- (F) -- (E) -- cycle;% Right
}
\makeatother


% pragmatics
\newcommand{\speaking}[1]{\eqparbox{name}{\textsc{\lowercase{#1}\space}}}
\newcommand{\alname}[1]{\eqparbox{name}{\textsc{\lowercase{#1}}}}
\newcommand{\HPSGTTR}{HPSG$_{\text{TTR}}$\xspace}

\newcommand{\ttrtype}[1]{\textit{#1}}
\newcommand{\avmel}{\q<\quad\q>} %% shortcut for empty lists in AVM
\newcommand{\ttrmerge}{\ensuremath{\wedge_{\textit{merge}}}}
\newcommand{\Cat}[2][0.1pt]{%
  \begin{scope}[y=#1,x=#1,yscale=-1, inner sep=0pt, outer sep=0pt]
   \path[fill=#2,line join=miter,line cap=butt,even odd rule,line width=0.8pt]
  (151.3490,307.2045) -- (264.3490,307.2045) .. controls (264.3490,291.1410) and (263.2021,287.9545) .. (236.5990,287.9545) .. controls (240.8490,275.2045) and (258.1242,244.3581) .. (267.7240,244.3581) .. controls (276.2171,244.3581) and (286.3490,244.8259) .. (286.3490,264.2045) .. controls (286.3490,286.2045) and (323.3717,321.6755) .. (332.3490,307.2045) .. controls (345.7277,285.6390) and (309.3490,292.2151) .. (309.3490,240.2046) .. controls (309.3490,169.0514) and (350.8742,179.1807) .. (350.8742,139.2046) .. controls (350.8742,119.2045) and (345.3490,116.5037) .. (345.3490,102.2045) .. controls (345.3490,83.3070) and (361.9972,84.4036) .. (358.7581,68.7349) .. controls (356.5206,57.9117) and (354.7696,49.2320) .. (353.4652,36.1439) .. controls (352.5396,26.8573) and (352.2445,16.9594) .. (342.5985,17.3574) .. controls (331.2650,17.8250) and (326.9655,37.7742) .. (309.3490,39.2045) .. controls (291.7685,40.6320) and (276.7783,24.2380) .. (269.9740,26.5795) .. controls (263.2271,28.9013) and (265.3490,47.2045) .. (269.3490,60.2045) .. controls (275.6359,80.6368) and (289.3490,107.2045) .. (264.3490,111.2045) .. controls (239.3490,115.2045) and (196.3490,119.2045) .. (165.3490,160.2046) .. controls (134.3490,201.2046) and (135.4934,249.3212) .. (123.3490,264.2045) .. controls (82.5907,314.1553) and (40.8239,293.6463) .. (40.8239,335.2045) .. controls (40.8239,353.8102) and (72.3490,367.2045) .. (77.3490,361.2045) .. controls (82.3490,355.2045) and (34.8638,337.3259) .. (87.9955,316.2045) .. controls (133.3871,298.1601) and   (137.4391,294.4766) .. (151.3490,307.2045) -- cycle;
\end{scope}%
}
%% leicht modifiziert nach Def. von Sebastian Nordhoff:
% \newcommand{\lueckingbox}[3]{\parbox[t][][t]{0.7cm}{\raggedright
%     \strut#1}\parbox[t][][t]{7.7cm}{\strut#2}\parbox[t][][t]{3cm}{\raggedright\strut#3}\bigskip\\}
\newcommand{\lueckingbox}[3]{\parbox[t][][t]{0.7cm}{\raggedright
    \strut\vspace*{-\baselineskip}\newline#1}\parbox[t][][t]{7.7cm}{\strut\vspace*{-\baselineskip}\newline#2}\parbox[t][][t]{3cm}{\raggedright\strut\vspace*{-\baselineskip}\newline#3}\bigskip\\}




% KdK
\newcommand{\smiley}{:)}

\renewbibmacro*{index:name}[5]{%
  \usebibmacro{index:entry}{#1}
    {\iffieldundef{usera}{}{\thefield{usera}\actualoperator}\mkbibindexname{#2}{#3}{#4}{#5}}}

% \newcommand{\noop}[1]{}

% chngcntr.sty otherwise gives error that these are already defined
%\let\counterwithin\relax
%\let\counterwithout\relax

% the space of a left bracket for glossings
\newcommand{\LB}{\hphantom{[}}

\newcommand{\LF}{\mbox{$[\![$}}

\newcommand{\RF}{\mbox{$]\!]_F$}}

\newcommand{\RT}{\mbox{$]\!]_T$}}





% Manfred's

\newcommand{\kommentar}[1]{}

\newcommand{\bsp}[1]{\emph{#1}}
\newcommand{\bspT}[2]{\bsp{#1} `#2'}
\newcommand{\bspTL}[3]{\bsp{#1} (lit.: #2) `#3'}

\newcommand{\noidi}{§}

\newcommand{\refer}[1]{(\ref{#1})}

%\newcommand{\avmtype}[1]{\multicolumn{2}{l}{\type{#1}}}
\newcommand{\attr}[1]{\textsc{#1}}

%\newcommand{\srdefault}{\mbox{\begin{tabular}{@{}c@{}}{\large <}\\[-1.5ex]$\sqcap$\end{tabular}}}
\newcommand{\srdefault}{$\stackrel{<}{\sqcap}$}


%% \newcommand{\myappcolumn}[2]{
%% \begin{minipage}[t]{#1}#2\end{minipage}
%% }

%% \newcommand{\appc}[1]{\myappcolumn{3.7cm}{#1}}


% Jong-Bok


% clean that up and do not use \def (killing other stuff defined before)
%\if 0
%\newcommand\DEL{\textsc{del}}
%\newcommand\del{\textsc{del}}

\newcommand\conn{\textsc{conn}}
\newcommand\CONN{\textsc{conn}}
\newcommand\CONJ{\textsc{conj}}
\newcommand\LITE{\textsc{lex}}
\newcommand\lite{\textsc{lex}}
\newcommand\HON{\textsc{hon}}

%\newcommand\CAUS{\textsc{caus}}
%\newcommand\PASS{\textsc{pass}}
\newcommand\NPST{\textsc{npst}}
%\newcommand\COND{\textsc{cond}}



\newcommand\hdlite{\textsc{head-lex construction}}
\newcommand\hdlight{\textsc{head-light} Schema}
\newcommand\NFORM{\textsc{nform}}

\newcommand\RELS{\textsc{rels}}
%\newcommand\TENSE{\textsc{tense}}


%\newcommand\ARG{\textsc{arg}}
\newcommand\ARGs{\textsc{arg0}}
\newcommand\ARGa{\textsc{arg}}
\newcommand\ARGb{\textsc{arg2}}
\newcommand\TPC{\textsc{top}}
%\newcommand\PROG{\textsc{prog}}

\newcommand\LIGHT{\textsc{light}\xspace}
\newcommand\pst{\textsc{pst}}
%\newcommand\PAST{\textsc{pst}}
%\newcommand\DAT{\textsc{dat}}
%\newcommand\CONJ{\textsc{conj}}
\newcommand\nominal{\textsc{nominal}}
\newcommand\NOMINAL{\textsc{nominal}}
\newcommand\VAL{\textsc{val}}
%\newcommand\val{\textsc{val}}
\newcommand\MODE{\textsc{mode}}
\newcommand\RESTR{\textsc{restr}}
\newcommand\SIT{\textsc{sit}}
\newcommand\ARG{\textsc{arg}}
\newcommand\RELN{\textsc{rel}}
%\newcommand\REL{\textsc{rel}}
%\newcommand\RELS{\textsc{rels}}
%\newcommand\arg-st{\textsc{arg-st}}
\newcommand\xdel{\textsc{xdel}}
\newcommand\zdel{\textsc{zdel}}
\newcommand\sug{\textsc{sug}}
%\newcommand\IMP{\textsc{imp}}
%\newcommand\conn{\textsc{conn}}
%\newcommand\CONJ{\textsc{conj}}
%\newcommand\HON{\textsc{hon}}
\newcommand\BN{\textsc{bn}}
\newcommand\bn{\textsc{bn}}
\newcommand\pres{\textsc{pres}}
\newcommand\PRES{\textsc{pres}}
\newcommand\prs{\textsc{pres}}
%\newcommand\PRS{\textsc{pres}}
\newcommand\agt{\textsc{agt}}
%\newcommand\DEL{\textsc{del}}
%\newcommand\PRED{\textsc{pred}}
\newcommand\AGENT{\textsc{agent}}
\newcommand\THEME{\textsc{theme}}
%\newcommand\AUX{\textsc{aux}}
%\newcommand\THEME{\textsc{theme}}
%\newcommand\PL{\textsc{pl}}
\newcommand\SRC{\textsc{src}}
\newcommand\src{\textsc{src}}
\newcommand{\FORMjb}{\textsc{form}}
\newcommand{\formjb}{\FORM}
\newcommand\GCASE{\textsc{gcase}}
\newcommand\gcase{\textsc{gcase}}
\newcommand\SCASE{\textsc{scase}}
\newcommand\PHON{\textsc{phon}}
%\newcommand\SS{\textsc{ss}}
\newcommand\SYN{\textsc{syn}}
%\newcommand\LOC{\textsc{loc}}
\newcommand\MOD{\textsc{mod}}
\newcommand\INV{\textsc{inv}}
%\newcommand\L{\textsc{l}}
%\newcommand\CASE{\textsc{case}}
\newcommand\SPR{\textsc{spr}}
\newcommand\COMPS{\textsc{comps}}
%\newcommand\comps{\textsc{comps}}
\newcommand\SEM{\textsc{sem}}
\newcommand\CONT{\textsc{cont}}
\newcommand\SUBCAT{\textsc{subcat}}
\newcommand\CAT{\textsc{cat}}
%\newcommand\C{\textsc{c}}
%\newcommand\SUBJ{\textsc{subj}}
\newcommand\subjjb{\textsc{subj}}
%\newcommand\SLASH{\textsc{slash}}
\newcommand\LOCAL{\textsc{local}}
%\newcommand\ARG-ST{\textsc{arg-st}}
%\newcommand\AGR{\textsc{agr}}
\newcommand\PER{\textsc{per}}
%\newcommand\NUM{\textsc{num}}
%\newcommand\IND{\textsc{ind}}
\newcommand\VFORM{\textsc{vform}}
\newcommand\PFORM{\textsc{pform}}
\newcommand\decl{\textsc{decl}}
%\newcommand\loc{\textsc{loc   }}
% \newcommand\   {\textsc{  }}

%\newcommand\NEG{\textsc{neg}}
\newcommand\FRAMES{\textsc{frames}}
%\newcommand\REFL{\textsc{refl}}

\newcommand\MKG{\textsc{mkg}}

%\newcommand\BN{\textsc{bn}}
\newcommand\HD{\textsc{hd}}
\newcommand\NP{\textsc{np}}
\newcommand\PF{\textsc{pf}}
%\newcommand\PL{\textsc{pl}}
\newcommand\PP{\textsc{pp}}
%\newcommand\SS{\textsc{ss}}
\newcommand\VF{\textsc{vf}}
\newcommand\VP{\textsc{vp}}
%\newcommand\bn{\textsc{bn}}
\newcommand\cl{\textsc{cl}}
%\newcommand\pl{\textsc{pl}}
\newcommand\Wh{\ital{Wh}}
%\newcommand\ng{\textsc{neg}}
\newcommand\wh{\ital{wh}}
%\newcommand\ACC{\textsc{acc}}
%\newcommand\AGR{\textsc{agr}}
\newcommand\AGT{\textsc{agt}}
\newcommand\ARC{\textsc{arc}}
%\newcommand\ARG{\textsc{arg}}
\newcommand\ARP{\textsc{arc}}
%\newcommand\AUX{\textsc{aux}}
%\newcommand\CAT{\textsc{cat}}
%\newcommand\COP{\textsc{cop}}
%\newcommand\DAT{\textsc{dat}}
\newcommand\NEWCOMMAND{\textsc{def}}
%\newcommand\DEL{\textsc{del}}
\newcommand\DOM{\textsc{dom}}
\newcommand\DTR{\textsc{dtr}}
%\newcommand\FUT{\textsc{fut}}
\newcommand\GAP{\textsc{gap}}
%\newcommand\GEN{\textsc{gen}}
%\newcommand\HON{\textsc{hon}}
%\newcommand\IMP{\textsc{imp}}
%\newcommand\IND{\textsc{ind}}
%\newcommand\INV{\textsc{inv}}
\newcommand\LEX{\textsc{lex}}
\newcommand\Lex{\textsc{lex}}
%\newcommand\LOC{\textsc{loc}}
%\newcommand\MOD{\textsc{mod}}
\newcommand\MRK{{\nr MRK}}
%\newcommand\NEG{\textsc{neg}}
\newcommand\NEW{\textsc{new}}
%\newcommand\NOM{\textsc{nom}}
%\newcommand\NUM{\textsc{num}}
%\newcommand\PER{\textsc{per}}
%\newcommand\PST{\textsc{pst}}
\newcommand\QUE{\textsc{que}}
%\newcommand\REL{\textsc{rel}}
\newcommand\SEL{\textsc{sel}}
%\newcommand\SEM{\textsc{sem}}
%\newcommand\SIT{\textsc{arg0}}
%\newcommand\SPR{\textsc{spr}}
%\newcommand\SRC{\textsc{src}}
\newcommand\SUG{\textsc{sug}}
%\newcommand\SYN{\textsc{syn}}
%\newcommand\TPC{\textsc{top}}
%\newcommand\VAL{\textsc{val}}
%\newcommand\acc{\textsc{acc}}
%\newcommand\agt{\textsc{agt}}
\newcommand\cop{\textsc{cop}}
%\newcommand\dat{\textsc{dat}}
\newcommand\foc{\textsc{focus}}
%\newcommand\FOC{\textsc{focus}}
\newcommand\fut{\textsc{fut}}
\newcommand\hon{\textsc{hon}}
\newcommand\imp{\textsc{imp}}
\newcommand\kes{\textsc{kes}}
%\newcommand\lex{\textsc{lex}}
%\newcommand\loc{\textsc{loc}}
\newcommand\mrk{{\nr MRK}}
%\newcommand\nom{\textsc{nom}}
%\newcommand\num{\textsc{num}}
\newcommand\plu{\textsc{plu}}
\newcommand\pne{\textsc{pne}}
%\newcommand\pst{\textsc{pst}}
\newcommand\pur{\textsc{pur}}
%\newcommand\que{\textsc{que}}
%\newcommand\src{\textsc{src}}
%\newcommand\sug{\textsc{sug}}
\newcommand\tpc{\textsc{top}}
%\newcommand\utt{\textsc{utt}}
%\newcommand\val{\textsc{val}}
%% \newcommand\LITE{\textsc{lex}}
%% \newcommand\PAST{\textsc{pst}}
%% \newcommand\POSP{\textsc{pos}}
%% \newcommand\PRS{\textsc{pres}}
%% \newcommand\mod{\textsc{mod}}%
%% \newcommand\newuse{{`kes'}}
%% \newcommand\posp{\textsc{pos}}
%% \newcommand\prs{\textsc{pres}}
%% \newcommand\psp{{\it en\/}}
%% \newcommand\skes{\textsc{kes}}
%% \newcommand\CASE{\textsc{case}}
%% \newcommand\CASE{\textsc{case}}
%% \newcommand\COMP{\textsc{comp}}
%% \newcommand\CONJ{\textsc{conj}}
%% \newcommand\CONN{\textsc{conn}}
%% \newcommand\CONT{\textsc{cont}}
%% \newcommand\DECL{\textsc{decl}}
%% \newcommand\FOCUS{\textsc{focus}}
%% %\newcommand\FORM{\textsc{form}} duplicate
%% \newcommand\FREL{\textsc{frel}}
%% \newcommand\GOAL{\textsc{goal}}
\newcommand\HEAD{\textsc{head}}
%% \newcommand\INDEX{\textsc{ind}}
%% \newcommand\INST{\textsc{inst}}
%% \newcommand\MODE{\textsc{mode}}
%% \newcommand\MOOD{\textsc{mood}}
%% \newcommand\NMLZ{\textsc{nmlz}}
%% \newcommand\PHON{\textsc{phon}}
%% \newcommand\PRED{\textsc{pred}}
%% %\newcommand\PRES{\textsc{pres}}
%% \newcommand\PROM{\textsc{prom}}
%% \newcommand\RELN{\textsc{pred}}
%% \newcommand\RELS{\textsc{rels}}
%% \newcommand\STEM{\textsc{stem}}
%% \newcommand\SUBJ{\textsc{subj}}
%% \newcommand\XARG{\textsc{xarg}}
%% \newcommand\bse{{\it bse\/}}
%% \newcommand\case{\textsc{case}}
%% \newcommand\caus{\textsc{caus}}
%% \newcommand\comp{\textsc{comp}}
%% \newcommand\conj{\textsc{conj}}
%% \newcommand\conn{\textsc{conn}}
%% \newcommand\decl{\textsc{decl}}
%% \newcommand\fin{{\it fin\/}}
%% %\newcommand\form{\textsc{form}}
%% \newcommand\gend{\textsc{gend}}
%% \newcommand\inf{{\it inf\/}}
%% \newcommand\mood{\textsc{mood}}
%% \newcommand\nmlz{\textsc{nmlz}}
%% \newcommand\pass{\textsc{pass}}
%% \newcommand\past{\textsc{past}}
%% \newcommand\perf{\textsc{perf}}
%% \newcommand\pln{{\it pln\/}}
%% \newcommand\pred{\textsc{pred}}


%% %\newcommand\pres{\textsc{pres}}
%% \newcommand\proc{\textsc{proc}}
%% \newcommand\nonfin{{\it nonfin\/}}
%% \newcommand\AGENT{\textsc{agent}}
%% \newcommand\CFORM{\textsc{cform}}
%% %\newcommand\COMPS{\textsc{comps}}
%% \newcommand\COORD{\textsc{coord}}
%% \newcommand\COUNT{\textsc{count}}
%% \newcommand\EXTRA{\textsc{extra}}
%% \newcommand\GCASE{\textsc{gcase}}
%% \newcommand\GIVEN{\textsc{given}}
%% \newcommand\LOCAL{\textsc{local}}
%% \newcommand\NFORM{\textsc{nform}}
%% \newcommand\PFORM{\textsc{pform}}
%% \newcommand\SCASE{\textsc{scase}}
%% \newcommand\SLASH{\textsc{slash}}
%% \newcommand\SLASH{\textsc{slash}}
%% \newcommand\THEME{\textsc{theme}}
%% \newcommand\TOPIC{\textsc{topic}}
%% \newcommand\VFORM{\textsc{vform}}
%% \newcommand\cause{\textsc{cause}}
%% %\newcommand\comps{\textsc{comps}}
%% \newcommand\gcase{\textsc{gcase}}
%% \newcommand\itkes{{\it kes\/}}
%% \newcommand\pass{{\it pass\/}}
%% \newcommand\vform{\textsc{vform}}
%% \newcommand\CCONT{\textsc{c-cont}}
%% \newcommand\GN{\textsc{given-new}}
%% \newcommand\INFO{\textsc{info-st}}
%% \newcommand\ARG-ST{\textsc{arg-st}}
%% \newcommand\SUBCAT{\textsc{subcat}}
%% \newcommand\SYNSEM{\textsc{synsem}}
%% \newcommand\VERBAL{\textsc{verbal}}
%% \newcommand\arg-st{\textsc{arg-st}}
%% \newcommand\plain{{\it plain}\/}
%% \newcommand\propos{\textsc{propos}}
%% \newcommand\ADVERBIAL{\textsc{advl}}
%% \newcommand\HIGHLIGHT{\textsc{prom}}
%% \newcommand\NOMINAL{\textsc{nominal}}

\newenvironment{myavm}{\begingroup\avmvskip{.1ex}
  \selectfont\begin{avm}}%
{\end{avm}\endgroup\medskip}
\newcommand\pfix{\vspace{-5pt}}


\newcommand{\jbsub}[1]{\lower4pt\hbox{\small #1}}
\newcommand{\jbssub}[1]{\lower4pt\hbox{\small #1}}
\newcommand\jbtr{\underbar{\ \ \ }\ }


%\fi

% cl

\newcommand{\delphin}{\textsc{delph-in}}


% YK -- CG chapter

\newcommand{\grey}[1]{\colorbox{mycolor}{#1}}
\definecolor{mycolor}{gray}{0.8}

\newcommand{\GQU}[2]{\raisebox{1.6ex}{\ensuremath{\rotatebox{180}{\textbf{#1}}_{\scalebox{.7}{\textbf{#2}}}}}}

\newcommand{\SetInfLen}{\setpremisesend{0pt}\setpremisesspace{10pt}\setnamespace{0pt}}

\newcommand{\pt}[1]{\ensuremath{\mathsf{#1}}}
\newcommand{\ptv}[1]{\ensuremath{\textsf{\textsl{#1}}}}

\newcommand{\sv}[1]{\ensuremath{\bm{\mathcal{#1}}}}
\newcommand{\sX}{\sv{X}}
\newcommand{\sF}{\sv{F}}
\newcommand{\sG}{\sv{G}}

\newcommand{\syncat}[1]{\textrm{#1}}
\newcommand{\syncatVar}[1]{\ensuremath{\mathit{#1}}}

\newcommand{\RuleName}[1]{\textrm{#1}}

\newcommand{\SemTyp}{\textsf{Sem}}

\newcommand{\E}{\ensuremath{\bm{\epsilon}}\xspace}

\newcommand{\greeka}{\upalpha}
\newcommand{\greekb}{\upbeta}
\newcommand{\greekd}{\updelta}
\newcommand{\greekp}{\upvarphi}
\newcommand{\greekr}{\uprho}
\newcommand{\greeks}{\upsigma}
\newcommand{\greekt}{\uptau}
\newcommand{\greeko}{\upomega}
\newcommand{\greekz}{\upzeta}

\newcommand{\Lemma}{\ensuremath{\hskip.5em\vdots\hskip.5em}\noLine}
\newcommand{\LemmaAlt}{\ensuremath{\hskip.5em\vdots\hskip.5em}}

\newcommand{\I}{\iota}

\newcommand{\sem}{\ensuremath}

\newcommand{\NoSem}{%
\renewcommand{\LexEnt}[3]{##1; \syncat{##3}}
\renewcommand{\LexEntTwoLine}[3]{\renewcommand{\arraystretch}{.8}%
\begin{array}[b]{l} ##1;  \\ \syncat{##3} \end{array}}
\renewcommand{\LexEntThreeLine}[3]{\renewcommand{\arraystretch}{.8}%
\begin{array}[b]{l} ##1; \\ \syncat{##3} \end{array}}}

\newcommand{\hypml}[2]{\left[\!\!#1\!\!\right]^{#2}}

%%%%for bussproof
\def\defaultHypSeparation{\hskip0.1in}
\def\ScoreOverhang{0pt}

\newcommand{\MultiLine}[1]{\renewcommand{\arraystretch}{.8}%
\ensuremath{\begin{array}[b]{l} #1 \end{array}}}

\newcommand{\MultiLineMod}[1]{%
\ensuremath{\begin{array}[t]{l} #1 \end{array}}}

\newcommand{\hypothesis}[2]{[ #1 ]^{#2}}

\newcommand{\LexEnt}[3]{#1; \ensuremath{#2}; \syncat{#3}}

\newcommand{\LexEntTwoLine}[3]{\renewcommand{\arraystretch}{.8}%
\begin{array}[b]{l} #1; \\ \ensuremath{#2};  \syncat{#3} \end{array}}

\newcommand{\LexEntThreeLine}[3]{\renewcommand{\arraystretch}{.8}%
\begin{array}[b]{l} #1; \\ \ensuremath{#2}; \\ \syncat{#3} \end{array}}

\newcommand{\LexEntFiveLine}[5]{\renewcommand{\arraystretch}{.8}%
\begin{array}{l} #1 \\ #2; \\ \ensuremath{#3} \\ \ensuremath{#4}; \\ \syncat{#5} \end{array}}

\newcommand{\LexEntFourLine}[4]{\renewcommand{\arraystretch}{.8}%
\begin{array}{l} \pt{#1} \\ \pt{#2}; \\ \syncat{#4} \end{array}}

\newcommand{\ManySomething}{\renewcommand{\arraystretch}{.8}%
\raisebox{-3mm}{\begin{array}[b]{c} \vdots \,\,\,\,\,\, \vdots \\
\vdots \end{array}}}

\newcommand{\lemma}[1]{\renewcommand{\arraystretch}{.8}%
\begin{array}[b]{c} \vdots \\ #1 \end{array}}

\newcommand{\lemmarev}[1]{\renewcommand{\arraystretch}{.8}%
\begin{array}[b]{c} #1 \\ \vdots \end{array}}

\newcommand{\p}{\ensuremath{\upvarphi}}

% clashes with soul package
\newcommand{\yusukest}{\textbf{\textsf{st}}}

\newcommand{\shortarrow}{\xspace\hskip-1.2ex\scalebox{.5}[1]{\ensuremath{\bm{\rightarrow}}}\hskip-.5ex\xspace}

\newcommand{\SemInt}[1]{\mbox{$[\![ \textrm{#1} ]\!]$}}

\newcommand{\HypSpace}{\hskip-.8ex}
\newcommand{\RaiseHeight}{\raisebox{2.2ex}}
\newcommand{\RaiseHeightLess}{\raisebox{1ex}}

\newcommand{\ThreeColHyp}[1]{\RaiseHeight{\Bigg[}\HypSpace#1\HypSpace\RaiseHeight{\Bigg]}}
\newcommand{\TwoColHyp}[1]{\RaiseHeightLess{\Big[}\HypSpace#1\HypSpace\RaiseHeightLess{\Big]}}

\newcommand{\LemmaShort}{\ensuremath{ \ \vdots} \ \noLine}
\newcommand{\LemmaShortAlt}{\ensuremath{ \ \vdots} \ }

\newcommand{\fail}{**}
\newcommand{\vs}{\raisebox{.05em}{\ensuremath{\upharpoonright}}}
\newcommand{\DerivSize}{\small}

% This is not needed, we just take unicode symbols
% The result of the code below came out wrong anyway.
% St. Mü. 10.06.2021
%
% \def\maru#1{{\ooalign{\hfil
%   \ifnum#1>999 \resizebox{.25\width}{\height}{#1}\else%
%   \ifnum#1>99 \resizebox{.33\width}{\height}{#1}\else%
%   \ifnum#1>9 \resizebox{.5\width}{\height}{#1}\else #1%
%   \fi\fi\fi%
% \/\hfil\crcr%
% \raise.167ex\hbox{\mathhexbox20D}}}}

\newenvironment{samepage2}%
 {\begin{flushleft}\begin{minipage}{\linewidth}}
 {\end{minipage}\end{flushleft}}

\newcommand{\cmt}[1]{\textsl{\textbf{[#1]}}}
\newcommand{\trns}[1]{\textbf{#1}\xspace}
\newcommand{\ptfont}{}
\newcommand{\gp}{\underline{\phantom{oo}}}
\newcommand{\mgcmt}{\marginnote}

\newcommand{\term}[1]{\emph{\isi{#1}}}

\newcommand{\citeposs}[1]{\citeauthor{#1}'s \citeyearpar{#1}}

% for standalone compilations Felix: This is in the class already
%\let\thetitle\@title
%\let\theauthor\@author 
\makeatletter
\newcommand{\togglepaper}[1][0]{ 
\bibliography{../Bibliographies/stmue,../localbibliography,
collection.bib}
  %% hyphenation points for line breaks
%% Normally, automatic hyphenation in LaTeX is very good
%% If a word is mis-hyphenated, add it to this file
%%
%% add information to TeX file before \begin{document} with:
%% %% hyphenation points for line breaks
%% Normally, automatic hyphenation in LaTeX is very good
%% If a word is mis-hyphenated, add it to this file
%%
%% add information to TeX file before \begin{document} with:
%% \include{localhyphenation}
\hyphenation{
A-la-hver-dzhie-va
ac-cu-sa-tive
anaph-o-ra
ana-phor
ana-phors
an-te-ced-ent
an-te-ced-ents
affri-ca-te
affri-ca-tes
ap-proach-es
Atha-bas-kan
Athe-nä-um
Be-schrei-bung
Bona-mi
Chi-che-ŵa
com-ple-ments
con-straints
Cope-sta-ke
Da-ge-stan
Dor-drecht
er-klä-ren-de
Flick-inger
Ginz-burg
Gro-ning-en
Has-pel-math
Jap-a-nese
Jon-a-than
Ka-tho-lie-ke
Ko-bon
krie-gen
Kroe-ger
Le-Sourd
moth-er
Mül-ler
Nie-mey-er
Ørs-nes
Par-a-digm
Prze-piór-kow-ski
phe-nom-e-non
re-nowned
Rie-he-mann
un-bound-ed
Ver-gleich
with-in
}

% listing within here does not have any effect for lfg.tex % 2020-05-14

% why has "erklärende" be listed here? I specified langid in bibtex item. Something is still not working with hyphenation.


% to do: check
%  Alahverdzhieva


% biblatex:

% This is a LaTeX frontend to TeX’s \hyphenation command which defines hy- phenation exceptions. The ⟨language⟩ must be a language name known to the babel/polyglossia packages. The ⟨text ⟩ is a whitespace-separated list of words. Hyphenation points are marked with a dash:

% \DefineHyphenationExceptions{american}{%
% hy-phen-ation ex-cep-tion }

\hyphenation{
A-la-hver-dzhie-va
ac-cu-sa-tive
anaph-o-ra
ana-phor
ana-phors
an-te-ced-ent
an-te-ced-ents
affri-ca-te
affri-ca-tes
ap-proach-es
Atha-bas-kan
Athe-nä-um
Be-schrei-bung
Bona-mi
Chi-che-ŵa
com-ple-ments
con-straints
Cope-sta-ke
Da-ge-stan
Dor-drecht
er-klä-ren-de
Flick-inger
Ginz-burg
Gro-ning-en
Has-pel-math
Jap-a-nese
Jon-a-than
Ka-tho-lie-ke
Ko-bon
krie-gen
Kroe-ger
Le-Sourd
moth-er
Mül-ler
Nie-mey-er
Ørs-nes
Par-a-digm
Prze-piór-kow-ski
phe-nom-e-non
re-nowned
Rie-he-mann
un-bound-ed
Ver-gleich
with-in
}

% listing within here does not have any effect for lfg.tex % 2020-05-14

% why has "erklärende" be listed here? I specified langid in bibtex item. Something is still not working with hyphenation.


% to do: check
%  Alahverdzhieva


% biblatex:

% This is a LaTeX frontend to TeX’s \hyphenation command which defines hy- phenation exceptions. The ⟨language⟩ must be a language name known to the babel/polyglossia packages. The ⟨text ⟩ is a whitespace-separated list of words. Hyphenation points are marked with a dash:

% \DefineHyphenationExceptions{american}{%
% hy-phen-ation ex-cep-tion }

  \memoizeset{
    memo filename prefix={hpsg-handbook.memo.dir/},
    % readonly
  }
  \papernote{\scriptsize\normalfont
    \@author.
    \titleTemp. 
    To appear in: 
    Stefan Müller, Anne Abeillé, Robert D. Borsley \& Jean-Pierre Koenig (eds.)
    HPSG Handbook
    Berlin: Language Science Press. [preliminary page numbering]
  }
  \pagenumbering{roman}
  \setcounter{chapter}{#1}
  \addtocounter{chapter}{-1}
}
\makeatother

\makeatletter
\newcommand{\togglepaperminimal}[1][0]{ 
  \bibliography{../Bibliographies/stmue,
                ../localbibliography,
collection.bib}
  %% hyphenation points for line breaks
%% Normally, automatic hyphenation in LaTeX is very good
%% If a word is mis-hyphenated, add it to this file
%%
%% add information to TeX file before \begin{document} with:
%% %% hyphenation points for line breaks
%% Normally, automatic hyphenation in LaTeX is very good
%% If a word is mis-hyphenated, add it to this file
%%
%% add information to TeX file before \begin{document} with:
%% \include{localhyphenation}
\hyphenation{
A-la-hver-dzhie-va
ac-cu-sa-tive
anaph-o-ra
ana-phor
ana-phors
an-te-ced-ent
an-te-ced-ents
affri-ca-te
affri-ca-tes
ap-proach-es
Atha-bas-kan
Athe-nä-um
Be-schrei-bung
Bona-mi
Chi-che-ŵa
com-ple-ments
con-straints
Cope-sta-ke
Da-ge-stan
Dor-drecht
er-klä-ren-de
Flick-inger
Ginz-burg
Gro-ning-en
Has-pel-math
Jap-a-nese
Jon-a-than
Ka-tho-lie-ke
Ko-bon
krie-gen
Kroe-ger
Le-Sourd
moth-er
Mül-ler
Nie-mey-er
Ørs-nes
Par-a-digm
Prze-piór-kow-ski
phe-nom-e-non
re-nowned
Rie-he-mann
un-bound-ed
Ver-gleich
with-in
}

% listing within here does not have any effect for lfg.tex % 2020-05-14

% why has "erklärende" be listed here? I specified langid in bibtex item. Something is still not working with hyphenation.


% to do: check
%  Alahverdzhieva


% biblatex:

% This is a LaTeX frontend to TeX’s \hyphenation command which defines hy- phenation exceptions. The ⟨language⟩ must be a language name known to the babel/polyglossia packages. The ⟨text ⟩ is a whitespace-separated list of words. Hyphenation points are marked with a dash:

% \DefineHyphenationExceptions{american}{%
% hy-phen-ation ex-cep-tion }

\hyphenation{
A-la-hver-dzhie-va
ac-cu-sa-tive
anaph-o-ra
ana-phor
ana-phors
an-te-ced-ent
an-te-ced-ents
affri-ca-te
affri-ca-tes
ap-proach-es
Atha-bas-kan
Athe-nä-um
Be-schrei-bung
Bona-mi
Chi-che-ŵa
com-ple-ments
con-straints
Cope-sta-ke
Da-ge-stan
Dor-drecht
er-klä-ren-de
Flick-inger
Ginz-burg
Gro-ning-en
Has-pel-math
Jap-a-nese
Jon-a-than
Ka-tho-lie-ke
Ko-bon
krie-gen
Kroe-ger
Le-Sourd
moth-er
Mül-ler
Nie-mey-er
Ørs-nes
Par-a-digm
Prze-piór-kow-ski
phe-nom-e-non
re-nowned
Rie-he-mann
un-bound-ed
Ver-gleich
with-in
}

% listing within here does not have any effect for lfg.tex % 2020-05-14

% why has "erklärende" be listed here? I specified langid in bibtex item. Something is still not working with hyphenation.


% to do: check
%  Alahverdzhieva


% biblatex:

% This is a LaTeX frontend to TeX’s \hyphenation command which defines hy- phenation exceptions. The ⟨language⟩ must be a language name known to the babel/polyglossia packages. The ⟨text ⟩ is a whitespace-separated list of words. Hyphenation points are marked with a dash:

% \DefineHyphenationExceptions{american}{%
% hy-phen-ation ex-cep-tion }

  \memoizeset{
    memo filename prefix={hpsg-handbook.memo.dir/},
    % readonly
  }
  \papernote{\scriptsize\normalfont
    \@author.
    \@title. 
    To appear in: 
    Stefan Müller, Anne Abeillé, Robert D. Borsley \& Jean-Pierre Koenig (eds.)
    HPSG Handbook
    Berlin: Language Science Press. [preliminary page numbering]
  }
  \pagenumbering{roman}
  \setcounter{chapter}{#1}
  \addtocounter{chapter}{-1}
}
\makeatother




% In case that year is not given, but pubstate. This mainly occurs for titles that are forthcoming, in press, etc.
\renewbibmacro*{addendum+pubstate}{% Thanks to https://tex.stackexchange.com/a/154367 for the idea
  \printfield{addendum}%
  \iffieldequalstr{labeldatesource}{pubstate}{}
  {\newunit\newblock\printfield{pubstate}}
}

\DeclareLabeldate{%
    \field{date}
    \field{year}
    \field{eventdate}
    \field{origdate}
    \field{urldate}
    \field{pubstate}
    \literal{nodate}
}

%\defbibheading{diachrony-sources}{\section*{Sources}} 

% if no langid is set, it is English:
% https://tex.stackexchange.com/a/279302
\DeclareSourcemap{
  \maps[datatype=bibtex]{
    \map{
      \step[fieldset=langid, fieldvalue={english}]
    }
  }
}


% for bibliographies
% biber/biblatex could use sortname field rather than messing around this way.
\newcommand{\SortNoop}[1]{}


% Doug Ball

\newcommand{\elist}{\q<\ \ \q>}

\newcommand{\esetDB}{\q\{\ \ \q\}}


\makeatletter

\newcommand{\nolistbreak}{%

  \let\oldpar\par\def\par{\oldpar\nobreak}% Any \par issues a \nobreak

  \@nobreaktrue% Don't break with first \item

}

\makeatother


% intermediate before Frank's trees are fixed
% This will be removed!!!!!
%\newcommand{\tree}[1]{} % ignore them blody trees
%\usepackage{tree-dvips}


\newcommand{\nodeconnect}[2]{}
\newcommand{\nodetriangle}[2]{}



% Doug relative clauses
%% I've compiled out almost all my private LaTeX command, but there are some
%% I found hard to get rid of. They are defined here.
%% There are few others which defined in places in the document where they have only
%% local effect (e.g. within figures); their names all end in DA, e.g. \MotherDA
%% There are a lot of \labels -- they are all of the form \label{sec:rc-...} or
%% \label{x:rc-...} or similar, so there should be no clashes.

% Subscripts -- scriptsize italic shape lowered by .25ex 
\newcommand{\subscr}[1]{\raisebox{-.5ex}{\protect{\scriptsize{\itshape #1\/}}}}
% A boxed subscript, for avm tags in normal text
\newcommand{\subtag}[1]{\subscr{\idx{#1}}}

%% Sets and tuples: I use \setof{} to get brackets that are upright, not slanted
%\newcommand{\setof}[1]{\ensuremath{\lbrace\,\mathit{#1}\,\rbrace}}
% 11.10.2019 EP: Doug requested replacement of existing \setof definition with the following:
%\newcommand{\setof}[1]{\begin{avm}\{\textcolor{red}{#1}\}\end{avm}}
% 31.1.2019 EP: Doug requested re-replacement of the above \textcolour version with the following:
\newcommand{\setof}[1]{\begin{avm}\{#1\}\end{avm}}

\newcommand{\tuple}[1]{\ensuremath{\left\langle\,\mbox{\textit{#1}}\,\right\rangle}}

% Single pile of stuff, optional arugment is psn (e.g. t or b)
% e.g. to put a over b over c in a centered column, top aligned, do:
%   \cPile[t]{a\\b\\c} 
\newcommand{\cPile}[2][]{%
  \begingroup%
  \renewcommand{\arraystretch}{.5}\begin{tabular}[#1]{@{}c@{}}#2\end{tabular}%
  \endgroup%
}

%% for linguistic examples in running text (`linguistic citation'):
\newcommand{\lic}[1]{\textit{#1}}

%% A gap marked by an underline, raised slightly
%% Default argument indicates how long the line should be:
\newcommand{\uGap}[1][3ex]{\raisebox{.25em}{\underline{\hspace{#1}}}\xspace}

%% \TnodeDA{XP}{avmcontents} -- in a Tree, put a node label next to an AVM
\newcommand{\TnodeDA}[2]{#1~\begin{avm}{#2}\end{avm}}

%% This allows tipa stuff to be put in \emph -- we need to change to cmr first.
%% It is used in the discussion of Arabic.
\newcommand{\emphtipa}[1]{{\fontfamily{cmr}\emph{\tipaencoding #1}}} 



 
 
\definecolor{lsDOIGray}{cmyk}{0,0,0,0.45}


% morphology.tex:
% Berthold

\newcommand{\dnode}[1]{\rnode{#1}{\fbox{#1}}}
\newcommand{\tnode}[1]{\rnode{#1}{\textit{#1}}}

\newcommand{\tl}[2]{#2}

\newcommand{\rrr}[3]{%
  \psframebox[linestyle=none]{%
    \avmoptions{center}
    \begin{avm}
      \[mud & \{ #1 \}\\
      ms & \{ #2 \}\\
      mph & \<  #3 \> \]
    \end{avm}
  }
}
\newcommand{\rr}[2]{%
  \psframebox[linestyle=none]{%
    \avmoptions{center}
    \begin{avm}
      \[mud & \{ #1 \}\\
      mph & \<  #2 \> \]
    \end{avm}
  }
}
 

% Frank Richter
\newtheorem{mydef}{Definition}

\long\def\set[#1\set=#2\set]%
{%
\left\{%
\tabcolsep 1pt%
\begin{tabular}{l}%
#1%
\end{tabular}%
\left|%
\tabcolsep 1pt%
\begin{tabular}{l}%
#2%
\end{tabular}%
\right.%
\right\}%
}

\newcommand{\einruck}{\\ \hspace*{1em}}


%\newcommand{\NatNum}{\mathrm{I\hspace{-.17em}N}}
\newcommand{\NatNum}{\mathbb{N}}
\newcommand{\Aug}[1]{\widehat{#1}}
%\newcommand{\its}{\mathrm{:}}
% Felix 14.02.2020
\DeclareMathOperator{\its}{:}

\newcommand{\sequence}[1]{\langle#1\rangle}

\newcommand{\INTERPRETATION}[2]{\sequence{#1\mathsf{U}#2,#1\mathsf{S}#2,#1\mathsf{A}#2,#1\mathsf{R}#2}}
\newcommand{\Interpretation}{\INTERPRETATION{}{}}

\newcommand{\Inte}{\mathsf{I}}
\newcommand{\Unive}{\mathsf{U}}
\newcommand{\Speci}{\mathsf{S}}
\newcommand{\Atti}{\mathsf{A}}
\newcommand{\Reli}{\mathsf{R}}
\newcommand{\ReliT}{\mathsf{RT}}

\newcommand{\VarInt}{\mathsf{G}}
\newcommand{\CInt}{\mathsf{C}}
\newcommand{\Tinte}{\mathsf{T}}
\newcommand{\Dinte}{\mathsf{D}}

% this was missing from ash's stuff.

%% \def \optrulenode#1{
%%   \setbox1\hbox{$\left(\hbox{\begin{tabular}{@{\strut}c@{\strut}}#1\end{tabular}}\right)$}
%%   \raisebox{1.9ex}{\raisebox{-\ht1}{\copy1}}}



\newcommand{\pslabel}[1]{}

\newcommand{\addpagesunless}{\todostefan{add pages unless you cite the
 work as such}}

% dg.tex
% framed boxes as used in dg.tex
% original idea from stackexchange, but modified by Saso
% http://tex.stackexchange.com/questions/230300/doing-something-like-psframebox-in-tikz#230306
\tikzset{
  frbox/.style={
    rounded corners,
    draw,
    thick,
    inner sep=5pt,
    anchor=base,
  },
}

% get rid of these morewrite messages:
% https://tex.stackexchange.com/questions/419489/suppressing-messages-to-standard-output-from-package-morewrites/419494#419494
\ExplSyntaxOn
\cs_set_protected:Npn \__morewrites_shipout_ii:
  {
    \__morewrites_before_shipout:
    \__morewrites_tex_shipout:w \tex_box:D \g__morewrites_shipout_box
    \edef\tmp{\interactionmode\the\interactionmode\space}\batchmode\__morewrites_after_shipout:\tmp
  }
\ExplSyntaxOff


% This is for places where authors used bold. I replace them by \emph
% but have the information where the bold was. St. Mü. 09.05.2020
\newcommand{\textbfemph}[1]{\emph{#1}}



% Felix 09.06.2020: copy code from the third line into localcommands.tex:
% https://github.com/langsci/langscibook#defined-environments-commands-etc
% Does not work with texlive 2020, is done with sed in Makefile
%\patchcmd{\mkbibindexname}{\ifdefvoid{#3}{}{\MakeCapital{#3} }}{\ifdefvoid{#3}{}{#3 }}{}{\AtEndDocument{\typeout{mkbibindexname could not be patched.}}}



\let\textnobf\textit
% instead of "in bold" write "in italics"
\newcommand{\bolddescriptionintext}{italics\xspace}

% Berthold
\newcommand{\mathplus}{+}
% \mbox{\normalfont +}}
\newcommand{\emdash}{--\xspace}
\newcommand{\emdashUS}{--\xspace}


% Stefan to get the space remvoed infront of the : in Bargmann NPN discussion
%\DeclareMathSymbol{:}{\mathord}{operators}{"3A}
% used {:\,} instead


% for cxg.tex needed for includonly to find the counter.
\newcounter{croftyears} 




% Needed for bibtex entry for Jackendoff's xbar syntax. Without it the bar would be off in itialics.

% https://tex.stackexchange.com/questions/95014/aligning-overline-to-italics-font/95079#95079
% \newbox\usefulbox

% \makeatletter
%     \def\getslant #1{\strip@pt\fontdimen1 #1}

%     \def\skoverline #1{\mathchoice
%      {{\setbox\usefulbox=\hbox{$\m@th\displaystyle #1$}%
%         \dimen@ \getslant\the\textfont\symletters \ht\usefulbox
%         \divide\dimen@ \tw@ 
%         \kern\dimen@ 
%         \overline{\kern-\dimen@ \box\usefulbox\kern\dimen@ }\kern-\dimen@ }}
%      {{\setbox\usefulbox=\hbox{$\m@th\textstyle #1$}%
%         \dimen@ \getslant\the\textfont\symletters \ht\usefulbox
%         \divide\dimen@ \tw@ 
%         \kern\dimen@ 
%         \overline{\kern-\dimen@ \box\usefulbox\kern\dimen@ }\kern-\dimen@ }}
%      {{\setbox\usefulbox=\hbox{$\m@th\scriptstyle #1$}%
%         \dimen@ \getslant\the\scriptfont\symletters \ht\usefulbox
%         \divide\dimen@ \tw@ 
%         \kern\dimen@ 
%         \overline{\kern-\dimen@ \box\usefulbox\kern\dimen@ }\kern-\dimen@ }}
%      {{\setbox\usefulbox=\hbox{$\m@th\scriptscriptstyle #1$}%
%         \dimen@ \getslant\the\scriptscriptfont\symletters \ht\usefulbox
%         \divide\dimen@ \tw@ 
%         \kern\dimen@ 
%         \overline{\kern-\dimen@ \box\usefulbox\kern\dimen@ }\kern-\dimen@ }}%
%      {}}
%     \makeatother




\newcommand{\acknowledgmentsEN}{Acknowledgements}
\newcommand{\acknowledgmentsUS}{Acknowledgments}

% to put two examples next to eachother
%\newcommand{\shortbox}[3][-.7]{
%    \parbox[t]{.4\textwidth}{
%      \vspace{#1\baselineskip} #2\strut~~ #3}%
%}

\newcommand{\twomulticolexamples}[2]{
\begin{tabular}[t]{@{}l@{~~}l@{\hspace{1em}}l@{~~}l@{}}
a. & \parbox[t]{.4\textwidth}{#1} & b. & \parbox[t]{.4\textwidth}{#2}\\
\end{tabular}
}




% This does a linebreak for \gll for long sentences leaving space for the language at the right
% margin.
% St.Mü. 17.06.2021
\newcommand{\longexampleandlanguage}[2]{%
\begin{tabularx}{\linewidth}[t]{@{}X@{}p{\widthof{(#2)}}@{}}%
\begin{minipage}[t]{\linewidth}%
#1%
\end{minipage} & (\ili{#2})%
\end{tabularx}}



\renewcommand{\indexccg}{\is{Categorial Grammar (CG)!Combinatorial \textasciitilde{} (CCG)}\xspace}
\newcommand{\indexccgstart}{\is{Categorial Grammar (CG)!Combinatorial \textasciitilde{} (CCG)|(}\xspace}
\newcommand{\indexccgend}{\is{Categorial Grammar (CG)!Combinatorial \textasciitilde{} (CCG)|)}\xspace}
\renewcommand{\indexmp}{\is{Minimalism}\xspace}


\newcommand{\gisu}{Giuseppe Varaschin\xspace}

\newcommand{\NPi}{NP$\mkern-1mu_i$\xspace}
\newcommand{\NPj}{NP$\mkern-1.5mu_j$\xspace}
  %% -*- coding:utf-8 -*-

%%%%%%%%%%%%%%%%%%%%%%%%%%%%%%%%%%%%%%%%%%%%%%%%%%%%%%%%%%%%
%
% gb4e

% fixes problem with to much vertical space between \zl and \eal due to the \nopagebreak
% command.
\makeatletter
\def\@exe[#1]{\ifnum \@xnumdepth >0%
                 \if@xrec\@exrecwarn\fi%
                 \if@noftnote\@exrecwarn\fi%
                 \@xnumdepth0\@listdepth0\@xrectrue%
                 \save@counters%
              \fi%
                 \advance\@xnumdepth \@ne \@@xsi%
                 \if@noftnote%
                        \begin{list}{(\thexnumi)}%
                        {\usecounter{xnumi}\@subex{#1}{\@gblabelsep}{0em}%
                        \setcounter{xnumi}{\value{equation}}}
% this is commented out here since it causes additional space between \zl and \eal 06.06.2020
%                        \nopagebreak}%
                 \else%
                        \begin{list}{(\roman{xnumi})}%
                        {\usecounter{xnumi}\@subex{(iiv)}{\@gblabelsep}{\footexindent}%
                        \setcounter{xnumi}{\value{fnx}}}%
                 \fi}
\makeatother

% the texlive 2020 langsci-gb4e adds a newline after \eas, the texlive 2017 version was OK.
% \makeatletter
% \def\eas{\ifnum\@xnumdepth=0\begin{exe}[(34)]\else\begin{xlist}[iv.]\fi\ex\begin{tabular}[t]{@{}p{.98\linewidth}@{}}}
% \makeatother



%%%%%%%%%%%%%%%%%%%%%%%%%%%%%%%%%%%%%%%%%%%%%%%%%%%%%%%%%%
%
% biblatex

% biblatex sets the option autolang=hyphens
%
% This disables language shorthands. To avoid this, the hyphens code can be redefined
%
% https://tex.stackexchange.com/a/548047/18561

\makeatletter
\def\hyphenrules#1{%
  \edef\bbl@tempf{#1}%
  \bbl@fixname\bbl@tempf
  \bbl@iflanguage\bbl@tempf{%
    \expandafter\bbl@patterns\expandafter{\bbl@tempf}%
    \expandafter\ifx\csname\bbl@tempf hyphenmins\endcsname\relax
      \set@hyphenmins\tw@\thr@@\relax
    \else
      \expandafter\expandafter\expandafter\set@hyphenmins
      \csname\bbl@tempf hyphenmins\endcsname\relax
    \fi}}
\makeatother


% the package defined \attop in a way that produced a box that has textwidth
%
\def\attop#1{\leavevmode\begin{minipage}[t]{.995\linewidth}\strut\vskip-\baselineskip\begin{minipage}[t]{.995\linewidth}#1\end{minipage}\end{minipage}}


%%%%%%%%%%%%%%%%%%%%%%%%%%%%%%%%%%%%%%%%%%%%%%%%%%%%%%%%%%%%%%%%%%%%


% Don't do this at home. I do not like the smaller font for captions.
% This does not work. Throw out package caption in langscibook
% \captionsetup{%
% font={%
% stretch=1%.8%
% ,normalsize%,small%
% },%
% width=\textwidth%.8\textwidth
% }
% \setcaphanging


  \togglepaper[14]
}{}


\author{Robert D. Borsley\affiliation{University of Essex and Bangor University}\lastand Berthold Crysmann\affiliation{Centre national de la recherche scientifique (CNRS)}}
\title{Unbounded dependencies} 

\abstract{Unbounded dependencies of the kind that are found in \emph{wh}-interrogatives, relative clauses, and other constructions have been a major focus of research in HPSG. They typically involve a gap of some kind and some distinctive higher structure, often involving a filler in a non-argument position with the properties of the gap. HPSG has developed detailed proposals about the bottom of the dependency, the middle, and the top. In the case of the top of the dependency, complex hierarchies of phrase types have been employed to handle the distinctive properties of the various unbounded dependency constructions. Analyses have also been developed for unbounded dependencies with a resumptive pronoun, the special properties of \emph{wh}-interrogatives, extraposition phenomena, and filler-gap mismatches.}


%% \avmfont{\normalfont\scshape}
%% \avmvalfont{\normalfont\itshape}
%% \avmsortfont{\normalfont\small\itshape}

\begin{document}
\maketitle
\label{chap-udc}

% for udc.tex
% set this locally
\psset{nodesep=2pt} %,linewidth=0.8pt,arrowscale=2}
\psset{linewidth=0.5pt}


\section{Introduction}
\label{sec:Intro} 

Since \citet{ross_j67} and \citet{Chomsky:77}, it has been clear that many
languages have a variety of constructions involving an unbounded (or
long distance) dependency (henceforth UD). \emph{Wh}-interrogatives
and relative clauses are important examples, but, as we will see,
there are many others. Typically these constructions contain a gap
(in the sense that a dependent is missing) and some distinctive
higher structure, and neither can appear without the other. The
following illustrate:

\largerpage
\eal
\label{ex:UDC:basic}
\ex[]{\label{ex:what-did-you-put-on-the-table}
What did you put \trace{} on the table?}  % \ex[]{Who did you talk to \trace{}?}
\ex[*]{You put \trace{} on the table?}  % \ex[*]{You talked to \trace{}.}
\ex[*]{What did you put it on the table?}  % \ex[*]{Who did you talk to him?}
\zl

\noindent In (\ref{ex:UDC:basic}a) there is a gap (indicated by the
underscore) in object position and the distinctive higher structure
involves the interrogative pronoun \emph{what} and the pre-subject
auxiliary \emph{did}. (\ref{ex:UDC:basic}b), where the gap is
present but not the distinctive higher structure, is ungrammatical,
as is (\ref{ex:UDC:basic}c), where the distinctive higher structure
appears but not the gap.  The interrogative pronoun \textit{what} in
(\ref{ex:what-did-you-put-on-the-table}) is known as a filler, a constituent in a
non-argument position with the properties of the gap.  But the
distinctive higher structure does not always include a filler.
\ili{English} relative clauses may or may not have a filler:

\ea
\label{ex:UDC:2} 
the book [(which) you put \trace{} on the table]
\z 
\noindent As we will see below, there are also UD constructions which
never have a filler.  When there is a filler in a UD construction, it
normally has all the properties of the associated gap. Thus, in the
following, the filler and the gap are of the same category:

\eal
\label{ex:UDC:3} 
\ex[]{ [\textsubscript{NP} Who] did Kim talk to \trace{} (NP)?}
\ex[]{ [\textsubscript{PP} To whom] did Kim talk \trace{} (PP)?}
\ex[]{ [\textsubscript{AP} How long] is this piece of string \trace{} (AP)?}  
\ex[]{\label{ex-how-quickly-did-you}
{}[\textsubscript{AdvP} How quickly] did you do it \trace{}  (AdvP)?}
\zl

\noindent They typically match
in other respects as well. For example, if they are nominal, they
match in number, as the following illustrate:

\begin{exe}
  \ex \begin{xlist} \label{ex:UDC:4} \ex[]{
      [\textsubscript{NP[\textit{sg}]} Which student] do you
      think \trace{} (NP[\textit{sg}]) knows the answer?}
    \ex[]{ [\textsubscript{NP[\textit{pl}]} Which students]
      do you think \trace{} (NP[\textit{pl}]) know the
      answer?}  \end{xlist} \end{exe}

\noindent In languages with
grammatical gender or morphological case, they also share these
properties.  In addition to syntactic properties, unbounded
dependencies also establish matching of semantic properties: i.e.,
in (\ref{ex:what-did-you-put-on-the-table}), the filler \textit{what} is understood to
fill an argument role of \textit{put}, just as an in situ complement
would.  The term \emph{unbounded}\is{dependency!unbounded} is used here because the gap and the
distinctive higher structure with which it is associated can be
indefinitely far apart. The following illustrate:

\eal
\label{ex:UDC:5} 
\ex What does she regret that she put \trace{} on the table?
\ex What did she say she regrets that she put \trace{} on the table?
\ex What do you think she says she regrets that she put \trace{} on the table?
\zl

\largerpage
\noindent There are, however, some
restrictions here commonly referred to as island phenomena. These
are discussed by \crossrefchaptert{islands}.  There are a few
further points that we should make at the outset. We have focused so
far on UD constructions where an obligatory dependent, a subject or
complement, is missing. But UDCs are certainly not restricted to
subjects and complements.  There are examples where the filler has
an adjunct role such as (\ref{ex-how-quickly-did-you}) or the following:

\ea
\label{ex:UDC:6}
      % \avm{
      % 	\{\normalfont Where \\ \normalfont When \\ \normalfont How \\ \normalfont Why & \}
      % } 
$\begin{array}{@{}l@{}}
\left\{%
\begin{tabular}{@{}l@{}}
  Where\hspace{-.09em}\mbox{}\\
  When\\
  How\\
  Why\\
\end{tabular}\right\}
\end{array}$
did you talk to Lee \trace{}?
\z

\noindent 
There are also UD constructions with no gap at all. Instead they
have a so-called \term{resumptive pronoun} (RP). The following \ili{Welsh}
example with the RP in italics illustrates:

\ea
\label{ex:UDC:7}
\gll Pa    ddyn werthodd               Ieuan y   ceffyl iddo             \textnobf{fo}?\\
     which man  sell.\textsc{past.3sg} Ieuan the horse  to\textsc{.3sg.m} he\\
\glt `Which man did Ieuan sell the horse to?'
\z

\largerpage
\noindent
Finally, we should note that there are some cases where filler and gap
do not match.

\eal
\label{ex:UDC:8} 
\ex[]{ Kim will sing, which Lee won't \trace{}.}  
\ex[*]{Which won't Lee \trace{}?}
\zl

\noindent 
In (\ref{ex:UDC:8}a) the filler is a nominal expression, but the
gap is a non-finite VP. The \emph{wh}-interrogative in
(\ref{ex:UDC:8}b) shows that it is not normally possible to have a
nominal filler associated with a VP gap, but in (\ref{ex:UDC:8}a) it
is fine.  We explore the HPSG approach to these matters in the
following pages. In Section~\ref{sec:UDC:BasicApproach}, we outline
the basic HPSG approach to UDs. Then in
Section~\ref{sec:UDC:MoreOnGaps}, we focus on the nature of gaps,
i.e.\ the bottom of the dependency, and in Section~\ref{sec:UDC:Middle}
we look more closely at the middle of UDs. In
Section~\ref{sec:UDC:Top}, we consider the top of UDs and highlight
the variety of UD constructions. In
Section~\ref{sec:UDC:ResumptivePronouns}, we look at resumptive
pronouns. Then, in Section~\ref{sec:UDC:MoreWh}, we consider some
further aspects of \emph{wh}"=interrogatives, including pied-piping
and \emph{wh}-in-situ phenomena, Section~\ref{sec:UDC:Extraposition} deals with extraposition, and, in
Section~\ref{sec:UDC:FillerGapMismatches}, we take a look at
filler-gap mismatches.  Finally in
Section~\ref{sec:UDC:ConcludingRemarks}, we summarise the chapter,
followed by an appendix comparing HPSG to SBCG.  

\section{The basic approach}
\label{sec:UDC:BasicApproach} 

An analysis of UDs needs an
account of gaps, of the structures at the top of UDs, and of the
connection between them. Central to the HPSG approach is the feature
\textsc{slash}, occasionally called \textsc{gap} in some recent works,
which provides information about the presence of UD gaps inside a
constituent.\footnote{The basic approach derives from the earlier
  Generalised Phrase Structure Grammar (GPSG) framework
  \citep*{Gazdar85} and can be traced back to \citet{gazdar_g81}. The
  feature's name equally derives from this heritage, referring to the
  GPSG notation whereby X/Y stands for a category X containing a gap of
  category Y.} 
Much HPSG work assumes the feature geometry {in} (\ref{ex:UDC:9}), following \citet[Chapter~4]{Pollard:Sag:94}:

\ea
\label{ex:UDC:9}
HPSG feature geometry: nonlocal and local features\\
\avm{
	[\type*{synsem}
	local &	[\type*{local}
			category & category \\
			content & content ] \\
	nonlocal &	[\type*{nonlocal}
				slash & \setOf{local} \\
				\ldots ] ]
}
\z

\noindent
As this indicates, \textsc{slash} is part of the value of the feature \textsc{nonlocal}.
Its value is a set of \textit{local} feature structures. If we use traditional
category labels as abbreviations for local feature structures, we can
say that a constituent containing an NP gap is [\textsc{slash} \{NP\}], a
constituent containing a PP gap is [\textsc{slash} \{PP\}], and so on.

Turning to gaps, a central question is whether there is a
phonologically empty element in the constituent structure or nothing
at all. Both positions have been developed within HPSG, but probably
the view that there is nothing at all in constituent structure is the more
widely assumed position. We will adopt that for now and return to the
issues in Section~\ref{sec:UDC:MoreOnGaps}. Assuming this position,
example (\ref{ex:what-did-you-put-on-the-table}),
repeated here as (\ref{ex:UDC:10}), will contain a V with just a
single complement sister, namely the predicative PP
\emph{on the table}.

\begin{exe}
\ex \label{ex:UDC:10}
What did you put \trace{} on the table?
\end{exe}

\noindent
Because the V node in Figure~\ref{fig:UDC:11} contains an NP gap, it
will be [\textsc{slash} \{NP\}], and so will the constituents that
contain it, with the exception of the complete sentence. Thus, we have
the schematic structure illustrated in Figure~\ref{fig:UDC:11}.

\begin{figure}
  \centering
\begin{forest}
sm edges without translation
	[{S[\slasch \{ \}]}
		[{NP[\local \rnode{1}{\ibox{1}}]}
			[what]]
		[{S[\slasch \{ \rnode{2}{\ibox{1}} \}]}
			[V[did]]
			[NP[you]]
			[{VP[\slasch \{ \rnode{3}{\ibox{1}} \}]}
				[{V[\slasch \{ \rnode{4}{\ibox{1}} \}]}
					[put]]
				[PP[on the table, roof]]]]]
\end{forest}
\caption{\label{fig:UDC:11}Extraction by \textsc{slash} feature percolation}
\end{figure}

 %% \Tree [ [ what ].{NP\\\begin{avm}
%%     \[local \@1 \]
%%   \end{avm}} [ [ did ].V  [ you ].NP [ [ put ].{V\\\begin{avm}
%%     \[slash \{ \@1 \} \]
%%   \end{avm} }  \qroof{on the
%% table}.{PP}
%% ].{VP\\\begin{avm}
%%     \[slash \{ \@1 \} \]
%%   \end{avm}} ].{S\\\begin{avm}
%%     \[slash \{ \@1 \} \]
%%   \end{avm}} ].{S\\\begin{avm}
%%     \[slash \{ \} \]
%%   \end{avm} }
%  \includegraphics{chapters/BB-udc-basic-tree-crop}

Obviously, we need to ask what ensures that the \textsc{slash} feature plays just
the right role here. First, however, we need to say more about gaps.

On the view of gaps we are focusing on here, they are only represented
on argument structure, i.e. 
\textsc{arg-st} lists (see \crossrefchapteralt[Section~\ref{properties:lexemes-and-words}]{properties} and \crossrefchapteralt{arg-st}). Thus, the verb \emph{put} in (\ref{ex:UDC:10})
 has a gap in its
\textsc{arg-st} list and therefore only a PP in its \textsc{comps} list and
in constituent structure. Gaps have the feature make up given in
(\ref{ex:UDC:12}):
\ea
\label{ex:UDC:12}
Representation of gaps, according to \citet[161]{Pollard:Sag:94}:\\
\avm{
    [local & \1 \\
    nonlocal &	[slash & \{ \1 \} ] ]
}
\z  


\noindent
Thus, \emph{put} in (\ref{ex:UDC:10}) will have an element of this
form in an \textsc{arg-st} list where \ibox{1} is the \localv of an
NP.

Returning now to \textsc{slash}, a widely assumed approach involves the following
assumptions:

\eal
\label{ex:UDC:13}
\ex
The \textsc{slash} value of a head is normally the same as the union of the \textsc{slash} values of its
arguments.

\ex
The \textsc{slash} value of a phrase is normally the same as that of its head.
\zl

\noindent
We will consider how these ideas are formalised in
Section~\ref{sec:UDC:Middle}. For now we will just discuss their
implications for the analysis of (\ref{ex:UDC:10}).  Essentially they
mean that it has the following more elaborate analysis, as given in Figure~\ref{fig:UDC:14}.

 \begin{figure}
   \centering
\begin{forest}
sm edges without translation
[{S[\slasch \{ \}]}
  [{NP[\local \rnode{1}{\ibox{1}}]} [what]]
  [{S[\slasch \{ \rnode{2}{\ibox{1}} \}]}%, l sep+=\baselineskip
    [{V[\slasch \{ \rnode{3}{\ibox{1}} \}]} [did]]
    [{NP[\slasch \{ \}]} [you]]
    [{VP[\slasch \{ \rnode{4}{\ibox{1}} \}]}
      [{V[\slasch \{ \rnode{5}{\ibox{1}} \}]} [put]]
      [{PP[\slasch \{ \}]} [on the table, roof]]]]]
\end{forest}
%\nccurve[angleA=45,angleB=130]{<->}{1}{2}
%\nccurve[angleA=270,angleB=70]{<->}{2}{3}
%\nccurve[angleA=45,angleB=130]{<->}{3}{4}
%\nccurve[angleA=270,angleB=80]{<->}{4}{5}
\caption{\label{fig:UDC:14}Head-driven \textsc{slash} feature percolation}
\end{figure}



Clause (\ref{ex:UDC:13}a) is responsible for the \textsc{slash} values
on P and both Vs, while clause (\ref{ex:UDC:13}b)
is responsible for the \textsc{slash} values on PP, VP, and the lower S. This
approach to the distribution of \textsc{slash} crucially involves heads and is
commonly said to be head-driven.

The lower S in Figures~\ref{fig:UDC:11} and~\ref{fig:UDC:14} is the head of
the higher S, but they do not have the same value for
\textsc{slash}. This is because they represent the top of the
dependency. If information about gaps were available above the top of
the dependency, it would be possible to have another filler higher in
the tree, as in (\ref{ex:UDC:15}).

\begin{exe}
  \ex[*]{What do you wonder what Kim saw \trace{}?}   \label{ex:UDC:15}
\end{exe}

\noindent
The top of the dependency in Figures~\ref{fig:UDC:11}
and~\ref{fig:UDC:14} is a head-filler phrase and
the constraint on head-filler phrases needs to ensure that the higher S
is [\textsc{slash} \{ \}]. One might propose the following constraint:\footnote{
We use shorthands rather than full AVMs. For example \SLASH is located under
  \textsc{synsem|nonloc} and \comps under \textsc{synsem|loc|cat}. See \crossrefchapterw[Section~\ref{prop:sec-elements}]{properties}
  for details.
}

\ea
%\label{fig:UDC:16}
Head-Filler Schema (singleton \textsc{slash} set):\\
%   \small
\emph{head-filler-phrase}\is{schema!Filler-Head} \impl\\
\avm{
	[slash & \{ \} \\
	hd-dtr & \1	[comps & < >\\
                         slash & \{ \2 \} ]\\
	dtrs & < [local & \2 ], \1 > ]
}
\z

\noindent
This says that a head-filler phrase is \textsc{slash} \{  \} and has a head
daughter which has a saturated \textsc{comps} list and has a single local feature structure in its
\textsc{slash} set and a non-head daughter whose \textsc{local} value is the local feature
structure in the \textsc{slash} set of the head. Standardly, however, a slightly
more general constraint is assumed along the following lines:

\eas
\label{fig:UDC:17}%
Head-Filler Schema\is{schema!Filler-Head}:\footnote{Some HPSG work employs a \textsc{to-bind} feature
  on the head of a phrase to identify information about gaps that
  should not be passed up to the mother. But much recent work uses
  stipulations in certain phrase types and lexical entries to do this
  work and dispenses with this feature.

}\\
\emph{head-filler-phrase} \impl\\
\avm{
[slash & \3 \\
 hd-dtr & \1 [comps & < > \\
              slash & \{ \2 \} \cupAVM \3 ] \\
 dtrs & < [local & \2 ], \1 > ]
}
\zs

\noindent
This allows the \textsc{slash} set of the head to contain more than one member
and any additional members form the \textsc{slash} set of the whole phrase \iboxb{3}. This
is necessary for an example like (\ref{ex:UDC:18}) from \citet[473]{Chaves:12}, where indices
are used to link fillers and gaps.

\ea
\label{ex:UDC:18}
This is the person who$_i$ I can't remember [which papers]$_j$ I sent copies of \_$_j$ to \_$_i$.
\z

\noindent
Examples of this form often seem unacceptable, but this is probably a
processing matter, see \citet[Section~3]{Chaves:12} for discussion.
See also Section~\ref{sec:UDC:ResumptivePronouns} for long
relativisation with resumption in \ili{Hausa} or Modern Standard \ili{Arabic}.

\section{More on gaps}
\label{sec:UDC:MoreOnGaps}


We now look more closely at the nature of gaps. The central question
here is: what exactly are gaps? We noted in the last section that it
has been widely assumed that gaps are only represented in
\textsc{arg-st} lists, but that some HPSG work assumes that they are
empty categories, often called traces. There is a third possibility
which might be considered, namely that gaps are represented in
\textsc{arg-st} lists and in \textsc{valence} lists,
i.e.\ \textsc{subj} and \textsc{comps} lists, but not in constituent
structures. However, it seems that this position has rarely been
considered. One complicating factor is that there seem to be
differences between complement gaps and both subject and adjunct
gaps. A consequence of this is that the question ``what are gaps?''
could have different answers for different sorts of gaps, and in fact
different answers have sometimes been given.

Complement gaps seem to have had rather more attention than subject or
adjunct gaps, perhaps because there are many different kinds of
complements, hence many different kinds of complement gaps. We will look
first at complement gaps, and in particular, the gap in (\ref{ex:UDC:basic}), repeated
here as (\ref{ex:UDC:19}).

\begin{exe}
\ex \label{ex:UDC:19}
What did you put \trace{} on the table?
\end{exe}

\noindent
Probably the most widely assumed position is that gaps are only
represented in \textsc{arg-st} lists
(see \citealt[Section~4.1]{Sag:97},
\citealt*[Section~2.2]{Bouma:Malouf:Sag:01},
\citealt[Chapter~5.1]{Ginzburg:Sag:01} and \citealt[508]{Sag:10a}). On
this view, the verb \textit{put} will have the
following syntactic properties:

\ea
\label{udc:ex-slashed-verb-traceless}
Representation of a slashed verb (traceless):\\
\avm{
[head & verb \\
 subj & < \1 >\\
 comps & < \2 >\\
 slash & \{ \3 \} \\
 arg-st & < \1 NP, NP[local & \3 \\ 
	      slash & \{ \3 \} ], \2 PP > ]
}
\z

\noindent
We ignore the \textsc{comps} feature and the issue of what ensures that the verb
here has the same \textsc{slash} value as the gap. We will discuss the latter in
the next section.

The view that gaps are empty categories was a feature of early HPSG
work, notably \citet[Chapter~4]{Pollard:Sag:94}, and it has been assumed in some
more recent work, e.g.\ \citet[191,385]{Levine:Hukari:06}, \citet{Borsley:09a},
\citet[Section~4.2]{Borsley:13}, and \citet{Mueller:14b}. On
this view, the VP will have the following structure:

\begin{figure}
%\centering
\oneline{%
\begin{forest}
sm edges without translation
	[%
	\avm{
		[head & verb \\
		slash & \{ \1 \} ]
	}
		[%
		\avm{
			[head & verb \\
			slash & \{ \} \\
			arg-st & < \2\,NP, \3 PP > ]
		}
			[put]]
		[%
		\avm{
			[synsem & \2	[local  & \1 \\
		                         nonloc & [slash & \{ \1 \} ] ] ]
		}
			[\trace]]
		[%
		\avm{[synsem & \3 ]} [on the table, roof]] ]
\end{forest}
}
    \caption{\label{fig:UDC:21}Representation of a slashed VP (with trace)}  
\end{figure}

  %% \resizebox{\linewidth}{!}{
%% \Tree [ [ put ].{\begin{avm}
%%     \[head & verb\\
%%       slash & \{ ~ \}\\
%%       arg-st & \< \@2 NP, \@3 PP \>\]
%%   \end{avm}
%% }
%% [ $e$ ].{
%%   \begin{avm}
%%     \[synsem & \@2\[local & \@1 \\nonlocal & \[slash & \{ \@1 \}\]\]\]
%%   \end{avm}
%% }
%% \qroof{on the table}.{
%%   \begin{avm}
%%     \[synsem & \@3\]
%%   \end{avm}
%% } 
%% ].{\begin{avm}
%%     \[head & verb\\
%%       slash & \{ \@1 \}\]
%%   \end{avm}
%% }
%% }

\largerpage
\noindent
Again we ignore the \textsc{comps} feature and how the VP here has the same \textsc{slash}
value as the gap.

It is not easy to choose between these two approaches. One argument in
favour of the first view, advanced, for example, in \citet[Section~3.5.2]{Bouma:Malouf:Sag:01}, is that it makes it unsurprising that a gap cannot be one
conjunct of a coordinate structure, as in the following:

\begin{exe} \ex \begin{xlist} \label{ex:UDC:22}
\ex[*]{Which of her books did you find both [[a review of
\trace{}] and \trace{}]?}

\ex[*]{Which of her books did you find [\trace{} and [a review of
\trace{}]]?}
\end{xlist}
\end{exe}

\noindent
It is not obvious why this should be impossible if gaps are empty
categories.\footnote{Coordination is a problem for any empty category,
  not just the empty categories that represent gaps in some HPSG
  work. Various empty categories have been proposed in the HPSG
  literature, most prominently the empty relativiser of
  \citet[Chapter~5]{Pollard:Sag:94}. \citet[Section~15.3.5]{Sag:Wasow:ea:03}
  propose that African American Vernacular English\il{English!African American Vernacular} has a
  phonologically empty form of the copula. This analysis requires some
  mechanism to prevent this form from appearing as a conjunct. It is
  likely that a mechanism that can do this will also prevent the empty
  categories that represent gaps from being conjuncts.
 }

 A second argument in favour of a traceless approach comes from
 languages which morphologically treat slashed transitives on a par
 with intransitives, like \ili{Hausa} \citep{crysmann_b04yom} or
 \ili{Mauritian} Creole \ili{French} \citep{Henri10}. In \ili{Hausa}
 and \ili{Mauritian}, verbs morphologically register whether a direct
 object is realised locally or not: in both languages, a ``short''
 form is used with locally realised direct objects, whereas the long
 form is used with intransitives as well as in the case of object
 extraction. Consider the following examples from \ili{Hausa},
 partially adapted from \citet[632--633]{newman_p00}:

\ea
\gll Sun hūtā̀.\\
  \textsc{3pl.cpl} rest.\textsc{a}\\\hfill(Hausa)
\glt `They rested.' \hfill \label{ex:UDC:Hau:intr}
\z
\eal
\label{ex:UDC:Hau:tr}
    \ex\label{ex:UDC:Hau:tr:a} 
    \gll Sun rāzànā.\footnotemark\\
         \textsc{3pl.cpl} terrorise.\textsc{a}\\\hfill{(Hausa)}
         \footnotetext{\citet[632]{newman_p00}}
    \glt `They terrorised (someone).'
    \ex\label{ex:UDC:Hau:tr:b} 
    \gll Sun rāzànà far̃ar-hū̀lā.\footnotemark\\
         \textsc{3pl.cpl} terrorise.\textsc{c} civilian(s)\\
         \footnotetext{\citet[632]{newman_p00}}
    \glt `They terrorised the civilians.'
    \ex 
    \gll Far̃ar-hū̀lā nḕ sukà rāzànā.\\
         civilian(s) \textsc{foc} \textsc{3pl.cpl} terrorise.\textsc{a}\\
    \glt `The civilians, they terrorised.' 
\zl

\noindent
\ili{Hausa} verbs are lexically transitive or intransitive, and they
are classified into one of seven morphological grades.\footnote{We
  restrict discussion here to grade 1, although the syntactic pattern
  is systematic across grades, only giving rise to different patterns
  of exponence. See the \ili{Hausa} grammars by \citet{newman_p00} and
  \citet{jaggar01:_hausa} for details, and \citet{crysmann_b04yom} for
  evidence in favour of a morphological treatment. } Intransitives
only have a single form (A-form), which is characterised by a long
vowel (in grade 1), cf.\ (\ref{ex:UDC:Hau:intr}). Transitives, however,
display an alternation depending on the mode of realisation of the
direct object: if used intransitively, they pattern with intransitive
verbs in using the A-form (long vowel in grade 1), but with an in situ
direct object (\ref{ex:UDC:Hau:tr:a}), they obligatorily surface in
the C-form (\ref{ex:UDC:Hau:tr:b}), which has a short vowel in grade
1. Once the direct object is extracted, we find the long vowel A-form
again, in parallel to the intransitive use of transitives and true
intransitives. In sum, the morphology of \ili{Hausa} treats complement
extraction on a par with argument suppression or lexical
intransitives, i.e.\ as if the direct object complement simply were
not there. Similar observations appear to hold for \ili{Mauritian}
\citep[Section~4.2.3]{Henri2010a-u}. Thus, if nonlocal realisation
corresponds to lexical valence reduction, the \ili{Hausa} (and
\ili{Mauritian}) facts are straightforwardly accounted for, whereas
the generalisation would be lost, if gaps were considered
phonologically empty syntactic elements.

%\largerpage[2]
However, the lexical approach to argument extraction has some possibly
non-trivial implications for other lexical sub-theories of HPSG that
make crucial reference to valence lists, which includes lexical
theories of agreement and case. This is because gaps will not be
present on the valence lists of word-level signs. The theory of
ergativity proposed by \citet[Section~5.2]{Manning:Sag:99} in terms of
mapping between \textsc{arg-st} and valence lists is actually
formulated as constraints on lexemes, since e.g.\ the linking of the
highest argument to the first element on \textsc{comps} (ergative
subject) needs to be specified independently of whether this argument
is realised by a local or a non-local dependency. The same holds of
course for the linking of objects in accusative
languages.\footnote{\citet{Crysmann:09} exploits the fact that
  extracted arguments do not appear on the valence lists of word-level
  signs and formulates local case assignment for \ili{Nias} as a
  constraint on \textit{word}, effectively exempting topicalised
  arguments from objective case assignment.}

Similar considerations apply to agreement: if agreement treats local
and non-local arguments alike, it is clear that agreement controllers
cannot be identified in a general fashion in terms of the valence
features of word-level signs: thus, if agreement relations need to make
reference to valence rather than argument structure, this can only be
established at the level of lexemes.\footnote{
{Lexemes are basic lexical items. Lexemes of inflectable parts of speech are mapped to
  words. See \crossrefchaptert[Section~\ref{properties:lexemes-and-words}]{properties} for
  more on the notion of lexeme.}
} The relevant evidence comes from
languages, where the highest argument on \textsc{arg-st} does not
necessarily correspond to the highest grammatical function, i.e.
\textsc{subj} valence: while some ergative languages display agreement
with the highest argument on \textsc{arg-st}, e.g.\ \ili{Udi}
\citep{harris_a84udi}, \ili{Archi} \citep{kibrik94:_archi} shows agreement
with the absolutive argument, suggesting that \textsc{subj} is the
right place to establish the relation. % \footnote{\citet{Borsley:16:Archi} goes a step
  % further and argues that agreement in this language should be
  % expressed at the level of constituent structure. }
In \ili{Nias} \citep{Crysmann:09}, we find agreement with \textsc{subj} in the realis, and with
the least oblique argument in the irrealis (\textsc{arg-st}). Finally,
in \ili{Welsh}, we observe a parallelism in the agreement between subjects
of finite verbs and the objects of prepositions and non-finite verbs:
according to \citet[Section~4]{Borsley89}, a unified treatment can be given if
subjects of finite verbs are the first element on \textsc{comps}, an
assumption that directly captures \ili{Welsh} VSO word order.\footnote{
  \citet[Section~5.4]{Borsley:16:Archi} argues on rather different grounds that
  agreement in the Caucasian language \ili{Archi} involves constraints on
  constituent structure, which will favour a trace-based perspective on
  extraction. }

Given the broad empirical support for valence lists as one of the loci
of case and agreement constraints, it is clear that these constraints
must hold for lexemes, not words under a traceless, lexical
approach to unbounded dependencies.



%\bigskip
We\label{udc:page-subject-gaps-start} turn now to subject gaps. Here a central question is: ``how similar or
how different are they to complement gaps?'' The following illustrate a
well-known contrast, which suggests that they may be significantly
different:

\begin{exe} \ex \begin{xlist} \label{ex:UDC:25}
\ex[]{  Who do you think Kim saw \trace{}?}

\ex[]{ Who do you think \trace{} saw Kim?}
\end{xlist}
\end{exe}

\begin{exe} \ex \begin{xlist} \label{ex:UDC:26}
\ex[]{Who do you think that Kim saw \trace{}?}

\ex[*]{Who do you think that \trace{} saw Kim?}
\end{xlist}
\end{exe}

%\largerpage[2]
\noindent
The examples in (\ref{ex:UDC:25}) show that a gap is possible in
object position in a complement clause whether or not it is introduced
by \emph{that}.
% \itdopt{Say that this is the that trace effect? To make Minimalists
%   feel welcome?}
In contrast, the examples in (\ref{ex:UDC:26}) suggest
that a gap is only possible in subject position in a complement clause
if it is not introduced by
\emph{that}. \citet[Chapter~4.4]{Pollard:Sag:94} approach this
contrast by stipulating that gaps cannot appear in subject
position. This accounts for the ungrammaticality of examples like
(\ref{ex:UDC:26}b). Examples like (\ref{ex:UDC:26}a) are allowed by
allowing verbs like \textit{think} to take a VP complement and have a
non-empty value for
\textsc{slash}. \citet[Chapter~5.1.3]{Ginzburg:Sag:01} offer a very
different account, in which subject gaps appear both in
\textsc{arg-st} list and \textsc{subj} lists. They suggest that
examples like (\ref{ex:UDC:26}b) are ungrammatical because \emph{that}
cannot combine with a constituent which has a non-empty \textsc{subj}
list.

An important fact about subject gaps is that they are not completely
impossible in a complement clause introduced by \emph{that}. In
particular, they are acceptable if \emph{that} is followed by an
adverbial constituent. The following illustrates:

\begin{exe}
\ex \label{ex:UDC:27}
Who did you say that tomorrow \trace{} would regret his words?
\end{exe}

\noindent
\citet[Chapter~5.1.3]{Ginzburg:Sag:01} offer an account of such examples, but
\citet[Chapter~2.3.2]{Levine:Hukari:06} argue that it is unsatisfactory. More generally, they
argue that subject gaps are like complement gaps in various respects
and therefore should have the same basic analysis. They propose an
analysis with an empty category for both types of gap. Thus, their
approach differs both from the widely assumed approach, which has no
empty categories, and the approach of Pollard and Sag, which has them
in complement position but not in subject position.\label{udc:page-subject-gaps-end}

We turn now to adjunct gaps. It is not obvious that there is a gap in
examples like (\ref{ex:UDC:6}) repeated as (\ref{ex:UDC:28}), because
no obligatory constituent is missing.\footnote{This position has initially
  been taken in \citet[176--180]{Pollard:Sag:94}. } 

\ea
\label{ex:UDC:28}
$\begin{array}{@{}l@{}}
\left\{%
\begin{tabular}{@{}l@{}}
  Where\hspace{-.09em}\mbox{}\\
  When\\
  How\\
  Why\\
\end{tabular}\right\}
\end{array}$
did you talk to Lee \trace{}?
\z

\noindent
However, \cite{hukari.levine:adjunct} show that such examples may display
what are often called \emph{extraction path effects}\is{extraction path effect}, certain phonological or
morphosyntactic phenomena which appear between a gap and the associated
higher structure (see the discussion of example (\ref{ex:UDC:34}) on
page \pageref{ex:UDC:34}). Hence, it seems that they must
involve a filler-gap dependency, on a par with examples with a
complement gap.

Of course, there are a variety of positions that are compatible with
this conclusion. \citet[\page 12]{Bouma:Malouf:Sag:01} and \citet[168,
fn.~2]{Ginzburg:Sag:01} propose that verbal adjuncts are optional
extra complements.  On this view, the gaps in the examples in
(\ref{ex:UDC:28}) are complement gaps.  \citet{Levine:03} and
\citet[Chapter~3.5--3.6]{Levine:Hukari:06} argue against this approach with examples
like the following:

\begin{exe}
\ex \label{ex:UDC:29}
In how many seconds flat do you think that [Robin found a chair,
sat down, and took off her logging boots]?
\end{exe}

\noindent
This is a query about the total time taken by three distinct events.
\citeauthor{Levine:Hukari:06} propose a fairly traditional analysis of verbal
adjuncts in which they are modifiers of VP, and combine this with the
assumption that gaps are empty categories. The interpretation of
examples like (\ref{ex:UDC:29}) follows straightforwardly on this
analysis. If indeed argument extraction contrasts with adjunct
extraction in terms of whether the gap is introduced  lexically (on
\textsc{arg-st}) or phrasally, this may
provide a direct account of the fact that the use of a resumptive
strategy in extraction is by and large restricted to arguments. As
discussed by \citet{Crysmann:Reintges:14}, resumptives are obligatory
for arguments in \ili{Coptic}, whereas gap-type extraction is the only
possibility for modifiers.
 

A rather different approach is developed in \citet{Chaves:09}. Like
\citet[Chapter~3]{Levine:Hukari:06}, he assumes that verbal adjuncts are
modifiers of VP, but he rejects the idea that gaps are empty
categories. He shows in particular that the possibility for a filler
to correspond to a group is neither limited to adjunct extraction nor
to events, but may also be observed with NP complements whose gaps are
properly contained within each conjunct, as shown by the following
examples:

\begin{exe} \ex \begin{xlist} \label{ex:UDC:31}
\ex[]{ Setting aside illegal poaching for a moment, how many
sharks\textsubscript{\emph{i} + \emph{j}} do you estimate
[[\trace{}\emph{\textsubscript{i}} died naturally] and
[\trace{}\emph{\textsubscript{j}} were killed recreationally]]?}


\ex[]{ [[Which pilot]\emph{\textsubscript{i}} and [which
sailor]\emph{\textsubscript{j}}] will Joan invite \trace{}
\emph{\textsubscript{i}} and Greta entertain \trace{}
\emph{\textsubscript{j}}
(respectively)?}
\end{xlist}
\end{exe}

% \begin{exe}
%   \ex The [ships$_{x+y}$ that [[a U-boat destroyed \trace{}$_x$ ] and [a
%   kamikaze blew up \trace{}$_y$ ]] were not insured. 
% \end{exe}

\noindent
He suggests that the treatment of coordination must be relaxed in such
a way as to permit the creation of group individuals and group events
on the mother's \textsc{slash} where the daughters' \textsc{slash}
values contain the individual or event variables of the group's
members.  This provides an account of complement extraction as in
(\ref{ex:UDC:31}), but it also provides a
straightforward account of the cumulative scoping facts in
(\ref{ex:UDC:29}).
 

% Instead of a set valued
% \textsc{slash} feature, he has a
% list-valued GAP feature, and he allows words to have modifiers in
% their GAP lists. To accommodate examples like (\ref{ex:UDC:29}) he
% proposes an analysis of coordination which allows structures of the
% following form, where \avmbox{1} combines the properties of \avmbox{2} and
% \avmbox{3}.


% % \pdfmargincomment{BC: The preceding paragraph seems to be lacking crucial information.\textCR 
% %  As far as I remember Chaves relegates the solution to a more sophisticated semantics. 
% %  I am not sure, that we cannot just rename GAP to SLASH. I feel that "combines the properties" should be spelled out. }

% % \pdfmargincomment[color=red]{BB: I'll have to try to figure out how to spell out 'combines the properties".}

% \begin{exe} 
%   \ex \label{ex:UDC:30} 
% %   \pdfmargincomment{BC: Do you really want lists? \textCR
% %   BC: What are the constraints on \avmbox{1}, \avmbox{2}, \avmbox{3} ? \textCR
% %   BC: Why \textsc{gap}?}
  
% %   \pdfmargincomment[color=red]{BB: We can clearly note that it doesn't matter whether the feature is called SLASH or GAP or anything else (maybe by saying 'sometimes called gap' when we first mention SLASH). 

% % I'll have to try to figure out how to spell out 'combines the properties".}
%   \Tree [ {\begin{avm}
%       \[gap & \< \@2 \>\]
%     \end{avm}
%   } {
%     \begin{avm}
%       \[gap & \< \@3 \>\]
%     \end{avm}
%   } ].{
%     \begin{avm}
%       \[gap & \< \@1 \>\]
%     \end{avm}
%   }

% \end{exe}



\section{The middle of the dependency}
\label{sec:UDC:Middle}

In the middle of an unbounded dependency we typically have a phrase
(or a clause) with the same value for \textsc{slash} as a non-head daughter. As
we noted in Section~\ref{sec:UDC:BasicApproach}, it is widely assumed
that this relation is mediated by the head daughter. The \textsc{slash} value
of a head is normally the same as that of its arguments, and the \textsc{slash}
value of a phrase is normally the same as that of its head. However,
as we will see, this head-driven approach to the distribution of \textsc{slash}
hasn't always been adopted.

Central to the head-driven approach is the \textsc{slash} Amalgamation Principle,
which we can formulate as follows, following \citet[199]{Ginzburg:Sag:01}:

\ea
\label{fig:UDC:32}\label{udc:slash-amalgamation-principle}
\textsc{slash} Amalgamation Principle\is{principle!slash amalgamation@\SLASH Amalgamation}:\\
   \textit{word} \impl ~$\slash$%
\avm{
[synsem & [nonloc & [slash & \1 \cupAVM \ldots{} \cupAVM \tag{n}]]\\
 arg-st & <[nonloc & [slash & \1], \ldots, [nonloc & [slash & \tag{n} ]] > ] ]
}
\z

\noindent
This is a default constraint, as indicated by the `$\slash$'\is{$\slash$}. Essentially,
it says that by default the \textsc{slash} value of a word
is the union of the \textsc{slash} values of its arguments.  Being
merely a
default constraint will accommodate examples like the following:

\begin{exe}
\ex \label{ex:UDC:33}
The professor is hard [to talk to \trace{}].
\end{exe}

\noindent
Here, the adjective \emph{hard} takes an infinitival complement with a
non-empty \textsc{slash} feature 
but this  \textsc{slash} feature
is not passed on any further, but rather coindexed with the subject of
the adjective.\footnote{The
  non-local nature of \textit{tough}-constructions appears to be a
  peculiarity of \ili{English}: similar constructions in \ili{German} and \ili{French}
  do exist, but they feature local (passive-like)\is{passive} dependencies. See
  \citet{abeille_a-godard_d-miller_p-sag_i95} and
  \citet{aguila-multner18} for \ili{French}, as well as \citet[Section~3.1.5]{Mueller:02b}
  for \ili{German}. Even for \ili{English}, the unboundedness of the construction
  has been challenged: \citet{Grover:95} questions the acceptability
  of \ili{English} \textit{tough}-constructions involving a UDC out of
  finite clauses and suggests a local account instead.}

\largerpage
To ensure that the \textsc{slash} value of a phrase is normally the
same as that of its head, much work employs a Slash Inheritance
Principle\is{principle!slash Inheritance@\SLASH Inheritance}, which stipulates that a phrase and its head have the same
value for \textsc{slash} except at the top of a dependency (see, e.g.\
\citealt{Bouma:Malouf:Sag:01}: 20). An alternative approach developed
in \citet[Chapter~5.1]{Ginzburg:Sag:01} uses the Generalised Head
Feature Principle\is{principle!Head Feature!Generalized} for this purpose.\footnote{
  {See also \crossrefchapterw[\pageref{properties:ex-generalized-head-feature-principle}]{properties} for an explicit formulation of the constraint.}
} This says that a headed phrase and
its head daughter have the same \textsc{synsem} values unless some
other constraint requires something different. Among other things,
this ensures that a headed phrase and its head daughter normally have
the same value for \textsc{slash}.

One\label{page-start-extraction-path-effects} argument in favour of a head-driven approach to the distribution
of \textsc{slash} is so-called extraction path effects\is{extraction path effect}, certain phonological or
morphosyntactic phenomena which appear between a gap and the
associated higher structure (see \citealt{hukari.levine:adjunct};
\citealt[Section~3.2]{Bouma:Malouf:Sag:01}). \ili{Irish} provides one of many
examples that have been discussed. In \ili{Irish}, the verbal particle
\emph{goN} only occurs with structures that do not contain gaps, while
\emph{aL} only occurs between a filler and a gap. The following 
illustrate \citep[26]{BMS2001a}:\footnote{In some accounts these particles are taken to be
  complementisers. The N indicates that \emph{go} triggers nasal
  mutation while L indicates that \emph{a} triggers lenition.}


\eal
\label{ex:UDC:34}
\ex[]{
\gll Shíl    mé goN         mbeadh   sé ann.\\
     thought I \textsc{prt} would.be he there\\\hfill(\ili{Irish})
\glt `I thought that he would be there.'
}
\ex[]{
\gll {a}n fear aL          shíl    mé aL          bheadh   ann\\
     the  man \textsc{prt} thought I \textsc{prt} would.be there\\
\glt `the man that I thought would be here'
}
\zl

\noindent
Within a head-driven approach to \textsc{slash}, this is just a contrast between a
verb which is [\textsc{slash} \{ \}] and a verb which is [\textsc{slash}
\{[ ]\}], and is completely unproblematic.

Early HPSG assumed an approach to \textsc{slash} which was not
head-driven (see \citealt[Chapter~4]{Pollard:Sag:94}), and related
approaches are assumed in \citet{Levine:Hukari:06} and \citet[497]{Chaves:12}.  A problem
with a head-driven approach is that it says nothing about examples
where an unbounded dependency crosses the boundary of a non-headed
phrase such as a coordinate structure. Thus, it does not deal with
examples of asymmetric coordination like the following:

\begin{exe} \ex \begin{xlist} \label{ex:UDC:35}
\ex[]{ How much can you [drink \trace{}] and [still stay sober]]?}

\ex[]{ How many lakes can we [[destroy \trace{}] and [not arouse public antipathy]]?}
\end{xlist}
\end{exe}
% \pdfmargincomment{BC: Goldsmith (1985) suggests that these are possibly not coordinations. See the discussion in P&S 94} 

% \pdfmargincomment[color=red]{BB: I suspect Chaves argues that they are gaps but I will have to check.}

\noindent
Early HPSG \citep[Chapter~4]{Pollard:Sag:94} accounts for the distribution of
\textsc{slash} by means of the Nonlocal Feature Principle\is{principle!Nonlocal Feature}, and related
principles are proposed by \citet[354]{Levine:Hukari:06} and
\citet[497]{Chaves:12}. These principles ensure that the \textsc{slash}
value of a phrase reflects the \textsc{slash} values of all its
daughters (using set union) and apply equally to headed and non-headed
structures. Thus, the examples in (\ref{ex:UDC:35}) are no problem for
these latter approaches. However, they seem to require some extra element to
handle extraction path effects. So, it is not easy to choose between
these approaches and the head-driven approach.\label{page-end-extraction-path-effects}

\largerpage
A further point that we should emphasise here is that both approaches to
the distribution of \textsc{slash} allow structures like the one in Figure~\ref{fig:UDC:36}.

\begin{figure}
  \centering
\begin{forest}
sm edges without translation
	[\avm{[slash & \{ \1 \}]}
		[\ldots]
		[\avm{[slash & \{ \1 \}]} ]
		[\ldots]
		[\avm{[slash & \{ \1 \}]} ]
		[\ldots]
	]		
\end{forest}
  \caption{\label{fig:UDC:36}Across-the-board (ATB) extraction\is{across-the-board extraction}: conflation of \textsc{slash} values} 
\end{figure}

\noindent
In other words, both allow more than one daughter of a phrase with a
non-empty \textsc{slash} value to have the same value. This means that we expect
structures in which a single filler is associated with more than one
gap. Thus, examples like the following are no problem:

\begin{exe} \ex \begin{xlist} \label{ex:UDC:37}
\ex[]{ What did Kim [[ cook \trace{} for two hours] and [eat \trace{} in four
minutes]]?}

\ex[]{
\label{ex-which-person-did-you-invite-without-thinking}
Which person did you [invite \trace{} [without thinking \trace{} would
actually come]]?}
\end{xlist}
\end{exe}

\noindent
Example (\ref{ex:UDC:37}a), where the two gaps are in a coordinate structure
is standardly said to be a case of \isi{across-the-board extraction} {\citep{Ross67a,Williams78a}}. (\ref{ex-which-person-did-you-invite-without-thinking}) is traditionally seen as involving an
ordinary gap followed by a parasitic gap. However, for HPSG, all these
gaps have essentially the same status (see \citealt{Levine:Hukari:06}
and \citealt{Chaves:12} for extensive discussion).

\section{The top of the dependency: The diversity of unbounded
dependency constructions}
\label{sec:UDC:Top}

We now look more closely at the top of unbounded dependencies. This is
where most of the diversity of unbounded dependency constructions
resides. They are largely the same at the bottom of the dependency and
in the middle, but at the top of the dependency, they differ from each
other in a variety of ways. We noted at the outset that the
distinctive higher structure in an unbounded dependency construction
may contain a filler, but does not always. In other words, it may be a
head-filler phrase, but it may not, and there are a number of other
possibilities. Moreover, head-filler phrases can have quite different
properties in different constructions.

%\largerpage
In the introduction to this chapter we mentioned
\emph{wh}-interrogatives and relative clauses\footnote{
{See also \crossrefchapterw{relative-clauses} for a more detailed discussion of relative clauses.}} as two examples of
unbounded dependency constructions. In \ili{English} the former always involve
a head-filler-phrase,\footnote{On some analyses of examples like the
  following, \emph{who} is just a subject and not a filler:
\ea
Who knows the answer?
\z
However, for other work, this is a filler just like the
  \emph{wh}-elements discussed here.
} while the later sometimes do but
sometimes do not. There are \emph{wh}-relatives and
non-\emph{wh}-relatives of various kinds. \ili{English}
\emph{wh}-interrogatives and \emph{wh}-relatives look quite similar.
They seem to involve many of the same lexical items: \emph{who},
\emph{which}, \emph{when}, \emph{where}, \emph{why}, and, as the
following show, both may be finite or non-finite:

\eal
\label{ex:UDC:38}
\ex[]{  Who should I talk to \trace{}?}

\ex[]{ I wondered [who to talk to \trace{}].}
\zl

\eal
\label{ex:UDC:39}
\ex[]{
someone [who I should talk to \trace{}]
}
\ex[]{
\label{ex:someone-to-whom-to-talk}
someone [to whom to talk \trace{}]
}
\zl

\noindent
But there are differences. \emph{Wh}-interrogatives, but not
\emph{wh}-relatives, allow \emph{what} and \emph{how}:

\eal
\label{ex:UDC:40}
\ex[]{ What did Kim say \trace{}?}
\ex[*]{the thing [what Kim said \trace{}]}
\zl

\eal
\label{ex:UDC:41}
\ex[]{How did Lee do it \trace{}? }
\ex[*]{the way [how Lee did it \trace{}]}
\zl

\noindent
In \emph{wh}-interrogatives, \emph{which} combines with a following
nominal except in cases of ellipsis. Thus, in (\ref{ex:UDC:42}), \emph{book} is
necessary unless it is clear that books are under discussion.

\ea
\label{ex:UDC:42}
Which book did Kim buy \trace{}?
\z
\noindent
Notice also that non-finite \emph{wh}-relatives only allow a PP as a
filler. Thus, (\ref{ex:UDC:43}) is not possible as an alternative to (\ref{ex:someone-to-whom-to-talk}).

\begin{exe}
  \ex[*]{someone [who to talk to \trace{}]}   \label{ex:UDC:43}
\end{exe}

%\largerpage
\noindent
Thus, the fillers in the two constructions differ in a number of ways.
The heads also differ in that \emph{wh}-interrogatives have auxiliary +
subject order in main clauses (unless the \emph{wh}-phrase is the
subject), something which does not occur in \emph{wh}-relatives.

\emph{Wh}-interrogatives and \emph{wh}-relatives are not the only
unbounded dependency constructions that involve a head-filler phrase.
Topicalisation sentences such as the following are another:

\begin{exe} \ex \begin{xlist} \label{ex:UDC:44}
\ex Beer, I like \trace{}.
\ex To London, I went \trace{}.
\end{xlist}
\end{exe}

\noindent
Unlike\label{udc:page-correlatives-start} \emph{wh}-interrogatives and \emph{wh}-relatives, these are
always finite. Also required to be finite are what have been called
\emph{the}-clauses (\citealp{Borsley:04}; \citealp[490--494, 524--527]{Sag:10a};
\citealp{Borsley:11}; {\crossrefchapteralt[Section~\ref{coord:sec-comparative-correlatives}]{coordination}}), the components of
comparative correlatives\is{comparative correlative} such as (\ref{ex:UDC:45}).

\begin{exe}
\ex \label{ex:UDC:45}
The more I read \trace{}, the more I understand \trace{}.
\end{exe}

\noindent
\emph{The}-clauses have the unusual property that they may contain
the complementiser \emph{that}:

\begin{exe}
\ex \label{ex:UDC:46}
The more that I read \trace{}, the more that I understand \trace{}.
\end{exe}

\noindent
Obviously, this is not possible in \emph{wh}-interrogatives and
\emph{wh}-relatives.

\begin{exe} \ex \begin{xlist} \label{ex:UDC:47}
\ex[*]{I wonder [who that Lee saw \trace{}].}

\ex[*]{the man [who that Lee saw \trace{}]}
\end{xlist}
\end{exe}

\noindent
Within HPSG the obvious approach to the sorts of facts we have just
highlighted involves a number of subtypes of the type
\emph{head-filler-phrase}, as in Figure~\ref{fig:UDC:48}.


\begin{figure}
  \centering

\begin{forest}
type hierarchy
[head-filler-phrase
  [wh-interr-cl]
  [wh-rel-cl]
  [top-cl]
  [the-cl]]
\end{forest}

  \caption{\label{fig:UDC:48}Hierarchy of head-filler phrases}
  
\end{figure}


As was noted in \crossrefchapteralt{properties}, much HPSG work assumes two distinct sets of
phrase types. Assuming this position, \emph{wh-interr-cl} will not just
be a subtype of \emph{head-filler-ph(rase)} but also a subtype of
\emph{interr-cl}, the type \emph{wh-rel-cl} will also be a subtype of
\emph{rel-cl}, and \emph{top-cl} and \emph{the-cl} will both be subtypes
of \emph{decl-cl}. This gives the type hierarchy in Figure~\ref{fig:UDC:49}.

\begin{figure}
  \centering

%  \includegraphics{figures/BB-extraction-function-hier-crop}
\begin{forest} type hierarchy
  [,phantom, for children={!1.l*=1.5}
    [head-filler-phrase
      [wh-interr-cl, edge to=!r2]
    ]
    [interr-cl
      [wh-rel-cl, edge to=!r1, edge to=!r3, no edge]
    ]
    [rel-cl
      [top-cl, edge to=!r1, edge to=!r4, no edge]
    ]
    [decl-cl
      [the-cl, edge to=!r1]
    ]
  ]
\end{forest}
  \caption{\label{fig:UDC:49}Hierarchy of extraction clause types (preliminary)}
\end{figure}


\noindent
Constraints on \emph{interr-cl} will capture the properties that all
interrogatives share, most obviously interrogative semantics.
Constraints on \emph{rel-cl} will capture what all relatives have in
common, especially modifying an appropriate nominal
constituent.\footnote{Non-restrictive relatives can also modify various
  kinds of non-nominal constituents. See \citew{Arnold:04},
  \citew{Arnold:Borsley:08}, and
  \crossrefchapterw[Section~\ref{sec:rc-non-restr-suppl}]{relative-clauses}}.% \inlinetodoopt{Stefan: And there are also free relatives functioning as arguments.}
Finally, constraints on \emph{decl-cl}
will capture the properties on declaratives, especially declarative
semantics.  Constraints on \emph{wh-interr-cl} and \emph{wh-rel-cl}
will ensure that their fillers take the appropriate form. Constraints
on \emph{top-cl} and \emph{the-cl} will restrict their fillers and
also require their heads to be finite. Further complexity is probably
necessary to handle all the facts noted above. To ensure that
non-finite \emph{wh}-relatives only allow a PP filler while finite
\emph{wh}-relatives allow either an NP or a PP filler, it is probably
necessary to postulate two subtypes of \emph{wh-rel-cl}. As for the
fact that \emph{the}-clauses may contain the complementiser
\emph{that}, one way to deal with this is to postulate a subtype of
\emph{head-filler-phrase}, \emph{standard-head-filler-phrase}, with
\emph{wh-interr-cl}, \emph{wh-rel-cl}, and \emph{top-cl} as its
subtypes. This new type will be subject to a constraint preventing its
head from containing a complementiser. The type \emph{the-cl} will not
be a subtype of this new type and hence will be able to contain a
complementiser (see \citealt[13--15]{Borsley:11} for discussion). All this suggests
the type hierarchy in Figure~\ref{fig:UDC:50}. 
%
\begin{figure}
  \centering
%  \includegraphics{figures/BB-extraction-function-hier-full-crop}
\begin{forest} 
type hierarchy
  [,phantom, 
    [head-filler-phrase,
      for siblings={!1.l*=2.5}, !11.l*=1.5,
      [standard-head-filler-phrase, anchor=base east, inner xsep=0
        [wh-interr-cl, edge to=!r2]
      ]
    ]
    [interr-cl
      [wh-rel-cl, edge to=!r11, edge to=!r3, no edge
        [fin-wh-rel-cl]
        [inf-wh-rel-cl]
      ]
    ]
    [rel-cl
      [top-cl, edge to=!r11, edge to=!r4, no edge]
    ]
    [decl-cl
      [the-cl, edge to=!r1]
    ]
  ]
\end{forest}
  \caption{\label{fig:UDC:50}Hierarchy of extraction clause types (final)}
\end{figure}
%
%
This is complex, but then the facts are complex, as we have seen.
Crucially, such a hierarchy allows a straightforward account of both the
similarities and the differences among these constructions.
\label{udc:page-correlatives-end}

We\label{udc:page-no-filler-start} turn now to cases where there is no filler. We start with the
so-called \emph{tough} construction, exemplified by (\ref{ex:UDC:33}), repeated here as
(\ref{ex:UDC:51}).\is{adjective|(}

\begin{exe}
\ex \label{ex:UDC:51}
The professor is hard [to talk to \trace{}].
\end{exe}

\noindent
Here, there is a gap following the preposition \emph{to}, and the
initial NP \emph{the professor} is understood as the object of
\emph{to}. But this NP is not a filler, but a subject. Like any subject,
it is preceded by an auxiliary in an interrogative:

\begin{exe}
\ex \label{ex:UDC:52}
 Is the professor hard [to talk to \trace{}]?
\end{exe}
 
\noindent
Moreover, it is clear that it cannot share a local feature structure
with the gap, since it is in a position associated with nominative case,
whereas the gap is in a position associated with accusative case. This
suggests that adjectives like \emph{hard} may take an infinitival
complement with a \textsc{slash} value containing a nominal local feature
structure which is coindexed with its subject. The coindexing will
ensure that the subject has the right interpretation without getting
into difficulties over case. It seems, then, that we need something like
the lexical description in (\ref{fig:UDC:53}) in order to account for \emph{hard} in examples like (\ref{ex:UDC:51})
and (\ref{ex:UDC:52}):
  
% \inlinetodoobl{JP: Should we explicitely prevent the \slasch-NP to correspond to the \subj? Depends
%   on rest of \slasch constraints. Stefan: Do not understand the problem.}
\ea
\label{fig:UDC:53}
Lexical representation of \textit{tough} adjectives (preliminary):\\
\avm{
[synsem & [local|cat & [head  & adj \\
                        subj  & < NP[index \tag{i}] > \\
          		comps & < VP[vform & inf \\
          			     slash & \{ NP[index & \tag{i}] \} ] > ] ] ]
}
\z

\noindent
But there is more to be said here. \emph{Hard} and its infinitival
complement are the top of a dependency. It is essential that the AP
\emph{hard to talk to} should not have the same \textsc{slash} value
as the infinitival complement \emph{to talk to}. How this should be
prevented depends on what approach to the distribution of
\textsc{slash} values is assumed.  However, if this involves a default
\textsc{slash} Amalgamation Principle\is{principle!slash amalgamation@\SLASH Amalgamation} of the kind discussed in
Section~\ref{sec:UDC:Middle}, it is a fairly simple matter. A default
\textsc{slash} Amalgamation Principle ensures that the \textsc{slash}
value of a word is normally the same as the \textsc{slash} value of
its arguments. We can override the principle in the present case by
giving adjectives like \emph{hard} lexical descriptions of the
following form:

\ea
\label{ex:UDC:54}
Lexical representation of \textit{tough} adjectives (final):\\*
\avm{
[synsem & [local|cat & [head  & adj \\
			subj  & < NP[index \tag{i}]> \\
                        comps & < VP[vform & inf\\
                                     slash & \{ NP[index & \tag{i} ] \} \cupAVM \1 ] > ]\\
	   nonlocal & [slash & \1 ] ] ]
}
\z  

\noindent
This ensures that the \textsc{slash} value of such adjectives is the \textsc{slash} value of
the infinitival complement minus the NP that is coindexed with its
subject. Where this NP is the only item in the complement's \textsc{slash} value,
the adjective will be \mbox[\textsc{slash} \{~\}], and so will the AP that it
heads. However, it is possible to have an additional item in the \textsc{slash}
value, as in the following example, adapted from  \citet[169]{ps2}:

\begin{exe}
\ex \label{ex:UDC:55}
 Which violin is this sonata [easy to play \trace{} on \trace{}]?
\end{exe}

\largerpage
\noindent
Here, \emph{which violin} is understood as the object of \emph{on} and
\emph{this sonata} as the object of \emph{play}. The infinitival
complement \emph{to play on} will have two items in its \textsc{slash} set, one
associated with \emph{which violin} and one associated with \emph{this
sonata}. The constraint in (\ref{ex:UDC:54}) will ensure that only the former appears in the \textsc{slash} set
of \emph{easy}, and hence only this appears in the \textsc{slash} set of
\emph{easy to play on}.

The term ``lexical binding of \textsc{slash}'' is often applied to situations like
this in which a lexical item makes some structure the top of a
dependency. This is a plausible approach to adjectives like \emph{hard}
and also to adjectives modified by \emph{too} or \emph{enough}, as in the
following:

\eal
\ex \label{ex:UDC:56}
Lee is too important for you to talk to.
\ex \label{ex:UDC:57}
Lee is important enough for you to talk to.
\zl
\is{adjective|)}

\noindent
Lexical binding is also a plausible approach to relative clauses which
have not a filler, but a complementiser. This may include \ili{English}
\emph{that} relatives such as that in (\ref{ex:UDC:58}) (although some HPSG work,
e.g.\ \citealt[Section~5.4]{Sag:97}, has analysed \emph{that} as a relative pronoun and hence
a filler):

\begin{exe}
\ex \label{ex:UDC:58}
 the man [that you talked to \trace{}]
\end{exe}

\noindent
If relative \emph{that} is a complementiser, and complementisers, are
heads, as in much HPSG work, it can be given a lexical description
like the one in (\ref{fig:UDC:59}):

\ea
\label{fig:UDC:59}
Lexical representation of relative complementiser \textit{that}:\\*
\avm{
[synsem & [local|cat & [head & [\type*{complementiser}\\
				mod N$'$[index & \tag{i} ] ] \\
                        subj & < > \\
			comps & < S[vform & fin\\
				    slash & \{ NP[index & \tag{i} ] \} \cupAVM \1 ] > ] \\
	   nonlocal & [slash & \1 ] ] ]
}
\z

\noindent
This says that \emph{that} takes a finite clause as its complement and
modifies an NP, that the \textsc{slash} value of the clause includes an NP which
is coindexed with the antecedent noun selected via \textsc{mod}, and that any additional members of the
complement's \textsc{slash} set form the \textsc{slash} set of \emph{that}. Normally there
will be no other members and \emph{that} will be [\textsc{slash}
\{\}].\footnote{This is essentially the approach that is taken to
  relatives in Modern Standard \ili{Arabic} in \citet{Alqurashi:Borsley:12}.}\label{udc:page-no-filler-end}

\largerpage
Further issues arise with zero relatives, which contain neither a filler
nor a complementiser, such as the following \ili{English} example:

\begin{exe}
\ex \label{ex:UDC:60}
 the man [you talked to \trace{}]
\end{exe}

\noindent
For \citet[Section~6]{Sag:97}, these are one type of non-\emph{wh}"=relative and are
required to have a \textsc{mod} value coindexed with an NP in the \textsc{slash} value of
the head daughter. But an issue arises about semantics. Assuming the
main verb in a zero relative has the same semantic interpretation as
elsewhere, a zero relative will have clausal semantics and not the
modifier semantics that one might think is necessary for a nominal
modifier. Sag's solution is to propose a special subtype of
\emph{head-adjunct-phrase} called \emph{head-relative-phrase}, which
allows a relative clause with clausal semantics to combine with a
nominal and be interpreted in the right way. One might well wonder how
satisfactory this approach is.

\citet[Section~5.4]{Sag:10a} shows that it is a simple matter to assign modifier semantics
to a relative clause where the basic clause is the daughter of some
other element, as it is when there is a filler or a complementiser. The
basic clause can have clausal semantics, and the mother can have modifier
semantics. This suggests that zero relatives, too, might be analysed as
daughters of another element with modifier semantics. One might do this,
as \citet[531]{Sag:10a} notes, with a special unary branching phrase
type \citep[Section~10.3.2]{Mueller99a}.
Alternatively, one might postulate a phonologically null counterpart of
relative \emph{that}.\footnote{This is the approach that is taken to
  zero relatives in Modern Standard \ili{Arabic} in \citet[Section~4]{Alqurashi:Borsley:12}.}

There are various other issues about the top of the dependency.
Consider, for example, cleft sentences such as (\ref{ex:UDC:61}).

\begin{exe}
\ex \label{ex:UDC:61}
 It was on the table that he placed the book \trace{}.
\end{exe}

\noindent
Clefts consist of \emph{it}, a form of \emph{be}, a focused constituent,
and a clause with a gap. In (\ref{ex:UDC:61}) the focused constituent is a PP and so
is the gap. It looks, then, as if the focused constituent shares its
main properties with the gap in the way that a filler would. However,
there are also clefts where it is clear that the focused constituent
does not share an index with the gap. Consider e.g.\ the following:

\begin{exe}
\ex \label{ex:UDC:62}
It's me that \trace{} likes beer.
\end{exe}

%\largerpage
\noindent
Here the focused constituent is first person singular, but the gap is
third person singular, as shown by the form of the following verb. Given
the standard assumption that person, number and gender features are a
property of indices, it follows that they cannot have the same index.
There are important challenges here.

Agreement in \ili{German} may shed some more light on this: 

\eal
\label{ex:GermanRelAgr}
\ex[]{
\gll Da    habe              ich,  der               /  die sonst immer rechtzeitig kommt, doch tatsächlich verschlafen.\\
     there have.\textsc{1sg} I     who.\textsc{sg.m} {} who.\textsc{sg.f} otherwise always on.time come.\textsc{3sg} indeed verily overslept\\
\glt `I,  who is otherwise always on time, have indeed overslept.'} 
\ex[]{
\gll Da habe ich,  der ich sonst immer rechtzeitig komme, doch tatsächlich verschlafen.\\
     there have.\textsc{1sg} I who.\textsc{sg.m} I otherwise always on.time come.\textsc{1sg} indeed verily overslept\\
\glt `I,  who is otherwise always on time, have indeed overslept.'}
\zl

\noindent 
In (\ref{ex:GermanRelAgr}a), we find a reduced agreement pattern in
number and gender between the relative pronoun and the antecedent
noun, to the exclusion of person. Within the relative clause, however, we find full person/number subject agreement on the verb. In (\ref{ex:GermanRelAgr}b), however, the relative pronoun is post-modified by the pronoun \textit{ich} `I', triggering full  agreement with both the antecedent noun and the embedded verb. 
\ili{French}, by contrast, observes full agreement of all three \textsc{index} features: 

\begin{exe}
 \ex \label{ex:FrenchRelSAgr} \gll C'est moi qui suis venu(e).\\
  it's me who am come.\textsc{m/f}\\
\glt `It's me who came.' 
\end{exe}
 
\noindent
Thus, relative pronouns and complementisers seem to differ cross-linguistically as to the features
which show agreement with the antecedent.  

Also quite challenging are free relatives\is{relative clause!free|(}. They look rather like
head-filler phrases. The initial constituent of a free relative behaves
like a filler, reflecting the properties of the gap.

\eal
\ex \label{ex:UDC:63}
whichever student you think knows/*know the answer
\ex \label{ex:UDC:64}
whichever students you think know/*knows the answer
\zl

\noindent
But the initial constituent also behaves like a head, determining the
distribution of the free relative.

\begin{exe}
\ex \label{ex:UDC:65}
\begin{xlist}
  \ex[]{Kim will buy what(ever) Lee buys.}
  \ex[*]{Kim will buy where(ever) Lee goes.}
  
\end{xlist}
\end{exe}


\begin{exe}
\ex \label{ex:UDC:66}
\begin{xlist}
  \ex[]{Kim will go where(ever) Lee goes.}
  \ex[*]{Kim will go what(ever) Lee buys.}
\end{xlist}
\end{exe}

\noindent
In case languages like \ili{German}, the matching effect generally includes case
specifications \citep{Mueller:99a}. % In order to reconcile 
% mismatches between the case requirement that is internal to the relative
% clause and the one that is selected for the free relative as whole,
% \citet{Mueller:99a} suggests a unary schema that mediates between the
% different case requirements.  

Most work on free relatives has assumed that the initial constituent is a filler and not a head
\citep{Groos:Riemsdijk:81,Grosu:2003} or a head and not a filler
\citep{Bresnan:Grimshaw:78}. But the obvious suggestion is that it is
both a filler and a head, a position espoused in
\citet[Chapter~12.6]{Huddleston02}. This idea can be implemented within HPSG by
analysing free relatives and head-filler phrases as subtypes of
\type{filler-phrase}, as shown in Figure~\ref{fig:UDC:67}. See \citet{borsley:2020} for an application of this approach to Welsh.


\begin{figure}
  \centering
\begin{forest}
type hierarchy
[filler-phrase 
  [head-filler-phrase]
  [free-relative]]
\end{forest}

\caption{\label{fig:UDC:67}Hierarchy of filler phrases}
\end{figure}

\type{filler-phrase} will be subject to a constraint like that proposed earlier
for head-filler phrases except that it will say nothing about the
head-daughter. \type{head-filler-phrase} will be subject to a constraint
identifying the second daughter as the head, while \type{free-relative}\is{schema!free-relative} will
be subject to a constraint identifying the first daughter as the head
(among other things).\footnote{The constraint on free relatives will also
  need to ensure that the first daughter takes the appropriate form and
  that the second daughter is finite.}

Naturally, there may be complications here. \ili{German}, for example,
has some free relatives in which the case of the \emph{wh}"=element
differs from that which the position of the free relative leads one to
expect: e.g. free relatives with a dative or PP filler can be used in
contexts where a less oblique argument is required, like a nominative
or accusative NP \citep[Section~3]{Bausewein90}. This looks like a problem for
the idea that the initial constituent is a head, but it may not be if
we adopt the Generalised Head Feature Principle\is{principle!Head Feature!Generalized} of \citet[\page 33]{GSag2000a-u}
and regard the difference in \textsc{head} and/or \textsc{case} values
between head daughter and mother as specific overrides enforced by the
free-relative rule.\footnote{\citet{Mueller:99a} pursues a rather
  different approach to German free relatives, in which the initial
  constituent is not a head. Differences between the initial
  constituent and the free relative are unproblematic for this
  approach, but it needs a mechanism to account for the similarities
  between them.}%
\is{relative clause!free|)}

\section{Resumptive pronouns}
\label{sec:UDC:ResumptivePronouns}

Ever since \citet{Vaillette:01}, resumption has been treated as an
unbounded dependency within HPSG, on a par with \textsc{slash}
dependencies, rather than as a case of anaphoric binding. The main
motivation for treating resumption similar to extraction lies with the
fact that in a variety of languages dependencies involving a
pronominal at the bottom of the dependency behave similarly to UDCs
involving a gap at the extraction site.

\citet{Vaillette:01} investigates resumption in \ili{Hebrew} and shows on
the basis of across-the-board (ATB) extraction\is{across-the-board extraction}, parasitic gaps, and
crossover that resumptive dependencies are indistinguishable from gap
dependencies except for their reduced sensitivity to extraction
islands. In order to reconcile the UDC-like properties of resumption
with the difference in island sensitivity, he introduces a dedicated
non-local feature \textsc{resump}. While using separate features for
resumptive pronouns and gaps easily makes them distinguishable for the
purposes of island constraints, it certainly has the drawback that
formulation of the ATB constraint becomes quite cumbersome.
The following example illustrates mixing of gaps and resumptives in
ATB extraction in \ili{Hebrew}: 

\begin{exe}
\ex \label{ex:HebATB}{
\longexampleandlanguage{
\gll kol profesor$_i$ še dani roce lehazmin \trace{}$_i$ aval lo maarix \textnobf{ʔoto}$_i$ maspik\footnotemark\\
     every professor that Dani wants to.invite {} but not esteems him enough\\}{Hebrew}
\footnotetext{\citet[78]{sells_p84}}
\glt `every professor that Dani wants to invite but doesn't respect enough'
}
\end{exe}

\noindent
Subsequent work on \ili{Persian} \citep{Taghvaipour2005a}, \ili{Hausa} \citep{Crysmann:12}, and \ili{Welsh}
\citep{Borsley:13} essentially follows
Vaillette, using ATB extraction as the main indicator for treating
resumptive dependencies in a similar way to gap dependencies. What all these
works have in common is that they rely on a single non-local
feature, namely \textsc{slash} for both types of dependencies. In
particular, these authors argue that mixing of strategies, as
illustrated in (\ref{ex:HebATB}) for \ili{Hebrew} and in (\ref{ex:HauATB})
for \ili{Hausa}, suggests that both extraction
strategies should be captured using a single non-local feature,
i.e.\ \textsc{slash}. Despite this commonality, however, approaches
differ as to how gap and resumptive dependencies are distinguished, if
at all.

In his work on \ili{Welsh} unbounded dependencies,
\citet{Borsley.2010} observes that the choice between gap and
resumptive pronoun is essentially determined by properties of the
immediate environment of the bottom of the dependency: i.e.\ while
possessors of nouns and complements of prepositions require a
resumptive element when extracted, subjects, as well as direct objects
of finite and non-finite verbs, only extract by means of filler-gap
dependencies. Thus, the distribution of gaps vs.\ resumptives is
practically disjoint.

Furthermore, he reports evidence that resumptives and gaps also
pattern alike with respect to island constraints: while extraction out
of the clausal complement in a complex NP is fine, with either a gap
or a resumptive at the bottom, extraction out of a relative clause leads to
ungrammaticality, again, independent of whether we find a gap or a
resumptive.\footnote{The examples are from \citet[91--92]{Borsley.2010}.}

% Cite ex. (44) and (45)
\eal
\ex[]{
\label{ex-dymar-r-dyn-resumptive}
\longexampleandlanguage{
\gll Dyma ’r dyn y credodd Dafydd [y si [y gwelodd Mair (o)]].\\
     here.is the man \textsc{prt} believe.\textsc{past.3sg} Dafydd \spacebr{}the rumour
     \spacebr{}\textsc{prt} see.\textsc{past.3sg} Mair \hphantom{(}he\\}{Welsh}
%\footnotetext{\citet[91]{Borsley.2010}}
\glt `Here’s the man who David believed the rumour that Mair saw.'}
\ex[]{ 
\gll Dyma ’r dyn y credodd Dafydd [y si [y cest ti ’r llythyr ’na ganddo (fo)]].\\
     here.is the man \textsc{prt} believe.\textsc{past.3sg} Dafydd \spacebr{}the rumour
     \spacebr{}\textsc{prt} get.\textsc{past.2sg} you the letter \textsc{dem} with.\textsc{3sg.m} \hphantom{(}him\\
%\footnotetext{\citet[92]{Borsley.2010}}
\glt `Here’s the man who David believed the rumour that you got that letter from.'}
\ex[*]{ 
\gll Dyma ’r ffenest darais i [’r bachgen [dorrodd (hi) ddoe]].\\
     that.is the window hit.\textsc{past.1sg} I \spacebr{}the boy \spacebr{}break.\textsc{past.3sg} \hphantom{(}she yesterday\\
%\footnotetext{\citet[92]{Borsley.2010}}
    }
\zl

\noindent
Moreover, with respect to the across-the-board (ATB)\is{across-the-board extraction} constraint,
resumptives and gaps show the same behaviour as observed for \ili{Hebrew},
easily permitting mixing. In addition, \ili{Welsh} also has certain
extraction path effects which are the same in gap and resumptive
dependencies (see \citealp{Borsley.2010} for details).

Given that the distribution of gaps and resumptives is regulated by
the locally selecting head at the bottom of the dependency and that
there is no need to distinguish the two types of dependencies along
the extraction path (middle), \citet[97]{Borsley.2010} formulates what is
probably the most simple and straightforward approach to
resumption. In essence, he proposes ``that we need structures in which
a slashed preposition or noun has not a slashed argument but a
pronominal argument coindexed with its \textsc{slash} value''. Consequently,
he extends Slash Amalgamation to optionally include a
\textsc{slash} element coindexed with an unslashed pronominal
argument. This move licenses \ili{Welsh} resumptives in a structure like the
one in Figure~\ref{fig:WelshResump} below.

\begin{figure}
  \centering
\begin{forest}
sm edges without translation
[%
\avm{
   [head & \1 prep $\lor$ noun \\
    slash & \{ \2 NP$_i$ \}]}
	[%
	\avm{
	  [head & \1 \\
	   slash & \{ \2 \} \\
	   arg-st & < \ldots, \3, \ldots{} >] }]
        [\ldots]
	[\ibox{3} NP$_i$]
	[\ldots]]
\end{forest}

    \caption{\label{fig:WelshResump}Representation of Welsh resumptives}  
\end{figure}

\largerpage
Thus, the only difference between gaps and resumptives on his account
is that the former give rise to a reentrancy of an element in
\textsc{slash} with a \textsc{local} value on \textsc{arg-st}, whereas
the latter merely involve reentrancy of \textsc{index} values (between
an NP \textit{local} on \textsc{slash} and an NP \textit{synsem} on
\textsc{arg-st}).

The respective distribution of gaps and resumptives are finally
accounted for by means of constraints on the binding theoretical
status of the element at the bottom of the the dependency,
i.e.\ \textit{ppro} for resumptives and \textit{npro} for gaps. See \crossrefchapterw{binding} on
Binding Theory in HPSG.

Borsley's decision to locate the resumptive function on the selecting
head, rather than on the pronominal, not only provides a good match for
the \ili{Welsh} data, but it also addresses McCloskey's generalisation
\citep[192]{mccloskey02:_resum_succes_cyclic_local_operat} that resumptives
are always the ordinary pronouns, since no lexical ambiguity between
slashed and unslashed pronouns is
involved.\footnote{Cf.\ e.g.\ \citet[\page 54--55]{AbeilleGodard07} for an ambiguity
  approach, treating reentrancy of \textsc{local} and \textsc{slash}
  as optional for \ili{French} pronouns. }

In contrast to Borsley, who developed his theory of resumption on the
basis of a language where the distribution of gaps vs.\ resumptives is
entirely regulated by the immediate local environment and no
difference in island sensitivity could be observed,
\citet{Crysmann:12} developed an alternative account for \ili{Hausa}, a
language where the distributions of gaps and resumptives partially
overlap at the bottom of the dependency and where resumptive
dependencies observe different locality constraints when compared to
filler-gap dependencies.

\largerpage
\ili{Hausa} patterns with a number of resumptive languages, including \ili{Welsh}, in that use of a
resumptive element is obligatory for
complements of a preposition or the possessor of a noun. With direct and indirect objects, however,
both resumptives and gaps are possible, as \citegen[534]{jaggar01:_hausa} examples in (\mex{1}) show: 

\eal
\label{ex:HauIO}
\ex{
\longexampleandlanguage{
\gll mutā̀nên dà sukà ƙi sayar wà \trace{} dà àbinci sukà fìta\\
     men \textsc{rel} \textsc{3.p.cpl} refuse sell to {} with food \textsc{3.p.cpl} left\\}{Hausa}
%\footnotetext{\citet[534]{jaggar01:_hausa}}
\glt `The men they refused to sell food to left.'} 
\ex{
\gll mutā̀nên dà sukà ƙi sayar \textnobf{musù} dà àbinci sukà fìta\\
      men \textsc{rel} \textsc{3.p.cpl} refuse sell to.them with food \textsc{3.p.cpl} left\\
%\footnotetext{\citet[534]{jaggar01:_hausa}}
\glt `The men they refused to sell food to left.'}
\zl


\noindent
In (\ref{ex:HauIO}), both a bare dative marker \textit{wà} `to' is
possible (with a gap), and a dative pronoun \textit{musù} `to.them'.

Moreover, gap and resumptive dependencies do behave differently with
respect to strong islands: while extraction out of a relative clause
or \emph{wh}"=island is impossible for gap dependencies,
relativisation out of these islands is perfectly fine with
resumptives.

\ea
\label{ex:HauResLongIO}
\longexampleandlanguage{
\gll Gā̀      tābōbîn$_j$ dà          Àli ya                 san  mùtumìn$_i$ dà           zâi$_i$            yī \textnobf{musù}$_j$ /  *{wà \trace{}$_j$}   kwālī\footnotemark\\
     here.is cigarettes \textsc{rel} Ali \textsc{3.s.m.cpl} know man         \textsc{rel} \textsc{3.s.m.fut} do to.them           {} \hphantom{*}to     box\\}{Hausa}
\footnotetext{\citet[84]{tuller_l86}}
    \glt `Here are the cigarettes that Ali knows the man that (he) will make a box for.'
\z

\noindent
\citet{Crysmann2012b-u} further emphasises that relativisation (which
may escape strong islands) resembles anaphoric relations, whereas
filler-gap dependencies, as observed with \emph{wh}"=fronting, require
matching of category as well. He therefore correlates relative
complementisers and resumptives with minimal \textsc{index} sharing,
whereas filler-head structures, as well as gaps will require sharing
of entire \textsc{local} values: while filler-head structures impose
this stricter constraint at the top of the dependency, gaps obviously
do so at the bottom. In order to express constraints on locality,
\citet{Crysmann:12,Crysmann:16} proposes that \textsc{slash} elements
(of type \textit{local}) should be distinguished as to their weight,
cf.~Figure~\ref{fig:local}: while the type \textit{local} always
minimally includes indexical information, its subtypes
\textit{full-local} and \textit{weak-local} differ as to the amount of
additional information that must or must not be present. For
\textit{full-local}, which is the appropriate value introduced by
\textit{synsem} (cf.~Figure~\ref{fig:synsem}), this includes
categorial and full semantic information, whereas exactly categorial
information is excluded for \textit{weak-local}.



%\begin{figure}[htb]
%  \centering
%  
%  \begin{tabular}{cc}
%    \multicolumn{2}{c}{
%    \rnode{l}{\begin{avm}
%      \[\asort{local} 
%      cont & \[index & ind \]\]
%    \end{avm}}
%}\\[4em]
%    \rnode{f}{\begin{avm}
%      \[\asort{full-local}
%        cat & cat\\
%      \]
%    \end{avm}}
%    &
%      \rnode{w}{\begin{avm}
%      \[\asort{weak-local}
%        cont & \[rels \< ~ \>\]
%      \]
%    \end{avm}}
%  \end{tabular}
%  \psset{angleA=-90,angleB=90,nodesep=1pt,arm=0pt,linewidth=.5pt}
%  \ncdiag{l}{w}
%  \ncdiag{l}{f}
%
%
%
%  \caption{\label{fig:local}Hierarchy of \textit{local} \citep{Crysmann:12}}
%
%\end{figure}



\begin{figure}
	\centering

\begin{forest}
[%
\avm{
	[\type*{local} 
	cont &	[index & ind ] ]
}%,l sep=1.5cm
	[%
	\avm{
		[\type*{full-local}
		cat & cat ]
	}]
	[%
	\avm{
		[\type*{weak-local}
		cont &	[rels <  > ] ]
	}]
]
\end{forest}  
  
%\inlinetodoobl{Stefan: The types are different in the cited reference.}
\caption{\label{fig:local}Hierarchy of \textit{local} \citep[\page 202]{Crysmann:16}}
  
\end{figure}

\begin{figure}
  \centering
\begin{forest}
sm edges without translation
[%
\avm{
	[\type*{synsem}
	loc & full-local \\
	nonloc & non-local ]
}
	[%
	\avm{
		[\type*{slashed}
		loc    &	[cont|index & \1] \\
		nonloc &	[slash & \{[cont|index & \1 ] \} ] ]
	}
		[%
		\avm{
			[\type*{gap}
			loc    & \1 full-local \\
			nonloc & [slash & \{ \1 \} ] ]
		} ]
		[\type{resump}] ] ]
\end{forest}
%\inlinetodoobl{Stefan: This is not in the cited reference.}
\caption{\label{fig:synsem}Hierarchy of \textit{synsem} objects \citep[\page 202]{Crysmann:16}}

\end{figure}

The hierarchy of \textit{local} types provides for the possibility
that \textit{local} types on \textsc{slash} may only be partially
specified: while gaps and filler-head structure require full
reentrancy of a (\textit{full-local}) \textsc{local} value,
resumptives may be non"=committal with respect to the weight
distinction, only imposing the minimal index-sharing constraint. This
ensures that both resumptives and gaps can be found at the bottom of a
strong UDC with e.g.\ a \emph{wh}"=filler. Conversely, islands can narrow down
the nature of \textsc{slash} elements to only pass on a \textsc{slash}
set of \textit{weak-local}, such that resumptives, but not gaps, will
be licensed at the bottom in the case of long relativisation.

Underspecification of \textit{local} at the bottom of a resumptive
dependency permits mixing of gap and resumptive strategies in
ATB extraction, as illustrated by the example below:

\ea{
\label{ex:HauATB}
\longexampleandlanguage{
\gll [àbōkī-n-ā]{$_i$} dà [[na zìyartā̀ \trace{}$_i$] àmmā [bàn sā̀mē \textnobf{shì}$_i$ à gidā  ba]]\footnotemark\\
    \spacebr{}friend-\textsc{l-1.s.gen} \textsc{rel} \hphantom{[[}\textsc{1.s.cpl} visit {} but
    \spacebr{}\textsc{1.s.neg.cpl} find \textsc{3.s.m.do} at home \textsc{neg}\\}{Hausa}
\footnotetext{\citet[539]{newman_p00}}
\glt `my friend that I visited but did not find at home'
}
\z

The obvious question is, of course, how these two approaches can be
harmonised in order to yield a unified HPSG theory of resumption.  It
is clear that the theory advanced by \citet{Crysmann:12} makes a more
fine-grained distinction with regard to \textsc{slash} elements and
should therefore be able to trivially account for languages where
there is no difference in locality restrictions between resumptive and
gap dependencies. In the case of \ili{Welsh}, it will suffice to strengthen
the constraints of strong islands, such as relative clauses, to block
passing of any \textit{local} on \textsc{slash}, rather than merely
restricting it to \textit{weak-local}. The other area where the theories
need to be brought closer together concerns the issue of McCloskey's
generalisation, which is straightforwardly derived by a syntactic
theory of resumption, such as Borsley's. Some work in this direction
has already been done: \citet{Crysmann:16} suggests replacing his
original ambiguity approach with an underspecification approach,
essentially following \citet{Borsley.2010} in locating the
disambiguation between pronoun and resumptive function on the
selecting head. While there are still differences of implementation,
general agreement has been obtained that it should indeed be the head
that decides on the pronominal's function, whether this is done via
disjunctively amalgamating the index of a pronominal argument
\citep{Borsley.2010,Alotaibi:Borsley:13}, or else via a more elaborate
system of \textit{synsem} types that integrates more nicely with
standard \textsc{slash} amalgamation \citep{Crysmann:16}.

Similar consensus has been reached with respect to the need to have
more fine-grained control on locality, again irrespective of
implementation details: while \citet{Alotaibi:Borsley:13} exploited
constraints on case marking in order to capture the difference in
locality of resumptives and gaps in Modern Standard \ili{Arabic}, the
weight-based analysis by \citet{Crysmann:17} provides a more
principled account of the data, essentially obviating stipulative
nominative case assignment that fails to correspond to any overtly
observable case marking.

% That claim is the main argument of the thesis and distributed over
% several chapters 
Some questions still remain: \citet[Section~6.5]{Taghvaipour2005a}
suggests that in \ili{Persian}, the distribution of gaps vs.\ resumptives is
partly determined by the constructional properties of the top of the
dependency, showing different patterns for \emph{wh}-extraction, free
relatives and ordinary relatives, and suggests that constructional
properties of the top need to be transmitted via \textsc{slash}.
However, percolation of constructional information across the tree
does not play nicely with basic assumptions of locality within
HPSG. It remains to be seen how the case of \ili{Persian} can be analysed
within the scope of the theories outlined above.

Another case study that deserves integration into the current HPSG
theory of resumption concerns so-called hybrid chains in \ili{Irish}
\citep{assmann10:_does_chain_hybrid_irish_suppor}: in this language,
the most deeply embedded complementisers register the difference
between gaps and resumptives at the bottom, yet complementisers
further up can switch between ``resumptive marking'' and ``gap
marking''. While the authors use a single \textsc{slash} feature for
both types of dependency, the objects in this set remain incompatible,
thereby necessitating a great deal of disjunction. In order to bring
this analysis fully in line with current HPSG, underspecification
techniques may be fruitfully explored.


%\begin{itemize}
%\item Evidence for a \textsc{slash} analysis %\citep{Vaillette:01,Vaillette:02,Taghvaipour2005a,Borsley:10,Borsley:13,Crysmann:12,Crysmann:16,Crysmann:17} 
%\item The nature of resumptive pronouns \citep{Borsley:10,Alotaibi:Borsley:13,Crysmann:12,Crysmann:16,Crysmann:17} 
%\end{itemize}

\section{More on \emph{wh}"=interrogatives}
\label{sec:UDC:MoreWh}

\subsection{Pied piping}

So far, we have concentrated on unbounded dependencies as witnessed by
extraction, captured in HPSG by \textsc{slash} feature
inheritance. Another type of unbounded dependency involves
pied-piping, as illustrated in (\ref{ex:WhPiedPiping}b--d) and
(\ref{ex:RelPiedPiping}b--d), taken from \citet[184]{Ginzburg:Sag:01}.

\begin{exe}
  \ex \label{ex:WhPiedPiping}
  \begin{xlist}
    \ex{I wonder [[\textit{what}] inspired them]. }
    \ex{I wonder [[\textit{whose} cousin] ate the pastry].} 
    \ex{I wonder [[\textit{whose} cousin's dog] ate the pastry].} 
    \ex{I wonder [[to \textit{whom}] they dedicated the building]}
  \end{xlist}
\end{exe}

\begin{exe}
  \ex \label{ex:RelPiedPiping}
  \begin{xlist}
    \ex{the book [[\textit{which}] inspired them] }
    \ex{the person [[\textit{whose} cousin] ate the pastry]} 
    \ex{the person [[\textit{whose} cousin's dog] ate the pastry]} 
    \ex{the person [[to \textit{whom}] they dedicated the building]}
  \end{xlist}
\end{exe}

\noindent
In (\ref{ex:WhPiedPiping}) the \emph{wh}"=word, a pronoun or determiner, that
marks the (embedded) \emph{wh}"=interrogative clause may be arbitrarily deeply embedded inside the filler.  

With relative clauses too, as witnessed by (\ref{ex:RelPiedPiping}), the
relative pronoun may be embedded inside the filler, and, again,
arbitrarily deep. Furthermore, regardless of the level of embedding,
the relative pronoun is coreferent with the antecedent noun, such that
a mechanism is called for that can establish this token identity in a
non-local fashion. This is most evident in languages where relative
pronouns undergo agreement with the antecedent noun, as e.g.\ in
\ili{German}:

\eal
\ex
\gll das Buch, [\textit{das} mich inspirierte]\\
     \textsc{def.n.s} book(\textsc{n})\textsc{.sg} \hphantom[\textsc{rel.n.sg} me inspired\\\hfill(\ili{German})
\glt `the book that inspired me'
\ex
\gll die Person, [\textit{die} mich inspirierte]\\
     \textsc{def.f.sg} person(\textsc{f})\textsc{.sg} \hphantom[\textsc{rel.f.sg} me inspired\\   
\glt `the person that inspired me'
\ex
\gll das Buch, [[\textit{dessen} / *\textit{deren} Rezension] mir gefiel]\\
     \textsc{def.n.sg} book(\textsc{n})\textsc{.sg} \hphantom{[[}\textsc{rel.n.sg.poss} {} \hphantom{*}\textsc{rel.f.sg.poss} review(\textsc{f}).\textsc{sg} me pleased\\
\glt `the book the review of which I liked'
\ex
\gll die Autorin, [[\textit{deren} / *\textit{dessen} Roman] mir gefiel]\\
     \textsc{def.f.sg} author(\textsc{f})\textsc{.sg} \hphantom{[[}\textsc{rel.f.sg.poss} {} \hphantom{*}\textsc{rel.m.sg.poss} novel(\textsc{m}) me pleased\\
\glt `the (female) author whose novel I liked'
\zl

\noindent
In order to capture the fact that the filler of a \emph{wh}"=clause
must contain a \emph{wh}"=word, or that the relative pronoun contained
within the filler of a relative clause must structure-share its
\textsc{index} with the antecedent noun, HPSG builds on previous work
in GPSG \citep[Chapter~5.2]{Gazdar85}, postulating the non-local features
\textsc{que}/\textsc{wh} and \textsc{rel}. \citet[\page 164]{Pollard:Sag:94} have proposed
a single Nonlocal Feature Principle\is{principle!Nonlocal Feature} that generalises from
\textsc{slash} feature percolation to inheritance of \textsc{que} and
\textsc{rel}, defining the value of each non-local feature of the
mother as the set union of the nonlocal features of the
daughters. See, however, \citet[Section~4.2]{Sag:97} and
\citet[Chapter~7]{Ginzburg:Sag:01} for a head-driven formulation of nonlocal feature percolation.

One observation regarding pied piping in languages such as \ili{English} or
\ili{German} pertains to the fact that \emph{wh}"=words tend to surface in
the left periphery of the filler, e.g.~(\ref{ex:WhPiedPipingOblique}a). \citet[194,~fn.~26]{Ginzburg:Sag:01} suggest that
amalgamation of \textsc{que/wh} is restricted to the least oblique
element on \textsc{arg-st}. This enables them to rule out
(\ref{ex:WhPiedPipingOblique}b) while still being able to account for
standard pied-piping with prepositional phrases
(\ref{ex:WhPiedPiping}d).

\begin{exe}
  \ex \label{ex:WhPiedPipingOblique}
  \begin{xlist}
  \ex[]{ I wonder [[whose picture] was on display].}
  \ex[*]{I wonder [[my picture of whom] was on display].}
  \end{xlist}
\end{exe}

\largerpage
\noindent
Indeed, from a cross-linguistic perspective, pied-piping of
prepositions appears to be the far less marked option when compared to
preposition stranding, which appears to be a peculiarity of
\ili{English} \citep[cf.][]{Riemsdijk78a}. This is supported not only by the ban on preposition
stranding in \ili{German}, \ili{French}, and many other languages, but it is also
corroborated by the distribution of resumptives (see Section~\ref{sec:UDC:ResumptivePronouns}).

To summarise, pied piping in HPSG is understood as a phenomenon that
involves a second unbounded dependency: in addition to a
\textsc{slash} dependency between the pied-piped filler and the
extraction site, just like the ones we have discussed throughout this
chapter, \textsc{que} or \textsc{rel} establish dependencies within
the filler itself.\footnote{
See also \crossrefchapterw[Section~\ref{rc:sec-internal-structure}]{relative-clauses} on pied
  piping.
}

\subsection{Multiple \emph{wh}"=questions}

% G&S on \ili{English}

While in languages such as \ili{English}, only one \emph{wh}"=phrase may be
fronted per interrogative clause (and typically one phrase is indeed
fronted), it is nevertheless possible to ask multiple questions, with
additional \emph{wh}"=phrases remaining in situ, as witnessed by
\textit{what} in (\ref{ex:WhMult}).

\begin{exe}
  \ex Who asked who saw what? \label{ex:WhMult}
\end{exe}

\noindent
According to the theory of \citet{Ginzburg:Sag:01}, only fillers in
interrogative clauses are \emph{wh}"=marked, and \emph{wh}"=marking
serves to ensure that a \emph{wh}"=quantifier contained in the filler
is interpreted as a parameter of the local interrogative clause. In
situ \emph{wh}"=phrases, by contrast, are still quantifiers, so they
may scope higher than their syntactic position suggests.
\citet[Section~5.3]{Ginzburg:Sag:01} follow \citet[Section~8.2]{Pollard:Sag:94} in adopting a
\isi{Cooper storage}, which enables them to have the in situ
\emph{wh}"=quantifier in (\ref{ex:WhMult}) retrieved either as a
parameter of the embedded interrogative clause, or as a parameter of
the matrix question. The \textsc{wh} feature thus not only ensures
that a \emph{wh}"=interrogative is marked as such by a filler
containing a \emph{wh}"=word, but it also fixes the semantic scope of
ex-situ \emph{wh}"=phrases to their syntactic
scope.\footnote{\citet{kathol:scope-marking} uses the \textsc{que}
  feature in his analysis of partial \emph{wh}"=fronting in \ili{German}.
 }
In situ \emph{wh}"=quantifiers, by contrast, are permitted to take
arbitrarily wide scope.

\largerpage[1.1]
\begin{sloppypar}
In \ili{Slavic} languages such as \ili{Russian} or \ili{Serbo-Croatian} \citep{Penn:99},
there does not appear to be a constraint on the number of
simultaneously fronted \emph{wh}"=phrases, as illustrated by the
examples in (\ref{ex:WhMultFront}) taken from \citew[163]{Penn:99}.
\end{sloppypar}

\eal
\label{ex:WhMultFront}
\ex[]{
\gll Ko koga si mislio da je voleo?\\
     who whom \textsc{cl.2sg} thought \textsc{comp} \textsc{cl.3sg}
     loved\\\hfill (Serbo-Croatian)
\glt `Who did you think loved whom?'}
\ex[*]{\gll Ko si koga mislio da je voleo?\\
     who  \textsc{cl.2sg} whom thought \textsc{comp} \textsc{cl.3sg} loved\\
   }
\zl

\noindent
Given that HPSG's nonlocal features, and in particular \textsc{slash}
and \textsc{que}/\textsc{wh}, are set valued, multiple \emph{wh}"=fronting is a
rather expected property. In fact, the grammar of \ili{English}
interrogatives as proposed by \citet{Ginzburg:Sag:01} specifically
stipulates that there be only a singleton \textsc{wh} set, and that
head-filler structures cannot be recursive.

The point where \ili{Slavic} multiple fronting poses a challenge is its
interaction with second position clitics: it seems, as witnessed by
the contrast in (\ref{ex:WhMultFront}), that multiple fronted
\emph{wh}"=phrases are treated as a constituent, as far as
linearisation is concerned. \citet{Penn:99} proposes a topological
analysis based on extended word order domains
(\citealt{Reape:90,Reape94a,kathol_a00}{; \crossrefchapteralt[Section~\ref{sec-domains}]{order}}) in order to reconcile multiple fronted
constituents with the second position property: in essence, multiple
fillers are assigned to the same initial topological field and
linearisation of clitics proceeds relative to that same initial field.

% Multiple ex-situ wh 

\subsection{\emph{Wh} in situ}
\label{sec:UDC:WhInSitu}

In the previous subsections, as in most of this chapter, we have
capitalised on ex situ \emph{wh}"=constructions. However, even in
languages like \ili{English}, and even more in \ili{French}, we do find
constructions with clear interrogative semantics where nonetheless the
\emph{wh}"=phrase stays in situ. Moreover, in languages such as
\ili{Japanese} or \ili{Coptic} \ili{Egyptian}, in situ realisation is the norm, rather
than the exception. In this subsection we shall therefore discuss how
HPSG's theory of unbounded dependencies has been put to use to account
for this phenomenon.

In languages such as \ili{English}, where standard \emph{wh}"=interrogatives
are signalled by a \emph{wh}"=phrase ex situ (i.e.\ by a
\emph{wh}"=filler), \citet[Chapter~7]{Ginzburg:Sag:01} identify two types of in
situ \emph{wh}"=questions in \ili{English}: so called reprise (or ``echo'')
questions, which typically mimic the syntax and semantics of the
speech act they are modelled on (e.g.\ an assertion, an order etc.),
and direct in situ interrogatives, the latter being more strongly
restricted
pragmatically. %\pdfmargincomment{BC: We should cite some work on \ili{French} here.}

%\largerpage
However, \emph{wh} in situ may even be an unmarked, or even the default
option for the expression of \emph{wh}"=interrogatives:
\citet[Section~6.2]{Johnson:Lappin:97}, studying Iraqi\il{Arabic!Iraqi|(} Arabic, made the important
observation that \emph{wh}"=fronting is optional in this language,
posing a challenge for transformational\is{Minimalism} models at the time.  In Iraqi
Arabic\il{Arabic!Iraqi}, a \emph{wh}"=interrogative may be realised ex situ, as in
(\ref{ex:Iraqi}a) or in situ, as in (\ref{ex:Iraqi}b).

\eal
\label{ex:Iraqi}
\ex[]{
\gll Mona shaafat meno?\footnotemark\\
     Mona saw whom\\\hfill{(Iraqi\il{Arabic!Iraqi|(} Arabic)}
\footnotetext{\citet[318]{Johnson:Lappin:97}}
\glt `Who did Mona see?'}
\ex[]{
\gll Meno shaafat Mona?\footnotemark\\
     who saw Mona\\
\footnotetext{\citet[320]{Johnson:Lappin:97}}
\glt `Who did Mona see?'}
\zl

\noindent
They propose a straightforward analysis within HPSG, suggesting to
drop what can be regarded as a parochial constraint of \ili{English} and
related languages, and allow \textsc{que} feature percolation from the
right clausal daughter.
 
What is more, they note that \emph{wh} in situ and ex situ strategies
do observe different locality restrictions, thereby lending further
support to a difference in the type of nonlocal feature
involved. While feature percolation for in situ
\emph{wh}"=constructions cannot escape finite clauses (cf.\ the
contrast in (\ref{ex:IraqiLoc}a,b), ex situ \emph{wh}"=interrogatives,
involving a \textsc{slash} dependency, are obviously not subject to
this restriction, as witnessed by (\ref{ex:IraqiLoc}c).\footnote{The examples in (\ref{ex:IraqiLoc}) are from \citew[318]{Johnson:Lappin:97}.}


    
\eal \label{ex:IraqiLoc}
\ex[]{
\gll Mona raadat tijbir Su'ad tisa'ad meno?\\
      Mona wanted to.force Su'ad to.help who\\\hfill{(Iraqi\il{Arabic!Iraqi|(} Arabic)}
\glt `Who did Mona want to force Su'ad to help?'}
\ex[*]{
\gll Mona tsawwarat Ali ishtara sheno?\\
     Mona thought Ali bought what\\
}
\ex[]{
\gll Sheno tsawwarit Mona Ali ishtara?\\
      what thought Mona Ali bought\\
\glt `What did Mona think Ali bought?'}
\zl

Yet, even this constraint, while valid for Iraqi Arabic\il{Arabic!Iraqi|)}, must be
considered language-specific: \citet{Crysmann:Reintges:14} study
Coptic Egyptian\il{Egyptian!Coptic}, where \emph{wh} in situ is the norm. They observe that the
scope of an in situ \emph{wh}"=phrase is determined by the position of
a relative complementiser and note that it can easily escape finite
clauses, as shown in (\ref{ex:Coptic}).

\ea
\label{ex:Coptic}
\gll ere əm=mɛɛʃe tʃoː əmmɔ=s [tʃe ang nim]?\footnotemark\\
     \textsc{rel} \textsc{def.pl}=crowd say \textsc{prep=3f.sg} \spacebr{}that I who\\\hfill{(Coptic Egyptian\il{Egyptian!Coptic})}
\footnotetext{\citet[72]{Crysmann:Reintges:14}}
\glt `Who do the crowds say that I am?'  (Luke 9,18)
\end{exe}

\noindent
Their analysis builds on \citet{Johnson:Lappin:97}, yet suggests that \textsc{que} percolation in this language may be as unrestricted as \textsc{slash} percolation.

% \pdfmargincomment{BC: Still missing: Partial-\emph{wh}-movement \citep{McDaniel:89}}

\section{Extraposition}
\label{sec:UDC:Extraposition}

Another\is{extraposition|(} non-local dependency is extraposition, the displacement of a
constituent towards the right. Extraposition is most often observed
with heavy constituents, such as relative clauses or complement
clauses, but it has also been attested with lighter constituents such
as prepositional phrases and non-finite VPs. In \ili{German}, where
extraposition is particularly common in general
\citep{uszkoreit:etal:98}, extraposed material can be extremely light,
including adverbs and NPs (see \citealt[Section~13.1]{Mueller99a} and
\citealt[\page ix--xi]{Mueller2002b} for examples).


%\largerpage
Apart from the obvious difference in the linear direction of the
process, extraposition also contrasts with e.g.\ filler-gap
dependencies with respect to the domain of locality: e.g.\ island
constraints that have been claimed to hold for extraction to the left,
such as the \isi{Complex NP Constraint} \citep[Section~4.1]{ross_j67}, clearly do not hold
with complement clause nor relative clause extraposition, as the
following examples by \citep[\page 4, 11]{Keller:94a} and
\citet[G.][\page 219]{mueller96:_extrap_and_succes_cyclic} show:

\eal
\ex[]{
\gll Planck hat [die Entdeckung \_$_i$] gemacht, [daß Licht Teilchennatur hat.]$_i$\footnotemark\\
     Planck has \spacebr{}the discovery {} made \spacebr{}that light particle.nature has\\\hfill{(German)}
\footnotetext{\citet[4]{Keller:94a}}
\glt `Planck made the discovery that light has particle properties.'}
\ex[*]{
\gll [Daß Licht Teilchennatur hat]$_i$  hat Planck [die Entdeckung \_$_i$] gemacht.\footnotemark\\
      \spacebr{}that light particle.nature has  has Planck \spacebr{}the discovery {} made \\
\footnotetext{\citet[11]{Keller:94a}}}
\zl
\eal
\ex[]{
\gll Ich habe [eine Frau   \_$_i$] getroffen, [die das Stück gelesen hat]$_i$.\footnotemark\\
     I   have \spacebr{}a woman {} met \spacebr{}who the play read has\\
\footnotetext{\citet[G.][\page 219]{mueller96:_extrap_and_succes_cyclic}}
\glt `I met the woman who has read the play.'}
\ex[*]{
\gll [die das Stück gelesen hat]$_i$, habe ich [eine Frau \_$_i$] getroffen.\footnotemark\\
      \spacebr{}who the play read has have I \spacebr{}a woman {} met \\
\footnotetext{\citet[G.][\page 219]{mueller96:_extrap_and_succes_cyclic}}
}
\zl

Conversely, while extraction to the left can easily cross finite clause
boundaries (\mex{1}), extraposition is said to be clause-bound, i.e.\ subject to
the \isi{Right Roof Constraint} \citep[Section~5.1.2]{ross_j67}.   

\ea
\gll Was$_i$ hat Hans gesagt, [daß           wir \_$_i$ kaufen sollten]?\\
     what    has Hans said    \spacebr{}that we {} buy should\\\hfill{(German)}
\glt `What did Hans say that we should buy?'
\z
\eal
\ex[]{
\gll [Daß Peter sich auf das Fest \_$_i$ gefreut hat, [das Maria veranstaltet hat,]$_i$ ] hat niemanden gewundert.\footnotemark\\
      \spacebr{}that Peter SELF on the party {} looked.forward has \spacebr{}which Maria organised has {} has no.one surprised\\
\footnotetext{\citew[\page 11]{wiltschko94:_extrap_in_german}, \citew[\page 10]{Keller:94a}}
\glt `That Peter was looking forward to the party that Maria had organised, did not surprise anyone.'}
\ex[*]{\gll [Daß Peter sich auf das Fest \_$_i$ gefreut hat],  hat niemanden gewundert, [das Maria veranstaltet hat]$_i$\footnotemark\\
      \spacebr{}that Peter SELF on the party {} looked.forward has  has no.one surprised \spacebr{}which Maria organised has\\
\footnotetext{\citew[\page 11]{wiltschko94:_extrap_in_german}, \citew[\page 10]{Keller:94a}}}
\zl



\subsection{Extraposition via non-local features}
\label{udc:sec-extra-feature}

Given the non-local nature of extraposition, a natural approach to
this construction is by means of non-local features. Because
extraposition differs from extraction in both direction and locality,
\citet{Keller:95} and \citet[Section~13.2]{Mueller99a}
have proposed a distinct non-local feature \textsc{extra} to capture
this rightward"=oriented dependency. 
Similar to lexical \textsc{slash} introduction, \citet[\page 303]{Keller:95} assumes two lexical
extraposition rules, one for  complement extraposition, the other for
adjunct extraposition. 

%%% Original version
% \ea
% Complement Extraposition Lexical Rule\\

% \avm{
% [comps & \1 \+ <[loc & \4 [cat [head & verb $\lor$ prep \\
% 			         comps & < > ] ] ] > \+ \2 \\
% \punk{nonloc|inher|extra}{\3} ]}
% $\mapsto$ \\
% \flushright \avm{[comps & \1 \+ \2 \\
%              \punk{nonloc|inher|extra}{\3 \cupAVM \{ \4 \}} ]}
% \z
% % \inlinetodoopt{Stefan: This LR would not work for relational nouns, since they select a complement
% %   and the adjuncts modify an \nbar. Mention this?}
% \eas
% Adjunct Extraposition Lexical Rule\\
% \avm{
% 	[loc & \2 [cat|head & noun $\lor$ verb ] \\
%     \punk{nonloc|inher|extra}{\1} ]}
% $\mapsto$ \\
% \flushright\avm{
% [loc|cont \3\\
%  nonloc|inher|extra & \1 \cupAVM \{[cat & [head & [\type*{prep $\lor$ rel}
%                                                   mod|loc & \2 ] \\
%                                    cont & \3 ] ] \} ]
% }
% \zs

\ea
Complement Extraposition Lexical Rule\is{lexical rule!Complement Extraposition}:\\

\avm{
[comps & \1 \+ <[loc & \4 [cat [head & verb $\lor$ prep \\
			         comps & < > ] ] ] > \+ \2 \\
\punk{nonloc|extra}{\3} ]}
$\mapsto$ \\
\flushright\avm{[comps & \1 \+ \2 \\
             \punk{nonloc|extra}{\3 \cupAVM \{ \4 \}} ]}
\z
% \inlinetodoopt{Stefan: This LR would not work for relational nouns, since they select a complement
%   and the adjuncts modify an \nbar. Mention this?}
\eas
Adjunct Extraposition Lexical Rule:\\
\avm{
	[loc & \2 [cat|head & noun $\lor$ verb ] \\
    \punk{nonloc|extra}{\1} ]}
$\mapsto$ \\
\flushright\avm{
[loc|cont \3\\
 nonloc|extra & \1 \cupAVM \{[cat & [head & [\type*{prep $\lor$ rel}
                                                  mod|loc & \2 ] \\
                                   cont & \3 ] ] \} ]
}
\zs


The complement extraposition rule is straightforward: it removes a
valency from the \textsc{comps} list and inserts its \textsc{local}
value into the \textsc{extra} set. 

As for adjunct extraposition, the lexical rule  equally inserts an
element into the \textsc{extra} set, yet constrains it to be a
modifier that selects for the local value of the lexical head (via
\textsc{mod}). 

Since \textsc{extra} is a nonlocal feature, percolation up the tree,
i.e.\ the middle of the dependency, is handled by the Nonlocal Feature
Principle\is{principle!Nonlocal Feature}
\citep[\page 164]{Pollard:Sag:94}.

At the top, the Head-Extra Schema will bind all extraposition
dependencies, which are realised as extraposed daughters.\footnote{We
  give a slightly simplified version of the schema, ignoring the
  \textsc{periphery} feature that was introduced to control for spurious
  ambiguity that could arise from string-vacuous extraposition. See
  \citet[\page 304--305]{Keller:95} for details and
  \citet{Crysmann2005a-u} for an alternative solution. }
\begin{exe}
  \ex Head-Extra Schema:\is{schema!Head-Extra}

%\inlinetodoobl{Stefan: \textsc{to-bind} is never explained.}
% \avm{
% 	[synsem & [nonloc|extra & \{ \} ] \\
%          dtrs & [head-dtr & [synsem|nonloc|to-bind|extra & \upshape \textsf{loc(\1)} ] \\
% 	 extra-dtrs & \1 ] ]
%        }
       \avm{
	[synsem & [nonloc|extra & \tag{x} ] \\
         dtrs & [head-dtr & [synsem|nonloc|extra & \{\tag{1}, \ldots{}, \tag{n}\}
         \cupAVM \tag{x}] \\
	 non-hd-dtrs & < [synsem|loc & \tag{1}], \ldots{}, [synsem|loc & \tag{n}]> ] ]
}
\end{exe}

Order of extraposed daughters amongst each other and with respect to
the head is regulated by linear precedence statements (see
\crossrefchapterw[Section~\ref{sec-id-lp}]{order} on linear precedence
constraints).

\citet{Keller:95} discusses how salient differences between extraction
and extraposition can be captured quite straightforwardly: to account,
e.g., for the clause-boundedness, it will be sufficient to restrict
the \textsc{extra} set of clausal signs to be the empty
set. Similarly, since extraposition (\textsc{extra}) and extraction
(\textsc{slash}) are implemented by different features, locality
constraints imposed on \textsc{slash} will not hold for extraposition.


\subsection{Extraposition as word order variation}

An entirely different approach to extraposition has emerged as part of
the HPSG work on linearisation using complex order domains. Following
\citet{Reape:94}, who suggested that linearisation in scrambling
languages such as \ili{German} should operate on larger domains than local
trees of depth one, \citet{Kathol:95b,kathol_a00} and \citet{KP95a}
have explored its suitability as a model for extraposition in \ili{German}.

The connection between scrambling and extraposition does have some
initial plausibility for freer word order languages such as \ili{German},
since the maximal domain of extraposition, i.e.\ the clause, coincides
with that of scrambling. However, even for \ili{German}, extraposition from
NPs already necessitates special mechanisms, such as partial
compaction, that are specific to extraposition and have no analogous
motivation for scrambling, where only union and total compaction are
used.\footnote{In linearisation-based HPSG, domain union creates an
  extended order domain, whereas compaction closes the domain by
  collapsing the list of domain objects into a single one. See \crossrefchapterw[Section~\ref{sec-domains}]{order} for
  explanation of linearization-based HPSG in general and
  \crossrefchapterw[Section~\ref{sec-partial-compaction}]{order} for a detailed discussion of the specific linearization-based
  approach to extraposition mentioned above.
} Once we
approach languages such as \ili{English} that display a much stricter order,
yet still allow extraposition, a scrambling approach to extraposition
becomes highly questionable.


\subsection{Generalised modification}
\label{udc:sec-generalised-modification}

Another line of proposals capitalises on the differences between
complement and adjunct extraposition: as argued by
\citet[284]{kiss_t02nllt}, the non-locality observed with relative clause
extraposition in \ili{German}, as in (\ref{ex:ExtraNP}a)  does not translate to complement extraposition
in equal measure, cf.~(\ref{ex:ExtraNP}b).
\eal
\label{ex:ExtraNP}

\ex[]{
\longexampleandlanguage{
\gll Man hat [die Frau [des Boten \_$_i$]] beschimpft, [der den Befehl überbrachte]$_i$.\footnotemark\\
     one has \spacebr{}the wife \spacebr{}of.the messenger {}  scolded \spacebr{}who the order delivered\\}{German}
\footnotetext{\citet[259]{haider96:_downr_down_to_right}}
\glt `People have scolded the wife of the messenger who delivered the order.'}
\ex[*] {
\gll Man hat [den Überbringer [der Mitteilung \_$_i$ ]] beschimpft, [daß die Erde rund ist]$_i$.\footnotemark\\
      one has \spacebr{}the messenger \spacebr{}of.the message {} {} insulted \spacebr{}that the earth round is\\
\footnotetext{\citet[282]{kiss_t02nllt}}
\glt `The messenger was insulted who delivered the message that the world is a sphere.'}
\zl

While acceptable examples of complement extraposition
from complex NPs can be found (see example
(\ref{ex-versuch-eines-beweises-der-vermutung}) below, extraposition
from adjuncts yields much sharper contrasts, which have not yet been
contested:

\eal
\label{ex:ExtraAdj}
\ex[*]{\label{ex:Beobachtungen:dass}
\gll Hier habe ich [bei             [den Beobachtungen \_$_i$ ]] faul  auf der Wiese gelegen, [daß die Erde rund ist]$_i$.\footnotemark\\
     here have I   \spacebr{}during \spacebr{}the observations {} {} lazily on the lawn laid \spacebr{}that the  earth round is\\
\footnotetext{\citet[283]{kiss_t02nllt}}
\glt `I was lying here lazily on the lawn during the observations that the world is a sphere.'
}
\ex[]{
\gll Hier habe ich [bei [vielen Versuchen$_i$ ]] faul auf der  Wiese gelegen, bei denen$_i$ die Schwerkraft überwunden wurde.\footnotemark\\
     here have I \spacebr{}during \spacebr{}many attempts {} lazily on the lawn laid during which  the gravity overcome was\\
\footnotetext{\citet[285]{kiss_t02nllt}}
\glt `I was lying here lazily on the lawn, during many attempts at which gravity was overcome.'}
\zl


Interestingly enough, complement extraposition (\ref{ex:Beobachtungen:dass}) appears to pattern with
leftward extraction (\ref{ex:ExtractAdj}) in this respect, which underlines the
extraction-like property of complement extraposition:

\ea[*]{
\label{ex:ExtractAdj}
\gll Das Verlies  hat er, [als er \_$_i$ verließ], gelacht.\footnotemark\\
     the dungeons has he  \spacebr{}when he {}  left laughed\\
\footnotetext{\citet[261]{haider96:_downr_down_to_right}}
\glt {Intended:} `He laughed when he left the dungeons.'
}
\z

Furthermore, Kiss observes that relative clause extraposition may give
rise to split antecedents, and therefore concludes that this process
should be better understood as an anaphoric one, rather than as
extraction to the right. 

Similar in spirit to \citet{culicover90:_extrap_and_compl_princ},
\citet{kiss_t02nllt} suggests that relative clause extraposition can
target any referential index introduced within the clause the relative {clause}
attaches to.  To that end, he proposes a set valued \textsc{anchor}
feature that indiscriminately percolates up the tree the index (and
handle) of any nominal expression. In situ and extraposed relative
clauses then semantically bind one of the \textsc{index/handle} pairs
contained in the \textsc{anchor} set of the head they syntactically
adjoin to.\footnote{See
  \crossrefchaptert[Section~\ref{sec-minimal-recursion-semantics}]{semantics}
  for an overview of Minimal Recursion Semantics, the meaning
  description language assumed in Kiss'
  approach.}$^,$\footnote{\citet{crysmann_b04rlc} proposes to
  synthesise the approach by \citet{kiss_t02nllt} with that of
  \citet{Keller:95}, using a two-step percolation mechanism that
  effectively controls for spurious ambiguity.}

\begin{figure}
  
  { \newbox\onebox \newbox\twobox \newbox\onetwobox \newbox\emptybox

    \setbox\onebox=\hbox{[\textsc{ancs} \{\fbox{i}\}]}
    \setbox\twobox=\hbox{[\textsc{ancs} \{\fbox{j}\}]}
    \setbox\onetwobox=\hbox{[\textsc{ancs} \{\fbox{i},\fbox{j}\}]}
    \setbox\emptybox=\hbox{[\textsc{ancs} \{ \}]}
%\centering
    \oneline{
\begin{forest}
sm edges
[{V\usebox{\onetwobox}}
   [{V\usebox{\onetwobox}} 
    [{NP\usebox{\onetwobox}} 
       [{D\usebox{\emptybox}} [den;the] ] 
       [{N\usebox{\onetwobox}} 
         [{N\usebox{\onebox}} [Beweis$_i$;proof] ] 
         [{NP\usebox{\twobox}} 
           [{D\usebox{\emptybox}} [der;of.the] ]
           [{N\usebox{\twobox}} [Theorie$_j$;theory] ] ] ] ]
     [{V\usebox{\emptybox}} [erbracht;produced] ] ] 
  [S [an die$_j$ niemand glaubt;in which nobody believes,roof] ] ]
\end{forest}
}
}
\caption{Anchor percolation in relative clause extraposition \citep{Kiss2005a}}
\end{figure}

The claim about the locality of complement extraposition has not been
left unchallenged: Müller (\citeyear[\page 206]{Mueller99a};
\citeyear[\page 10]{Mueller2004d}) presents examples of complement
clause extraposition that equally defy the \isi{Complex NP Constraint}.

\ea
\label{ex-versuch-eines-beweises-der-vermutung} 
{\gll Ich habe [von [dem Versuch [eines Beweises [der Vermutung \_$_i$ ]]]] gehört, [daß es Zahlen gibt, die die folgenden Bedingungen erfüllen]$_i$.\footnotemark\\
      I have \spacebr{}of \spacebr{}the attempt \spacebr{}of.a proof \spacebr{}of.the hypothesis {} {}
      heard \spacebr{}that there numbers exist which the following conditions
      fulfil\\
\footnotetext{St. \citet[223]{mueller_s02lc}}
      \glt `I have heard of the attempt at a proof of the hypothesis
      that there are numbers which fulfil the following conditions.'
    }  
    
    % \ex[]{\gll Für das Volk der Deutschen Demokratischen Republik ist
    %   dabei [die einmütige Bekräftigung [der Auffassung \_$_i$ ]]
    %   wichtig, [daß es die Interessen des Friedens und der
    %   Sicherheit erfordern, ... ]$_i$ \\
    % For the people of.the \ili{German} Democratic Republic is there the
    % unanimous emphasis of.the position {} {} important that it the
    % interest of.the peace and of.the security necessitate, ...\\
    % `It is important for the people of the GDR to emphasise,
    % unanimously, the position that it is necessary, in the interests
    % of peace and security ...' \hfill
    %   \citep[St.][223]{mueller_s02lc}}
\z

Consequently, he suggests that complement extraposition and adjunct
extraposition should both be handled by the same mechanism, i.e.\ a
non-local \textsc{extra} feature \parencites{Keller:95}[Section~13.2]{Mueller99a}.

\citet{crysmann_b09xtra} challenges Müller's unified analysis on the
grounds that it severely overgenerates.  While he concedes that
non-local complement extraposition is indeed possible, he argues that
the two processes still need to be distinguished, because (i) only
adjunct extraposition may target split antecedents and (ii)
complements cannot extrapose out of adjuncts, whereas adjunct
extraposition observes no such constraint.  He further notes that
non-local complement extraposition is subject to stronger bridging
requirements than adjunct extraposition, both semantic and prosodic:
as illustrated in (\ref{ex:ExtraAff}), acceptability greatly improves
with the semantic affinity between the complex NP from which
extraposition proceeds and the verb that governs it.

\eal
\label{ex:ExtraAff}
\ex[]{
\gll Er hat [ein Buch [über die Theorie \_$_i$ ]] gelesen, [daß Licht Teilchennatur hat]$_i$.\footnotemark\\
     he has \spacebr{}a book \spacebr{}about the theory {} {} read \spacebr{}that light {particle nature} has\\
\footnotetext{\citet[381]{Crysmann2013a}}
\glt `He has read a book about the theory that light has particle properties.'}
\ex[*]{
\gll Er hat [ein Buch [über die Theorie \_$_i$ ]] geklaut, [daß Licht Teilchennatur hat]$_i$.\footnotemark\\
     he has \spacebr{}a book \spacebr{}about the theory {} {} stolen \spacebr{}that light {particle nature} has\\
\footnotetext{\citet[381]{Crysmann2013a}}
\glt `He has stolen a book about the theory that light has particle properties.'}
\zl

\eal
\label{ex:RemnantSemantic}
\ex[]{
\gll [Über Syntax]$_i$  hat Max sich [ein Buch  \_$_i$ ] ausgeliehen.\footnotemark\\
     \spacebr{}about syntax has Max \textsc{self} \spacebr{}a book {} {} borrowed\\
\footnotetext{\citet[148]{kuthy01}}
\glt `It's about syntax that Max has borrowed a book.'}
\ex[*]{
\gll [Über  Syntax]$_i$  hat Max [ein Buch  \_$_i$ ] geklaut.\footnotemark\\
     \spacebr{}about syntax has Max \spacebr{}a book {} {} stolen\\
\footnotetext{\citet[148]{kuthy01}}
\glt `It's about syntax that Max has stolen a book.'}
\zl

While this effect for complement extraposition is similar to what has
been observed for PP extraction out of NPs \citep{deKuthy2002a}, cf.\ the
examples in (\ref{ex:RemnantSemantic}), it is of
note that no such contrasts can be found for adjunct extraposition:

\eal
\ex{
\gll Er hat [ein Buch [über die Theorie \_$_i$ ]]    gelesen, [die derzeit kontrovers diskutiert wird]$_i$.\footnotemark\\
    he has \spacebr{}a book(\textsc{n}) \spacebr{}about the theory(\textsc{f)} {} {} read \spacebr{}{which.\textsc{f}} currently controversially discussed is\\
\footnotetext{\citet[381]{Crysmann2013a}}
\glt `He has read a book about the theory which is under considerable debate at present.'}  
\ex{
\gll Er hat [ein Buch [über die Theorie \_$_i$ ]] geklaut, [die derzeit kontrovers diskutiert wird]$_i$.\footnotemark\\
    he has \spacebr{}a book(\textsc{n}) \spacebr{}about the theory(\textsc{f}) {} {} stolen \spacebr{}{which.\textsc{f}} currently controversially discussed is\\
\footnotetext{\citet[381]{Crysmann2013a}}
\glt `He has stolen a book about the theory which is under considerable debate at present.'}


\end{xlist}
\end{exe}


\citet{crysmann_b09xtra} unifies the anaphoric approach of
\citet{kiss_t02nllt} for adjunct extraposition with the
rightward-extraction approach of \citet{Keller:95} and
\citet{Mueller99a}, and suggests that both processes should be modelled
by the same set-valued non-local feature (\textsc{extra}), but that
elements on that set should be distinguished as to whether they are
mainly anaphoric elements (\textit{weak-local}), or full-fledged
\textit{local} values (\textit{full-local}), cf.\
Section~\ref{sec:UDC:ResumptivePronouns}.  Under this perspective,
extraposed adjuncts are expected to escape extraction islands\is{island} (such as
adjunct islands), as well as to modify split antecedents, simply
because they involve a grammaticalised anaphoric process, not
extraction. Conversely, complement extraposition involves an
extraction-like dependency, making it more prone to island
constraints, which may be bridged (complex NPs) or not (adjunct
islands).%
\is{extraposition|)}

% Walker:  


\section{Filler-gap mismatches}
\label{sec:UDC:FillerGapMismatches}

As noted in the introduction, there are unbounded dependency
constructions in which a filler apparently does not match the
associated gap. In this section we will look briefly at two examples
of such mismatches.

An interesting type of example is what \citet{Arnold:Borsley:10} call
auxiliary-stranding relative clauses (ASRCs). The following
illustrate:

\begin{exe}
  \ex \label{ex:UDC:ASRC}
  \begin{xlist}
    \ex Kim will sing, which Lee won't \trace{}.
    
    \ex Kim has sung, which Lee hasn't \trace{}.
    
    \ex Kim is singing, which Lee isn't \trace{}.
    
    \ex Kim is clever, which Lee isn't \trace{}.
    
    \ex Kim is in Spain, which Lee isn't \trace{}.
    
    \ex Kim wants to go home, which Lee doesn't want to \trace{}.
  \end{xlist}
\end{exe}

\noindent
\emph{Which} in these examples appears to be the ordinary nominal
\emph{which}, but the gap is a VP in (\ref{ex:UDC:ASRC}a), (\ref{ex:UDC:ASRC}b), (\ref{ex:UDC:ASRC}c) and (\ref{ex:UDC:ASRC}f), an AP in
(\ref{ex:UDC:ASRC}d), and a PP in (\ref{ex:UDC:ASRC}e). One response to these data might be to propose
that \emph{which} in such examples is not the normal nominal
\emph{which}, but a pronominal counterpart of the categories which appear
as complements of an auxiliary, mainly various kinds of VP. It is clear,
however, that ordinary VP complements of an auxiliary cannot appear as
fillers in a relative clause, as shown by the (b) examples in the
following:

\begin{exe} \ex \begin{xlist} 
\ex[]{This is the book, which Kim will read \trace{}.}

\ex[*]{This is the book, [read which] Kim will \trace{}.}
\end{xlist}
\end{exe}

\begin{exe} \ex \begin{xlist} 
\ex[]{ This is the book, which Kim has read \trace{}.}

\ex[*]{This is the book, [read which] Kim has \trace{}.}
\end{xlist}
\end{exe}
\noindent
\begin{exe} \ex \begin{xlist} 
\ex[]{This is the book, which Kim is reading \trace{}.}

\ex[*]{This is the book, [reading which] Kim is \trace{}.}
\end{xlist}
\end{exe}

\noindent
Thus, this does not seem a viable approach.

\citet{Arnold:Borsley:10} propose that these examples involve a special
kind of gap. As noted above, in a normal gap, the \textsc{local} value and the
\textsc{slash} value match. However, as \citet{webelhuth08} noted, there is no
reason why we should not under some circumstances have what he calls a
``dishonest gap''\is{gap!dishonest}, one whose \textsc{local} value
does not match its  \textsc{slash} element.
Developing this approach, \citet{Arnold:Borsley:10} propose that when an
auxiliary has an unrealised complement, the complement optionally has a
certain kind of nominal in \textsc{slash}, which is realised as
relative \emph{which}. When \textsc{slash} has the empty set as its value, the
result is an auxiliary complement ellipsis sentence. When \textsc{slash} contains a nominal element, we have a dishonest gap, because the value of \textsc{local} is
whatever the auxiliary requires, normally a VP of some kind, and the
result is an auxiliary-stranding relative clause.

A rather different type of example, discussed, among others, by
\citet[Chapter~2]{Bresnan01}, \citet[25--26]{Bouma:Malouf:Sag:01}, and \citet{Webelhuth:12},
is the following:

\begin{exe}
\ex  \label{ex:UDC:ThatHeMightBeWrong}  That he might be wrong, he didn't think of \trace{}.
\end{exe}

\noindent
Here, the apparent filler is a clause, but as the following shows, only
an overt NP and not an overt clause is possible in the position of the
gap.

\begin{exe}
\ex
\begin{xlist}\ex []{ He didn't think of the matter.}
\ex[*]{ He didn't think of that he might be wrong.}
\end{xlist}
\end{exe}

\noindent
The most detailed HPSG discussion of such examples is \citet{Webelhuth:12}.
Webelhuth argues on the basis of examples like the following that initial
clauses cannot be associated with a clausal gap:

\begin{exe} \ex \begin{xlist} 
\ex[]{ He was unhappy [that Sue was late again].}
\ex[*]{[That Sue was late again] he was unhappy.}
\end{xlist}
\end{exe}

\begin{exe} \ex \begin{xlist} 
\ex[]{Mary informed Bill [that Sue was late again].}
\ex[*]{[That Sue was late again] Mary informed Bill.}
\end{xlist}
\end{exe}

\begin{exe} \ex \begin{xlist} 
\ex[]{ It seems [that John is guilty].}
\ex[*]{[That John is guilty] it seems.}
\end{xlist}
\end{exe}

\noindent
Thus, initial clauses can only be associated with a nominal
gap. \citet[25--26]{Bouma:Malouf:Sag:01} propose an analysis in which
an NP gap has an S in its \textsc{slash} value. In other words, they propose a
dishonest gap.  \citet{Webelhuth:12} argues against this approach and
proposes an analysis in which an S[\textsc{slash} \{NP\}] in which the NP
has a clausal interpretation can combine with a finite clause. Thus,
Figure~\ref{fig:UDC:Tree:ThatHeMightBeWrong} gives the schematic
structure for (\ref{ex:UDC:ThatHeMightBeWrong}).

\begin{figure}
	\centering
\begin{forest}
sm edges without translation
	[S
		[S [That he might be wrong, roof]]
		[%
		\avm{S [slash \{NP\}]}
			[he didn't think of, roof]]]
\end{forest}    
	 \caption{\label{fig:UDC:Tree:ThatHeMightBeWrong}``Dishonest'' gap }
\end{figure}
 

On this analysis, the initial clause is not a filler, and the
construction is not a head-filler phrase. However, the analysis involves
a normal unbounded dependency except at the top. In contrast, the Arnold
and Borsley analysis of ASRCs outlined earlier involves a normal
unbounded dependency except at the bottom.


\section{Concluding remarks}
\label{sec:UDC:ConcludingRemarks}

The preceding pages have, among other things, highlighted the fact
that there are some unresolved issues in the HPSG approach to
unbounded dependencies. In particular, there is disagreement about
whether or not gaps are empty categories and about whether or not the
middle of a dependency is head-driven. It is important, therefore, to
emphasise that a number of matters seem reasonably clear. In
particular, it is generally accepted that unbounded dependencies
involve a set- or list-valued feature called \textsc{slash} or, in
some recent work, \textsc{gap}. It is also generally accepted that
this is true of all types of unbounded dependencies, including those
with a filler and those without, those with a gap and those with a
resumptive pronoun, as well as dependencies with or without some kind
of mismatch between filler and gap. Finally, it is generally accepted
that the hierarchies of phrase types that are a central feature of
HPSG provide an appropriate way to capture both the similarities among
the many unbounded dependency constructions and the variety of ways in
which they differ. The general approach seems to compare quite
favourably with the approaches that have been developed within other
frameworks.

\section*{Appendix: Unbounded dependencies in Sign-Based Construction Grammar}
\label{udc:sec-SBCG}

\is{Construction Grammar (CxG)!Sign-Based|(}
This chapter has concentrated on the approach to unbounded
dependencies that has been developed with Constructional HPSG. As
has been discussed in a number of chapters,\footnote{
{See \citew[Section~\ref{prop:sec-sbcg}]{chapters/properties} and \citew[Section~\ref{cxg:sec-sbcg}]{chapters/cxg} for a general comparison of
  Constructional HPSG and SBCG. \citet{chapters/evolution} discuss the
  evolution of HPSG and the pages \pageref{page-sbcg-start}--\pageref{page-sbcg-end} deal with the
  HPSG variant SBCG.}
} a version of HPSG called
Sign-Based Construction Grammar (SBCG) was developed in the 2000s,
which differs from Constructional HPSG in a number of ways
\citep{Sag:12}. Among other things, it has a somewhat different
treatment of unbounded dependencies. In this appendix, we outline the
main ways in which SBCG is different in this area.

\begin{sloppypar}
Unlike Constructional HPSG, SBCG makes a fundamental distinction
between signs and constructions. Constructions are objects which
associate a mother sign (\textsc{mtr}) with a list of daughter signs
(\textsc{dtrs}), one of which may be a head daughter
(\textsc{hd-dtr}). Headed constructions thus take the following form:
\end{sloppypar}

\begin{exe}
  \ex \label{ex:UDC:SBCG:cx}
\avm{
	[\type*{cx}
      mtr & sign \\
      dtrs & \listOf{sign} \\
    hd-dtr & sign ]
}

\end{exe}
\noindent
Constructions are utilised by the Sign Principle\is{principle!Sign}, which can be
formulated as follows:

\begin{exe}
  \ex \label{ex:UDC:SBCG:SignPrinciple} Signs are well formed if either

  \begin{xlist}
    \ex they match some lexical entry, or \ex they match the mother of
    some construction.
  \end{xlist}
\end{exe}

\noindent
Constructions and the Sign Principle are features of SBCG which
are lacking in Constructional HPSG. Hence, they are
complications. But they allow simplifications. In particular, they
allow a simpler notion of sign without the features \textsc{dtrs} and
\textsc{hd-dtr}. This in turn allows the framework to dispense with
\textit{synsem} and \textit{local} objects. The \textsc{arg-st}
feature and the \textsc{valence} feature, which replaces \textsc{subj}
and \textsc{comps}, take lists of signs and not \type{synsem} objects as
their value. More importantly in the present context, the \textsc{gap}
feature, which replaces \textsc{slash}, takes as its value a list of
signs and not \type{local} objects.

One might suppose that this view of \textsc{gap} would entail that a
filler and the associated gap have all the same syntactic and semantic
properties, unlike within Constructional HPSG, where they only
share the syntactic and semantic properties that are part of a
\textit{local} object and hence not the \textsc{wh} feature in
\emph{wh}"=interrogatives. However, the framework allows constraints
to stipulate that certain objects are the same except for some
specified features. The constraint of the filler-head construction,
which corresponds to HPSG’s head-filler phrase, stipulates that the
sign that is the filler is identical to the sign in the \textsc{gap}
list of its sister, except for the value of the \textsc{wh} feature
and the \textsc{rel} feature used in relative clauses \citep[\page 166]{Sag:12}. Thus, filler
and gap differ in the same way in SBCG and Constructional HPSG,
but for different reasons.

At the bottom of the dependency, things are rather different. The SBCG
analysis allows a member of the \textsc{arg-st} list of a lexical head
to appear not as a member of the word’s \textsc{valence} list, but as
a member of its \textsc{gap} list. We can illustrate with
\textit{read} in the following examples:

\begin{exe}
  \ex \label{ex:UDC:SBCG:ex}
  \begin{xlist}
    \ex I will read the book.
    
    \ex Which book will you read?
  \end{xlist}

\end{exe}

\noindent
In (\ref{ex:UDC:SBCG:ex}a), \textit{read} has the values in
(\ref{ex:UDC:SBCG:nogap}) for the three features:

% \inlinetodoobl{Stefan: This needs fixing by Felix. The space between \ibox{1} and NP is larger than
%   the space between bracket and \ibox{1}.}
% avm todo
\begin{exe}
  \ex \label{ex:UDC:SBCG:nogap}
\avm{
	[arg-st & < \1 NP, \2 NP > \\
	valence & < \1, \2 > \\
	gap & < > ]
}

\end{exe}

\noindent
Here, \textsc{arg-st} and \textsc{valence} have the same value, and
the value of \textsc{gap} is the empty list. In
(\ref{ex:UDC:SBCG:ex}b), the three features have the following values:

% \inlinetodoopt{Stefan: In \citet[\page 163]{Sag:12} the NPs in \argst have an empty GAP list.}
\begin{exe}
  \ex \label{ex:UDC:SBCG:gap}
\avm{
  [arg-st & < \1 NP, \2 NP > \\
  valence & < \1 > \\
  gap & < \2 > ]
}

\end{exe}
	
\noindent
The second member of the \textsc{arg-st} list appears not in the
\textsc{valence}, but in the \textsc{gap} list. This is rather
different from HPSG. As discussed in
Section~\ref{sec:UDC:BasicApproach}, HPSG gaps have a non-empty
\textsc{slash} value. Here, gaps are just ordinary signs which appear
in a \textsc{gap} list and not in a \textsc{valence} list.

This is an interesting alternative to the approach outlined in the
main body of this chapter. However, it would need to be extended to
account for some of the phenomena considered here.
\is{Construction Grammar (CxG)!Sign-Based|)}



\section*{Abbreviations}

\begin{tabularx}{.99\textwidth}{@{}lX}
\textsc{prt} & particle\\
\textsc{resump} & resumptive element \\
\end{tabularx}



\section*{\acknowledgmentsEN}

We would like to thank a reviewer and the editors of the handbook, in particular
Jean-Pierre Koenig and Stefan Müller for their comments and
corrections on various versions of this chapter. 





 

{\sloppy
\printbibliography[heading=subbibliography,notkeyword=this]
}



\end{document}

%      <!-- Local IspellDict: en_GB-ise-w_accents -->


%%% Local Variables:
%%% mode: latex
%%% TeX-master: "../main-udc"
%%% End:
