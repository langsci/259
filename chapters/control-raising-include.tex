%% -*- coding:utf-8 -*-
\abstract{The distinction between raising and control predicates has been a hallmark of syntactic
  theory since \citew{Rosenbaum67a-u} and \citew{Postal1974}. Unlike transformational analyses, HPSG
  treats the difference as mainly a semantic one: raising verbs (\word{seem}, \word{begin},
  \word{expect}) do not semantically select their subject (or object) nor assign them a semantic
  role, while control verbs (\word{want}, \word{promise}, \word{persuade}) semantically select all
  their syntactic arguments. On the syntactic side, raising verbs share their subject (or object)
  with the unexpressed subject of their non"=finite complement while control verbs only coindex them
  \citep{PollardandSag1994}. We will provide creole data (from Mauritian) which support a phrasal
  analysis of their complement, and argue against a clausal (or small clause) analysis
  \citep{HenriandLaurens2011}. The distinction is also relevant for non"=verbal predicates such as
  adjectives (\word{likely} vs.\ \word{eager}). The raising analysis naturally extends to copular
  constructions (\word{become}, \word{consider}) and most auxiliary verbs
  \citep{PollardandSag1994,Sagetal2020}.} 

\begin{document}
\maketitle
\label{chap-control-raising}


%\section{Introduction}
%\label{control-sec-intro}

\section{The distinction between raising and control predicates}

\subsection{The main distinction between raising and control verbs}

In a broad sense, \emph{control} refers to a relation of referential dependence between an unexpressed
subject (the controlled element) and an expressed or unexpressed constituent (the controller); the
referential properties of the controlled element, including possibly the property of having no
reference at all,\itd{expletives?} are determined by those of the controller \citep[372]{Bresnan1982}. Verbs taking
non"=finite complements usually determine the interpretation of the unexpressed subject of the
non"=finite verb. With \word{want}, the subject is understood as the subject of the infinitive,
while with \word{persuade} it is the object, as shown by the reflexives in (\ref{ex-equi}). They are
called \emph{control verbs}, and \emph{John} is called the \emph{controller} in (\ref{equi1}) while
it is \emph{Mary} in (\ref{equi2}).  

\eal
\label{ex-equi}
\ex John wants to buy himself a coat. \label{equi1}
\ex John persuaded Mary to buy herself / * himself a coat.\label{equi2}
\zl

Another type of verb also takes a non-finite complement and identifies its subject (or its object)
with the unexpressed subject of the non-finite verb. Since \citet{Postal1974}, they are called
``raising verbs''. What semantic role the missing subject has if any is determined by the lower
verb, or if that is a raising verb, an even lower verb. In (\ref{seem}) the subject of the
infinitive (\emph{like}) is understood to the be the subject of \word{seem}, while in (\ref{exp})
the subject of the non-finite verb (\emph{buy}) is understood to be the object of
\word{expect}. Verbs like \word{seem} are called ``subject-to-subject-raising verbs'' (or
``subject-raising verbs''), while verbs like \word{expect} are called ``subject-to-object-raising
verbs'' (or ``object-raising verbs''). 

\eal
\ex John seemed to like himself.\label{seem}
\ex John expected Mary to buy herself / * himself a coat. \label{exp}
\zl
 
Raising and control constructions differ from other constructions in which the missing subject
remains vague (\ref{arbitrary}) and which are a case of ``arbitrary'' or ``anaphoric'' control
\parencites[\page 75--76]{Chomsky1981}[\page 379]{Bresnan1982}\footnote{\citet{Bresnan1982} proposes
  a non-transformational analysis and renames ``raising'' ``functional control'' and ``control''
  (obligatory) ``anaphoric control''.%
}. 
 
\ea
Buying a coat can be expensive.\label{arbitrary}
\z
  
A number of syntactic and semantic properties show how control verbs like \emph{want}, \emph{hope},
\emph{force}, \emph{persuade}, \emph{promise}, differ from raising verbs like \emph{see},
\emph{seem}, \emph{start}, \emph{believe}, \emph{expect} \citep{Rosenbaum67a-u,Postal1974,Bresnan1982}.\footnote{%
  The same distinction is available for verbs taking a gerund-participle complement: 
  \emph{Kim remembered seeing Lee.} (control) vs.\  \emph{Kim started singing.} (raising).%
}
The key point is that there is a semantic role associated with the subject position with verbs like
\emph{want} but not with verbs like \emph{seem} and with the post-verbal position with verbs like
\emph{persuade} but not with verbs like \emph{expect}.  The consequence is that more or less any NP
is possible as subject of \emph{seem} and as the post-verbal NP after \emph{expect}. This includes
expletive \emph{it} and \emph{there} and idiom-chunks\itd{Stefan: Isn't the idiom-chunk the whole
  idiom? Then I would write here ``parts of idiom chunks''}.

Let us first consider non"=referential subjects: meteorological \word{it} is selected by predicates
such as \word{rain}. It can be the subject of \word{start}, \word{seem}, but not of \word{hope},
\word{want}. It can be the object of \word{expect}, \word{believe} but not of \word{force},
\word{persuade}:
	
\eal
\ex[]{
It rained this morning.
}
\ex[]{
It seems/started to rain this morning.
} \label{rain1}
\ex[]{
We expect it to rain tomorrow. 
} \label{rain2}
\zl
\eal
\ex[\#]{
It wants/hopes to rain tomorrow.
} \label{rain3}
\ex[\#]{
The sorcerer forced it to rain.
} \label{rain4}
\zl
 	
The same contrast holds with an idiomatic subject such as \word{the cat} in the expression \word{the
cat is out of the bag} `the secret is out'. It can be the subject of \word{seem} or the object of
\word{expect}, with its idiomatic meaning. If it is the subject of \word{want} or the object of
\word{persuade}, the idiomatic meaning is lost and only the literal meaning remains. 
 
\eal
\judgewidth{\#}
\ex[]{
The cat is out of the bag.
}
\ex[]{
The cat seems to be out of the bag. 
} \label{cat1}
\ex[]{
We expected the cat to be out of the bag. 
} \label{cat2}
\ex[\#]{
The cat wants to be out of the bag.\hfill(non"=idiomatic)
} \label{cat3}
\ex[\#]{
We persuaded the cat to be out of the bag.\hfill(non"=idiomatic)
} \label{cat4}
\zl

Let us now look at non-nominal subjects: \emph{be obvious} allows for a sentential subject and
\emph{be a good place to hide} allows for a prepositional subject. They are possible with raising
verbs, as in the following: 
 
\eal 
\ex[]{
[That Kim is a spy] seemed to be obvious.
}
\ex[]{
[Under the bed] is a good place to hide.}
\ex[]{
Kim expects [under the bed] to be a good place to hide.
} \label{under}
\zl

\noindent
But they would not be possible with control verbs:
\eal
\ex[\#]{
[That Kim is a spy] wanted to be obvious.
}
\ex[\#]{
Kim persuaded [under the bed] to be a good place to hide.
}
\zl

In languages such as \ili{German} which allow them, subjectless constructions can be embedded under
raising verbs but not under control verbs \citep[\page 48]{Mueller2002b}; subjectless passive
\emph{gearbeitet} `worked' can thus appear under \emph{scheinen} `seem' but not under
\emph{versuchen} `try': 

\eal
\label{german1}
\ex[]{ 
\gll weil gearbeitet wurde\\
     because worked was\\\hfill(\ili{German})
\glt 'because work was being done'
}
\ex[]{ 
\gll Dort schien noch gearbeitet zu werden.\\
     there seemed yet worked to be\\
\glt `Work seemed to still be being done there.’
}
\ex[*]{
\gll Der Student versucht, gearbeitet zu werden\\
     the student tries worked to be\\
\glt Intended: `The student tries to get the work done.'
}
\zl
 
\noindent
All this shows that the subject (or the object) of a raising verb is only selected by the
non"=finite verb.

A related difference is that when control and raising sentences have a corresponding sentence with a finite clause complement, they have rather different related sentences.
%It seemed [that Kim impressed Sandy]
%Kim hoped [that he would go home]
%Kim expected [that Sandy would go home]
%Kim persuaded Sandy [that he/she should home]
With control verbs, the non"=finite complement may often be replaced by a sentential complement (with is own subject), while it is not possible with raising verbs:

\eal
\ex[]{
Kim hoped [to impress Sandy] / [that he impressed Sandy].\label{hope1}
}
\ex[]{
Kim seemed [to impress Sandy] / *[that he impressed Sandy].
}
\zl

\eal
\ex[]{
Kim promised Sandy [to come] / [that he will come]. \label{promise1}
}
\ex[]{
Kim expected Sandy [to come] / *[that she will come].
}
\zl

With some raising verbs, on the other hand, a sentential complement is possible with an expletive
subject (\ref{seem4}), or with no postverbal object (\ref{expect3}). 

\eal
\ex[]{
It seemed [that Kim impressed Sandy]. \label{seem4}
}
\ex[]{
Kim expected [that Sandy will come].\label{expect3}
}
\zl

\noindent
This shows that the control verbs can have a subject (or an object) different from the subject of
the embedded verb, but not the raising verbs.\footnote{%
  Another contrast proposed by
  \citet{Jacobson1990}\iaddpages is that control verbs may allow for a null complement (\emph{She tried.}), or
  a non"=verbal complement (\emph{They wanted a raise.}), while raising verbs may not (\emph{*She
    seemed}). However, some raising verbs may have a null complement (\emph{She just started.}) as
  well as some auxiliaries (\emph{She doesn't.}) which can be analysed as raising verbs (see
  Section~\ref{sec-auxiliaries-as-raising-verbs} below).%
} 

\subsection{More on control verbs}

For control verbs, the choice of the controller is determined by the semantic class of the verb
(\citealt[Chapter~3]{PollardandSag1992} and also \citealt{JackendoffandCulicover2003}).  Verbs of
influence (\word{permit}, \word{forbid}) are object-control while verbs of commitment
(\word{promise}, \word{try}) as in (\ref{commit}) and orientation (\word{want}, \word{hate}) as in
(\ref{orient}) are subject-control, as shown by the reflexive in the following examples: 

\itd{Stefan: What about: ``John promised his son to go to the movies together.''}
\eal
\ex\label{ex-John-promised-Mary-to-buy}
John promised Mary to buy himself / * herself a coat. \label{commit}
\ex\label{ex-John-permitted-Mary-to-buy} 
John permitted Mary to buy herself / * himself a coat.\label{orient}
\zl
 
  The classification of control verbs is cross-linguistically widespread \citep{VanValinandLapolla1997}, but \ili{Romance} verbs of mental representation and speech report are an exception in being subject-control without having a commitment or an orientation component.


\begin{exe}
\ex \begin{xlist}
\ex 
\gll Marie dit {ne pas} \^etre convaincue.\\
     Marie says \ig{neg} be convinced \\\hfill(\ili{French})
\glt `Marie says she is not convinced.'	
\ex 
\gll Paul pensait  avoir compris. \\
     Paul thought have understood \\
\glt `Paul thought he understood.'
 \end{xlist}
\end{exe}

It is worth noting that for object-control verbs, the controller may also be the complement of a preposition \citep[\page 139]{PollardandSag1994}:

\begin{exe}
\ex Kim appealed [to Sandy] to cooperate. \label{to}
\end{exe}


%\eal
%\ex[]{
%Leslie wants this / a raise.
%}
%\ex[]{
%Leslie tried.
%}
%\ex[*]{
%Leslie seemed.
%}
%\ex[*]{
%Leslie seemed this.
%}
%\zl
 
\citet[\page 401]{Bresnan1982}, who attributes the generalization to Visser, also suggests that
object-control verbs may passivize (and become subject-control) while subject"=control verbs do not
(with a verbal complement). 

\itd{Stefan: Example (\mex{1}c) is a mis-attribution. The example is due to \citet{JB1976a-u} and this is stated in
  PS94. I removed the reference to \citep[\page 285]{PollardandSag1994} and added the original
  source to the text.}
\eal
\ex[]{
Mary was persuaded to leave (by John).
}\label{persuade-pass}
\ex[*]{
Mary was promised to leave (by John).
}\label{promise-pass}
\ex[]{Pat was promised to be allowed to leave. 
}\label{promise-pass2}
\zl
However, there are counterexamples like (\ref{promise-pass2}) from \citet{JB1976a-u}\addpages and the generalization does not seem to
hold crosslinguistically (see \citew[\page 129]{Mueller2002b} for counterexamples in \ili{German}). 
 
\subsection{More on raising verbs}
\label{sec-more-on-raising-verbs}

From a cross-linguistic point of view, raising verbs usually belong to other semantic classes than
control verbs. The distinction between subject-raising and object-raising also has some semantic
basis: verbs marking tense, aspect, modality (\word{start}, \word{cease}, \word{keep}) are
subject-raising, while causative and perception verbs (\word{let}, \word{see}) are usually
object-raising:

\eal
\ex John started to like himself.
\ex It started to rain.
\ex John let it appear that he was tired.
\ex John let Mary buy herself / * himself a coat.
\zl
	

Transformational analyses posit distinct syntactic structures for raising and control sentences:
subject-raising verbs select a sentential complement (and no subject), while subject-control verbs
select a subject and a sentential complement \citep{Postal1974, Chomsky81a}. With subject-raising
verbs, the embedded clause's subject is supposed to move to the position of matrix verb subject,
hence the  term ``raising''. They also posit two distinct structures for object-control and
object-raising verbs: while object-control verbs select two complements, object-raising verbs only
select a sentential complement and an exceptional case marking (ECM) rule assigns case to the
embedded clause's subject of \word{expect} verbs).
% a co-indexing rule between the empty subject (PRO) of the infinitive and the subject (\word{promise}) or the object (\word{persuade}) of the matrix verb for control verbs.
In this approach, both subject- and object-raising verbs have a sentential complement:
	
\itd{Stefan: Are there sources for these analyses?}
\eal
\ex subject-raising:\\
{}[\sub{NP} $e$ ] seems [\sub{S} John to leave ] 
$\leadsto$  
{}[\sub{NP} John$_{i}$ ] seems [\sub{S} $e_{i}$ to leave ]	
%\ex subject-control:  
%{}[\sub{NP} John$_{i}$ ] wants [\sub{S} PRO$_{i}$ to leave ]	
%\zl
%\eal
\ex object-raising (ECM):\\
We expected [\sub{S} John to leave ] 	
%\ex object-control: We persuaded  
%{}[\sub{NP} John$_{i}$ ]  [\sub{S} PRO$_{i}$ to leave ]	
\zl

However, the putative correspondence between source and target for raising structures is not
systematic: \word{seem} may take a sentential complement (with an expletive subject)as in
(\ref{seem4}) but the other subject-raising verbs (aspectual and modal verbs) do not.

\eal
\ex[]{
Paul started to understand.
}
\ex[*]{
It started [that Paul understands].
}
\zl
 
Similarly, while some object-raising verbs (\word{expect, see}) may take a sentential complement (\ref{expect3}), others do not (\word{let}, \word{make}, \word{prevent}).
 
\itd{Stefan: Why do you have (\mex{1}) in addition to (\mex{2}) and no such example in addition to (\mex{0})?}
\eal
\ex[]{
We expect Paul to understand.	\label{ex-we-expect-paul-to-unterstand}
}
\ex[]{
We expect [that Paul understands]. \label{ex-we-expect-that-paul-understands}
}
\zl
\eal
\ex[]{
We let Paul sleep.
}
\ex[*]{
We let [that Paul sleeps].
}
\zl

Furthermore, in transformational analyses, it is often assumed that the subject of the non"=finite verb  is supposed to raise to receive case from the matrix verb.
 But the subject of \word{seem} or \word{start} need not bear case  since it can be a non-nominal subject (\ref{under}).
%\eal
%\ex[]{[Drinking one liter of water each day] seems to benefit your health.}
%\zl
Data from languages with ``quirky'' case such as \ili{Icelandic}, also show that subjects of subject-raising verbs in fact keep the quirky case assigned by the embedded verb  \citep{Zaenenetal1985}, contrary to the subject of subject-control verbs which are assigned case by the matrix verb and are thus in the nominative. A verb like \word{need} takes an accusative subject, and a raising verb (\word{seem}) takes an accusative subject as well when combined with \word{need} (\ref{need}). With a control verb (\word{hope}), on the other hand, the subject must be nominative (\ref{hope-i}).
\inlinetodoobl{
Stefan: You attributed the examples in (\mex{1}) to \citew[\page 138--139]{PollardandSag1994} but
they are not from there. PS discuss other examples. Are they from Zaenen?%
}

\eal
\ex 
\gll Mig vantar peninga.\footnotemark\\
     I.\textsc{acc} need money.\textsc{acc} \\\hfill(\ili{Icelandic})
\ex 
\gll Mig virdast vanta peninga. \label{need} \\
     I.\textsc{acc} seem need money.\textsc{acc} \\
\ex 
\gll Eg vonast till ad vanta ekki peninga. \label{hope-i} \\
     I.\textsc{nom} hope for to need not money.\textsc{acc} \\
\glt `I hope I won't need money.'
\zl

Turning now to object-raising verbs, when a finite sentential complement is possible, the structure
is not the same as  with a non"=finite complement. Heavy NP shift is possible with a non"=finite
complement, and not with a sentential complement \parencites[\page 423]{Bresnan1982}[\page 113]{PollardandSag1994}: this shows that \word{expect} has two complements in (\ref{ex-we-expect-paul-to-unterstand}) and only one in (\ref{ex-we-expect-that-paul-understands}).

\eal
\ex[]{
We expected [all students] [to understand].
}
\ex[]{
We expected [to understand] [all those who attended the class]. \label{HNPS}
}
\ex[]{ 
We expected [that [all those who attended the class] understand].
}
\ex[*]{
We expected [that understand [all those who attended the class]].
}
\zl

Fronting also shows that the NP VP sequence does not behave as a single constituent, unlike the finite complement:

\eal
\ex[]{
That Paul understood, I did not expect.
}
\ex[*]{
Paul to understand, I did not expect.
}
\zl


This shows that object-raising verbs are better analysed as ditransitive verbs and that the subject of the non"=finite verb has all properties of an object of the matrix verb. It is an accusative in \ili{English} (\word{him}, \word{her}) (\ref{pro}) and it can passivize, like the object of an object-control verb (\ref{passive}).

\begin{exe}
\ex
\begin{xlist} \label{pro}
\ex We expect him to understand.
\ex  We persuaded him to work on this.
\end{xlist}
\ex \begin{xlist} \label{passive}
\ex  He was expected to understand.
\ex  He was persuaded to work on this.
\end{xlist}
	
\end{exe}


To conclude, the movement (raising) analysis of subject-raising verbs, as well as the ECM analysis of object-raising verbs are motivated by semantic considerations: an NP which receives a semantic role from a verb should be a syntactic argument of this verb. But they lead to syntactic structures which are not motivated (assuming a systematic availability of a sentential complementation) and/or make wrong empirical predictions (that the postverbal sequence of ECM verb behaves as one constituent instead of two).
 
\subsection{Raising and control non-verbal predicates}\label{nonverbal}

Non-verbal predicates taking a non"=finite complement may also fall under the raising/control distinction.  Adjectives such as \word{likely} have raising properties: they do not select the category of their subject, nor assign it a semantic role, contrary to adjectives like \word{eager}. Meteorological \word{it} is thus compatible with \word{likely}, but not with \word{eager}. In the following examples, the subject of the adjective is the same as the subject of the copula (see Section~\ref{sec-copular-constructions} below).

\eal
\ex[]{
It is likely to rain.
}
\ex[]{
John is likely / eager to work here.
}
\ex[*]{
It is eager to rain.
}
\zl

The same contrast may be found with  nouns taking a non"=finite complement. Nouns such as \word{tendency} have raising properties: they do not select the category of their subject, nor assign it a semantic role, contrary to nouns like \word{desire}. Meteorological \word{it} is thus compatible with the former, but not with the latter. In the following examples, the subject of the predicative noun is the same as the subject of the light verb \emph{have}.


\eal
\ex[]{
John has a tendency to lie.
}
\ex[]{
John has a desire to win.
}
\ex[]{
It has a tendency / * desire to rain at this time of year.
}
\zl

\section{An HPSG analysis}


In a nutshell, the HPSG analysis rests on a few leading ideas: non"=finite complements are
unsaturated VPs (a verb phrase with a non"=empty \subjl); a syntactic argument need not be assigned
a semantic role; control and raising verbs have the same syntactic arguments; raising verbs do not
assign a semantic role to the syntactic argument that functions as the subject of their non"=finite
complement. We continue to use the term \emph{raising}, but it is just a cover term, since no raising
is taking place in HPSG analyses. 

As a result\itd{Stefan: of what?},  raising means full identity of syntactic and semantic
information (\type{synsem}) \crossrefchapterp{properties}\itd{Stefan: What for are you citing this
  chapter? Give the reader a clue. And me for linking to the page.} with the unexpressed subject, while
control involves identity of semantic indices (discourse referents) between the controller and the
unexpressed subject. Co-indexing is compatible with the controller and the controlled subject not
bearing the same case (\ref{hope-i}) or having different parts of speech (\ref{to}), as it is the
case for pronouns and antecedents (see \crossrefchapteralp{binding}). This would not be possible
with raising verbs, where there is full sharing of syntactic and semantic features between the
subject (or the object) of the matrix verb and the (expected) subject of the non"=finite verb. In
\ili{German}, the nominal complement of a raising verb like \emph{sehen} `see' must agree in case
with the subject of the infinitive, as shown by the adverbial phrase \emph{einer nach dem anderen} `one after the other' which
agrees in case with the unexpressed subject of the infinitive, but it can have a different case with
a control verb like \emph{erlauben} `allow', as the following examples from \citew[\page
47--48]{Mueller2002b} show: 


\eal
\label{german2}
\ex 
\gll Der Wächter  sah den Einbrecher     und seinen Helfer            einen       / *  einer nach dem anderen weglaufen\\
     the watchman saw the burglar.\ACC{} and his    accomplice.\ACC{} one.\ACC{} {} {} one.\NOM{} after the other run.away\\\hfill(\ili{German})
\glt `The watchman saw the burglar and his accomplice run away, one after the other.'
\ex
\gll Der Wächter erlaubte den Einbrechern, einer nach dem anderen wegzulaufen.\\
     the watchman  allowed the burglars.\DAT{} one.\NOM{} after the other away.to.run\\
\glt `The watchman allowed the burglars to run away, one after the other.'
\zl

\itd{Stefan: I think it would be good to give the reader some guidance here. Say what subsections
  you have and why they are here. I was a bit surprised to see the section on Mauritian since it
  looks like a specific case study. But it has some overall relevance and this could be made clear here.}

\subsection{The HPSG analysis of ``raising'' verbs}
\label{control:sec-HPSG-anaylsis-of-raising}

Subject-raising-verbs (and object-raising verbs) can be defined as subtypes inheriting from
verb"=lexeme and subject-raising"=lexeme (or object raising"=lexeme)
types. Figure~\ref{raising:fig-verb-hier2} shows parts of a possible type hierarchy.


\begin{figure}
\begin{forest}
type hierarchy
[lexeme
  [part-of-speech,partition
     [verb-lx, name=A1 ] 
     [adj-lx]
     [noun-lx] 
     [\ldots]] 
  [arg-selection,partition 
     [intr-lx
      	[subj-rsg-lx
      	  [sr-v-lx, name=B1 ]]
        [\ldots] ]
     [tr-lx
       [obj-rsg-lx
         [or-v-lx, name=B2 ]]
       [\ldots]	]
     [\ldots]]]
\draw (A1.south)-- (B1.north);
\draw (B2) to [bend left= 6] (A1);
\end{forest}
\caption{\label{raising:fig-verb-hier2}A type hierarchy for subject- and object-raising verbs}
\end{figure}


As in \crossrefchapterw[Section~\ref{properties:lexemes-and-words}]{properties}, upper case letters
are used for the two dimensions of classification, and \type{verb-lx}, \type{intr-lx}, \type{tr-lx},
\type{subj-rsg-lx}, \type{obj-rsg-lx}, \type{or-v-lx} and \type{sr-v-lx} abbreviate
\type{verb"=lexeme}, \type{intransitive"=lexeme}, \type{transitive"=lexeme},
\type{subject-raising"=lexeme}, \type{object"=raising"=lexeme}, \type{object"=raising"=verb"=lexeme}
and \type{subject-raising"=verb"=lexeme}, respectively.  The constraints on the types \type{subj-rsg-lx} and
\type{obj-rsg-lx} are as follows:\footnote{%
\append  is used for list concatenation. The category of the complement is not specified as a VP
since it may be a V in some \ili{Romance} languages with a flat structure \citep{AG2003a-u} and in
some verb final langages where the matrix verb and the non"=finite verb form a verbal complex
(\ili{German}, \ili{Dutch}, \ili{Japanese}, \ili{Persian}, \ili{Korean}, see
\crossrefchapterw{order} on constituent order). Furthermore, the same lexical types\itd{Stefan: the
  ``same''? You define new types in (\ref{ex-subj-pred-v}). Maybe better write ``similar
  types''. But after all, the copula types should be subtypes. Here is an open issue.} will also be
used for copular verbs that take non"=verbal predicative complements, see
Section~\ref{sec-copular-constructions}.%
}

\itd{Stefan: What about ``Kim seems to Sandy to be smart.''? \type{subj-rsg-lx} whould allow for an
  optional object.}
\eal
\label{rsg}
\ex \type{subj-rsg-lx} \impl\\
\avm{ [ \argst  \1 \+ < [subj & \1 ] > ]} \label{rais-1}
\ex \type{obj-rsg-lx}  \impl\\
\avm{ [ \argst  < NP > \+ \1 \+ < [subj & \1 ] > ]} \label{rais2}
\zl


The subject of the non"=finite complement shares its \type{synsem} with the subject of
the subject-raising verb in (\ref{rais-1}), and with the object of object raising verb in
(\ref{rais2}). This means that they share their syntactic and semantic features: they have the same
semantic index (if any), but also the same part of speech, the same case etc. Thus a subject
appropriate for the non"=finite verb is appropriate as a subject (or an object) of the raising verb:
this allows for expletive (\ref{rain1}, \ref{rain2}) or idiomatic (\ref{cat1}, \ref{cat2}) subjects,
as well as non"=nominal subjects (\ref{under}). If the embedded verb is subjectless, as in
(\ref{german1}), this information is shared too (\ibox{1} can be the empty list). 

A subject-raising verb (\word{seem}) and an object"=raising verb (\word{expect}) inherit from
\type{subj-rsg-v} and \type{obj-rsg-v}\itd{Stefan: Schouldn't these be \type{subj-rsg-lx} and \type{obj-rsg-lx}?} respectively; their lexical descriptions look as follows,
assuming a MRS semantics (\citealp{CFPS2005a} and \crossrefchapteralp{semantics}):

\itd{Stefan: Why does the type \type{subj-rsg-v-word} exist? Usually one would assume the lexeme
  types and then there are inflectional lexical rules. The output of the lexical rule is of type
  \type{inflected-word} or something of this kind. In any case it is something that is the same for
  all verbs, not something specific to subject raising verbs. I would present a lexical item for the
lexeme. This would make it possible to use the lexeme type.}
\eas
\type{seem}:\\
\avm{
[\type*{subj-rsg-v-word}
subj & <\1 > \\
comps & <\2 VP[head & [vform & inf] \\
		       subj  & <\1> \\
		       cont  & [ind & \3] ]>\\
arg-st & <\1, \2> \\
cont &	[ind & s \\
		rels &	\{[\type*{seem-rel}
				soa & \3]\}]]
}
\zs

\itd{Stefan: Can we use numbers instead of the \ibox{i} or just the index $i$? \ibox{i} looks
  strange \ldots}
\eas
\type{expect}:\\
\avm{
[\type*{obj-rsg-v-word}
subj &	<\1 NP$_{\tag{i}}$> \\
comps &	<\2, \3 VP[head & [vform & inf] \\
			   subj  & <\2> \\
			   cont  & [ind & \4] ]> \\
arg-st & <\1, \2, \3> \\
cont &	[ind & s \\
		rels & \{[\type*{expect-rel}
			  exp & \tag{i} \\
			  soa & \4]\} ] ]
}
\zs
They take a VP and not a clausal complement, which means that the complements expected by the
infinitive are realized locally but not its subject. The  corresponding simplified trees are as
shown in Figures~\ref{cons-1} and~\ref{cons2}. Notice that the syntactic structures are the same. 
\itd{Stefan: ``Notice that the syntactic structures are the same.'' They are not.}
\begin{figure}
% \begin{tikzpicture}[baseline, sibling distance=2pt, level distance=60pt, scale=.9]
% 	\Tree
% 	[.{\begin{avm}
% 		\[phon & \phonliste{ Paul seems to sleep }\\
% 			subj & \eliste \\
% 			comps & \eliste\]
% 		\end{avm}}
% 		{\begin{avm}\[phon & \phonliste{ Paul } \\
% 			synsem & \@1 \]
% 		\end{avm}}
% 		[.{\begin{avm}
% 			\[phon & \phonliste{ seems to sleep }\\
% 			subj & \<\@1\>\]
% 			\end{avm}}
% 		 {\begin{avm}
% 			\[phon & \phonliste{ seems } \\
% 			subj & \<\@1\>\\
% 			comps & \<\@2 \[subj & \< \@1 \>\]\>\\
% 			\]
% 			\end{avm}} 
% 		{\begin{avm}
% 			\[phon \phonliste{ to sleep }\\
% 				synsem \@2 \]	
% 			\end{avm}}  
% 		]
% 	]
% \end{tikzpicture}
\begin{forest}
[
\avm{
[S\\
\phon <Paul seems to sleep> \\
subj & < > \\
comps & < > ]
}
	[
	\avm{
	[NP \\
  	\phon <Paul> \\
	synsem & \1 ]
	}
	]
	[
	\avm{
	[VP \\
	\phon <seems to sleep> \\
	subj & <\1> \\
	comps & < > ]
	}
    	[
    	\avm{
        [V\\
        \phon <seems> \\
          subj & <\1> \\
          comps & <\2	[subj & <\1>]> ]
		}
		]
    	[
    	\avm{
        [VP \\
        \phon <to sleep> \\
        synsem & \2 ]	
		}
		]
	]
]
\end{forest}
\caption{\label{cons-1}A sentence with a subject-raising verb}
\end{figure}

\begin{figure}
\begin{forest}
[
\avm{
[S\\
\phon <Mary expected Paul to work> \\
        subj & < > \\
        comps & < > ]		
}
	[
	\avm{
	[NP \\
	\phon <Mary> \\
	synsem & \1 ]
	}
	]
	[
	\avm{
	[VP \\
	\phon <expected Paul to work> \\
	subj & <\1> \\
	comps & < > ]
	}
		[
		\avm{
		[V \\
		\phon <expected> \\
		subj & <\1 > \\
		comps & <\2, \3	[subj & <\2>]> ]		
		}
		]
		[
		\avm{
		[NP \\
		\phon <Paul> \\
		synsem & \2 ]
		}
		]
		[
		\avm{
		[VP \\
		\phon <to work> \\
              synsem & \3 ]
		}
		]
	]
]
\end{forest}	
\caption{\label{cons2}A sentence with an object"=raising verb}
\end{figure}

Raising verbs have in common a mismatch between syntactic and
semantic arguments: the raising verb has a subject (or an object) which is not one of its semantic
arguments (its \textsc{index} does not appear in the \cont feature of the raising verb). To constrain this type of
mismatch, \citet[140]{PollardandSag1994} propose the Raising Principle.

\ea
Raising Principle: Let X be a non"=expletive element subcategorized by Y, X is not assigned any
semantic role by Y iff Y also subcategorizes a complement which has X as its first argument.
\z

This principle was meant to prevent raising verbs from omitting their VP complement, unlike control
verbs \citep{Jacobson1990}\addpages. Without a non"=finite complement, the subject of \word{seem} is not
assigned any semantic role, which violates the Raising principle. However, some unexpressed (null)
complements are possible with some subject-raising verbs as well as VP ellipsis with \ili{English}
auxiliaries, which are analysed as subject-raising verbs (see
Section~\ref{sec-auxiliaries-as-raising-verbs} below and
\crossrefchaptert[Section~\ref{sec-analyses-of-pred-ellipsis}]{ellipsis} on predicate/argument ellipsis). So the
Raising Principle should be reformulated in terms of argument-structure and not valence features. 

\eal
\ex[]{John tried / * seems.
}
\ex[]{
John just started.
}
\ex[]{
John did.
}
\zl

For subject-raising verbs which allow for a sentential complement as well (with an expletive
subject) (\ref{seem4}), another lexical description is needed, and the same holds for
object"=raising verbs which allow a sentential complement (with no object) (\ref{expect3}). These
can be seen as valence alternations, which are available for some items (or some classes of items)
but not all (see \crossrefchapteralt{arg-st} on argument structure). 

\eal
\ex \emph{seem}:   [\argst \sliste{ NP[\type{it}], S }]
\ex \emph{expect}: [\argst \sliste{ NP, S }]
\zl

\subsection{The HPSG analysis of control verbs}

\citet{SagandPollard1991} propose a semantics-based control theory. The semantic class of the verb
determines whether it is subject-control or object-control: they distinguish verbs of orientation
(\emph{want}, \emph{hope}), verbs of commitment (\emph{promise}, \emph{try}) and verbs of influence
(\emph{persuade}, \emph{forbid}) based on the type of relation and semantic roles of their
arguments. Relational types for control predicates can be organized in a type hierarchy like the one
given in Figure~\ref{verb-hier3}.
\itd{Stefan: Where is this hierarchy from? It is not Pollard\&Sag. Do you have a reference? Why is
  it different from what P\&S did?}

\itd{Stefan: Having the \type{forbid-rel} as a subtype of \type{persuade-rel} does not work in the
  model-theoretic world, since all type have to be maximal in models. This would mean that all
  relations of type \type{persuade-rel} are always automatically \type{forbid-rel}.} 
\begin{figure}
	\begin{forest}type hierarchy
       [control-relation
      					[orientation-rel
      						[want-rel] 
      						 [hope-rel]
      						 [\ldots]   		
      					] 
      					[commitment-rel
      					 		[promise-rel
      					 			[try-rel]
      					 		[\ldots]
      					 	]
      					 	 [influence-rel
      					 		[persuade-rel
      					 			[forbid-rel]
      					 		]
      					 		[\ldots]
      					 	]
      					 	[\ldots]
      					]  
      	]
\end{forest}
\caption{\label{verb-hier3}A type hierarchy for control predicates}
\end{figure}

For example, \emph{want}, \emph{promise} and \emph{persuade} have a semantic content such as the following, with SOA meaning state-of-affairs and denoting the content of the non"=finite complement:

\itd{Stefan: Stating \type{relation} in the lexical items seems redundant since this is the most
  general type for somthing having \textsc{arg} as a feature and being the value of \textsc{soa}.}
\eal
%\word{want}:\\
\ex 
\avm{
[\type*{want-rel}
 experiencer & \1 \\
 soa [\type*{relation}  \\
      arg & \1] ]
}
\ex
%\word{promise}:\\
\avm{
[\type*{promise-rel}
 commitor & \1 \\
 commitee & \2 \\
 soa 	[\type*{relation}  \\
	 arg & \1]]
}
\ex
%\word{persuade}:\\*
\avm{
[\type*{persuade-rel}
 influencer & \1 \\
 influenced & \2 \\
 soa 	[\type*{relation}  \\
	 arg & \2]]
}	
\zl

According to this theory, the controller is the experiencer with verbs of orientation, the commitor
with verbs of commitment, and the influencer with verbs of influence. From the syntactic point of
view, two types of control predicates, \type{subject-cont-lx} and \type{object-cont-lx}, can be
defined as follows:

\eal
\label{cont}
\ex
\label{subj-cont-lx}
\type{subj-cont-lx} \impl\\
\avm{ [ \argst  < NP$_{\tag{i}}$, \ldots, [subj & < [ind & \tag{i} ] > ] > ]}
\itd{Stefan: \ibox{1} is not shared. Remove? Start with \ibox{1} rather than \ibox{0}? Like
  (\mex{1}c)? Replace non-shared tags with [].}
\ex
\type{obj-cont-lx} \impl\\ 
\avm{ [ \argst  < \0, \1 [ind & \tag{i}], [subj & < [ind & \tag{i} ] > ] > ]}
\ex
\type{obj-cont-lx} \impl\\ 
\avm{ [ \argst  < \1, [ind & \tag{i}], [subj & < [ind & \tag{i} ] > ] > ]}
\zl

The controller is the first argument with subject-control verbs, while it is the second argument
with object-control verbs. Contrary to the types defined for raising predicates in (\ref{rsg}), the
controller here is simply coindexed with the subject of the non"=finite complement. 
\itd{Stefan: ``This means it must have a semantic role''. This is not correct. Expletives also have
  an index. A better formulation would be: ``Since the controller is referential and since it is
  coindexed with the controlee, the controlee has to be referential as well.''. I would also talk
  about ``non-referential idiom parts'' since there are decomposable idioms with referential parts.} 
This means it
must have a semantic role (since it has a semantic index), thus expletives and idiom parts are not
allowed ((\ref{rain3}), (\ref{rain4}), (\ref{cat3}), (\ref{cat4})). This also implies that its
syntactic features may differ from those of the subject of the non"=finite complement: it may have a
different part of speech (a NP subject can be coindexed with a PP controller) as well as a different
case ((\ref{to}), (\ref{hope-i})).

Verbs of orientation and commitment inherit from the type \type{subj-cont-lx}, while verbs of
influence inherit from the type \type{subj-cont-lx}.  A subject-control verb (\word{want}) and an
object-control verb (\word{persuade}) inherit from \type{subj-cont-v} and \type{obj-cont-v}
respectively; their lexical descriptions are as follows:\footnote{To account for Visser's
  generalization (object-control verbs passivize while subject-control verbs do not),
  \citet{SagandPollard1991} analyse the subject of the infinitive as a reflexive, which must be
  bound by the controller. According to Binding Theory (see \crossrefchapteralp{binding}), the
  controller must be less oblique than the reflexive, hence less oblique than the VP complement
  which contains the reflexive: the controller can be the subject and the VP a complement as in
  (\ref{ex-John-promised-Mary-to-buy}) and (\ref{persuade-pass}); it can be the first complement
  when the VP is the second complement as in (\ref{ex-John-permitted-Mary-to-buy}), but it cannot be
  a \emph{by}-phrase, which is more oblique than the VP complement, as in (\ref{promise-pass}) (the
  \emph{by}-phrase should not be bound according to principle C, and the subject of the infinitive
  should be bound according to principle A).}

\eas
\type{want}:\\
\avm{
[\type*{subj-cont-v-word}
	subj  & < \1 NP$_{\tag{i}}$ > \\
	comps & < \2 VP	[head &	[vform & inf] \\
					subj &	<[ind & \tag{i}]> \\
					cont & [ind & \3] ]> \\
	arg-st & <\1, \2> \\
	cont &	[ind & s \\
			rels &	\{[\type*{want-rel}
					exp & \tag{i} \\
					soa & \3]\}] ]
}
\zs
\eas
\type{persuade}:\\*
\avm{
	[\type*{obj-cont-v-word} \\
	subj & <\1 NP$_{\tag{i}}$ > \\
	comps & <\2 NP$_{\tag{j}}$, \3 VP	[head &	[vform & inf] \\
	subj &	<[ind & \tag{j}]> \\
	cont & [ind & \4] ]>\\
	arg-st & <\1, \2, \3> \\
	cont &	[ind & s \\
			rels & \{[\type*{persuade-rel}
			agent & \tag{i} \\
			patient & \tag{j} \\
			soa & \4]\} ] ]
}
\zs

The corresponding structures for subject-control and object-control sentences are illustrated in Figures~\ref{sleep3} and~\ref{cons3}:

\begin{figure}
% fails 03.05.2020 23:23
% \tikzexternaldisable
%  \begin{tikzpicture}[baseline, sibling distance=2pt, level distance=60pt, scale=.9]
% 	\Tree
% 	[.{\begin{avm}
% 		\[phon & \phonliste{ Paul wants to sleep }\\
% 			subj & \eliste \\
% 			comps & \eliste\]
% 		\end{avm}}
% 		{\begin{avm}\[phon \phonliste{ Paul } \\
% 			synsem \@1 \]
% 		\end{avm}}
% 		[.{\begin{avm}
% 			\[phon & \phonliste{ wants to sleep }\\
% 			subj & \<\@1\>\]
% 			\end{avm}}
% 		 {\begin{avm}
% 			\[phon & \phonliste{ wants } \\
% 			subj & \<\@1 \[ cont|ind  \type{i} \] \>\\
% 			comps & \<\@2 \[subj & \< \normalfont NP_{i} \> \]\>\\
% 			\]
% 			\end{avm}} 
% 		{\begin{avm}
% 			\[phon & \phonliste{ to sleep }\\
% 				synsem & \@2  \]	
% 			\end{avm}}  
% 		]
% 	]
% \end{tikzpicture}
% This one fails forest even without externalization
\begin{forest}
[
\avm{
[S\\
\phon <Paul wants to sleep> \\
      subj & < > \\
      comps & < > ]
}
	[
	\avm{
	[NP\\
    \phon <Paul> \\
      synsem \1 ]
	}
	]
	[
	\avm{
	[VP\\
    \phon <wants to sleep> \\
        subj & <\1> \\
        comps & < > ]
	}
    	[
    	\avm{
        [V\\
        \phon <wants> \\
          subj & <\1 [ind & i] > \\
          comps & <\2 [subj & < NP$_{i}$ > ]> ]
		}
		] 
		[
		\avm{
        [VP\\
        \phon <to sleep> \\
          synsem & \2  ]	
		}
		]
	]
]
\end{forest}
% \begin{forest}
% [\avm{
%    [phon & \phonliste{ Paul wants to sleep }\\
%     subj & \eliste \\
%     comps & \eliste ]
%   }
%   [\avm{
%     [phon \phonliste{ Paul } \\
%      synsem \1 ]
%     }]
%   [\avm{
%       [phon & \phonliste{ wants to sleep }\\
%        subj & < \1 >]
%     }
%     [\avm{
%       [phon  & \phonliste{ wants } \\
%        subj  & < \1 [ cont|ind  \type{i} ] >\\
%        comps & < \2 [ subj & < \normalfont NP$_{i}$ > ] >\\
%         \]
%       }] 
%     [\avm{
%         [phon   & \phonliste{ to sleep }\\
%          synsem & \2  ]	
%      }] ] ]
% \end{forest}
\caption{\label{sleep3}A sentence with a subject-control verb}
\end{figure}



%The corresponding trees are given in Figure~\ref{cons2} and~\ref{cons3}. Notice that the syntactic structures are the same.

%\begin{figure}
% \begin{tikzpicture}[baseline, sibling distance=2pt, level distance=60pt, scale=.9]
% 	\Tree
% 	[.{\begin{avm}
% \[phon & \phonliste{ Mary expected Paul to work }\\
% subj & \eliste\\
% comps & \eliste\]		
% \end{avm}}
% 	{\begin{avm} \[phon & \phonliste{ Mary } \\
% 			synsem & \@3 \]
% 		\end{avm}}
% 	[.{\begin{avm}
% \[phon & \phonliste{ expected Paul to work }\\
% subj & \<\@3 NP\>\\
% comps & \eliste\]		
% \end{avm}}
% 	{\begin{avm}
% \[phon & \phonliste{ expected } \\
% subj & \<\@3 NP\>\\
% comps & \<\@1, \@2 \[
% 		 subj & \@1 \]\>\]		
% \end{avm}}
% 	{\begin{avm} \[phon & \phonliste{ Paul } \\
% 			synsem & \@1 \]
% 		\end{avm}}
% 	{\begin{avm}
% 			\[phon & \phonliste{ to work }\\
% 				synsem & \@2 \]	
% 			\end{avm}}
% 	] ]
% \end{tikzpicture}	
\begin{figure}
% fails 04.05.2020 06:30
% \tikzexternaldisable
% \begin{tikzpicture}[baseline, sibling distance=2pt, level distance=60pt, scale=.9]
% 	\Tree
% 	[.{\begin{avm}
% \[phon & \phonliste{ Mary persuaded Paul to work }\\
% subj & \eliste\\
% comps & \eliste\]		
% \end{avm}}
% 	{\begin{avm}\[phon & \phonliste{ Mary } \\
% 	synsem & \@3
% 			\]
% 		\end{avm}}
% 	[.{\begin{avm}
% \[phon & \phonliste{ persuaded Paul to work }\\
% subj & \<\@3 NP\>\\
% comps & \eliste\]		
% \end{avm}}
% 	{\begin{avm}
% \[phon & \phonliste{ persuaded } \\
% subj & \<\@3 NP\>\\
% comps & \<\@1 \[cont|ind \type{i} \], \@2 \[
% 		 subj & \< NP$_{i}$ \> \]\>\]		
% \end{avm}}
% 	{\begin{avm}\[phon  \phonliste{ Paul } \\
% 		synsem \@1 \]
% 		\end{avm}}
% 	{\begin{avm}
% 			\[phon & \phonliste{ to work }\\
% 				synsem & \@2  \]	
% 			\end{avm}}
% 	] ]
% \end{tikzpicture}	
\oneline{
\begin{forest}
[
\avm{
[S\\
\phon <Mary persuaded Paul to work> \\
      subj & < > \\
      comps & < > ]		
}
	[
	\avm{
	[NP\\
	\phon <Mary> \\
	synsem & \3 ]
	}
	]
	[
	\avm{
	[VP\\
	\phon <persuaded Paul to work> \\
        subj & <\3 NP> \\
        comps & < >]
	}
		[
		\avm{
        [V\\
         \phon <persuaded> \\
          subj & <\3 NP>\\
          comps & <\1	[ind & i], \2	[subj & < NP$_{i}$> ]> ]		
		}
		]
		[
		\avm{
		[NP\\
		\phon <Paul> \\
          synsem \1 ]
		}
		]
		[
		\avm{
        [VP\\
        \phon <to work> \\
          synsem & \2  ]
		}
		]
	]
]
\end{forest}
}	
\caption{\label{cons3}A sentence with an object-control verb}
\end{figure}

In some \ili{Slavic} languages (\ili{Russian}, \ili{Czech}, \ili{Polish}), some subject-control
verbs may allow case sharing as well, as shown by predicate case agreement with quantified
(non"=nominative) subjects. As observed by \citet{Przepiorkowski2004} and
\citet{PrzepiorkowskiandRosen2005}, coindexing does not prevent full sharing: so the analysis may
allow for both cases, and a specific constraint may be added to enforce only case sharing and
\itd{Stefan: This is too dense. The reader is lacking lots of information.}
\itd{Stefan: There was a problem with this analysis. I remember Adam giving a talk at the HPSG
  conference and I pointed out to him that it did not work, but the paper was published already.}
prevent default (instrumental) case assignment to the embedded predicate.\footnote{% 
   The examples in (\mex{1}) are taken from \citew[ex (6)--(7)]{Przepiorkowski2004}.
}

\begin{exe}
\ex \begin{xlist}
\ex 
\gll Janek chce byé miły.\\
     Janek.\textsc{nom} wants be.\textsc{inf} nice.\textsc{nom} \\ \hfill(\ili{Polish})
\glt `Janek wants to be nice.'
\itd{Stefan: It is unclear how this can work with both acc and gen, since the NP will be either acc
  or gen. Looking on Adam's paper, there was no solution to this problem.}
\ex 
\gll Pięć dziewcząt chce być miłe / miłych. \\
     five.\textsc{acc} girls.\textsc{gen} wants be.\textsc{inf} nice.\textsc{acc} {} nice.\textsc{gen}\\
\glt `Five girls want to be nice.'
	\end{xlist}
		
\end{exe}


For control verbs which allow for a sentential complement as well ((\ref{hope1}), (\ref{promise1})),
another lexical description of the kind in (\mex{1}) is needed. These can be seen as valence alternations, which are
available for some items (or some classes of items) but not all (see \crossrefchapteralt{arg-st} on argument structure).

\eal
\ex \emph{want}: [\argst \sliste{ NP, S }]
\ex \emph{persuade}: [\argst \sliste{ NP, NP, S }]
\zl




\subsection{Raising and control in Mauritian} \label{sec-maurit}


\ili{Mauritian} is a \ili{French}-based creole, which has raising and control verbs, belonging
roughly to the same semantic classes as in \ili{English} or \ili{French}. Verbs marking aspect or
modality (\word{kontign} `continue', \word{aret} `stop') are subject-raising verbs and causative and
perception verbs (\emph{get} `watch') are object"=raising. Raising verbs differ from TMA (tense
modality aspect) markers by different properties: they are preceded by the negation, which follows
TMA; they can be coordinated unlike TMA \citep[\page 209]{HenriandLaurens2011}:

\eal
\ex[]{ 
\gll To pou kontign ou aret bwar? \\
     2\SG{} \IRR{} continue.\textsc{sf} or stop.\textsc{sf} drink.\textsc{lf}\\\hfill(\ili{Mauritian})
\glt `You will continue or stop drinking?'
}
\ex[*]{
\gll To'nn ou pou aret bwar? \\
     2\SG{}'\PRF{} or \IRR{} stop.\textsc{sf} drink.\textsc{lf}\\
\glt  `You have or will stop drinking?'
}
\zl
 
If their verbal complement has no external argument, as is the case with impersonal expressions such as \word{ena lapli} `to rain', then the raising verb itself has no external argument, contrary to a control verb like \word{sey} `try':

\eal
\ex[]{
\gll Kontign     ena lapli. \\
     continue.\textsc{sf} have.\textsc{sf} rain \\
\glt `It continued to rain.'
}
\ex[*]{
\gll Sey ena lapli. \\
     try have.\textsc{sf} rain \\
\glt Literally: `It tries to rain.'
}
\zl

Unlike in \ili{French}, its superstrate, in \ili{Mauritian},  verbs neither inflect for tense, mood and aspect nor for person, number, and
gender. But they have a short form and a long form (henceforth \textsc{sf} and \textsc{lf}), with
30\,\% verbs showing a syncretic form (\emph{bwar} `drink'). The following list of examples provides pairs of short and
long forms respectively:

\eal
\ex manz/manze `eat', koz/koze `talk', sant/sante `sing'
\ex pans/panse `think', kontign/kontigne `continue', konn/kone `know'
\zl

As described in \citet{Henri2010}, the verb form is determined by the construction: the short form is required before a phrasal complement and the long form appears otherwise.\footnote{\textit{yer} `yesterday' is an adjunct. See \citew{Hassamal2017} for an analysis of \ili{Mauritian} adverbs which treats as complements those triggering the verb short form.}


\begin{exe}
\ex \begin{xlist}
\ex 
\gll Zan sant sega / manz pom / trov so mama / pans Paris. \\
     Zan sing.\textsc{sf} sega {} eat.\textsc{sf} apple {} find.\textsc{sf} \POSS{} mother {} think.\textsc{sf} Paris \\
\glt `Zan sings a sega / eats an apple / finds his mother / thinks about Paris.'	
\ex 
\gll Zan sante / manze.\\
     Zan sing.\textsc{lf} {} eat.\textsc{lf}\\
\glt `Zan sings / eats.'
\ex 
\gll Zan ti zante yer. \\
Zan  \PRF{} sing.\textsc{lf} yesterday\\
\glt `Zan sang yesterday.'
\end{xlist}
\end{exe}


\citet{Henri2010} proposes to define two possible values (\type{sf} and \type{lf}) for the head
feature \vform, with the following lexical constraint (\type{nelist} stands for non-empty list):

\begin{exe}       
\ex
\avm{
	[vform & sf] \impl [comps & nelist]
}
\end{exe}
Interestingly, clausal complements do not trigger the verb short form (\citet{Henri2010}\addpages analyses them as extraposed). The complementizer (\emph{ki}) is optional.

\eal
\ex 
\gll Zan panse             (ki)               Mari pou    vini.\\
     Zan think.\textsc{lf} \hphantom{(}that Mari \FUT{} come.\textsc{lf}\\
\glt `Zan thinks that Mari will come.'
\ex 
\gll Mari trouve           (ki)                so      mama   tro      manze.\\
     Mari find.\textsc{lf} \hphantom{(}that  \POSS{} mother too.much eat.\textsc{lf}\\
\glt `Mari finds that her mother eats too much.'
\zl

On the other hand, subject-raising and subject-control verbs occur in a short form before a verbal complement.

\begin{exe}
\ex \begin{xlist}
\ex \gll Zan kontign sante.\\
Zan continue.\textsc{sf} sing.\textsc{lf}\\\jambox*{(subject-raising verb, p.\,198)}
\glt `Zan continues to sing.'
\ex \gll Zan sey sante.\\
Zan try.\textsc{sf} sing.\textsc{lf}\\\jambox{(subject-control verb)}
\glt `Zan tries to sing.'
\end{xlist}
\end{exe}

The same is true with object-control and object"=raising verbs:
\eal
\settowidth\jamwidth{(object"=raising verb, p.\,200)}
\ex \gll Zan inn fors Mari vini.\\
Zan \PRF{} force.\textsc{sf} Mari come.\textsc{lf}\\\jambox{(object-control verb)}
\glt `Zan has forced Mari to come.'
\ex \gll Zan pe get Mari dormi.\\
Zan \PROG{} watch.\textsc{sf} Mari sleep.\textsc{lf}\\\jambox{(object"=raising verb, p.\,200)}
\glt `Zan is watching Mari sleep.'
\end{xlist}
\end{exe}


Raising  and control verbs thus differ from verbs taking sentential complements. Their \textsc{sf} form is
predicted if they take unsaturated VP complements. Assuming the same lexical type hierarchy as
defined above, verbs like \word{kontign} `continue' and \word{sey} `try' inherit from
\itd{Stefan: \type{subj-rsg-lx}?}
\type{subj-rsg-v} and \type{subj-cont-v} respectively 
%and have the following lexical entries 
\footnote{Henri \& Laurens use Sign-based Construction Grammar (SBCG) (see
  \crossrefchapteralp[Section~\ref{prop:sec-sbcg}]{properties} and
  \crossrefchapteralp[Section~\ref{cxg:sec-sbcg}]{cxg}), but their analyses can be adapted to the
  feature geometry of Constructional HPSG \citep{Sag97a} assumed in this volume. The analysis of
  control verbs sketched here will be revised in Section~\ref{section-xarg} below.} \citet[\page
197]{HenriandLaurens2011} conclude that ``while \ili{Mauritian} data can be brought in accordance
with the open complement analysis, both morphological data on the control or raising verb and the
existence of genuine verbless clauses put up a big challenge for both the clause and small clause
analysis.''

%\begin{exe}
%\ex \word{kontign} `continue':\\
%\begin{avm}
%	\[head & sf\\
%subj & \<\@1 \> \\
%	comps & \<VP\[%head & verb \\
%		%marking & unmark\\
%		subj & \<\@1\> \\
%		cont & \[ind & \@2\] \]\>\\
%	cont & \[ind & s \\
%			rels & \{\[\asort{continue-rel}
%			arg & \@2\]\}\]
%	\]
%\end{avm}
%\ex \word{sey} `try':\\*
%\begin{avm}
%	\[head & sf\\
%subj & \<NP$_{\@i}$ \> \\
%	comps & \<VP\[%head & verb \\
%		%marking & unmark\\
%		subj & \<\[ind & \@i\]\> \\
%		cont & \[ind & \@2\] \]\>\\
%	cont & \[ind & s \\
%			rels & \{\[\asort{try-rel} \\
%			exp & \@i \\
%			arg & \@2\]\}\]
%	\]
%\end{avm}	
%\end{exe}



\subsection{Raising and control in prodrop and ergative languages}

The theory of raising and control presented above naturally extends to prodrop and ergative
languages.  Since \citet*{BMS2001a}, it is assumed\itd{Stefan: Not everybody assumes this. I do not.
  See also \citew{LH2006a}.} that syntactic arguments are listed in \argst and
that only canonical ones are present in the valence lists (\subj, \spr and \comps). See for example
\itd{Stefan: I think the UDC chapter does not contain such a discussion.}
\crossrefchaptert{udc} for an analysis of UDC with non"=canonical \type{synsem}. For \emph{pro}-drop languages,
it has been proposed, e.g. in \citep[\page 65]{ManningandSag1998} that null subject sentences have
an element representing the understood subject in the \argst list of the main verb but nothing in
the \subj list.

\eal
\ex 
\label{Italian}
\gll Vengo.\\
     come.\PRS.1\SG\\\hfill(\ili{Italian})
\glt `I come.'
\ex 
\label{Italian-raising}
\gll Posso venire.\\
     can.1\SG{} come.\INF\\
\glt `I can come.'
\ex 
\label{Italian-control}
\gll Voglio venire.\\ 
     want.1\SG{} come.\INF\\
\glt `I want to come.'
\zl

Assuming the lexical types for subj-rsg-lexmes and subj-cont-lexemes in (\ref{rsg}) and
(\ref{cont}), the verbal descriptions for (\ref{Italian-raising}) and (\ref{Italian-control}) are as
follows:
%Vengo : SUBJ <>, Comps <>, ARG-ST<[pro]>
%Posso : SUBJ  <>, Comps <2>, Arg-st <1[pro], 2VP[SUBJ <1>]>
%Voglio SUBJ  <>, Comps <2>, Arg-st <NPi[pro], 2VP[SUBJ <NPi>]>
\eal
\ex	
\type{posso} `can':\\
\avm{
	[subj & elist \\
	comps & <\2> \\
	arg-st & <\1[\type{pro}], \2[subj & \1]>]
}\label{rais1}
\ex 
\type{voglio} `want':\\
\avm{
	[subj & elist \\
	comps & < \2> \\
	arg-st & < NP$_{\tag{i}}$[\type{pro}], \2[subj & < [ind & \tag{i}] >] > ]
}
\zl


\ili{Balinese}\il{Balinese|(} offers an intriguing case of syntactic ergativity. It displays rigid SVO order,
regardless of the verb's voice form \citep{WechslerandArka1998}. In the agentive voice (AV), the
subject is the \argst initial member, while in the objective voice (OV), the verb is transitive, and
the subject is the initial NP, although it is not the first argument.\itd{Stefan: This is confusing
  since it is unclear what ``first'' refers to. In the utterance the subject is first but not in
  \argst. I would suggest ``although it is not the first element of the \argstl''.}  (see
\crossrefchapteralp[Section~\ref{arg-st-sec-ergativity}]{arg-st}):

\eal
\ex  
\gll Ida ng-adol bawi.\\
     3\SG{} \textsc{av}-sell pig\\ \hfill(\ili{Balinese})
\glt `He/She sold a pig.'
\ex 
\gll Bawi adol ida.\\
     pig \textsc{ov}.sell 3\SG \\
\glt `He/She sold a pig.' 
\zl

Different properties argue in favor of a subject status of the first NP in the objective
voice. Binding properties show that the agent is always the first element on the \argst list, see
\citew{WechslerandArka1998}, \citew{ManningandSag1998} and \crossrefchaptert{binding}. The objective
voice is also different from the passive: the passive may have a passive prefix, an agent
\emph{by}-phrase, and does not constrain the thematic role of its subject. The two verbal types can
be defined as follows (see \crossrefchapteralp[Section~\ref{arg-st-sec-ergativity}]{arg-st}):

\itd{Stefan: Both \ibox{1} and \ibox{2} may be the empty list or more generally lists of arbitrary length.}
\eal
\ex 
\type{av-verb} \impl\\
\avm{
	[subj & \1 \\
	comps & \2 \\
	arg-st & \1 \+ \2 ] 
}
\ex 
\type{ov-verb} \impl\\
\avm{
	[subj & \1 \\
	comps & \2 \\
	arg-st & \2 \+ \1 ]
}
\zl

In this analysis, the preverbal argument, whether the theme of an OV verb or the agent of an AV
verb, is the subject, and in many languages, only a subject can be raised or controlled
\citep{Zaenenetal1985}. Thus the first argument of the verb is controlled when the embedded verb is
in the agentive voice, and the second argument\itd{Stefan: What about ditransitive verbs?} is controlled when the verb is in the objective
voice.\footnote{%
  The examples in (\mex{1}) are taken from \citew[ex 25]{WechslerandArka1998}.
}


\begin{exe}
\ex \begin{xlist}
\ex 
\gll Tiang edot [ \trace{} teka].\\
     1 want     {} {} come\\\hfill(\ili{Balinese})
\glt `I want to come.'
\ex 
\gll Tiang edot [ \trace{}  meriksa dokter].\\
     1     want {} {}     \textsc{av}.examine doctor\\
\glt `I want to examine a doctor.'
\ex 
\gll Tiang edot [ \trace{} periksa dokter].\\
     1     want {} {}    \textsc{ov}.examine doctor\\
\glt `I want to be examined by a doctor.'
\end{xlist}
\end{exe}

Turning to \word{majanji} `promise', in this type of commitment relation, the promiser must have
semantic control over the action promised \citep{Farkas1988,Kroeger1993,SagandPollard1991}\addpages. The
promiser should therefore be the actor of the downstairs verb. This semantic constraint interacts
with the syntactic constraint that the controllee must be the subject, to predict that the
controlled VP must be in AV voice, which places the agent\itd{Stefan: Actor or agent?} in the subject role:\footnote{%
  The examples in (\mex{1}) are taken from \citew[ex 27]{WechslerandArka1998}.
}
\eal
\ex[]{
\gll Tiang majanji maang Nyoman pipis.\\
     1 promise \textsc{av}.give Nyoman money\\\hfill(\ili{Balinese})
\glt `I promised to give Nyoman money.' 
}
\ex[*]{ 
\gll Tiang majanji Nyoman baang pipis. \\
     1 promise Nyoman \textsc{ov}.give money \\
}
\ex[*]{ 
\gll Tiang majanji pipis baang Nyoman. \\
     1 promise money \textsc{ov}.give Nyoman\\ 
}
\zl
The same facts obtain for other control verbs such as \word{paksa} `force'.

\ili{Balinese} also has subject-raising verbs like \word{ngenah} `seem':\footnote{
The examples in (\mex{1}) are taken from \citew[ex 7]{WechslerandArka1998}.
}

\eal
\ex 
\gll Ngenah ia mobog.\\
     seem 3 lie\\\hfill(\ili{Balinese})
\glt `It seems that (s)he is lying.'
\ex 
\gll  Ia ngenah mobog.\\
      3 seem lie\\
\glt `(S)he seems to be lying.'
\zl

As predicted, the agent can be ``raised'' when the embedded verb is in the agentive voice, since it is the subject:\footnote{
The examples in (\mex{1}) are taken from \citew[ex 9]{WechslerandArka1998}.
}
%The same applies to a transitive verb in the agentive voice: the agent can %appear as the subject of \emph{ngenah} `seem' but not the patient.

\eal
\judgewidth{?*}
%\ex[]{ 
%\gll Ngenah sajan [ci ngengkebang kapelihan-ne].\\
  %   seem much \spacebr{}2 \textsc{av}.hide mistake-3\POSS\\\hfill\citep[ex 9]{WechslerandArka1998}
%\glt `It is very apparent that you are hiding his/her wrongdoing.'
%}
\ex[]{
\gll Ci ngenah sajan ngengkebang kapelihan-ne.\\
     2 seem much \textsc{av}.hide mistake-3\POSS\\
\glt `You seem to be hiding his/her wrongdoing.'
}
\ex[?*]{ 
\gll Kapelihan-ne ngenah sajan ci ngengkebang.\\
     mistake-3\POSS{} seem much 2 \textsc{av}.hide\\
}
\zl

On the other hand, only the patient can be ``raised'' (because that is the subject) when the embedded verb is in the objective voice:\footnote{
The examples in (\mex{1}) are taken from \citew[ex 8]{WechslerandArka1998}.
}

\eal
\judgewidth{?*}
%\ex[]{ 
%\gll Ngenah sajan [kapelihan-ne engkebang ci].\\
%     seem much \spacebr{}mistake-3\POSS{} \textsc{ov}.hide 2\\\hfill
%\glt `It is very apparent that you are hiding his/her wrongdoing.'
%}
\ex[]{
\gll Kapelihan-ne ngenah sajan engkebang ci.\\
     mistake-3\POSS{} seem much \textsc{ov}.hide 2 \\
\glt `His/her wrongdoings seem to be hidden by you.'
}
\ex[?*]{
\gll Ci ngenah sajan kapelihan-ne engkebang.\\
     2 seem much mistake-3\POSS{} \textsc{ov}.hide \\
}
\zl


Turning now to object-raising verbs, like \emph{tawang} `know',  they can occur in the agentive
voice with an embedded AV verb (\ref{av}), and with an embedded OV verb (\ref{ov}), unlike control
verbs like \emph{majanji} `promise'. 
%\ili{Balinese} also displays object"=raising. While the subject of \emph{mulih} %`go home' has been ``raised'' to the
They can also occur in the objective voice, when the subject of the embedded verb is raised.  In
(\ref{rais-av}), the embedded verb (\emph{nangkep} `arrest') is in the agentive voice and its
subject (\emph{polisi} `police') is raised to the subject of \emph{tawang} `know' in the objective
voice; in (\ref{rais-ov}), the embedded verb (\emph{tangkep} `arrest') is in the objective voice and
its subject (\emph{Wayan}) is raised to the subject of \itd{Stefan: The form of the verb was
  wrong. You had \emph{nawang}. I changed it to \emph{tawang} as it occurs in the example.}
\emph{tawang} 'know' in the objective voice \citep[ex 23]{WechslerandArka1998}.

\eal
%\gll
% Nyoman Santosa tawang           tiang  mulih.\\
  %   Nyoman Santosa \textsc{ov}.know 1      go.home\\\hfill\citep[ex 22]%{WechslerandArka1998}
%\glt `I knew that Nyoman Santosa went home.'
%\ex 
%\gll Tiang nawang           Nyoman Santosa mulih.\\
%     1     \textsc{av}.know Nyoman Santosa go.home\\
%\glt `I knew that Nyoman Santosa went home.'
\ex 
\label{av}
\gll Ia nawang          polisi lakar  nangkep            Wayan. \\
     3 \textsc{av}.know police \FUT{} \textsc{av}.arrest Wayan \\
\glt 'He knew that the police would arrest Wayan.'
\ex
\label{rais-av} 
\gll Polisi tawang=a           lakar  nangkep            Wayan. \\
     police \textsc{ov}.know=3 \FUT{} \textsc{av}.arrest Wayan\\

\ex
\label{ov}
\gll Ia nawang           Wayan lakar  tangkep            polisi.\\
     3  \textsc{av}.know Wayan \FUT{} \textsc{ov}.arrest police\\
\glt 'He knew that the police would arrest Wayan.'
\ex
\label{rais-ov}
\gll Wayan tawang=a           lakar  tangkep            polisi. \\
     Wayan \textsc{ov}.know=3 \FUT{} \textsc{ov}.arrest police\\ 
\zl


In \ili{Balinese}, the subject is always the controlled (or ``raised'') element but it is not
necessarily the first argument of the embedded verb. The semantic difference between control verbs
and raising verbs has a consequence for their complementation: raising verbs (which do not constrain
the semantic role of the raised argument) can take verbal complements either in the agentive or
objective voice, while object control verbs (which select an agentive argument) can only take a
verbal complement in the agentive voice. This difference is a result of the analysis of raising and
control presented above, and nothing else has to be added.\il{Balinese|)}
%As a result, it is always the subject of the embedded predicate that is coindexed or shared with an argument of the matrix verb, but the subject is not always the first syntactic argument.

%\eal
%\label{rsg-two}
%\ex	\type{subj-rsg-lx}	\impl \begin{avm} \[arg-st & \@1 \append \<\[subj & \@1\]\>\] \end{avm}
%\ex \type{obj-rsg-lx} \impl \begin{avm} \[arg-st & \<NP\> \append \@1 \append \<\[subj & \@1\]\>\] \end{avm}
%\zl


\subsection{\xarg and a revised HPSG analysis}\label{section-xarg}

\itd{Stefan: \xarg is never really introduced. You have to say that it is a \type{synsem} object
  structure shared with the subject (or whatever).}

Sometimes, obligatory control is also attested for verbal complements with an expressed subject.  As
noted by \citet{Zec87a-u,Farkas1988} and \citet[\page 115--116]{GH2001a-u}, in some languages, such
as \ili{Romanian}, \ili{Japanese} \citep{Kuno76a-u,Iida96a-u} or \ili{Persian} \citep{Karimi2008},
the expressed subject of a verbal complement may display obligatory control. This may be a challenge
for the theory of control presented here, since a clausal complement is a saturated complement, with
an empty \subjl, and the matrix verb cannot access the \subjv of the embedded verb. \citet[\page
89]{SP91a-u} proposed a semantic feature external-argument (\textsc{ext-arg}), which makes the index of the
subject argument available at the clausal level.  \citet{Sag2007a}\addpages proposed to introduce a Head
feature \xarg that takes as its value the first syntactic argument of the head verb, and is
accessible at the clause level.

This is adopted by \citet{HenriandLaurens2011}\addpages for \ili{Mauritian}.  After some subject-control verbs
like \word{pans} `think', the VP complement may have an optional pronominal subject which must be coindexed with the matrix subject. 
%It is not a clausal complement since the matrix verb is in the short form (\textsc{sf}) and not in the long form (see above).

\ea
\gll Zan$_{i}$ pans              pou           (li$_{i}$)            vini.\\
     Zan       think.\textsc{sf} \textsc{comp} \hphantom{(}3\SG{} come.\textsc{lf}  \\\hfill(p.\,202)
\glt `Zan thinks about coming.'
\z

\itd{Stefan: You wrote ``Using XARG, they propose for pans ‘think’ the following description.'' This
  was wrong. What they suggest is totally different in details. I adapted the text so that it is
  correct and cites correctly.}
Using \xarg, \citet[\page 214]{HenriandLaurens2011} propose a description for \word{pans} `think' that is
similar to what is given in (\mex{1}). The complement of
\word{pans} must have an \xarg coindexed with the subject of \word{pans}, but its \subjl is not
constrained: it can be a saturated verbal complement (whose \subjv is the empty list) or a VP
complement (whose \subjv is not the empty list).

\ea
\label{ex-pans-Maritian}
\type{pans} `think':\\
\avm{
	[subj & <NP$_{\tag{i}}$ > \\
	comps &	<[head &	[\type*{verb}\\
						xarg &	[ind & \tag{i}] ] \\
			marking & pou
			%cont & \[ind & \@2\]
			]>
	%cont & \[ind & s \\
		%	rels & \{\[\asort{think-rel}
		%	arg \@i \\
		%	arg & \@2\]\}\]
	]
}
\z

 See also \citew[\page 408--409]{Sag2007a} and \citew{KaySag2009} for the obligatory control of possessive determiners in \ili{English} expressions such as \emph{keep one's cool}, \emph{lose one's temper}, with an \xarg feature on nouns and NPs:
\begin{exe}
\ex \begin{xlist}
\ex John lost his / * her temper.
\ex Mary lost * his / her temper.
\end{xlist}
\end{exe}

Raising may also involve verbs taking a finite complement with a pronominal subject. It is the case in \ili{English} with \emph{look like} which has been called ``copy raising'' (\citealp{Rogers74a-u,Hornstein99a-u} a.o.): it takes a finite complement with an overt subject, but this subject must be coindexed with the matrix subject; it is a raising predicate, as shown by the possibility of the expletive \emph{there}:

\eal
\ex Peter looks like he's tired. / \# Mary is coming.
\ex There looks like there's going to be a storm.\footnote{
\citet[\page 407]{Sag2007a}
}
\zl

The verb \emph{look like} can thus have the subject of its sentential complement  be shared with its
own subject:

\itd{Stefan: Just an interesting thought: This would not work in SBCG since whole signs are
  shared. So you have \phon available in \ibox{1}.} 
\ea
\emph{look like}:\\
\avm{
[ \argst < \1, S[xarg & \1] > ]}
\z

This is also the case in \ili{English} tag questions, since the subject of the tag question must be
expressed and shared with that of the matrix clause:
\itd{Stefan: This would not work since the subject of the main clause has semantics if it is a full
  NP that is different from the semantics of a pronoun. So identification would fail. Ivan argued
  that agreement must be semantic. So syntactic features may differ as well.}
\eal
\ex Paul left, didn't he?
\ex It rained yesterday, didn't it ?
\zl


The types for subject-raising and subject-control verb lexemes in (\ref{rais-1}) and (\ref{subj-cont-lx}) will thus be revised as follows:\\
\eal
\ex \type{subj-rsg-lx}  \impl \avm{ [\argst \1 \+ < [x-arg & \1 ] > ] } 
\ex \type{subj-cont-lx} \impl \avm{ [\argst < NP$_{\tag{i}}$, \ldots, [x-arg & < [ind & \tag{i} ] > ] > ] }
\zl

\section{Copular constructions}
\label{sec-copular-constructions}

Copular verbs can also be considered as ``raising'' verbs \citep[\page 106]{Chomsky81a}.  While
attributive adjectives are adjoined to N or NP, predicative adjectives are complements of copular
verbs and share their subject with these verbs. Like raising verbs
(Section~\ref{sec-more-on-raising-verbs}), copular verbs come in two varieties: subject copular
verbs (\word{be}, \word{get}, \word{seem}, \ldots), and object copular verbs (\word{consider},
\word{prove}, \word{expect}, \ldots).

Let us review a few properties of copular constructions.
The adjective selects for the verb's subject or object: \word{likely} may select a nominal or a
sentential argument, while \word{expensive} only takes a nominal argument. As a result, \word{seem}
combined with \word{expensive} only takes a nominal subject, and \word{consider} combined with the
same adjective only takes a nominal object. 


\begin{exe}
\ex \label{storm}
\begin{xlist}
\ex{} [A storm] / [That it rains] seems likely.
\ex{} [This trip] / * [That he comes ] seems expensive.
\end{xlist}
\ex \begin{xlist}
\ex 	I consider [a storm] likely / likely [that it rains].
\ex 	I consider [this trip] expensive/ * expensive [that he comes].
\end{xlist}	
\end{exe}


A copular verb thus takes any subject (or object) allowed by the predicate: \emph{be} can take a PP
subject in \ili{English} (\ref{under2}), \emph{werden} takes no subject when combined with a
subjectless predicate like \emph{schlecht} `sick' in \ili{German} (\ref{german3}):

\eal
\ex{}[Under the bed] is a good place to hide \label{under2}
\ex
\label{german3} 
\gll Ihm        wurde schlecht.\footnotemark\\
     him.\DAT{} got   sick\\\hfill(\ili{German})
\footnotetext{\citet[\page 72]{Mueller2002b}}
\glt `He got sick.'
\zl

 In \ili{English}, \word{be} also has the properties of an auxiliary, see Section~\ref{control-sec-copula-verbs}.

\subsection{The problems with a small clause analysis}

To account for these properties, Transformational Grammar since \citet{Stowell1983}\addpages and
\citet{Chomsky1986}\addpages has proposed a clausal or \emph{small clause} analysis: the predicative
adjective heads a (small) clause; the subject of the adjective raises to the subject position of the
embedding clause (\ref{rais-transformational}) or stays in its subject position and receives accusative case from
the matrix verb via so-called Exceptional Case Marking\is{Exceptional Case Marking} (ECM) (\ref{ecm}).


\eal
\ex
\label{rais-transformational}
{}[\sub{NP} e] be [\sub{S} John sick] $\leadsto$  [\sub{NP} John ] is  [\sub{S} $e_{i}$ sick]
\ex
\label{ecm}
We consider [\sub{S} John sick]
\zl

It is true that the adjective may combine with its subject to form a verbless sentence. It is the
case in African American Vernacular English (AAVE)\il{English!African American Vernacular} \citep{Bender2001a}, in \ili{French} \citep{Laurens2008} and creole languages
\citep{HenriandAbeille2007}, in \ili{Slavic} languages \citep{Zec87a-u}, and in \ili{Semitic} languages (see
\citealp{Alqurashi:Borsley:14}), among others. 

\ea
\gll Magnifique ce chapeau !\\
     beautiful this hat\\\hfill{(\ili{French})}
\glt `What a beautiful hat!'
\z

\noindent
But this does not entail that \emph{be} takes a sentential complement. 


%a problem for the raising principle? In \ili{French}, and other \ili{Romance} languages (Abeillé and Godard 2000), the predicate can be pronominalized as a complement:\\

%\begin{exe}
%\ex \gll Paul est malade / médecin / en forme.
%Paul is sick / a doctor / in a good shape\\
%\ex \gll Paul l'est.
%Paul it is\\
%\glt Paul is so
%\end{exe}

\citet[Chapter~3]{PollardandSag1994} present several arguments against a (small) clause analysis. The putative sentential source is sometimes attested (\ref{cons1}) but more often ungrammatical:

	
\eal
\ex[]{
John gets / becomes sick.
}
\ex[*]{
It gets / becomes that John is sick.
}
\ex[]{
\label{cons1}
John considers Lou a friend / that Lou is a friend.
}
\ex[]{
Paul regards Mary as crazy.
}
\ex[*]{
Paul regards that Mary is crazy.
}
\zl

	
When a clausal complement is possible, its properties differ from those of the putative small clause. Pseudo-clefting shows that \textit{Lou a friend} is not a constituent in (\ref{consider}).

\eal
\ex[]{
We consider Lou a friend.\label{consider}
}
\ex[*]{
What we consider is Lou a friend.
}
\ex[]{
We consider [that Lou is a friend].
}
\ex[]{
What we consider is [that Lou is a friend].
}
\zl

Following \citet{Bresnan1982}\addpages, \citet[113]{PollardandSag1994} also show that Heavy-NP shift
applies to the putative subject of the small clause, exactly as it applies to the first complement
of a ditransitive verb:

\eal
\ex We would consider [any candidat] [acceptable].
\ex We would consider [acceptable]  [any candidate who supports the proposed amendment].
\ex I showed [all the cookies] [to Dana].
\ex I showed [to Dana]  [all the cookies that could be made from betel nuts and molasses].  
\zl

\itd{Stefan: Why quotes around subject?}
Indeed, the ``subject'' of the adjective with object"=raising verbs has all the properties of an
object: it bears accusative case and it can be the subject of a passive:

\eal
\ex We consider him / * he guilty.
\ex We consider that he / * him is guilty.
\ex He was proved guilty (by the jury).	
\zl
	

Furthermore, the matrix verb may select the head of the putative small clause, which is not the case
with verbs taking a clausal complement, and which violates the 
\itd{Stefan: ``locality of subcategorization'' is not explained. Why is it important? What is it?
  cite Sag with various proposals.}
locality of subcategorization \citep[\page 102]{PollardandSag1994}. The
verb \word{expect} takes a predicative adjective but not a preposition or a nominal predicate (\ref{ex-expect}),
\word{get} selects a predicative adjective or a preposition (\ref{ex-get}), but not a predicative nominal, while
\word{prove} selects a predicative noun or adjective but not a preposition (\ref{ex-prove}).


\eal
\label{ex-expect}
\ex I expect that man (to be) dead  by tomorrow. \citep[\page 102]{PollardandSag1994}
\ex I expect that island *(to be) off the route. (p.\,103)
\ex I expect that island *(to be) a good vacation spot. (p.\,103)
\zl
\ea
\label{ex-get}
John got political / * a success. (p.\,105)	
\z
\eal
\label{ex-prove}
\ex Tracy proved the theorem (to be) false. (p.\,100)
\ex I proved the weapon *(to be) in his possession.	(p.\,101)
\zl
	


\subsection{An HPSG analysis of copular verbs}
\label{control-sec-copula-verbs}
	
Copular verbs such as \word{be} or \word{consider} are analysed as subtypes of subject-raising verbs
and object"=raising verbs respectively and hence, the constraints in (\ref{rsg}) apply. They share their subject (or object) with the
unexpressed subject of their predicative complement. Instead of taking a VP complement, they take a
predicative complement (\prd $+$), which they may select the category of.  The two lexical types for
verbs that take a predicative complement are as follows:

\inlinetodoobl{Stefan: subj-prd-lx?}
\itd{Stefan: This is unnecessary if these types are subtypes of (\ref{rsg}). The only thing one
  would have to say is that the last element of the \argstl of respective heads have to be
  \prd+. One type would be sufficient.}
\itd{Stefan: pred-v or prd-lx?}
\eal
\label{ex-subj-pred-v}
\ex \type{subj-pred-v} \impl\\
\avm{ 
[ argst & \1 \+ < [subj & \1\\
                   prd  & $+$ ] > ]}
\ex \type{obj-pred-v} \impl\\ 
\avm{
[ argst & < NP > \+ \1 \+ < [subj & \1\\
                             prd  & $+$ ] > ]}
\zl

A copular verb like \word{be} or \word{seem} does not assign any semantic role to its subject, while
verbs like \word{consider} or \word{expect} do not assign any semantic role to their object. For
more details, see \citew{PollardandSag1994}\addpages,
\textcites[Section~2.2.7]{Mueller2002b}[]{MuellerPredication} and \citew{VanEynde2015}.  The lexical descriptions for predicative \word{seem} and
predicative \word{consider} inherit from the \type{subject-pred-v} type and \type{object-pred-v}
type respectively, and are as follows: \inlinetodostefan{Stefan: fix these lexical items. Raise
  everything? Use append for consider. Should this be \argst? it is easier to show a word that is
  later used in the figure ; AA: I use append in the types, but for \ili{English} verbs, we can
  assume that there are no
  subjectless predicates.\\
  Stefan: But this assumption is not in the theory. Hence you have to use append to reflect what you
  stated in the types, don't you?\\
  AA arg-st added: the types are more general and the words are more specified.\\
  Stefan: Yes, but then you have to explain it. Say something about additional constraints and where
they come from.}

\itd{Stefan: Doesn't \emph{seem} take an optional PP? It also may add an argument to the
  \type{seem-rel}. I could refer to the lexical item from the binding chapter where I use
  \emph{seem} with PP.}
\eas
\type{seem}:\\
\avm{
[\type*{subj-pred-v-word}
subj  & < \1 > \\
comps & < \2 [head & [prd & $+$] \\
	      subj & < \1 > \\
	      cont & [ind & \3]] > \\
arg-st & <\1, \2> \\
cont &	[ind & s \\
	 rels & \{[\type*{seem-rel} 
	           soa & \3]\} ] ]
}
\zs

\eas
\type{consider}:\\
\avm{
[\type*{obj-pred-v-word}
subj & < \1 NP$_{i}$ > \\
comps & < \2, \3 [head &	[prd & $+$] \\
	          subj & < \2 > \\
		  cont & [ind & \4] ]> \\
arg-st & < \1, \2, \3 > \\
cont   & [ind & s \\
	  rels & \{[\type*{consider-rel} 
	            exp & i \\
		    soa & \4]\} ] ]
}
\zs

	
The subject of \word{seem} is unspecified: it can be any category selected by the predicative
complement; the same holds for the first complement of \word{consider}:\itd{Stefan: Say something
  about the fact that it has to be exactly one lement. Where does this information come from?}
it can be any category
selected by the predicative complement (see examples in (\ref{storm}) above).  \word{Consider} selects
a subject and two complements, but only takes two semantic arguments: one corresponding to its
subject, and one corresponding to its predicative complement. It does not assign a semantic role to
its non"=predicative complement.

Let us take the example \textit{Paul seems happy}. As a predicative adjective, \word{happy} has a
\headf [\prd $+$] and its \subjf is not the empty list: it subcategorizes for a nominal subject and
assigns a semantic role to it, as shown in (\ref{happy2}).
	
\eas
\label{happy2}
\type{happy}:\\
\avm{
[\phon <happy> \\
head &	[\type*{adj}
	 	prd & $+$] \\
subj & <NP$_{i}$> \\
comps & < > \\
cont & [ind & s \\
rels &	\{[\type*{happy-rel}
		exp & i]\} ] ]
}
\zs

In the trees in the Figures~\ref{fig-happy} and~\ref{fig-cons}, the \subjf of \word{happy} is
shared with the \subjf of \word{seem} and the first element of the \comps list of
\word{consider}.\footnote{In what follows, we ignore adjectives taking complements. As noted in section 1 ref\, adjectives may take a non"=finite VP complement and fall under a control or raising type: as a subject-raising adjective, \word{likely} shares the \textsc{synsem} value of its subject with the expected subject of its VP complement; as a subject-control adjective, \word{eager} coindexes both subjects.
Such adjectives thus inherit from subj-rsg"=lexeme and subj-control"=lexeme type, respectively, as well as from adjective"=lexeme type. In some languages, copular constructions are complex predicates, which means that the copular verb inherits the complements of the adjective as well, see \citew{AG2001b-u}.}
\inlinetodostefan{Stefan: Is \ibox{1} a list or an element of a list? If the complete
\subjv is supposed to be raised, my fix is technically not correct. AA seems ok to me, the numbers are synsem descriptions, these figures are similar to the previous ones. Stefan: In the type
constraints you are sharing the whole SUBJ list. Here you are sharing elements. I think it should
look like in my Figure~\ref{fig-happy-fixed}.\\
AA: I prefer my figure 7 since it is the same as figure 2.\\
Stefan: But this is not a valid reason, if it is wrong.}


\begin{figure}
% \begin{tikzpicture}[baseline, sibling distance=2pt, level distance=60pt, scale=.9]
% 	\Tree
% 	[.{\begin{avm}
% 		\[phon & \phonliste{ Paul seems happy }\\
% 			subj & \eliste \\
% 			comps & \eliste\]
% 		\end{avm}}
% 		{\begin{avm} \[phon & \phonliste{ Paul } \\
% 			synsem & \@1 \]
% 		\end{avm}}
% 		[.{\begin{avm}
% 			\[phon & \phonliste{ seems happy }\\
% 			subj & \<\@1\>\]
% 			\end{avm}}
% 		 {\begin{avm}
% 			\[phon & \phonliste{ seems } \\
% 			subj & \<\@1\>\\
% 			comps & \<\@2 \[subj & \<\@1\>\]\>\\
% 			\]
% 			\end{avm}} 
% 		{\begin{avm}
% 			\[phon & \phonliste{ happy }\\
% 				synsem & \@2  \]	
% 			\end{avm}}  
% 		]
% 	]
% \end{tikzpicture}
\begin{forest}
[
\avm{
[S\\
\phon <Paul seems happy> \\
subj & < > \\
comps & < > ]
}
	[
	\avm{
	[NP\\
	\phon <Paul> \\
	synsem & \1 ]
	}
	]
	[
	\avm{
	[VP\\
	\phon <seems happy> \\
	subj & <\1> \\
	comps & < > ]
	}
		[
		\avm{
        [V\\
        \phon <seems> \\
		subj & <\1> \\
		comps & <\2	[subj & <\1>] > ]
		}
		]
		[
		\avm{
        [AP\\
        \phon <happy> \\
		synsem & \2 ]	
		}
		]
	]
]
\end{forest}
\caption{\label{fig-happy}A sentence with an intransitive copular verb}
\end{figure}

\begin{figure}
\begin{forest}
[
\avm{
[S\\
\phon <Paul seems happy> \\
      subj & < > \\
      comps & < > ]
}
	[
	\avm{
	[NP\\
	\phon <Paul> \\
       synsem & \1 ]
	}
	]
	[
	\avm{
	[VP\\
	\phon <seems happy> \\
         subj  & \2 <\1> \\
         comps & < > ]
	}
		[
		\avm{
		[V\\
		\phon <seems>  \\
		subj  & \2 \\
		comps & <\3	[subj & \2] > ]
		}
		] 
		[
		\avm{
		[AP\\
		\phon <happy> \\
		synsem & \3 ]	
		}
		]
	]
]
\end{forest}
\caption{\label{fig-happy-fixed}A sentence with an intransitive copular verb}
\end{figure}



\begin{figure}
% \begin{tikzpicture}[baseline, sibling distance=2pt, level distance=60pt, scale=.9]
% 	\Tree
% 	[.{\begin{avm}
% \[phon & \phonliste{ Mary considers Paul happy }\\
% subj & \eliste\\
% comps & \eliste\]		
% \end{avm}}
% 	{\begin{avm}\[phon & \phonliste{ Mary }\\
% 			synsem & \@3 \]
% 		\end{avm}}
% 	[.{\begin{avm}
% \[phon & \phonliste{ considers Paul happy }\\
% subj & \<\@3 NP\>\\
% comps & \eliste\]		
% \end{avm}}
% 	{\begin{avm}
% \[phon & \phonliste{ considers } \\
% subj & \<\@3 NP\>\\
% comps & \<\@1, \@2 \[
% 		 subj & \<\@1\> \]\>\]		
% \end{avm}}
% 	{\begin{avm} \[phon & \phonliste{ Paul } \\
% 			synsem & \@1 \]
% 		\end{avm}}
% 	{\begin{avm}
% 			\[phon & \phonliste{ happy }\\
% 				synsem & \@2 \]	
% 			\end{avm}}
% 	] ]
% \end{tikzpicture}
\begin{forest}
[
\avm{
[S\\
\phon <Mary considers Paul happy> \\
      subj & < > \\
      comps & < > ]		
}
	[
	\avm{
	[NP\\
	\phon <Mary> \\
	synsem & \1 ]
	}
	]
	[
	\avm{
	[VP\\
	\phon <considers Paul happy> \\
	subj & <\1> \\
	comps & < > ]
	}
		[
		\avm{
		[V\\
		\phon <considers> \\
		subj  & <\1> \\
		comps & <\2, \3[subj & <\2> ] > ]		
		}
		]
		[
		\avm{
		[NP\\
		\phon <Paul> \\
			synsem & \2 ]
		}
		]
		[
		\avm{
		[AP\\
		\phon <happy> \\
		synsem & \3 ]	
		}
		]
	]
]
\end{forest}	
\caption{\label{fig-cons}A sentence with a transitive copular verb}
\end{figure}

\citet{PollardandSag1994}\addpages mention a few verbs taking a predicative complement which can be
considered as control verbs. A verb like \word{feel} selects a nominal subject and assigns a
semantic role to it.

\begin{exe}
\ex John feels tired / in a good mood.
\end{exe}

\noindent
It inherits from the subject-control-verb type (\ref{cont}); its lexical description is given in (\mex{1}):

\eas
\type{feel}:\\
\avm{
	[subj & <\1 NP$_{\tag{i}}$ > \\
	comps & <\2	[head &	[prd & $+$] \\
				subj &	<[ind & \tag{i}]> \\
				cont &	[ind & \3] ]>\\
	arg-st & <\1, \2> \\
	cont &	[ind & s \\
			rels &	\{[\type*{feel-rel}
			exp & \tag{i} \\
			soa & \3]\} ] ]
}
\zs


\subsection{Copular verbs in Mauritian}

As shown by \citet{HenriandLaurens2011}, and as noted earlier, \ili{Mauritian} data
argue\itd{Stefan: Data argue?} in favor
of a non"=clausal analysis.\itd{Stefan: Where exactly did you show this? What you suggested in
  (\ref{ex-pans-Maritian}) was an embedded clause.} A copular verb takes a short form before an
attributive\itd{Stefan: predicative complement?} complement, and
a long form before a clausal one.\itd{Stefan: reference examples in (\mex{1}).} Despite the lack of inflection on the embedded verb, and the
possibility of subject prodrop, clausal complements differ from non"=clausal complements by the
following properties: they do not trigger the matrix verb short form (SF), they may be introduced by
the complementizer \word{ki}, their subject is a weak pronoun (\word{mo} `I', \word{to} `you'). On
the other hand, a VP or AP complement cannot be introduced by \word{ki}, and an NP complement must
be realized as a strong pronoun (\word{mwa} `me', \word{twa} `you'). See Section~\ref{sec-maurit}
above for the alternation between verb short form (\textsc{sf}) and long form (\textsc{lf}).

\eal
\ex 
\gll Mari ti res  malad.\\
     Mari \textsc{pst} remain.\textsc{sf} sick\\\hfill\citep[\page 198]{HenriandLaurens2011}
\glt `Mari remained sick.'

\ex 
\gll Mari trouv  so mama malad\\
     Mari find.\textsc{sf} \POSS{} mother sick\\
\glt `Mari finds her mother sick.'

\ex 
\gll Mari trouve (ki) mo malad\\
     Mari find.\textsc{lf} \hphantom{(}that 1\SG.\textsc{wk} sick\\
\glt `Mari finds that I am sick.'

\ex 
\gll Mari trouv            ki   mwa               malad\\
     Mari find.\textsc{sf} that 1\SG.\textsc{str} sick\\
\glt `Mari finds me sick.'
\zl

\itd{Stefan: The typo is in the original, but ``clause'' is singular and ``small clauses'' is plural.}
\citet[\page 218]{HenriandLaurens2011} conclude that ``Complements of raising and control verbs
systematically pattern with non-clausal phrases such as NPs or PPs. This kind of evidence is seldom
available in world's languages because heads are not usually sensitive to the properties of their
complements. The analysis as clause or small clauses is also problematic because of the existence of
genuine verbless clauses in \ili{Mauritian} which pattern with verbal clauses and not with
complements of raising and control verbs.''




\section{Auxiliaries as raising verbs}
\label{sec-auxiliaries-as-raising-verbs}

Following \citep{Ross69a-u,Gazdaretal1982, Sagetal2020}, \word{be}, \word{do}, \word{have}, and
modals (e.g., \word{can}, \word{should}) in HPSG are not considered a special part of speech
(\type{Aux} or \type{Infl}) but verbs with the head property in
(\ref{ex-head-value-of-aux-elements}):

\begin{exe}
\ex \label{ex-head-value-of-aux-elements}
  \type{auxiliary-verb} \impl
\avm{
	[head &	[aux & $+$] ]
}
 \end{exe}
 
 \ili{English} auxiliaries take VP (or XP) complements and do not select their subject,\itd{Stefan:
   Well, they do. After all the subject is raised and then the auxiliary selects it. It does not
   assign it a semantic role.} just like
 subject-raising verbs. They are thus compatible with non"=referential subjects, such as
 meteorological \word{it} and existential \textit{there}. They select the verb form of their
 non"=finite complements: \textit{have} selects a past participle, \textit{be} a gerund-participle,
 \textit{can} and \textit{will} a bare form.

	
\begin{exe}
\ex \begin{xlist}
\ex Paul has left.
\ex Paul is leaving.
\ex Paul can leave.
\ex It will rain.
\ex There can be a riot.
\end{xlist}	
\end{exe}

In this approach, \ili{English} auxiliaries are subtypes of subject-raising-verbs, and thus take a VP (or
XP) complement and share their subject with the unexpressed subject of the non"=finite verb.\footnote{ \emph{Be} is an auxiliary and a subj-raising verb with a \prd$+$ complement, see
  Section~\ref{control-sec-copula-verbs} above, or a gerund-participle VP complement,
  different from the identity \emph{be} which is not a raising verb (see \citew{VanEynde2008a} and
  \citew{MuellerPredication} on predication). A verb like \emph{dare}, shown to be an auxiliary by
  its postnominal negation, is not a raising verb but a subject-control verb:
\itd{Stefan: Why is this important?}
\eal
\ex[]{
He is lazy and sleeping.
}
\ex[]{
I dare not be late.
}
\ex[\#]{
It will not dare rain.
}
\zllast
}
The lexical descriptions for the auxiliaries \word{will} and \word{have} look as follows: 

\ea
\type{will}:\\*
\avm{
[head &	[aux &  $+$] \\
 subj & < \1 > \\
 comps & < \2 VP[head & [vform & bse] \\
                 subj & < \1 > \\
	         cont & [ind & \3] ]> \\
	arg-st & < \1, \2 >\\
	cont &	[ind & s \\
		 rels & \{[\type*{future-rel}
		           soa & \3]\} ] ]
}
\z
\eas
\type{have}:\\*
\avm{
	[head &	[aux & $+$] \\
	subj & <\1> \\
	comps & <\2 VP	[head &	[vform & past-part] \\
					subj & <\1> \\
					cont &	[ind & \3] ]>\\
	arg-st & <\1, \2>\\
	cont &	[ind & s \\
			rels & \{[\type*{perfect-rel}
			soa & \3]\} ] ]
}
\zs

To account for their NICE (\isi{negation}, \isi{inversion}, \isi{contraction}, \isi{ellipsis}) properties, \citet{KS2002a} % ref to whole paper OK
use a binary head feature \aux, so that only [\aux $+$] verbs may allow for subject inversion
(\ref{inv}), sentential negation (\ref{neg}), contraction or VP ellipsis (\ref{ell}). See
\crossrefchapterw[Section~\ref{sec-head-movement-vs-flat}]{order} on subject inversion,
\crossrefchapterw[Section~\ref{sec-sentential-negation}]{negation} on negation and
\crossrefchapterw[Section~\ref{sec-analyses-of-pred-ellipsis}]{ellipsis} on post-auxiliary ellipsis.\footnote{Copular \word{be} has
  the NICE properties (\textit{Is John happy?}), it is an auxiliary verb with [\prd $+$]
  complement. Since \emph{to} allows for VP ellipsis, it is also analysed as an auxiliary verb:
  \emph{John promised to work and he started to}. See \citew*{GPS82a-u}.\itd{add pages}} 

\eal
\ex[]{
Is Paul working? \label{inv}
}
\ex[*]{
Keeps Paul working?
}
\ex[]{
Paul is (probably) not working.\label{neg}
}
\ex[*]{
Paul keeps (probably) not working.
}
\ex[]{
John promised to come and he will. \label{ell}
}
\ex[*]{
John promised to come and he seems.
}
\zl

\noindent
Subject raising verbs such as \word{seem}, \word{keep} or \word{start} are [\aux $-$].

\citet{Sagetal2020} revised this analysis and proposed a new analysis couched in \sbcg (\citealp{Sag2012a}; see also \crossrefchapteralp[Section~\ref{sec-sbcg}]{cxg}). The descriptions used below were translated into the feature geometry of Constructional HPSG \citep{Sag97a}, which is used in this volume. In their approach, the head feature \aux is both lexical and constructional: the constructions restricted to auxiliaries require their head to be [\aux $+$], while the constructions available for all verbs are [\aux $-$]. In this approach, non"=auxiliary verbs are lexically specified as [\aux $-$]:

\begin{exe}
\ex \type{non-auxiliary-verb} \impl
\avm{
	[head &	[aux & $-$ \\
			inv & $-$ ] ]
}
\end{exe}

 Auxiliary verbs, on the other hand are unspecified for the feature \aux, and are contextually specified; except for unstressed \word{do}  which is [\aux $+$] and must occur in constructions restricted to auxiliaries.

\eal
\settowidth\jamwidth{(tritra trulla la. Nobody will ever read this.)}
\ex[]{
Paul is working. \jambox{[\aux $-$]}
}
\ex[]{
Is Paul working? \jambox{[\aux $+$]}
} \label{inv1}
\ex[*]{
John does work. \jambox{[\aux $-$]}
}
\ex[]{
Does John work? \jambox{[\aux $+$]}
}\label{inv2}
\zl

Subject inversion is handled by a subtype of head-subject-complement phrase, which is independently
needed for verb initial languages like \ili{Welsh} \parencites{Borsley99c-u}{SWB2003a}.\footnote{As
  noted in \crossrefchapterw[\page \pageref{page-properties:aux-inversion}]{properties}, in some HPSG work, e.g. \citew[409--414]{SWB2003a},
  examples like (\ref{inv1}) and (\ref{inv2}) are analysed as involving an auxiliary verb with two
  complements and no subject. This approach has no need for an additional phrase type, but it
  requires an alternative valence description for auxiliary verbs.} It is a specific (non"=binary)
construction,\is{branching!non-binary} of which other constructions such as \type{polar-interrogative-clause} are subtypes,
and whose head must be [\textsc{inv} $+$].
\itd{Stefan: Shouldn't the \textsc{inv} constraint be in (\mex{1})?}
\ea
\type{initial-aux-cx} \impl
\avm{
[subj     & < > \\
 comps    & < > \\
 head-dtr & \1[aux & $+$ \\
	       subj & \2\\
               comps & \3 ] \\
 dtrs & < \1 > \+ \2 \+ \3 ]
}
\z
       
Most auxiliaries are lexically unspecified for the feature INV and allow for both constructions
(non"=inverted and inverted), while the 1st person \word{aren't} is obligatory inverted (lexically
marked as [\textsc{inv} $+$]) and the modal \word{better} obligatory non"=inverted (lexically marked
as [\textsc{inv} $-$]):

\eal
\ex[]{
Aren't I dreaming?
}
\ex[*]{
I aren't dreaming.
}
\ex[]{
We better be carefull.
}
\ex[*]{
Better we be carefull?
}
\zl

While the distinction is not always easy to make between VP ellipsis and null complement anaphora
(\textit{Paul tried}), \citeauthor{Sagetal2020} observe that certain elliptical constructions are
\itd{Stefan: Ellipsis chapter (p.\,\pageref{sec-analyses-of-noncon}) points here and does not have
  any analysis on pseudogapping.}
restricted to auxiliaries, for example pseudogapping (see also \crossrefchaptert{ellipsis} and \citealt{Miller2014a-u}).

\eal
\ex[]{
John can eat more pizza than Mary can tacos.
}
\ex[]{Larry might read the short story, but he won’t the play.
}
\ex[*]{
Ann seems to buy more bagels than Sue seems cupcakes.
}
\zl

This could be captured by having the relevant auxiliairies optionally inherit the complements of their verbal complement.\footnote{See \citew{KimandSag2002} for a comparison of \ili{French} and \ili{English} auxilaries, \citew{AG2002b-u} for a thorough analysis of \ili{French} auxiliaries as ``generalized'' raising verbs, inheriting not only the subject but also any complement from the past participle; such generalized raising was first suggested by \citet{HN89a,HN94a} for \ili{German} and has been adopted since in various analyses of verbal complexes in \ili{German} \citep{Kiss95a,Meurers2000b,Kathol2001a,Mueller99a,Mueller2002b}, \ili{Dutch} \citep{BvN98a} and \ili{Persian} \citep[Section~4]{MuellerPersian}. See also \crossrefchaptert{complex-predicates}.}
A revised version of \emph{will} with complement inheritance could be the following:
\inlinetodoobl{Anne: The VP should be typed \type{pro}. Stefan: I do not know how to do this. I did it like JBK, hope this is correct. But you should explain this somewhere.}
\ea
\emph{will} (pseudogapping):\\
\avm{
[ arg-st < \1, VP[\type{pro}, \subj < \1 >, comps \2 ] > \+ \2 ]
}
\z
As observed by \citet{ArnoldandBorsley2008}, auxiliaries can be stranded in certain non-restrictive
relative clauses such as (\ref{aux1}), no such possibility is open to non-auxiliary verbs
(\ref{nonaux}) (see also \crossrefchapteralt[\page \pageref{page-relative-clauses:stranded-aux}]{relative-clauses}):

\eal
\ex[]{
Kim was singing, which Lee wasn't. \label{aux1}
}
\ex[*]{
Kim tried to impress Lee, which Sandy didn't try. \label{nonaux}
}
\zl

The HPSG analysis sketched here captures a very wide range of facts, and expresses both generalizations (\ili{English} auxiliaries are subtypes of subject-raising verbs) and lexical idiosyncrasies (copula \emph{be} takes non"=verbal complements, 1st person \emph{aren't} triggers obligatory inversion etc).


	
\section{Conclusion}

Complements of ``raising'' and control verbs have been either analyzed as clauses \citep{Chomsky81a}\addpages
or small clauses \citep{Stowell81a-u,Stowell1983}\addpages in Mainstream Generative Grammar.  As in LFG
\citep{Bresnan1982}, ``raising'' and control predicates are analysed as taking non-clausal open
complements in HPSG \citep{PollardandSag1994}\addpages, with sharing or coindexing the (unexpressed) subject
of the embedded predicate with their own subject (or object). This leads to a more accurate analysis
of ``object"=raising'' verbs as ditransitive, without the need for an exceptional case marking
device. This analysis naturally extends to pro-drop and ergative languages; it also makes correct
empirical predictions for languages marking clausal complementation differently from VP
complementation. A rich hierarchy of lexical types enables verbs and adjectives taking non"=finite
or predicative complements to inherit from a raising type or a control type. The Raising Principle
prevents any other kind of non"=canonical linking between semantic argument and syntactic
argument. A semantics-based control theory predicts which predicates are subject-control and which
object-control. The ``subject-raising'' analysis has been successfully extended to copular and
auxiliary verbs, without the need for an Infl category.




\section*{Abbreviations}

\begin{tabularx}{.45\textwidth}{@{}lX}
\textsc{av} & Agentive Voice\\
\textsc{lf} & long form\\ 
\textsc{ov} & Objective Voice\\
\textsc{sf} & short form\\
\textsc{str} & strong\\
\textsc{wk} & weak\\

\end{tabularx}

\section*{Acknowledgements}

\itd{Name the coeditors?}
I am grateful to the reviewers and the coeditors of the volume for their helpful comments.
{\sloppy
\printbibliography[heading=subbibliography,notkeyword=this] 
}

\end{document}
%      <!-- Local IspellDict: en_US-w_accents -->
