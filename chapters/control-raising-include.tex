%% -*- coding:utf-8 -*-



\begin{document}
\maketitle
\label{chap-control-raising}

%\itdobl{Stefan: Abstracts should be standalone pieces of text that can go to databases without the references. So references should not be included here.\\
%Stefan: Please change all occurances of academic ``we'' into ``I'' or passive. I would remove the quotes around raising and control. I know that you hate Minimalist terminology, but in HPSG we also raise things. We just raise on ARG-ST or COMPS. I think it is ok to use raise and control without quotes. Anne: it is not a matter of sentiment, reusing the movement metaphor does not give credit to the HPSG analysis}

%\section{Introduction}
%\label{control-sec-intro}

\section{The distinction between raising and control predicates}
\label{sec-distinction-raising-control}

\subsection{The main distinction between raising and control verbs}

In a broad sense, \emph{control} refers to a relation of referential dependence between an unexpressed
subject (the controlled element) and an expressed or unexpressed constituent (the controller); the
referential properties of the controlled element, including possibly the property of having no
reference at all,
%\itdobl{But if the controlled element does not have a reference, it is an expletive and then we would have raising, wouldn't we? I think, what you have here is Bresnan's definition, but HPSG is
 % not using this.\\ Anne: I added this on Bob's request, that is control "in a broad sense", before going into the distinction %between 'control' and 'raising'}
are determined by those of the controller \citep[372]{Bresnan1982}. Verbs taking
non"=finite complements usually determine the interpretation of the unexpressed subject of the
non"=finite verb. With \word{want}, the subject is understood as the subject of the infinitive,
while with \word{persuade} it is the object, as shown by the reflexives in (\ref{ex-equi}). They are
called ``control verbs'', and \emph{John} is called the ``controller'' in (\ref{equi1}) while
\emph{Mary} is the controller in (\ref{equi2}).  

\eal
\label{ex-equi}
\ex John wants to buy himself a coat. \label{equi1}
\ex John persuaded Mary to buy herself / * himself a coat.\label{equi2}
\zl

Another type of verb also takes a non-finite complement and identifies its subject (or its object)
with the unexpressed subject of the non-finite verb. Since \citet{Postal1974}, they have been called
``raising verbs''. What semantic role the missing subject has, if any, is determined by the lower
verb, or if that is a raising verb, an even lower verb. In (\ref{ex-john-seemed-to-like-himself}) the subject of the
infinitive (\emph{like}) is understood to be the subject of \word{seem}, while in (\ref{exp})
the subject of the non-finite verb (\emph{buy}) is understood to be the object of
\word{expect}. Verbs like \word{seem} are called ``subject-to-subject-raising verbs'' (or
``subject-raising verbs''), while verbs like \word{expect} are called ``subject-to-object-raising
verbs'' (or ``object-raising verbs''). 

\eal
\ex John seemed to like himself.\label{ex-john-seemed-to-like-himself}
\ex John expected Mary to buy herself / * himself a coat. \label{exp}
\zl
 
Raising and control constructions differ from other constructions in which the missing subject
remains vague (\ref{arbitrary}) and which are a case of ``arbitrary'' or ``anaphoric'' control
\parencites[\page 75--76]{Chomsky1981}[\page 379]{Bresnan1982}.\footnote{\citet{Bresnan1982} proposes
  a non-transformational analysis and renames ``raising'' to ``functional control'' and ``control''
  (obligatory) to ``anaphoric control''. See also \crossrefchapterw[Section~\ref{lfg:sec-raising-control}]{lfg}.} 
 
\ea
Buying a coat can be expensive.\label{arbitrary}
\z
  
A number of syntactic and semantic properties distinguish control verbs like \emph{want}, \emph{hope},
\emph{force}, \emph{persuade}, \emph{promise} from raising verbs like \emph{see},
\emph{seem}, \emph{start}, \emph{believe}, \emph{expect} \citep{Rosenbaum67a-u,Postal1974,Bresnan1982}.\footnote{%
  The same distinction is available for verbs taking a gerund-participle complement: 
  \emph{Kim remembered seeing Lee.} (control) vs.\  \emph{It started raining.} (raising).}
% \itdobl{The phase verbs are coming in two versions: a control and a raising version \citep{Perlmutter70}. Maybe use
% \emph{It started raining.} since this would make it clear that the verb is a raising verb.\\
% Anne: I'm not saying phase verbs come in two versions, I just analyse them here as raising verbs; i
% changed the ex to it...rain\\
% Stefan: I would just remove \emph{start} here. But \eph{start} is used later. So maybe leave it at
% this.}

The key point is that there is a semantic role associated with the subject of verbs like
\emph{want} but not of verbs like \emph{seem} and with the first complement of verbs like
\emph{persuade} but not of verbs like \emph{expect}.  The consequence is that more or less any NP
is possible as subject of \emph{seem} and as the first NP after \emph{expect}. This includes
expletive \emph{it} and \emph{there} and non-referential parts of idioms.

Let us first consider expletive subjects: meteorological \word{it} is selected by predicates
such as \word{rain}. It can be the subject of \word{start}, \word{seem}, but not of \word{hope},
\word{want}. It can be the object of \word{expect}, \word{believe} but not of \word{force},
\word{persuade}:
	
\eal
\ex[]{
It rained this morning.
}
\ex[]{
It seems/started to rain this morning.
} \label{rain1}
\ex[]{
We expect it to rain tomorrow. 
} \label{rain2}
\zl
\eal
\ex[\#]{
It wants/hopes to rain tomorrow.
} \label{rain3}
\ex[\#]{
The sorcerer persuaded it to rain.
} \label{rain4}
\zl
 	
The same contrast holds with an idiomatic subject such as \word{the cat} in the expression \word{the
cat is out of the bag} `the secret is out'. It can be the subject of \word{seem} or the object of
\word{expect}, with its idiomatic meaning. If it is the subject of \word{want} or the object of
\word{persuade}, the idiomatic meaning is lost and only the literal meaning remains. 
 
\eal
\judgewidth{\#}
\ex[]{
The cat is out of the bag.
}
\ex[]{
The cat seems to be out of the bag. 
} \label{cat1}
\ex[]{
We expected the cat to be out of the bag. 
} \label{cat2}
\ex[\#]{
The cat wants to be out of the bag.\hfill(non"=idiomatic)
} \label{cat3}
\ex[\#]{
We persuaded the cat to be out of the bag.\hfill(non"=idiomatic)
} \label{cat4}
\zl

Let us now look at non-nominal subjects: \emph{be obvious} allows for a sentential subject (\ref{ex-that-Kim-is-a-spy-seemed-to-be-obvious}) and
\emph{be a good place to hide} allows for a prepositional subject (\ref{under}). They are possible with raising
verbs, as in the following: 
 
\eal
\ex[]{
[That Kim is a spy] is obvious.
} 
\ex[]{
[That Kim is a spy] seemed to be obvious.
} \label{ex-that-Kim-is-a-spy-seemed-to-be-obvious}
\zl
\eal
\ex[]{
[Under the bed] is a good place to hide.}
\ex[]{
Kim expects [under the bed] to be a good place to hide.
} \label{under}
\zl


\noindent
But they would not be possible with control verbs:
\eal
\ex[\#]{
[That Kim is a spy] wanted to be obvious.
}
\ex[\#]{
Kim persuaded [under the bed] to be a good place to hide.
}
\zl

In languages such as \ili{German} which allow them, subjectless constructions can be embedded under
raising verbs but not under control verbs \citep[\page 48]{Mueller2002b}; subjectless passive
\emph{gearbeitet} `worked' can thus appear under \emph{scheinen} `seem' but not under
\emph{versuchen} `try': 

\eal
\label{german1}\label{ex-raising-with-subjectless-verbs}
\ex[]{ 
\gll weil gearbeitet wurde\\
     because worked was\\\hfill(\ili{German})
\glt `because work was being done'
}
\ex[]{ 
\gll Dort schien noch gearbeitet zu werden.\\
     there seemed yet worked to be\\
\glt `Work seemed to still be being done there.’
}
\ex[*]{
\gll Der Student versucht, gearbeitet zu werden.\\
     the student tries worked to be\\
\glt Intended: `The student tries to get the work done.'
}
\zl
 
\noindent
All this shows that the kind of subject (or object) that a raising verb may take depends only on the embedded
non"=finite verb.

Let us now look at possible paraphrases: when control and raising sentences have a corresponding sentence with a finite clause complement, they have rather different related sentences.
%It seemed [that Kim impressed Sandy]
%Kim hoped [that he would go home]
%Kim expected [that Sandy would go home]
%Kim persuaded Sandy [that he/she should home]
With control verbs, the non"=finite complement may often be replaced by a sentential complement (with its own subject), while this is not possible with raising verbs:

\eal
\ex[]{
Bill hoped [to impress Sandy] / [that he impressed Sandy].\label{hope1}
}
\ex[]{
Bill seemed [to impress Sandy] / *[that he impressed Sandy].
}
\zl

\eal
\ex[]{
Bill promised Sandy [to come] / [that he would come]. \label{promise1}
}
\ex[]{
Bill expected Sandy [to come] / *[that she would come].
}
\zl
% \itdobl{Stefan: The difference between she/he is confusing. The examples are not minimal pairs. I
%   think having ``he'' in (\mex{0}b) would make the point as well.\\Anne: But expect is object
%   control.\\Stefan: Good. Maybe replace Sandy by a uniquely female propername as well.}

With some raising verbs, on the other hand, a sentential complement is possible with an expletive
subject (\ref{seem4}) or with no postverbal object (\ref{expect3}). 

\eal
\ex[]{
It seemed [that Kim impressed Sandy]. \label{seem4}
}
\ex[]{
Kim expected [that Sandy would come].\label{expect3}
}
\zl

\noindent
This shows that the control verbs can have a subject (or an object) different from the subject of
the embedded verb, but that the raising verbs cannot.\footnote{%
  Another contrast proposed by
 \citew[\page 444]{Jacobson1990} is that control verbs may allow for a null complement (\emph{She tried.}) or
  a non"=verbal complement (\emph{They wanted a raise.}), while raising verbs may not (\emph{*She
    seemed.}). However, some raising verbs may have a null complement (\emph{It just
    started} (\emph{to rain}).) as well as some auxiliaries (\emph{She doesn't.}) which can be analyzed as raising verbs (see
  Section~\ref{sec-auxiliaries-as-raising-verbs} below).%
} 

\subsection{More on control verbs}

For control verbs, the choice of the controller is determined by the semantic class of the verb
(\citealt[Chapter~3]{PollardandSag1994} and also \citealt{JackendoffandCulicover2003}).  Verbs of
influence (\word{permit}, \word{forbid}) are cases of object control while verbs of commitment
(\word{promise}, \word{try}) as in (\ref{commit}) and orientation (\word{want}, \word{hate}) as in
(\ref{orient}) display subject control, as shown by the reflexive in the following examples:%
%\itd{Stefan: What about: ``John promised his son to go to the movies together.''}
\footnote{Some verbs may be ambiguous and allow for subject control (\emph{John proposed to come
    later.}), object control (\emph{John proposed to Mary to wash herself.}), and joint
  control (\emph{John proposed to Mary to go to the movies together.}). For the joint control case, a cumulative
  (i+j) index is needed, as is also the case with long distance dependencies; see
  \citew[Chapter~3]{CP2020a-u} and \crossrefchapterw[\pageref{ex:UDC:31}]{udc}:
\ea
Setting aside illegal poaching for a moment, how many sharks$_{i+j}$ do you estimate [[$_i$ died
naturally] and [$_j$ were killed recreationally]]?
\zlast
}
\eal
\ex\label{ex-John-promised-Mary-to-buy}
John promised Mary to buy himself / * herself a coat. \label{commit}
\ex\label{ex-John-permitted-Mary-to-buy} 
John permitted Mary to buy herself / * himself a coat.\label{orient}
\zl
 
  This classification of control verbs is cross-linguistically widespread \citep{VanValinandLapolla1997}, but \ili{Romance} verbs of mental representation and speech report are an exception in being subject-control without having a commitment or an orientation component.


\begin{exe}
\ex \begin{xlist}
\ex 
\gll Marie dit {ne pas} \^etre convaincue.\\
     Marie says \ig{neg} be convinced \\\hfill(\ili{French})
\glt `Marie says she is not convinced.'	
\ex 
\gll Paul pensait  avoir compris. \\
     Paul thought have understood \\
\glt `Paul thought he understood.'
 \end{xlist}
\end{exe}

It is worth noting that for object-control verbs, the controller may also be the complement of a preposition \citep[\page 139]{PollardandSag1994}:

\begin{exe}
\ex Kim appealed [to Sandy] to cooperate. \label{to}
\end{exe}


%\eal
%\ex[]{
%Leslie wants this / a raise.
%}
%\ex[]{
%Leslie tried.
%}
%\ex[*]{
%Leslie seemed.
%}
%\ex[*]{
%Leslie seemed this.
%}
%\zl
 
\citet[\page 401]{Bresnan1982}, who attributes the generalization to Visser, also suggests that
object-control verbs may passivize (and become subject-control) while subject"=control verbs cannot
(with a verbal complement). However, there are counterexamples like (\ref{promise-pass2}) adapted
from \citet[\page 255]{JB1976a-u}, and the generalization does not seem to hold crosslinguistically
(see \citealt[\page 129]{Mueller2002b} for counterexamples in \ili{German}).

\eal
\ex[]{
Mary was persuaded to leave (by John).
}\label{persuade-pass}
\ex[*]{
Mary was promised to leave (by John).
}\label{promise-pass}
\ex[]{Pat was promised to be allowed to leave. 
}\label{promise-pass2}
\zl

 
\subsection{More on raising verbs}
\label{sec-more-on-raising-verbs}

From a cross-linguistic point of view, raising verbs usually belong to other semantic classes than
control verbs. The distinction between subject-raising and object-raising also has some semantic
basis: verbs marking tense, aspect, modality (\word{start}, \word{cease}, \word{keep}) are
subject-raising, while causative and perception verbs (\word{let}, \word{see}) are usually
object-raising:

\eal
\ex John started to like himself.
\ex It started to rain.
\ex John let it appear that he was tired.
\ex John let Mary buy herself / * himself a coat.
\zl
	

Transformational analyses posit distinct syntactic structures for raising and control sentences:
subject-raising verbs select a sentential complement (and no subject), while subject-control verbs
select a subject and a sentential complement \parencites[\iaddpages]{Postal1974}[\iaddpages]{Chomsky81a}.\footnote{We disregard here the Movement Theory of Control \citep{Hornstein99a-u}; see \citew{Landau2000a-u} for criticism.}
With subject-raising
verbs, the embedded clause's subject is supposed to move to the position of matrix verb subject,
hence the  term ``raising''. Transformational analyses also posit two distinct structures for object-control and
object-raising verbs: while object-control verbs select two complements, object-raising verbs only
select a sentential complement, and an exceptional case marking (ECM) rule assigns case to the
embedded clause's subject.
% of \word{expect} verbs).
% a co-indexing rule between the empty subject (PRO) of the infinitive and the subject (\word{promise}) or the object (\word{persuade}) of the matrix verb for control verbs.
In this approach, both subject- and object-raising verbs have a sentential complement:
	
% \itd{Stefan: Are there sources for these analyses? Anne: would be better with John barred instead of ei in S.\\
% Stefan: Like this? Sources are still missing. References. The e$_i$ notation is used later in the
% chapter. Maybe keep things uniform.\\ Anne: this is fine.\\
% Stefan: OK, but references are missing. Is this Chomsky's analysis?}
\eal
\ex subject-raising:\\
{}[\sub{NP} $e$ ] seems [\sub{S} John to leave] 
$\leadsto$  
{}[\sub{NP} John] seems [\sub{S} \st{John} to leave]
%\ex subject-control:  
%{}[\sub{NP} John$_{i}$ ] wants [\sub{S} PRO$_{i}$ to leave ]	
%\zl
%\eal
\ex object-raising (ECM):\\
We expected [\sub{S} John to leave] 	
%\ex object-control: We persuaded  
%{}[\sub{NP} John$_{i}$ ]  [\sub{S} PRO$_{i}$ to leave ]	
\zl

% \itd{Anne:should not be indented}
\noindent
However, the putative correspondence between source structure (before movement) and target structure (after movement) for raising verbs is not
systematic: \word{seem} may take a sentential complement (with an expletive subject) as in
(\ref{seem4}), but the other subject-raising verbs (aspectual and modal verbs) do not.

\eal
\ex[]{
Paul started to understand.
}
\ex[*]{
It started [that Paul understands].
}
\zl
 
%\itd{Anne:should not be indented}
 \noindent
 Similarly, while some object-raising verbs (\word{expect, see}) may take a sentential complement as in (\ref{expect3}), others do not (\word{let}, \word{make}, \word{prevent}).
 
%\itd{Stefan: Why do you have (\mex{1}) in addition to (\mex{2}) and no such example in addition to (\mex{0})?}
%\eal
%\ex[]{We expect Paul to understand.	\label{ex-we-expect-paul-to-unterstand}}
%\ex[]{We expect [that Paul understands]. \label{ex-we-expect-that-paul-understands}}
%\zl
\eal
\ex[]{
We let Paul sleep.
}
\ex[*]{
We let [that Paul sleeps].
}
\zl

Furthermore, in transformational analyses, it is often assumed that the subject of the non"=finite verb must raise to receive case from the matrix verb.
 But the subject of \word{seem} or \word{start} need not bear case, since it can be a non-nominal subject (\ref{under}).
%\eal
%\ex[]{[Drinking one liter of water each day] seems to benefit your health.}
%\zl
Data from languages with ``quirky'' case such as \ili{Icelandic} also show that subjects of
subject-raising verbs in fact keep the quirky case assigned by the embedded verb
\citep{Zaenenetal1985}\addpages, in contrast to the subject of subject-control verbs, which are assigned
case by the matrix verb and are thus in the nominative. A verb like \word{need} takes an accusative
subject, and a raising verb (\word{seem}) takes an accusative subject as well when combined with
\word{need} (\ref{need}). With a control verb (\word{hope}), on the other hand, the subject must be
nominative (\ref{hope-i}).\footnote{
The examples in (\mex{1}) are from  \citew*[\page 386--387]{SWB2003a}.
}

\eal
\ex 
\gll Hana             vantar peninga.\\
     she.\textsc{acc} lacks  money.\textsc{acc} \\\hfill(\ili{Icelandic})
\glt `She lacks money.'
\ex 
\gll Hana             virðist vanta   peninga.\label{need} \\
     she.\textsc{acc} seems   to.lack money.\textsc{acc} \\
\glt `She seems to lack money.'
\ex 
\gll Eg             vonast till ad vanta ekki peninga. \label{hope-i} \\
     I.\textsc{nom} hope   for  to lack  not  money.\textsc{acc} \\
\glt `I hope I won't lack money.'
\zl

Finally, the possibility of an intervening PP between the matrix verb and the non-finite verb should
block subject movement, according to Chain formation or \isi{Relativized Minimality}
\citep{Rizzi1986,Rizzi1990b-u}.
% whole paper/book
\eal
\ex Carol seems to Kim to be gifted.
\ex Carol$_i$ seems to herself$_i$ [e$_i$ to have been quite fortunate].\footnote{
From \citet[\page 50]{McGinnis2004a-u}.
}
\zl
Turning now to object-raising verbs, when a finite sentential complement is possible, the structure
is not the same as  with a non"=finite complement. Heavy NP shift is possible with a non"=finite
complement, but not with a sentential complement \parencites[\page 423]{Bresnan1982}[\page 113]{PollardandSag1994}: this shows that \word{expect} has two complements in (\ref{ex-we-expect-all-students-to-unterstand}) and only one in (\ref{ex-we-expect-that-all-understand}).

\eal
\ex[]{\label{ex-we-expect-all-students-to-unterstand}
We expected [all students] [to understand].
}
\ex[]{
We expected [to understand] [all those who attended the class]. \label{HNPS}
}
\ex[]{\label{ex-we-expect-that-all-understand} 
We expected [that [all those who attended the class] understand].
}
\ex[*]{
We expected [that understand [all those who attended the class]].
}
\zl

%Bob says it the lack of 'that' that makes it *: Fronting also shows that the NP VP sequence does not behave as a single constituent, unlike the finite complement:
%\eal
%\ex[]{
%That Paul understood, I did not expect.}
%\ex[*]{
%Paul to understand, I did not expect.}
%\zl


%\itd{Anne:should not be indented}
\noindent
This shows that object-raising verbs are better analyzed as ditransitive verbs. This analysis predicts that the subject of the non"=finite verb has all properties of an object of the matrix verb. It is an accusative in \ili{English} (\word{him, her}) (\ref{pro}) and it can passivize, like the object of an object-control verb (\ref{passive}).

\begin{exe}
\ex
\begin{xlist} \label{pro}
\ex We expect him to understand.
\ex  We persuaded him to work on this.
\end{xlist}
\ex \begin{xlist} \label{passive}
\ex  He was expected to understand.
\ex  He was persuaded to work on this.
\end{xlist}
	
\end{exe}


To conclude, the movement (raising) analysis of subject-raising verbs and the ECM analysis of object-raising verbs are motivated by the idea that an NP which receives a semantic role from a verb should be a syntactic argument of this verb.
%an NP which receives a semantic role from a verb should be a syntactic argument of this verb. 
But they lead to syntactic structures which are not motivated (assuming a systematic availability of a sentential complementation) and/or make wrong empirical predictions (that the postverbal sequence of ECM verb behaves as one constituent instead of two).
%There have been debates among minimalist grammarians as to unifying the analysis for raising and control verbs mention Hornstein analysis which proposes a movement analysis for both, and criticisms by Landau who proposes a (non movement) PRO analysis for both (?) Relevant data that we ignore here: joint or partial control: John wants to go to the movies together; "backward" control in certain lg (with a subjunctive complement) pro-i tries that John-i leaves
 
\subsection{Raising and control non-verbal predicates}\label{nonverbal}

Non-verbal predicates taking a non"=finite complement may also fall under the raising/control distinction.  Adjectives such as \word{likely} have raising properties: they neither select the category of their subject nor assign it a semantic role, in contrast to adjectives like \word{eager}. Meteorological \word{it} is thus compatible with \word{likely}, but not with \word{eager}. In the following examples, the subject of the adjective is the same as the subject of the copula (see Section~\ref{sec-copular-constructions} below).

\eal
\ex[]{
It is likely to rain.
}
\ex[]{
John is likely / eager to work here.
}
\ex[*]{
It is eager to rain.
}
\zl

The same contrast may be found with  nouns taking a non"=finite complement. Nouns such as \word{tendency} have raising properties: they neither select the category of their subject nor assign it a semantic role, in contrast to nouns like \word{desire}. Meteorological \word{it} is thus compatible with the former, but not with the latter. In the following examples, the subject of the predicative noun is the same as the subject of the light verb \emph{have}.


\eal
\ex[]{
John has a tendency to lie.
}
\ex[]{
John has a desire to win.
}
\ex[]{
It has a tendency / * desire to rain at this time of year.
}
\zl
%\itdopt{Stefan: There is also ``cert to win'' in \citew{MuellerPredication}.}

\section{An HPSG analysis}


In a nutshell, the HPSG analysis rests on a few leading ideas: non"=finite complements are
unsaturated VPs (a verb phrase with a non"=empty \subjl); a syntactic argument need not be assigned
a semantic role; control and raising verbs have the same syntactic arguments; raising verbs do not
assign a semantic role to the syntactic argument that functions as the subject of their non"=finite
complement. 
%We\itdobl{no we Anne: ok but I dont understand why} 
I continue to use the term \emph{raising}, but it is just a cover term, since no raising move
is taking place in HPSG analyses.
%\itdobl{Stefan: Well, there is raising: we raise arguments from an
%  embedded \comps/\argst to a higher one. Some authors even have a RAISED$+$/$-$ feature. Anne: I want to keep this, we agree that we disagree on this I think, I mean there is no movement in HPSG this is important for the naive reader} 

%As a result\itd{Stefan: of what?}, 
In HPSG terminology, raising means full identity of syntactic and semantic
information (\type{synsem}) \crossrefchapterp[\pageref{ex:prop22}--\pageref{ex:prop24}]{properties} with the unexpressed subject, while
control involves identity of semantic indices (discourse referents) between the controller and the
unexpressed subject. Co-indexing is compatible with the controller and the controlled subject not
bearing the same case (\ref{hope-i}) or belonging to different parts of speech (\ref{to}), as is the
case for pronouns and antecedents (see \crossrefchapteralp{binding} on Binding Theory). This would not be possible
with raising verbs, where there is full sharing of syntactic and semantic features between the
subject (or the object) of the matrix verb and the (expected) subject of the non"=finite verb. In
\ili{German}, the nominal complement of a raising verb like \emph{sehen} `see' must agree in case
with the subject of the infinitive, as shown by the adverbial phrase \emph{einer nach dem anderen} `one after the other' which
agrees in case with the unexpressed subject of the infinitive, but it can have a different case with
a control verb like \emph{erlauben} `allow', as the following examples from \citew[\page
47--48]{Mueller2002b} show: 


\eal
\label{german2}
\ex 
\gll Der Wächter  sah den Einbrecher     und seinen Helfer            einen       / *  einer nach dem anderen weglaufen\\
     the watchman saw the burglar.\ACC{} and his    accomplice.\ACC{} one.\ACC{} {} {} one.\NOM{} after the other run.away\\\hfill(\ili{German})
\glt `The watchman saw the burglar and his accomplice run away, one after the other.'
\ex
\gll Der Wächter erlaubte den Einbrechern, einer nach dem anderen wegzulaufen.\\
     the watchman  allowed the burglars.\DAT{} one.\NOM{} after the other away.to.run\\
\glt `The watchman allowed the burglars to run away, one after the other.'
\zl

%\itd{Stefan: I think it would be good to give the reader some guidance here. Say what subsections  you have and why they are here. I was a bit surprised to see the section on Mauritian since it  looks like a specific case study. But it has some overall relevance and this could be made clear here.}
%We\itdobl{No academic ``we''.} 
I will first present in more detail the HPSG analysis of raising and
control verbs, then provide creole data (from Mauritian) which support a phrasal analysis of their
complement, then discuss the implication of control/raising for pro-drop and ergative languages, to end up
with a revised HPSG analysis, based on sharing \xarg instead of \subj. 

\subsection{The HPSG analysis of ``raising'' verbs}
\label{control:sec-HPSG-anaylsis-of-raising}

Subject-raising-verbs (and object-raising verbs) can be defined as subtypes inheriting from
\type{verb"=lexeme} and \type{subject-raising"=lexeme} (or \type{object-raising"=lexeme})
types. Figure~\ref{raising:fig-verb-hier2} shows parts of a possible type hierarchy.
%
\begin{figure}
\begin{forest}
type hierarchy
[lexeme
  [part-of-speech,partition
     [verb-lx, name=A1 ] 
     [adj-lx
       [sr-a-lx,tier=sr-v-lx, edge to=srl
         [likely,instance]]]
     [noun-lx] 
     [\ldots]] 
  [arg-selection,partition 
     [intr-lx
      	[subj-rsg-lx,name=srl
      	  [sr-v-lx, name=B1,tier=sr-v-lx
            [seem,instance] ]]
        [\ldots] ]
     [tr-lx
       [obj-rsg-lx
         [or-v-lx, name=B2 
           [expect, instance]]
       [\ldots]	]
     [\ldots]]]]
\draw (A1.south)-- (B1.north);
\draw (B2) to [bend left= 6] (A1);
\end{forest}
\caption{\label{raising:fig-verb-hier2}A type hierarchy for subject- and object-raising verbs}
\end{figure}
%\itd{Anne: in this figure, is it possible to add  sr-a-lx as a subtype of adj-lx and subj-rsg-lx, as well as the items likey, seem and expect as examples in leaves?}
%\itd{Anne: should not be indented}
As in \crossrefchapterw[Section~\ref{properties:lexemes-and-words}]{properties}, upper case letters
are used for the two dimensions of classification, and \type{verb-lx}, \type{intr-lx}, \type{tr-lx},
\type{subj-rsg-lx}, \type{obj-rsg-lx}, \type{or-v-lx} and \type{sr-v-lx} abbreviate
\type{verb"=lexeme}, \type{intransitive"=lexeme}, \type{transitive"=lexeme},
\type{subject-raising"=lexeme}, \type{object"=raising"=lexeme}, \type{object"=raising"=verb"=lexeme}
and \type{subject-raising"=verb"=lexeme}, respectively.
%\itdobl{Stefan: Add \type{sr-a-lx}.}
The figure also shows three examples (\emph{likely}, \emph{seem} and \emph{expect}) inheriting from \type{sr-a-lx}, \type{sr-v-lx}, and \type{or-v-lx}, respectively.  The constraints on the types \type{subj-rsg-lx} and
\type{obj-rsg-lx} are as follows:\footnote{%
\isi{\append}  is used for list concatenation. The category of the complement is not specified as a VP
because it may be a V in some \ili{Romance} languages with a flat structure \citep{AG2003a-u} and in
some verb"=final languages where the matrix verb and the non"=finite verb form a verbal complex
(\ili{German}, \ili{Dutch}, \ili{Japanese}, \ili{Persian}, \ili{Korean}; see
\crossrefchapteralp{order} on constituent order and \crossrefchapteralp{complex-predicates} on complex predicates). Furthermore, other subtypes of these lexical types
will also be used for copular verbs that take non"=verbal predicative complements; see
Section~\ref{sec-copular-constructions}.%
}

%\itd{Stefan: What about ``Kim seems to Sandy to be smart.''? \type{subj-rsg-lx} whould allow for an optional object.}
\eal
\label{rsg}
\ex \type{subj-rsg-lx} \impl
\avm{ [ \argst  \1 \+ < \ldots, [subj & \1 ] >  ]} \label{rais-1}
\ex \type{obj-rsg-lx}  \impl
\avm{ [ \argst  < NP > \+ \1 \+ < [subj & \1 ] > ]} \label{rais2}
\zl

\noindent
%The \subj value ([1]) of the non"=finite verb is shared
%with the first argument (the subject) of the subject-raising verb in (\ref{rais-1}), and with the second argument (the object) of the
%object-raising verb in (\ref{rais2}). 
The \subj value of the non"=finite verb is appended to the beginning of the \argst and, provided
\ibox{1} contains an element, this means that the subject of the embedded verb is also the subject
of the subject-raising verb in (\ref{rais-1}). Similarly, if \ibox{1} is a singleton list, the subject of the
non-finite verb will be the second element of the \argst list of the object-raising verb in (\ref{rais2}).

This means that  their syntactic and semantic features are shared: they have the
same semantic index, but also the same part of speech, the same case, etc. Thus a subject appropriate for the
non"=finite verb is appropriate as a subject (or an object) of the raising verb: this allows for
expletive ((\ref{rain1}), (\ref{rain2})) or idiomatic ((\ref{cat1}), (\ref{cat2})) subjects, as well
as non"=nominal subjects (\ref{under}). If the embedded verb is subjectless, as in (\ref{german1}),
this information is shared too (\ibox{1} can be the empty list). The dots in (\ref{rsg}) account for
a possible PP complement as in \textit{Kim seems to Sandy to be smart.}, which we ignore in what
follows.

A subject-raising verb (\word{seem}) and an object"=raising verb (\word{expect}) inherit from
\type{sr-v-lx} and \type{or-v-lx}
%\itd{Stefan: Schouldn't these be \type{subj-rsg-lx} and \type{obj-rsg-lx}?}
 respectively, which are subtypes of \type{subj-rsg-lx} and \type{obj-rsg-lx} (see Figure~\ref{raising:fig-verb-hier2}); their lexical descriptions are as follows,
assuming a simplified MRS semantics (\citealp{CFPS2005a} and \crossrefchapteralp[Section~\ref{semantics:sec-mrs}]{semantics}):

% \itd{Stefan: Why does the type \type{subj-rsg-v-word} exist? Usually one would assume the lexeme  types and then there are inflectional lexical rules. The output of the lexical rule is of type
% \type{inflected-word} or something of this kind. In any case it is something that is the same for
% all verbs, not something specific to subject raising verbs. I would present a lexical item for the
% lexeme. This would make it possible to use the lexeme type.\\
% Anne: the pb is that I want to show valence features and they are not available for lexemes, only for words.\\
% Stefan: Ah, I see. Then maybe do not give a type at all and say: ``Lexical item for the word
% \emph{seem}''\\
% Anne: no, it is a lexical description and it is important to show how it derives from the lexeme
% type.}
\eas
Lexical description of \emph{seem} (\type{sr-v-lx}):\\
\avm{
[%\type*{sr-v-word}
subj  & <\1 > \\
comps & <\2 VP[head  & [vform & inf] \\
	       subj  & <\1> \\
	       cont & [ind &  \3] ]>\\
arg-st & <\1, \2> \\
cont &	[%ind & s \\
		rels &	<[\type*{seem-rel}
			 soa & \3]>]]
}
\zs
%\itdobl{Stefan: For all the semantic contributions to work, the index should fill a role in the predicate introduced in \rels. This is systematically missing from all semantic representations here.\\
%Anne: I dont udnerstand what you want, maybe Jean-Pierre can help.\\Stefan: s has to be an argument of \type{seem-rel}. Like this:\avm{[ind & s \\ rels &	<[\type*{seem-rel}
 %         event & s\\        soa & \3]>]}
%Otherwise the s would be unconnected.
%Anne: got rid of s then}
% \itdobl{Stefan: The semantics is still not correct. The index is used for embedding. So
%   \type{seem-rel} should have a SOA argument that refers to the INDEX of the embedded verb. If you
%   do not want to do this, then drop MRS and go back to what JP suggested. Then we would drop RELS
%   and the list and have just the \type{seem-rel} AVM as value of CONT. The embedded SOA would then
%   be the CONT value of the embedded VP. AA: I thought JP prefered CONT than Index for the embedded verb, I dont care}

%\itd{Stefan: Can we use numbers instead of the \ibox{i} or just the index $i$? \ibox{i} looks  strange \ldots Anne: ok for just the index i}
\eas
Lexical description of \emph{expect} (\type{or-v-lx}):\\
\avm{
[%\type*{or-v-word}
subj &	<\1 \NPi> \\
comps &	<\2, \3 VP[head  & [vform & inf] \\
	           subj  & <\2> \\
		   cont & [ind & \4] ]> \\
arg-st & <\1, \2, \3> \\
cont &	[%ind & s \\
		rels & <[\type*{expect-rel}
			  exp &  $i$\\
			  soa & \4]> ] ]
}
\zs
They take a VP and not a clausal complement, which means that the embedded infinitive has its complements realized locally (if any) but not its subject. The  corresponding simplified trees are as
shown in Figures~\ref{cons-1} and~\ref{cons2}. Notice that the syntactic structures are the same as for control verbs (Figures~\ref{sleep3} and~\ref{cons3}). 
%\itd{to Stefan: please do not change the figures, I put the SUBJ info on the inf verb on purpose}
%\itd{Stefan: ``Notice that the syntactic structures are the same.'' They are not.}
\begin{figure}
% \begin{tikzpicture}[baseline, sibling distance=2pt, level distance=60pt, scale=.9]
% 	\Tree
% 	[.{\begin{avm}
% 		\[phon & \phonliste{ Paul seems to sleep }\\
% 			subj & \eliste \\
% 			comps & \eliste\]
% 		\end{avm}}
% 		{\begin{avm}\[phon & \phonliste{ Paul } \\
% 			synsem & \@1 \]
% 		\end{avm}}
% 		[.{\begin{avm}
% 			\[phon & \phonliste{ seems to sleep }\\
% 			subj & \<\@1\>\]
% 			\end{avm}}
% 		 {\begin{avm}
% 			\[phon & \phonliste{ seems } \\
% 			subj & \<\@1\>\\
% 			comps & \<\@2 \[subj & \< \@1 \>\]\>\\
% 			\]
% 			\end{avm}} 
% 		{\begin{avm}
% 			\[phon \phonliste{ to sleep }\\
% 				synsem \@2 \]	
% 			\end{avm}}  
% 		]
% 	]
% \end{tikzpicture}
\begin{forest}
[
\avm{
[S\\
\phon <Paul seems to sleep> \\
subj & < > \\
comps & < > ]
}
	[
	\avm{
	[NP \\
  	\phon <Paul> \\
	synsem & \1 ]
	}
	]
	[
	\avm{
	[VP \\
	\phon <seems to sleep> \\
	subj & <\1> \\
	comps & < > ]
	}
    	[
    	\avm{
        [V\\
        \phon <seems> \\
          subj & <\1> \\
          comps & <\2> ]
		}
		]
    	[
    	\avm{
        [VP \\
        \phon <to sleep> \\
        synsem & \2	[subj & <\1>]]	
		}
		]
	]
]
\end{forest}
\caption{\label{cons-1}A sentence with a subject-raising verb}
\end{figure}

\begin{figure}
\begin{forest}
[
\avm{
[S\\
\phon <Mary expected Paul to work> \\
        subj & < > \\
        comps & < > ]		
}
	[
	\avm{
	[NP \\
	\phon <Mary> \\
	synsem & \1 ]
	}
	]
	[
	\avm{
	[VP \\
	\phon <expected Paul to work> \\
	subj & <\1> \\
	comps & < > ]
	}
		[
		\avm{
		[V \\
		\phon <expected> \\
		subj & <\1 > \\
		comps & <\2, \3> ]		
		}
		]
		[
		\avm{
		[NP \\
		\phon <Paul> \\
		synsem & \2 ]
		}
		]
		[
		\avm{
		[VP \\
		\phon <to work> \\
              synsem & \3 [subj & <\2>]]
		}
		]
	]
]
\end{forest}	
\caption{\label{cons2}A sentence with an object"=raising verb}
\end{figure}

Raising verbs have in common a mismatch between syntactic and
semantic arguments: the raising verb has a subject (or an object) which is not one of its semantic
arguments (its \textsc{index} does not appear in the \cont feature of the raising verb). To constrain this type of
mismatch, \citet[140]{PollardandSag1994} propose the Raising Principle.

\eanoraggedright
Raising Principle\is{principle!Raising}: Let X be a non"=expletive element subcategorized by Y; X is not assigned any
semantic role by Y iff Y also subcategorizes a complement which has X as its first argument.
\z

\noindent
This principle was meant to prevent raising verbs from omitting their VP complement, unlike control
verbs \citep[444]{Jacobson1990}. Without a non"=finite complement, the subject of \word{seem} is not
assigned any semantic role, which violates the Raising principle. However, some unexpressed (null)
complements are possible with some subject-raising verbs as well as VP ellipsis with \ili{English}
auxiliaries, which are analyzed as subject-raising verbs (see
Section~\ref{sec-auxiliaries-as-raising-verbs} below and
\crossrefchapteralp[Section~\ref{sec-analyses-of-pred-ellipsis}]{ellipsis} on predicate/argument ellipsis). So the
Raising Principle should be reformulated in terms of argument structure (which includes unexpressed arguments) and not valence
features.
%\itdobl{Stefan: I guess something is missing here. The reader does not know why \argst helps here.}

\eal
\ex[]{John tried / * seems.
}
\ex[]{
John just started.
}
\ex[]{
John did.
}
\zl

For subject-raising verbs which allow for a sentential complement as well (with an expletive
subject) (\ref{seem4}), another lexical description is needed (see (\mex{1}a)), and the same holds for
object"=raising verbs which allow a sentential complement (with no object) ((\ref{expect3}); see (\mex{1}b)). These
can be seen as valence alternations, which are available for some items (or some classes of items)
but not all (see \crossrefchapteralt{arg-st} on argument structure).

\eal
\ex \emph{seem}:   [\argst \sliste{ NP[\type{it}], S }]
\ex \emph{expect}: [\argst \sliste{ NP, S }]
\zl

\subsection{The HPSG analysis of control verbs}

\citet{SagandPollard1991} propose a semantics-based control theory in which the semantic class of the verb
determines whether it is subject-control or object-control. They distinguish verbs of orientation
(\emph{want}, \emph{hope}), verbs of commitment (\emph{promise}, \emph{try}) and verbs of influence
(\emph{persuade}, \emph{forbid}) based on the type of relation and semantic roles of their
arguments. Relational types for control predicates can be organized in a type hierarchy like the one
given in Figure~\ref{verb-hier3}, adapted from \citet[78]{SagandPollard1991}.\footnote{For further semantic classification of main predicates in order to account for optional control in languages such as Modern \ili{Greek} and Modern Standard \ili{Arabic}, see \citew{GreshlerMelnikWintner2017a-u}.}
%\itd{Stefan: Where is this hierarchy from? It is not Pollard\&Sag. Do you have a reference? Why is it different from what P\&S did? Anne: it is very similar, it is better for illustration purposes}
%\itd{Stefan: Having the \type{forbid-rel} as a subtype of \type{persuade-rel} does not work in the model-theoretic world, since all type have to be maximal in models. This would mean that all
 % relations of type \type{persuade-rel} are always automatically \type{forbid-rel}.} 
\begin{figure}
\oneline{%
\begin{forest}
type hierarchy
       [control-relation
      					[orientation-rel
      						[want-rel] 
      						 [hope-rel]
      						 [\ldots]   		
      					] 
      					[commitment-rel
      					 		[promise-rel]
      					 			[try-rel]
      					 		[\ldots]
      					 	]
      					 	 [influence-rel
      					 		[persuade-rel]
      				 			[forbid-rel]
      					 		[\ldots]
      					 	]
      					 	[\ldots]
      					]  
      	]
\end{forest}}
\caption{\label{verb-hier3}A type hierarchy for control predicates}
\end{figure}

For example, \emph{want}, \emph{promise} and \emph{persuade} have semantic content such as the
following, where \textsc{soa} means state-of-affairs and denotes the content of the non"=finite
complement:\footnote{
  The fact that \textsc{soa} has a value of type \type{relation} follows
  from the general setup of AVMs that is specified as the so-called signature of the grammar and need not be
  given here (see \crossrefchapteralt[Section~\ref{sec-signatures}]{formal-background}). I state it
  nevertheless for reasons of exposition.
}

%\itd{Stefan: Stating \type{relation} in the lexical items seems redundant since this is the most general type for something having \textsc{arg} as a feature and being the value of \textsc{soa}.\\
%Anne: I guess it is useful for the reader, it is not the same rel as that of the matrix verb.\\
%Stefan: I still find it confusing.\\
%Anne: you're not the target reader. JPK says ok}
\eal
%\word{want}:\\
\ex 
\avm{
[\type*{want-rel}
 experiencer & \1 \\
 soa [\type*{relation}  \\
      arg & \1] ]
}
\ex
%\word{promise}:\\
\avm{
[\type*{promise-rel}
 commitor & \1 \\
 commitee & \2 \\
 soa 	[\type*{relation}  \\
	 arg & \1]]
}
\ex
%\word{persuade}:\\*
\avm{
[\type*{persuade-rel}
 influencer & \1 \\
 influenced & \2 \\
 soa 	[\type*{relation}  \\
	 arg & \2]]
}	
\zl

According to this theory, the controller is the experiencer with verbs of orientation, the commitor
with verbs of commitment, and the influencer with verbs of influence. From the syntactic point of
view, two types of control predicates, \type{subject-cont-lx} and \type{object-cont-lx}, can be
defined as follows:

\eal
\label{cont}
\ex
\label{subj-contr-lx}
\type{subj-contr-lx} \impl\\
\avm{ [ \argst  < \NPi, \ldots, [subj & < [ind & $i$ ] > ] > ]}
%\itd{Stefan: \ibox{1} is not shared. Remove? Start with \ibox{1} rather than \ibox{0}? Like (\mex{1}c)? Replace non-shared tags with [].}
\ex
\type{obj-contr-lx} \impl\\ 
\avm{ [ \argst  < ![]!, XP$_i$, [subj & < [ind & $i$ ] > ] > ]}
%\ex
%\type{obj-cont-lx} \impl\\ 
%\avm{ [ \argst  < \1, [ind & \tag{i}], [subj & < [ind & \tag{i} ] > ] > ]}
\zl
%\itdopt{Stefan: Maybe use contr rather than cont to avoid ``content''.}

The controller is the first argument with subject-control verbs, while it is the second argument
with object-control verbs. Contrary to the types defined for raising predicates in (\ref{rsg}), the
controller here is simply coindexed with the subject of the non"=finite complement. Since the
controller is referential and since it is coindexed with the controlee, the controlee has to be
referential as well. This means it must have a semantic role (since it has a referential index),
thus expletives and (non referential) idiom parts are not allowed ((\ref{rain3}), (\ref{rain4}),
(\ref{cat3}), (\ref{cat4})). This also implies that its syntactic features may differ from those of
the subject of the non"=finite complement: it may have a different part of speech (a NP subject can
be coindexed with a PP controller) as well as a different case ((\ref{to}), (\ref{hope-i})).

Verbs of orientation and commitment inherit from the type \type{subj-contr-lx}, while verbs of
influence inherit from the type \type{obj-contr-lx}.  A subject-control verb (\word{want}) and an
object-control verb (\word{persuade}) inherit from \type{sc-v-lx} and \type{oc-v-lx}
respectively; their lexical descriptions are as follows:
%\footnote{To account for Visser's
 % generalization (object-control verbs passivize while subject-control verbs do not),
 % \citet{SagandPollard1991} analyse the subject of the infinitive as a reflexive, which must be
 % bound by the controller. According to Binding Theory (see \crossrefchapteralp{binding}), the
 % controller must be less oblique than the reflexive, hence less oblique than the VP complement
 % which contains the reflexive: the controller can be the subject and the VP a complement as in
 % (\ref{ex-John-promised-Mary-to-buy}) and (\ref{persuade-pass}); it can be the first complement
%  when the VP is the second complement as in (\ref{ex-John-permitted-Mary-to-buy}), but it cannot be
%  a \emph{by}-phrase, which is more oblique than the VP complement, as in (\ref{promise-pass}) (the
 % \emph{by}-phrase should not be bound according to principle C, and the subject of the infinitive
 % should be bound according to principle A).
%\itdopt{Stefan: This account of Visser's generalization did not work. I had a discussion of it in my
 % binding chapter. If you want, I can add it back in.}}


%\itdobl{RELS should be a list since it is a list everywhere else in the volume. The types of the
 % lexical items should be removed.Anne: I changed the set to lists; the types are important to show how the lexicon is structured: the words %inherit from lexeme types}
\eas
Lexical description of \emph{want} (\type{sc-v-lx}):\\
\avm{
[subj  & < \1 \NPi > \\
	comps & < \2 VP[head  &	[vform & inf] \\
		        subj  &	<[ind & i]> \\
			cont & [ind &  \3] ]> \\
	arg-st & <\1, \2> \\
	cont &	[%ind & s \\
			rels &	<[\type*{want-rel}
					exp & i \\
					soa & \3]>] ]
}
\zs
\eas
Lexical description of \emph{persuade} (\type{oc-v-lx}):\\*
\avm{
	[subj & <\1 \NPi > \\
	comps & <\2 \NPj, \3 VP[head  & [vform & inf] \\
	                        subj  &	<[ind & j]> \\
	                        cont  & [ind &  \4] ]>\\
	arg-st & <\1, \2, \3> \\
	cont &	[%ind & s \\
		 rels & <[\type*{persuade-rel}
			agent & i \\
			patient & j \\
			soa & \4]> ] ]
}
\zs
The corresponding structures for subject-control and object-control sentences are illustrated in Figures~\ref{sleep3} and~\ref{cons3}.
%
\begin{figure}
\begin{forest}
[
\avm{
[S\\
\phon <Paul wants to sleep> \\
      subj & < > \\
      comps & < > ]
}
	[
	\avm{
	[NP\\
    \phon <Paul> \\
      synsem \1  [ind & i]]
	}
	]
	[
	\avm{
	[VP\\
    \phon <wants to sleep> \\
        subj & <\1> \\
        comps & < > ]
	}
    	[
    	\avm{
        [V\\
        \phon <wants> \\
          subj & <\1 NP> \\
          comps & <\2 > ]
		}
		] 
		[
		\avm{
        [VP\\
        \phon <to sleep> \\
          synsem & \2 [subj & < NP$_{i}$ > ] ]	
		}
		]
	]
]
\end{forest}
% \begin{forest}
% [\avm{
%    [phon & \phonliste{ Paul wants to sleep }\\
%     subj & \eliste \\
%     comps & \eliste ]
%   }
%   [\avm{
%     [phon \phonliste{ Paul } \\
%      synsem \1 ]
%     }]
%   [\avm{
%       [phon & \phonliste{ wants to sleep }\\
%        subj & < \1 >]
%     }
%     [\avm{
%       [phon  & \phonliste{ wants } \\
%        subj  & < \1 [ cont|ind  \type{i} ] >\\
%        comps & < \2 [ subj & < \normalfont NP$_{i}$ > ] >\\
%         \]
%       }] 
%     [\avm{
%         [phon   & \phonliste{ to sleep }\\
%          synsem & \2  ]	
%      }] ] ]
% \end{forest}
\caption{\label{sleep3}A sentence with a subject-control verb}
\end{figure}
%
%
%The corresponding trees are given in Figure~\ref{cons2} and~\ref{cons3}. Notice that the syntactic structures are the same.
%
\begin{figure}
\oneline{
\begin{forest}
[
\avm{
[S\\
\phon <Mary persuaded Paul to work> \\
      subj & < > \\
      comps & < > ]		
}
	[
	\avm{
	[NP\\
	\phon <Mary> \\
	synsem & \3 ]
	}
	]
	[
	\avm{
	[VP\\
	\phon <persuaded Paul to work> \\
        subj & <\3 NP> \\
        comps & < >]
	}
		[
		\avm{
        [V\\
         \phon <persuaded> \\
          subj & <\3 NP>\\
          comps & <\1, \2 > ]		
		}
		]
		[
		\avm{
		[NP\\
		\phon <Paul> \\
          synsem \1 [ind & i]]
		}
		]
		[
		\avm{
        [VP\\
        \phon <to work> \\
          synsem & \2 [subj & < NP$_{i}$> ] ]
		}
		]
	]
]
\end{forest}
}	
\caption{\label{cons3}A sentence with an object-control verb}
\end{figure}

In some \ili{Slavic} languages (\ili{Russian}, \ili{Czech}, \ili{Polish}), the predicative adjective must share case with the subject of the copular verb (\mex{1}a): some subject-control
verbs may allow case sharing like subject-raising verbs (\mex{1}b), unlike object"=control verbs (\mex{1}c).
%as shown by predicate case agreement with quantified (non"=nominative) subjects. 
As proposed by \citet{Przepiorkowski2004} and
\citet{PrzepiorkowskiandRosen2005}, coindexing does not prevent full sharing, so the analysis may
allow for both (shared) nominative and (default) instrumental case for the unexpressed subject and the predicative adjective, and a specific constraint may be added to enforce only (nominative) case sharing with the relevant set of verbs.\footnote{% 
   The examples in (\mex{1}) are taken from \citew[ex (6)--(7)]{Przepiorkowski2004}.
}
%\itd{Stefan: This is too dense. The reader is lacking lots of information.}
%\itd{Stefan: There was a problem with this analysis. I remember Adam giving a talk at the HPSG conference and I pointed out to him that it did not work, but the paper was published already.}
%prevent default (instrumental) case assignment to the embedded predicate

\begin{exe}
\ex \begin{xlist}
\ex 
\gll Janek jest miły.\\
     Janek.\textsc{nom} is nice.\textsc{nom} \\ \hfill(\ili{Polish})
\glt `Janek is nice.'
\ex 
\gll Janek              zaczal  /  chce  by'c            miły.\\
     Janek.\textsc{nom} started {} wants be.\textsc{inf} nice.\textsc{nom} \\ 
\glt `Janek started / wants to be nice.'
\ex 
\gll Janek              kazal   Tomkowi            by'c            miłym              / *milemu.\\
     Janek.\textsc{nom} ordered Tomek.\textsc{dat} be.\textsc{inf} nice.\textsc{inst} {} \hphantom{*}nice.\textsc{dat} \\ 
\glt `Janek ordered Tomek to be nice.'
%\itd{Stefan: It is unclear how this can work with both acc and gen, since the NP will be either acc or gen. Looking on Adam's paper, there was no solution to this problem. Anne: it is fine, five girls may be acc or gen=> are nice:acc/gen}
%\ex 
%\gll Pięć dziewcząt chce być miłe / miłych. \\
 %    five.\textsc{acc} girls.\textsc{gen} want be.\textsc{inf} nice.\textsc{acc} {} nice.\textsc{gen}\\
%\glt `Five girls want to be nice.'
	\end{xlist}
\end{exe}


For control verbs which allow for a sentential complement as well ((\ref{hope1}), (\ref{promise1})),
another lexical description of the kind in (\mex{1}) is needed. These can be seen as valence alternations, which are
available for some items (or some classes of items) but not all (see \crossrefchapteralt{arg-st} on argument structure).

\eal
\ex \emph{want}: [\argst \sliste{ NP, S }]
\ex \emph{promise}: [\argst \sliste{ NP, NP, S }]
\zl




\subsection{Raising and control verbs in Mauritian}\label{sec-maurit}

\il{Mauritian|(}%
\ili{Mauritian}, which is a \ili{French}-based creole, provides some evidence for a phrasal (and not sentential) analysis of the verbal complement of raising and control verbs.
\ili{Mauritian} raising and control verbs belong
roughly to the same semantic classes as in \ili{English} or \ili{French}. Verbs marking aspect or
modality (\word{kontign} `continue', \word{aret} `stop') are subject-raising verbs, and causative and
perception verbs (\emph{get} `watch') are object"=raising. Raising verbs have different properties from TMA (tense
modality aspect) markers: they are preceded by the negation, which follows
TMA, and they can be coordinated, unlike TMA \citep[\page 209]{HenriandLaurens2011}:

\eal
\ex[]{ 
\gll To pou kontign ou aret bwar? \\
     2\SG{} \IRR{} continue.\textsc{sf} or stop.\textsc{sf} drink.\textsc{lf}\\\hfill(\ili{Mauritian})
\glt `You will continue or stop drinking?'
}
\ex[*]{
\gll To'nn ou pou aret bwar? \\
     2\SG{}'\PRF{} or \IRR{} stop.\textsc{sf} drink.\textsc{lf}\\
\glt  `You have or will stop drinking?'
}
\zl
 
If their verbal complement has no external argument, as is the case with impersonal expressions such as \word{ena lapli} `to rain', then the raising verb itself has no external argument, in contrast to a control verb like \word{sey} `try':

\eal
\ex[]{
\gll Kontign     ena lapli. \\
     continue.\textsc{sf} have.\textsc{sf} rain \\
\glt `It continued to rain.'
}
\ex[*]{
\gll Sey ena lapli. \\
     try have.\textsc{sf} rain \\
\glt Literally: `It tries to rain.'
}
\zl

Verb morphology in \ili{Mauritian} provides an argument for the phrasal (and not clausal) status of the complement of both control and raising verbs. Unlike in \ili{French}, its superstrate, in \ili{Mauritian},  verbs inflect neither for tense, mood and aspect nor for person, number, and
gender. But they have a short form and a long form (henceforth \textsc{sf} and \textsc{lf}), with
30\% of verbs showing a syncretic form (as for example \emph{bwar} `drink'). The following list of examples provides pairs of short and
long forms respectively:

\eal
\ex manz/manze `eat', koz/koze `talk', sant/sante `sing'
\ex pans/panse `think', kontign/kontigne `continue', konn/kone `know'
\zl

As described in \citet[Chapter~4]{Henri2010}, the verb form is determined by the construction: the
short form is required before a non-clausal complement,
%\itdopt{Stefan: What does ``phrasal complement'' mean? sega and pom are words. Anne: they are also noun phrases}
  and the long form appears otherwise.\footnote{\textit{yer} `yesterday' is an adjunct. See \citew{Hassamal2017} for an analysis of \ili{Mauritian} adverbs which treats as complements those that trigger the verb short form.}


\begin{exe}
\ex \begin{xlist}
\ex 
\gll Zan sant             [sega]         /  manz            [pom]           /  trov             [so mama] / pans [Paris]. \\
     Zan sing.\textsc{sf} \spacebr{}sega {} eat.\textsc{sf} \spacebr{}apple {} find.\textsc{sf} \spacebr{}\POSS{} mother {} think.\textsc{sf} \spacebr{}Paris \\
\glt `Zan sings a sega / eats an apple / finds his mother / thinks about Paris.'	
\ex 
\gll Zan sante / manze.\\
     Zan sing.\textsc{lf} {} eat.\textsc{lf}\\
\glt `Zan sings / eats.'
\ex 
\gll Zan ti zante yer. \\
Zan  \PRF{} sing.\textsc{lf} yesterday\\
\glt `Zan sang yesterday.'
\end{xlist}
\end{exe}


\citet[\page 258]{Henri2010} proposes to define two possible values (\type{sf} and \type{lf}) for the head
feature \vform, with a lexical constraint on verbs simplified as follows (\type{nelist} stands for non-empty list):

\ea
\label{sf-constraint}
\avm{
[\type*{v-word}\\
vform & sf]} \impl 
\avm{
[comps & nelist]
}
\z
% \itdobl{Stefan: You misquoted Henri since she has a different formulation. Your version does not  work. The reason is that your constraint applies to all AVMs with VFORM sf. This includes verbs
%   that have been combined with complements. The constraint causes a contradiction, which makes it impossible to analyze sentences with VFORM sf verbs. What Henri has is a constraint applying to
%  lexical verbs only. This is what you need.\\
% Anne: I say it is a lexical constraint, so it applies to verbs, not to phrases.\\
% Stefan: I would remove the \type{v-} of \type{v-word} since it is redundant (head value
% \type{verb}follows from the presence of \vform and it may be wrong since there is \type{v-lx} but
% the words are derived by lexical rules.\\
% Anne: redundancy does not hurt, might help the naive reader.\\
% Stefan: There should not be any redundancy for explanatory purposes in the formal part. You can be
% as redundant as you want in the prose, but the technical part should be in the way we want the
% theory to be. AA I  dont agree, this is why I add relation in 36
% }

Interestingly, clausal complements do not trigger the verb short form (\citealp[131]{Henri2010} analyses them as extraposed). The complementizer (\emph{ki}) is optional.

\eal
\ex 
\gll Zan panse             [(ki)               Mari pou    vini].\\
     Zan think.\textsc{lf} \hphantom{[(}that Mari \FUT{} come.\textsc{lf}\\
\glt `Zan thinks that Mari will come.'
\ex 
\gll Mari trouve           [(ki)                so      mama   tro      manze].\\
     Mari find.\textsc{lf} \hphantom{[(}that  \POSS{} mother too.much eat.\textsc{lf}\\
\glt `Mari finds that her mother eats too much.'
\zl

On the other hand, subject-raising and subject-control verbs occur in a short form before a verbal complement.

\begin{exe}
\ex \begin{xlist}
\ex \gll Zan kontign [sante].\\
Zan continue.\textsc{sf} \spacebr{}sing.\textsc{lf}\\\jambox*{(subject-raising verb, p.\,198)}
\glt `Zan continues to sing.'
\ex \gll Zan sey [sante].\\
Zan try.\textsc{sf} \spacebr{}sing.\textsc{lf}\\\jambox{(subject-control verb)}
\glt `Zan tries to sing.'
\end{xlist}
\end{exe}

The same is true with object-control and object"=raising verbs:
\eal
\settowidth\jamwidth{(object"=raising verb, p.\,200)}
\ex \gll Zan inn fors [Mari] [vini].\\
Zan \PRF{} force.\textsc{sf} \spacebr{}Mari \spacebr{}come.\textsc{lf}\\\jambox{(object-control verb)}
\glt `Zan has forced Mari to come.'
\ex \gll Zan pe get [Mari] [dormi].\\
Zan \PROG{} watch.\textsc{sf} \spacebr{}Mari \spacebr{}sleep.\textsc{lf}\\\jambox{(object"=raising verb, p.\,200)}
\glt `Zan is watching Mari sleep.'
\end{xlist}
\end{exe}


Raising  and control verbs thus differ from verbs taking sentential complements. Their \textsc{sf} form is
predicted if they take unsaturated VP complements. Assuming the same lexical type hierarchy as
defined above, verbs like \word{kontign} `continue' and \word{sey} `try' inherit from
%\itd{Stefan: \type{subj-rsg-lx}?}
\type{sr-v-lx} and \type{sc-v-lx}
%\itdopt{Stefan: The type sc-v-lx has never been mentioned
 % before. One can get it by analogy but may be it would be better to explain this explicitely.Anne: it is mentioned for English in (37)} 
 respectively.\footnote{\citeauthor{HenriandLaurens2011} use Sign-based Construction Grammar (SBCG) (see
  \crossrefchapteralp[Section~\ref{prop:sec-sbcg}]{properties} and
  \crossrefchapteralp[Section~\ref{cxg:sec-sbcg}]{cxg}), but their analyses can be adapted to the
  feature geometry of Constructional HPSG \citep{Sag97a} assumed in this volume. The analysis of
  control verbs sketched here will be revised in Section~\ref{section-xarg} below.}
  %\citet[\page197]{HenriandLaurens2011} conclude that ``while \ili{Mauritian} data can be brought in accordance with the open complement analysis, both morphological data on the control or raising verb and the existence of genuine verbless clauses put up a big challenge for both the clause and small clause analysis.''.
\il{Mauritian|)}


\subsection{Raising and control in pro-drop and ergative languages}

The theory of raising and control presented above naturally extends to pro-drop and ergative
languages.  But a distinction must be made between subject and first syntactic argument. Since \citet*{BMS2001a}, it is widely assumed
%\itd{Stefan: Not everybody assumes this. I do not.  See also \citew{LH2006a}.Anne: Bob suggests 'widely' assumed} 
  that syntactic arguments are listed in \argst and
that only canonical ones are present in the valence lists (\subj, \spr and \comps). See the Argument Realization Principle (ARP) in \crossrefchapterw[\pageref{properties:ex-ARP}]{properties}.
%\crossrefchaptert{udc} for an analysis of UDC with non"=canonical \type{synsem}. 
For pro-drop languages,
it has been proposed, \eg in \citep[\page 65]{ManningandSag1998}, that null subject verbs have
a first argument having the non-canonical \type{synsem} type \type{pro}, representing the unexpressed subject in the \argst list, but nothing in
their \subj list.

\eal
\ex 
\label{Italian}
\gll Vengo.\\
     come.\PRS.1\SG\\\hfill(\ili{Italian})
\glt `I come.'
\ex 
\label{Italian-raising}
\gll Posso venire.\\
     can.1\SG{} come.\INF\\
\glt `I can come.'
\ex 
\label{Italian-control}
\gll Voglio venire.\\ 
     want.1\SG{} come.\INF\\
\glt `I want to come.'
\zl

Assuming the lexical types for \type{sr-v-lx} and \type{sc-v-lx} in (\ref{rsg}) and
(\ref{cont}), the verbal descriptions for (\ref{Italian-raising}) and (\ref{Italian-control}) are as
follows:
%Vengo: SUBJ <>, Comps <>, ARG-ST<[pro]>
%Posso: SUBJ  <>, Comps <2>, Arg-st <1[pro], 2VP[SUBJ <1>]>
%Voglio SUBJ  <>, Comps <2>, Arg-st <NPi[pro], 2VP[SUBJ <NPi>]>
\eal
\ex	
\type{posso} `can' (\type{sr-v-lx}):\\
\avm{
	[subj & <> \\
	comps & < \2 > \\
	arg-st & < \1![\type{pro}]!, \2[subj & \1]>]
}\label{rais1}
\ex 
\type{voglio} `want' (\type{sc-v-lx}):\\
\avm{
	[subj & <> \\
	comps & < \2 > \\
	arg-st & < \NPi![\type{pro}]!, \2[subj & < [ind & $i$] >] > ]
}
%\itdobl{Stefan: Remove types.Anne: I need them to show where theses entries come from}
\zl


\ili{Balinese}\il{Balinese|(}, an ergative language, provides another example of non-canonical subjects.  \citet{WechslerandArka1998} argue that the subject is not necessarily the first syntactic argument in this language. A transitive verb has two verb forms, called ``voice'', and there is  rigid SVO order,
regardless of the verb's voice form. In the agentive voice (AV), the
subject is the \argst initial member, while in the objective voice (OV), the verb is transitive, and
the subject is the initial NP, although it is not the first element of the \argstl.
%\itd{Stefan: This is confusing since it is unclear what ``first'' refers to. In the utterance the subject is first but not in  \argst. I would suggest ``although it is not the first element of the \argstl''.} 
   (see
\crossrefchapteralp[Section~\ref{arg-st-sec-ergativity}]{arg-st}):

\eal
\ex  
\gll Ida ng-adol bawi.\\
     3\SG{} \textsc{av}-sell pig\\ \hfill(\ili{Balinese})
\glt `He/She sold a pig.'
\ex 
\gll Bawi adol ida.\\
     pig \textsc{ov}.sell 3\SG \\
\glt `He/She sold a pig.' 
\zl

Different properties argue in favor of a subject status of the first NP in the objective
voice. Binding properties show that the agent is always the first element on the \argst list; see
\citew{WechslerandArka1998}, \citew{ManningandSag1998} and \crossrefchaptert[Section~\ref{binding:toba-batak}]{binding}. The objective
voice is also different from the passive: the passive may have a passive prefix and an agent
\emph{by}-phrase, and it does not constrain the thematic role of its subject. The two verbal types can
be defined as follows (see \crossrefchapteralp[Section~\ref{arg-st-sec-ergativity}]{arg-st}):

%\itd{Stefan: Both \ibox{1} and \ibox{2} may be the empty list or more generally lists of arbitrary  length.\\
%Anne: not relevant here; I'm summarizing W\&A's paper.\\
%Stefan: Chapter~9 does not have these constraints. W\&A have a %general constraint but mention subjects explicitely. This is not done in your implicational constraints below. But you can say:
%``Together with a constraint requiring that the SUBJ list has to have exactly one element, the following lexical items are licensed.'' Anne I prefer not to go into that, but it is the general case, it is true that they can be pro drop as well}
\eal
\ex 
\type{av-verb} \impl\\
\avm{
	[subj & \1 \\
	comps & \2 \\
	arg-st & \1 \+ \2 ] 
}
\ex 
\type{ov-verb} \impl\\
\avm{
	[subj & \1 \\
	comps & \2 \\
	arg-st & \2 \+ \1 ]
}
\zl

Together with a constraint stating that the \subj list has at most one element, these constraints license the following two verb forms:

\eal
\ex 
Lexical description of \emph{ng-adol} `sell.AV':\\
\avm{
	[subj & <\NPi> \\
	comps & <\NPj>\\
	arg-st & <\NPi, \NPj> ] 
}
\ex 
Lexical description of \emph{adol} `sell.OV':\\
\avm{
	[subj & <\NPj> \\
	comps & <\NPi>\\
	arg-st & <\NPi, \NPj> ]
}
\zl
In this analysis, the preverbal argument, whether the theme of an OV verb or the agent of an AV
verb, is the subject, and as in many languages, only a subject can be raised or controlled
\citep{Chomsky1981,Zaenenetal1985}. Thus the first argument of the verb is controlled when the embedded verb is
in the agentive voice, and the second argument
%\itd{Stefan: What about ditransitive verbs? Anne see below} 
is controlled when the verb is in the objective
voice.\footnote{%
  The examples in (\mex{1}) are taken from \citew[ex 25]{WechslerandArka1998}.
}


\begin{exe}
\ex \begin{xlist}
\ex 
\gll Tiang edot [teka].\\
     1 want     \spacebr{}come\\\hfill(\ili{Balinese})
\glt `I want to come.'
\ex 
\gll Tiang edot [meriksa dokter].\\
     1     want \spacebr{}\textsc{av}.examine doctor\\
\glt `I want to examine a doctor.'
\ex 
\gll Tiang edot [periksa dokter].\\
     1     want \spacebr{}\textsc{ov}.examine doctor\\
\glt `I want to be examined by a doctor.'
\end{xlist}
\end{exe}



%\ili{Balinese} also has subject-raising verbs like \word{ngenah} `seem':\footnote{
%The examples in (\mex{1}) are taken from \citew[ex 7]{WechslerandArka1998}.
%}
%\eal
%\ex 
%\gll Ngenah ia mobog.\\
 %    seem 3 lie\\\hfill(\ili{Balinese})
%\glt `It seems that (s)he is lying.'
%\ex 
%\gll  Ia ngenah mobog.\\
 %     3 seem lie\\
%\glt `(S)he seems to be lying.'
%\zl

Similarly, only the agent can be ``raised'' when the embedded verb is in the agentive voice, since it is the subject. And only the patient can be ``raised'' (because that is the subject) when the embedded verb is in the objective voice:\footnote{
The examples in (\mex{1}) are taken from \citew[391--392]{WechslerandArka1998}.
}
%The same applies to a transitive verb in the agentive voice: the agent can %appear as the subject of \emph{ngenah} `seem' but not the patient.

\eal
%\judgewidth{?*}
%\ex[]{ 
%\gll Ngenah sajan [ci ngengkebang kapelihan-ne].\\
  %   seem much \spacebr{}2 \textsc{av}.hide mistake-3\POSS\\\hfill\citep[ex 9]{WechslerandArka1998}
%\glt `It is very apparent that you are hiding his/her wrongdoing.'
%}
\ex[]{
\gll Ci ngenah sajan ngengkebang kapelihan-ne.\\
     2 seem much \textsc{av}.hide mistake-3\POSS\\\hfill(\ili{Balinese})
\glt `You seem to be hiding his/her wrongdoing.'
}
%\ex[?*]{ 
%\gll Kapelihan-ne ngenah sajan ci ngengkebang.\\
%     mistake-3\POSS{} seem much 2 \textsc{av}.hide\\}
%\zl
%\eal
%\judgewidth{?*}
%\ex[]{ 
%\gll Ngenah sajan [kapelihan-ne engkebang ci].\\
%     seem much \spacebr{}mistake-3\POSS{} \textsc{ov}.hide 2\\\hfill
%\glt `It is very apparent that you are hiding his/her wrongdoing.'
%}
\ex[]{
\gll Kapelihan-ne ngenah sajan engkebang ci.\\
     mistake-3\POSS{} seem much \textsc{ov}.hide 2 \\
\glt `His/her wrongdoings seem to be hidden by you.'
}
%\ex[?*]{
%\gll Ci ngenah sajan kapelihan-ne engkebang.\\
 %    2 seem much mistake-3\POSS{} \textsc{ov}.hide \\}
\zl

Turning now to ditransitive verbs, \word{majanji} `promise' denotes a commitment relation, so the promiser must have
semantic control over the action promised \parencites{Farkas1988,Kroeger1993}[\page 78]{SagandPollard1991}\addpages. The
promiser should therefore be the agent of the lower verb. This semantic constraint interacts
with the syntactic constraint that the controllee must be the subject, predicting that the
lower verb must be in agentive voice, with an agentive subject:\footnote{%
  The examples in (\mex{1}) are taken from \citew[398--399]{WechslerandArka1998}.
}
\eal
\ex[]{
\gll Tiang majanji maang Nyoman pipis.\\
     1 promise \textsc{av}.give Nyoman money\\\hfill(\ili{Balinese})
\glt `I promised to give Nyoman money.' 
}
\ex[*]{ 
\gll Tiang majanji Nyoman baang pipis. \\
     1 promise Nyoman \textsc{ov}.give money \\
}
\ex[*]{ 
\gll Tiang majanji pipis baang Nyoman. \\
     1 promise money \textsc{ov}.give Nyoman\\ 
}
\zl
The same facts obtain for other control verbs such as \word{paksa} `force'.
Turning now to object-raising verbs like \emph{tawang} `know',  these can occur in the agentive
voice with an embedded AV verb (\ref{av}) and with an embedded OV verb (\ref{ov}), unlike control
verbs like \emph{majanji} `promise'. 
%\ili{Balinese} also displays object"=raising. While the subject of \emph{mulih} %`go home' has been ``raised'' to the
They can also occur in the objective voice when the subject of the embedded verb is raised.  In
(\ref{rais-av}), the embedded verb (\emph{nangkep} `arrest') is in the agentive voice, and its
subject (\emph{polisi} `police') is raised to the subject of \emph{tawang} `know' in the objective
voice; in (\ref{rais-ov}), the embedded verb (\emph{tangkep} `arrest') is in the objective voice, and
its subject (\emph{Wayan}) is raised to the subject of \emph{tawang} `know' in the objective voice \citep[ex 23]{WechslerandArka1998}.

\eal
%\gll
% Nyoman Santosa tawang           tiang  mulih.\\
  %   Nyoman Santosa \textsc{ov}.know 1      go.home\\\hfill\citep[ex 22]%{WechslerandArka1998}
%\glt `I knew that Nyoman Santosa went home.'
%\ex 
%\gll Tiang nawang           Nyoman Santosa mulih.\\
%     1     \textsc{av}.know Nyoman Santosa go.home\\
%\glt `I knew that Nyoman Santosa went home.'
\ex 
\label{av}
\gll Ia nawang          polisi lakar  nangkep            Wayan. \\
     3 \textsc{av}.know police \FUT{} \textsc{av}.arrest Wayan \\\hfill(\ili{Balinese})
\glt `He knew that the police would arrest Wayan.'
\ex
\label{rais-av} 
\gll Polisi tawang=a           lakar  nangkep            Wayan. \\
     police \textsc{ov}.know=3 \FUT{} \textsc{av}.arrest Wayan\\

\ex
\label{ov}
\gll Ia nawang           Wayan lakar  tangkep            polisi.\\
     3  \textsc{av}.know Wayan \FUT{} \textsc{ov}.arrest police\\
\glt `He knew that the police would arrest Wayan.'
\ex
\label{rais-ov}
\gll Wayan tawang=a           lakar  tangkep            polisi. \\
     Wayan \textsc{ov}.know=3 \FUT{} \textsc{ov}.arrest police\\ 
\zl


In \ili{Balinese}, the subject is always the controlled (or ``raised'') element, but it is not
necessarily the first argument of the embedded verb. The semantic difference between control verbs
and raising verbs has a consequence for their complementation: raising verbs (which do not constrain
the semantic role of the raised argument) can take verbal complements either in the agentive or
objective voice, like subject-control verbs, while object-control verbs (which select an agentive argument) can only take a
verbal complement in the agentive voice. This difference is a result of the analysis of raising and
control presented above, and nothing else has to be added.\il{Balinese|)}
%As a result, it is always the subject of the embedded predicate that is coindexed or shared with an argument of the matrix verb, but the subject is not always the first syntactic argument.



\subsection{\xarg and an alternative HPSG analysis}\label{section-xarg}

Sometimes, obligatory control is also attested for verbal complements with an expressed subject.  As
noted by \citet{Zec87a-u}, \citet{Farkas1988} and \citet[\page 115--116]{GH2001a-u}, in some languages, such
as \ili{Romanian}, \ili{Japanese} \citep{Kuno76a-u,Iida96a-u} or \ili{Persian} \citep{Karimi2008},
the expressed subject of a verbal complement may display obligatory control. This may be a challenge
for the theory of control presented here, since a clausal complement is a saturated complement with
an empty \subjl, and the matrix verb cannot access the \subjv of the embedded verb. \citet[\page
89]{SP91a-u} proposed a semantic feature external-argument (\textsc{ext-arg}), which makes the index of the
subject argument available at the clausal level.  \citet[409]{Sag2007a} proposed to introduce a Head
feature \xarg that takes as its value the first syntactic argument of the head verb and is
accessible at the clause level.

This is adopted by \citet[Section~6]{HenriandLaurens2011} for \ili{Mauritian}.  After some
subject-control verbs like \word{pans} `think', the embedded verb may have an optional clitic
subject which must be coindexed with the matrix subject.  It is not a clausal complement since the
matrix verb is in the short form (\textsc{sf}) and not in the long form (see (\ref{sf-constraint})
above).

\ea
\gll Zan$_{i}$ pans          (*ki)           (li$_{i}$)            vini.\footnotemark\\
     Zan       think.\textsc{sf} \hphantom{(*}that \hphantom{(}3\SG{} come.\textsc{lf}  \\\hfill(\ili{Mauritian})
\footnotetext{
From \citet[\page 202]{HenriandLaurens2011}.
}
\glt `Zan thinks about coming.'
\z

%\itd{Stefan: You wrote ``Using XARG, they propose for pans ‘think’ the following description.'' This  was wrong. What they suggest is totally different in details. I adapted the text so that it is  correct and cites correctly.}
%\itd{Anne:no indent}
\noindent
Using \xarg, \citet[\page 214]{HenriandLaurens2011} propose a description for \word{pans} `think' that is
simplified in (\mex{1}). The complement of
\word{pans} must have an \xarg coindexed with the subject of \word{pans}, but its \subjl is not
constrained: it can be a saturated verbal complement (whose \subjv is the empty list) or a VP
complement (whose \subjv is not the empty list).

\ea
\label{ex-pans-Maritian}
Lexical description of \type{pans} `think':\\
\avm{
	[subj & <\NPi > \\
	comps &	<[head [\type*{verb}\\
			xarg &	[ind & $i$] ]	]>
		 % marking  \type{pou}~~~~~~~~~~~~~
			%cont & \[ind & \@2\]
	%cont & \[ind & s \\
		%	rels & \{\[\asort{think-rel}
		%	arg \@i \\
		%	arg & \@2\]\}\]
	]
}
\z

%\itd{Anne: no new paragraph}
\noindent
See also \citew[\page 408--409]{Sag2007a} and \citew{KaySag2009} for the obligatory control of possessive determiners in \ili{English} expressions such as \emph{keep one's cool}, \emph{lose one's temper}, with an \xarg feature on nouns and NPs:
\begin{exe}
\ex \begin{xlist}
\ex John lost his / * her temper.
\ex Mary lost * his / her temper.
\end{xlist}
\end{exe}

This coindexing can also be extended to some subject-raising verbs such as \emph{look like}, which
has been called ``copy raising'' (\citealp{Rogers74a-u,Hornstein99a-u} a.o.): \emph{look like} takes a finite
complement with an overt subject, and this pronominal subject must be coindexed with the matrix
subject;
%\itdobl{Stefan: And be a pronoun? ``Peter looks like Peter is tired.''?\\
%Anne: maybe ok:  Peter looks like his wallet has been stolen?\\
%Stefan: You sent me these:  These would not work with your analysis since the pronoun is not the
%XARG. Right?} 
it is a raising predicate, as shown by the possibility of the expletive \emph{there}:

\eal
\ex Peter looks like he's tired. / \# Mary is coming.
\ex There looks like there's going to be a storm.\footnote{
From \citet[\page 407]{Sag2007a}. The full picture is a bit more complicated, since the bound pronoun in
the finite complement is not necessarily the subject, as pointed out to me by Philip Miller
(p.c. 2021), on the basis of naturally occurring examples such as: 
\eal
\ex He looks like someone took him out back and beat the crap out of him. (COCA)
\ex He is the shiniest person I have ever seen. He looks like his mother polished him before she sent
him off to school. (COCA) 
\zl
I do not go further on this particular case.
}
\zl


%the verb \emph{look like} can thus have the subject of its sentential complement coindexed with its own subject:

%\itd{Stefan: Just an interesting thought: This would not work in SBCG since whole signs are  shared. So you have \phon available in \ibox{1}.} 
%\ea
%Lexical description of \emph{look (like)}:\\
%\avm{[ arg-st <\NPi, S[marking & like\\
%                    xarg & [cont & [\type{pron}\\
% ind & $_i$]] ] > ]}\z
This bears some similarity with \ili{English} tag questions: the  subject of the tag question must be
pronominal and coindexed with that of the matrix clause (see \citealt{BF99a}, and this chapter Section~\ref{sec-auxiliaries-as-raising-verbs} on auxiliary verbs):
%\itd{Stefan: This would not work since the subject of the main clause has semantics if it is a full NP that is different from the semantics of a pronoun. So identification would fail. Ivan argued that agreement must be semantic. So syntactic features may differ as well.}
\eal \label{extag}
\ex Paul left, didn't he?\label{ex-paul-left-didnt-he}
\ex It rained yesterday, didn't it?
\zl

\noindent
To account for such cases, the types for subject-raising and subject-control verb lexemes in
(\ref{rais-1}) and (\ref{subj-contr-lx}) can thus be revised  as follows. 
Assuming a tripartition of \type{index} with \type{referential}, \type{there} and \type{it}
\citep[\page 138]{ps2}, 
% the only difference being that the \textsc{index} of the subject of control verbs must
% be a referential NP:
the only difference between subject raising and subject control being that the \textsc{index} of the subject of control verbs must be a
referential NP:\footnote{This coindexing follows from the fact that control verbs assign a semantic role to their
subject and the subject is coindexed with the subject of the controlled verb. Some authors have independently argued that some verbs have either a control-like or a raising-like behavior depending on the agentivity of their subject; see \citew{Perlmutter70} for \ili{English} aspectual verbs (\emph{begin}, \emph{stop}) and \citew[56]{Ruwet1991a-u} for \ili{French} verbs like \emph{menacer} (`threaten') and \emph{promettre} (`promise').}
\eal
\ex \type{sr-v-lx}  \impl \avm{ [\argst < XP$_i$, \ldots, [xarg & [ind & $i$] ] > ] } 
%\ex \type{sc-v-lx} \impl \avm{ [\argst < \NPi, \ldots, [xarg & [ind & $i$ \type{referential} ]  ] >
%] }
\ex \type{sc-v-lx} \impl \avm{ [\argst < \NPi, \ldots, [xarg & [ind & $i$ ]  ] > ] }
\zl
Note that this approach does not for those languages allowing subjectless verbs (see example (\ref{ex-raising-with-subjectless-verbs})).
% \itdobl{Stefan: But this does not work for all languages. It sounds as if this was asssumed to be the
%   final version now. Anne: are you thinking of German subjectless verbs again? it says 'to account
%   for the cases above'\\
% Stefan: But this is an overview chapter on raising. It is possible to give a general account for
% raising that works for all languages. The fact that English has to have a subject is independent
% from raising and should not be mixed in lexical constraints.}
% \itdobl{Stefan: The fact that the xarg of the controlled verb has to be referential follows from the
%   fact that the control verb assigns a role to the controlee. It does not have to be stated in the
%   lexical item as constraint on the embedded verb.}

\section{Copular constructions}
\label{sec-copular-constructions}

Copular verbs can also be considered as ``raising'' verbs \citep[\page 106]{Chomsky81a}.  While
attributive adjectives are adjoined to N or NP, predicative adjectives are complements of copular
verbs and share their subject with these verbs. Like raising verbs
(Section~\ref{sec-more-on-raising-verbs}), copular verbs come in two varieties: subject copular
verbs (\word{be}, \word{get}, \word{seem}), and object copular verbs (\word{consider},
\word{prove}, \word{expect}).

Let us review a few properties of copular constructions.
The adjective selects for the verb's subject or object: \word{likely} may select a nominal or a
sentential argument, while \word{expensive} only takes a nominal argument. As a result, \word{seem}
combined with \word{expensive} only takes a nominal subject, and \word{consider} combined with the
same adjective only takes a nominal object. 


\begin{exe}
\ex \label{storm}
\begin{xlist}
\ex{} [A storm] / [That it will rain] seems likely.
\ex{} [This trip] / * [That he comes] seems expensive.
\end{xlist}
\ex \begin{xlist}
\ex 	I consider [a storm] likely / likely [that it will rain].
\ex 	I consider [this trip] expensive/ * expensive [that he comes].
\end{xlist}	
\end{exe}


A copular verb thus takes any subject (or object) allowed by the predicate: \emph{be} can take a PP
subject in \ili{English} with a proper predicate like `a good place to hide' (\ref{under2}), and \emph{werden} takes no subject when combined with a
subjectless predicate like \emph{schlecht} `sick' in \ili{German} (\ref{german3}):

\eal
\ex{}[Under the bed] is a good place to hide \label{under2}
\ex
\label{german3} 
\gll Ihm        wurde schlecht.\footnotemark\\
     him.\DAT{} got   sick\\\hfill(\ili{German})
\footnotetext{From \citet[\page 72]{Mueller2002b}.}
\glt `He got sick.'
\zl

\noindent
In \ili{English}, \word{be} also has the properties of an auxiliary; see Section~\ref{control-sec-copula-verbs}.

\subsection{The problems with a small clause analysis}

To account for the above properties, Transformational Grammar since \citet{Stowell1983}\addpages and
\citet{Chomsky1986}\addpages has proposed a clausal or \emph{small clause} analysis: the second predicate
(the predicative adjective) heads a (small) clause; its subject raises to the subject position of the
matrix verb (\ref{rais-transformational}) or stays in its embedded position and receives accusative case from
the matrix verb via exceptional case marking\is{Exceptional Case Marking}, ECM, as seen above (\ref{ecm}).


\eal
\ex
\label{rais-transformational}
{}[\sub{NP} e] be [\sub{S} John sick] $\leadsto$  [\sub{NP} John ] is  [\sub{S} \st{John} sick]
\ex
\label{ecm}
We consider [\sub{S} John sick].
\zl

It is true that the adjective may combine with its subject to form a verbless sentence; this happens
in African American Vernacular English (AAVE)\il{English!African American Vernacular} \citep{Bender2001a}\addpages, in \ili{French} \citep{Laurens2008}\addpages, in creole languages
\citep[134]{HenriandAbeille2007}, in \ili{Slavic} languages \citep{Zec87a-u}\addpages and in \ili{Semitic} languages.
%(see \citealp{Alqurashi:Borsley:14}\addpages), among others. 

\ea
\gll Magnifique ce chapeau !\\
     beautiful this hat\\\hfill{(\ili{French})}
\glt `What a beautiful hat!'
\z

\noindent
But this does not entail that copular verbs like \emph{be} take a sentential complement. 


%a problem for the raising principle? In \ili{French}, and other \ili{Romance} languages (Abeillé and Godard 2000), the predicate can be pronominalized as a complement:\\

%\begin{exe}
%\ex \gll Paul est malade / médecin / en forme.
%Paul is sick / a doctor / in a good shape\\
%\ex \gll Paul l'est.
%Paul it is\\
%\glt Paul is so
%\end{exe}

Several arguments can be presented against a (small) clause
analysis. The putative sentential source is sometimes attested (\ref{cons1}) but more often
ungrammatical:
%\itdopt{Stefan: Example from ps94? Page numbers? Anne: no they are not}
	
\eal
\ex[]{
John gets / becomes sick.
}
\ex[*]{
It gets / becomes that John is sick.
}
\ex[]{
\label{cons1}
John considers Lou a friend / that Lou is a friend.
}
\ex[]{
Paul regards Mary as crazy.
}
\ex[*]{
Paul regards that Mary is crazy.
}
\zl

	
When a clausal complement is possible, its properties differ from those of the putative small
clause. Pseudo-clefting shows that \textit{Lou a friend} is not a constituent in
(\ref{consider}). (\ref{consider}) does not mean exactly the same as (\ref{consider2}). Furthermore, as pointed out by \citet{Williams83a}, the embedded predicate can be
questioned independently of the first NP, which would be very unusual if it were the head of a small
clause (\ref{w}). 

\ealnoraggedright
\ex[]{
We consider Lou a friend.\label{consider}
}
\ex[*]{
What we consider is Lou a friend.
}
\ex[]{
We consider [that Lou is a friend]. \label{consider2}
}
\ex[]{
What we consider is [that Lou is a friend].
}
\ex[]{
What do you consider Lou?
}\label{w}
\zl

Following \citet[420--423]{Bresnan1982},
\citet[113]{PollardandSag1994} also show that Heavy-NP shift
applies to the putative subject of the small clause, exactly as it applies to the first complement
of a ditransitive verb:

\eal
\ex We would consider [any candidate] [acceptable].
\ex We would consider [acceptable]  [any candidate who supports the proposed amendment].
\ex I showed [all the cookies] [to Dana].
\ex I showed [to Dana]  [all the cookies that could be made from betel nuts and molasses].  
\zl

% \itd{Stefan: Why quotes around subject? AA: becasue it is a subject for small clause analysis; in
%   hpsg it is a member of the Subj list of the adjective, realized as a complement of the verb.\\
% Stefan: Yes, but it is the subject of the adjective. We would say: ``The subject of the adjective is
% represented under \subj, wouldn't we?}
Indeed, the ``subject'' of the adjective with object"=raising verbs has all the properties of an
object: it bears accusative case and it can be the subject of a passive:

\eal
\ex We consider him / * he guilty.
\ex We consider that he / * him is guilty.
\ex He was proven guilty (by the jury).	
\zl
	

Furthermore, the matrix verb may select the head of the putative small clause, which is not the case
with verbs taking a clausal complement, and which violates the 
%\itd{Stefan: ``locality of subcategorization'' is not explained. Why is it important? What is it? cite Sag with various proposals.}
locality of subcategorization \parencites[\page 102]{PollardandSag1994}{Sag2007a}. The
verb \word{expect} takes a predicative adjective but not a preposition or a nominal predicate (\ref{ex-expect});
\word{get} selects a predicative adjective or a preposition (\ref{ex-get}), but not a predicative nominal; and
\word{prove} selects a predicative noun or adjective but not a preposition (\ref{ex-prove}).


\eal
\label{ex-expect}
\ex I expect that man (to be) dead  by tomorrow. \citep[\page 102]{PollardandSag1994}
\ex I expect that island *(to be) off the route. (p.\,103)
\ex I expect that island *(to be) a good vacation spot. (p.\,103)
\zl
\ea
\label{ex-get}
John got political / * a success. (p.\,105)	
\z
\eal
\label{ex-prove}
\ex Tracy proved the theorem (to be) false. (p.\,100)
\ex I proved the weapon *(to be) in his possession.	(p.\,101)
\zl
	


\subsection{An HPSG analysis of copular verbs}
\label{control-sec-copula-verbs}
	
Copular verbs such as \word{be} or \word{consider} are analyzed as subtypes of subject-raising verbs
and object"=raising verbs respectively and hence, the constraints in (\ref{rsg}) apply. They share their subject (or object) with the
unexpressed subject of their predicative complement. Instead of taking a VP complement, they take a
predicative complement (\prd $+$), which they may select the category of.  We can thus define a general type for verbs taking a predicative complement as in (\ref{ex-subj-pred-v}), 
%with \type{list} meaning any list  of arguments before the predicative complement, 
and then two subtypes of verbs taking a predicative complement: \type{s-pred-v-lx}  for verbs like \word{be}, which also inherit from subject-raising verbs, and \type{o-pred-v-lx} for verbs like \word{consider}, which also inherit from object-raising verbs.

%\itd{Stefan: This is unnecessary if these types are subtypes of (\ref{rsg}). The only thing one would have to say is that the last element of the \argstl of respective heads have to be  \prd+. One type would be sufficient.}
%\itd{Stefan: pred-v or prd-lx?}
\eal
\label{ex-subj-pred-v}
\type{pred-lx} \impl  \avm{ [arg-st & \upshape \sliste{ \ldots, [\prd $+$] } ]}
%\ex \type{s-pred-v-lx} \impl \type{sr-v-lx} \& \type{pred-lx}
%\ex \type{o-pred-v-lx} \impl  \type{or-v-lx} \& \type{pred-lx}
\zl
%\itdobl{Stefan: b and c are part of the signature and do not have to be given as separate implicational constraints.\\Anne: but I dont have a signature here, so I need to say this somehow so that the reader can understand.\\Stefan: Yes, but you should not say this in technical terms since otherwise the reader may think
%that this is the technically correct way of specifying things.}

A copular verb like \word{be} or \word{seem} does not assign any semantic role to its subject, while
verbs like \word{consider} or \word{expect} do not assign any semantic role to their object. For
more details, see \citew[Chapter~3]{PollardandSag1994},
\textcites[Section~2.2.7]{Mueller2002b}[]{MuellerPredication} and \citew{VanEynde2015}.  The lexical descriptions for predicative \word{seem} and
predicative \word{consider} inherit from the \type{s-pred-v-lx} type and \type{o-pred-v-lx}
type respectively, and are simplified as shown below. 
%\inlinetodostefan{Stefan: fix these lexical items. Raise everything? Use append for consider. Should this be \argst? it is easier to show a word that is later used in the figure ; Anne: I use append in the types, but for \ili{English} verbs, we can assume that there are no  subjectless  predicates.\\  Stefan: But this assumption is not in the theory. Hence you have to use append to  reflect what you  stated in the types, don't you?\\
%Anne: arg-st added: the types are more general and the words are more specified.\\ Stefan: Yes, but then you have to explain it. Say something about additional constraints and where they come from.}

%\itd{Stefan: Doesn't \emph{seem} take an optional PP? It also may add an argument to the  \type{seem-rel}. I could refer to the lexical item from the binding chapter where I use  \emph{seem} with PP.}
As in Section~\ref{control:sec-HPSG-anaylsis-of-raising}, we ignore here a possible PP complement
(\textit{John seems smart to me}). With the assumption that the \subjl contains exactly one
  element in English, the following lexical descriptions result:
% (see
%\crossrefchapterw{binding}\itdobl{for Stefan: check and add section} ).
%\itdobl{Stefan: I would add the following line here: ``With the assumption that the \subj list of English verbs contains exactly one element, the following lexical items for the words \emph{seem} and \emph{consider} are licensed:'' And the types should be removed. Anne: I need the types to distinguish them from other instances}
\eas
Lexical description of \emph{seem} (\type{s-pred-v-lx}):\\
\avm{
[subj  & < \1 > \\
comps & < \2 [head & [prd & $+$] \\
	      subj & < \1 > \\
	      cont & [ind &  \3]] > \\
arg-st & <\1, \2> \\
cont &	[%ind & s \\
	 rels & < [\type*{seem-rel} 
	           soa & \3] > ] ]
}
\zs

\eas
Lexical description of \emph{consider} (\type{o-pred-v-lx}):\\
\avm{
[subj & < \1 \NPi > \\
comps & < \2, \3 [head &	[prd & $+$] \\
	          subj & < \2 > \\
		  cont & [ ind &  \4] ]> \\
arg-st & < \1, \2, \3 > \\
cont   & [%ind & s \\
	  rels & < [\type*{consider-rel} 
	            exp & $i$ \\
		    soa & \4] > ] ]
}
\zs

	
The subject of \word{seem} is unspecified: it can be any category selected by the predicative
complement; the same holds for the first complement of \word{consider}
% \itd{Stefan: Say something about the fact that it has to be exactly one element. Where does this information come from? Anne: this is not important here}
(see examples in (\ref{storm}) above).  \word{Consider} selects
a subject and two complements, but only takes two semantic arguments: one corresponding to its
subject, and one corresponding to its predicative complement. It does not assign a semantic role to
its non"=predicative complement.

Let us take the example \textit{Paul seems happy}. As a predicative adjective, \word{happy} has a
\headf [\prd $+$] and its \subjf is not the empty list: it subcategorizes for a nominal subject and
assigns a semantic role to it, as shown in (\ref{happy2}).
	
\eas
\label{happy2}%
Lexical description of \type{happy}:\\
\avm{
[\phon <happy> \\
head &	[\type*{adj}
	 	prd & $+$] \\
subj & <\NPi> \\
comps & < > \\
cont & [%ind & s \\
rels &	< [\type*{happy-rel}
		exp & $i$] > ] ]
}
\zs

In the trees in the Figures~\ref{fig-happy} and~\ref{fig-cons}, the \subjf of \word{happy} is
shared with the \subjf of \word{seem} and the first element of the \comps list of
\word{consider}.\footnote{In what follows, I ignore adjectives taking complements. As noted in Section~\ref{sec-distinction-raising-control}, adjectives may take a non"=finite VP complement and fall under a control or raising type: as a subject-raising adjective, \word{likely} shares the \textsc{synsem} value of its subject with the expected subject of its VP complement; as a subject-control adjective, \word{eager} coindexes both subjects.
Such adjectives thus inherit from \type{subj-rsg"=lexeme} and \type{subj-control"=lexeme},
respectively, as well as from \type{adjective"=lexeme}. In some languages, copular constructions are
complex predicates, which means that the copular verb inherits the complements of the adjective as
well; see \citew{AG2001b-u} and \crossrefchapterw[Section~\ref{cp:sec-copula-romance} and~\ref{cp:sec-copula-German}]{complex-predicates}.}
%\inlinetodostefan{Stefan: Is \ibox{1} a list or an element of a list? If the complete \subjv is supposed to be raised, my fix is technically not correct. AA seems ok to me, the numbers are synsem descriptions, these figures are similar to the previous ones. Stefan: In the type constraints you are sharing the whole SUBJ list. Here you are sharing elements. I think it should
%look like in my Figure~\ref{fig-happy-fixed}.\\
%AA: I prefer my figure 7 since it is the same as figure 2.\\
%Stefan: But this is not a valid reason, if it is wrong. AA: it is not wrong}


\begin{figure}
\begin{forest}
[
\avm{
[S\\
\phon <Paul seems happy> \\
subj & < > \\
comps & < > ]
}
	[
	\avm{
	[NP\\
	\phon <Paul> \\
	synsem & \1 ]
	}
	]
	[
	\avm{
	[VP\\
	\phon <seems happy> \\
	subj & <\1> \\
	comps & < > ]
	}
		[
		\avm{
        [V\\
        \phon <seems> \\
		subj & <\1> \\
		comps & <\2 > ]
		}
		]
		[
		\avm{
        [AP\\
        \phon <happy> \\
		synsem & \2[subj & <\1>] ]	
		}
		]
	]
]
\end{forest}
\caption{\label{fig-happy}A sentence with an intransitive copular verb}
\end{figure}

\begin{figure}
\begin{forest}
[
\avm{
[S\\
\phon <Mary considers Paul happy> \\
      subj & < > \\
      comps & < > ]		
}
	[
	\avm{
	[NP\\
	\phon <Mary> \\
	synsem & \1 ]
	}
	]
	[
	\avm{
	[VP\\
	\phon <considers Paul happy> \\
	subj & <\1> \\
	comps & < > ]
	}
		[
		\avm{
		[V\\
		\phon <considers> \\
		subj  & <\1> \\
		comps & <\2, \3> ]		
		}
		]
		[
		\avm{
		[NP\\
		\phon <Paul> \\
			synsem & \2 ]
		}
		]
		[
		\avm{
		[AP\\
		\phon <happy> \\
		synsem & \3[subj & <\2> ] ]	
		}
		]
	]
]
\end{forest}	
\caption{\label{fig-cons}A sentence with a transitive copular verb}
\end{figure}

\citet[\page 133]{PollardandSag1994} mention a few verbs taking a predicative complement which can be
considered as control verbs. A verb like \word{feel} selects a nominal subject and assigns a
semantic role to it.

\begin{exe}
\ex John feels tired / at ease.
\end{exe}

\noindent
It inherits from the subject-control-verb type (\ref{cont}); its lexical description is given in (\mex{1}):

\eas
\type{feel} (\type{sc-v-lx}):\\
\avm{
	[subj & <\1 \NPi > \\
	comps & <\2	[head  &	[prd & $+$] \\
			 subj  &	<[ind & $i$]> \\
			 cont & [ind &	 \3] ]>\\
	arg-st & <\1, \2> \\
	cont &	[%ind & s \\
			rels &	< [\type*{feel-rel}
			exp & $i$ \\
			soa & \3] > ] ]
}
\zs


\subsection{Copular verbs in Mauritian}

As shown by \citet{HenriandLaurens2011}, and as was the case for other raising verbs (see Section~\ref{sec-maurit}), \ili{Mauritian} data
provide a strong argument in favor of a non"=clausal analysis.
%\itd{Stefan: Where exactly did you show this? What you suggested in (\ref{ex-pans-Maritian}) was an  embedded clause.\\
%Anne: no, see also ex (\ref{sf-constraint}).\\
%Stefan: (\ref{ex-pans-Maritian}) allows for both complete sentences and VPs. (\ref{sf-constraint}) says that there has to be an argument.Anne: a non sentential cplt; for copular verbs it is explained below anyway} 
A copular verb takes a short form before a
predicative complement and
a long form before a clausal one. Despite the lack of inflection on the embedded verb and the
possibility of subject pro-drop, clausal complements differ from non"=clausal complements by the
following properties: they do not trigger the matrix verb short form, they may be introduced by
the complementizer \word{ki} and their subject is a weak pronoun (\word{mo} `I', \word{to} `you'). On
the other hand, a VP or AP complement cannot be introduced by \word{ki}, and an NP complement must
be realized as a strong pronoun
(\word{mwa} `me', \word{twa} `you'). So \word{malad} `sick' is an adjectival complement in
(\ref{ex-anne1}), (\ref{ex-anne2}) and (\ref{ex-anne4}) and not a small clause and \emph{trouv} `find' takes
two complements in (\ref{ex-anne2}) and (\ref{ex-anne4}) and \emph{trouve} `find' one clausal complement in (\ref{ex-anne3}). See Section~\ref{sec-maurit}
above for the alternation between verb short form (\textsc{sf}) and long form (\textsc{lf}).

\eal
\ex 
\gll Mari ti res  malad.\\
     Mari \textsc{pst} remain.\textsc{sf} sick\\\hfill\citep[\page 198]{HenriandLaurens2011}
\glt `Mari remained sick.' \label{ex-anne1}

\ex 
\gll Mari trouv  so mama malad\\
     Mari find.\textsc{sf} \POSS{} mother sick\\
\glt `Mari finds her mother sick.' \label{ex-anne2}

\ex 
\gll Mari trouve (ki) mo malad\\
     Mari find.\textsc{lf} \hphantom{(}that 1\SG.\textsc{wk} sick\\
\glt `Mari finds that I am sick.' \label{ex-anne3}

\ex 
\gll Mari trouv              mwa               malad\\
     Mari find.\textsc{sf}  1\SG.\textsc{str} sick\\
\glt `Mari finds me sick.' \label{ex-anne4}
\zl

%\itd{Stefan: The typo is in the original, but ``clause'' is singular and ``small clauses'' is plural.}
\citet[\page 218]{HenriandLaurens2011} conclude that ``Complements of raising and control verbs
systematically pattern with non-clausal phrases such as NPs or PPs. This kind of evidence is seldom
available in world's languages because heads are not usually sensitive to the properties of their
complements. The analysis as clause or small clauses is also problematic because of the existence of
genuine verbless clauses in \ili{Mauritian} which pattern with verbal clauses and not with
complements of raising and control verbs''.


\section{Auxiliaries as raising verbs}
\label{sec-auxiliaries-as-raising-verbs}

%\itdobl{Stefan: Something is wrong in the following paragraph. Better: ``not considered to have a
 % special part of speech but are verbs with the head property'' }
Following \citet{Ross69a-u}, \citet{Gazdaretal1982} and \citet{Sagetal2020}, \word{be}, \word{do}, \word{have} and
modals (\eg \word{can}, \word{should}) in HPSG are not considered to have a special part of speech
(\type{Aux} or \type{Infl})\footnote{Having Infl as a syntactic category and sentences defined as IP does not account for languages without inflection, nor for verbless sentences; see for example \citew{Laurens2008}.} but are verbs with the head feature [\textsc{aux} $+$].
%(\ref{ex-head-value-of-aux-elements}):
%\begin{exe}
%\ex \label{ex-head-value-of-aux-elements}
%  \type{aux-verb} \impl 
  %\type{sr-v-lx} \& 
%\avm{[head &	[aux & $+$] ]}
 %\end{exe}
%\itdobl{Stefan: I said this already in some discussion, but the subtype relation is part of the type hierarchy (the signature, see formal background). So \type{sr-v-lx} should not be stated here. You can mention the inheritance relation in prose.}
 
 \ili{English} auxiliaries take VP (or XP) complements and neither impose categorial restrictions on their subject nor assign it a semantic role,
 %\itd{Stefan:  Well, they do. After all the subject is raised and then the auxiliary selects it. It does not assign it a semantic role. Anne: ok they may do for agreement}
   just like
 other subject-raising verbs. They are thus compatible with non"=referential subjects, such as
 meteorological \word{it} and existential \textit{there}. They select the verb form of their
 non"=finite complements: \textit{have} selects a past participle, \textit{be} a gerund-participle and
 \textit{can} and \textit{will} a bare form.

	
\begin{exe}
\ex \begin{xlist}
\ex Paul has left.
\ex Paul is leaving.
\ex Paul can leave.
\ex It will rain.
\ex There can be a riot.
\end{xlist}	
\end{exe}

In this approach, \ili{English} auxiliaries are subtypes of subject-raising verbs and thus take a VP (or
XP) complement and share their subject with the unexpressed subject of the non"=finite verb (see Section~\ref{control:sec-HPSG-anaylsis-of-raising}).\footnote{ \emph{Be} is an auxiliary and a subject-raising verb with a \prd$+$ complement (see
  Section~\ref{control-sec-copula-verbs} above) or a gerund-participle VP complement,
  different from the identity \emph{be} which is not a raising verb (see \citealp{VanEynde2008a} and
  \citealp{MuellerPredication} on predication). A verb like \emph{dare}, shown to be an auxiliary by
  its postnominal negation, is not a raising verb but a subject-control verb:
\eal
\ex[]{
He is lazy and sleeping.
}
\ex[]{
I dare not be late.
}
\ex[\#]{
It will not dare rain.
}
\zllast
}
The lexical descriptions for the auxiliaries \word{will} and \word{have} are as follows: 

\ea
Lexical description of \emph{will} (\type{sr-v-lx}):\\*
\avm{
[head &	[aux &  $+$] \\
 subj & < \1 > \\
 comps & < \2 VP[head & [vform & bse] \\
                 subj & < \1 > \\
	         cont & [ind & \3] ]> \\
	arg-st & < \1, \2 >\\
	cont &	[ind & s \\
		 rels & < [\type*{future-rel}
		           soa & \3] > ] ]
}
%\itdobl{Stefan: Remove type? Anne I need to say it is the aux, not some other lxm}
\z
\eas
Lexical description of \emph{have} (\type{sr-v-lx}):\\*
\avm{
	[head &	[aux & $+$] \\
	subj & <\1> \\
	comps & <\2 VP	[head &	[vform & past-part] \\
			 subj & <\1> \\
			 cont &	[ind & \3] ]>\\
	arg-st & <\1, \2>\\
	cont &	[ind & s \\
			rels & < [\type*{perfect-rel}
			          soa & \3] > ] ]
}
%\itdobl{Stefan: Remove type?}
\zs

To account for their NICE (\isi{negation}, \isi{inversion}, \isi{contraction} (\emph{isn't}, \emph{won't}), \isi{ellipsis}) properties, \citet{KS2002a} % ref to whole paper OK
use a binary head feature \aux, so that only [\aux $+$] verbs may allow for subject inversion
(\ref{inv}), sentential negation (\ref{neg}), contraction or VP ellipsis (\ref{ell}). See
\crossrefchapterw[Section~\ref{sec-head-movement-vs-flat}]{order} on subject inversion,
\crossrefchapterw[Section~\ref{sec-sentential-negation}]{negation} on negation and
\crossrefchapterw[Section~\ref{sec-analyses-of-pred-ellipsis}]{ellipsis} on post-auxiliary ellipsis.\footnote{Copular \word{be} has
  the NICE properties (\textit{Is John happy?}); it is an auxiliary verb with [\prd $+$]
  complement. Since \emph{to} allows for VP ellipsis, it is also analyzed as an auxiliary verb:
  \emph{John promised to work and he started to}. See \citew*[600]{GPS82a-u} and \citew{Levine2012a-u}.} 

\eal
\ex[]{
Is Paul working? \label{inv}
}
\ex[*]{
Keeps Paul working?
}
\ex[]{
Paul is (probably) not working.\label{neg}
}
\ex[*]{
Paul keeps (probably) not working.
}
\ex[]{
John promised to come and he will. \label{ell}
}
\ex[*]{
John promised to come and he seems.
}
\zl

\noindent
Subject raising verbs such as \word{seem}, \word{keep} or \word{start} are [\aux $-$].

\citet{Sagetal2020} revised this analysis and proposed a new analysis couched in \sbcg (\citealp{Sag2012a}; see also \crossrefchapteralp[Section~\ref{sec-sbcg}]{cxg}). The descriptions used below were translated into the feature geometry of Constructional HPSG \citep{Sag97a}, which is used in this volume. In their approach, the head feature \aux is both lexical and constructional: the constructions restricted to auxiliaries require their head to be [\aux $+$], while the constructions available for all verbs are [\aux $-$]. In this approach, non"=auxiliary verbs are lexically specified as [\aux $-$] and [\inv $-$].

%\begin{exe} \ex \type{non-auxiliary-verb} \impl
%\avm{	[head &	[aux & $-$ \\
%			inv & $-$ ] ]} \end{exe}

 Auxiliary verbs, on the other hand, are unspecified for the feature \aux and are contextually specified, except for unstressed \word{do}, which is [\aux $+$] and must occur in constructions restricted to auxiliaries.

\eal
\settowidth\jamwidth{(tritra trulla la. Nobody will ever read this.)}
\ex[]{
Paul is working. \jambox{[\aux $-$]}
}
\ex[]{
Is Paul working? \jambox{[\aux $+$]}
} \label{inv1}
\ex[*]{
John does work. \jambox{[\aux $-$]}
}
\ex[]{
Does John work? \jambox{[\aux $+$]}
}\label{inv2}
\zl

\subsection{Subject inversion and English auxiliaries}

Subject inversion is handled by a subtype of head-subject-complement phrase, which is independently
needed for verb initial languages like \ili{Welsh} \parencites[\page 285]{Borsley99c-u}[\page 410]{SWB2003a}.\footnote{As
  noted in \crossrefchapterw[\page \pageref{page-properties:aux-inversion}]{properties}, in some HPSG work, \eg \citew[409--414]{SWB2003a},
  examples like (\ref{inv1}) and (\ref{inv2}) are analyzed as involving an auxiliary verb with two
  complements and no subject. This approach has no need for an additional phrase type, but it
  requires an alternative valence description for auxiliary verbs.} It is a specific (non"=binary)
construction,\is{branching!non-binary} of which other constructions such as \type{polar-interrogative-clause} are subtypes,
and whose head must be [\textsc{inv} $+$].
%\itd{Stefan: Shouldn't the \textsc{inv} constraint be in (\mex{1})?}
\ea
\type{initial-aux-ph} \impl\\
%\avm{
%[subj     & < > \\
% comps    & < > \\
% head-dtr & \1[head & [aux & $+$ \\ 
 %                      inv & $+$]\\
%	       subj & \2\\
%               comps & \3 \ldots{} \tag{n} ] \\
% dtrs & < \1 > \+ < synsem \2> \+ < synsem \3> \ldots{} \+ < synsem \tag{n}> ]
%}
%\itdobl{Stefan: This is wrong. \subj is a list of synsems. I do not know how to write this up properly. With the assumption that there has to be at least one element in SUBJ, it would be the following. COMPS is a list as well. So this has to be changed.}
\avm{
[subj     & < > \\
 comps    & < > \\
 head-dtr & \1[head & [aux & $+$ \\ 
                       inv & $+$]\\
	       subj & < \2 >\\
               comps & < \3, \ldots, \tag{n} > ] \\
 dtrs & < \1, [synsem \2 ], [synsem \3], \ldots{}, [synsem \tag{n}] > ]
}
\z
%I use \synsem for \type{synsem} features.\\
Most auxiliaries are lexically unspecified for the feature \textsc{inv} and allow for both constructions
(non"=inverted and inverted), while the first person \word{aren't} is obligatorily inverted (lexically
marked as [\textsc{inv} $+$]) and the modal \word{better} obligatorily non"=inverted (lexically marked
as [\textsc{inv} $-$]):

\eal
\ex[]{
Aren't I dreaming?
}
\ex[*]{
I aren't dreaming.
}
\ex[]{
We better be careful.
}
\ex[*]{
Better we be careful?
}
\zl

As for tag questions (\emph{Paul left, didn't he?} (\ref{ex-paul-left-didnt-he})), they can be defined as special adjuncts, coindexing their subject with that of the sentence they adjoin to, using the \xarg feature (see above Section~\ref{section-xarg}).
%\itd{Anne:I'm simplifying with NEG +, if you could replace by POL 1 in the modified clause and POL -1 in the tag-aux, it would be better.Stefan: Oh, have there been solutions to this? I made one up. I also added a line explaining what it does.}
\ea
\type{tag-aux-lx} \impl\\
\avm{
[head & [inv & $+$\\ 
          tense & \2\\
          pol & \rel{not}(\1)\\
         mod & [xarg & $i$\\
                tense & \2\\
                pol & \1]]\\
 subj     & < cont & [\type{pron}\\
 ind & $_i$]> \\
 comps    & < > \\]
}
\z
\rel{not} is a function that returns `$+$' for the input `$-$' and `$-$' for the input `$+$'. I use coindexing of \textsc{tense} to ensure time concordance between the main verb and the tag auxiliary. \textsc{pron} denotes a subject with a pronominal content.

\subsection{English auxiliaries and ellipsis}

While the distinction is not always easy to make between VP ellipsis (\emph{Paul can}) and null complement anaphora
(\emph{Paul tried}), \citeauthor{Sagetal2020} observe that certain elliptical constructions are
restricted to auxiliaries, for example pseudogapping (see also
\crossrefchapteralp[Section~\ref{ellipsis-sec-Predicate-ellipsis-and-argument-ellipsis}]{ellipsis} and
\citealt{Miller2014a-u}).

\eal
\ex[]{
John can eat more pizza than Mary can tacos. \citep[ex. 52]{Sag2020a}
}
\ex[]{Larry might read the short story, but he won’t the play.
}
\ex[*]{
Ann seems to buy more bagels than Sue seems cupcakes.
}
\zl

\noindent
This could be captured by having the relevant auxiliaries optionally inherit the complements of their verbal complement.\footnote{See \citew{KimandSag2002} for a comparison of \ili{French} and \ili{English} auxilaries and \citew{AG2002b-u} for a thorough analysis of \ili{French} auxiliaries as ``generalized'' raising verbs, inheriting not only the subject but also any complement from the past participle. Such generalized raising was first suggested by \citet{HN89a,HN94a} for \ili{German} and has been adopted since in various analyses of verbal complexes in \ili{German} \citep{Kiss95a,Meurers2000b,Kathol2001a,Mueller99a,Mueller2002b}, \ili{Dutch} \citep{BvN98a} and \ili{Persian} \citep[Section~4]{MuellerPersian}. See also \crossrefchaptert{complex-predicates}.}
An additional lexical description of \emph{will} with complement inheritance could be the following, using the non-canonical \type{synsem} type \type{pro} for the unexpressed VP:
\ea
Lexical description of elliptical \emph{will} (VPE or pseudogapping):\\
\avm{
[subj & < \1 > \\
comps & \2 \\
arg-st & < \1, VP[\type*{pro}\\
                  subj & < \1 >\\
                  comps & \2 ] > \+ \2 ]
}
\z
If the list \ibox{2} is empty, this entry deals with VP ellipsis (\emph{I will}), if it is not empty, it deals with pseudogapping (\emph{I will the play}).

As observed by \citet{ArnoldandBorsley2008}, auxiliaries can be stranded in certain non-restrictive
relative clauses such as (\ref{aux1}), whereas no such possibility is open to non-auxiliary verbs
(\ref{nonaux}) (see also \crossrefchapteralt[\page \pageref{page-relative-clauses:stranded-aux}]{relative-clauses}):

\eal
\ex[]{
Kim was singing, which Lee wasn't. \label{aux1}
}
\ex[*]{
Kim tried to impress Lee, which Sandy didn't try. \citep[ex. 54a]{Sag2020a}\label{nonaux}
}
\zl

The HPSG analysis sketched here captures a very wide range of facts, and expresses both generalizations (\ili{English} auxiliaries are subtypes of subject-raising verbs) and lexical idiosyncrasies (copula \emph{be} takes non"=verbal complements, first person \emph{aren't} triggers obligatory inversion, etc.).


	
\section{Conclusion}

Complements of ``raising'' and control verbs have been either analyzed as clauses \citep{Chomsky81a}\addpages
or small clauses (Stowell \citeyear{Stowell81a-u,Stowell1983}\addpages) in Mainstream Generative Grammar.  As in LFG
\citep{Bresnan1982}, ``raising'' and control predicates are analyzed as taking non-clausal open
complements in HPSG \citep[Chapter~3]{PollardandSag1994}, with sharing or coindexing the (unexpressed) subject
of the embedded predicate with their own subject (or object). This leads to a more accurate analysis
of ``object"=raising'' verbs as ditransitive, without the need for an exceptional case marking
device. This analysis naturally extends to pro-drop and ergative languages; it also makes correct
empirical predictions for languages that mark clausal complementation differently from VP
complementation. A rich hierarchy of lexical types enables verbs and adjectives taking non"=finite
or predicative complements to inherit from a raising type or a control type. The Raising Principle
prevents any other kind of non"=canonical linking between semantic argument and syntactic
argument. A semantics-based control theory predicts which predicates are subject-control and which
object-control. The ``subject-raising'' analysis has been successfully extended to copular and
auxiliary verbs, which are subtypes of raising verbs, without the need for an Infl category.



\section*{Abbreviations}

\begin{tabularx}{.45\textwidth}{@{}lX}
\textsc{av} & agentive voice\\
\textsc{lf} & long form\\ 
\textsc{ov} & objective voice\\
\textsc{sf} & short form\\
\textsc{str} & strong\\
\textsc{wk} & weak\\

\end{tabularx}

\section*{Acknowledgements}

%\itd{Name the coeditors?}
I am grateful to the reviewers, Bob Borsley, Jean-Pierre Koening and Stefan Müller for their helpful comments.
{\sloppy
\printbibliography[heading=subbibliography,notkeyword=this] 
}

\end{document}


%      <!-- Local IspellDict: en_US-w_accents -->
