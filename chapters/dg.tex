%% -*- coding:utf-8 -*-

\documentclass[output=paper
%	        ,collection
%	        ,collectionchapter
 	        ,biblatex
                ,babelshorthands
                ,newtxmath
                ,draftmode
                ,colorlinks, citecolor=brown
]{langscibook}

\IfFileExists{../localcommands.tex}{%hack to check whether this is being compiled as part of a collection or standalone
  \usepackage{../nomemoize}
  % add all extra packages you need to load to this file 

% the ISBN assigned to the digital edition
\usepackage[ISBN=9783961102556]{ean13isbn} 

\usepackage{graphicx}
\usepackage{tabularx}
\usepackage{amsmath} 

%\usepackage{tipa}      % Davis Koenig
\usepackage{xunicode} % Provide tipa macros (BC)

\usepackage{multicol}

% Berthold morphology
\usepackage{relsize}
%\usepackage{./styles/rtrees-bc} % forbidden forest 08.12.2019

% provides logo priniting commands
\usepackage{langsci-basic}

\usepackage{langsci-optional} 
% used to be in this package
\providecommand{\citegen}{}
\renewcommand{\citegen}[2][]{\citeauthor{#2}'s (\citeyear*[#1]{#2})}
\providecommand{\lsptoprule}{}
\renewcommand{\lsptoprule}{\midrule\toprule}
\providecommand{\lspbottomrule}{}
\renewcommand{\lspbottomrule}{\bottomrule\midrule}
\providecommand{\largerpage}{}
\renewcommand{\largerpage}[1][1]{\enlargethispage{#1\baselineskip}}

\usepackage{./styles/biblatex-series-number-checks}


\usepackage{langsci-lgr}

\newcommand{\MAS}{\textsc{m}\xspace} % \M is taken by somebody

%\usepackage{./styles/forest/forest}
\usepackage{langsci-forest-setup}

% is loaded in main etc.
% \usepackage{nomemoize} 
% \memoizeset{
%   memo filename prefix={chapters/hpsg-handbook.memo.dir/},
%   register=\todo{O{}+m},
%   prevent=\todo,
% }

\usepackage{tikz-cd}

\usepackage{./styles/tikz-grid}
\usetikzlibrary{shadows}


% removed with texlive 2020 06.05.2020
% %\usepackage{pgfplots} % for data/theory figure in minimalism.tex
% % fix some issue with Mod https://tex.stackexchange.com/a/330076
% \makeatletter
% \let\pgfmathModX=\pgfmathMod@
% \usepackage{pgfplots}%
% \let\pgfmathMod@=\pgfmathModX
% \makeatother

\usepackage{subcaption}

% Stefan Müller's styles
\usepackage{./styles/merkmalstruktur,./styles/makros.2020,./styles/my-xspace,./styles/article-ex,
./styles/eng-date}

\usepackage{varioref}
\newcommand\refORregion[2]{%
 \vrefpagenum\firstnum{#1}%
 \vrefpagenum\secondnum{#2}%
\ifthenelse{\equal\firstnum\secondnum}%
{\pageref{#1}}%
{\pageref{#1}--\pageref{#2}}%
}

% I am sick of fiddeling arround with babel. I want these shorthands also to work in commands I
% define. St.Mü. 13.08.2020
% e.g. with \iwithini
\usepackage{german}
\selectlanguage{USenglish}

\usepackage{./styles/abbrev}


% Has to be loaded late since otherwise footnotes will not work

%%%%%%%%%%%%%%%%%%%%%%%%%%%%%%%%%%%%%%%%%%%%%%%%%%%%
%%%                                              %%%
%%%           Examples                           %%%
%%%                                              %%%
%%%%%%%%%%%%%%%%%%%%%%%%%%%%%%%%%%%%%%%%%%%%%%%%%%%%
% remove the percentage signs in the following lines
% if your book makes use of linguistic examples
\usepackage{langsci-gb4e} 



% This introduces labels which makes hyperlinks work so that proofreading is easier.
%\makeatletter
%\newcommand{\mex}[1]{\ref{ex-\the\c@chapter-\the\numexpr\c@equation+#1}\relax}
%\newcommand{\eaautolabel}{\label{ex-\the\c@chapter-\the\numexpr\c@equation+1}}
%\makeatother

%\let\oldea\ea
%\def\ea{\oldea\eaautolabel}

%\let\oldeal\eal
%\def\eal{\oldeal\eaautolabel}


% Crossing out text
% uncomment when needed
%\usepackage{ulem}

\usepackage{./styles/additional-langsci-index-shortcuts}

% this is the completely redone avm package
\usepackage{langsci-avm}
\avmsetup{columnsep=.3ex,style=narrow}

\avmdefinecommand{phon}[phon]
  {
    attributes  = \itshape%,
%    delimfactor = 900,
%    delimfall   = 10pt
}

\avmdefinecommand{form}[form]
  {
    attributes  = \itshape%,
%    delimfactor = 900,
%    delimfall   = 10pt
}

% \set was already taken
\avmdefinecommand{avmset}[set]{ attributes=\itshape } % define a new \set command
\avmdefinecommand{list}[list]{ attributes=\itshape } % define a new \list command
   % Note: the label "list" will be output in whatever font is currently active.

% \avm{
% 	[subj  & \1 \\
% 	comps & \2 \- \list*(gap-ss) \\ % Produce a \list
% 	deps  & < \1 > \+ \2
% 	]
% }


\avmdefinecommand{nelist}[ne-list]{ attributes=\itshape } % define a new \nelist command
   % Note: the label "ne-list" will be output in whatever font is currently active.



% https://github.com/langsci/langsci-avm/issues/33#issuecomment-671201576
%\avmsetup{extraskip=0pt}

% if you have to use both langsci-avm and avm
% \usepackage{langsci-avm} % Load pkg with meaning A of conflicting cmd
% \let\lavm\avm % Send the conflicting command to an alternative
% \let\avm\undefined % Send the conflicting cmd to be \undefined
% \usepackage{avm} % Load pkg with meaning B for conf. cmd 

%\let\asort\type*

% remove this, once we really do without avm
%\usepackage{./styles/avm+}

% copied over from avm+.sty
% some relation operators:
%\newcommand{\append}[0]{\ensuremath{\oplus\hspace{.24em}}}
%\newcommand{\shuffle}[0]{\ensuremath{\bigcirc\hspace{.24em}}}

\newcommand{\append}[0]{\ensuremath{\oplus}\xspace}
\newcommand{\shuffle}[0]{\ensuremath{\bigcirc}\xspace}


% command to fontify relations in avms 
\newcommand{\rel}[1]{\texttt{#1}}
%\def\relfont{\slshape}%
%\def\relfont{\ttdefault}%


\let\idx\ibox
\let\avmbox\ibox

% command to fontify attributes in ordinary text
%\newcommand{\attrib}[1]{\textsc{#1}}


% some relation operators:
%\newcommand{\append}[0]{\ensuremath{\oplus\hspace{.24em}}}
%\newcommand{\shuffle}[0]{\ensuremath{\bigcirc\hspace{.24em}}}

\def\relfont{\slshape}%
%
% command to fontify relations in avms 
%\newcommand{\rel}[1]{{\relfont #1}}



% \renewcommand{\tpv}[1]{{\avmjvalfont\itshape #1}}

% % no small caps please
% \renewcommand{\phonshape}[0]{\normalfont\itshape}

% \regAvmFonts

\usepackage{theorem}

\newtheorem{mydefinition}{Def.}
\newtheorem{principle}{Principle}

{\theoremstyle{break}
%\newtheorem{schema}{Schema}
\newtheorem{mydefinition-break}[mydefinition]{Def.}
\newtheorem{principle-break}[principle]{Principle}
}


%% \newcommand{schema}[2]{
%% \begin{minipage}{\textwidth}
%% {\textbf{Schema~\theschema}}]\hspace{.5em}\textbf{(#1)}\\
%% #2
%% \end{minipage}}


% This avoids linebreaks in the Schema
\newcounter{schemacounter}
\makeatletter
\newenvironment{schema}[1][]
  {%
   \refstepcounter{schemacounter}%
   \par\bigskip\noindent
   \minipage{\linewidth}%
   \textbf{Schema~\theschemacounter\hspace{.5em} \ifx&#1&\else(#1)\fi}\par
  }{\endminipage\par\bigskip\@endparenv}%
\makeatother

%\usepackage{subfig}





% Davis Koenig Lexikon

\usepackage{tikz-qtree,tikz-qtree-compat} % Davis Koenig remove

\usepackage{shadow}



\usepackage[english]{isodate} % Andy Lücking
\usepackage[autostyle]{csquotes} % Andy
%\usepackage[autolanguage]{numprint}

%\defaultfontfeatures{
%    Path = /usr/local/texlive/2017/texmf-dist/fonts/opentype/public/fontawesome/ }

%% https://tex.stackexchange.com/a/316948/18561
%\defaultfontfeatures{Extension = .otf}% adds .otf to end of path when font loaded without ext parameter e.g. \newfontfamily{\FA}{FontAwesome} > \newfontfamily{\FA}{FontAwesome.otf}
%\usepackage{fontawesome} % Andy Lücking
\usepackage{pifont} % Andy Lücking -> hand

\usetikzlibrary{decorations.pathreplacing} % Andy Lücking
\usetikzlibrary{matrix} % Andy 
\usetikzlibrary{positioning} % Andy
\usepackage{tikz-3dplot} % Andy

% pragmatics
\usepackage{eqparbox} % Andy
\usepackage{enumitem} % Andy
\usepackage{longtable} % Andy
\usepackage{tabu} % Andy              needs to be loaded before hyperref as of texlive 2020

% tabu-fix
% to make "spread 0pt" work
% -----------------------------
\RequirePackage{etoolbox}
\makeatletter
\patchcmd
	\tabu@startpboxmeasure
	{\bgroup\begin{varwidth}}%
	{\bgroup
	 \iftabu@spread\color@begingroup\fi\begin{varwidth}}%
	{}{}
\def\@tabarray{\m@th\def\tabu@currentgrouptype
    {\currentgrouptype}\@ifnextchar[\@array{\@array[c]}}
%
%%% \pdfelapsedtime bug 2019-12-15
\patchcmd
	\tabu@message@etime
	{\the\pdfelapsedtime}%
	{\pdfelapsedtime}%
	{}{}
%
%
\makeatother
% -----------------------------


% Manfred's packages

%\usepackage{shadow}

\usepackage{tabularx}
\newcolumntype{L}[1]{>{\raggedright\arraybackslash}p{#1}} % linksbündig mit Breitenangabe


% Jong-Bok

%\usepackage{xytree}

\newcommand{\xytree}[2][dummy]{Let's do the tree!}

% seems evil, get rid of it
% defines \ex is incompatible with gb4e
%\usepackage{lingmacros}

% taken from lingmacros:
\makeatletter
% \evnup is used to line up the enumsentence number and an entry along
% the top.  It can take an argument to improve lining up.
\def\evnup{\@ifnextchar[{\@evnup}{\@evnup[0pt]}}

\def\@evnup[#1]#2{\setbox1=\hbox{#2}%
\dimen1=\ht1 \advance\dimen1 by -.5\baselineskip%
\advance\dimen1 by -#1%
\leavevmode\lower\dimen1\box1}
\makeatother


% YK -- CG chapter

%\usepackage{xspace}
\usepackage{bm}
\usepackage{ebproof}


% Antonio Branco, remove this
\usepackage{epsfig}

% now unicode
%\usepackage{alphabeta}





\usepackage{pst-node}


% fmr: additional packages
%\usepackage{amsthm}


% Ash and Steve: LFG
\usepackage{./styles/lfg/dalrymple}

\RequirePackage{graphics}
%\RequirePackage{./styles/lfg/trees}
%% \RequirePackage{avm}
%% \avmoptions{active}
%% \avmfont{\sc}
%% \avmvalfont{\sc}
\RequirePackage{./styles/lfg/lfgmacrosash}

\usepackage{./styles/lfg/glue}

%%%%%%%%%%%%%%%%%%%%%%%%%%%%%%
%% Markup
%%%%%%%%%%%%%%%%%%%%%%%%%%%%%%
\usepackage[normalem]{ulem} % For thinks like strikethrough, using \sout

% \newcommand{\high}[1]{\textbf{#1}} % highlighted text
\newcommand{\high}[1]{\textit{#1}} % highlighted text
%\newcommand{\term}[1]{\textit{#1}\/} % technical term
\newcommand{\qterm}[1]{``{#1}''} % technical term, quotes
%\newcommand{\trns}[1]{\strut `#1'} % translation in glossed example
\newcommand{\trnss}[1]{\strut \phantom{\sqz{}} `#1'} % translation in ungrammatical glossed example
\newcommand{\ttrns}[1]{(`#1')} % an in-text translation of a word
\newcommand{\LFGfeat}[1]{\mbox{\textsc{\MakeLowercase{#1}}}}     % feature name
%\newcommand{\val}[1]{\mbox{\textsc{\MakeLowercase{#1}}}}    % f-structure value
\newcommand{\featt}[1]{\mbox{\textsc{\MakeLowercase{#1}}}}     % feature name
\newcommand{\vall}[1]{\mbox{\textsc{\textup{\MakeLowercase{#1}}}}}    % f-structure value
\newcommand{\mg}[1]{\mbox{\textsc{\MakeLowercase{#1}}}}    % morphological gloss
%\newcommand{\word}[1]{\textit{#1}}       % mention of word
\providecommand{\kstar}[1]{{#1}\ensuremath{^*}}
\providecommand{\kplus}[1]{{#1}\ensuremath{^+}}
\newcommand{\template}[1]{@\textsc{\MakeLowercase{#1}}}
\newcommand{\templaten}[1]{\textsc{\MakeLowercase{#1}}}
\newcommand{\templatenn}[1]{\MakeUppercase{#1}}
\newcommand{\tempeq}{\ensuremath{=}}
\newcommand{\predval}[1]{\ensuremath{\langle}\textsc{#1}\ensuremath{\rangle}}
\newcommand{\predvall}[1]{{\rm `#1'}}
\newcommand{\lfgfst}[1]{\ensuremath{#1\,}}
\newcommand{\scare}[1]{``#1''} % scare quotes
\newcommand{\bracket}[1]{\ensuremath{\left\langle\mathit{#1}\right\rangle}}
\newcommand{\sectionw}[1][]{Section#1} % section word: for cap/non-cap
\newcommand{\tablew}[1][]{Table#1} % table word: for cap/non-cap
\newcommand{\lfgglue}{LFG+Glue}
\newcommand{\hpsgglue}{HPSG+Glue}
\newcommand{\gs}{GS}
%\newcommand{\func}[1]{\ensuremath{\mathbf{#1}}}
\newcommand{\func}[1]{\textbf{#1}}
\renewcommand{\glue}{Glue}
%\newcommand{\exr}[1]{(\ref{ex:#1}}
\newcommand{\exra}[1]{(\ref{ex:#1})}


%%%%%%%%%%%%%%%%%%%%%%%%%%%%%%
% Notation
%\newcommand{\xbar}[1]{$_{\mbox{\textsc{#1}$^{\raisebox{1ex}{}}$}}$}
\newcommand{\xprime}[2][]{\textup{\mbox{{#2}\ensuremath{^\prime_{\hspace*{-.0em}\mbox{\footnotesize\ensuremath{\mathit{#1}}}}}}}}
\providecommand{\xzero}[2][]{#2\ensuremath{^0_{\mbox{\footnotesize\ensuremath{\mathit{#1}}}}}}



\let\leftangle\langle
\let\rightangle\rangle

%\newcommand{\pslabel}[1]{}

% remove when finished
\usepackage{proofread}
  %add all your local new commands to this file

% The orchid-id is specified and then extracted by scripts for zenodo.
\newcommand{\orcid}[1]{} 

% do not show the chapter number. It is redundant, since most references to figures are within the
% same chapter.
\renewcommand{\thefigure}{\arabic{figure}}


% Don't do this at home. I do not like the smaller font for captions.
% I just removed loading the caption packege in langscibook.cls
%% \captionsetup{%
%% font={%
%% stretch=1%.8%
%% ,normalsize%,small%
%% },%
%% width=.8\textwidth
%% }

\makeatletter
\def\blx@maxline{77}
\makeatother


\let\citew\citet

\newcommand{\page}{}

\newcommand{\todostefan}[1]{\todo[color=orange!80]{\footnotesize #1}\xspace}
\newcommand{\todosatz}[1]{\todo[color=red!40]{\footnotesize #1}\xspace}

\newcommand{\inlinetodostefan}[1]{\todo[color=green!40,inline]{\footnotesize #1}\xspace}

\newcommand{\inlinetodoopt}[1]{\todo[color=green!40,inline]{\footnotesize #1}\xspace}
\newcommand{\inlinetodoobl}[1]{\todo[color=red!40,inline]{\footnotesize #1}\xspace}

\newcommand{\itd}[1]{\inlinetodoobl{#1}}
\newcommand{\itdobl}[1]{\inlinetodoobl{#1}}
\newcommand{\itdopt}[1]{\inlinetodoopt{#1}}

\newcommand{\itdsecond}[1]{}

\newcommand{\itddone}[1]{}
%\let\itddone\itdopt
\newcommand{\LATER}[1]{}



% A. Red: Simple typos, errors in the AVMs (only a couple) to take care of on the editorial side, no need to contact the authors
% B.: Green: Wording changes which do not necessarily require authors’ approval, but are not just typos/errors
% C.: Blue: Comments to the author that they don’t have to take care of, but after all, the authors might be interested to have the comments for future revisions. 
% D.: Purple: Comments to the editors about something we need to keep in mind or do. Nothing for you

\newcommand{\colorcodingexplanation}{\todo[color=green!40,inline]{%
Explanation of colors of bubbles and text:\\
A.: Red: Things that have to be fixed/commented upon.\\
B.: Green: optional comments\\
C.: Blue: Comments to the author that they don’t have to take care of, but after all, the authors
might be interested to have the comments for future revisions.\\
Explanation of colors of text:\\
Red: newly added material (crossreferences to other chapters and other references)\\
Orange: changed material, please check\\
Blue: suggestions for deletion\\
Please also check margin notes.
}}
% D.: Purple: Comments to the editors about something we need to keep in mind or do. Nothing for you


\newcommand{\itdgreen}[1]{\todo[color=green!40,inline]{\footnotesize #1}\xspace}
\newcommand{\itdblue}[1]{\todo[color=blue!40,inline]{\footnotesize #1}\xspace}

% for editing, remove later
\usepackage{xcolor}
\newcommand{\added}[1]{{\red #1}}
\newcommand{\addedthis}{\todostefan{added this}}

\newcommand{\changed}[1]{\textcolor{orange}{#1}}
\newcommand{\deleted}[1]{\textcolor{blue}{#1}}


% \newcommand{\addpages}{\todostefan{add pages}}
% %\newcommand{\iaddpages}{\inlinetodoobl{add pages}}
% \newcommand{\iaddpages}{\yel[add pages]{pages}\xspace}
% \newcommand{\addref}{\todostefan{add reference}}
% \newcommand{\inlineaddpages}{\inlinetodostefan{add pages}}
% \newcommand{\addglosses}{\todostefan{add glosses}}

\newcommand{\addpages}{\xspace}%np
\newcommand{\iaddpages}{\xspace}%islands und understudied languages
\newcommand{\addref}{\xspace}
\newcommand{\inlineaddpages}{\xspace}
% not used \newcommand{\addglosses}{}


%\newcommand{\spacebr}{\hphantom{[}}

\newcommand{\danishep}{\jambox{(\ili{Danish})}}
\newcommand{\english}{\jambox{(\ili{English})}}
\newcommand{\german}{\jambox{(\ili{German})}}
\newcommand{\yiddish}{\jambox{(\ili{Yiddish})}}
\newcommand{\welsh}{\jambox{(\ili{Welsh})}}

% Cite and cross-reference other chapters
\newcommand{\crossrefchaptert}[2][]{\citet*[#1]{chapters/#2}, Chapter~\ref{chap-#2} of this volume} 
\newcommand{\crossrefchapterp}[2][]{(\citealp*[#1]{chapters/#2}, Chapter~\ref{chap-#2} of this volume)}
\newcommand{\crossrefchapteralt}[2][]{\citealt*[#1]{chapters/#2}, Chapter~\ref{chap-#2} of this volume}
\newcommand{\crossrefchapteralp}[2][]{\citealp*[#1]{chapters/#2}, Chapter~\ref{chap-#2} of this volume}

\newcommand{\crossrefcitet}[2][]{\citet*[#1]{chapters/#2}} 
\newcommand{\crossrefcitep}[2][]{\citep*[#1]{chapters/#2}}
\newcommand{\crossrefcitealt}[2][]{\citealt*[#1]{chapters/#2}}
\newcommand{\crossrefcitealp}[2][]{\citealp*[#1]{chapters/#2}}


% example of optional argument:
% \crossrefchapterp[for something, see:]{name}
% gives: (for something, see: Author 2018, Chapter~X of this volume)



\let\crossrefchapterw\crossrefchaptert



% Davis Koenig

\let\ig=\textsc
\let\tc=\textcolor

% evolution, Flickinger, Pollard, Wasow

\let\citeNP\citet

% Adam P

%\newcommand{\toappear}{Forthcoming}
\newcommand{\pg}[1]{p.\,#1}
\renewcommand{\implies}{\rightarrow}

\newcommand*{\rref}[1]{(\ref{#1})}
\newcommand*{\aref}[1]{(\ref{#1}a)}
\newcommand*{\bref}[1]{(\ref{#1}b)}
\newcommand*{\cref}[1]{(\ref{#1}c)}

\newcommand{\msadam}{.}
\newcommand{\morsyn}[1]{\textsc{#1}}

\newcommand{\aux}{\textsc{aux}\xspace}

\newcommand{\nom}{\morsyn{nom}}
\newcommand{\acc}{\morsyn{acc}}
\newcommand{\dat}{\morsyn{dat}}
\newcommand{\gen}{\morsyn{gen}}
\newcommand{\ins}{\morsyn{ins}}
%\newcommand{\aploc}{\morsyn{loc}}
\newcommand{\voc}{\morsyn{voc}}
\newcommand{\ill}{\morsyn{ill}}
\renewcommand{\inf}{\morsyn{inf}}
\newcommand{\passprc}{\morsyn{passp}}

%\newcommand{\Nom}{\msadam\nom}
%\newcommand{\Acc}{\msadam\acc}
%\newcommand{\Dat}{\msadam\dat}
%\newcommand{\Gen}{\msadam\gen}
\newcommand{\Ins}{\msadam\ins}
\newcommand{\Loc}{\msadam\loc}
\newcommand{\Voc}{\msadam\voc}
\newcommand{\Ill}{\msadam\ill}
\newcommand{\PassP}{\msadam\passprc}

\newcommand{\Aux}{\textsc{aux}}

%\newcommand{\princ}[1]{\textnormal{\textsc{#1}}} % for constraint names
\newcommand{\princ}[1]{\textnormal{#1}} % for constraint names
\newcommand{\notion}[1]{\emph{#1}}
\renewcommand{\path}[1]{\textnormal{\textsc{#1}}}
\newcommand{\ftype}[1]{\textit{#1}}
\newcommand{\fftype}[1]{{\scriptsize\textit{#1}}}
\newcommand{\la}{$\langle$}
\newcommand{\ra}{$\rangle$}
%\newcommand{\argst}{\path{arg-st}}
\newcommand{\phtm}[1]{\setbox0=\hbox{#1}\hspace{\wd0}}
\newcommand{\prep}[1]{\setbox0=\hbox{#1}\hspace{-1\wd0}#1}


% Rui

\newcommand{\spc}[0]{\hspace{-1pt}\underline{\hspace{6pt}}\,}
\newcommand{\spcs}[0]{\hspace{-1pt}\underline{\hspace{6pt}}\,\,}
\newcommand{\bad}[1]{\leavevmode\llap{#1}}
\newcommand{\COMMENT}[1]{}


% Rui coordination
\newcommand{\subl}[1]{$_{\scriptstyle \textsc{#1}}$}



% Andy Lücking gesture.tex
\newcommand{\Pointing}{\ding{43}}
% Giotto: "Meeting of Joachim and Anne at the Golden Gate" - 1305-10 
\definecolor{GoldenGate1}{rgb}{.13,.09,.13} % Dress of woman in black
\definecolor{GoldenGate2}{rgb}{.94,.94,.91} % Bridge
\definecolor{GoldenGate3}{rgb}{.06,.09,.22} % Blue sky
\definecolor{GoldenGate4}{rgb}{.94,.91,.87} % Dress of woman with shawl
\definecolor{GoldenGate5}{rgb}{.52,.26,.26} % Joachim's robe
\definecolor{GoldenGate6}{rgb}{.65,.35,.16} % Anne's robe
\definecolor{GoldenGate7}{rgb}{.91,.84,.42} % Joachim's halo

\makeatletter
\newcommand{\@Depth}{1} % x-dimension, to front
\newcommand{\@Height}{1} % z-dimension, up
\newcommand{\@Width}{1} % y-dimension, rightwards
%\GGS{<x-start>}{<y-start>}{<z-top>}{<z-bottom>}{<Farbe>}{<x-width>}{<y-depth>}{<opacity>}
\newcommand{\GGS}[9][]{%
\coordinate (O) at (#2-1,#3-1,#5);
\coordinate (A) at (#2-1,#3-1+#7,#5);
\coordinate (B) at (#2-1,#3-1+#7,#4);
\coordinate (C) at (#2-1,#3-1,#4);
\coordinate (D) at (#2-1+#8,#3-1,#5);
\coordinate (E) at (#2-1+#8,#3-1+#7,#5);
\coordinate (F) at (#2-1+#8,#3-1+#7,#4);
\coordinate (G) at (#2-1+#8,#3-1,#4);
\draw[draw=black, fill=#6, fill opacity=#9] (D) -- (E) -- (F) -- (G) -- cycle;% Front
\draw[draw=black, fill=#6, fill opacity=#9] (C) -- (B) -- (F) -- (G) -- cycle;% Top
\draw[draw=black, fill=#6, fill opacity=#9] (A) -- (B) -- (F) -- (E) -- cycle;% Right
}
\makeatother


% pragmatics
\newcommand{\speaking}[1]{\eqparbox{name}{\textsc{\lowercase{#1}\space}}}
\newcommand{\alname}[1]{\eqparbox{name}{\textsc{\lowercase{#1}}}}
\newcommand{\HPSGTTR}{HPSG$_{\text{TTR}}$\xspace}

\newcommand{\ttrtype}[1]{\textit{#1}}
\newcommand{\avmel}{\q<\quad\q>} %% shortcut for empty lists in AVM
\newcommand{\ttrmerge}{\ensuremath{\wedge_{\textit{merge}}}}
\newcommand{\Cat}[2][0.1pt]{%
  \begin{scope}[y=#1,x=#1,yscale=-1, inner sep=0pt, outer sep=0pt]
   \path[fill=#2,line join=miter,line cap=butt,even odd rule,line width=0.8pt]
  (151.3490,307.2045) -- (264.3490,307.2045) .. controls (264.3490,291.1410) and (263.2021,287.9545) .. (236.5990,287.9545) .. controls (240.8490,275.2045) and (258.1242,244.3581) .. (267.7240,244.3581) .. controls (276.2171,244.3581) and (286.3490,244.8259) .. (286.3490,264.2045) .. controls (286.3490,286.2045) and (323.3717,321.6755) .. (332.3490,307.2045) .. controls (345.7277,285.6390) and (309.3490,292.2151) .. (309.3490,240.2046) .. controls (309.3490,169.0514) and (350.8742,179.1807) .. (350.8742,139.2046) .. controls (350.8742,119.2045) and (345.3490,116.5037) .. (345.3490,102.2045) .. controls (345.3490,83.3070) and (361.9972,84.4036) .. (358.7581,68.7349) .. controls (356.5206,57.9117) and (354.7696,49.2320) .. (353.4652,36.1439) .. controls (352.5396,26.8573) and (352.2445,16.9594) .. (342.5985,17.3574) .. controls (331.2650,17.8250) and (326.9655,37.7742) .. (309.3490,39.2045) .. controls (291.7685,40.6320) and (276.7783,24.2380) .. (269.9740,26.5795) .. controls (263.2271,28.9013) and (265.3490,47.2045) .. (269.3490,60.2045) .. controls (275.6359,80.6368) and (289.3490,107.2045) .. (264.3490,111.2045) .. controls (239.3490,115.2045) and (196.3490,119.2045) .. (165.3490,160.2046) .. controls (134.3490,201.2046) and (135.4934,249.3212) .. (123.3490,264.2045) .. controls (82.5907,314.1553) and (40.8239,293.6463) .. (40.8239,335.2045) .. controls (40.8239,353.8102) and (72.3490,367.2045) .. (77.3490,361.2045) .. controls (82.3490,355.2045) and (34.8638,337.3259) .. (87.9955,316.2045) .. controls (133.3871,298.1601) and   (137.4391,294.4766) .. (151.3490,307.2045) -- cycle;
\end{scope}%
}
%% leicht modifiziert nach Def. von Sebastian Nordhoff:
% \newcommand{\lueckingbox}[3]{\parbox[t][][t]{0.7cm}{\raggedright
%     \strut#1}\parbox[t][][t]{7.7cm}{\strut#2}\parbox[t][][t]{3cm}{\raggedright\strut#3}\bigskip\\}
\newcommand{\lueckingbox}[3]{\parbox[t][][t]{0.7cm}{\raggedright
    \strut\vspace*{-\baselineskip}\newline#1}\parbox[t][][t]{7.7cm}{\strut\vspace*{-\baselineskip}\newline#2}\parbox[t][][t]{3cm}{\raggedright\strut\vspace*{-\baselineskip}\newline#3}\bigskip\\}




% KdK
\newcommand{\smiley}{:)}

\renewbibmacro*{index:name}[5]{%
  \usebibmacro{index:entry}{#1}
    {\iffieldundef{usera}{}{\thefield{usera}\actualoperator}\mkbibindexname{#2}{#3}{#4}{#5}}}

% \newcommand{\noop}[1]{}

% chngcntr.sty otherwise gives error that these are already defined
%\let\counterwithin\relax
%\let\counterwithout\relax

% the space of a left bracket for glossings
\newcommand{\LB}{\hphantom{[}}

\newcommand{\LF}{\mbox{$[\![$}}

\newcommand{\RF}{\mbox{$]\!]_F$}}

\newcommand{\RT}{\mbox{$]\!]_T$}}





% Manfred's

\newcommand{\kommentar}[1]{}

\newcommand{\bsp}[1]{\emph{#1}}
\newcommand{\bspT}[2]{\bsp{#1} `#2'}
\newcommand{\bspTL}[3]{\bsp{#1} (lit.: #2) `#3'}

\newcommand{\noidi}{§}

\newcommand{\refer}[1]{(\ref{#1})}

%\newcommand{\avmtype}[1]{\multicolumn{2}{l}{\type{#1}}}
\newcommand{\attr}[1]{\textsc{#1}}

%\newcommand{\srdefault}{\mbox{\begin{tabular}{@{}c@{}}{\large <}\\[-1.5ex]$\sqcap$\end{tabular}}}
\newcommand{\srdefault}{$\stackrel{<}{\sqcap}$}


%% \newcommand{\myappcolumn}[2]{
%% \begin{minipage}[t]{#1}#2\end{minipage}
%% }

%% \newcommand{\appc}[1]{\myappcolumn{3.7cm}{#1}}


% Jong-Bok


% clean that up and do not use \def (killing other stuff defined before)
%\if 0
%\newcommand\DEL{\textsc{del}}
%\newcommand\del{\textsc{del}}

\newcommand\conn{\textsc{conn}}
\newcommand\CONN{\textsc{conn}}
\newcommand\CONJ{\textsc{conj}}
\newcommand\LITE{\textsc{lex}}
\newcommand\lite{\textsc{lex}}
\newcommand\HON{\textsc{hon}}

%\newcommand\CAUS{\textsc{caus}}
%\newcommand\PASS{\textsc{pass}}
\newcommand\NPST{\textsc{npst}}
%\newcommand\COND{\textsc{cond}}



\newcommand\hdlite{\textsc{head-lex construction}}
\newcommand\hdlight{\textsc{head-light} Schema}
\newcommand\NFORM{\textsc{nform}}

\newcommand\RELS{\textsc{rels}}
%\newcommand\TENSE{\textsc{tense}}


%\newcommand\ARG{\textsc{arg}}
\newcommand\ARGs{\textsc{arg0}}
\newcommand\ARGa{\textsc{arg}}
\newcommand\ARGb{\textsc{arg2}}
\newcommand\TPC{\textsc{top}}
%\newcommand\PROG{\textsc{prog}}

\newcommand\LIGHT{\textsc{light}\xspace}
\newcommand\pst{\textsc{pst}}
%\newcommand\PAST{\textsc{pst}}
%\newcommand\DAT{\textsc{dat}}
%\newcommand\CONJ{\textsc{conj}}
\newcommand\nominal{\textsc{nominal}}
\newcommand\NOMINAL{\textsc{nominal}}
\newcommand\VAL{\textsc{val}}
%\newcommand\val{\textsc{val}}
\newcommand\MODE{\textsc{mode}}
\newcommand\RESTR{\textsc{restr}}
\newcommand\SIT{\textsc{sit}}
\newcommand\ARG{\textsc{arg}}
\newcommand\RELN{\textsc{rel}}
%\newcommand\REL{\textsc{rel}}
%\newcommand\RELS{\textsc{rels}}
%\newcommand\arg-st{\textsc{arg-st}}
\newcommand\xdel{\textsc{xdel}}
\newcommand\zdel{\textsc{zdel}}
\newcommand\sug{\textsc{sug}}
%\newcommand\IMP{\textsc{imp}}
%\newcommand\conn{\textsc{conn}}
%\newcommand\CONJ{\textsc{conj}}
%\newcommand\HON{\textsc{hon}}
\newcommand\BN{\textsc{bn}}
\newcommand\bn{\textsc{bn}}
\newcommand\pres{\textsc{pres}}
\newcommand\PRES{\textsc{pres}}
\newcommand\prs{\textsc{pres}}
%\newcommand\PRS{\textsc{pres}}
\newcommand\agt{\textsc{agt}}
%\newcommand\DEL{\textsc{del}}
%\newcommand\PRED{\textsc{pred}}
\newcommand\AGENT{\textsc{agent}}
\newcommand\THEME{\textsc{theme}}
%\newcommand\AUX{\textsc{aux}}
%\newcommand\THEME{\textsc{theme}}
%\newcommand\PL{\textsc{pl}}
\newcommand\SRC{\textsc{src}}
\newcommand\src{\textsc{src}}
\newcommand{\FORMjb}{\textsc{form}}
\newcommand{\formjb}{\FORM}
\newcommand\GCASE{\textsc{gcase}}
\newcommand\gcase{\textsc{gcase}}
\newcommand\SCASE{\textsc{scase}}
\newcommand\PHON{\textsc{phon}}
%\newcommand\SS{\textsc{ss}}
\newcommand\SYN{\textsc{syn}}
%\newcommand\LOC{\textsc{loc}}
\newcommand\MOD{\textsc{mod}}
\newcommand\INV{\textsc{inv}}
%\newcommand\L{\textsc{l}}
%\newcommand\CASE{\textsc{case}}
\newcommand\SPR{\textsc{spr}}
\newcommand\COMPS{\textsc{comps}}
%\newcommand\comps{\textsc{comps}}
\newcommand\SEM{\textsc{sem}}
\newcommand\CONT{\textsc{cont}}
\newcommand\SUBCAT{\textsc{subcat}}
\newcommand\CAT{\textsc{cat}}
%\newcommand\C{\textsc{c}}
%\newcommand\SUBJ{\textsc{subj}}
\newcommand\subjjb{\textsc{subj}}
%\newcommand\SLASH{\textsc{slash}}
\newcommand\LOCAL{\textsc{local}}
%\newcommand\ARG-ST{\textsc{arg-st}}
%\newcommand\AGR{\textsc{agr}}
\newcommand\PER{\textsc{per}}
%\newcommand\NUM{\textsc{num}}
%\newcommand\IND{\textsc{ind}}
\newcommand\VFORM{\textsc{vform}}
\newcommand\PFORM{\textsc{pform}}
\newcommand\decl{\textsc{decl}}
%\newcommand\loc{\textsc{loc   }}
% \newcommand\   {\textsc{  }}

%\newcommand\NEG{\textsc{neg}}
\newcommand\FRAMES{\textsc{frames}}
%\newcommand\REFL{\textsc{refl}}

\newcommand\MKG{\textsc{mkg}}

%\newcommand\BN{\textsc{bn}}
\newcommand\HD{\textsc{hd}}
\newcommand\NP{\textsc{np}}
\newcommand\PF{\textsc{pf}}
%\newcommand\PL{\textsc{pl}}
\newcommand\PP{\textsc{pp}}
%\newcommand\SS{\textsc{ss}}
\newcommand\VF{\textsc{vf}}
\newcommand\VP{\textsc{vp}}
%\newcommand\bn{\textsc{bn}}
\newcommand\cl{\textsc{cl}}
%\newcommand\pl{\textsc{pl}}
\newcommand\Wh{\ital{Wh}}
%\newcommand\ng{\textsc{neg}}
\newcommand\wh{\ital{wh}}
%\newcommand\ACC{\textsc{acc}}
%\newcommand\AGR{\textsc{agr}}
\newcommand\AGT{\textsc{agt}}
\newcommand\ARC{\textsc{arc}}
%\newcommand\ARG{\textsc{arg}}
\newcommand\ARP{\textsc{arc}}
%\newcommand\AUX{\textsc{aux}}
%\newcommand\CAT{\textsc{cat}}
%\newcommand\COP{\textsc{cop}}
%\newcommand\DAT{\textsc{dat}}
\newcommand\NEWCOMMAND{\textsc{def}}
%\newcommand\DEL{\textsc{del}}
\newcommand\DOM{\textsc{dom}}
\newcommand\DTR{\textsc{dtr}}
%\newcommand\FUT{\textsc{fut}}
\newcommand\GAP{\textsc{gap}}
%\newcommand\GEN{\textsc{gen}}
%\newcommand\HON{\textsc{hon}}
%\newcommand\IMP{\textsc{imp}}
%\newcommand\IND{\textsc{ind}}
%\newcommand\INV{\textsc{inv}}
\newcommand\LEX{\textsc{lex}}
\newcommand\Lex{\textsc{lex}}
%\newcommand\LOC{\textsc{loc}}
%\newcommand\MOD{\textsc{mod}}
\newcommand\MRK{{\nr MRK}}
%\newcommand\NEG{\textsc{neg}}
\newcommand\NEW{\textsc{new}}
%\newcommand\NOM{\textsc{nom}}
%\newcommand\NUM{\textsc{num}}
%\newcommand\PER{\textsc{per}}
%\newcommand\PST{\textsc{pst}}
\newcommand\QUE{\textsc{que}}
%\newcommand\REL{\textsc{rel}}
\newcommand\SEL{\textsc{sel}}
%\newcommand\SEM{\textsc{sem}}
%\newcommand\SIT{\textsc{arg0}}
%\newcommand\SPR{\textsc{spr}}
%\newcommand\SRC{\textsc{src}}
\newcommand\SUG{\textsc{sug}}
%\newcommand\SYN{\textsc{syn}}
%\newcommand\TPC{\textsc{top}}
%\newcommand\VAL{\textsc{val}}
%\newcommand\acc{\textsc{acc}}
%\newcommand\agt{\textsc{agt}}
\newcommand\cop{\textsc{cop}}
%\newcommand\dat{\textsc{dat}}
\newcommand\foc{\textsc{focus}}
%\newcommand\FOC{\textsc{focus}}
\newcommand\fut{\textsc{fut}}
\newcommand\hon{\textsc{hon}}
\newcommand\imp{\textsc{imp}}
\newcommand\kes{\textsc{kes}}
%\newcommand\lex{\textsc{lex}}
%\newcommand\loc{\textsc{loc}}
\newcommand\mrk{{\nr MRK}}
%\newcommand\nom{\textsc{nom}}
%\newcommand\num{\textsc{num}}
\newcommand\plu{\textsc{plu}}
\newcommand\pne{\textsc{pne}}
%\newcommand\pst{\textsc{pst}}
\newcommand\pur{\textsc{pur}}
%\newcommand\que{\textsc{que}}
%\newcommand\src{\textsc{src}}
%\newcommand\sug{\textsc{sug}}
\newcommand\tpc{\textsc{top}}
%\newcommand\utt{\textsc{utt}}
%\newcommand\val{\textsc{val}}
%% \newcommand\LITE{\textsc{lex}}
%% \newcommand\PAST{\textsc{pst}}
%% \newcommand\POSP{\textsc{pos}}
%% \newcommand\PRS{\textsc{pres}}
%% \newcommand\mod{\textsc{mod}}%
%% \newcommand\newuse{{`kes'}}
%% \newcommand\posp{\textsc{pos}}
%% \newcommand\prs{\textsc{pres}}
%% \newcommand\psp{{\it en\/}}
%% \newcommand\skes{\textsc{kes}}
%% \newcommand\CASE{\textsc{case}}
%% \newcommand\CASE{\textsc{case}}
%% \newcommand\COMP{\textsc{comp}}
%% \newcommand\CONJ{\textsc{conj}}
%% \newcommand\CONN{\textsc{conn}}
%% \newcommand\CONT{\textsc{cont}}
%% \newcommand\DECL{\textsc{decl}}
%% \newcommand\FOCUS{\textsc{focus}}
%% %\newcommand\FORM{\textsc{form}} duplicate
%% \newcommand\FREL{\textsc{frel}}
%% \newcommand\GOAL{\textsc{goal}}
\newcommand\HEAD{\textsc{head}}
%% \newcommand\INDEX{\textsc{ind}}
%% \newcommand\INST{\textsc{inst}}
%% \newcommand\MODE{\textsc{mode}}
%% \newcommand\MOOD{\textsc{mood}}
%% \newcommand\NMLZ{\textsc{nmlz}}
%% \newcommand\PHON{\textsc{phon}}
%% \newcommand\PRED{\textsc{pred}}
%% %\newcommand\PRES{\textsc{pres}}
%% \newcommand\PROM{\textsc{prom}}
%% \newcommand\RELN{\textsc{pred}}
%% \newcommand\RELS{\textsc{rels}}
%% \newcommand\STEM{\textsc{stem}}
%% \newcommand\SUBJ{\textsc{subj}}
%% \newcommand\XARG{\textsc{xarg}}
%% \newcommand\bse{{\it bse\/}}
%% \newcommand\case{\textsc{case}}
%% \newcommand\caus{\textsc{caus}}
%% \newcommand\comp{\textsc{comp}}
%% \newcommand\conj{\textsc{conj}}
%% \newcommand\conn{\textsc{conn}}
%% \newcommand\decl{\textsc{decl}}
%% \newcommand\fin{{\it fin\/}}
%% %\newcommand\form{\textsc{form}}
%% \newcommand\gend{\textsc{gend}}
%% \newcommand\inf{{\it inf\/}}
%% \newcommand\mood{\textsc{mood}}
%% \newcommand\nmlz{\textsc{nmlz}}
%% \newcommand\pass{\textsc{pass}}
%% \newcommand\past{\textsc{past}}
%% \newcommand\perf{\textsc{perf}}
%% \newcommand\pln{{\it pln\/}}
%% \newcommand\pred{\textsc{pred}}


%% %\newcommand\pres{\textsc{pres}}
%% \newcommand\proc{\textsc{proc}}
%% \newcommand\nonfin{{\it nonfin\/}}
%% \newcommand\AGENT{\textsc{agent}}
%% \newcommand\CFORM{\textsc{cform}}
%% %\newcommand\COMPS{\textsc{comps}}
%% \newcommand\COORD{\textsc{coord}}
%% \newcommand\COUNT{\textsc{count}}
%% \newcommand\EXTRA{\textsc{extra}}
%% \newcommand\GCASE{\textsc{gcase}}
%% \newcommand\GIVEN{\textsc{given}}
%% \newcommand\LOCAL{\textsc{local}}
%% \newcommand\NFORM{\textsc{nform}}
%% \newcommand\PFORM{\textsc{pform}}
%% \newcommand\SCASE{\textsc{scase}}
%% \newcommand\SLASH{\textsc{slash}}
%% \newcommand\SLASH{\textsc{slash}}
%% \newcommand\THEME{\textsc{theme}}
%% \newcommand\TOPIC{\textsc{topic}}
%% \newcommand\VFORM{\textsc{vform}}
%% \newcommand\cause{\textsc{cause}}
%% %\newcommand\comps{\textsc{comps}}
%% \newcommand\gcase{\textsc{gcase}}
%% \newcommand\itkes{{\it kes\/}}
%% \newcommand\pass{{\it pass\/}}
%% \newcommand\vform{\textsc{vform}}
%% \newcommand\CCONT{\textsc{c-cont}}
%% \newcommand\GN{\textsc{given-new}}
%% \newcommand\INFO{\textsc{info-st}}
%% \newcommand\ARG-ST{\textsc{arg-st}}
%% \newcommand\SUBCAT{\textsc{subcat}}
%% \newcommand\SYNSEM{\textsc{synsem}}
%% \newcommand\VERBAL{\textsc{verbal}}
%% \newcommand\arg-st{\textsc{arg-st}}
%% \newcommand\plain{{\it plain}\/}
%% \newcommand\propos{\textsc{propos}}
%% \newcommand\ADVERBIAL{\textsc{advl}}
%% \newcommand\HIGHLIGHT{\textsc{prom}}
%% \newcommand\NOMINAL{\textsc{nominal}}

\newenvironment{myavm}{\begingroup\avmvskip{.1ex}
  \selectfont\begin{avm}}%
{\end{avm}\endgroup\medskip}
\newcommand\pfix{\vspace{-5pt}}


\newcommand{\jbsub}[1]{\lower4pt\hbox{\small #1}}
\newcommand{\jbssub}[1]{\lower4pt\hbox{\small #1}}
\newcommand\jbtr{\underbar{\ \ \ }\ }


%\fi

% cl

\newcommand{\delphin}{\textsc{delph-in}}


% YK -- CG chapter

\newcommand{\grey}[1]{\colorbox{mycolor}{#1}}
\definecolor{mycolor}{gray}{0.8}

\newcommand{\GQU}[2]{\raisebox{1.6ex}{\ensuremath{\rotatebox{180}{\textbf{#1}}_{\scalebox{.7}{\textbf{#2}}}}}}

\newcommand{\SetInfLen}{\setpremisesend{0pt}\setpremisesspace{10pt}\setnamespace{0pt}}

\newcommand{\pt}[1]{\ensuremath{\mathsf{#1}}}
\newcommand{\ptv}[1]{\ensuremath{\textsf{\textsl{#1}}}}

\newcommand{\sv}[1]{\ensuremath{\bm{\mathcal{#1}}}}
\newcommand{\sX}{\sv{X}}
\newcommand{\sF}{\sv{F}}
\newcommand{\sG}{\sv{G}}

\newcommand{\syncat}[1]{\textrm{#1}}
\newcommand{\syncatVar}[1]{\ensuremath{\mathit{#1}}}

\newcommand{\RuleName}[1]{\textrm{#1}}

\newcommand{\SemTyp}{\textsf{Sem}}

\newcommand{\E}{\ensuremath{\bm{\epsilon}}\xspace}

\newcommand{\greeka}{\upalpha}
\newcommand{\greekb}{\upbeta}
\newcommand{\greekd}{\updelta}
\newcommand{\greekp}{\upvarphi}
\newcommand{\greekr}{\uprho}
\newcommand{\greeks}{\upsigma}
\newcommand{\greekt}{\uptau}
\newcommand{\greeko}{\upomega}
\newcommand{\greekz}{\upzeta}

\newcommand{\Lemma}{\ensuremath{\hskip.5em\vdots\hskip.5em}\noLine}
\newcommand{\LemmaAlt}{\ensuremath{\hskip.5em\vdots\hskip.5em}}

\newcommand{\I}{\iota}

\newcommand{\sem}{\ensuremath}

\newcommand{\NoSem}{%
\renewcommand{\LexEnt}[3]{##1; \syncat{##3}}
\renewcommand{\LexEntTwoLine}[3]{\renewcommand{\arraystretch}{.8}%
\begin{array}[b]{l} ##1;  \\ \syncat{##3} \end{array}}
\renewcommand{\LexEntThreeLine}[3]{\renewcommand{\arraystretch}{.8}%
\begin{array}[b]{l} ##1; \\ \syncat{##3} \end{array}}}

\newcommand{\hypml}[2]{\left[\!\!#1\!\!\right]^{#2}}

%%%%for bussproof
\def\defaultHypSeparation{\hskip0.1in}
\def\ScoreOverhang{0pt}

\newcommand{\MultiLine}[1]{\renewcommand{\arraystretch}{.8}%
\ensuremath{\begin{array}[b]{l} #1 \end{array}}}

\newcommand{\MultiLineMod}[1]{%
\ensuremath{\begin{array}[t]{l} #1 \end{array}}}

\newcommand{\hypothesis}[2]{[ #1 ]^{#2}}

\newcommand{\LexEnt}[3]{#1; \ensuremath{#2}; \syncat{#3}}

\newcommand{\LexEntTwoLine}[3]{\renewcommand{\arraystretch}{.8}%
\begin{array}[b]{l} #1; \\ \ensuremath{#2};  \syncat{#3} \end{array}}

\newcommand{\LexEntThreeLine}[3]{\renewcommand{\arraystretch}{.8}%
\begin{array}[b]{l} #1; \\ \ensuremath{#2}; \\ \syncat{#3} \end{array}}

\newcommand{\LexEntFiveLine}[5]{\renewcommand{\arraystretch}{.8}%
\begin{array}{l} #1 \\ #2; \\ \ensuremath{#3} \\ \ensuremath{#4}; \\ \syncat{#5} \end{array}}

\newcommand{\LexEntFourLine}[4]{\renewcommand{\arraystretch}{.8}%
\begin{array}{l} \pt{#1} \\ \pt{#2}; \\ \syncat{#4} \end{array}}

\newcommand{\ManySomething}{\renewcommand{\arraystretch}{.8}%
\raisebox{-3mm}{\begin{array}[b]{c} \vdots \,\,\,\,\,\, \vdots \\
\vdots \end{array}}}

\newcommand{\lemma}[1]{\renewcommand{\arraystretch}{.8}%
\begin{array}[b]{c} \vdots \\ #1 \end{array}}

\newcommand{\lemmarev}[1]{\renewcommand{\arraystretch}{.8}%
\begin{array}[b]{c} #1 \\ \vdots \end{array}}

\newcommand{\p}{\ensuremath{\upvarphi}}

% clashes with soul package
\newcommand{\yusukest}{\textbf{\textsf{st}}}

\newcommand{\shortarrow}{\xspace\hskip-1.2ex\scalebox{.5}[1]{\ensuremath{\bm{\rightarrow}}}\hskip-.5ex\xspace}

\newcommand{\SemInt}[1]{\mbox{$[\![ \textrm{#1} ]\!]$}}

\newcommand{\HypSpace}{\hskip-.8ex}
\newcommand{\RaiseHeight}{\raisebox{2.2ex}}
\newcommand{\RaiseHeightLess}{\raisebox{1ex}}

\newcommand{\ThreeColHyp}[1]{\RaiseHeight{\Bigg[}\HypSpace#1\HypSpace\RaiseHeight{\Bigg]}}
\newcommand{\TwoColHyp}[1]{\RaiseHeightLess{\Big[}\HypSpace#1\HypSpace\RaiseHeightLess{\Big]}}

\newcommand{\LemmaShort}{\ensuremath{ \ \vdots} \ \noLine}
\newcommand{\LemmaShortAlt}{\ensuremath{ \ \vdots} \ }

\newcommand{\fail}{**}
\newcommand{\vs}{\raisebox{.05em}{\ensuremath{\upharpoonright}}}
\newcommand{\DerivSize}{\small}

% This is not needed, we just take unicode symbols
% The result of the code below came out wrong anyway.
% St. Mü. 10.06.2021
%
% \def\maru#1{{\ooalign{\hfil
%   \ifnum#1>999 \resizebox{.25\width}{\height}{#1}\else%
%   \ifnum#1>99 \resizebox{.33\width}{\height}{#1}\else%
%   \ifnum#1>9 \resizebox{.5\width}{\height}{#1}\else #1%
%   \fi\fi\fi%
% \/\hfil\crcr%
% \raise.167ex\hbox{\mathhexbox20D}}}}

\newenvironment{samepage2}%
 {\begin{flushleft}\begin{minipage}{\linewidth}}
 {\end{minipage}\end{flushleft}}

\newcommand{\cmt}[1]{\textsl{\textbf{[#1]}}}
\newcommand{\trns}[1]{\textbf{#1}\xspace}
\newcommand{\ptfont}{}
\newcommand{\gp}{\underline{\phantom{oo}}}
\newcommand{\mgcmt}{\marginnote}

\newcommand{\term}[1]{\emph{\isi{#1}}}

\newcommand{\citeposs}[1]{\citeauthor{#1}'s \citeyearpar{#1}}

% for standalone compilations Felix: This is in the class already
%\let\thetitle\@title
%\let\theauthor\@author 
\makeatletter
\newcommand{\togglepaper}[1][0]{ 
\bibliography{../Bibliographies/stmue,../localbibliography,
collection.bib}
  %% hyphenation points for line breaks
%% Normally, automatic hyphenation in LaTeX is very good
%% If a word is mis-hyphenated, add it to this file
%%
%% add information to TeX file before \begin{document} with:
%% %% hyphenation points for line breaks
%% Normally, automatic hyphenation in LaTeX is very good
%% If a word is mis-hyphenated, add it to this file
%%
%% add information to TeX file before \begin{document} with:
%% \include{localhyphenation}
\hyphenation{
A-la-hver-dzhie-va
ac-cu-sa-tive
anaph-o-ra
ana-phor
ana-phors
an-te-ced-ent
an-te-ced-ents
affri-ca-te
affri-ca-tes
ap-proach-es
Atha-bas-kan
Athe-nä-um
Be-schrei-bung
Bona-mi
Chi-che-ŵa
com-ple-ments
con-straints
Cope-sta-ke
Da-ge-stan
Dor-drecht
er-klä-ren-de
Flick-inger
Ginz-burg
Gro-ning-en
Has-pel-math
Jap-a-nese
Jon-a-than
Ka-tho-lie-ke
Ko-bon
krie-gen
Kroe-ger
Le-Sourd
moth-er
Mül-ler
Nie-mey-er
Ørs-nes
Par-a-digm
Prze-piór-kow-ski
phe-nom-e-non
re-nowned
Rie-he-mann
un-bound-ed
Ver-gleich
with-in
}

% listing within here does not have any effect for lfg.tex % 2020-05-14

% why has "erklärende" be listed here? I specified langid in bibtex item. Something is still not working with hyphenation.


% to do: check
%  Alahverdzhieva


% biblatex:

% This is a LaTeX frontend to TeX’s \hyphenation command which defines hy- phenation exceptions. The ⟨language⟩ must be a language name known to the babel/polyglossia packages. The ⟨text ⟩ is a whitespace-separated list of words. Hyphenation points are marked with a dash:

% \DefineHyphenationExceptions{american}{%
% hy-phen-ation ex-cep-tion }

\hyphenation{
A-la-hver-dzhie-va
ac-cu-sa-tive
anaph-o-ra
ana-phor
ana-phors
an-te-ced-ent
an-te-ced-ents
affri-ca-te
affri-ca-tes
ap-proach-es
Atha-bas-kan
Athe-nä-um
Be-schrei-bung
Bona-mi
Chi-che-ŵa
com-ple-ments
con-straints
Cope-sta-ke
Da-ge-stan
Dor-drecht
er-klä-ren-de
Flick-inger
Ginz-burg
Gro-ning-en
Has-pel-math
Jap-a-nese
Jon-a-than
Ka-tho-lie-ke
Ko-bon
krie-gen
Kroe-ger
Le-Sourd
moth-er
Mül-ler
Nie-mey-er
Ørs-nes
Par-a-digm
Prze-piór-kow-ski
phe-nom-e-non
re-nowned
Rie-he-mann
un-bound-ed
Ver-gleich
with-in
}

% listing within here does not have any effect for lfg.tex % 2020-05-14

% why has "erklärende" be listed here? I specified langid in bibtex item. Something is still not working with hyphenation.


% to do: check
%  Alahverdzhieva


% biblatex:

% This is a LaTeX frontend to TeX’s \hyphenation command which defines hy- phenation exceptions. The ⟨language⟩ must be a language name known to the babel/polyglossia packages. The ⟨text ⟩ is a whitespace-separated list of words. Hyphenation points are marked with a dash:

% \DefineHyphenationExceptions{american}{%
% hy-phen-ation ex-cep-tion }

  \memoizeset{
    memo filename prefix={hpsg-handbook.memo.dir/},
    % readonly
  }
  \papernote{\scriptsize\normalfont
    \@author.
    \titleTemp. 
    To appear in: 
    Stefan Müller, Anne Abeillé, Robert D. Borsley \& Jean-Pierre Koenig (eds.)
    HPSG Handbook
    Berlin: Language Science Press. [preliminary page numbering]
  }
  \pagenumbering{roman}
  \setcounter{chapter}{#1}
  \addtocounter{chapter}{-1}
}
\makeatother

\makeatletter
\newcommand{\togglepaperminimal}[1][0]{ 
  \bibliography{../Bibliographies/stmue,
                ../localbibliography,
collection.bib}
  %% hyphenation points for line breaks
%% Normally, automatic hyphenation in LaTeX is very good
%% If a word is mis-hyphenated, add it to this file
%%
%% add information to TeX file before \begin{document} with:
%% %% hyphenation points for line breaks
%% Normally, automatic hyphenation in LaTeX is very good
%% If a word is mis-hyphenated, add it to this file
%%
%% add information to TeX file before \begin{document} with:
%% \include{localhyphenation}
\hyphenation{
A-la-hver-dzhie-va
ac-cu-sa-tive
anaph-o-ra
ana-phor
ana-phors
an-te-ced-ent
an-te-ced-ents
affri-ca-te
affri-ca-tes
ap-proach-es
Atha-bas-kan
Athe-nä-um
Be-schrei-bung
Bona-mi
Chi-che-ŵa
com-ple-ments
con-straints
Cope-sta-ke
Da-ge-stan
Dor-drecht
er-klä-ren-de
Flick-inger
Ginz-burg
Gro-ning-en
Has-pel-math
Jap-a-nese
Jon-a-than
Ka-tho-lie-ke
Ko-bon
krie-gen
Kroe-ger
Le-Sourd
moth-er
Mül-ler
Nie-mey-er
Ørs-nes
Par-a-digm
Prze-piór-kow-ski
phe-nom-e-non
re-nowned
Rie-he-mann
un-bound-ed
Ver-gleich
with-in
}

% listing within here does not have any effect for lfg.tex % 2020-05-14

% why has "erklärende" be listed here? I specified langid in bibtex item. Something is still not working with hyphenation.


% to do: check
%  Alahverdzhieva


% biblatex:

% This is a LaTeX frontend to TeX’s \hyphenation command which defines hy- phenation exceptions. The ⟨language⟩ must be a language name known to the babel/polyglossia packages. The ⟨text ⟩ is a whitespace-separated list of words. Hyphenation points are marked with a dash:

% \DefineHyphenationExceptions{american}{%
% hy-phen-ation ex-cep-tion }

\hyphenation{
A-la-hver-dzhie-va
ac-cu-sa-tive
anaph-o-ra
ana-phor
ana-phors
an-te-ced-ent
an-te-ced-ents
affri-ca-te
affri-ca-tes
ap-proach-es
Atha-bas-kan
Athe-nä-um
Be-schrei-bung
Bona-mi
Chi-che-ŵa
com-ple-ments
con-straints
Cope-sta-ke
Da-ge-stan
Dor-drecht
er-klä-ren-de
Flick-inger
Ginz-burg
Gro-ning-en
Has-pel-math
Jap-a-nese
Jon-a-than
Ka-tho-lie-ke
Ko-bon
krie-gen
Kroe-ger
Le-Sourd
moth-er
Mül-ler
Nie-mey-er
Ørs-nes
Par-a-digm
Prze-piór-kow-ski
phe-nom-e-non
re-nowned
Rie-he-mann
un-bound-ed
Ver-gleich
with-in
}

% listing within here does not have any effect for lfg.tex % 2020-05-14

% why has "erklärende" be listed here? I specified langid in bibtex item. Something is still not working with hyphenation.


% to do: check
%  Alahverdzhieva


% biblatex:

% This is a LaTeX frontend to TeX’s \hyphenation command which defines hy- phenation exceptions. The ⟨language⟩ must be a language name known to the babel/polyglossia packages. The ⟨text ⟩ is a whitespace-separated list of words. Hyphenation points are marked with a dash:

% \DefineHyphenationExceptions{american}{%
% hy-phen-ation ex-cep-tion }

  \memoizeset{
    memo filename prefix={hpsg-handbook.memo.dir/},
    % readonly
  }
  \papernote{\scriptsize\normalfont
    \@author.
    \@title. 
    To appear in: 
    Stefan Müller, Anne Abeillé, Robert D. Borsley \& Jean-Pierre Koenig (eds.)
    HPSG Handbook
    Berlin: Language Science Press. [preliminary page numbering]
  }
  \pagenumbering{roman}
  \setcounter{chapter}{#1}
  \addtocounter{chapter}{-1}
}
\makeatother




% In case that year is not given, but pubstate. This mainly occurs for titles that are forthcoming, in press, etc.
\renewbibmacro*{addendum+pubstate}{% Thanks to https://tex.stackexchange.com/a/154367 for the idea
  \printfield{addendum}%
  \iffieldequalstr{labeldatesource}{pubstate}{}
  {\newunit\newblock\printfield{pubstate}}
}

\DeclareLabeldate{%
    \field{date}
    \field{year}
    \field{eventdate}
    \field{origdate}
    \field{urldate}
    \field{pubstate}
    \literal{nodate}
}

%\defbibheading{diachrony-sources}{\section*{Sources}} 

% if no langid is set, it is English:
% https://tex.stackexchange.com/a/279302
\DeclareSourcemap{
  \maps[datatype=bibtex]{
    \map{
      \step[fieldset=langid, fieldvalue={english}]
    }
  }
}


% for bibliographies
% biber/biblatex could use sortname field rather than messing around this way.
\newcommand{\SortNoop}[1]{}


% Doug Ball

\newcommand{\elist}{\q<\ \ \q>}

\newcommand{\esetDB}{\q\{\ \ \q\}}


\makeatletter

\newcommand{\nolistbreak}{%

  \let\oldpar\par\def\par{\oldpar\nobreak}% Any \par issues a \nobreak

  \@nobreaktrue% Don't break with first \item

}

\makeatother


% intermediate before Frank's trees are fixed
% This will be removed!!!!!
%\newcommand{\tree}[1]{} % ignore them blody trees
%\usepackage{tree-dvips}


\newcommand{\nodeconnect}[2]{}
\newcommand{\nodetriangle}[2]{}



% Doug relative clauses
%% I've compiled out almost all my private LaTeX command, but there are some
%% I found hard to get rid of. They are defined here.
%% There are few others which defined in places in the document where they have only
%% local effect (e.g. within figures); their names all end in DA, e.g. \MotherDA
%% There are a lot of \labels -- they are all of the form \label{sec:rc-...} or
%% \label{x:rc-...} or similar, so there should be no clashes.

% Subscripts -- scriptsize italic shape lowered by .25ex 
\newcommand{\subscr}[1]{\raisebox{-.5ex}{\protect{\scriptsize{\itshape #1\/}}}}
% A boxed subscript, for avm tags in normal text
\newcommand{\subtag}[1]{\subscr{\idx{#1}}}

%% Sets and tuples: I use \setof{} to get brackets that are upright, not slanted
%\newcommand{\setof}[1]{\ensuremath{\lbrace\,\mathit{#1}\,\rbrace}}
% 11.10.2019 EP: Doug requested replacement of existing \setof definition with the following:
%\newcommand{\setof}[1]{\begin{avm}\{\textcolor{red}{#1}\}\end{avm}}
% 31.1.2019 EP: Doug requested re-replacement of the above \textcolour version with the following:
\newcommand{\setof}[1]{\begin{avm}\{#1\}\end{avm}}

\newcommand{\tuple}[1]{\ensuremath{\left\langle\,\mbox{\textit{#1}}\,\right\rangle}}

% Single pile of stuff, optional arugment is psn (e.g. t or b)
% e.g. to put a over b over c in a centered column, top aligned, do:
%   \cPile[t]{a\\b\\c} 
\newcommand{\cPile}[2][]{%
  \begingroup%
  \renewcommand{\arraystretch}{.5}\begin{tabular}[#1]{@{}c@{}}#2\end{tabular}%
  \endgroup%
}

%% for linguistic examples in running text (`linguistic citation'):
\newcommand{\lic}[1]{\textit{#1}}

%% A gap marked by an underline, raised slightly
%% Default argument indicates how long the line should be:
\newcommand{\uGap}[1][3ex]{\raisebox{.25em}{\underline{\hspace{#1}}}\xspace}

%% \TnodeDA{XP}{avmcontents} -- in a Tree, put a node label next to an AVM
\newcommand{\TnodeDA}[2]{#1~\begin{avm}{#2}\end{avm}}

%% This allows tipa stuff to be put in \emph -- we need to change to cmr first.
%% It is used in the discussion of Arabic.
\newcommand{\emphtipa}[1]{{\fontfamily{cmr}\emph{\tipaencoding #1}}} 



 
 
\definecolor{lsDOIGray}{cmyk}{0,0,0,0.45}


% morphology.tex:
% Berthold

\newcommand{\dnode}[1]{\rnode{#1}{\fbox{#1}}}
\newcommand{\tnode}[1]{\rnode{#1}{\textit{#1}}}

\newcommand{\tl}[2]{#2}

\newcommand{\rrr}[3]{%
  \psframebox[linestyle=none]{%
    \avmoptions{center}
    \begin{avm}
      \[mud & \{ #1 \}\\
      ms & \{ #2 \}\\
      mph & \<  #3 \> \]
    \end{avm}
  }
}
\newcommand{\rr}[2]{%
  \psframebox[linestyle=none]{%
    \avmoptions{center}
    \begin{avm}
      \[mud & \{ #1 \}\\
      mph & \<  #2 \> \]
    \end{avm}
  }
}
 

% Frank Richter
\newtheorem{mydef}{Definition}

\long\def\set[#1\set=#2\set]%
{%
\left\{%
\tabcolsep 1pt%
\begin{tabular}{l}%
#1%
\end{tabular}%
\left|%
\tabcolsep 1pt%
\begin{tabular}{l}%
#2%
\end{tabular}%
\right.%
\right\}%
}

\newcommand{\einruck}{\\ \hspace*{1em}}


%\newcommand{\NatNum}{\mathrm{I\hspace{-.17em}N}}
\newcommand{\NatNum}{\mathbb{N}}
\newcommand{\Aug}[1]{\widehat{#1}}
%\newcommand{\its}{\mathrm{:}}
% Felix 14.02.2020
\DeclareMathOperator{\its}{:}

\newcommand{\sequence}[1]{\langle#1\rangle}

\newcommand{\INTERPRETATION}[2]{\sequence{#1\mathsf{U}#2,#1\mathsf{S}#2,#1\mathsf{A}#2,#1\mathsf{R}#2}}
\newcommand{\Interpretation}{\INTERPRETATION{}{}}

\newcommand{\Inte}{\mathsf{I}}
\newcommand{\Unive}{\mathsf{U}}
\newcommand{\Speci}{\mathsf{S}}
\newcommand{\Atti}{\mathsf{A}}
\newcommand{\Reli}{\mathsf{R}}
\newcommand{\ReliT}{\mathsf{RT}}

\newcommand{\VarInt}{\mathsf{G}}
\newcommand{\CInt}{\mathsf{C}}
\newcommand{\Tinte}{\mathsf{T}}
\newcommand{\Dinte}{\mathsf{D}}

% this was missing from ash's stuff.

%% \def \optrulenode#1{
%%   \setbox1\hbox{$\left(\hbox{\begin{tabular}{@{\strut}c@{\strut}}#1\end{tabular}}\right)$}
%%   \raisebox{1.9ex}{\raisebox{-\ht1}{\copy1}}}



\newcommand{\pslabel}[1]{}

\newcommand{\addpagesunless}{\todostefan{add pages unless you cite the
 work as such}}

% dg.tex
% framed boxes as used in dg.tex
% original idea from stackexchange, but modified by Saso
% http://tex.stackexchange.com/questions/230300/doing-something-like-psframebox-in-tikz#230306
\tikzset{
  frbox/.style={
    rounded corners,
    draw,
    thick,
    inner sep=5pt,
    anchor=base,
  },
}

% get rid of these morewrite messages:
% https://tex.stackexchange.com/questions/419489/suppressing-messages-to-standard-output-from-package-morewrites/419494#419494
\ExplSyntaxOn
\cs_set_protected:Npn \__morewrites_shipout_ii:
  {
    \__morewrites_before_shipout:
    \__morewrites_tex_shipout:w \tex_box:D \g__morewrites_shipout_box
    \edef\tmp{\interactionmode\the\interactionmode\space}\batchmode\__morewrites_after_shipout:\tmp
  }
\ExplSyntaxOff


% This is for places where authors used bold. I replace them by \emph
% but have the information where the bold was. St. Mü. 09.05.2020
\newcommand{\textbfemph}[1]{\emph{#1}}



% Felix 09.06.2020: copy code from the third line into localcommands.tex:
% https://github.com/langsci/langscibook#defined-environments-commands-etc
% Does not work with texlive 2020, is done with sed in Makefile
%\patchcmd{\mkbibindexname}{\ifdefvoid{#3}{}{\MakeCapital{#3} }}{\ifdefvoid{#3}{}{#3 }}{}{\AtEndDocument{\typeout{mkbibindexname could not be patched.}}}



\let\textnobf\textit
% instead of "in bold" write "in italics"
\newcommand{\bolddescriptionintext}{italics\xspace}

% Berthold
\newcommand{\mathplus}{+}
% \mbox{\normalfont +}}
\newcommand{\emdash}{--\xspace}
\newcommand{\emdashUS}{--\xspace}


% Stefan to get the space remvoed infront of the : in Bargmann NPN discussion
%\DeclareMathSymbol{:}{\mathord}{operators}{"3A}
% used {:\,} instead


% for cxg.tex needed for includonly to find the counter.
\newcounter{croftyears} 




% Needed for bibtex entry for Jackendoff's xbar syntax. Without it the bar would be off in itialics.

% https://tex.stackexchange.com/questions/95014/aligning-overline-to-italics-font/95079#95079
% \newbox\usefulbox

% \makeatletter
%     \def\getslant #1{\strip@pt\fontdimen1 #1}

%     \def\skoverline #1{\mathchoice
%      {{\setbox\usefulbox=\hbox{$\m@th\displaystyle #1$}%
%         \dimen@ \getslant\the\textfont\symletters \ht\usefulbox
%         \divide\dimen@ \tw@ 
%         \kern\dimen@ 
%         \overline{\kern-\dimen@ \box\usefulbox\kern\dimen@ }\kern-\dimen@ }}
%      {{\setbox\usefulbox=\hbox{$\m@th\textstyle #1$}%
%         \dimen@ \getslant\the\textfont\symletters \ht\usefulbox
%         \divide\dimen@ \tw@ 
%         \kern\dimen@ 
%         \overline{\kern-\dimen@ \box\usefulbox\kern\dimen@ }\kern-\dimen@ }}
%      {{\setbox\usefulbox=\hbox{$\m@th\scriptstyle #1$}%
%         \dimen@ \getslant\the\scriptfont\symletters \ht\usefulbox
%         \divide\dimen@ \tw@ 
%         \kern\dimen@ 
%         \overline{\kern-\dimen@ \box\usefulbox\kern\dimen@ }\kern-\dimen@ }}
%      {{\setbox\usefulbox=\hbox{$\m@th\scriptscriptstyle #1$}%
%         \dimen@ \getslant\the\scriptscriptfont\symletters \ht\usefulbox
%         \divide\dimen@ \tw@ 
%         \kern\dimen@ 
%         \overline{\kern-\dimen@ \box\usefulbox\kern\dimen@ }\kern-\dimen@ }}%
%      {}}
%     \makeatother




\newcommand{\acknowledgmentsEN}{Acknowledgements}
\newcommand{\acknowledgmentsUS}{Acknowledgments}

% to put two examples next to eachother
%\newcommand{\shortbox}[3][-.7]{
%    \parbox[t]{.4\textwidth}{
%      \vspace{#1\baselineskip} #2\strut~~ #3}%
%}

\newcommand{\twomulticolexamples}[2]{
\begin{tabular}[t]{@{}l@{~~}l@{\hspace{1em}}l@{~~}l@{}}
a. & \parbox[t]{.4\textwidth}{#1} & b. & \parbox[t]{.4\textwidth}{#2}\\
\end{tabular}
}




% This does a linebreak for \gll for long sentences leaving space for the language at the right
% margin.
% St.Mü. 17.06.2021
\newcommand{\longexampleandlanguage}[2]{%
\begin{tabularx}{\linewidth}[t]{@{}X@{}p{\widthof{(#2)}}@{}}%
\begin{minipage}[t]{\linewidth}%
#1%
\end{minipage} & (\ili{#2})%
\end{tabularx}}



\renewcommand{\indexccg}{\is{Categorial Grammar (CG)!Combinatorial \textasciitilde{} (CCG)}\xspace}
\newcommand{\indexccgstart}{\is{Categorial Grammar (CG)!Combinatorial \textasciitilde{} (CCG)|(}\xspace}
\newcommand{\indexccgend}{\is{Categorial Grammar (CG)!Combinatorial \textasciitilde{} (CCG)|)}\xspace}
\renewcommand{\indexmp}{\is{Minimalism}\xspace}


\newcommand{\gisu}{Giuseppe Varaschin\xspace}

\newcommand{\NPi}{NP$\mkern-1mu_i$\xspace}
\newcommand{\NPj}{NP$\mkern-1.5mu_j$\xspace}
  %% -*- coding:utf-8 -*-

%%%%%%%%%%%%%%%%%%%%%%%%%%%%%%%%%%%%%%%%%%%%%%%%%%%%%%%%%%%%
%
% gb4e

% fixes problem with to much vertical space between \zl and \eal due to the \nopagebreak
% command.
\makeatletter
\def\@exe[#1]{\ifnum \@xnumdepth >0%
                 \if@xrec\@exrecwarn\fi%
                 \if@noftnote\@exrecwarn\fi%
                 \@xnumdepth0\@listdepth0\@xrectrue%
                 \save@counters%
              \fi%
                 \advance\@xnumdepth \@ne \@@xsi%
                 \if@noftnote%
                        \begin{list}{(\thexnumi)}%
                        {\usecounter{xnumi}\@subex{#1}{\@gblabelsep}{0em}%
                        \setcounter{xnumi}{\value{equation}}}
% this is commented out here since it causes additional space between \zl and \eal 06.06.2020
%                        \nopagebreak}%
                 \else%
                        \begin{list}{(\roman{xnumi})}%
                        {\usecounter{xnumi}\@subex{(iiv)}{\@gblabelsep}{\footexindent}%
                        \setcounter{xnumi}{\value{fnx}}}%
                 \fi}
\makeatother

% the texlive 2020 langsci-gb4e adds a newline after \eas, the texlive 2017 version was OK.
% \makeatletter
% \def\eas{\ifnum\@xnumdepth=0\begin{exe}[(34)]\else\begin{xlist}[iv.]\fi\ex\begin{tabular}[t]{@{}p{.98\linewidth}@{}}}
% \makeatother



%%%%%%%%%%%%%%%%%%%%%%%%%%%%%%%%%%%%%%%%%%%%%%%%%%%%%%%%%%
%
% biblatex

% biblatex sets the option autolang=hyphens
%
% This disables language shorthands. To avoid this, the hyphens code can be redefined
%
% https://tex.stackexchange.com/a/548047/18561

\makeatletter
\def\hyphenrules#1{%
  \edef\bbl@tempf{#1}%
  \bbl@fixname\bbl@tempf
  \bbl@iflanguage\bbl@tempf{%
    \expandafter\bbl@patterns\expandafter{\bbl@tempf}%
    \expandafter\ifx\csname\bbl@tempf hyphenmins\endcsname\relax
      \set@hyphenmins\tw@\thr@@\relax
    \else
      \expandafter\expandafter\expandafter\set@hyphenmins
      \csname\bbl@tempf hyphenmins\endcsname\relax
    \fi}}
\makeatother


% the package defined \attop in a way that produced a box that has textwidth
%
\def\attop#1{\leavevmode\begin{minipage}[t]{.995\linewidth}\strut\vskip-\baselineskip\begin{minipage}[t]{.995\linewidth}#1\end{minipage}\end{minipage}}


%%%%%%%%%%%%%%%%%%%%%%%%%%%%%%%%%%%%%%%%%%%%%%%%%%%%%%%%%%%%%%%%%%%%


% Don't do this at home. I do not like the smaller font for captions.
% This does not work. Throw out package caption in langscibook
% \captionsetup{%
% font={%
% stretch=1%.8%
% ,normalsize%,small%
% },%
% width=\textwidth%.8\textwidth
% }
% \setcaphanging


  \togglepaper[33]
}{}


\hyphenation{analy-sis}

\author{Richard Hudson\affiliation{University College London}}
\title{HPSG and Dependency Grammar}

% \chapterDOI{} %will be filled in at production

%\epigram{Change epigram in chapters/03.tex or remove it there }
\abstract{HPSG assumes Phrase Structure (PS), a partonomy, in contrast with Dependency Grammar (DG),
  which recognises Dependency Structure (DS), with direct relations between individual words and no
  multi-word phrases. The chapter presents a brief history of the two approaches, showing that DG
  matured in the late nineteenth century, long before the influential work by Tesnière, while Phrase
  Structure Grammar (PSG) started somewhat later with Bloomfield's enthusiastic adoption of Wundt's
  ideas. Since DG embraces almost as wide a range of approaches as PSG, the rest of the chapter
  focuses on one version of DG, Word Grammar. The chapter argues that classical DG needs to be
  enriched in ways that bring it closer to PSG: each dependent actually adds an extra node to the
  head, but the nodes thus created form a taxonomy, not a partonomy; coordination requires strings;
  and in some languages the syntactic analysis needs to indicate phrase boundaries. Another proposed
  extension to bare DG is a separate system of relations for controlling word order, which is
  reminiscent of the PSG distinction between dominance and precedence. The ``head-driven'' part of
  HPSG corresponds in Word Grammar to a taxonomy of dependencies which distinguishes grammatical
  functions, with complex combinations similar to HPSG's re-entrancy. The chapter reviews and
  rejects the evidence for headless phrases, and ends with the suggestion that HPSG could easily
  move from PS to DG.}
%\maketitle


\begin{document}
\maketitle
\label{chap-dg}


%%%%%%%%%%%%%%%%%%%%%%%%%%%%%%%%%%%%%%%%%%%%%%%%%%%%%
%%%%%%%%%%%%%%%%%%%%%%%%%%%%%%%%%%%%%%%%%%%%%%%%%%%%%
\section{Introduction}
\label{sec:1}

HPSG is firmly embedded, both theoretically and historically, in the phrase"=structure (PS) tradition of syntactic analysis, but it also has some interesting theoretical links to the dependency"=structure (DS) tradition. This is the topic of the present chapter, so after a very simple comparison of PS and DS and a glance at the development of these two traditions in the history of syntax, I consider a number of issues where the traditions interact.

The basis for PS analysis is the part-whole relation between smaller units (including words) and larger phrases, so the most iconic notation uses boxes \citep[6]{MuellerGT-Eng2}. In contrast, the basis for DS analysis is the asymmetrical dependency relation between two words, so in this case an iconic notation inserts arrows between words. (Although the standard notation in both traditions uses trees, these are less helpful because the lines are open to different interpretations.) The two analyses of a very simple sentence are juxtaposed in Figure~\ref{fig:1}. As in HPSG attribute-value matrices (AVMs), each rectangle represents a unit of analysis.

\begin{figure}
  \centering
  \begin{tikzpicture}[
      every node/.style=frbox,
      node distance=0.5em,
      font=\strut,
    ]
    \node(many) {many};
    \node(students) [base right=of many]{students};
    \node(enjoy) [base right=1.5em of students]{enjoy};
    \node(syntax) [base right=of enjoy]{syntax};
    \node(ms)[fit=(many)(students)]{};
    \node(es)[fit=(enjoy)(syntax)]{};
    \node[fit=(ms)(es)]{};
  \end{tikzpicture}

\vspace{\baselineskip}

\begin{tikzpicture}[node distance=.2cm]
\node[draw](many) at (0,0){\strut{many}};
\node[draw](students) [right=of many]{\strut students};
\node[draw](enjoy) [right=of students]{\strut enjoy};
\node[draw](syntax) [right= of enjoy]{\strut syntax};
\draw[->] (enjoy)[out=north,in=north] to (syntax);
\draw[->] (enjoy)[out=north,in=north] to (students);
\draw[->] ([xshift=-.2cm]students.north)[out=north,in=north] to (many);
\end{tikzpicture}
%
	\caption{Phrase structure and dependency structure contrasted}
	\label{fig:1}
\end{figure}

In both approaches, each unit has properties such as a classification, a meaning, a form and
relations to other items, but these properties may be thought of in two different ways. In PS
analyses, an item contains its related items, so it also contains its other properties – hence the
familiar AVMs contained within the box for each item. But in DS analyses, an item's related items
are outside it, sitting alongside it in the analysis, so, for consistency, other properties may be
shown as a network in which the item concerned is just one atomic node. This isn't the only possible
notation, but it is the basis for the main DS theory that I shall juxtapose with HPSG, Word Grammar.

What, then, are the distinctive characteristics of the two traditions? In the following summary I
use \emph{item} to include any syntagmatic unit of analysis including morphemes, words and phrases
(though this chapter will not discuss the possible role of morphemes). The following generalisations
apply to classic examples of the two approaches: PS as defined by Chomsky in terms of labelled
bracketed strings \citep{Chomsky57a}, and DS as defined by \citeauthor{Tesniere59a-u}
(\citeyear{Tesniere59a-u,Tesniere2015a-u}). These generalisations refer to ``direct relations'',
which are shown by single lines in standard tree notation; for example, taking a pair of words such
as \emph{big book}, they are related directly in DS, but only indirectly via a mother phrase in
PS. A phenomenon such as agreement is not a relation in this sense, but it applies to word-pairs
which are identified by their relationship; so even if two sisters agree, this does not in itself
constitute a direct relation between them.

\begin{enumerate}
\item\label{it:1} Containment: in PS, but not in DS, if two items are directly related, one must
  contain the other. For instance, a PS analysis of \emph{the book} recognises a direct relation (of
  dominance) between \emph{book} and \emph{the book}, but not between \emph{book} and \emph{the},
  which are directly related only by linear precedence. In contrast, a DS analysis does recognise a
  direct relation between \emph{book} and \emph{the} (in addition to the linear precedence
  relation).
	
\item\label{it:2} Continuity: therefore, in PS, but not in DS, all the items contained in a larger
  one must be adjacent.
	
\item\label{it:3} Asymmetry: in both DS and PS, a direct relation between two items must be
  asymmetrical, but in DS the relation (between two words) is dependency whereas in PS the relevant relation is the
  part-whole relation.
\end{enumerate}

These generalisations imply important theoretical claims which can be tested; for instance,
\ref{it:2} claims that there are no discontinuous phrases, which is clearly false. On the other
hand, \ref{it:3} claims that there can be no exocentric or headless phrases, so DS has to consider
apparent counter-examples such as the NPN construction, coordination and verbless sentences (see
Sections~\ref{sec:4.2} and~\ref{sec:5.1} for discussion, and also
\crossrefchapteralt{coordination}).

The contrasts in \ref{it:1}--\ref{it:3} apply without reservation to ``plain vanilla''
\citep{Zwicky1985} versions of DS and PS, but as we shall see in the history section, very few
theories are plain vanilla. In particular, there are versions of HPSG that allow phrases to be
discontinuous \citep{Reape94a,Kathol2000a,Mueller95c,Babel}. Nevertheless, the fact is that HPSG
evolved out of more or less pure PS, that it includes \emph{phrase structure} in its name, and that
it is never presented as a version of DS.

On the other hand, the term \emph{head-driven} points immediately to dependency: an asymmetrical
relation driven by a head word. Even if HPSG gives some constructions a headless analysis
\citep[654--666]{MuellerGT-Eng2}, the fact remains that it treats most constructions as headed.
This chapter reviews the relations between HPSG and the very long DS tradition of grammatical
analysis. The conclusion will be that in spite of its PS roots, HPSG implicitly (and sometimes even
explicitly) recognises dependencies; and it may not be a coincidence that one of the main
power-bases of HPSG is Germany, where the DS tradition is also at its strongest
\citep[359]{MuellerGT-Eng2}.

Where, then, does this discussion leave the notion of a phrase? In PS, phrases are basic units of
the analysis, alongside words; but even DS recognises phrases indirectly because they are easily
defined in terms of dependencies as a word plus all the words which depend, directly or indirectly,
on it. Although phrases play no part in a DS analysis, it is sometimes useful to be able to refer to
them informally (in much the same way that some PS grammars refer to grammatical functions
informally while denying them any formal status).

Why, then, does HPSG use PS rather than DS? As far as I know, PS was simply default syntax in the
circles where HPSG evolved, so the choice of PS isn't the result of a conscious decision by the
founders, and I hope that this chapter will show that this is a serious question which deserves
discussion.\footnote{% 
  Indeed, I once wrote a paper (which was never published) called ``Taking the PS out of HPSG'' – a
  title I was proud of until I noticed that PS was open to misreading, not least as ``Pollard and
  Sag''. Carl and Ivan took it well, and I think Carl may even have entertained the possibility that
  I might be right – possibly because he had previously espoused a theory called ``Head Grammar''
  (HG). See also \crossrefchapterw[Section~\ref{evolution:sec-head-grammar}]{evolution} on
    Head Grammar and the evolution of HPSG.

I hasten to add that while the PS view might have been the approach available at the time,
  there have been many researchers thinking carefully about issues concerning general phrase structure
  vs.\ dependency. For example, one general dependency structure is argued to be insufficient to
  account for complex predicates (\citealt{AG2010a-u}; \crossrefchapteralt{complex-predicates}) and negation
  (\citealt{KS2002a}; \crossrefchapteralt{negation}). See also \citew[Section~11.7]{MuellerGT-Eng4} for discussion of analyses of
  \citet{Eroms2000a}, \citet{GO2009a}, and others, and a general comparison of phrase structure and dependency approaches.
}


Unfortunately, the historical roots and the general dominance of PS have so far discouraged discussion of this fundamental question.

HPSG is a theoretical package where PS is linked intimately to a collection of other assumptions;
and the same is true for any theory which includes DS, including my own Word Grammar
\citep{Hudson84a-u,Hudson90a-u,Hudson1998,Hudson2007a-u,Hudson2010b-u,Gisborne2010,GisborneTBA,Duran-Eppler2011,TraugottTrousdale2013}. Among
the other assumptions of HPSG I find welcome similarities, not least the use of default inheritance
in some versions of the theory. I shall argue below that inheritance offers a novel solution to one
of the outstanding challenges for the dependency tradition.

The next section sets the historical scene. This is important because it's all too easy for students
to get the impression (mentioned above) that PS is just default syntax, and maybe even the same as
``traditional grammar''. We shall see that grammar has a very long and rather complicated history in
which the default is actually DS rather than PS. Later sections then address particular issues
shared by HPSG and the dependency tradition.


%%%%%%%%%%%%%%%%%%%%%%%%%%%%%%%%%%%%%%%%%%%%%%%%%%%%%
%%%%%%%%%%%%%%%%%%%%%%%%%%%%%%%%%%%%%%%%%%%%%%%%%%%%%
\section{Dependency and constituency in the history of syntax}
\label{sec:2}

The relevant history of syntax starts more than two thousand years ago in Greece. (Indian syntax may
have started even earlier, but it is hardly relevant because it had so little impact on the European
tradition.) Greek and Roman grammarians focused on the morphosyntactic properties of
individual words, but since these languages included a rich case system, they were aware of the
syntactic effects of verbs and prepositions governing particular cases. However, this didn't lead
them to think about syntactic relations, as such; precisely because of the case distinctions, they
could easily distinguish a verb's dependents in terms of their cases: ``its nominative'', ``its
accusative'' and so on \citep[29]{Robins1967}. Both the selecting verb or preposition and the item
carrying the case inflection were single words, so the Latin grammar of Priscian, written about 500
AD and still in use a thousand years later, recognised no units larger than the word: ``his model of
syntax was word-based – a dependency model rather than a constituency model''
\citep[91]{Law2003}. However, it was a dependency model without the notion of dependency as a
relation between words. 

The dependency relation, as such, seems to have been first identified by the Arabic grammarian
Sibawayh in the eighth century \citep{Owens1988,Kouloughli1999}. However, it is hard to rule out the
possibility of influence from the then"=flourishing Paninian tradition in India, and in any case it
doesn't seem to have had any more influence on the European tradition than did Panini's syntax, so
it is probably irrelevant. 

In Europe, grammar teaching in schools was based on parsing (in its original sense), an activity
which was formalised in the ninth century \citep{Luhtala1994}. The activity of parsing was a
sophisticated test of grammatical understanding which earned the central place in school work that
it held for centuries – in fact, right up to the 1950s (when I myself did parsing at school) and
maybe beyond. In HPSG terms, school children learned a standard list of attributes for words of
different classes, and in parsing a particular word in a sentence, their task was to provide the
values for its attributes, including its grammatical function (which would explain its case). In the
early centuries the language was Latin, but more recently it was the vernacular (in my case,
English). 

Alongside these purely grammatical analyses, the Ancient World had also recognised a logical one,
due to Aristotle, in which the basic elements of a proposition (\emph{logos}) are the logical
subject (\emph{onoma}) and the predicate (\emph{rhēma}). For Aristotle a statement such as
``Socrates ran'' requires the recognition both of the person Socrates and of the property of
running, neither of which could constitute a statement on its own \citep[30–31]{Law2003}. By the
twelfth century, grammarians started to apply a similar analysis to sentences; but in recognition of
the difference between logic and grammar they replaced the logicians' \emph{subiectum} and
\emph{praedicatum} by \emph{suppositum} and \emph{appositum} – though the logical terms would creep
into grammar by the late eighteenth century \citep[168]{Law2003}. This logical approach produced the
first top-down analysis in which a larger unit (the logician's proposition or the grammarian's
sentence) has parts, but the parts were still single words, so \emph{onoma} and \emph{rhēma} can now
be translated as ``noun'' and ``verb''. If the noun or verb was accompanied by other words, the
older dependency analysis applied. 

The result of this confusion of grammar with logic was a muddled hybrid analysis in the
Latin/Greek tradition which combines a headless subject"=predicate analysis with a headed
analysis elsewhere, and which persists even today in some school grammars; this confusion took
centuries to sort out in grammatical theory. For the subject and verb, the prestige of Aristotle and
logic supported a subject-verb division of the sentence (or clause) in which the subject noun and
the verb were both equally essential – a very different analysis from modern first-order logic in
which the subject is just one argument (among many) which depends on the predicate. Moreover the
grammatical tradition even includes a surprising number of analyses in which the subject noun is the
head of the construction, ranging from the modistic grammarians of the twelfth century
\citep[83]{Robins1967}, through Henry Sweet \citep[17]{Sweet1891}, to no less a figure than Otto
Jespersen in the twentieth \citep{Jespersen37a-u}, who distinguished ``junction'' (dependency) from
``nexus'' (predication) and treated the noun in both constructions as ``primary''. 

The first grammarians to recognise a consistently dependency-based analysis for the rest of the
sentence (but not for the subject and verb) seem to have been the French \emph{encyclopédistes} of
the eighteenth century \citep{Kahane2020a-u}, and, by the nineteenth century, much of Europe accepted a
theory of sentence structure based on dependencies, but with the subject-predicate analysis as an
exception – an analysis which by modern standards is muddled and complicated. Each of these units
was a single word, not a phrase, and modern phrases were recognised only indirectly by allowing the
subject and predicate to be expanded by dependents; so nobody ever suggested there might be such a
thing as a noun phrase until the late nineteenth century. Function words such as prepositions had no
proper position, being treated typically as though they were case inflections.

The invention of syntactic diagrams in the nineteenth century made the inconsistency of the hybrid
analysis obvious. The first such diagram was published in a German grammar of Latin for school
children \citep{Billroth1832}, and the nineteenth century saw a proliferation of diagramming
systems, including the famous Reed-Kellogg diagrams which are still taught (under the simple name
``diagramming'') in some American schools \citep{ReedKellog1890}; indeed, there is a website which
generates such diagrams, one of which is reproduced in Figure~\ref{fig:2}.% 
%
\footnote{See a small selection of diagramming systems at
  \url{http://dickhudson.com/sentence-diagramming/} (last access 2021-03-31), and the website is
  Sentence Diagrammer by 1aiway.} 
%
The significant feature of this diagram is the special treatment given to the relation between the
subject and predicate (with the verb \emph{are} sitting uncomfortably between the two), with all the
other words in the sentence linked by more or less straightforward dependencies. (The geometry of
these diagrams also distinguishes grammatical functions.) 
 
 \begin{figure}
 	\centering
\begin{tikzpicture}
\draw (0,0) to [edge label=sentences] (4,0);
\draw (4,.5) to (4,-.5);
\draw (2,0) to [edge label=~~~like,sloped] (3,-1);
\draw (3,-1) to [edge label=this] (5,-1);
\draw (4,0) to [edge label=are] (6,0);
\draw (6,0) to (5.5,.5);
\draw (6,0) to [edge label=easy,sloped] (8,0);
\draw (7,0) to [edge label=to,sloped] (8,-1);
\draw (8,-1) to [edge label=diagram] (10,-1);
\end{tikzpicture}
	\caption{Reed-Kellogg diagram by Sentence Diagrammer}
	\label{fig:2}
 \end{figure}
 
One particularly interesting (and relevant) fact about Reed and Kellogg is that they offer an analysis of \emph{that old wooden house} in which each modifier creates a new unit to which the next modifier applies: \emph{wooden house}, then \emph{old wooden house} \citep[18]{Percival1976} – a clear hint at more modern structures (including the ones proposed in Section~\ref{sec:4.1}), albeit one that sits uncomfortably with plain-vanilla dependency structure.
 
However, even in the nineteenth century, there were grammarians who questioned the hybrid tradition
which combined the subject-predicate distinction with dependencies. Rather remarkably, three
different grammarians seem to have independently reached the same conclusion at roughly the same
time: hybrid structures can be replaced by a homogeneous structure if we take the finite verb as the
root of the whole sentence, with the subject as one of its dependents. This idea seems to have been
first proposed in print in 1873 by the Hungarian Sámuel Brassai
\citep{Imrenyi2013a,ImrenyiVladar2020a-u}; in 1877 by the Russian Aleksej Dmitrievsky
\citep{Seriot2004}; and in 1884 by the German Franz Kern \citep{Kern1884a-u}. Both Brassai and
Kern used diagrams to present their analyses, and used precisely the same tree structures which
Lucien Tesnière in France called \emph{stemmas} nearly fifty years later
\citep{Tesniere59a-u,Tesniere2015a-u}. The diagrams have both been redrawn here as
Figures~\ref{fig:3} and~\ref{fig:4}.

\begin{figure}
	\centering
\begin{forest}
[\emph{tenebat}\\governing verb
	[\emph{flentem}\\dependent]
	[\emph{Uxor}\\dependent
		[\emph{amans}\\attribute]
		[\emph{ipsa}\\attribute]
		[\emph{flens}\\attribute
			[\emph{acrius}\\tertiary\\dependent]
		]
	]
	[$\overbrace{\emph{imbre cadente}}$\\dependent
		[\emph{usque}\\secondary\\dependent]
		[$\overbrace{\emph{per genas}}$\\secondary\\dependent
			[\emph{indignas}\\tertiary\\dependent]
		]
	]
]
\end{forest}
\caption{A verb-rooted tree published in 1873 by Brassai, quoted from \citew[\page 174]{ImrenyiVladar2020a-u}}\label{fig:3}
\end{figure}
 
Brassai's proposal is contained in a school grammar of Latin, so the example is also from Latin – an
extraordinarily complex sentence which certainly merits a diagram because the word order obscures the grammatical relations, which can be reconstructed only by paying attention to the morphosyntax. For example, \emph{flentem} and \emph{flens} both mean `crying', but their distinct case marking links them to different nouns, so the nominative \emph{flens} can modify nominative \emph{uxor} (woman), while the accusative \emph{flentem} defines a distinct individual glossed as `the crying one'.

\ea
\label{ex:1}
\gll Uxor am-ans fl-ent-em fl-ens acr-ius ips-a ten-eb-at, imbr-e per in-dign-as usque cad-ent-e gen-as.\\
     wife\textsc{.f.sg.nom} love\textsc{-ptcp.f.sg.nom} cry\textsc{-ptcp-m.sg.acc} cry\textsc{-ptcp.f.sg.nom} bitterly-more self\textsc{-f.sg.nom} hug\textsc{-pst-3sg} shower\textsc{-m.sg.abl} on un-becoming\textsc{-f.pl.acc} continuously fall\textsc{-ptcp-m.sg.abl} cheeks\textsc{-f.pl.acc}\\
%\gll Uxor amans flentem flens acr-ius       ipsa tenebat, imbre  per indignas   usque        cadente genas.\\
%     wife love  cry     cry   bitterly-more self hug      shower on  unbecoming continuously fall    cheeks\\
	\glt `The wife, herself even more bitterly crying, was hugging the crying one, while a shower [of tears] was falling on her unbecoming cheeks [i.e.\ cheeks to which tears are unbecoming].'
\z

Brassai's diagram, including grammatical functions as translated by the authors
\citep{ImrenyiVladar2020a-u}, is in Figure~\ref{fig:3}. The awkward horizontal braces should not be seen
as a nod in the direction of classical PS, given that the bracketed words are not even adjacent in
the sentence analysed. Kern's tree in Figure~\ref{fig:4}, on the other hand, is for the German
sentence in (\ref{ex:2}).

\ea
\label{ex:2}
\gll Ein-e               stolz-e                 Krähe                            schmück-t-e sich mit d-en aus-ge-fall-en-en Feder-n d-er Pfau-en.\\
     a\textsc{-f.sg.nom} proud\textsc{-f.sg.nom} crow(\textsc{f})\textsc{.sg.nom} decorate\textsc{-pst}\textsc{-3sg} self\textsc{.acc} with the\textsc{-pl.dat} out-\textsc{ptcp}-fall-\textsc{ptcp}-\textsc{pl.dat} feather\textsc{-pl.dat} the\textsc{-pl.gen} peacock-\textsc{pl.gen}\\\hfill(\ili{German})

%	\gll Eine stolze Krähe schmückte sich              mit  den ausgefallenen Federn der Pfauen.\\
%	     a    proud  crow  decorated self.\textsc{acc} with the.\textsc{dat} out-fallen.\dat{} feathers.\dat{} the.\textsc{gen} peacocks.\textsc{gen}\\
\glt `A proud crow decorated himself with the dropped feathers of the peacocks.'
\z

Once again, the original diagram includes function terms which are translated in this diagram into English.

\begin{figure}
	\centering
\begin{forest}
[finite verb\\\emph{schmückte}
	[subject word\\\emph{Krähe}
		[counter\\\emph{eine}]
		[attributive adjective\\\emph{stolze}]
	]
	[object\\\emph{sich}]
	[case with preposition\\\emph{mit Federn}
		[pointer\\\emph{den}]
		[attributive adjective\\(participle)\\\emph{ausgefallenen}]
		[genitive\\\emph{Pfauen}
			[pointer\\\emph{der}]
		]
	]
]
\end{forest}
\caption{A verb-rooted tree from \citet[\page 30]{Kern1884a-u}}
\label{fig:4}
\end{figure}


Once again the analysis gives up on prepositions, treating \emph{mit Federn} `with feathers' as a
single word, but Figure~\ref{fig:4} is an impressive attempt at a coherent analysis which would have
provided an excellent foundation for the explosion of syntax in the next century. According to the
classic history of dependency grammar, in this approach,

\begin{quotation} [\dots] the sentence is not a basic grammatical unit, but merely results from
  combinations of words, and therefore [\dots] the only truly basic grammatical unit is the word. A
  language, viewed from this perspective, is a collection of words and ways of using them in
  word-groups, i.e., expressions of varying length. \citep{Percival2007}
\end{quotation}

But the vagaries of intellectual history and geography worked against this intellectual
breakthrough. When Leonard Bloomfield was looking for a theoretical basis for syntax, he could have
built on what he had learned at school:

\begin{quotation} [\dots] we do not know and may never know what system of grammatical analysis
  Bloomfield was exposed to as a schoolboy, but it is clear that some of the basic conceptual and
  terminological ingredients of the system that he was to present in his 1914 and 1933 books were
  already in use in school grammars of English current in the United States in the nineteenth
  century. Above all, the notion of sentence ``analysis'', whether diagrammable or not, had been
  applied in those grammars. \citep{Percival2007}
\end{quotation}

And when he visited Germany in 1913--1914, he might have learned about Kern's ideas, which were
already influential there. But instead, he adopted the syntax of the German psychologist
Wilhelm Wundt. Wundt's theory applied to meaning rather than syntax, and was based on a single idea:
that every idea consists of a subject and a predicate. For example, a phrase meaning ``a sincerely
thinking person'' has two parts: \emph{a person} and \emph{thinks sincerely}; and the latter breaks
down, regardless of the grammar, into the noun \emph{thought} and \emph{is sincere}
\citep[\page 239]{Percival1976}.

For all its reliance on logic rather than grammar, the analysis is a clear precursor to
neo-Bloomfieldian trees: it recognises a single consistent part-whole relationship (a partonomy)
which applies recursively. This, then, is the beginning of the PS tradition: an analysis based
purely on meaning as filtered through a speculative theory of cognition – an unpromising start for a
theory of syntax. However, Bloomfield's school experience presumably explains why he combined
Wundt's partonomies with the hybrid structures of Reed-Kellogg diagrams in his classification of
structures as endocentric (headed) or exocentric (headless). For him, exocentric constructions
include the subject-predicate structure and preposition phrases, both of which were problematic in
sentence analysis at school. Consequently, his Immediate Constituent Analysis (ICA) perpetuated the
old hybrid mixture of headed and headless structures.

The DS elements of ICA are important in evaluating the history of PS, because they contradict the
standard view of history expressed here:

\begin{quotation}
  Within the Bloomfieldian tradition, there was a fair degree of consensus regarding the application
  of syntactic methods as well as about the analyses associated with different classes of
  constructions. Some of the general features of IC analyses find an obvious reflex in subsequent
  models of analysis. Foremost among these is the idea that structure involves a part–whole relation
  between elements and a larger superordinate unit, rather than an asymmetrical dependency relation
  between elements at the same level. \citep[202–203]{BlevinsSag2013}
\end{quotation}
%
This quotation implies, wrongly, that ICA rejected DS altogether.

What is most noticeable about the story so far is that, even in the 1950s, we still haven't seen an
example of pure phrase structure. Every theory visited so far has recognised dependency relations in
at least some constructions. Even Bloomfieldian ICA had a place for dependencies, though it
introduced the idea that dependents might be phrases rather than single words and it rejected the
traditional grammatical functions such as subject and object. Reacting against the latter gap, and
presumably remembering their schoolroom training, some linguists developed syntactic theories which
were based on constituent structure but which did have a place for grammatical functions, though not
for dependency as such. The most famous of these theories are Tagmemics \citep{Pike1954} and
Systemic Functional Grammar \citep{Halliday1961,Halliday67b-u}. However, in spite of its very
doubtful parentage and its very brief history, by the 1950s virtually every linguist in America
seemed to accept without question the idea that syntactic structure was a partonomy.

This is the world in which Noam Chomsky introduced phrase structure, which he presented as a
formalisation of ICA, arguing that ``customarily, linguistic description on the syntactic level is
formulated in terms of constituent analysis (parsing)'' \citep[26]{Chomsky57a}. But such analysis
was only ``customary'' among the Bloomfieldians, and was certainly not part of the classroom
activity of parsing \citep[147]{Matthews1993}.

Chomsky's phrase structure continued the drive towards homogeneity which had led to most of the
developments in syntactic theory since the early nineteenth century. Unfortunately, Chomsky
dismissed both dependencies and grammatical functions as irrelevant clutter, leaving nothing but
part-whole relations, category-labels, continuity and sequential order.

Rather remarkably, the theory of phrase structure implied the (psychologically implausible) claim
that sideways relations such as dependencies between individual words are impossible in a syntactic
tree – or at least that, even if they are psychologically possible, they can (and should) be ignored
in a formal model. Less surprisingly, having defined PS in this way, Chomsky could easily prove that
it was inadequate and needed to be greatly expanded beyond the plain-vanilla version. His solution
was the introduction of transformations, but it was only thirteen years before he also recognised
the need for some recognition of head-dependent asymmetries in X-bar theory \citep{Chomsky70a}. At
the same time, others had objected to transformations and started to develop other ways of making PS
adequate. One idea was to include grammatical functions; this idea was developed variously in LFG
\citep{Bresnan78a,Bresnan2001a}, Relational Grammar \citep{PP83a-u,Blake1990} and Functional Grammar
\citep{Dik1989,Siewierska1991}. Another way forward was to greatly enrich the categories
\citep{Harman63a} as in GPSG \citep{GKPS85a} and HPSG \citep{ps2}.

Meanwhile, the European ideas about syntactic structure culminating in Kern's tree diagram developed
rather more slowly. Lucien Tesnière in France wrote the first full theoretical discussion of DS in
1939, but it was not published till 1959 \citep{Tesniere59a-u,Tesniere2015a-u}, complete with
stemmas looking like the diagrams produced seventy years earlier by Brassai and Kern. Somewhat
later, these ideas were built into theoretical packages in which DS was bundled with various other
assumptions about levels and abstractness. Here the leading players were from Eastern Europe, where
DS flourished: the Russian Igor Mel’čuk \citep{Melcuk88a-u}, who combined DS with multiple
analytical levels, and the Czech linguists Petr Sgall, Eva Hajičová and Jarmila Panevova
\citep{Sgall&co1986}, who included information structure. My own theory Word Grammar (developed,
exceptionally, in the UK), also stems from the 1980s
\citep{Hudson84a-u,Hudson90a-u,Sugayama2002,Hudson2007a-u,Gisborne2008,Rosta2008,Gisborne2010,Hudson2010b-u,Gisborne2011,Duran-Eppler2011,TraugottTrousdale2013,Duran-Eppler&co2016,Hudson2016,Hudson2017,Hudson2018a,Gisborne2019}. This
is the theory which I compare below with HPSG, but it is important to remember that other DS
theories would give very different answers to some of the questions that I raise.

DS certainly has a low profile in theoretical linguistics, and especially so in anglophone countries, but there is an area of linguistics where its profile is much higher (and which is of particular interest to the HPSG community): natural-language processing \citep{KMcDN2009a-u}. For example:

\begin{itemize}
	\item the Wikipedia entry for ``Treebank'' classifies 228 of its 274 treebanks as using DS.%
	%
	\footnote{\url{https://en.wikipedia.org/wiki/Treebank} (last access 2021-04-06).}%
	%
	
	
	\item The ``Universal dependencies'' website lists almost 200 dependency-based treebanks
          for over 100 languages.%
	%
	\footnote{\url{https://universaldependencies.org/} (last access January 2021-04-06).}%
	%
	
	
	\item Google's n-gram facility allows searches based on dependencies.%
	%
	\footnote{\url{https://books.google.com/ngrams/info} and search for ``dependency'' (last access 2021-04-06).}%
	%
	
	
	\item The Stanford Parser \citep{ChenManning2014,deMarneffe&co2014} uses DS.%
	%
	\footnote{\url{https://nlp.stanford.edu/software/stanford-dependencies.shtml} (last access 2021-04-06).}%
	%
\end{itemize}

The attraction of DS in NLP is that the only units of analysis are words, so at least these units are given in the raw data and the overall analysis can immediately be broken down into a much simpler analysis for each word. This is as true for a linguist building a treebank as it was for a school teacher teaching children to parse words in a grammar lesson. Of course, as we all know, the analysis actually demands a global view of the entire sentence, but at least in simple examples a bottom-up word-based view will also give the right result.

To summarise this historical survey, PS is a recent arrival, and is not yet a hundred years old. Previous syntacticians had never considered the possibility of basing syntactic analysis on a partonomy. Instead, it had seemed obvious that syntax was literally about how words (not phrases) combined with one another.


%%%%%%%%%%%%%%%%%%%%%%%%%%%%%%%%%%%%%%%%%%%%%%%%%%%%%
%%%%%%%%%%%%%%%%%%%%%%%%%%%%%%%%%%%%%%%%%%%%%%%%%%%%%
\section{HPSG and Word Grammar}
\label{sec:3}

The rest of this chapter considers a number of crucial issues that differentiate PS from DS by
focusing specifically on how they distinguish two particular manifestations of these traditions,
HPSG and Word Grammar (WG). The main question is, of course, how strong the evidence is for the PS
basis of HPSG, and how easily this basis could be replaced by DS.

The comparison requires some understanding of WG, so what follows is a brief tutorial on the parts
of the theory which will be relevant in the following discussion. Like HPSG, WG combines claims
about syntactic relations with a number of other assumptions; but for WG, the main assumption is the
Cognitive Principle:

\eanoraggedright
\label{it:CogPrin}
The Cognitive Principle:\\
Language uses the same general cognitive processes  and resources as general cognition, and has access to all of them.
\z
%
This principle is of course merely a hypothesis which may turn out to be wrong, but so far it seems
correct \citep[494]{MuellerGT-Eng2}, and it is more compatible with HPSG than with the innatist
ideas underlying Chomskyan linguistics \citep*{Berwick:Chomsky2013a-u}. In WG, it plays an important
part because it determines other parts of the theory.

On the one hand, cognitive psychologists tend to see knowledge as a network of related concepts
\citep[252]{Reisberg2007}, so WG also assumes that the whole of language, including grammar, is a
conceptual network (\citealt[1]{Hudson84a-u}; \citeyear[1]{Hudson2007a-u}). One of the consequences
is that the AVMs of HPSG are presented instead as labelled network links; for example, we can
compare the elementary example in (\mex{1}) of the HPSG lexical item for a \ili{German} noun
\citep[264]{MuellerGT-Eng2} with an exact translation using WG notation.

\ea
\label{fig:5}
AVM for the \ili{German} noun \emph{Grammatik}:\\
\avm{
[\type*{word}
phonology & <\type{Grammatik}\,> \\
syntax-semantics  \ldots &	[\type*{local}
							category &	[\type*{category}
										head &	[\type*{noun}
												case & \1] \\
										spr & < Det![case \1]! > \\
										\ldots ] \\
							content & \ldots[\type*{grammatik}
											inst & X ] ] ]
}
%\vspace{-\baselineskip}
\z

HPSG regards AVMs as equivalent to networks, so translating this AVM into network notation is
straightforward; however, it is visually complicated, so I take it in two steps. First I introduce
the basic notation in Figure~\ref{fig:6}: a small triangle showing that the lexeme
\textsc{grammatik} ``isa'' word, and a headed arrow representing a labelled attribute (here,
``phonology'') and pointing to its value. The names of entities and attributes are enclosed in
rectangles and ellipses respectively.

\begin{figure}
	\centering
\begin{tikzpicture}[node distance=3cm]
\node[draw](word) at (0,1.5){word};
\node[draw](grammatik) at (0,0){\textsc{grammatik}};
\node[draw,ellipse](phonology) [right of=grammatik]{phonology};
\node[draw](Grammatik) [right of=phonology]{\emph{Grammatik}};
\draw[<-,>=open triangle 90 reversed] (word) to (grammatik);
\draw (phonology) to (grammatik);
\draw[->] (phonology) to (Grammatik);
\end{tikzpicture}
	\caption{The German noun \emph{Grammatik} `grammar' in a WG network}
	\label{fig:6}
\end{figure}

The rest of the AVM translates quite smoothly (ignoring the list for \spr), giving Figure~\ref{fig:7}, though an actual WG analysis would be rather different in ways that are irrelevant here.

\begin{figure}
	\centering
\begin{tikzpicture}[node distance=1.8cm]
%relative placement of nodes starting from 1
\node[draw](1) at (0,0){1};
\node[draw,ellipse](case1)[above left of=1]{case};
\node[draw,ellipse](case2)[above right of=1]{case};
\node[draw](noun)[above left of=case1]{\emph{noun}};
\node[draw](det)[above right of=case2]{\emph{det}};
\node[draw,ellipse](head)[above right of=noun]{head};
\node[draw,ellipse](spr)[above left of=det]{spr};
\node[draw](category1)[above left of=spr]{\emph{category}};
\node[draw,ellipse](category2)[above right of=category1]{category};
\node[draw](local)[above right of=category2]{\emph{local}};
\node[draw,ellipse](content)[below right of=local]{content};
\node[draw](grammatik1)[below right of=content]{\emph{grammatik}};
\node[draw,ellipse](inst)[below of=grammatik1]{inst};
\node[draw](X)[below of=inst]{X};
\node[draw,ellipse](synsem)[above of=local]{syntax-semantics};
\node[draw](grammatik2)[above of=synsem]{\textsc{grammatik}};
%arrows (I didnt manage to find a way to whitefill the ellipses afterwards, so every arrow is split up :/)
\draw (grammatik2) to (synsem);
\draw[->] (synsem) to (local);
\draw (local) to (category2);
\draw (local) to (content);
\draw[->] (category2) to (category1);
\draw[->] (content) to (grammatik1);
\draw (grammatik1) to (inst);
\draw[->] (inst) to (X);
\draw (category1) to (head);
\draw (category1) to (spr);
\draw[->] (head) to (noun);
\draw[->] (spr) to (det);
\draw (noun) to (case1);
\draw (det) to (case2);
\draw[->] (case1) to (1);
\draw[->] (case2) to (1);
\end{tikzpicture}
	\caption{The German noun \emph{Grammatik} `grammar' in a WG network}
	\label{fig:7}
\end{figure}

The other difference based on cognitive psychology between HPSG and WG is that many cognitive psychologists argue that concepts are built around prototypes \citep{Rosch1973,Taylor1995}, clear cases with a periphery of exceptional cases. This claim implies the logic of default inheritance \citep{BCdP93a-ed}, which is popular in AI, though less so in logic. In HPSG, default inheritance is accepted by some \citep{LC99a} but not by others \citep[403]{MuellerGT-Eng2}, whereas in WG it plays a fundamental role, as I show in Section~\ref{sec:4.1} below. WG uses the \rel{isa} relation to carry default inheritance, and avoids the problems of non-monotonic inheritance by restricting inheritance to node-creation \citep[18]{Hudson2018a}. Once again, the difference is highly relevant to the comparison of PS and DS because one of the basic questions is whether syntactic structures involve partonomies (based on whole:part relations) or taxonomies (based on the \rel{isa} relation). (I argue in Section~\ref{sec:4.1} that taxonomies exist within the structure of a sentence thanks to \rel{isa} relations between tokens and sub-tokens.)

Default inheritance leads to an interesting comparison of the ways in which the two theories treat attributes. On the one hand, they both recognise a taxonomy in which some attributes are grouped together as similar; for example, the HPSG analysis in (\ref{fig:5}) classifies the attributes \textsc{category} and \textsc{content} as \textsc{local}, and within \textsc{category} it distinguishes the \textsc{head} and \textsc{specifier} attributes. In WG, attributes are called relations, and they too form a taxonomy. The simplest examples to present are the traditional grammatical functions, which are all subtypes of ``dependent''; for example, ``object'' \rel{isa} ``complement'', which \rel{isa} ``valent'', which \rel{isa} ``dependent'', as shown in Figure~\ref{fig:8} (which begs a number of analytical questions such as the status of depictive predicatives, which are not complements).

\begin{figure}
	\centering
\begin{tikzpicture}[>=open triangle 90 reversed]
\node[shape=ellipse,draw](object) at (0,0) {object};
\node[shape=ellipse,draw](complement) at (0,1.5) {complement};
\node[shape=ellipse,draw](valent) at (0,3) {\strut{valent}};
\node[shape=ellipse,draw](dependent) at (0,4.5) {dependent};
\node[shape=ellipse,draw](predicative) at (3,0) {predicative};
\node[shape=ellipse,draw](subject) at (3,1.5) {subject};
\node[shape=ellipse,draw](adjunct) at (3,3) {adjunct};
\draw[<-] (dependent) to (valent);
\draw (adjunct) to ([yshift=-.13cm]dependent.south);
\draw[<-] (valent) to (complement);
\draw (subject) to ([yshift=-.13cm]valent.south);
\draw[<-] (complement) to (object);
\draw (predicative) to ([yshift=-.13cm]complement.south);
\end{tikzpicture}
	\caption{A WG taxonomy of grammatical functions}
	\label{fig:8}
\end{figure}

In spite of the differences in the categories recognised, the formal similarity is striking. On the
other hand, there is also an important formal difference in the roles played by these taxonomies. In
spite of interesting work on default inheritance \citep{LC99a}, most versions of HPSG allow
generalisations but not exceptions (``If one formulates a restriction on a supertype, this
automatically affects all of its subtypes''; \citealt[275]{MuellerGT-Eng2}), whereas in WG the usual
logic of default inheritance applies so exceptions are possible. These are easy to illustrate from
word order, which (as explained in Section~\ref{sec:4.4}) is normally inherited from dependencies: a
verb's subject normally precedes it, but an inverted subject (the subject of an inverted auxiliary
verb, as in \emph{did he}) follows it.

Another reason for discussing default inheritance and the \rel{isa} relation is to explain that WG,
just like HPSG, is a constraint-based theory. In HPSG, a sentence is grammatical if it can be
modelled given the structures and lexicon provided by the grammar, which are combined with each
other by inserting less complex structures into daughter slots of more complex
structures. Similarly, in WG it is grammatical if its word tokens can all be inherited from entries
in the grammar (which also includes the entire lexicon). Within the grammar, these may involve
overrides, but overrides between the grammar and the word tokens imply some degree of
ungrammaticality. For instance, \emph{He slept} is grammatical because all the properties of
\emph{he} and \emph{slept} (including their syntactic properties such as the word order that can be
inherited from their grammatical function) can be inherited directly from the grammar, whereas
*\emph{Slept he} is ungrammatical in that the order of words is exceptional, and the exception is
not licensed by the grammar.

This completes the tutorial on WG, so we are now ready to consider the issues that distinguish HPSG
from this particular version of DS. In preparation for this discussion, I return to the three
distinguishing assumptions about classical PS and DS theories given earlier as~\ref{it:1}
to~\ref{it:3}, and repeated here:

\begin{enumerate}
	\item Containment: in PS, but not in DS, if two items are directly related, one must contain the other.
	
	\item Continuity: therefore, in PS, but not in DS, all the items contained in a larger one must be adjacent.
	
	\item Asymmetry: in both DS and PS, a direct relation between two items must be asymmetrical, but in DS the relation (between two words) is dependency whereas in PS it is the part-whole relation.
\end{enumerate}

These distinctions will provide the structure for the discussion:

\begin{itemize}
	\item  Containment and continuity:
	\begin{itemize}
		\item semantic phrasing
		
		\item coordination
		
		\item phrasal edges
		
		\item word order
	\end{itemize}

	\item Asymmetry:
	\begin{itemize}
		\item structure sharing and raising/lowering
		
		\item headless phrases
		
		\item complex dependency
		
		\item grammatical functions
	\end{itemize}
\end{itemize}


%%%%%%%%%%%%%%%%%%%%%%%%%%%%%%%%%%%%%%%%%%%%%%%%%%%%%
%%%%%%%%%%%%%%%%%%%%%%%%%%%%%%%%%%%%%%%%%%%%%%%%%%%%%
\section{Containment and continuity (PS but not DS)}
\label{sec:4}

%%%%%%%%%%%%%%%%%%%%%%%%%%%%%%%%
\subsection{Semantic phrasing}
\label{sec:4.1}

One apparent benefit of PS is what I call ``semantic phrasing'' \citep[146–151]{Hudson90a-u}, in
which the effect of adding a dependent to a word modifies that word's meaning to produce a different
meaning. For instance, the phrase \emph{typical French house} does not mean `house which is both
typical and French', but rather `French house which is typical (of French houses)'
\citep[\page 486]{Dahl80a}. In other words, even if the syntax does not need a node corresponding to the
combination \emph{French house}, the semantics does need one.

For HPSG, of course, this is not a problem, because every dependent is part of a new structure,
semantic as well as syntactic \citep{MuellerEvaluating}; so the syntactic phrase \emph{French house}
has a content which is `French house'. But for DS theories, this is not generally possible, because
there is no syntactic node other than those for individual words – so, in this example, one node for
\emph{house} and one for \emph{French} but none for \emph{French house}.

Fortunately for DS, there is a solution: create extra word nodes but treat them as a taxonomy, not a
partonomy \citep{Hudson2018a}. To appreciate the significance of this distinction, the connection
between the concepts ``finger'' and ``hand'' is a partonomy, but that between ``index finger'' and
``finger'' is a taxonomy; a finger is part of a hand, but it is not a hand, and conversely an index
finger is a finger, but it is not part of a finger.

In this analysis, then, the token of \emph{house} in \emph{typical French house} would be factored into three distinct nodes:

\begin{itemize}
\item \emph{house}: \label{it:house} an example of the lexeme \textsc{house}, with the inherited meaning `house'.
	
\item \emph{house+F}: \label{it:house+f} the word \emph{house} with \emph{French} as its dependent, meaning `French house'.
	
\item \emph{house+t}: \label{it:house+t} the word \emph{house+F} with \emph{typical} as its dependent, meaning `typical example of a French house'.
\end{itemize}

\noindent
(It is important to remember that the labels are merely hints to guide the analyst, and not part of
the analysis; so the last label could have been \emph{house+t+F} without changing the analysis at
all. One of the consequences of a network approach is that the only substantive elements in the
analysis are the links between nodes, rather than the labels on the nodes.) These three nodes can be
justified as distinct categories because each combines a syntactic fact with a semantic one: for
instance, \emph{house} doesn't simply mean `French house', but has that meaning because it has the
dependent \emph{French}. The alternative would be to add all the dependents and all the meanings to
a single word node as in earlier versions of WG \citep[146–151]{Hudson90a-u}, thereby removing all
the explanatory connections; this seems much less plausible psychologically. The proposed WG
analysis of \emph{typical French house} is shown in Figure~\ref{fig:9}, with the syntactic structure
on the left and the semantics on the right.

\begin{figure}
	\centering
\begin{tikzpicture}[node distance=2cm]
%first row
\node[draw](typical) at (0,0){\emph{typical}};
\node[draw](french)[right of=typical]{\emph{French}};
\node[draw](house1)[right of=french]{\emph{house}};
\node[draw,ellipse](sense1)[right of=house1]{sense};
\node[draw](house2)[right of=sense1]{`house'};
%second row
\node[draw](housef)[above of=house1]{\emph{house}+\emph{F}};
\node[draw,ellipse](sense2)[right of=housef]{sense};
\node[draw, align=center](fhouse)[above of=house2]{`French\\house'};
%third row
\node[draw](houset)[above of=housef]{\emph{house}+\emph{t}};
\node[draw,ellipse](sense3)[right of=houset]{sense};
\node[draw, align=center](tfhouse)[above of=fhouse]{`typical\\French\\house'};
%arrows
\draw[<-,>=open triangle 90 reversed] (house1) to (housef);
\draw[<-,>=open triangle 90 reversed] (housef) to (houset);
\draw[<-,>=open triangle 90 reversed] (house2) to (fhouse);
\draw[<-,>=open triangle 90 reversed] (fhouse) to (tfhouse);
\draw (house1) to (sense1);
\draw (housef) to (sense2);
\draw (houset) to (sense3);
\draw[->] (sense1) to (house2);
\draw[->] (sense2) to (fhouse);
\draw[->] (sense3) to (tfhouse);
\draw[->] (housef) to[out=west,in=north] (french);
\draw[->] (houset) to[out=west,in=north] (typical);
\end{tikzpicture}
	\caption{\emph{typical French house} in WG}
	\label{fig:9}
\end{figure}

Unlike standard DG analyses \citep{MuellerEvaluating}, the number of syntactic nodes in this
analysis is the same as in an HPSG analysis, but crucially these nodes are linked by the \rel{isa}
relation, and not as parts to wholes – in other words, the hierarchy is a taxonomy, not a
partonomy. As mentioned earlier, the logic is default inheritance, and the default semantics has
\rel{isa} links parallel to those in syntax; thus the meaning of \emph{house+F} (\emph{house} as
modified by \emph{French}) \rel{isa} the meaning of \emph{house} – in other words, a French house is
a kind of house. But the default can be overridden by exceptions such as the meanings of adjectives
like \emph{fake} and \emph{former}, so a fake diamond is not a diamond (though it looks like one)
and a former soldier is no longer a soldier.\footnote{
See also \crossrefchapterw[Section~\ref{semantics-sec-adjunct-scope}]{semantics} on adjunct scope.
} The exceptional semantics is licensed by the grammar –
the stored network – so the sentence is fully grammatical. All this is possible because of the same
default inheritance that allows irregular morphology and syntax.

%%%%%%%%%%%%%%%%%%%%%%%%%%%%%%%%
\subsection{Coordination}
\label{sec:4.2}

Another potential argument for PS, and against DS, is based on coordination: coordination is a
symmetrical relationship, not a dependency, and it coordinates phrases rather than single words. For
instance, in (\ref{ex:3}) the coordination clearly links the VPs \emph{came in} to \emph{sat down}
and puts them on equal grammatical terms; and it is this equality that allows them to share the
subject \emph{Mary}.

\begin{exe}
	\ex \label{ex:3} Mary came in and sat down.
\end{exe}
%
But of course, in a classic DS analysis \emph{Mary} is also attached directly to \emph{came},
without an intervening VP node, so \emph{came in} is not a complete syntactic item and this approach
to coordination fails, so we have a prima facie case against DS. (For coordination in HPSG, see
\crossrefchapteralt{coordination}.)

Fortunately, there is a solution: sets \citep[404--421]{Hudson90a-u}. We know from the vast
experimental literature (as well as from everyday experience) that the human mind is capable of
representing ordered sets (strings) of words, so all we need to assume is that we can apply this
ability in the case of coordination. The members of a set are all equal, so their relation is
symmetrical; and the members may share properties (e.g.\ a person's children constitute a set united
by their shared relation to that person as well as by a multitude of other shared
properties). Moreover, sets may be combined into supersets, so both conjuncts such as \emph{came in}
and \emph{sat down} and coordinations (\emph{came in and sat down}) are lists. According to
this analysis, then, the two lists (\emph{came}, \emph{in}) and (\emph{sat}, \emph{down}) are united
by their shared subject, Mary, and combine into the coordination ((\emph{came}, \emph{in})
(\emph{sat}, \emph{down})). The precise status of the conjunction \emph{and} remains to be
determined. The proposed analysis is shown in network notation in Figure~\ref{fig:10}.

\begin{figure}
	\centering
\begin{forest}
for tree={l sep=2cm}
[((\emph{came}{,} \emph{in}){,} (\emph{sat}{,} \emph{down})),s sep=1.25cm, l sep=1.5cm,draw,name=cisd
	[(\emph{came}{,} \emph{in}),draw,name=ci
		[\emph{came},draw,name=came]
		[\emph{in},draw,name=in]
	]
	[(\emph{sat}{,} \emph{down}),draw,name=sd
		[\emph{sat},draw,name=sat]
		[\emph{down},draw,name=down]
	]
]
\node[draw](and)[right of=in]{\strut\emph{and}};
\node[draw](mary)[left of=came,node distance=1.3cm]{\strut\emph{Mary}};
\draw[->](came) to[out=north,in=north]([xshift=-.2cm]in.north);
\draw[->](sat) to[out=north,in=north]([xshift=-.2cm]down.north);
\draw[->](came) to[out=north,in=north]([xshift=.4cm]mary.north);
\draw[->,dashed](sat) to[out=north,in=north]([xshift=.2cm]mary.north);
\draw[->,dashed](sd.west) to[out=south west,in=north](mary);
\draw[->](ci) to[out=west,in=north]([xshift=-.2cm]mary.north);
\draw[->](cisd) to[out=west,in=north]([xshift=-.4cm]mary.north);
\end{forest}
	\caption{Coordination with sets}
	\label{fig:10}
\end{figure}

Once again, inheritance plays a role in generating this diagram. The \rel{isa} links have been
omitted in Figure~\ref{fig:10} to avoid clutter, but they are shown in Figure~\ref{fig:11}, where
the extra \rel{isa} links are compensated for by removing all irrelevant matter and the dependencies
are numbered for convenience. In this diagram, the dependency d1 from \emph{came} to \emph{Mary} is
the starting point, as it is established in processing during the processing of \emph{Mary came} –
long before the coordination is recognised; and the endpoint is the dependency d5 from \emph{sat} to
\emph{Mary}, which is simply a copy of d1, so the two are linked by isa. (It will be recalled from
Figure~\ref{fig:8} that dependencies form a taxonomy, just like words and word classes, so \rel{isa}
links between dependencies are legitimate.) The conjunction \emph{and} creates the three set nodes,
and general rules for sets ensure that properties – in this case, dependencies – can be shared by
the two conjuncts.

It's not yet clear exactly how this happens, but one possibility is displayed in the diagram: d1
licenses d2 which licenses d3 which licenses d4 which licenses d5. Each of these licensing relations
is based on isa. Whatever the mechanism, the main idea is that the members of a set can share a
property; for example, we can think of a group of people sitting in a room as a set whose members
share the property of sitting in the room. Similarly, the set of strings \emph{came in} and
\emph{sat down} share the property of having \emph{Mary} as their subject.

\begin{figure}
	\centering
\begin{tikzpicture}[node distance=2.5cm]
%first row
\node[draw](mary) at(0,0){\emph{\strut Mary}};
\node[draw](came)[right of=mary]{\emph{\strut came}};
\node[draw](sat)[right of=came, node distance=3cm]{\emph{\strut sat}};
%other nodes
\node[draw](ci) at (3,2.8){(\emph{came}, \emph{in})};
\node[draw](sd) at (6,4){(\emph{sat}, \emph{down})};
\node[draw](cisd)[above left of=sd, node distance=2cm]{((\emph{came}, \emph{in}), (\emph{sat}, \emph{down}))};
%arrows
\draw[->] (came) to[in=north, out=north] node[draw,ellipse,fill=white](d1){d1} ([xshift=.4cm]mary.north);
\draw[->] (ci) to[in=north, out=west] node[draw,pos=.2,ellipse,fill=white](d2){d2} (mary);
\draw[->] (cisd) to[in=north, out=west] node[draw,pos=.214,ellipse,fill=white](d3){d3} ([xshift=-.4cm]mary.north);
%
\draw[dashed,->] (sd) to[in=north, out=west] node[draw,pos=.095, ellipse,solid,fill=white](d4){d4} (-.2,1)% here could go the specs of the triangle thingy
to ([xshift=-.2cm]mary.north);
\draw[dashed,->] (sat) to[in=north, out=north] node[draw,pos=.305,ellipse,solid,fill=white](d5){d5}
%(x,y)% triangle specs
([xshift=.2cm]mary.north);
%
\draw[<-,>=open triangle 90 reversed] (d1) to (d2);
\draw[<-,>=open triangle 90 reversed] (d2) to (d3);
\draw[<-,>=open triangle 90 reversed] (d3) to (d4);
\draw[<-,>=open triangle 90 reversed] (d4) to (d5);
\end{tikzpicture}
	\caption{Coordination with inherited dependencies}
	\label{fig:11}
%\itdblue{JP: IMHO, this is amphibological use of ISA, as the relation between a member of a set and a set (in Figure 10), is not an ISA relation.}
\end{figure}

The proposed analysis may seem to have adopted phrases in all but name, but this is not so because
the conjuncts have no grammatical classification, so coordination is not restricted to coordination
of like categories. This is helpful with examples like (\ref{ex:4}) where an adjective is
coordinated with an NP and a PP.

\begin{exe}
	\ex \label{ex:4} Kim was intelligent, a good linguist and in the right job.
\end{exe}

The possibility of coordinating mixed categories is a well-known challenge for PS-based analyses
such as HPSG: ``Ever since Sag et al. (1985), the underlying intuition was that what makes
Coordination of Unlikes acceptable is that each conjunct is actually well-formed when combined
individually with the shared rest'' \citep[61]{Crysmann2003c}. Put somewhat more precisely, the
intuition is that what coordinated items share is not their category but their function
\citep[414]{Hudson90a-u}. This is more accurate because simple combinability isn't enough; for
instance, \emph{we ate} can combine with an object or with an adjunct, but the functional difference
prevents them from coordinating:

\begin{exe}
	\ex[]{We ate a sandwich.}\label{ex:5}

	\ex[]{We ate at midday.}\label{ex:6}

	\ex[*]{We ate a sandwich and at midday.}\label{ex:7}
\end{exe}

\noindent
Similarly, \emph{a linguist} can combine as dependent with many verbs, but these can only coordinate
if their relation to \emph{a linguist} is the same:

\begin{exe}
	\ex[]{She became a linguist.}\label{ex:8}

	\ex[]{She met a linguist.}\label{ex:9}

	\ex[*]{She became and met a linguist.}\label{ex:10}
\end{exe}

\noindent
It is true that HPSG can accommodate the coordination of unlike categories by redefining categories
so that they define functions rather than traditional categories; for example, if ``predicative'' is
treated as a category, then the problem of (\ref{ex:4}) disappears because \emph{intelligent},
\emph{a good linguist} and \emph{in the right job} all belong to the category
``predicative''. However, this solution generates as many problems as it solves. For example, why is
the category ``predicative'' exactly equivalent to the function with the same name, whereas
categories such as ``noun phrase'' have multiple functions? And how does this category fit into a
hierarchy of categories so as to bring together an arbitrary collection of categories which are
otherwise unrelated: nominative noun phrase, adjective phrase and preposition phrase?

Moreover, since the WG analysis is based on arbitrary strings and sets rather than phrases, it
easily accommodates ``incomplete'' conjuncts (\citealt[405]{Hudson90a-u}; \citealt{Hudson1982})
precisely because there is no expectation that strings are complete phrases. This claim is born out
by examples such as (\ref{ex:11}) (meaning `\dots\ and parties for foreign girls \dots').

\begin{exe}
	\ex \label{ex:11} We hold parties for foreign \emph{boys on Tuesdays} and \emph{girls on Wednesdays}.
\end{exe}
In this example, the first conjunct is the string (\emph{boys}, \emph{on}, \emph{Tuesdays}), which
is not a phrase defined by dependencies; the relevant phrases are \emph{parties for foreign boys}
and \emph{on Tuesdays}.

This sketch of a WG treatment of coordination ignores a number of important issues (raised by
reviewers) such as joint interpretation (\ref{ex:12}) and special choice of pronoun forms
(\ref{ex:13}).

\begin{exe}
	\ex \label{ex:12} John and Mary are similar.

	\ex \label{ex:13} Between you and I, she likes him.
\end{exe}

\noindent
These issues have received detailed attention in WG (\citealt[Chapter~5]{Hudson84a-u};
\citeyear{Hudson88a}; \citeyear[Chapter~14]{Hudson90a-u}; \citeyear{Hudson1995}; \citeyear[175--181,
304--307]{Hudson2010b-u}), but they are peripheral to this chapter.


%%%%%%%%%%%%%%%%%%%%%%%%%%%%%%%%
\subsection{Phrasal edges}
\label{sec:4.3}

One of the differences between PS and DS is that, at least in its classic form, PS formally
recognises phrasal boundaries, and a PS tree can even be converted to a bracketed string where the
phrase is represented by its boundaries. In contrast, although standard DS implies phrases (since a
phrase can be defined as a word and all the words depending on it either directly or indirectly), it
doesn't mark their boundaries.

This turns out to be problematic in dealing with \ili{Welsh} soft mutation
\citep{Tallerman2009}. Tallerman's article is one of the few serious discussions by a PS advocate of
the relative merits of PS and DS, so it deserves more consideration than space allows here. It
discusses examples such as (\ref{ex:14}) and (\ref{ex:15}), where the emphasised words are
morphologically changed by soft mutation in comparison with their underlying forms shown in
brackets.

\begin{exe}
\ex \label{ex:14}
\gll Prynodd                      y   ddynes \emph{\smash{delyn}}. (telyn)\\
     buy.\textsc{pst}.3\textsc{s} the woman  harp\\\hfill(\ili{Welsh)}
\glt `The woman bought a harp.'

\ex \label{ex:15}
\gll Gwnaeth                     y   ddynes [\emph{werthu} telyn]. (gwerthu)\\
     do.\textsc{pst}.3\textsc{s} the woman  \spacebr{}sell.\textsc{inf} harp\\
\glt `The woman sold a harp.'
\end{exe}

\noindent
Soft mutation is sensitive to syntax, so although `harp' is the object of a preceding verb in both
examples, it is mutated when this verb is finite (\emph{prynodd}) and followed by a subject, but not
when the verb has no subject because it is non-finite (\emph{werthu}). Similarly, the non-finite
verb `sell' is itself mutated in example (\ref{ex:15}) because it follows a subject, in contrast
with the finite verbs which precede the subject and have no mutation.

A standard PS explanation for such facts (and many more) is the ``XP Trigger Hypothesis'': that soft
mutation is triggered on a subject or complement (but not an adjunct) immediately after an XP
boundary \citep[226]{BorsleyTallermanWillis2007}. The analysis contains two claims: that mutation
affects the first word of an XP, and that it is triggered by the end of another XP. The first claim
seems beyond doubt: the mutated word is simply the first word, and not necessarily the
head. Examples such as (\ref{ex:16}) are conclusive.

\ea
\label{ex:16}
\gll Dw                          i [\emph{lawn}   mor grac  â  chi]. (llawn)\\
     be.\textsc{prs}.1\textsc{s} I \spacebr{}full as  angry as you\\\hfill(\ili{Welsh)}
\glt `I'm just as angry as you.'
\z

\noindent
The second claim is less clearly correct; for instance, it relies on controversial assumptions about
null subjects and traces in examples such as (\ref{ex:17}) and (\ref{ex:18}) (where \emph{t} and
\emph{pro} stand for a trace and a null subject respectively, but have to be treated as full phrases
for purposes of the XP Trigger Hypothesis in order to explain the mutation following them).

\begin{exe}
\ex \label{ex:17}
\gll Pwy brynodd \emph{t} delyn? (telyn)\\
     who buy.\textsc{pst}.3\textsc{s} {} harp\\\hfill(\ili{Welsh)}
\glt `Who bought a harp?'
\ex \label{ex:18}
\gll Prynodd \emph{pro} delyn. (telyn)\\
     buy.\textsc{pst}.3\textsc{s} {} harp\\
\glt `He/she bought a harp.'
\end{exe}

\noindent
But suppose both claims were true. What would this imply for DS? All it shows is that we need to be
able to identify the first word in a phrase (the mutated word) and the last word in a phrase (the
trigger). This is certainly not possible in WG as it stands, but the basic premise of WG is that the
whole of ordinary cognition is available to language, and it's very clear that ordinary cognition
allows us to recognise beginnings and endings in other domains, so why not also in language?
Moreover, beginnings and endings fit well in the framework of ideas about linearisation that are
introduced in the next subsection.

The \ili{Welsh} data, therefore, do not show that we need phrasal nodes complete with attributes and
values. Rather, edge phenomena such as \ili{Welsh} mutation show that DS needs to be expanded, but
not that we need the full apparatus of PS. Exactly how to adapt WG is a matter for future research,
not for this chapter.


%%%%%%%%%%%%%%%%%%%%%%%%%%%%%%%%
\subsection{Word order}
\label{sec:4.4}

In both WG and some variants of HPSG, dominance and linearity are separated, but this separation
goes much further in WG. In basic HPSG, linearisation rules apply only to sisters, and if the binary
branching often assumed for languages such as \ili{German} \citep[Section~10.3]{MuellerGT-Eng2}
reduces these to just two, the result is clearly too rigid given the freedom of ordering found in
many languages. It is true that solutions are available \citep[Chapter~10]{MuellerGT-Eng2}, such as
allowing alternative binary branchings for the same word combinations \crossrefchapterp[Section~\ref{sec-binary-flat}]{order} or combining binary branching
with flat structures held in lists, but these solutions involve extra complexity in other parts of
the theory such as additional lists. For instance, one innovation is the idea of linearisation
domains \citep{Reape94a,Kathol2000a,Babel}, which allow a verb and its arguments and adjuncts to be
members of the same linearisation domain and hence to be realized in any order
(\citealt[302]{MuellerGT-Eng2}; \crossrefchapteralt[Section~\ref{sec-domains}]{order}). These
proposals bring HPSG nearer to DS, where flat structures are inevitable and free order is the
default (subject to extra order constraints).

WG takes the separation of linearity from dominance a step further by introducing two new syntactic
relations dedicated to word order: ``position'' and ``landmark'', each of which points to a node in
the overall network \citep{Hudson2018a}. As its name suggests, a word's landmark is the word from
which it takes its position, and is normally the word on which it depends (as in the HPSG list of
dependents); what holds phrases together by default is that dependents keep as close to their
landmarks as possible, because a general principle bans intersecting landmark relations. Moreover,
the word's ``position'' relative to its landmark may either be free or defined as either ``before''
or ``after''.

However, this default pattern allows exceptions, and because ``position'' and ``landmark'' are
properties, they are subject to default inheritance which allows exceptions such as raising and
extraction (discussed in Section~\ref{sec:5.2}). To give an idea of the flexibility allowed by these
relations, I start with the very easy \ili{English} example in Figure~\ref{fig:12}, where ``lm''
and ``psn'' stand for ``landmark'' and ``position'', and ``<'' and ``>'' mean ``before'' and
``after''.

\begin{figure}
	\centering
\begin{tikzpicture}[node distance=2cm]
%fig 1 repeated
\node[draw](many) at (0,0){\strut{\emph{many}}};
\node[draw](students) [right of=many]{\strut \emph{students}};
\node[draw](enjoy) [right of=students]{\strut \emph{enjoy}};
\node[draw](syntax) [right of=enjoy]{\strut \emph{syntax}};
\draw[->] (enjoy)[out=north,in=north] to (syntax);
\draw[->] (enjoy)[out=north,in=north] to (students);
\draw[->] ([xshift=-.2cm]students.north)[out=north,in=north] to (many);
%lower ellipses
\node[draw,ellipse](psn1)[below of=many]{psn};
\node[draw,ellipse](psn2)[below of=students]{psn};
\node[draw,ellipse](psn3)[below of=enjoy]{psn};
\node[draw,ellipse](psn4)[below of=syntax]{psn};
\draw[dashed] (many) to (psn1);
\draw[dashed] (students) to (psn2);
\draw[dashed] (enjoy) to (psn3);
\draw[dashed] (syntax) to (psn4);
%lower rechtangles
\node[draw,inner sep=.3cm](1) [below of=psn1, node distance=1cm]{};
\node[draw,inner sep=.3cm](2) [below of=psn2, node distance=1cm]{};
\node[draw,inner sep=.3cm](3) [below of=psn3, node distance=1cm]{};
\node[draw,inner sep=.3cm](4) [below of=psn4, node distance=1cm]{};
\draw[->,dashed] (psn1) to (1);
\draw[->,dashed] (psn2) to (2);
\draw[->,dashed] (psn3) to (3);
\draw[->,dashed] (psn4) to (4);
%lowest ellipses
\node[draw,ellipse](1a)[right of=1, node distance=1cm]{<};
\node[draw,ellipse](2a)[right of=2, node distance=1cm]{<};
\node[draw,ellipse](3a)[right of=3, node distance=1cm]{>};
\draw[dashed] (1) to (1a);
\draw[->,dashed] (1a) to (2);
\draw[dashed] (2) to (2a);
\draw[->,dashed] (2a) to (3);
\draw[<-,dashed] (3) to (3a);
\draw[dashed] (3a) to (4);
%higher ellipses
\node[draw,ellipse](lm1)[above of=1a]{lm};
\node[draw,ellipse](lm2)[above of=2a]{lm};
\node[draw,ellipse](lm3)[above of=3a]{lm};
\draw[dashed] (many) to (lm1);
\draw[->,dashed] (lm1) to (students);
\draw[dashed] (students) to (lm2);
\draw[->,dashed] (lm2) to (enjoy);
\draw[<-,dashed] (enjoy) to (lm3);
\draw[dashed] (lm3) to (syntax);
\end{tikzpicture}
	\caption{Basic word order in English}
	\label{fig:12}
\end{figure}

It could be objected that this is a lot of formal machinery for such a simple matter as word order. However, it is important to recognise that the conventional left-right ordering of writing is just a written convention, and that a mental network (which is what we are trying to model in WG) has no left-right ordering. Ordering a series of objects (such as words) is a complex mental operation, which experimental subjects often get wrong, so complex machinery is appropriate.
Moreover, any syntactician knows that language offers a multiplicity of complex relations between
dependency structure and word order. To take an extreme example, non-configurational languages pose
problems for standard versions of HPSG (for which Bender suggests solutions) as illustrated by a
\ili{Wambaya} sentence, repeated here as (\ref{ex:19}) \parencites[\page
8]{Bender2008a}{Nordlinger1998}:\footnote{
  See also \crossrefchapterw[Section~\ref{sec-free-without-domains}]{order} for a discussion of
    Bender's approach and \crossrefchapterw[Section~\ref{sec-warlpiri}]{order} for an analysis
    of the phenomenon in linearization-based HPSG.
}

\begin{exe}
\ex \label{ex:19}
\gll Ngaragana-nguja ngiy-a gujinganjanga-ni jiyawu ngabulu\\
     grog\textsc{-prop}.\textsc{iv}.\textsc{acc} 3\textsc{sg}.\textsc{nm}.\textsc{a}-\textsc{pst} mother-\textsc{ii}.\textsc{erg} give milk.\textsc{iv}.\textsc{acc}\\\hfill(\ili{Wambaya})
\glt `(His) mother gave (him) milk with grog in it.'
\end{exe}

\noindent
The literal gloss shows that both `grog' and `milk' are marked as accusative, which is enough to allow the former to modify the latter in spite of their separation. The word order is typical of many Australian non-configurational languages: totally free within the clause except that the auxiliary verb (glossed here as \textsc{3sg.pst}) comes second (after one dependent word or phrase). Such freedom of order is easily accommodated if landmarks are independent of dependencies: the auxiliary verb is the root of the clause's dependency structure (as in \ili{English}), and also the landmark for every word that depends on it, whether directly or (crucially) indirectly. Its second position is due to a rule which requires it to precede all these words by default, but to have just one ``preceder''. A simplified structure for this sentence (with \ili{Wambaya} words replaced by \ili{English} glosses) is shown in Figure~\ref{fig:13}, with dotted arrows below the words again showing landmark and position relations. The dashed horizontal line separates this sentence structure from the grammar that generates it. In words, an auxiliary verb requires precisely one preceder, which \rel{isa} descendant. ``Descendant'' is a transitive generalisation of ``dependent'', so a descendant is either a dependent or a dependent of a descendant. The preceder precedes the auxiliary verb, but all other descendants follow it.

\begin{figure}
	\centering
\begin{tikzpicture}[node distance= 3cm]
\draw[dashed] (-1,1) to (10,1);
%%%%%%%%%%%
%below dashed line
%first row
\node[draw](grog) at (0,-2){\strut \emph{grog}};
\node[draw](3rd)[right of=grog]{\strut \emph{3\textsc{sg.pst}}};
\node[draw](mother)[right of=3rd, node distance=2.5cm]{\strut \emph{mother}};
\node[draw](give)[right of=mother, node distance=2cm]{\strut \emph{give}};
\node[draw](milk)[right of=give, node distance=2cm]{\strut \emph{milk}};
%squares
\node[draw,inner sep=.3cm](1)[below of=grog]{};
\node[draw,inner sep=.3cm](2)[below of=3rd]{};
\node[draw,inner sep=.3cm](3)[below of=mother]{};
\node[draw,inner sep=.3cm](4)[below of=3, node distance=1.25cm]{};
\node[draw,inner sep=.3cm](5)[below of=4, node distance=1.25cm]{};
%arrows
\draw[->,dashed] (grog) to (1);
\draw[->,dashed] (3rd) to (2);
\draw[->,dashed] (mother) to (3);
\draw[->,dashed] (give.south) to (4.north);
\draw[->,dashed] (milk.south) to (5.north);
%
\draw[->,dashed] (milk.south) to[in=300,out=225] ([xshift=.2cm]3rd.south);
\draw[->,dashed] (give.south) to[in=300, out=225] ([xshift=.4cm]3rd.south);
\draw[->,dashed] (mother.south) to[in=300, out=225] ([xshift=.6cm]3rd.south);
\draw[->,dashed] (grog.south) to[in=south, out=south] ([xshift=-.2cm]3rd.south);
%arrow with small circle
\draw[->] ([xshift=-.2cm]3rd.north) to[in=40, out=135] node[draw,circle,fill=white,inner sep=.2cm](circ){} ([xshift=.2cm]grog.north);
%
\draw[->] (3rd.north) to[in=north, out=north] (give);
\draw[->] ([xshift=-.2cm]give.north) to[in=north, out=north] (mother);
\draw[->] ([xshift=.2cm]give.north) to[in=135, out=40] (milk.north);
\draw[->] ([xshift=.2cm]milk.north) to[in=40, out=135] ([xshift=-.2cm]grog.north);
%arrows with ellipses
\draw[->,dashed] (1) to node[draw,solid,ellipse,fill=white](a){<} (2);
\draw[->,dashed] (3) to (2);
\node[draw,solid,ellipse,fill=white](b)[left of=3, node distance=.78cm]{>};
\draw[->,dashed] (4) to node[draw,near start,solid,ellipse,fill=white](c){>} (2);
\draw[->,dashed] (5) to node[draw,near start,solid,ellipse,fill=white](d){>} (2.south east);
%above dashed line
%%%%%%%%%%%
%middle row
\node[draw](one) at (0,4){\strut\ ~1 ~ };
\node[draw](aux)[right of=one]{\strut aux verb};
\node[draw,inner sep=.3cm](1a)[right of=aux]{};
%lower row
\node[draw,inner sep=.3cm](1b)[below of=1a, node distance=2cm]{};
\node[draw,inner sep=.3cm](1c) at (3.5,2){};
\node[draw,inner sep=.3cm](1d) at (1,2){};
%arrows with ellipses
%ellipses with words
\draw[->] (aux) to[in=north, out=north] node[draw,ellipse,fill=white](prec){preceder} (one);
\draw[->] (aux) to[in=north, out=north] node[draw,ellipse,fill=white](desc){descendant} (1a);
%ellipses with <
\draw[->,dashed] (1b) to node[draw,solid,ellipse,fill=white](aa){>} (1c);
\draw[->,dashed] (1d) to node[draw,solid,ellipse,fill=white](ab){<} (1c);
%triangle "arrows"
\draw[<-,>=open triangle 90 reversed] (one) to (grog);
\draw[<-,>=open triangle 90 reversed] (prec) to (circ);
\draw[<-,>=open triangle 90 reversed] (desc) to (prec);
\draw[<-,>=open triangle 90 reversed] (aux) to (3rd);
%arrows
\draw[->,dashed] (1a) to (1b);
\draw[->,dashed] ([xshift=.5cm]aux.south) to (1c);
\draw[->,dashed] (one) to (1d);
\draw[->,dashed] (one.south east) to[in=south west, out=south east] (aux.south west);
\draw[->,dashed] (1a.south west) to[in=south east, out=south west] (aux.south east);
\end{tikzpicture}
	\caption{A non-configurational structure}
	\label{fig:13}
\end{figure}

Later sections will discuss word order, and will reinforce the claims of this subsection: that
plain-vanilla versions of either PS or DS are woefully inadequate and need to be supplemented in
some way.

This completes the discussion of ``containment'' and ``continuity'', the characteristics of
classical PS which are missing in DS. We have seen that the continuity guaranteed by PS is also
provided by default in WG by a general ban on intersecting landmark relations; but, thanks to
default inheritance, exceptions abound. HPSG offers a similar degree of flexibility but using
different machinery such as word-order domains \citep{Reape94a}; see also
\crossrefchapterw{order}. An approach to Wambaya not using linearisation domains but rather
projection of valence information is discussed in Section~\ref{sec-free-without-domains} of \citew{chapters/order}. Moreover, WG offers a great deal of flexibility in other relations: for
example, a word may be part of a string (as in coordination) and its phrase's edges may need to be
recognised structurally (as in \ili{Welsh} mutation).


%%%%%%%%%%%%%%%%%%%%%%%%%%%%%%%%%%%%%%%%%%%%%%%%%%%%%
%%%%%%%%%%%%%%%%%%%%%%%%%%%%%%%%%%%%%%%%%%%%%%%%%%%%%
\section{Asymmetry and functions}
\label{sec:5}

This section considers the characteristics of DS which are missing from classical PS: asymmetrical relations between words and their dependents. Does syntactic theory need these notions? It's important to distinguish here between two different kinds of asymmetry that are recognised in HPSG. One is the kind which is inherent to PS and the part-whole relation, but the other is inherent to DS but an optional extra in PS: the functional asymmetry between the head and its dependents. HPSG, like most other theories of syntax, does recognise this asymmetry and indeed builds it into the name of the theory, but more recently this assumption has come under fire within the HPSG community for reasons considered below in Section~\ref{sec:5.1}.

But if the head/dependent distinction is important, are there any other functional distinctions between parts that ought to be explicit in the analysis? In other words, what about grammatical functions such as subject and object? As Figure~\ref{fig:8} showed, WG recognises a taxonomy of grammatical functions which carry important information about word order (among other things), so functions are central to WG analyses. Many other versions of DS also recognise functional distinctions; for example, Tesnière distinguished actants from circumstantials, and among actants he distinguished subjects, direct objects and indirect objects \citep[xlvii]{Tesniere2015a-u}. But the only functional distinction which is inherent to DS is the one between head and dependents, so other such distinctions are an optional extra in DS – just as they are in PS, where many theories accept them. But HPSG leaves them implicit in the order of elements in \argst (like phrases in DS), so this is an issue worth raising when comparing HPSG with the DS tradition.


%%%%%%%%%%%%%%%%%%%%%%%%%%%%%%%%
\subsection{Headless phrases}
\label{sec:5.1}

Bloomfield assumed that phrases could be either headed (endocentric) or not (exocentric). According to WG (and other DS theories), there are no headless phrases. Admittedly, utterances may contain unstructured lists (e.g.\ \emph{one two three four} \dots), and quotations may be unstructured strings, as in (\ref{ex:20}), but presumably no-one would be tempted to call such strings ``phrases'', or at least not in the sense of phrases that a grammar should generate.

\begin{exe}
	\ex \label{ex:20} He said ``One, two, three, testing, testing, testing.''
\end{exe}
%
Such strings can be handled by the mechanism already introduced for coordination, namely ordered sets.

The WG claim, then, is that when words hang together syntactically, they form phrases which always
have a head. Is this claim tenable? There are a number of potential counterexamples including
(\ref{ex:21})--(\ref{ex:24}):

\settowidth\jamwidth{(Arnold \& Borsley, 2014)}
\eal
\ex \label{ex:21} \emph{The rich} get richer.\footnote{\citew[403]{MuellerGT-Eng2}}

\ex \label{ex:22} \emph{The more you eat}, the fatter you get.\footnote{\citew[\page 164]{Fillmore1986}}

\ex \label{ex:23} In they came, \emph{student after student}.\footnote{\citew[8]{Jackendoff2008a}}

\ex \label{ex:24} \emph{However intelligent the students}, a lecture needs to be
clear.\footnote{Adapted from \citew[\page 28]{AB2014a-u}.}
\zl

\noindent
All these examples can in fact be given a headed analysis, as I shall now explain, starting with
(\ref{ex:21}). \emph{The rich} is allowed by \emph{the}, which has a special sub-case which allows a
single adjective as its complement, meaning either ``generic people'' or some contextually defined
notion (such as ``apples'' in \emph{the red} used when discussing apples); this is not possible with
any other determiner. In the determiner-headed analysis of standard WG, this is unproblematic as the
head is \emph{the}.

The comparative correlative in (\ref{ex:22}) is clearly a combination of a subordinate clause
followed by a main clause \citep{CJ99a-u}, but what are the heads of the two clauses? The obvious
dependency links the first \emph{the} with the second (hence ``correlative''), so it is at least
worth considering an analysis in which this dependency is the basis of the construction and, once
again, the head is \emph{the}. Figure~\ref{fig:14} outlines a possible analysis, though it should be
noted that the dependency structures are complex. The next section discusses such complexities,
which are a reaction to complex functional pressures; for example, it is easy to see that the
fronting of \emph{the less} reduces the distance between the two correlatives. Of course, there is
no suggestion here that this analysis applies unchanged to every translation equivalent of our
comparative correlative; for instance, \ili{French} uses a coordinate structure without an
equivalent of \emph{the}: \emph{Plus \dots\ et plus \dots} (\citealt{Abeille:Borsley:08}; \crossrefchapteralt[Section~\ref{coord:sec-comparative-correlatives}]{coordination}).

\begin{figure}
	\centering
\begin{forest}
%[node distance=1cm]
%\node[draw](the1) at (0,0){\strut the};
%\node[draw](harder)[right of=the1]{\strut harder};
%\node[draw](he1)[right of=harder]{\strut he};
%\node[draw](works)[right of=he1]{\strut works};
%\node[draw](the2)[right of=works]{\strut the};
%\node[draw](less)[right of=the2]{\strut less};
%\node[draw](he2)[right of=less]{\strut he};
%\node[draw](learns)[right of=he2]{\strut learns};
wg
[,phantom,for tree={font=\it} 
	[the]
	[more]
	[you]
	[eat]
	[the]
	[fatter]
	[you]
	[get]
]
\draw[->] (the)[out=north,in=north] to (more);
\draw[->] (the)[out=north,in=north] to ([xshift=.2cm]eat.north);
\draw[->] ([xshift=-.2cm]the2.north)[out=north,in=north] to ([xshift=-.2cm]the.north);
\draw[->] (eat)[out=north,in=north] to (you);
\draw[->] (eat)[out=north,in=north] to ([xshift=.2cm]more.north);
\draw[->] ([xshift=.2cm]the2.north)[out=north,in=north] to (fatter);
%\draw[->] (the2)[out=north,in=north] to ([xshift=.2cm]get.north);
\draw[->] (get)[out=north,in=north] to (the2);
\draw[->] (get)[out=north,in=north] to (you2);
\draw[->] (get)[out=north,in=north] to ([xshift=.2cm]fatter.north);
\end{forest}
	\caption{A WG sketch of the comparative correlative}
	\label{fig:14}
\end{figure}

Example\label{dg:page-npn-construction} (\ref{ex:23}) is offered by Jackendoff as a clear case of
headlessness, but there is an equally obvious headed analysis of \emph{student after student} in
which the structure is the same as in commonplace NPN examples like \emph{box of matches}. The only
peculiarity of Jackendoff's example is the lexical repetition, which is beyond most theories of
syntax. For WG, however, the solution is easy: the second N token \rel{isa} the first, which allows
default inheritance. This example illustrates an idiomatic but generalisable version of the NPN
pattern in which the second N \rel{isa} the first and the meaning is special; as expected, the pattern is
recursive. The grammatical subnetwork needed to generate the syntactic structure for such examples
is shown (with solid lines) in Figure~\ref{fig:15}; the semantics is harder and needs more
research. What this diagram shows is that there is a subclass of nouns called here
``noun\textsubscript{npn}'', which is special in having as its complement a preposition with the
special property of having another copy of the same noun\textsubscript{npn} as its complement. The
whole construction is potentially recursive because the copy itself inherits the possibility of a
preposition complement, but the recursion is limited by the fact that this complement is optional
(shown as ``0,1'' inside the box, meaning that its quantity is either 0 (absent) or 1
(present)). Because the second noun \rel{isa} the first, if it has a prepositional complement this is also
a copy of the first preposition – hence \emph{student after student after student}, whose structure
is shown in Figure~\ref{fig:15} with dashed lines.

\begin{figure}
	\centering
\begin{tikzpicture}[node distance=1.5cm]
\node[draw](noun) at (0,0){\strut noun};
%first row
\node[draw](nounnpn)[below of=noun, node distance=2.5cm]{\strut noun\textsubscript{npn}};
\node[draw,ellipse](c1)[above right of=nounnpn]{c};
\node[draw](1a)[below right of=c1]{1};
\node[draw](prep)[above of=1a, node distance=2.5cm]{\strut preposition};
\node[draw,ellipse](c2)[above right of=1a]{c};
\node[draw](1b)[below right of=c2]{1};
\node[draw,ellipse](c3)[above right of=1b]{c};
\node[draw](10)[below right of=c3]{1,0};
%second row
\node[draw,dashed](student1)[below of=nounnpn, node distance=2.6cm]{\strut \emph{student}};
\node[draw,ellipse,dashed](c4)[above right of=student1]{c};
\node[draw,dashed](after1)[below right of=c4]{\strut \emph{after}};
\node[draw,ellipse,dashed](c5)[above right of=after1]{c};
\node[draw,dashed](student2)[below right of=c5]{\strut \emph{student}};
\node[draw,ellipse,dashed](c6)[above right of=student2]{c};
\node[draw,dashed](after2)[below right of=c6]{\strut \emph{after}};
\node[draw,ellipse,dashed](c7)[above right of=after2]{c};
\node[draw,dashed](student3)[below right of=c7]{\strut \emph{student}};
%%arrows
\draw[<-,>=open triangle 90 reversed] (noun) to (nounnpn);
\draw[<-,>=open triangle 90 reversed] (prep) to (1a);
\draw[<-,>=open triangle 90 reversed,dashed] (nounnpn) to (student1);
\draw (1b.south west) to[out=south west,in=south east] ([yshift=-.13cm]nounnpn.south);
%first row
\draw (nounnpn) to (c1);
\draw[->] (c1) to (1a.north west);
\draw (1a.north east) to (c2);
\draw[->] (c2) to (1b.north west);
\draw (1b.north east) to (c3);
\draw[->] (c3) to (10.north west);
%second row
\draw[dashed] (student1) to (c4);
\draw[->,dashed] (c4) to (after1);
\draw[dashed] (after1) to (c5);
\draw[->,dashed] (c5) to (student2);
\draw[dashed] (student2) to (c6);
\draw[->,dashed] (c6) to (after2);
\draw[dashed] (after2) to (c7);
\draw[->,dashed] (c7) to (student3);
\end{tikzpicture}
	\caption{The NPN construction in Word Grammar}
	\label{fig:15}
\end{figure}

The ``exhaustive conditional'' or ``unconditional'' in (\ref{ex:24}) clearly has two parts:
\emph{however smart} and \emph{the students}, but which is the head? A verb could be added, giving
\emph{however smart the students are}, so if we assumed a covert verb, that would provide a head,
but without a verb it is unclear – and indeed this is precisely the kind of subject-predicate
structure that stood in the way of dependency analysis for nearly two thousand years.

However, there are good reasons for rejecting covert verbs in general. For instance, in \ili{Arabic}
a predicate adjective or nominal is in different cases according to whether ``be'' is overt:
accusative when it is overt, nominative when it is covert. Moreover, the word order is different in
the two constructions: the verb normally precedes the subject, but the verbless predicate follows
it. In \ili{Arabic}, therefore, a covert verb would simply complicate the analysis; but if an
analysis without a covert verb is possible for \ili{Arabic}, it is also possible in \ili{English}.

Moreover, even \ili{English} offers an easy alternative to the covert verb based on the structure
where the verb \textsc{be} is overt. It is reasonably uncontroversial to assume a raising analysis
for examples such as (\ref{ex:25}) and (\ref{ex:26}), so (\ref{ex:27}) invites a similar analysis
\citep{MuellerPredication,MuellerCopula}.

\eal
\ex \label{ex:25} He keeps talking.

\ex \label{ex:26} He is talking.

\ex \label{ex:27} He is cold.
\zl

\noindent
But a raising analysis implies a headed structure for \emph{he ... cold} in which \emph{he} depends (as subject) on \emph{cold}. Given this analysis, the same must be true even where there is no verb, as in example (\ref{ex:24})'s \emph{however smart the students} or so-called ``Mad-Magazine sentences'' like (\ref{ex:28}) \citep{Lambrecht:90}.%
%
\footnote{A reviewer asks what excludes alternatives such as *\emph{He smart?} and *\emph{Him smart.} (i.e.\ as a statement). The former is grammatically impossible because \emph{he} is possible only as the subject of a tensed verb, but presumably the latter is excluded by the pragmatic constraints on the ``Mad-magazine'' construction.}%
%

\begin{exe}
	\ex \label{ex:28} What, him smart? You're joking!
\end{exe}

\noindent
Comfortingly, the facts of exhaustive conditionals support this analysis because the subject is
optional, confirming that the predicate is the head:

\begin{exe}
	\ex \label{ex:29} However smart, nobody succeeds without a lot of effort.
\end{exe}

\noindent
In short, where there is just a subject and a predicate, without a verb, then the predicate is the
head.

Clearly it is impossible to prove the non-existence of headless phrases, but the examples considered
have been offered as plausible examples, so if even they allow a well-motivated headed analysis, it
seems reasonable to hypothesise that all phrases have heads.


%%%%%%%%%%%%%%%%%%%%%%%%%%%%%%%%
\subsection{Complex dependency}
\label{sec:5.2}\label{dg-sec-complex-dependency}

The differences between HPSG and WG raise another question concerning the geometry of sentence structure, because the possibilities offered by the part-whole relations of HPSG are more limited than those offered by the word-word dependencies of WG. How complex can dependencies be? Is there a theoretical limit such that some geometrical patterns can be ruled out as impossible? Two particular questions arise:

\begin{enumerate}
	\item \label{it:4} Can a word depend on more than one other word? This is of course precisely what structure sharing allows, but this only allows ``raising'' or ``lowering'' within a single chain of dependencies. Is any other kind of ``double dependency'' possible?
	
	\item \label{it:5} Is mutual dependency possible?
\end{enumerate}

\noindent
The answer to both questions is yes for WG, but is less clear for HPSG.

Consider the dependency structure for an example such as (\ref{ex:30}).

\begin{exe}
	\ex \label{ex:30} I wonder who came.
\end{exe}
	
\noindent
In a dependency analysis, the only available units are words, so the clause \emph{who came} has no status in the analysis and is represented by its head. In WG, this is \emph{who}, because this is the word that links \emph{came} to the rest of the sentence.

Of interest in (\ref{ex:30}) are three dependencies:

\begin{enumerate}
	\item \label{it:6} \emph{who} depends on \emph{wonder} because \emph{wonder} needs an interrogative complement – i.e.\ an interrogative word such as \emph{who} or \emph{whether}; so \emph{who} is the object of \emph{wonder}.
	
	\item \label{it:7} \emph{who} also depends on \emph{came}, because it is the subject of \emph{came}.
	
	\item \label{it:8} \emph{came} depends on \emph{who}, because interrogative pronouns allow a following finite verb (or, for most but not all pronouns, an infinitive, as in \emph{I wonder who to invite}). Since this is both selected by the pronoun and optional (as in \emph{I wonder who}), it must be the pronoun's complement, so \emph{came} is the complement of \emph{who}.
\end{enumerate}

\noindent
Given the assumptions of DS, and of WG in particular, each of these dependencies is quite obvious and uncontroversial when considered in isolation. The problem, of course, is that they combine in an unexpectedly complicated way; in fact, this one example illustrates both the complex conditions defined above: \emph{who} depends on two words which are not otherwise syntactically connected (\emph{wonder} and \emph{came}), and \emph{who} and \emph{came} are mutually dependent. A WG analysis of the relevant dependencies is sketched in Figure~\ref{fig:16} (where ``s'' and ``c'' stand for ``subject'' and ``complement'').

\begin{figure}
	\centering
\begin{tikzpicture}[node distance=1.3cm]
\node[draw](I) at (0,0){\strut ~~\emph{I}~~ };
\node[draw,ellipse](s1)[above right of=I]{s};
\node[draw](wonder)[below right of=s1]{\strut \emph{wonder}};
\node[draw,ellipse](c1)[above right of=wonder]{c};
\node[draw](who)[below right of=c1]{\strut \emph{who}};
\node[draw,ellipse](c2)[above right of=who]{c};
\node[draw](came)[below right of=c2]{\strut \emph{came}};
\node[draw,ellipse](s2)[above of=c2]{s};
\draw (wonder) to (s1);
\draw[->] (s1) to (I);
\draw (wonder) to (c1);
\draw[->] (c1) to (who);
\draw (who) to (c2);
\draw[->] (c2) to (came);
\draw (came) to (s2);
\draw[->] (s2) to (who);
\end{tikzpicture}
	\caption{Complex dependencies in a relative clause}
	\label{fig:16}
\end{figure}

A similar analysis applies to relative clauses. For instance, in (\ref{ex:31}), the relative pronoun \emph{who} depends on the antecedent \emph{man} as an adjunct and on \emph{called} as its subject, while the ``relative verb'' \emph{called} depends on \emph{who} as its obligatory complement.

\begin{exe}
	\ex \label{ex:31} I knew the man who called.
\end{exe}

Pied-piping presents well-known challenges. Take, for example, (\ref{ex:32}) \citep[212]{ps2}.

\begin{exe}
	\ex \label{ex:32} Here's the minister [[in [the middle [of [whose sermon]]]] the dog barked]
\end{exe}

\noindent
According to WG, \emph{whose} (which as a determiner is head of the phrase \emph{whose sermon}) is both an adjunct of its antecedent \emph{minister} and also the head of the relative verb \emph{barked}, just as in the simpler example. The challenge is to explain the word order: how can \emph{whose} have dependency links to both \emph{minister} and \emph{barked} when it is surrounded, on both sides, by words on which it depends? Normally, this would be impossible, but pied-piping is special. The WG analysis \citep{Hudson2018a} locates the peculiarities of pied-piping entirely in the word order, invoking a special relation ``pipee'' which transfers the expected positional properties of the relative pronoun (the ``piper'') up the dependency chain – in this case, to the preposition \emph{in}.

And so we finish this review of complex dependencies by answering the question that exercised the
minds of the Arabic grammarians in the Abbasid Ca\-liph\-ate: is mutual dependency possible?
The arrow notation of WG allows grammars to generate the relevant structures, so the answer is yes,
and HPSG can achieve the same effect by means of re-entrancy (see
  \citew[\page 50]{ps2} for the mutual selection of determiner and noun); so this conclusion reflects
another example of theoretical convergence.


%%%%%%%%%%%%%%%%%%%%%%%%%%%%%%%%
\subsection{Grammatical functions}
\label{sec:5.3}

As I have already explained, more or less traditional grammatical functions such as subject and
adjunct play a central part in WG, and more generally, they are highly compatible with any version
of DS, because they are all sub-divisions of the basic function ``dependent''. This being so, we can
define a taxonomy of functions such as the one in Figure~\ref{fig:8}, parts of which are developed in Figure~\ref{fig:17} to accommodate an example of the very specific functions which are needed in any complete grammar: the second complement of \emph{from}, as in \emph{from London to Edinburgh}, which may be unique to this particular preposition.

\begin{figure}
	\centering
\begin{tikzpicture}[node distance=1.5cm]
%first row
\node[shape=ellipse,draw](object) at (0,0){object};
\node[shape=ellipse,draw](comp2)[right of=object, node distance=4cm]{2\textsuperscript{nd} complement of \emph{from}};
%second row
\node[shape=ellipse,draw](subject)[above of=object]{subject};
\node[shape=ellipse,draw](comp1)[above of=comp2]{complement};
%third row
\node[shape=ellipse,draw](adj)[above of=subject]{adjunct};
\node[shape=ellipse,draw](valent)[above of=comp1]{valent};
%fourth row
\node[shape=ellipse,draw](dep)[above of=valent]{dependent};
%arrows
\draw[<-,>=open triangle 90 reversed] (dep)    to (valent);
\draw[<-,>=open triangle 90 reversed] (valent) to (comp1);
\draw[<-,>=open triangle 90 reversed] (comp1)  to (comp2);
%
\draw (object.north)  to ([yshift=-.13cm]comp1.south);
\draw (subject.north) to ([yshift=-.13cm]valent.south);
\draw (adj.north)     to ([yshift=-.13cm]dep.south);
\end{tikzpicture}
	\caption{A taxonomy of grammatical functions}
	\label{fig:17}
\end{figure}

HPSG also recognises a taxonomy of functions by means of three lists attached to any head word:

\begin{description}
\item[\textnormal{\textsc{spr}:}] \label{it:spr} the word's specifier, i.e.\ for a noun its determiner and (in some versions of HPSG) for a verb its subject
	
\item[\textnormal{\textsc{subj}:}] \label{it:subj} the word's subject, i.e.\ the subject of a verb
  (in some versions of HPSG) and the subject of certain other predicates (\eguk adjectives)

\item[\textnormal{\textsc{comps}:}] \label{it:comps} its complements
	
\item[\textnormal{\textsc{arg-st}:}] \label{it:arg-st} its specifier, its subject, and its complements, i.e.\ in
  WG terms, its valents.
\end{description}

\noindent
The third list concatenates the first two, so the same analysis could be achieved in WG by a
taxonomy in which \textsc{spr} and \textsc{comps} both \rel{isa} \textsc{arg-st}. However, there are
also two important differences: in HPSG, adjuncts have a different status from other dependents, and
these three general categories are lists.

Adjuncts are treated differently in the two theories. In WG, they are dependents, and located in the
same taxonomy as valents; so in HPSG terms they would be listed among the head word's attributes,
along with the other dependents but differentiated by not being licensed by the head. But HPSG
reverses this relationship by treating the head as a \textsc{mod} (``modified'') of the adjunct. For
example, in (\ref{ex:33}) \emph{she} and \emph{it} are listed in the \textsc{arg-st} of \emph{ate}
but \emph{quickly} is not mentioned in the AVM of \emph{ate}; instead, \emph{ate} is listed as
\textsc{mod} of \emph{quickly}.

\begin{exe}
	\ex \label{ex:33} She ate it quickly.
\end{exe}

\noindent
This distinction, inherited from Categorial Grammar, correctly reflects the facts of government:
\emph{ate} governs \emph{she} and \emph{it}, but not \emph{quickly}. It also reflects one possible
analysis of the semantics, in which \emph{she} and \emph{it} provide built-in arguments of the
predicate ``eat'', while \emph{quickly} provides another predicate ``quick'', of which the whole
proposition \emph{eat}(\emph{she}, \emph{it}) is the argument. Other semantic analyses are of course
possible, including one in which ``manner'' is an optional argument; but the proposed analysis is
consistent with the assumptions of HPSG.


On the other hand, HPSG also recognises a \textsc{head-daughter} in schemata like
the Specifier-Head, the Filler-Head, the Head"=Complement and the Head"=Adjunct Schema and in
the construction which includes \emph{quickly}, the latter is not the head. So what unifies
arguments and adjuncts is the fact of not being heads (being members of the \textsc{non-head-dtrs}
list in some versions of HPSG). In contrast, DS theories (including WG) agree
in recognising adjuncts as dependents, so arguments and adjuncts are unified by this category, which
is missing from most versions of HPSG, though not from all \citep*{BMS2001a}. The DS analysis follows
from the assumption that dependency isn't just about government, nor is it tied to a logical
analysis based on predicates and arguments. At least in WG, the basic characteristic of a dependent
is that it modifies the meaning of the head word, so that the resultant meaning is (typically) a
hyponym of the head's unmodified meaning. Given this characterisation, adjuncts are core dependents;
for instance \emph{big book} is a hyponym of \emph{book} (i.e.\ ``big book'' \rel{isa} ``book''), and
\emph{she ate it quickly} is a hyponym of \emph{she ate it}. The same characterisation also applies
to arguments: \emph{ate it} is a hyponym of \emph{ate}, and \emph{she ate it} is a hyponym of
\emph{ate it}. (Admittedly hyponymy is merely the default, and as explained in Section~\ref{sec:4.1}
it may be overridden by the details of particular adjuncts such as \emph{fake} as in \emph{fake
  diamonds}; but exceptions are to be expected.)


Does the absence in HPSG of a unifying category ``dependent'' matter? So long as \textsc{head} is
available, we can express word-order generalisations for head-final and head-initial languages, and
maybe also for ``head-medial'' languages such as \ili{English} \citep[172]{Hudson2010b-u}. At least
in these languages, adjuncts and arguments follow the same word-order rules, but although it is
convenient to have a single cover term ``dependent'' for them, it is probably not essential. So
maybe the presence of \textsc{head} removes the need for its complement term, \textsc{dependent}.

The other difference between HPSG and WG lies in the way in which the finer distinctions among
complements are made. In HPSG they are shown by the ordering of elements in a list, whereas WG
distinguishes them as further subcategories in a taxonomy. For example, in HPSG the direct object is
identified as the second NP in the \textsc{arg-st} list, but in WG it is a sub-category of
``complement'' in the taxonomy of Figure~\ref{fig:17}. In this case, each approach seems to offer
something which is missing from the other.

On the one hand, the ordered lists of HPSG reflect the attractive ranking of dependents offered by
Relational Grammar \citep{PP83a-u,Blake1990} in which arguments are numbered from 1 to 3 and can be
``promoted'' or ``demoted'' on this scale. The scale had subjects at the top and remote adjuncts at
the bottom, and appeared to explain a host of facts from the existence of argument-changing
alternations such as passivisation \citep{Levin93a-u} to the relative accessibility of different
dependents to relativisation \citep{KC77a}. An ordered list, as in \textsc{arg-st}, looks like a
natural way to present this ranking of dependents.

On the other hand, the taxonomy of WG functions has the attraction of open-endedness and
flexibility, which seems to be in contrast with the HPSG analysis which assumes a fixed and universal list of
dependency types defined by the order of elements in the various categories discussed previously
(\textsc{spr}, \textsc{comps} and \textsc{arg-st}). A universal list of categories seems to require an
explanation: Why a universal list? Why this particular list? How does the list develop in a
learner's mind? In contrast, a taxonomy can be learned entirely from experience, can vary across
languages, and can accommodate any amount of minor variation. Of these three attractions, the
easiest to illustrate briefly is the third. Take once again the \ili{English} preposition
\emph{from}, as in (\ref{ex:34}).
%
\begin{exe}
\ex \label{ex:34} From London to Edinburgh is four hundred miles.
\end{exe}
%
\noindent
Here \emph{from} seems to have two complements: \emph{London} and \emph{to Edinburgh}. Since they
have different properties, they must be distinguished, but how? The easiest and arguably correct
solution is to create a special dependency type just for the second complement of \emph{from}. This
is clearly unproblematic in the flexible WG approach, where any number of special dependency types
can be added at the foot of the taxonomy, but much harder if every complement must fit into a
universal list. So, HPSG seems to have a problem here, but on closer inspection this is not the
case: first, there is no claim that \argst is universal. For example, \citet{KM15a-u} discuss \ili{Oneida}
(Iroquoian) and argue that this language does not have syntactic valence and hence it would not make
sense to assume an \argstl, which entails that \argst is not universal. (See also
\citew{MuellerCoreGram} and
\crossrefchapteralt[Section~\ref{sec-empty-els-innate-knowledge}]{minimalism} on the non-assumption
of innate language-specific knowledge in HPSG.) \citet{KC77a} discussed
the obliqueness order as a universal tendency and it plays a role in various phenomena:
relativization, case assignment, agreement, pronoun binding (see the chapters on these phenomena by
\citealt{chapters/case}, \citealt{chapters/agreement}, \citealt{chapters/binding}) and an order is also
needed for capturing generalizations on linking \citep*{chapters/arg-st}. But apart from this there
is no label or specific category information attached to say the third element in the
\argstl. The general setting also allows for subjectless \argst lists as needed in grammars of
German. The respective lexemes would have an object at the first position of the \argstl.
English \emph{from} is also unproblematic: the second element in an \argstl can be
anything. A respective specification can be lexeme specific or specific for a class of lexemes (see
Chapters by \citet*{chapters/idioms} on idioms and by \citet*{chapters/arg-st} on linking).

To summarise the discussion, therefore, HPSG and WG offer fundamentally different treatments of
grammatical functions with two particularly salient differences. In the treatment of adjuncts, there
are reasons for preferring the WG approach in which adjuncts and arguments are grouped together
explicitly as dependents. But in distinguishing different types of complement, the HPSG lists seem
to complement the taxonomy of WG, each approach offering different benefits. This is clearly an area
needing further research.


%%%%%%%%%%%%%%%%%%%%%%%%%%%%%%%%%%%%%%%%%%%%%%%%%%%%%
%%%%%%%%%%%%%%%%%%%%%%%%%%%%%%%%%%%%%%%%%%%%%%%%%%%%%
\section{HPSG without PS?}
\label{sec:6}

This chapter on HPSG and DS raises a fundamental question for HPSG: does it really need PS? Most
introductory textbooks present PS as an obvious and established approach to syntax, but it is only
obvious because these books ignore the DS alternative: the relative pros and cons of the two
approaches are rarely assessed. Even if PS is in fact better than DS, this can't be described as
``established'' (in the words of one of my reviewers) until its superiority has been
demonstrated. This hasn't yet happened. The historical sketch showed very clearly that nearly two
thousand years of syntactic theory assumed DS, not PS, with one exception: the subject-predicate
analysis of the proposition (later taken to be the sentence). Even when PS was invented by
Bloomfield, it was combined with elements of DS, and Chomsky's PS, purified of all DS elements, only
survived from 1957 to 1970.

A reviewer also argues that HPSG is vindicated by the many large-scale grammars that use it (see
also \crossrefchapterw[Section~\ref{cl:resources}]{cl} for an overview). These
grammars are indeed impressive, but DS theories have also been implemented in the equally
large-scale projects listed in Section~\ref{sec:2}. In any case, the question is not whether HPSG is
a good theory, but rather whether it might be even better without its PS assumptions. The challenge
for HPSG, then, is to explain why PS is a better basis than DS. The debate has hardly started, so
its outcome is unpredictable; but suppose the debate favoured DS. Would that be the end of HPSG? Far
from it. It could survive almost intact, with just two major changes.

The first would be in the treatment of grammatical functions. It would be easy to bring all
dependents together in a list called \textsc{deps} \citep{BMS2001a} with \textsc{adjuncts} and
\textsc{comps} as sub-lists, or even with a separate subcategory for each sub-type of dependent
\citep{Hellan2017}.

The other change would be the replacement of phrasal boxes by a single list of words. (\ref{ex:35})
gives a list for the example with which we started (with round and curly brackets for ordered and
unordered sets, and a number of sub-tokens for each word):

\begin{exe}
	\ex \label{ex:35} ({\emph{many}, \emph{many}+\emph{h}} {\emph{students}, \emph{students}+\emph{a}}, {\emph{enjoy}, \emph{enjoy}+\emph{o}, \emph{enjoy}+\emph{s}}, \emph{syntax})
\end{exe}

\noindent
Each word in this list stands for a whole box of attributes which include syntactic dependency links
to other words in the list. The internal structure of the boxes would otherwise look very much like
standard HPSG, as in the schematic neo-HPSG structure in Figure~\ref{fig:18}. (To improve
readability by minimizing crossing lines, attributes and their values are separated as usual by a
colon, but may appear in either order.)

\begin{figure}
	\centering
\begin{tikzpicture}[node distance=2.5cm]
%first row
\node[draw, align=center](many1) at (0,0){\emph{many}\\\textsc{mod:} \{~~\}};
\node[draw, align=center](students1)[right of=many1]{\emph{students}\\\{~~\} :\textsc{deps}};
\node[draw, align=center](enjoy1)[right of=students1]{\emph{enjoy}\\\{~~\} :\textsc{sbj}\\\textsc{obj}: \{~~\}\\\textsc{deps}: \{~~\}};
\node[draw, align=center](syntax)[right of=enjoy1]{\emph{syntax}\\\textsc{deps}: \{~~\}};
%second row
\node[draw, align=center](many2)[above of=many1, node distance=3cm]{\emph{many$+$h}\\\textsc{mod}: \{~~\}};
\node[draw, align=center](students2)[right of=many2]{\emph{students$+$a}\\\{~~\} :\textsc{deps}};
\node[draw, align=center](enjoy2)[right of=students2]{\emph{enjoy}\\\{~~\} :\textsc{sbj}\\\textsc{obj}: \{~~\}\\\textsc{deps}: \{~~\}};
%third row
\node[draw, align=center](enjoy3)[above of=enjoy2, node distance=3cm]{\emph{enjoy}\\\{~~\} :\textsc{sbj}\\\textsc{obj}: \{~~\}\\\textsc{deps}: \{~~\}};
%arrows
\draw[<-,>=open triangle 90 reversed] (many1) to (many2);
\draw[<-,>=open triangle 90 reversed] (students1) to (students2);
\draw[<-,>=open triangle 90 reversed] (enjoy1) to (enjoy2);
\draw[<-,>=open triangle 90 reversed] (enjoy2) to (enjoy3);
%%%
\draw[->] (.4,2.8) to (students2);
\draw[->] (2,2.7) to[in=south east, out=south] (many2);
\draw[->] (4.5,6.2) to[in=north, out=west] (students2);
\draw[->] (5.5,2.7) to[in=north, out=east] (syntax);
%%%
\draw[->] (5.5,-.7) to (5.4,-.2);
\draw[->] (5.4,-.6) to[in=south, out=north] (4.7,.2);
%
\draw[->] (5.5,2.3) to (5.4,2.8);
\draw[->] (5.4,2.4) to[in=south, out=north] (4.7,3.2);
%
\draw[->] (5.5,5.3) to (5.4,5.8);
\draw[->] (5.4,5.4) to[in=south, out=north] (4.7,6.2);
%braces
\node(b1)[left of=many1, node distance=1cm]{(};
\node(b2) [right of=syntax, node distance=1cm]{)};
\end{tikzpicture}
	\caption{A neo-HPSG analysis}
	\label{fig:18}
\end{figure}

Figure~\ref{fig:18} can be read as follows:

\begin{itemize}
\item The items at the bottom of the structure (\emph{many}, \emph{students}, \emph{enjoy} and
  \emph{syntax}) are basic types stored in the grammar, available for modification by the
  dependencies. These four words are the basis for the ordered set in (\ref{ex:35}), and shown here
  by the round brackets, with the ordering shown by the left-right dimension. This list replaces the
  ordered partonomy of HPSG.

\item Higher items in the vertical taxonomy are tokens and sub-tokens, whose names show the
  dependency that defines them (\emph{h} for ``head'', \emph{a} for ``adjunct'', and so on). The
  taxonomy above \emph{enjoy} shows that \emph{enjoy}+\emph{s} \rel{isa} \emph{enjoy}+\emph{o} which isa
  \emph{enjoy}, just as in an HPSG structure where each dependent creates a new representation of
  the head by satisfying and cancelling a valency need and passing the remaining needs up to the new
  representation.

\item The taxonomy above \emph{students} shows that \emph{students}+\emph{a} is a version of
  \emph{students} that results from modification by \emph{many}, while the parallel one above
  \emph{many} shows that (following HPSG practice) \emph{many}+\emph{h} has the function of
  modifying \emph{students}.
\end{itemize}

\noindent
Roughly speaking, each boxed item in this diagram corresponds to an AVM in a standard HPSG analysis.

In short, modern HPSG could easily be transformed into a version of DS, with a separate AVM for each
word. As in DS, the words in a sentence would be represented as an ordered list interrelated partly
by the ordering and partly by the pairwise dependencies between them. This transformation is
undeniably possible. Whether it is desirable remains to be established by a programme of research
and debate which will leave the theory more robust and immune to challenge.



\section*{Abbreviations}



\begin{tabularx}{.99\textwidth}{@{}lX}
%\textsc{a}    & agent\\ is in LGR
\textsc{nm}   & non-masc. (class II--IV)\\
\textsc{ii}   & noun class II\\
\textsc{iv}   & noun class IV\\
\textsc{prop} & proprietive\\
\end{tabularx}


\section*{\acknowledgmentsEN}

I would like to take this opportunity to thank Stefan Müller for his unflagging insistence on getting everything right. 

{\sloppy
\printbibliography[heading=subbibliography,notkeyword=this] 
}
\end{document}


%      <!-- Local IspellDict: en_US-w_accents -->

%%% Local Variables:
%%% mode: latex
%%% TeX-master: t
%%% TeX-engine: xetex
%%% End:
