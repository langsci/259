%% -*- coding:utf-8 -*-

\documentclass[output=paper
%	        ,collection
%	        ,collectionchapter
 	        ,biblatex
                ,babelshorthands
                ,newtxmath
                ,draftmode
                ,colorlinks, citecolor=brown
]{langscibook}

\IfFileExists{../localcommands.tex}{%hack to check whether this is being compiled as part of a collection or standalone
  % add all extra packages you need to load to this file 

% the ISBN assigned to the digital edition
\usepackage[ISBN=9783961102556]{ean13isbn} 

\usepackage{graphicx}
\usepackage{tabularx}
\usepackage{amsmath} 

%\usepackage{tipa}      % Davis Koenig
\usepackage{xunicode} % Provide tipa macros (BC)

\usepackage{multicol}

% Berthold morphology
\usepackage{relsize}
%\usepackage{./styles/rtrees-bc} % forbidden forest 08.12.2019

% provides logo priniting commands
\usepackage{langsci-basic}

\usepackage{langsci-optional} 
% used to be in this package
\providecommand{\citegen}{}
\renewcommand{\citegen}[2][]{\citeauthor{#2}'s (\citeyear*[#1]{#2})}
\providecommand{\lsptoprule}{}
\renewcommand{\lsptoprule}{\midrule\toprule}
\providecommand{\lspbottomrule}{}
\renewcommand{\lspbottomrule}{\bottomrule\midrule}
\providecommand{\largerpage}{}
\renewcommand{\largerpage}[1][1]{\enlargethispage{#1\baselineskip}}

\usepackage{./styles/biblatex-series-number-checks}


\usepackage{langsci-lgr}

\newcommand{\MAS}{\textsc{m}\xspace} % \M is taken by somebody

%\usepackage{./styles/forest/forest}
\usepackage{langsci-forest-setup}

% is loaded in main etc.
% \usepackage{nomemoize} 
% \memoizeset{
%   memo filename prefix={chapters/hpsg-handbook.memo.dir/},
%   register=\todo{O{}+m},
%   prevent=\todo,
% }

\usepackage{tikz-cd}

\usepackage{./styles/tikz-grid}
\usetikzlibrary{shadows}


% removed with texlive 2020 06.05.2020
% %\usepackage{pgfplots} % for data/theory figure in minimalism.tex
% % fix some issue with Mod https://tex.stackexchange.com/a/330076
% \makeatletter
% \let\pgfmathModX=\pgfmathMod@
% \usepackage{pgfplots}%
% \let\pgfmathMod@=\pgfmathModX
% \makeatother

\usepackage{subcaption}

% Stefan Müller's styles
\usepackage{./styles/merkmalstruktur,./styles/makros.2020,./styles/my-xspace,./styles/article-ex,
./styles/eng-date}

\usepackage{varioref}
\newcommand\refORregion[2]{%
 \vrefpagenum\firstnum{#1}%
 \vrefpagenum\secondnum{#2}%
\ifthenelse{\equal\firstnum\secondnum}%
{\pageref{#1}}%
{\pageref{#1}--\pageref{#2}}%
}

% I am sick of fiddeling arround with babel. I want these shorthands also to work in commands I
% define. St.Mü. 13.08.2020
% e.g. with \iwithini
\usepackage{german}
\selectlanguage{USenglish}

\usepackage{./styles/abbrev}


% Has to be loaded late since otherwise footnotes will not work

%%%%%%%%%%%%%%%%%%%%%%%%%%%%%%%%%%%%%%%%%%%%%%%%%%%%
%%%                                              %%%
%%%           Examples                           %%%
%%%                                              %%%
%%%%%%%%%%%%%%%%%%%%%%%%%%%%%%%%%%%%%%%%%%%%%%%%%%%%
% remove the percentage signs in the following lines
% if your book makes use of linguistic examples
\usepackage{langsci-gb4e} 



% This introduces labels which makes hyperlinks work so that proofreading is easier.
%\makeatletter
%\newcommand{\mex}[1]{\ref{ex-\the\c@chapter-\the\numexpr\c@equation+#1}\relax}
%\newcommand{\eaautolabel}{\label{ex-\the\c@chapter-\the\numexpr\c@equation+1}}
%\makeatother

%\let\oldea\ea
%\def\ea{\oldea\eaautolabel}

%\let\oldeal\eal
%\def\eal{\oldeal\eaautolabel}


% Crossing out text
% uncomment when needed
%\usepackage{ulem}

\usepackage{./styles/additional-langsci-index-shortcuts}

% this is the completely redone avm package
\usepackage{langsci-avm}
\avmsetup{columnsep=.3ex,style=narrow}

\avmdefinecommand{phon}[phon]
  {
    attributes  = \itshape%,
%    delimfactor = 900,
%    delimfall   = 10pt
}

\avmdefinecommand{form}[form]
  {
    attributes  = \itshape%,
%    delimfactor = 900,
%    delimfall   = 10pt
}

% \set was already taken
\avmdefinecommand{avmset}[set]{ attributes=\itshape } % define a new \set command
\avmdefinecommand{list}[list]{ attributes=\itshape } % define a new \list command
   % Note: the label "list" will be output in whatever font is currently active.

% \avm{
% 	[subj  & \1 \\
% 	comps & \2 \- \list*(gap-ss) \\ % Produce a \list
% 	deps  & < \1 > \+ \2
% 	]
% }


\avmdefinecommand{nelist}[ne-list]{ attributes=\itshape } % define a new \nelist command
   % Note: the label "ne-list" will be output in whatever font is currently active.



% https://github.com/langsci/langsci-avm/issues/33#issuecomment-671201576
%\avmsetup{extraskip=0pt}

% if you have to use both langsci-avm and avm
% \usepackage{langsci-avm} % Load pkg with meaning A of conflicting cmd
% \let\lavm\avm % Send the conflicting command to an alternative
% \let\avm\undefined % Send the conflicting cmd to be \undefined
% \usepackage{avm} % Load pkg with meaning B for conf. cmd 

%\let\asort\type*

% remove this, once we really do without avm
%\usepackage{./styles/avm+}

% copied over from avm+.sty
% some relation operators:
%\newcommand{\append}[0]{\ensuremath{\oplus\hspace{.24em}}}
%\newcommand{\shuffle}[0]{\ensuremath{\bigcirc\hspace{.24em}}}

\newcommand{\append}[0]{\ensuremath{\oplus}\xspace}
\newcommand{\shuffle}[0]{\ensuremath{\bigcirc}\xspace}


% command to fontify relations in avms 
\newcommand{\rel}[1]{\texttt{#1}}
%\def\relfont{\slshape}%
%\def\relfont{\ttdefault}%


\let\idx\ibox
\let\avmbox\ibox

% command to fontify attributes in ordinary text
%\newcommand{\attrib}[1]{\textsc{#1}}


% some relation operators:
%\newcommand{\append}[0]{\ensuremath{\oplus\hspace{.24em}}}
%\newcommand{\shuffle}[0]{\ensuremath{\bigcirc\hspace{.24em}}}

\def\relfont{\slshape}%
%
% command to fontify relations in avms 
%\newcommand{\rel}[1]{{\relfont #1}}



% \renewcommand{\tpv}[1]{{\avmjvalfont\itshape #1}}

% % no small caps please
% \renewcommand{\phonshape}[0]{\normalfont\itshape}

% \regAvmFonts

\usepackage{theorem}

\newtheorem{mydefinition}{Def.}
\newtheorem{principle}{Principle}

{\theoremstyle{break}
%\newtheorem{schema}{Schema}
\newtheorem{mydefinition-break}[mydefinition]{Def.}
\newtheorem{principle-break}[principle]{Principle}
}


%% \newcommand{schema}[2]{
%% \begin{minipage}{\textwidth}
%% {\textbf{Schema~\theschema}}]\hspace{.5em}\textbf{(#1)}\\
%% #2
%% \end{minipage}}


% This avoids linebreaks in the Schema
\newcounter{schemacounter}
\makeatletter
\newenvironment{schema}[1][]
  {%
   \refstepcounter{schemacounter}%
   \par\bigskip\noindent
   \minipage{\linewidth}%
   \textbf{Schema~\theschemacounter\hspace{.5em} \ifx&#1&\else(#1)\fi}\par
  }{\endminipage\par\bigskip\@endparenv}%
\makeatother

%\usepackage{subfig}





% Davis Koenig Lexikon

\usepackage{tikz-qtree,tikz-qtree-compat} % Davis Koenig remove

\usepackage{shadow}



\usepackage[english]{isodate} % Andy Lücking
\usepackage[autostyle]{csquotes} % Andy
%\usepackage[autolanguage]{numprint}

%\defaultfontfeatures{
%    Path = /usr/local/texlive/2017/texmf-dist/fonts/opentype/public/fontawesome/ }

%% https://tex.stackexchange.com/a/316948/18561
%\defaultfontfeatures{Extension = .otf}% adds .otf to end of path when font loaded without ext parameter e.g. \newfontfamily{\FA}{FontAwesome} > \newfontfamily{\FA}{FontAwesome.otf}
%\usepackage{fontawesome} % Andy Lücking
\usepackage{pifont} % Andy Lücking -> hand

\usetikzlibrary{decorations.pathreplacing} % Andy Lücking
\usetikzlibrary{matrix} % Andy 
\usetikzlibrary{positioning} % Andy
\usepackage{tikz-3dplot} % Andy

% pragmatics
\usepackage{eqparbox} % Andy
\usepackage{enumitem} % Andy
\usepackage{longtable} % Andy
\usepackage{tabu} % Andy              needs to be loaded before hyperref as of texlive 2020

% tabu-fix
% to make "spread 0pt" work
% -----------------------------
\RequirePackage{etoolbox}
\makeatletter
\patchcmd
	\tabu@startpboxmeasure
	{\bgroup\begin{varwidth}}%
	{\bgroup
	 \iftabu@spread\color@begingroup\fi\begin{varwidth}}%
	{}{}
\def\@tabarray{\m@th\def\tabu@currentgrouptype
    {\currentgrouptype}\@ifnextchar[\@array{\@array[c]}}
%
%%% \pdfelapsedtime bug 2019-12-15
\patchcmd
	\tabu@message@etime
	{\the\pdfelapsedtime}%
	{\pdfelapsedtime}%
	{}{}
%
%
\makeatother
% -----------------------------


% Manfred's packages

%\usepackage{shadow}

\usepackage{tabularx}
\newcolumntype{L}[1]{>{\raggedright\arraybackslash}p{#1}} % linksbündig mit Breitenangabe


% Jong-Bok

%\usepackage{xytree}

\newcommand{\xytree}[2][dummy]{Let's do the tree!}

% seems evil, get rid of it
% defines \ex is incompatible with gb4e
%\usepackage{lingmacros}

% taken from lingmacros:
\makeatletter
% \evnup is used to line up the enumsentence number and an entry along
% the top.  It can take an argument to improve lining up.
\def\evnup{\@ifnextchar[{\@evnup}{\@evnup[0pt]}}

\def\@evnup[#1]#2{\setbox1=\hbox{#2}%
\dimen1=\ht1 \advance\dimen1 by -.5\baselineskip%
\advance\dimen1 by -#1%
\leavevmode\lower\dimen1\box1}
\makeatother


% YK -- CG chapter

%\usepackage{xspace}
\usepackage{bm}
\usepackage{ebproof}


% Antonio Branco, remove this
\usepackage{epsfig}

% now unicode
%\usepackage{alphabeta}





\usepackage{pst-node}


% fmr: additional packages
%\usepackage{amsthm}


% Ash and Steve: LFG
\usepackage{./styles/lfg/dalrymple}

\RequirePackage{graphics}
%\RequirePackage{./styles/lfg/trees}
%% \RequirePackage{avm}
%% \avmoptions{active}
%% \avmfont{\sc}
%% \avmvalfont{\sc}
\RequirePackage{./styles/lfg/lfgmacrosash}

\usepackage{./styles/lfg/glue}

%%%%%%%%%%%%%%%%%%%%%%%%%%%%%%
%% Markup
%%%%%%%%%%%%%%%%%%%%%%%%%%%%%%
\usepackage[normalem]{ulem} % For thinks like strikethrough, using \sout

% \newcommand{\high}[1]{\textbf{#1}} % highlighted text
\newcommand{\high}[1]{\textit{#1}} % highlighted text
%\newcommand{\term}[1]{\textit{#1}\/} % technical term
\newcommand{\qterm}[1]{``{#1}''} % technical term, quotes
%\newcommand{\trns}[1]{\strut `#1'} % translation in glossed example
\newcommand{\trnss}[1]{\strut \phantom{\sqz{}} `#1'} % translation in ungrammatical glossed example
\newcommand{\ttrns}[1]{(`#1')} % an in-text translation of a word
\newcommand{\LFGfeat}[1]{\mbox{\textsc{\MakeLowercase{#1}}}}     % feature name
%\newcommand{\val}[1]{\mbox{\textsc{\MakeLowercase{#1}}}}    % f-structure value
\newcommand{\featt}[1]{\mbox{\textsc{\MakeLowercase{#1}}}}     % feature name
\newcommand{\vall}[1]{\mbox{\textsc{\textup{\MakeLowercase{#1}}}}}    % f-structure value
\newcommand{\mg}[1]{\mbox{\textsc{\MakeLowercase{#1}}}}    % morphological gloss
%\newcommand{\word}[1]{\textit{#1}}       % mention of word
\providecommand{\kstar}[1]{{#1}\ensuremath{^*}}
\providecommand{\kplus}[1]{{#1}\ensuremath{^+}}
\newcommand{\template}[1]{@\textsc{\MakeLowercase{#1}}}
\newcommand{\templaten}[1]{\textsc{\MakeLowercase{#1}}}
\newcommand{\templatenn}[1]{\MakeUppercase{#1}}
\newcommand{\tempeq}{\ensuremath{=}}
\newcommand{\predval}[1]{\ensuremath{\langle}\textsc{#1}\ensuremath{\rangle}}
\newcommand{\predvall}[1]{{\rm `#1'}}
\newcommand{\lfgfst}[1]{\ensuremath{#1\,}}
\newcommand{\scare}[1]{``#1''} % scare quotes
\newcommand{\bracket}[1]{\ensuremath{\left\langle\mathit{#1}\right\rangle}}
\newcommand{\sectionw}[1][]{Section#1} % section word: for cap/non-cap
\newcommand{\tablew}[1][]{Table#1} % table word: for cap/non-cap
\newcommand{\lfgglue}{LFG+Glue}
\newcommand{\hpsgglue}{HPSG+Glue}
\newcommand{\gs}{GS}
%\newcommand{\func}[1]{\ensuremath{\mathbf{#1}}}
\newcommand{\func}[1]{\textbf{#1}}
\renewcommand{\glue}{Glue}
%\newcommand{\exr}[1]{(\ref{ex:#1}}
\newcommand{\exra}[1]{(\ref{ex:#1})}


%%%%%%%%%%%%%%%%%%%%%%%%%%%%%%
% Notation
%\newcommand{\xbar}[1]{$_{\mbox{\textsc{#1}$^{\raisebox{1ex}{}}$}}$}
\newcommand{\xprime}[2][]{\textup{\mbox{{#2}\ensuremath{^\prime_{\hspace*{-.0em}\mbox{\footnotesize\ensuremath{\mathit{#1}}}}}}}}
\providecommand{\xzero}[2][]{#2\ensuremath{^0_{\mbox{\footnotesize\ensuremath{\mathit{#1}}}}}}



\let\leftangle\langle
\let\rightangle\rangle

%\newcommand{\pslabel}[1]{}

% remove when finished
\usepackage{proofread}
  %add all your local new commands to this file

% The orchid-id is specified and then extracted by scripts for zenodo.
\newcommand{\orcid}[1]{} 

% do not show the chapter number. It is redundant, since most references to figures are within the
% same chapter.
\renewcommand{\thefigure}{\arabic{figure}}


% Don't do this at home. I do not like the smaller font for captions.
% I just removed loading the caption packege in langscibook.cls
%% \captionsetup{%
%% font={%
%% stretch=1%.8%
%% ,normalsize%,small%
%% },%
%% width=.8\textwidth
%% }

\makeatletter
\def\blx@maxline{77}
\makeatother


\let\citew\citet

\newcommand{\page}{}

\newcommand{\todostefan}[1]{\todo[color=orange!80]{\footnotesize #1}\xspace}
\newcommand{\todosatz}[1]{\todo[color=red!40]{\footnotesize #1}\xspace}

\newcommand{\inlinetodostefan}[1]{\todo[color=green!40,inline]{\footnotesize #1}\xspace}

\newcommand{\inlinetodoopt}[1]{\todo[color=green!40,inline]{\footnotesize #1}\xspace}
\newcommand{\inlinetodoobl}[1]{\todo[color=red!40,inline]{\footnotesize #1}\xspace}

\newcommand{\itd}[1]{\inlinetodoobl{#1}}
\newcommand{\itdobl}[1]{\inlinetodoobl{#1}}
\newcommand{\itdopt}[1]{\inlinetodoopt{#1}}

\newcommand{\itdsecond}[1]{}

\newcommand{\itddone}[1]{}
%\let\itddone\itdopt
\newcommand{\LATER}[1]{}



% A. Red: Simple typos, errors in the AVMs (only a couple) to take care of on the editorial side, no need to contact the authors
% B.: Green: Wording changes which do not necessarily require authors’ approval, but are not just typos/errors
% C.: Blue: Comments to the author that they don’t have to take care of, but after all, the authors might be interested to have the comments for future revisions. 
% D.: Purple: Comments to the editors about something we need to keep in mind or do. Nothing for you

\newcommand{\colorcodingexplanation}{\todo[color=green!40,inline]{%
Explanation of colors of bubbles and text:\\
A.: Red: Things that have to be fixed/commented upon.\\
B.: Green: optional comments\\
C.: Blue: Comments to the author that they don’t have to take care of, but after all, the authors
might be interested to have the comments for future revisions.\\
Explanation of colors of text:\\
Red: newly added material (crossreferences to other chapters and other references)\\
Orange: changed material, please check\\
Blue: suggestions for deletion\\
Please also check margin notes.
}}
% D.: Purple: Comments to the editors about something we need to keep in mind or do. Nothing for you


\newcommand{\itdgreen}[1]{\todo[color=green!40,inline]{\footnotesize #1}\xspace}
\newcommand{\itdblue}[1]{\todo[color=blue!40,inline]{\footnotesize #1}\xspace}

% for editing, remove later
\usepackage{xcolor}
\newcommand{\added}[1]{{\red #1}}
\newcommand{\addedthis}{\todostefan{added this}}

\newcommand{\changed}[1]{\textcolor{orange}{#1}}
\newcommand{\deleted}[1]{\textcolor{blue}{#1}}


% \newcommand{\addpages}{\todostefan{add pages}}
% %\newcommand{\iaddpages}{\inlinetodoobl{add pages}}
% \newcommand{\iaddpages}{\yel[add pages]{pages}\xspace}
% \newcommand{\addref}{\todostefan{add reference}}
% \newcommand{\inlineaddpages}{\inlinetodostefan{add pages}}
% \newcommand{\addglosses}{\todostefan{add glosses}}

\newcommand{\addpages}{\xspace}%np
\newcommand{\iaddpages}{\xspace}%islands und understudied languages
\newcommand{\addref}{\xspace}
\newcommand{\inlineaddpages}{\xspace}
% not used \newcommand{\addglosses}{}


%\newcommand{\spacebr}{\hphantom{[}}

\newcommand{\danishep}{\jambox{(\ili{Danish})}}
\newcommand{\english}{\jambox{(\ili{English})}}
\newcommand{\german}{\jambox{(\ili{German})}}
\newcommand{\yiddish}{\jambox{(\ili{Yiddish})}}
\newcommand{\welsh}{\jambox{(\ili{Welsh})}}

% Cite and cross-reference other chapters
\newcommand{\crossrefchaptert}[2][]{\citet*[#1]{chapters/#2}, Chapter~\ref{chap-#2} of this volume} 
\newcommand{\crossrefchapterp}[2][]{(\citealp*[#1]{chapters/#2}, Chapter~\ref{chap-#2} of this volume)}
\newcommand{\crossrefchapteralt}[2][]{\citealt*[#1]{chapters/#2}, Chapter~\ref{chap-#2} of this volume}
\newcommand{\crossrefchapteralp}[2][]{\citealp*[#1]{chapters/#2}, Chapter~\ref{chap-#2} of this volume}

\newcommand{\crossrefcitet}[2][]{\citet*[#1]{chapters/#2}} 
\newcommand{\crossrefcitep}[2][]{\citep*[#1]{chapters/#2}}
\newcommand{\crossrefcitealt}[2][]{\citealt*[#1]{chapters/#2}}
\newcommand{\crossrefcitealp}[2][]{\citealp*[#1]{chapters/#2}}


% example of optional argument:
% \crossrefchapterp[for something, see:]{name}
% gives: (for something, see: Author 2018, Chapter~X of this volume)



\let\crossrefchapterw\crossrefchaptert



% Davis Koenig

\let\ig=\textsc
\let\tc=\textcolor

% evolution, Flickinger, Pollard, Wasow

\let\citeNP\citet

% Adam P

%\newcommand{\toappear}{Forthcoming}
\newcommand{\pg}[1]{p.\,#1}
\renewcommand{\implies}{\rightarrow}

\newcommand*{\rref}[1]{(\ref{#1})}
\newcommand*{\aref}[1]{(\ref{#1}a)}
\newcommand*{\bref}[1]{(\ref{#1}b)}
\newcommand*{\cref}[1]{(\ref{#1}c)}

\newcommand{\msadam}{.}
\newcommand{\morsyn}[1]{\textsc{#1}}

\newcommand{\aux}{\textsc{aux}\xspace}

\newcommand{\nom}{\morsyn{nom}}
\newcommand{\acc}{\morsyn{acc}}
\newcommand{\dat}{\morsyn{dat}}
\newcommand{\gen}{\morsyn{gen}}
\newcommand{\ins}{\morsyn{ins}}
%\newcommand{\aploc}{\morsyn{loc}}
\newcommand{\voc}{\morsyn{voc}}
\newcommand{\ill}{\morsyn{ill}}
\renewcommand{\inf}{\morsyn{inf}}
\newcommand{\passprc}{\morsyn{passp}}

%\newcommand{\Nom}{\msadam\nom}
%\newcommand{\Acc}{\msadam\acc}
%\newcommand{\Dat}{\msadam\dat}
%\newcommand{\Gen}{\msadam\gen}
\newcommand{\Ins}{\msadam\ins}
\newcommand{\Loc}{\msadam\loc}
\newcommand{\Voc}{\msadam\voc}
\newcommand{\Ill}{\msadam\ill}
\newcommand{\PassP}{\msadam\passprc}

\newcommand{\Aux}{\textsc{aux}}

%\newcommand{\princ}[1]{\textnormal{\textsc{#1}}} % for constraint names
\newcommand{\princ}[1]{\textnormal{#1}} % for constraint names
\newcommand{\notion}[1]{\emph{#1}}
\renewcommand{\path}[1]{\textnormal{\textsc{#1}}}
\newcommand{\ftype}[1]{\textit{#1}}
\newcommand{\fftype}[1]{{\scriptsize\textit{#1}}}
\newcommand{\la}{$\langle$}
\newcommand{\ra}{$\rangle$}
%\newcommand{\argst}{\path{arg-st}}
\newcommand{\phtm}[1]{\setbox0=\hbox{#1}\hspace{\wd0}}
\newcommand{\prep}[1]{\setbox0=\hbox{#1}\hspace{-1\wd0}#1}


% Rui

\newcommand{\spc}[0]{\hspace{-1pt}\underline{\hspace{6pt}}\,}
\newcommand{\spcs}[0]{\hspace{-1pt}\underline{\hspace{6pt}}\,\,}
\newcommand{\bad}[1]{\leavevmode\llap{#1}}
\newcommand{\COMMENT}[1]{}


% Rui coordination
\newcommand{\subl}[1]{$_{\scriptstyle \textsc{#1}}$}



% Andy Lücking gesture.tex
\newcommand{\Pointing}{\ding{43}}
% Giotto: "Meeting of Joachim and Anne at the Golden Gate" - 1305-10 
\definecolor{GoldenGate1}{rgb}{.13,.09,.13} % Dress of woman in black
\definecolor{GoldenGate2}{rgb}{.94,.94,.91} % Bridge
\definecolor{GoldenGate3}{rgb}{.06,.09,.22} % Blue sky
\definecolor{GoldenGate4}{rgb}{.94,.91,.87} % Dress of woman with shawl
\definecolor{GoldenGate5}{rgb}{.52,.26,.26} % Joachim's robe
\definecolor{GoldenGate6}{rgb}{.65,.35,.16} % Anne's robe
\definecolor{GoldenGate7}{rgb}{.91,.84,.42} % Joachim's halo

\makeatletter
\newcommand{\@Depth}{1} % x-dimension, to front
\newcommand{\@Height}{1} % z-dimension, up
\newcommand{\@Width}{1} % y-dimension, rightwards
%\GGS{<x-start>}{<y-start>}{<z-top>}{<z-bottom>}{<Farbe>}{<x-width>}{<y-depth>}{<opacity>}
\newcommand{\GGS}[9][]{%
\coordinate (O) at (#2-1,#3-1,#5);
\coordinate (A) at (#2-1,#3-1+#7,#5);
\coordinate (B) at (#2-1,#3-1+#7,#4);
\coordinate (C) at (#2-1,#3-1,#4);
\coordinate (D) at (#2-1+#8,#3-1,#5);
\coordinate (E) at (#2-1+#8,#3-1+#7,#5);
\coordinate (F) at (#2-1+#8,#3-1+#7,#4);
\coordinate (G) at (#2-1+#8,#3-1,#4);
\draw[draw=black, fill=#6, fill opacity=#9] (D) -- (E) -- (F) -- (G) -- cycle;% Front
\draw[draw=black, fill=#6, fill opacity=#9] (C) -- (B) -- (F) -- (G) -- cycle;% Top
\draw[draw=black, fill=#6, fill opacity=#9] (A) -- (B) -- (F) -- (E) -- cycle;% Right
}
\makeatother


% pragmatics
\newcommand{\speaking}[1]{\eqparbox{name}{\textsc{\lowercase{#1}\space}}}
\newcommand{\alname}[1]{\eqparbox{name}{\textsc{\lowercase{#1}}}}
\newcommand{\HPSGTTR}{HPSG$_{\text{TTR}}$\xspace}

\newcommand{\ttrtype}[1]{\textit{#1}}
\newcommand{\avmel}{\q<\quad\q>} %% shortcut for empty lists in AVM
\newcommand{\ttrmerge}{\ensuremath{\wedge_{\textit{merge}}}}
\newcommand{\Cat}[2][0.1pt]{%
  \begin{scope}[y=#1,x=#1,yscale=-1, inner sep=0pt, outer sep=0pt]
   \path[fill=#2,line join=miter,line cap=butt,even odd rule,line width=0.8pt]
  (151.3490,307.2045) -- (264.3490,307.2045) .. controls (264.3490,291.1410) and (263.2021,287.9545) .. (236.5990,287.9545) .. controls (240.8490,275.2045) and (258.1242,244.3581) .. (267.7240,244.3581) .. controls (276.2171,244.3581) and (286.3490,244.8259) .. (286.3490,264.2045) .. controls (286.3490,286.2045) and (323.3717,321.6755) .. (332.3490,307.2045) .. controls (345.7277,285.6390) and (309.3490,292.2151) .. (309.3490,240.2046) .. controls (309.3490,169.0514) and (350.8742,179.1807) .. (350.8742,139.2046) .. controls (350.8742,119.2045) and (345.3490,116.5037) .. (345.3490,102.2045) .. controls (345.3490,83.3070) and (361.9972,84.4036) .. (358.7581,68.7349) .. controls (356.5206,57.9117) and (354.7696,49.2320) .. (353.4652,36.1439) .. controls (352.5396,26.8573) and (352.2445,16.9594) .. (342.5985,17.3574) .. controls (331.2650,17.8250) and (326.9655,37.7742) .. (309.3490,39.2045) .. controls (291.7685,40.6320) and (276.7783,24.2380) .. (269.9740,26.5795) .. controls (263.2271,28.9013) and (265.3490,47.2045) .. (269.3490,60.2045) .. controls (275.6359,80.6368) and (289.3490,107.2045) .. (264.3490,111.2045) .. controls (239.3490,115.2045) and (196.3490,119.2045) .. (165.3490,160.2046) .. controls (134.3490,201.2046) and (135.4934,249.3212) .. (123.3490,264.2045) .. controls (82.5907,314.1553) and (40.8239,293.6463) .. (40.8239,335.2045) .. controls (40.8239,353.8102) and (72.3490,367.2045) .. (77.3490,361.2045) .. controls (82.3490,355.2045) and (34.8638,337.3259) .. (87.9955,316.2045) .. controls (133.3871,298.1601) and   (137.4391,294.4766) .. (151.3490,307.2045) -- cycle;
\end{scope}%
}
%% leicht modifiziert nach Def. von Sebastian Nordhoff:
% \newcommand{\lueckingbox}[3]{\parbox[t][][t]{0.7cm}{\raggedright
%     \strut#1}\parbox[t][][t]{7.7cm}{\strut#2}\parbox[t][][t]{3cm}{\raggedright\strut#3}\bigskip\\}
\newcommand{\lueckingbox}[3]{\parbox[t][][t]{0.7cm}{\raggedright
    \strut\vspace*{-\baselineskip}\newline#1}\parbox[t][][t]{7.7cm}{\strut\vspace*{-\baselineskip}\newline#2}\parbox[t][][t]{3cm}{\raggedright\strut\vspace*{-\baselineskip}\newline#3}\bigskip\\}




% KdK
\newcommand{\smiley}{:)}

\renewbibmacro*{index:name}[5]{%
  \usebibmacro{index:entry}{#1}
    {\iffieldundef{usera}{}{\thefield{usera}\actualoperator}\mkbibindexname{#2}{#3}{#4}{#5}}}

% \newcommand{\noop}[1]{}

% chngcntr.sty otherwise gives error that these are already defined
%\let\counterwithin\relax
%\let\counterwithout\relax

% the space of a left bracket for glossings
\newcommand{\LB}{\hphantom{[}}

\newcommand{\LF}{\mbox{$[\![$}}

\newcommand{\RF}{\mbox{$]\!]_F$}}

\newcommand{\RT}{\mbox{$]\!]_T$}}





% Manfred's

\newcommand{\kommentar}[1]{}

\newcommand{\bsp}[1]{\emph{#1}}
\newcommand{\bspT}[2]{\bsp{#1} `#2'}
\newcommand{\bspTL}[3]{\bsp{#1} (lit.: #2) `#3'}

\newcommand{\noidi}{§}

\newcommand{\refer}[1]{(\ref{#1})}

%\newcommand{\avmtype}[1]{\multicolumn{2}{l}{\type{#1}}}
\newcommand{\attr}[1]{\textsc{#1}}

%\newcommand{\srdefault}{\mbox{\begin{tabular}{@{}c@{}}{\large <}\\[-1.5ex]$\sqcap$\end{tabular}}}
\newcommand{\srdefault}{$\stackrel{<}{\sqcap}$}


%% \newcommand{\myappcolumn}[2]{
%% \begin{minipage}[t]{#1}#2\end{minipage}
%% }

%% \newcommand{\appc}[1]{\myappcolumn{3.7cm}{#1}}


% Jong-Bok


% clean that up and do not use \def (killing other stuff defined before)
%\if 0
%\newcommand\DEL{\textsc{del}}
%\newcommand\del{\textsc{del}}

\newcommand\conn{\textsc{conn}}
\newcommand\CONN{\textsc{conn}}
\newcommand\CONJ{\textsc{conj}}
\newcommand\LITE{\textsc{lex}}
\newcommand\lite{\textsc{lex}}
\newcommand\HON{\textsc{hon}}

%\newcommand\CAUS{\textsc{caus}}
%\newcommand\PASS{\textsc{pass}}
\newcommand\NPST{\textsc{npst}}
%\newcommand\COND{\textsc{cond}}



\newcommand\hdlite{\textsc{head-lex construction}}
\newcommand\hdlight{\textsc{head-light} Schema}
\newcommand\NFORM{\textsc{nform}}

\newcommand\RELS{\textsc{rels}}
%\newcommand\TENSE{\textsc{tense}}


%\newcommand\ARG{\textsc{arg}}
\newcommand\ARGs{\textsc{arg0}}
\newcommand\ARGa{\textsc{arg}}
\newcommand\ARGb{\textsc{arg2}}
\newcommand\TPC{\textsc{top}}
%\newcommand\PROG{\textsc{prog}}

\newcommand\LIGHT{\textsc{light}\xspace}
\newcommand\pst{\textsc{pst}}
%\newcommand\PAST{\textsc{pst}}
%\newcommand\DAT{\textsc{dat}}
%\newcommand\CONJ{\textsc{conj}}
\newcommand\nominal{\textsc{nominal}}
\newcommand\NOMINAL{\textsc{nominal}}
\newcommand\VAL{\textsc{val}}
%\newcommand\val{\textsc{val}}
\newcommand\MODE{\textsc{mode}}
\newcommand\RESTR{\textsc{restr}}
\newcommand\SIT{\textsc{sit}}
\newcommand\ARG{\textsc{arg}}
\newcommand\RELN{\textsc{rel}}
%\newcommand\REL{\textsc{rel}}
%\newcommand\RELS{\textsc{rels}}
%\newcommand\arg-st{\textsc{arg-st}}
\newcommand\xdel{\textsc{xdel}}
\newcommand\zdel{\textsc{zdel}}
\newcommand\sug{\textsc{sug}}
%\newcommand\IMP{\textsc{imp}}
%\newcommand\conn{\textsc{conn}}
%\newcommand\CONJ{\textsc{conj}}
%\newcommand\HON{\textsc{hon}}
\newcommand\BN{\textsc{bn}}
\newcommand\bn{\textsc{bn}}
\newcommand\pres{\textsc{pres}}
\newcommand\PRES{\textsc{pres}}
\newcommand\prs{\textsc{pres}}
%\newcommand\PRS{\textsc{pres}}
\newcommand\agt{\textsc{agt}}
%\newcommand\DEL{\textsc{del}}
%\newcommand\PRED{\textsc{pred}}
\newcommand\AGENT{\textsc{agent}}
\newcommand\THEME{\textsc{theme}}
%\newcommand\AUX{\textsc{aux}}
%\newcommand\THEME{\textsc{theme}}
%\newcommand\PL{\textsc{pl}}
\newcommand\SRC{\textsc{src}}
\newcommand\src{\textsc{src}}
\newcommand{\FORMjb}{\textsc{form}}
\newcommand{\formjb}{\FORM}
\newcommand\GCASE{\textsc{gcase}}
\newcommand\gcase{\textsc{gcase}}
\newcommand\SCASE{\textsc{scase}}
\newcommand\PHON{\textsc{phon}}
%\newcommand\SS{\textsc{ss}}
\newcommand\SYN{\textsc{syn}}
%\newcommand\LOC{\textsc{loc}}
\newcommand\MOD{\textsc{mod}}
\newcommand\INV{\textsc{inv}}
%\newcommand\L{\textsc{l}}
%\newcommand\CASE{\textsc{case}}
\newcommand\SPR{\textsc{spr}}
\newcommand\COMPS{\textsc{comps}}
%\newcommand\comps{\textsc{comps}}
\newcommand\SEM{\textsc{sem}}
\newcommand\CONT{\textsc{cont}}
\newcommand\SUBCAT{\textsc{subcat}}
\newcommand\CAT{\textsc{cat}}
%\newcommand\C{\textsc{c}}
%\newcommand\SUBJ{\textsc{subj}}
\newcommand\subjjb{\textsc{subj}}
%\newcommand\SLASH{\textsc{slash}}
\newcommand\LOCAL{\textsc{local}}
%\newcommand\ARG-ST{\textsc{arg-st}}
%\newcommand\AGR{\textsc{agr}}
\newcommand\PER{\textsc{per}}
%\newcommand\NUM{\textsc{num}}
%\newcommand\IND{\textsc{ind}}
\newcommand\VFORM{\textsc{vform}}
\newcommand\PFORM{\textsc{pform}}
\newcommand\decl{\textsc{decl}}
%\newcommand\loc{\textsc{loc   }}
% \newcommand\   {\textsc{  }}

%\newcommand\NEG{\textsc{neg}}
\newcommand\FRAMES{\textsc{frames}}
%\newcommand\REFL{\textsc{refl}}

\newcommand\MKG{\textsc{mkg}}

%\newcommand\BN{\textsc{bn}}
\newcommand\HD{\textsc{hd}}
\newcommand\NP{\textsc{np}}
\newcommand\PF{\textsc{pf}}
%\newcommand\PL{\textsc{pl}}
\newcommand\PP{\textsc{pp}}
%\newcommand\SS{\textsc{ss}}
\newcommand\VF{\textsc{vf}}
\newcommand\VP{\textsc{vp}}
%\newcommand\bn{\textsc{bn}}
\newcommand\cl{\textsc{cl}}
%\newcommand\pl{\textsc{pl}}
\newcommand\Wh{\ital{Wh}}
%\newcommand\ng{\textsc{neg}}
\newcommand\wh{\ital{wh}}
%\newcommand\ACC{\textsc{acc}}
%\newcommand\AGR{\textsc{agr}}
\newcommand\AGT{\textsc{agt}}
\newcommand\ARC{\textsc{arc}}
%\newcommand\ARG{\textsc{arg}}
\newcommand\ARP{\textsc{arc}}
%\newcommand\AUX{\textsc{aux}}
%\newcommand\CAT{\textsc{cat}}
%\newcommand\COP{\textsc{cop}}
%\newcommand\DAT{\textsc{dat}}
\newcommand\NEWCOMMAND{\textsc{def}}
%\newcommand\DEL{\textsc{del}}
\newcommand\DOM{\textsc{dom}}
\newcommand\DTR{\textsc{dtr}}
%\newcommand\FUT{\textsc{fut}}
\newcommand\GAP{\textsc{gap}}
%\newcommand\GEN{\textsc{gen}}
%\newcommand\HON{\textsc{hon}}
%\newcommand\IMP{\textsc{imp}}
%\newcommand\IND{\textsc{ind}}
%\newcommand\INV{\textsc{inv}}
\newcommand\LEX{\textsc{lex}}
\newcommand\Lex{\textsc{lex}}
%\newcommand\LOC{\textsc{loc}}
%\newcommand\MOD{\textsc{mod}}
\newcommand\MRK{{\nr MRK}}
%\newcommand\NEG{\textsc{neg}}
\newcommand\NEW{\textsc{new}}
%\newcommand\NOM{\textsc{nom}}
%\newcommand\NUM{\textsc{num}}
%\newcommand\PER{\textsc{per}}
%\newcommand\PST{\textsc{pst}}
\newcommand\QUE{\textsc{que}}
%\newcommand\REL{\textsc{rel}}
\newcommand\SEL{\textsc{sel}}
%\newcommand\SEM{\textsc{sem}}
%\newcommand\SIT{\textsc{arg0}}
%\newcommand\SPR{\textsc{spr}}
%\newcommand\SRC{\textsc{src}}
\newcommand\SUG{\textsc{sug}}
%\newcommand\SYN{\textsc{syn}}
%\newcommand\TPC{\textsc{top}}
%\newcommand\VAL{\textsc{val}}
%\newcommand\acc{\textsc{acc}}
%\newcommand\agt{\textsc{agt}}
\newcommand\cop{\textsc{cop}}
%\newcommand\dat{\textsc{dat}}
\newcommand\foc{\textsc{focus}}
%\newcommand\FOC{\textsc{focus}}
\newcommand\fut{\textsc{fut}}
\newcommand\hon{\textsc{hon}}
\newcommand\imp{\textsc{imp}}
\newcommand\kes{\textsc{kes}}
%\newcommand\lex{\textsc{lex}}
%\newcommand\loc{\textsc{loc}}
\newcommand\mrk{{\nr MRK}}
%\newcommand\nom{\textsc{nom}}
%\newcommand\num{\textsc{num}}
\newcommand\plu{\textsc{plu}}
\newcommand\pne{\textsc{pne}}
%\newcommand\pst{\textsc{pst}}
\newcommand\pur{\textsc{pur}}
%\newcommand\que{\textsc{que}}
%\newcommand\src{\textsc{src}}
%\newcommand\sug{\textsc{sug}}
\newcommand\tpc{\textsc{top}}
%\newcommand\utt{\textsc{utt}}
%\newcommand\val{\textsc{val}}
%% \newcommand\LITE{\textsc{lex}}
%% \newcommand\PAST{\textsc{pst}}
%% \newcommand\POSP{\textsc{pos}}
%% \newcommand\PRS{\textsc{pres}}
%% \newcommand\mod{\textsc{mod}}%
%% \newcommand\newuse{{`kes'}}
%% \newcommand\posp{\textsc{pos}}
%% \newcommand\prs{\textsc{pres}}
%% \newcommand\psp{{\it en\/}}
%% \newcommand\skes{\textsc{kes}}
%% \newcommand\CASE{\textsc{case}}
%% \newcommand\CASE{\textsc{case}}
%% \newcommand\COMP{\textsc{comp}}
%% \newcommand\CONJ{\textsc{conj}}
%% \newcommand\CONN{\textsc{conn}}
%% \newcommand\CONT{\textsc{cont}}
%% \newcommand\DECL{\textsc{decl}}
%% \newcommand\FOCUS{\textsc{focus}}
%% %\newcommand\FORM{\textsc{form}} duplicate
%% \newcommand\FREL{\textsc{frel}}
%% \newcommand\GOAL{\textsc{goal}}
\newcommand\HEAD{\textsc{head}}
%% \newcommand\INDEX{\textsc{ind}}
%% \newcommand\INST{\textsc{inst}}
%% \newcommand\MODE{\textsc{mode}}
%% \newcommand\MOOD{\textsc{mood}}
%% \newcommand\NMLZ{\textsc{nmlz}}
%% \newcommand\PHON{\textsc{phon}}
%% \newcommand\PRED{\textsc{pred}}
%% %\newcommand\PRES{\textsc{pres}}
%% \newcommand\PROM{\textsc{prom}}
%% \newcommand\RELN{\textsc{pred}}
%% \newcommand\RELS{\textsc{rels}}
%% \newcommand\STEM{\textsc{stem}}
%% \newcommand\SUBJ{\textsc{subj}}
%% \newcommand\XARG{\textsc{xarg}}
%% \newcommand\bse{{\it bse\/}}
%% \newcommand\case{\textsc{case}}
%% \newcommand\caus{\textsc{caus}}
%% \newcommand\comp{\textsc{comp}}
%% \newcommand\conj{\textsc{conj}}
%% \newcommand\conn{\textsc{conn}}
%% \newcommand\decl{\textsc{decl}}
%% \newcommand\fin{{\it fin\/}}
%% %\newcommand\form{\textsc{form}}
%% \newcommand\gend{\textsc{gend}}
%% \newcommand\inf{{\it inf\/}}
%% \newcommand\mood{\textsc{mood}}
%% \newcommand\nmlz{\textsc{nmlz}}
%% \newcommand\pass{\textsc{pass}}
%% \newcommand\past{\textsc{past}}
%% \newcommand\perf{\textsc{perf}}
%% \newcommand\pln{{\it pln\/}}
%% \newcommand\pred{\textsc{pred}}


%% %\newcommand\pres{\textsc{pres}}
%% \newcommand\proc{\textsc{proc}}
%% \newcommand\nonfin{{\it nonfin\/}}
%% \newcommand\AGENT{\textsc{agent}}
%% \newcommand\CFORM{\textsc{cform}}
%% %\newcommand\COMPS{\textsc{comps}}
%% \newcommand\COORD{\textsc{coord}}
%% \newcommand\COUNT{\textsc{count}}
%% \newcommand\EXTRA{\textsc{extra}}
%% \newcommand\GCASE{\textsc{gcase}}
%% \newcommand\GIVEN{\textsc{given}}
%% \newcommand\LOCAL{\textsc{local}}
%% \newcommand\NFORM{\textsc{nform}}
%% \newcommand\PFORM{\textsc{pform}}
%% \newcommand\SCASE{\textsc{scase}}
%% \newcommand\SLASH{\textsc{slash}}
%% \newcommand\SLASH{\textsc{slash}}
%% \newcommand\THEME{\textsc{theme}}
%% \newcommand\TOPIC{\textsc{topic}}
%% \newcommand\VFORM{\textsc{vform}}
%% \newcommand\cause{\textsc{cause}}
%% %\newcommand\comps{\textsc{comps}}
%% \newcommand\gcase{\textsc{gcase}}
%% \newcommand\itkes{{\it kes\/}}
%% \newcommand\pass{{\it pass\/}}
%% \newcommand\vform{\textsc{vform}}
%% \newcommand\CCONT{\textsc{c-cont}}
%% \newcommand\GN{\textsc{given-new}}
%% \newcommand\INFO{\textsc{info-st}}
%% \newcommand\ARG-ST{\textsc{arg-st}}
%% \newcommand\SUBCAT{\textsc{subcat}}
%% \newcommand\SYNSEM{\textsc{synsem}}
%% \newcommand\VERBAL{\textsc{verbal}}
%% \newcommand\arg-st{\textsc{arg-st}}
%% \newcommand\plain{{\it plain}\/}
%% \newcommand\propos{\textsc{propos}}
%% \newcommand\ADVERBIAL{\textsc{advl}}
%% \newcommand\HIGHLIGHT{\textsc{prom}}
%% \newcommand\NOMINAL{\textsc{nominal}}

\newenvironment{myavm}{\begingroup\avmvskip{.1ex}
  \selectfont\begin{avm}}%
{\end{avm}\endgroup\medskip}
\newcommand\pfix{\vspace{-5pt}}


\newcommand{\jbsub}[1]{\lower4pt\hbox{\small #1}}
\newcommand{\jbssub}[1]{\lower4pt\hbox{\small #1}}
\newcommand\jbtr{\underbar{\ \ \ }\ }


%\fi

% cl

\newcommand{\delphin}{\textsc{delph-in}}


% YK -- CG chapter

\newcommand{\grey}[1]{\colorbox{mycolor}{#1}}
\definecolor{mycolor}{gray}{0.8}

\newcommand{\GQU}[2]{\raisebox{1.6ex}{\ensuremath{\rotatebox{180}{\textbf{#1}}_{\scalebox{.7}{\textbf{#2}}}}}}

\newcommand{\SetInfLen}{\setpremisesend{0pt}\setpremisesspace{10pt}\setnamespace{0pt}}

\newcommand{\pt}[1]{\ensuremath{\mathsf{#1}}}
\newcommand{\ptv}[1]{\ensuremath{\textsf{\textsl{#1}}}}

\newcommand{\sv}[1]{\ensuremath{\bm{\mathcal{#1}}}}
\newcommand{\sX}{\sv{X}}
\newcommand{\sF}{\sv{F}}
\newcommand{\sG}{\sv{G}}

\newcommand{\syncat}[1]{\textrm{#1}}
\newcommand{\syncatVar}[1]{\ensuremath{\mathit{#1}}}

\newcommand{\RuleName}[1]{\textrm{#1}}

\newcommand{\SemTyp}{\textsf{Sem}}

\newcommand{\E}{\ensuremath{\bm{\epsilon}}\xspace}

\newcommand{\greeka}{\upalpha}
\newcommand{\greekb}{\upbeta}
\newcommand{\greekd}{\updelta}
\newcommand{\greekp}{\upvarphi}
\newcommand{\greekr}{\uprho}
\newcommand{\greeks}{\upsigma}
\newcommand{\greekt}{\uptau}
\newcommand{\greeko}{\upomega}
\newcommand{\greekz}{\upzeta}

\newcommand{\Lemma}{\ensuremath{\hskip.5em\vdots\hskip.5em}\noLine}
\newcommand{\LemmaAlt}{\ensuremath{\hskip.5em\vdots\hskip.5em}}

\newcommand{\I}{\iota}

\newcommand{\sem}{\ensuremath}

\newcommand{\NoSem}{%
\renewcommand{\LexEnt}[3]{##1; \syncat{##3}}
\renewcommand{\LexEntTwoLine}[3]{\renewcommand{\arraystretch}{.8}%
\begin{array}[b]{l} ##1;  \\ \syncat{##3} \end{array}}
\renewcommand{\LexEntThreeLine}[3]{\renewcommand{\arraystretch}{.8}%
\begin{array}[b]{l} ##1; \\ \syncat{##3} \end{array}}}

\newcommand{\hypml}[2]{\left[\!\!#1\!\!\right]^{#2}}

%%%%for bussproof
\def\defaultHypSeparation{\hskip0.1in}
\def\ScoreOverhang{0pt}

\newcommand{\MultiLine}[1]{\renewcommand{\arraystretch}{.8}%
\ensuremath{\begin{array}[b]{l} #1 \end{array}}}

\newcommand{\MultiLineMod}[1]{%
\ensuremath{\begin{array}[t]{l} #1 \end{array}}}

\newcommand{\hypothesis}[2]{[ #1 ]^{#2}}

\newcommand{\LexEnt}[3]{#1; \ensuremath{#2}; \syncat{#3}}

\newcommand{\LexEntTwoLine}[3]{\renewcommand{\arraystretch}{.8}%
\begin{array}[b]{l} #1; \\ \ensuremath{#2};  \syncat{#3} \end{array}}

\newcommand{\LexEntThreeLine}[3]{\renewcommand{\arraystretch}{.8}%
\begin{array}[b]{l} #1; \\ \ensuremath{#2}; \\ \syncat{#3} \end{array}}

\newcommand{\LexEntFiveLine}[5]{\renewcommand{\arraystretch}{.8}%
\begin{array}{l} #1 \\ #2; \\ \ensuremath{#3} \\ \ensuremath{#4}; \\ \syncat{#5} \end{array}}

\newcommand{\LexEntFourLine}[4]{\renewcommand{\arraystretch}{.8}%
\begin{array}{l} \pt{#1} \\ \pt{#2}; \\ \syncat{#4} \end{array}}

\newcommand{\ManySomething}{\renewcommand{\arraystretch}{.8}%
\raisebox{-3mm}{\begin{array}[b]{c} \vdots \,\,\,\,\,\, \vdots \\
\vdots \end{array}}}

\newcommand{\lemma}[1]{\renewcommand{\arraystretch}{.8}%
\begin{array}[b]{c} \vdots \\ #1 \end{array}}

\newcommand{\lemmarev}[1]{\renewcommand{\arraystretch}{.8}%
\begin{array}[b]{c} #1 \\ \vdots \end{array}}

\newcommand{\p}{\ensuremath{\upvarphi}}

% clashes with soul package
\newcommand{\yusukest}{\textbf{\textsf{st}}}

\newcommand{\shortarrow}{\xspace\hskip-1.2ex\scalebox{.5}[1]{\ensuremath{\bm{\rightarrow}}}\hskip-.5ex\xspace}

\newcommand{\SemInt}[1]{\mbox{$[\![ \textrm{#1} ]\!]$}}

\newcommand{\HypSpace}{\hskip-.8ex}
\newcommand{\RaiseHeight}{\raisebox{2.2ex}}
\newcommand{\RaiseHeightLess}{\raisebox{1ex}}

\newcommand{\ThreeColHyp}[1]{\RaiseHeight{\Bigg[}\HypSpace#1\HypSpace\RaiseHeight{\Bigg]}}
\newcommand{\TwoColHyp}[1]{\RaiseHeightLess{\Big[}\HypSpace#1\HypSpace\RaiseHeightLess{\Big]}}

\newcommand{\LemmaShort}{\ensuremath{ \ \vdots} \ \noLine}
\newcommand{\LemmaShortAlt}{\ensuremath{ \ \vdots} \ }

\newcommand{\fail}{**}
\newcommand{\vs}{\raisebox{.05em}{\ensuremath{\upharpoonright}}}
\newcommand{\DerivSize}{\small}

% This is not needed, we just take unicode symbols
% The result of the code below came out wrong anyway.
% St. Mü. 10.06.2021
%
% \def\maru#1{{\ooalign{\hfil
%   \ifnum#1>999 \resizebox{.25\width}{\height}{#1}\else%
%   \ifnum#1>99 \resizebox{.33\width}{\height}{#1}\else%
%   \ifnum#1>9 \resizebox{.5\width}{\height}{#1}\else #1%
%   \fi\fi\fi%
% \/\hfil\crcr%
% \raise.167ex\hbox{\mathhexbox20D}}}}

\newenvironment{samepage2}%
 {\begin{flushleft}\begin{minipage}{\linewidth}}
 {\end{minipage}\end{flushleft}}

\newcommand{\cmt}[1]{\textsl{\textbf{[#1]}}}
\newcommand{\trns}[1]{\textbf{#1}\xspace}
\newcommand{\ptfont}{}
\newcommand{\gp}{\underline{\phantom{oo}}}
\newcommand{\mgcmt}{\marginnote}

\newcommand{\term}[1]{\emph{\isi{#1}}}

\newcommand{\citeposs}[1]{\citeauthor{#1}'s \citeyearpar{#1}}

% for standalone compilations Felix: This is in the class already
%\let\thetitle\@title
%\let\theauthor\@author 
\makeatletter
\newcommand{\togglepaper}[1][0]{ 
\bibliography{../Bibliographies/stmue,../localbibliography,
collection.bib}
  %% hyphenation points for line breaks
%% Normally, automatic hyphenation in LaTeX is very good
%% If a word is mis-hyphenated, add it to this file
%%
%% add information to TeX file before \begin{document} with:
%% %% hyphenation points for line breaks
%% Normally, automatic hyphenation in LaTeX is very good
%% If a word is mis-hyphenated, add it to this file
%%
%% add information to TeX file before \begin{document} with:
%% \include{localhyphenation}
\hyphenation{
A-la-hver-dzhie-va
ac-cu-sa-tive
anaph-o-ra
ana-phor
ana-phors
an-te-ced-ent
an-te-ced-ents
affri-ca-te
affri-ca-tes
ap-proach-es
Atha-bas-kan
Athe-nä-um
Be-schrei-bung
Bona-mi
Chi-che-ŵa
com-ple-ments
con-straints
Cope-sta-ke
Da-ge-stan
Dor-drecht
er-klä-ren-de
Flick-inger
Ginz-burg
Gro-ning-en
Has-pel-math
Jap-a-nese
Jon-a-than
Ka-tho-lie-ke
Ko-bon
krie-gen
Kroe-ger
Le-Sourd
moth-er
Mül-ler
Nie-mey-er
Ørs-nes
Par-a-digm
Prze-piór-kow-ski
phe-nom-e-non
re-nowned
Rie-he-mann
un-bound-ed
Ver-gleich
with-in
}

% listing within here does not have any effect for lfg.tex % 2020-05-14

% why has "erklärende" be listed here? I specified langid in bibtex item. Something is still not working with hyphenation.


% to do: check
%  Alahverdzhieva


% biblatex:

% This is a LaTeX frontend to TeX’s \hyphenation command which defines hy- phenation exceptions. The ⟨language⟩ must be a language name known to the babel/polyglossia packages. The ⟨text ⟩ is a whitespace-separated list of words. Hyphenation points are marked with a dash:

% \DefineHyphenationExceptions{american}{%
% hy-phen-ation ex-cep-tion }

\hyphenation{
A-la-hver-dzhie-va
ac-cu-sa-tive
anaph-o-ra
ana-phor
ana-phors
an-te-ced-ent
an-te-ced-ents
affri-ca-te
affri-ca-tes
ap-proach-es
Atha-bas-kan
Athe-nä-um
Be-schrei-bung
Bona-mi
Chi-che-ŵa
com-ple-ments
con-straints
Cope-sta-ke
Da-ge-stan
Dor-drecht
er-klä-ren-de
Flick-inger
Ginz-burg
Gro-ning-en
Has-pel-math
Jap-a-nese
Jon-a-than
Ka-tho-lie-ke
Ko-bon
krie-gen
Kroe-ger
Le-Sourd
moth-er
Mül-ler
Nie-mey-er
Ørs-nes
Par-a-digm
Prze-piór-kow-ski
phe-nom-e-non
re-nowned
Rie-he-mann
un-bound-ed
Ver-gleich
with-in
}

% listing within here does not have any effect for lfg.tex % 2020-05-14

% why has "erklärende" be listed here? I specified langid in bibtex item. Something is still not working with hyphenation.


% to do: check
%  Alahverdzhieva


% biblatex:

% This is a LaTeX frontend to TeX’s \hyphenation command which defines hy- phenation exceptions. The ⟨language⟩ must be a language name known to the babel/polyglossia packages. The ⟨text ⟩ is a whitespace-separated list of words. Hyphenation points are marked with a dash:

% \DefineHyphenationExceptions{american}{%
% hy-phen-ation ex-cep-tion }

  \memoizeset{
    memo filename prefix={hpsg-handbook.memo.dir/},
    % readonly
  }
  \papernote{\scriptsize\normalfont
    \@author.
    \titleTemp. 
    To appear in: 
    Stefan Müller, Anne Abeillé, Robert D. Borsley \& Jean-Pierre Koenig (eds.)
    HPSG Handbook
    Berlin: Language Science Press. [preliminary page numbering]
  }
  \pagenumbering{roman}
  \setcounter{chapter}{#1}
  \addtocounter{chapter}{-1}
}
\makeatother

\makeatletter
\newcommand{\togglepaperminimal}[1][0]{ 
  \bibliography{../Bibliographies/stmue,
                ../localbibliography,
collection.bib}
  %% hyphenation points for line breaks
%% Normally, automatic hyphenation in LaTeX is very good
%% If a word is mis-hyphenated, add it to this file
%%
%% add information to TeX file before \begin{document} with:
%% %% hyphenation points for line breaks
%% Normally, automatic hyphenation in LaTeX is very good
%% If a word is mis-hyphenated, add it to this file
%%
%% add information to TeX file before \begin{document} with:
%% \include{localhyphenation}
\hyphenation{
A-la-hver-dzhie-va
ac-cu-sa-tive
anaph-o-ra
ana-phor
ana-phors
an-te-ced-ent
an-te-ced-ents
affri-ca-te
affri-ca-tes
ap-proach-es
Atha-bas-kan
Athe-nä-um
Be-schrei-bung
Bona-mi
Chi-che-ŵa
com-ple-ments
con-straints
Cope-sta-ke
Da-ge-stan
Dor-drecht
er-klä-ren-de
Flick-inger
Ginz-burg
Gro-ning-en
Has-pel-math
Jap-a-nese
Jon-a-than
Ka-tho-lie-ke
Ko-bon
krie-gen
Kroe-ger
Le-Sourd
moth-er
Mül-ler
Nie-mey-er
Ørs-nes
Par-a-digm
Prze-piór-kow-ski
phe-nom-e-non
re-nowned
Rie-he-mann
un-bound-ed
Ver-gleich
with-in
}

% listing within here does not have any effect for lfg.tex % 2020-05-14

% why has "erklärende" be listed here? I specified langid in bibtex item. Something is still not working with hyphenation.


% to do: check
%  Alahverdzhieva


% biblatex:

% This is a LaTeX frontend to TeX’s \hyphenation command which defines hy- phenation exceptions. The ⟨language⟩ must be a language name known to the babel/polyglossia packages. The ⟨text ⟩ is a whitespace-separated list of words. Hyphenation points are marked with a dash:

% \DefineHyphenationExceptions{american}{%
% hy-phen-ation ex-cep-tion }

\hyphenation{
A-la-hver-dzhie-va
ac-cu-sa-tive
anaph-o-ra
ana-phor
ana-phors
an-te-ced-ent
an-te-ced-ents
affri-ca-te
affri-ca-tes
ap-proach-es
Atha-bas-kan
Athe-nä-um
Be-schrei-bung
Bona-mi
Chi-che-ŵa
com-ple-ments
con-straints
Cope-sta-ke
Da-ge-stan
Dor-drecht
er-klä-ren-de
Flick-inger
Ginz-burg
Gro-ning-en
Has-pel-math
Jap-a-nese
Jon-a-than
Ka-tho-lie-ke
Ko-bon
krie-gen
Kroe-ger
Le-Sourd
moth-er
Mül-ler
Nie-mey-er
Ørs-nes
Par-a-digm
Prze-piór-kow-ski
phe-nom-e-non
re-nowned
Rie-he-mann
un-bound-ed
Ver-gleich
with-in
}

% listing within here does not have any effect for lfg.tex % 2020-05-14

% why has "erklärende" be listed here? I specified langid in bibtex item. Something is still not working with hyphenation.


% to do: check
%  Alahverdzhieva


% biblatex:

% This is a LaTeX frontend to TeX’s \hyphenation command which defines hy- phenation exceptions. The ⟨language⟩ must be a language name known to the babel/polyglossia packages. The ⟨text ⟩ is a whitespace-separated list of words. Hyphenation points are marked with a dash:

% \DefineHyphenationExceptions{american}{%
% hy-phen-ation ex-cep-tion }

  \memoizeset{
    memo filename prefix={hpsg-handbook.memo.dir/},
    % readonly
  }
  \papernote{\scriptsize\normalfont
    \@author.
    \@title. 
    To appear in: 
    Stefan Müller, Anne Abeillé, Robert D. Borsley \& Jean-Pierre Koenig (eds.)
    HPSG Handbook
    Berlin: Language Science Press. [preliminary page numbering]
  }
  \pagenumbering{roman}
  \setcounter{chapter}{#1}
  \addtocounter{chapter}{-1}
}
\makeatother




% In case that year is not given, but pubstate. This mainly occurs for titles that are forthcoming, in press, etc.
\renewbibmacro*{addendum+pubstate}{% Thanks to https://tex.stackexchange.com/a/154367 for the idea
  \printfield{addendum}%
  \iffieldequalstr{labeldatesource}{pubstate}{}
  {\newunit\newblock\printfield{pubstate}}
}

\DeclareLabeldate{%
    \field{date}
    \field{year}
    \field{eventdate}
    \field{origdate}
    \field{urldate}
    \field{pubstate}
    \literal{nodate}
}

%\defbibheading{diachrony-sources}{\section*{Sources}} 

% if no langid is set, it is English:
% https://tex.stackexchange.com/a/279302
\DeclareSourcemap{
  \maps[datatype=bibtex]{
    \map{
      \step[fieldset=langid, fieldvalue={english}]
    }
  }
}


% for bibliographies
% biber/biblatex could use sortname field rather than messing around this way.
\newcommand{\SortNoop}[1]{}


% Doug Ball

\newcommand{\elist}{\q<\ \ \q>}

\newcommand{\esetDB}{\q\{\ \ \q\}}


\makeatletter

\newcommand{\nolistbreak}{%

  \let\oldpar\par\def\par{\oldpar\nobreak}% Any \par issues a \nobreak

  \@nobreaktrue% Don't break with first \item

}

\makeatother


% intermediate before Frank's trees are fixed
% This will be removed!!!!!
%\newcommand{\tree}[1]{} % ignore them blody trees
%\usepackage{tree-dvips}


\newcommand{\nodeconnect}[2]{}
\newcommand{\nodetriangle}[2]{}



% Doug relative clauses
%% I've compiled out almost all my private LaTeX command, but there are some
%% I found hard to get rid of. They are defined here.
%% There are few others which defined in places in the document where they have only
%% local effect (e.g. within figures); their names all end in DA, e.g. \MotherDA
%% There are a lot of \labels -- they are all of the form \label{sec:rc-...} or
%% \label{x:rc-...} or similar, so there should be no clashes.

% Subscripts -- scriptsize italic shape lowered by .25ex 
\newcommand{\subscr}[1]{\raisebox{-.5ex}{\protect{\scriptsize{\itshape #1\/}}}}
% A boxed subscript, for avm tags in normal text
\newcommand{\subtag}[1]{\subscr{\idx{#1}}}

%% Sets and tuples: I use \setof{} to get brackets that are upright, not slanted
%\newcommand{\setof}[1]{\ensuremath{\lbrace\,\mathit{#1}\,\rbrace}}
% 11.10.2019 EP: Doug requested replacement of existing \setof definition with the following:
%\newcommand{\setof}[1]{\begin{avm}\{\textcolor{red}{#1}\}\end{avm}}
% 31.1.2019 EP: Doug requested re-replacement of the above \textcolour version with the following:
\newcommand{\setof}[1]{\begin{avm}\{#1\}\end{avm}}

\newcommand{\tuple}[1]{\ensuremath{\left\langle\,\mbox{\textit{#1}}\,\right\rangle}}

% Single pile of stuff, optional arugment is psn (e.g. t or b)
% e.g. to put a over b over c in a centered column, top aligned, do:
%   \cPile[t]{a\\b\\c} 
\newcommand{\cPile}[2][]{%
  \begingroup%
  \renewcommand{\arraystretch}{.5}\begin{tabular}[#1]{@{}c@{}}#2\end{tabular}%
  \endgroup%
}

%% for linguistic examples in running text (`linguistic citation'):
\newcommand{\lic}[1]{\textit{#1}}

%% A gap marked by an underline, raised slightly
%% Default argument indicates how long the line should be:
\newcommand{\uGap}[1][3ex]{\raisebox{.25em}{\underline{\hspace{#1}}}\xspace}

%% \TnodeDA{XP}{avmcontents} -- in a Tree, put a node label next to an AVM
\newcommand{\TnodeDA}[2]{#1~\begin{avm}{#2}\end{avm}}

%% This allows tipa stuff to be put in \emph -- we need to change to cmr first.
%% It is used in the discussion of Arabic.
\newcommand{\emphtipa}[1]{{\fontfamily{cmr}\emph{\tipaencoding #1}}} 



 
 
\definecolor{lsDOIGray}{cmyk}{0,0,0,0.45}


% morphology.tex:
% Berthold

\newcommand{\dnode}[1]{\rnode{#1}{\fbox{#1}}}
\newcommand{\tnode}[1]{\rnode{#1}{\textit{#1}}}

\newcommand{\tl}[2]{#2}

\newcommand{\rrr}[3]{%
  \psframebox[linestyle=none]{%
    \avmoptions{center}
    \begin{avm}
      \[mud & \{ #1 \}\\
      ms & \{ #2 \}\\
      mph & \<  #3 \> \]
    \end{avm}
  }
}
\newcommand{\rr}[2]{%
  \psframebox[linestyle=none]{%
    \avmoptions{center}
    \begin{avm}
      \[mud & \{ #1 \}\\
      mph & \<  #2 \> \]
    \end{avm}
  }
}
 

% Frank Richter
\newtheorem{mydef}{Definition}

\long\def\set[#1\set=#2\set]%
{%
\left\{%
\tabcolsep 1pt%
\begin{tabular}{l}%
#1%
\end{tabular}%
\left|%
\tabcolsep 1pt%
\begin{tabular}{l}%
#2%
\end{tabular}%
\right.%
\right\}%
}

\newcommand{\einruck}{\\ \hspace*{1em}}


%\newcommand{\NatNum}{\mathrm{I\hspace{-.17em}N}}
\newcommand{\NatNum}{\mathbb{N}}
\newcommand{\Aug}[1]{\widehat{#1}}
%\newcommand{\its}{\mathrm{:}}
% Felix 14.02.2020
\DeclareMathOperator{\its}{:}

\newcommand{\sequence}[1]{\langle#1\rangle}

\newcommand{\INTERPRETATION}[2]{\sequence{#1\mathsf{U}#2,#1\mathsf{S}#2,#1\mathsf{A}#2,#1\mathsf{R}#2}}
\newcommand{\Interpretation}{\INTERPRETATION{}{}}

\newcommand{\Inte}{\mathsf{I}}
\newcommand{\Unive}{\mathsf{U}}
\newcommand{\Speci}{\mathsf{S}}
\newcommand{\Atti}{\mathsf{A}}
\newcommand{\Reli}{\mathsf{R}}
\newcommand{\ReliT}{\mathsf{RT}}

\newcommand{\VarInt}{\mathsf{G}}
\newcommand{\CInt}{\mathsf{C}}
\newcommand{\Tinte}{\mathsf{T}}
\newcommand{\Dinte}{\mathsf{D}}

% this was missing from ash's stuff.

%% \def \optrulenode#1{
%%   \setbox1\hbox{$\left(\hbox{\begin{tabular}{@{\strut}c@{\strut}}#1\end{tabular}}\right)$}
%%   \raisebox{1.9ex}{\raisebox{-\ht1}{\copy1}}}



\newcommand{\pslabel}[1]{}

\newcommand{\addpagesunless}{\todostefan{add pages unless you cite the
 work as such}}

% dg.tex
% framed boxes as used in dg.tex
% original idea from stackexchange, but modified by Saso
% http://tex.stackexchange.com/questions/230300/doing-something-like-psframebox-in-tikz#230306
\tikzset{
  frbox/.style={
    rounded corners,
    draw,
    thick,
    inner sep=5pt,
    anchor=base,
  },
}

% get rid of these morewrite messages:
% https://tex.stackexchange.com/questions/419489/suppressing-messages-to-standard-output-from-package-morewrites/419494#419494
\ExplSyntaxOn
\cs_set_protected:Npn \__morewrites_shipout_ii:
  {
    \__morewrites_before_shipout:
    \__morewrites_tex_shipout:w \tex_box:D \g__morewrites_shipout_box
    \edef\tmp{\interactionmode\the\interactionmode\space}\batchmode\__morewrites_after_shipout:\tmp
  }
\ExplSyntaxOff


% This is for places where authors used bold. I replace them by \emph
% but have the information where the bold was. St. Mü. 09.05.2020
\newcommand{\textbfemph}[1]{\emph{#1}}



% Felix 09.06.2020: copy code from the third line into localcommands.tex:
% https://github.com/langsci/langscibook#defined-environments-commands-etc
% Does not work with texlive 2020, is done with sed in Makefile
%\patchcmd{\mkbibindexname}{\ifdefvoid{#3}{}{\MakeCapital{#3} }}{\ifdefvoid{#3}{}{#3 }}{}{\AtEndDocument{\typeout{mkbibindexname could not be patched.}}}



\let\textnobf\textit
% instead of "in bold" write "in italics"
\newcommand{\bolddescriptionintext}{italics\xspace}

% Berthold
\newcommand{\mathplus}{+}
% \mbox{\normalfont +}}
\newcommand{\emdash}{--\xspace}
\newcommand{\emdashUS}{--\xspace}


% Stefan to get the space remvoed infront of the : in Bargmann NPN discussion
%\DeclareMathSymbol{:}{\mathord}{operators}{"3A}
% used {:\,} instead


% for cxg.tex needed for includonly to find the counter.
\newcounter{croftyears} 




% Needed for bibtex entry for Jackendoff's xbar syntax. Without it the bar would be off in itialics.

% https://tex.stackexchange.com/questions/95014/aligning-overline-to-italics-font/95079#95079
% \newbox\usefulbox

% \makeatletter
%     \def\getslant #1{\strip@pt\fontdimen1 #1}

%     \def\skoverline #1{\mathchoice
%      {{\setbox\usefulbox=\hbox{$\m@th\displaystyle #1$}%
%         \dimen@ \getslant\the\textfont\symletters \ht\usefulbox
%         \divide\dimen@ \tw@ 
%         \kern\dimen@ 
%         \overline{\kern-\dimen@ \box\usefulbox\kern\dimen@ }\kern-\dimen@ }}
%      {{\setbox\usefulbox=\hbox{$\m@th\textstyle #1$}%
%         \dimen@ \getslant\the\textfont\symletters \ht\usefulbox
%         \divide\dimen@ \tw@ 
%         \kern\dimen@ 
%         \overline{\kern-\dimen@ \box\usefulbox\kern\dimen@ }\kern-\dimen@ }}
%      {{\setbox\usefulbox=\hbox{$\m@th\scriptstyle #1$}%
%         \dimen@ \getslant\the\scriptfont\symletters \ht\usefulbox
%         \divide\dimen@ \tw@ 
%         \kern\dimen@ 
%         \overline{\kern-\dimen@ \box\usefulbox\kern\dimen@ }\kern-\dimen@ }}
%      {{\setbox\usefulbox=\hbox{$\m@th\scriptscriptstyle #1$}%
%         \dimen@ \getslant\the\scriptscriptfont\symletters \ht\usefulbox
%         \divide\dimen@ \tw@ 
%         \kern\dimen@ 
%         \overline{\kern-\dimen@ \box\usefulbox\kern\dimen@ }\kern-\dimen@ }}%
%      {}}
%     \makeatother




\newcommand{\acknowledgmentsEN}{Acknowledgements}
\newcommand{\acknowledgmentsUS}{Acknowledgments}

% to put two examples next to eachother
%\newcommand{\shortbox}[3][-.7]{
%    \parbox[t]{.4\textwidth}{
%      \vspace{#1\baselineskip} #2\strut~~ #3}%
%}

\newcommand{\twomulticolexamples}[2]{
\begin{tabular}[t]{@{}l@{~~}l@{\hspace{1em}}l@{~~}l@{}}
a. & \parbox[t]{.4\textwidth}{#1} & b. & \parbox[t]{.4\textwidth}{#2}\\
\end{tabular}
}




% This does a linebreak for \gll for long sentences leaving space for the language at the right
% margin.
% St.Mü. 17.06.2021
\newcommand{\longexampleandlanguage}[2]{%
\begin{tabularx}{\linewidth}[t]{@{}X@{}p{\widthof{(#2)}}@{}}%
\begin{minipage}[t]{\linewidth}%
#1%
\end{minipage} & (\ili{#2})%
\end{tabularx}}



\renewcommand{\indexccg}{\is{Categorial Grammar (CG)!Combinatorial \textasciitilde{} (CCG)}\xspace}
\newcommand{\indexccgstart}{\is{Categorial Grammar (CG)!Combinatorial \textasciitilde{} (CCG)|(}\xspace}
\newcommand{\indexccgend}{\is{Categorial Grammar (CG)!Combinatorial \textasciitilde{} (CCG)|)}\xspace}
\renewcommand{\indexmp}{\is{Minimalism}\xspace}


\newcommand{\gisu}{Giuseppe Varaschin\xspace}

\newcommand{\NPi}{NP$\mkern-1mu_i$\xspace}
\newcommand{\NPj}{NP$\mkern-1.5mu_j$\xspace}
  %% -*- coding:utf-8 -*-

%%%%%%%%%%%%%%%%%%%%%%%%%%%%%%%%%%%%%%%%%%%%%%%%%%%%%%%%%%%%
%
% gb4e

% fixes problem with to much vertical space between \zl and \eal due to the \nopagebreak
% command.
\makeatletter
\def\@exe[#1]{\ifnum \@xnumdepth >0%
                 \if@xrec\@exrecwarn\fi%
                 \if@noftnote\@exrecwarn\fi%
                 \@xnumdepth0\@listdepth0\@xrectrue%
                 \save@counters%
              \fi%
                 \advance\@xnumdepth \@ne \@@xsi%
                 \if@noftnote%
                        \begin{list}{(\thexnumi)}%
                        {\usecounter{xnumi}\@subex{#1}{\@gblabelsep}{0em}%
                        \setcounter{xnumi}{\value{equation}}}
% this is commented out here since it causes additional space between \zl and \eal 06.06.2020
%                        \nopagebreak}%
                 \else%
                        \begin{list}{(\roman{xnumi})}%
                        {\usecounter{xnumi}\@subex{(iiv)}{\@gblabelsep}{\footexindent}%
                        \setcounter{xnumi}{\value{fnx}}}%
                 \fi}
\makeatother

% the texlive 2020 langsci-gb4e adds a newline after \eas, the texlive 2017 version was OK.
% \makeatletter
% \def\eas{\ifnum\@xnumdepth=0\begin{exe}[(34)]\else\begin{xlist}[iv.]\fi\ex\begin{tabular}[t]{@{}p{.98\linewidth}@{}}}
% \makeatother



%%%%%%%%%%%%%%%%%%%%%%%%%%%%%%%%%%%%%%%%%%%%%%%%%%%%%%%%%%
%
% biblatex

% biblatex sets the option autolang=hyphens
%
% This disables language shorthands. To avoid this, the hyphens code can be redefined
%
% https://tex.stackexchange.com/a/548047/18561

\makeatletter
\def\hyphenrules#1{%
  \edef\bbl@tempf{#1}%
  \bbl@fixname\bbl@tempf
  \bbl@iflanguage\bbl@tempf{%
    \expandafter\bbl@patterns\expandafter{\bbl@tempf}%
    \expandafter\ifx\csname\bbl@tempf hyphenmins\endcsname\relax
      \set@hyphenmins\tw@\thr@@\relax
    \else
      \expandafter\expandafter\expandafter\set@hyphenmins
      \csname\bbl@tempf hyphenmins\endcsname\relax
    \fi}}
\makeatother


% the package defined \attop in a way that produced a box that has textwidth
%
\def\attop#1{\leavevmode\begin{minipage}[t]{.995\linewidth}\strut\vskip-\baselineskip\begin{minipage}[t]{.995\linewidth}#1\end{minipage}\end{minipage}}


%%%%%%%%%%%%%%%%%%%%%%%%%%%%%%%%%%%%%%%%%%%%%%%%%%%%%%%%%%%%%%%%%%%%


% Don't do this at home. I do not like the smaller font for captions.
% This does not work. Throw out package caption in langscibook
% \captionsetup{%
% font={%
% stretch=1%.8%
% ,normalsize%,small%
% },%
% width=\textwidth%.8\textwidth
% }
% \setcaphanging


  \togglepaper[2]
}{}


\title{The evolution of HPSG}
\author{%
	Dan Flickinger\affiliation{Stanford University}%
	\and Carl Pollard\affiliation{Ohio State Universitiy}
	\lastand Tom Wasow\affiliation{Stanford University}%
}

\abstract{HPSG was developed to express insights from theoretical linguistics in a precise formalism that was computationally tractable.  It drew ideas from a wide variety of traditions in linguistics, logic, and computer science.  Its chief architects were Carl Pollard and Ivan Sag, and its most direct precursors were Generalized Phrase Structure Grammar and Head Grammar.  The theory has been applied in the construction of computational systems for the analysis of a variety of languages; a few of these systems have been used in practical applications.  This chapter sketches the history of the development and application of the theory.}


\begin{document}
\maketitle
\label{chap-evolution}


\section{Introduction} 

From its inception in 1983, HPSG was intended to serve as a framework for the formulation and implementation of natural language grammars which are (i) linguistically motivated, (ii) formally explicit, and (iii) computationally tractable. These desiderata are reflective of HPSG's dual origins as an academic linguistic theory and as part of an industrial grammar implementation project with an eye toward potential practical applications. Here (i) means that the grammars are intended as scientific theories about the languages in question, and that the analyses the grammars give rise to are transparently relatable to the predictions (empirical consequences) of those theories. Thus HPSG shares the general concerns of the theoretical linguistics literature, including distinguishing between well-formed and ill-formed expressions and capturing linguistically significant generalizations.  (ii) means that the notation for the grammars and its interpretation have a precise grounding in logic, mathematics, and theoretical computer science, so that there is never any ambiguity about the intended meaning of a rule or principle of grammar, and so that grammars have determinate empirical consequences. (iii) means that the grammars can be translated into computer programs that can handle linguistic expressions embodying the full range of complex interacting phenomena that naturally occur in the target languages, and can do so with a tolerable cost in space and time resources.

The two principal architects of HPSG were Carl Pollard\aimention{Carl Pollard} and Ivan
Sag\aimention{Ivan Sag}, but a great many other people made important contributions to its
development.  Many, but by no means all, are cited in the chronology presented in the following
sections.  There are today a number of groups of HPSG researchers around the world, in many cases
involved in building HPSG-based computational systems.  While the number of practitioners is
relatively small, it is a very active community that holds annual meetings and publishes quite
extensively.\footnote{\added{See \url{https://hpsg.hu-berlin.de/HPSG-Bib/} for a list of HPSG
    publications.}} Hence, although Pollard no longer works on HPSG and Sag died in 2013, the theory
is very much alive, and still evolving.

\section{Precursors}

HPSG arose between 1983 and 1985 from the complex interaction between two lines of research in theoretical linguistics: (i) work on context-free Generative Grammar (CFG) initiated in the late 1970s by Gerald Gazdar and Geoffrey Pullum, soon joined by Ivan Sag, Ewan Klein, Tom Wasow, and others, resulting in the framework referred to as Generalized Phrase Structure Grammar (GPSG: \citealt*{GKPS85a}); and (ii) Carl Pollard's Stanford dissertation research, under Sag and Wasow's supervision, on Generalized Context-Free Grammar, and more specifically Head Grammar (HG: \citealt{Pollard84a-u}).

\subsection{Generalized Phrase Structure Grammar}

In the earliest versions of Generative Grammar \citep{Chomsky57a}, the focus was on motivating
transformations to express generalizations about classes of sentences.  In the 1960s, as generative
linguists began to attend more explicitly to meaning, a division arose between those advocating
using the machinery of transformations to capture semantic generalizations and those advocating the
use of other types of formal devices.  This division became quite heated, and was subsequently
dubbed ``the linguistic wars'' (see \citealt[Chapter 5]{Newmeyer:1980}; \citealt{Harris93a}\addedthis).  Much of the work in theoretical syntax and semantics during the 1970s explored ways to constrain the power of transformations (see especially \citealt{Chomsky73a} and \citealt{ChomLas1977}), and non-transformational approaches to the analysis of meaning (see especially \citealt{Montague74a-ed} and \citealt{Dowty79a}).

These developments led a few linguists to begin questioning the central role transformations had played in syntactic research of the preceding two decades (notably, \citealt{Bresnan78a}).  This questioning of Transformational Grammar (TG) culminated in a series of papers by Gerald Gazdar, which (in those pre-internet days) were widely distributed as paper manuscripts.  The project that they laid out was succinctly summarized in one of Gazdar's later publications as follows:

\begin{quote}
Consider eliminating the transformational component of a generative grammar. \citep[\page 155]{Gazdar81a}
\end{quote}

\noindent
The framework that emerged became known as \isi{Generalized Phrase Structure Grammar}; a good account of its development is Ted Briscoe's interview of Gazdar in November 2000.\footnote{\url{https://nlp.fi.muni.cz/~xjakub/briscoe-gazdar/}, 2021-01-15.}

GPSG developed in response to several criticisms leveled against transformational grammar. First, TG was highly underformalized, to the extent that it was unclear what its claims---and the empirical consequences of those claims---amounted to; CFG, by comparison, was a simple and explicit mathematical formalism. Second, given the TG architecture of a context-free base together with a set of transformations, the claimed necessity of transformations was standardly justified on the basis of arguments that CFGs were insufficiently expressive to serve as a general foundation for natural language (NL) grammar; but \citet{PG82a-u} showed all such arguments presented up to that time to be logically flawed or else based on false empirical claims. And third, closely related to the previous point, they showed that transformational grammarians had been insufficiently resourceful in exploiting what expressive power CFGs \emph{did} possess, especially through the use of complex categories bearing features whose values might themselves bear features of their own.  For example, \isi{coordinate constructions} and \isi{unbounded dependency constructions} had long served as prime exemplars of the need for transformations, but \citet{Gazdar81a} was able to show that both kinds of constructions, as well as interactions between them, did in fact yield straightforward analyses within the framework of a CFG.
 
Gazdar and Pullum's early work in this vein was quickly embraced by Sag and Wasow at Stanford University, both formally inclined former students of Chomsky's, who saw it as the logical conclusion of a trend in Chomskyan syntax toward constraining the transformational component. That trend, in turn, was a response, at least in part, to (i) the demonstration by \citet{PR73a-u} that \citegen{Chomsky65a} Standard Theory, when precisely formalized, was totally unconstrained, in the sense of generating all recursively enumerable languages; and (ii) the insight of \citet{Emonds76a-u} that most of the transformations proposed up to that time were ``structure-preserving'' in the sense that the trees they produced were isomorphic to ones that were base-generated. Besides directly addressing these issues of excess power and structure preservation, the hypothesis that NLs were context-free also had the advantage that CFGs were well-known by computer scientists to have decidable recognition problems and efficient parsing algorithms, facts which seemed to have some promise of bearing on questions of the psychological plausibility and computational tractability of the grammars in question.

Aside from serving as a framework for theoretical linguistic research, GPSG also provided the theoretical underpinnings for a natural language processing (NLP) project established in 1981 by Egon Loebner at Hewlett-Packard Laboratories in Palo Alto. This project, which led in due course to the first computer implementation of HPSG, is described below.

\subsection{Head Grammar}

Pollard, with a background in pure mathematics, \ili{Chinese} historical phonology, and
1930s--1950s-style American structural linguistics, arrived at Stanford in 1979 with the intention
of getting a PhD in \ili{Chinese} linguistics, but was soon won over to theoretical syntax by
Wasow and Sag. He had no exposure to Chomskyan linguistics, but was immediately attracted to the
emerging nontransformational approaches, especially the early GPSG papers and the contemporaneous
forms of CG in \citet{Bach79a,Bach80a} and \citeauthor{Dowty82a-u} \citeyearpar{Dowty82a-u,Dowty82b}, in part because of their formal simplicity and rigor, but also because the formalism of CFG was (and is) easy to read as a more technically precise rendering of structuralist ideas about syntax (as presented, e.g., in \citealt{Bloomfield33a-u} and \citealt{Hockett1958}).

Although \citet{PG82a-u} successfully refuted all published arguments to date that CFGs were inadequate for analyzing NLs, by the following year, Stuart Shieber had developed an argument (published in \citealt{Shieber85a}), which was (and remains) generally accepted as correct, that there could not be a CFG that accounted for the cross-serial dependencies in Swiss German\il{German!Swiss}; and Chris Culy showed, in his Stanford M.A. thesis (cf.\ \citealt{Culy85a}), that the presence of reduplicative compounding in \ili{Bambara} precluded a CF analysis of that language. At the same time, Bach and Dowty (independently) had been experimenting with generalizations of traditional A-B (Ajdukiewicz-Bar Hillel) CG which allowed for modes of combining strings (such as \isi{reduplication}, \isi{wrapping}, \isi{insertion}, \isi{cliticization}, and the like) in addition to the usual \isi{concatenation}. This latter development was closely related to a wider interest among nontransformational linguists of the time in the notion of \isi{discontinuous constituency}, and also had an obvious affinity to \citegen{Hockett54a} item-and-process conception of linguistic structure, albeit at the level of words and phrases rather than morphemes. One of the principal aims of Pollard's dissertation work was to provide a general framework for syntactic (and semantic) analysis that went beyond---but not too far beyond---the limits of CFG in a way that took such developments into account. 

Among the generalizations of CFG that Pollard studied, special attention was given to HGs, which differ from CFGs in two respects: (i) the role of strings was taken over by headed strings, essentially strings with a designation of one of its words as its head; and (ii) besides concatenation, headed strings can also be combined by inserting one string directly to the left or right of another string's head. An appendix of his dissertation \citep[Appendix~1]{Pollard84a-u} provided an analysis of discontinuous constituency in \ili{Dutch}, and that analysis also works for Swiss German\il{German!Swiss}. In another appendix, Pollard used a generalization of the \isi{CKY algorithm} to prove that the head languages (HLs, the languages analyzed by HGs) shared with CFLs the property of deterministic polynomial time recognition complexity, but of order $n^{7}$, subsequently reduced by \citet*{Kasamietal1989} to $n^{6}$, as compared with order $n^{3}$ for CFLs. For additional formal properties of HGs, see \citet{Roach1987}. \citet{VijayWeir1994} proved that HGs had the same weak generative capacities as three other grammar formalisms -- Combinatory Categorial Grammar \citep{Steedman87a-u,Steedman90a-u}, Lexicalized Tree-Adjoining Grammar \citep{Shabes90}, and Linear Indexed Grammar \citep{Gazdar88a-u} -- and the corresponding class of languages became known as \emph{\isi{mildly context sensitive}}.

Although the handling of linearization in HG seems not to have been pursued further within the HPSG framework, the ideas that (i) linearization had to involve data structures richer than strings of phoneme strings, and (ii) the way these structures were linearized had to involve operations other than mere concatenation, were implicit in subsequent HPSG work, starting with \citegen[\page 169]{ps} \isi{Constituent Order Principle} (which was really more of a promissory note than an actual principle).  These and related ideas would become more fully fleshed out a decade later within the \isi{linearization grammar} avatar of HPSG developed by \citet{Reape90a}, \citet{Reape92a}, \citet{Kathol95a,Kathol2000a}, and \citet{Mueller95c,Babel,Mueller99a,Mueller2004b}\addedthis. \added{(See also \crossrefchapterw[Section~\ref{sec-domains}]{order} on linearization approaches in HPSG.)} On the other hand, two other innovations of HG, both related to the system of syntactic features, were incorporated into HPSG, and indeed should probably be considered the defining characteristics of that framework, namely the list-valued \subcat and \slasch features, discussed below.


\section{The HP NL project}

Work on GPSG culminated in the \citeyear{GKPS85a} book \textit{Generalized Phrase Structure Grammar} by Gazdar,  Klein, Pullum, and Sag.  During the writing of that book, Sag taught a course on the theory, with participation of his co-authors.  The course was attended not only by Stanford students and faculty, but also by linguists from throughout the area around Stanford, including the Berkeley and Santa Cruz campuses of the University of California, as well as people from nearby industrial labs.  One of the attendees at this course was Anne Paulson\aimention{Anne Paulson}, a programmer from Hewlett-Packard (HP) Laboratories in nearby Palo Alto, who had some background in linguistics from her undergraduate education at Brown University.  Paulson told her supervisor at HP Labs, Egon Loebner\aimention{Egon Loebner}, that she thought the theory could be implemented and might be turned into something useful. Loebner, a multi-lingual polymathic engineer, had no background in linguistics, but he was intrigued, and invited Sag to meet and discuss setting up a natural language processing project at HP. Sag brought along Gazdar, Pullum, and Wasow. This led to the creation of the project that eventually gave rise to HPSG.  Gazdar, who would be returning to England relatively soon, declined the invitation to be part of the new project, but Pullum, who had taken a position at the University of California at Santa Cruz (about an hour's drive from Palo Alto), accepted.  So the project began with Sag, Pullum, and Wasow hired on a part-time basis to work with Paulson and two other HP programmers, John Lamping\aimention{John Lamping} and Jonathan King\aimention{Jonathan King}, to implement a GPSG of \ili{English} at HP Labs.  J. Mark Gawron\aimention{J. Mark Gawron}, a linguistics graduate student from Berkeley who had attended Sag's course, was very soon added to the team.

The initial stages consisted of the linguists and programmers coming up with a notation that would serve the purposes of both.  Once this was accomplished, the linguists set to work writing a grammar of \ili{English} in Lisp to run on the \mbox{DEC-20} mainframe computer that they all worked on.   The first publication coming out of this project was a 1982 Association for Computational Linguistics paper.  The paper's conclusion begins:
\begin{quote}
What we have outlined is a natural language system that is a direct implementation of a linguistic theory.  We have argued that in this case the linguistic theory has the special appeal of computational tractability (promoted by its context-freeness), and that the system as a whole offers the hope of a happy marriage of linguistic theory, mathematical logic, and advanced computer applications. \citep[\page 80]{Gaw:Kin:Lam:82}
\end{quote}
This goal was carried over into HPSG.

It should be mentioned that the HP group was by no means alone in these concerns.  The early 1980s was a period of rapid growth in computational linguistics (due at least in part to the rapid growth in the power and accessibility of computers).  In the immediate vicinity of Stanford and HP Labs, there were at least two other groups working on developing natural language systems that were both computationally tractable and linguistically motivated.  One such group was at the Xerox Palo Alto Research Center, where Ron Kaplan\aimention{Ron Kaplan} and Joan Bresnan\aimention{Joan Bresnan} (in collaboration with a number of other researchers, notably Martin Kay\aimention{Martin Kay}) were developing \isi{Lexical Functional Grammar};\footnote{\added{For a comparison of HPSG and LFG see \crossrefchapterp{lfg}. A handbook of LFG parallel to this handbook is in preparation \citep{LFGhandbook}.}}  the other was at SRI International, where a large subset of SRI's artificial intelligence researchers  (including Barbara Grosz\aimention{Barbara Grosz}, Jerry Hobbs\aimention{Jerry Hobbs}, Bob Moore\aimention{Bob Moore}, Hans Uszkoreit\aimention{Hans Uszkoreit}, Fernando Pereira\aimention{Fernando Pereira}, and Stuart Shieber\aimention{Stuart Shieber}) worked on natural language.  Thanks to the founding of the Center for the Study of Language and Information (CSLI) at Stanford in the early 1980s, there was a great deal of interaction among these three research groups.  Although some aspects of the work being done at the three non-Stanford sites were proprietary, most of the research was basic enough that there was a fairly free flow of ideas among the three groups about building linguistically motivated natural language systems.

Other projects seeking to develop theories combining computational tractability with linguistic motivation were also underway outside of the immediate vicinity of Stanford, notably at the Universities of Pennsylvania and Edinburgh.  Aravind Joshi and his students were working on Tree Adjoining Grammars \citep*{JLT75a-u,Joshi87a-u}, while Mark Steedman and others were developing Combinatory Categorial Grammar \citep{Steedman87a-u,Steedman90a-u}.

During the first few years of the HP NL project, several Stanford students were hired as part-time help.   One was Pollard, who was writing his doctoral dissertation under Sag's supervision.  Ideas from his thesis work played a major role in the transition from GPSG to HPSG.  Two other students who became very important to the project were Dan Flickinger\aimention{Dan Flickinger}, a doctoral student in linguistics, and Derek Proudian\aimention{Derek Proudian}, who was working on an individually-designed undergraduate major when he first began at HP and later became a master's student in computer science.  Both Flickinger and Proudian became full-time HP employees after finishing their degrees.   Over the years, a number of other HP employees also worked on the project and made substantial contributions.  They included Susan Brennan\aimention{Susan Brennan}, Lewis Creary\aimention{Lewis Creary}, Marilyn Friedman\aimention{Marilyn Friedman} (now Walker\aimention{Marilyn Walker}), Dave Goddeau\aimention{Dave Goddeau}, Brett Kessler\aimention{Brett Kessler}, Joachim Laubsch\aimention{Joachim Laubsch}, and John Nerbonne\aimention{John Nerbonne}.  Brennan, Walker, Kessler, and Nerbonne all later went on to academic careers at major universities, doing research dealing with natural language processing.

The HP NL project lasted until the early 1990s.  By then, a fairly large and robust grammar of \ili{English} had been implemented.  The period around 1990 combined an economic recession with what has sometimes been termed an ``AI winter'' -- that is, a period in which enthusiasm and hence funding for artificial intelligence research was at a particularly low ebb.  Since NLP was considered a branch of AI, support for it waned.  Hence, it was not surprising that the leadership of HP Labs decided to terminate the project.  Flickinger and Proudian came to an agreement with HP that allowed them to use the NLP technology developed by the project to launch a new start-up company, which they named \isi{Eloquent Software}.  They were, however, unable to secure the capital necessary to turn the existing system into a product, so the company never got off the ground.

\section{The emergence of HPSG}

A few important features of GPSG that were later carried over into HPSG are worth mentioning here. First, GPSG borrowed from Montague the idea that each phrase structure rule was to be paired with a semantic rule providing a recipe for computing the meaning of the mother from the meanings of its daughters \citep[\page 156]{Gazdar81a}; this design feature was shared with contemporaneous forms of Categorial Grammar (CG) being studied by such linguists as Emmon Bach \citep{Bach79a,Bach80a} and David Dowty \citep{Dowty82a-u,Dowty82b}.  Second, the specific inventory of features employed in GPSG for making fine-grained categorial distinctions (such as case, agreement, verb inflectional form, and the like), was largely preserved, though the technical implementation of morphosyntactic features in HPSG was somewhat different. And third, the \slasch feature, which originated in \citegen{Gazdar81a} derived categories (e.g. S/NP), and which was used to keep track of \isi{unbounded dependencies}, was generalized in HPSG to allow for multiple unbounded dependencies (as in the notorious violins-and-sonatas example in (\ref{ex-violins}) below). As will be discussed, this \slasch feature bears a superficial---and misleading---resemblance to the Categorial Grammar connectives written as `/' and `\textbackslash'. On the other hand, a centrally important architectural feature of GPSG absent from HPSG (and from HG) was the device of \isi{metarules}, higher-order rules used to generate the full set of context-free phrase structure rules (PSRs) from an initial inventory of basic PSRs. Among the metarules were ones used to introduce non-null \slasch values and propagate them upward through trees to a position where they were discharged by combination with a matching constituent called a \isi{filler} (analogous to a \emph{wh}-moved expression in TG).

A note is in order about the sometimes confusing use of the names \emph{Head Grammar} (\emph{HG}) and \emph{HPSG}. Strictly speaking, HG was a specific subtype of generalized CFG developed in Pollard's dissertation work, but the term \emph{HG} did not appear in academic linguistic publications with the exception of the \citet{PollardSag1983} WCCFL paper, which introduced the distinction between head features and binding features (the latter were incorporated into GPSG under the name \emph{foot features}). In the summer of 1982, Pollard had started working part time on the HP NL project; and the term \emph{HPSG} was first employed (by Pullum) in reference to an extensive reworking by Pollard and Paulson of the then-current HP GPSG implementation, incorporating some of the main features of Pollard's dissertation work in progress, carried out over the summer of 1983, while much of the HP NLP team (including Pullum and Sag) was away at the LSA Institute in Los Angeles.  The implication of the name change was that whatever this new system was, it was no longer GPSG.

Once this first HPSG implementation was in place, the NLP work at HP was considered to be within the framework of HPSG, rather than GPSG. After Pollard completed his dissertation, he continued to refer to \emph{HG} in invited talks as late as autumn 1984; but his talk at the (December~1984) LSA Binding Theory Symposium used \emph{HPSG} instead, and after that, the term \emph{HG} was supplanted by \emph{HPSG} (except in publications by non-linguists about formal language theory). One additional complication is that until the \citet*{GKPS85a} volume appeared, GPSG and HPSG were developing side by side, with  considerable interaction.  Pollard, together with Flickinger, Wasow, Nerbonne, and others, did HPSG; Gazdar and Klein did GPSG; and Sag and Pullum worked both sides of the street.

HPSG papers, about both theory and implementation, began to appear in 1985, starting with Pollard's WCCFL paper \emph{Phrase structure grammar without metarules} \citep{Pollard85a-u}, and his paper at the Categorial Grammar conference in Tucson \citep{Pollard88a}, comparing and contrasting HPSG with then-current versions of Categorial Grammar due to Bach, Dowty, and Steedman. These were followed by a trio of ACL papers documenting the current state of the HPSG implementation at HP Labs: \citet{creary-pollard:1985:ACL}, \citet*{FPW85a}, and \citet{PP85}. Of those three, the most significant in terms of its influence on the subsequent development of the HPSG framework was the second, which showed how the lexicon could be (and in fact was) organized using multiple-inheritance knowledge representation; Flickinger's Stanford dissertation \citep{Flickinger87} was an in-depth exploration of that idea.

\section{Early HPSG}

Setting aside implementation details, early HPSG can be characterized by the following architectural features:

\paragraph*{Elimination of metarules} Although \isi{metarules} were a central feature of GPSG, they were also problematic: \citet{UszPet1982} had shown  that if metarules were allowed to apply to their own outputs, then the resulting grammars were no longer guaranteed to generate CFLs; indeed, such grammars could generate all recursively enumerable languages. And so, in GPSG, the closure of a set of base phrase structure rules (PSRs) under a set of metarules was defined in such a way that no metarule could apply to a PSR whose own derivation involved an application of that metarule. This definition was intended to ensure that the closure of a finite set of PSRs remained finite, and therefore still constituted a CFG.

So, for example, the metarule STM1 was used in GPSG to convert a PSR into another PSR, one of whose daughters is [+NULL] (informally speaking, a ``trace''), and \isi{feature cooccurrence restrictions} (FCRs) guaranteed that such daughters would bear a \slasch value, and that this \slasch value would also appear on the mother. Unfortunately, the \isi{finite closure} definition described above does not preclude the possibility of derived PSRs whose mother carries multiple, in fact unboundedly many
\slasch values (e.g. NP/NP, (NP/NP)/NP, etc.). And this in turn leads to an infinite set of PSRs, outside the realm of CF-ness (see \citealt{Ristad86}).  Of course, one could rein in this excess power by imposing another FCR that disallows categories of the form (X/Y)/Z; but then there is no way to analyze sentences containing a constituent with two undischarged unbounded dependencies, such as the VP complement of \emph{easy} in the following example:

\ea
\label{ex-violins}
Violins this finely crafted, even the most challenging sonatas are easy to [play \_ on \_]. \added{(adapted from \citealt[\page 169]{ps2})}
\z

\noindent
GPSG avoided this problem by not analyzing such examples. In HPSG \citep{Pollard85a-u}, by contrast, such examples were analyzed straightforwardly by replacing GPSG's category-valued \slasch feature with one whose values were lists (or sets) of categories. This approach still gave rise to an infinite set of rules, but since maintaining context-freeness was no longer at stake, this was not seen as problematic. The infinitude of rules in HPSG arose not through a violation of finite closure (since there were no longer any metarules at all), but because each of the handful of schematic PSRs (see below) could be directly instantiated in an infinite number of ways, given that the presence of list-valued
features gave rise to an infinite set of categories.

\paragraph*{Lexical rules} GPSG, generalizing a suggestion of \citet{Flickinger1983}, constrained metarules to apply only to PSRs that introduced a lexical head. \citet{Pollard85a-u} took this idea a step further, noting that many proposed metarules could be reformulated as lexical rules that (among other effects) operated on the subcategorization frames (encoded by the \subcat feature discussed below) of lexical entries. The idea of capturing some linguistic generalizations by means of rules internal to the lexicon had been explored by generative grammarians since \citet{Jackendoff75a}; and lexical rules of essentially the kind Pollard proposed were employed by \citet{Bach83}, \citet{Dowty78a}, and others working in Categorial Grammar. Examples of constructions handled by metarules in GPSG but in HPSG by lexical rules included \isi{sentential extraposition}, \isi{subject extraction}, and the \isi{passive}. \citet*{FPW85a} argued for an architecture for the lexicon that combined lexical rules with multiple inheritance using a frame-based knowledge representation system \citep{Minsky1975}, on the basis of both overall grammar simplicity and efficient, easily modifiable implementation.

\paragraph*{CG-like treatment of subcategorization} GPSG treated subcategorization using an integer-valued feature called \subcat that in effect indexed each lexical item with the rule that introduced and provided its subcategorization frame; e.g. \emph{weep} was listed in the lexicon with \subcat value 1 while \emph{devour} was listed with \subcat value 2, and then PSRs of roughly the form in (\mex{1}) guaranteed that lexical heads would have the right kinds of complements.
\ea
\begin{tabular}[t]{l}
          VP $\rightarrow$ V[\subcat 1] \\
          VP $\rightarrow$ V[\subcat 2] NP
\end{tabular}
\z

\noindent
In HPSG, by contrast, the
\subcat feature directly characterized the grammatical arguments selected by a head (not just the
complements, but the subject too) as a list of categories, so that e.g.~\emph{weep} was listed as
V[\subcat \liste{ NP }] but \emph{devour} as V[\subcat \liste{ NP, NP }] (where the first occurrence of NP refers to the object and the second to the subject). This treatment of argument selection was inspired by Categorial Grammar, where the same verbs would have been categorized as NP{\textbackslash}S and (NP{\textbackslash}S)/NP respectively;\footnote{We adhere to the Lambek convention for functor categories, so that expressions seeking to combine with an A on the left to form a B are written ``A{\textbackslash}B'' (not ``B{\textbackslash}A'').} the main differences are that (i) the CG treatment also encodes the directionality of the argument relative to the head, and (ii) in HPSG, all the arguments appear on one list, while in CG they are ``picked up'' one at a time, with as many
connectives (/ or \textbackslash) as there are arguments. In particular, as in the CG of \citet{Dowty82b}, the subject was defined as the last argument, except that in HPSG, ``last'' now referred to the rightmost position on the \subcat list, not to the most deeply embedded connective. In HPSG, this
ordering of the categories on the \subcat list was related not just to CG, but also to the traditional grammatical notion of obliqueness, and also to the accessibility hierarchy of \citet{KC77a}.  See \citet[Section 4]{MWArgSt} for a more detailed discussion of these developments from GPSG to HPSG.

\paragraph*{Schematic rules} Unlike CFG but like CG, HPSG had only a handful of schematic rules. For example, in \citet{Pollard85a-u}, a substantial chunk of \ili{English} ``local'' grammar (i.e.~leaving aside unbounded dependencies) was handled by three rules: (i) a rule (used for subject-auxiliary inversion) that forms a sentence from an inverted (+\textsc{inv}) lexical head and all its arguments; (ii) a rule that forms a phrase from a head with \subcat list of length > 1 together with all its non-subject arguments; and (iii) a rule that forms a sentence from a head with a \subcat value of length one together with its single (subject) argument.

\paragraph*{List- (or set-) valued \slasch feature} The list-valued \slasch was introduced in \citet{Pollard85a-u} to handle multiple unbounded dependencies, instead of the GPSG category-valued \slasch (which in turn originated as the \emph{derived categories} of  \citet{Gazdar81a}, e.g.~S/NP). In spite of the notational similarity, though, the PSG \slasch is not an analog of the CG slashes / and \textbackslash \ (though HPSG's \subcat is, as explained above). In fact, HPSG's \slasch has no analog in the kinds of CGs being developed by Montague semanticists such as \citet{Bach79a,Bach80a} and \citet{Dowty82a-u} in the late 1970s and early 1980s, which followed the CGs of \citet{Bar-Hillel54-u} in having only rules for eliminating (or canceling) slashes as in (\mex{1}):

\ea
\begin{tabular}[t]{ccc}
A A{\textbackslash}B  & &     B/A A \\ \cline{1-1} \cline{3-3}
B & & B 
\end{tabular}
\z

\noindent
To find an analog to HPSG's \slasch in CG, we have to turn to the kinds of CGs invented by \citet{Lambek1958}, which unfortunately were not yet well-known to linguists (though that would soon change, starting with Lambek's appearance at the 1985 Categorial Grammar conference in Tucson). What sets apart grammars 
of this kind (and their elaborations by \citet{Moortgat89a-u}, \citet{OBW88a-ed}, \citet{Morrill94a-u}, and many others), is the existence of rules for hypothetical proof (not given here), which allow a hypothesized category occurrence introduced into a tree (thought of as a proof) to be discharged. 

In the Gentzen\addref style of natural deduction (see \citealt{Pollard:2013}), hypothesized categories are written to the left of the symbol $\vdash$ (turnstile), so that the two slash elimination rules above take the following form, where $\Gamma$ and $\Delta$ are lists of categories, and comma represents list concatenation as in (\mex{1}):

\ea
\begin{tabular}[t]{ccc}
$\Gamma$ $\vdash$ A \ \ $\Delta$ $\vdash$ A{\textbackslash}B  & &     $\Gamma$ $\vdash$ B/A \ \ $\Delta$ $\vdash$ A \\ \cline{1-1} \cline{3-3}
$\Gamma$,$\Delta$ $\vdash$ B & & $\Gamma$,$\Delta$ $\vdash$ B 
\end{tabular}
\z

\noindent
These rules serve to propagate hypotheses (analogous to linguists' traces) downward through the proof tree (downward because logicians' trees are
upside down with the conclusion, or ``root'', at the bottom). In HPSG notation, these same rules can be written as one rule (since \subcat is
nondirectional) in (\mex{1}):

\ea
\begin{tabular}[t]{c}
B[\subcat \liste{\ldots, A}, \slasch $\Gamma$] \ \ A[\slasch $\Delta$] \\ \cline{1-1}
B[\subcat \liste{\ldots}][\slasch $\Gamma$,$\Delta$]
\end{tabular}
\z

This in turn is a special case of an HPSG principle first known as the \isi{Binding Inheritance Principle} (BIP) and later as the \isi{Nonlocal Feature Principle} (binding features included \slasch as well as the features \textsc{que}
and \textsc{rel} used for tracking undischarged interrogative and relative pronouns). The original statement of the BIP \citep{Pollard:1986} treated \slasch as set- rather than list-valued:

\begin{quote}
The value of a binding feature on the mother is the union of the values of that feature on the daughters. \citep{Pollard:1986}\addedthis
\end{quote}

\noindent
For example, the doubly-gapped VP in the violins-and-sonatas example in (\ref{ex-violins}) is analyzed in HPSG roughly  as
is shown in Figure~\ref{fig-play-on}
\begin{figure}
\centerfit{%
\begin{forest}
sm edges
[{V[\subcat \sliste{ NP }, \slasch \sliste{ NP, NP }]}
  [{V[\subcat \sliste{ PP, NP, NP }]} [play]]
  [{NP[\slasch \sliste{ NP }]} [t]]
  [{PP[\slasch \sliste{ NP }]}
    [{P[\subcat \sliste{ NP }]} [on]]
    [{NP[\slasch \sliste{ NP }]} [t]]]]
\end{forest}
}
\caption{\label{fig-play-on}\emph{play on} as part of \emph{Violins this finely crafted, even the most challenging sonatas are easy to play on.}}
\end{figure}
and essentially the same way in Lambek-style CG:

\ea
\begin{tabular}[t]{ccccccc}
\multicolumn{2}{c}\emph{play} & \emph{t} & & \emph{on} & \ \ \ \ \ \ \emph{t} & \\
\multicolumn{2}{c}{$\vdash$ ((NP{\textbackslash}S)/PP)/NP} & NP $\vdash$ NP & & $\vdash$ PP/NP & \multicolumn{2}{c}{NP $\vdash$ NP} \\ \cline{1-3} \cline{5-7}
 & \multicolumn{2}{c}{NP $\vdash$ (NP{\textbackslash}S)/PP} & & \multicolumn{3}{c}{NP $\vdash$ PP} \\ \cline{2-6}
\multicolumn{7}{c}{NP,NP $\vdash$ (NP{\textbackslash}S)}
\end{tabular}
\z

\noindent
Aside from the binary branching of the \citeauthor{Lambek1958} analysis, the main difference is that HPSG traces of
the form A[\slasch \liste{ A }] correspond
to Lambek axioms of the form A $\vdash$ A, which is the standard mechanism for introducing hypotheses in Gentzen-style natural deduction.

An overview and elaboration of early HPSG is provided by the two books \citet{ps} and \citet{ps2}. Confusingly, the former is called \textit{Information-Based Syntax and Semantics, Volume 1: Fundamentals}, and the second simply \textit{Head-Driven Phrase Structure Grammar} (not \textit{Information-Based Syntax and Semantics, Volume 2}). The reason for the title change had to do with a change in the underlying mathematical theory of feature structures. In the first book, following work in theoretical computer science by \citet{RoundsKasper1986} and \citet{MoshierRounds1987}, feature structures were treated as data structures that supplied partial information about the linguistic objects being theorized about; this perspective in turn was based on \citegen{Scott1982} mathematical theory of computation in terms of what he called information systems. Subsequently, Paul King\aimention{Paul King} persuaded Pollard and Sag that it was more straightforward to distinguish between \isi{feature structures}, thought of as formal models of the linguistic objects, and \isi{feature descriptions} or formulas of feature logic, which provided partial information about them, as described in his Manchester dissertation \citep{King89}.  Although the formal issues involved in distinguishing between the two approaches are of interest in their own right, they seem not to have had a lasting effect on how theoretical linguists used HPSG, nor on how computational linguists implemented it. As for subject matter, \citet{ps} was limited to the most basic notions, including syntactic features and categories (including the distinction between head features and binding features); subcategorization and the distinction between arguments and adjuncts (the latter of which necessitated one more rule schema beyond the three proposed by \citealt{Pollard85a-u}); basic principles of grammar (especially the \isi{Head Feature Principle} and the \isi{Subcategorization Principle}); the \isi{obliqueness order} and \isi{constituent ordering}; and the \isi{organization of the lexicon} by means of a \isi{multiple inheritance} hierarchy and lexical rules\is{lexical rule}. \citet{ps2} used HPSG to analyze a wide range of phenomena, primarily in \ili{English}, that had figured prominently in the syntactic literature of the 1960s--1980s, including \isi{agreement}, \isi{expletive pronoun} constructions, \isi{raising}, \isi{control}, \isi{filler-gap constructions} (including \isi{island constraints} and \isi{parasitic gaps}); so-called \isi{binding theory} (the distribution of reflexive pronouns, non-reflexive pronouns, and non-pronominal NPs), and \isi{scope of quantificational NPs}. \added{These topics are also handled in respective chapters of this handbook \parencites{chapters/agreement}{chapters/control-raising}{chapters/udc}{chapters/islands}{chapters/binding}[Section~\ref{sec-scope-in-hpsg}]{chapters/semantics}.}

\section{Theoretical developments}
\label{evolution:sec-theoretical-developments}

Three decades of vigorous work since \cite{ps} developing the theoretical framework of HPSG receive detailed discussion throughout the present volume, but we highlight here two significant stages in that development.  The first is in Chapter 9 of \citew{ps2}, where a pair of major revisions to the framework presented in the first eight chapters are adopted, changing the analysis of valence and of unbounded dependencies.  Following \citet{Borsley87a,Borsley88b-u,Borsley89,Borsley90a}, Pollard and Sag moved to distinguish subjects from complements, and further to distinguish subjects from specifiers, thus replacing the single \subcat attribute with \subj, \spr, and \comps.  This formal distinction between subjects and complements enabled an improved analysis of unbounded dependencies, eliminating traces altogether by introducing three lexical rules for the extraction of subjects, complements, and adjuncts respectively.   It is this revised analysis of valence constraints that came to be viewed as part of the standard HPSG framework, though issues of valence representation cross-linguistically remain a matter of robust debate.

The second notable stage of development was the introduction of a type hierarchy of \emph{constructions}\is{construction} as descriptions of phrasal feature structures, employed first by \citet{Sag97a} in a richly detailed analysis of a wide variety of \isi{relative clause} phenomena in \ili{English}.  This extension from the lexicon of the use of descriptions of typed feature structures organized in hierarchies to syntactic rules preserved the ability to express general principles holding for rule schemata while also enabling expression of idiosyncratic properties of phrases.  In \crossrefchaptert{properties}, the version of the framework with this extended use of types is termed \term{Constructional HPSG}, including further elaboration by \citet{GSag2000a-u} to a comprehensive analysis of \isi{interrogatives} in \ili{English}.  

\section{The LinGO project}

In the early 1990s, a consortium of research centers in Germany secured funding from the \ili{German} government for a large project in spoken language \isi{machine translation}, called \verbmobil \citep{Wahlster2000a-ed}, which aimed to combine a variety of methods and frameworks in a single implemented state-of-the-art demonstrator system.  Grammars of \ili{German} and \ili{English} were to be implemented in HPSG, to be used both for parsing and for generation in the translation of human-human dialogues, with a \ili{German} grammar initially implemented by Pollard\aimention{Carl Pollard} and Tibor Kiss\aimention{Tibor Kiss} at IBM in Heidelberg, later replaced by one developed \added{by Stefan Müller and Walter Kasper} at the \ili{German} AI Research Center (DFKI), coordinator for the \verbmobil project.  The DFKI contracted in 1993 with Sag\aimention{Ivan Sag} at CSLI to design and implement the \ili{English} grammar, with Flickinger\aimention{Dan Flickinger} brought over from HP Labs to help lead the effort, forming a new research group at CSLI initially called ERGO (for \ili{English} Resource Grammar Online), later generalized to the name LinGO (Linguistic Grammars Online).  Early LinGO members included Wasow\aimention{Tom Wasow} and linguistics graduate student Rob Malouf\aimention{Rob Malouf}, who authored the initial implementation of the \ili{English} Resource Grammar (ERG)\is{English Resource Grammar (ERG)}, along with four other linguistics graduate students, Emily Bender\aimention{Emily Bender}, Kathryn Campbell-Kibler\aimention{Kathryn Campbell-Kibler}, Tony Davis\aimention{Tony Davis}, and Susanne Riehemann\aimention{Susanne Riehemann}.

During the first of the two four-year phases of the \verbmobil project, the focus was on designing and implementing core syntactic and semantic analyses, initially using the DISCO/PAGE platform \citep{DISCO94} developed at the DFKI, and largely informed by the framework presented in \citet{ps2}.  However, a more computationally useful semantic formalism emerged, called Minimal Recursion Semantics\indexmrs (MRS: \citealt*{CFPS2005a}), which Ann Copestake, formerly of the European \isi{ACQUILEX} project, helped to design.  Copestake also expanded the LKB\indexlkb system \citep{Copestake2002a} which had been used in ACQUILEX, to serve as the grammar development environment for the LinGO project, including both a parser and a generator for typed feature structure grammars.

The second four years of the \verbmobil project emphasized development of the generation capabilities of the ERG, along with steady expansion of linguistic coverage, and elaboration of the MRS framework.  LinGO contributors in this phase included Sag, Wasow, Flickinger, Malouf, Copestake, Riehemann, and Bender, along with a regular visitor and steady contributor from the DFKI, Stephan Oepen\aimention{Stephan Oepen}.  \verbmobil had meanwhile added \ili{Japanese} alongside \ili{German} \citep{MK2000a} and \ili{English} \citep*{FCS2000a} for more translation pairs, giving rise to another relatively broad-coverage HPSG grammar, Jacy, authored by Melanie Siegel at the DFKI \citep{Siegel2000a}.  Work continued at the DFKI, of course, on the \ili{German} HPSG grammar, written by Stefan Müller, adapted from his earlier Babel grammars \citep{Mueller99a}, and with semantics contributed by Walter Kasper.

Before the end of \verbmobil funding in 2000, the LinGO project had already begun to diversify into other application and research areas using the ERG, including over the next several years work on \isi{augmented/adaptive communication}, \isi{multiword expressions}, and hybrid processing with \isi{statistical methods}, variously funded by the National Science Foundation, the \ili{Scottish} government, and industrial partners including IBM and NTT.  At the turn of the millennium, Flickinger joined the software start-up boom, co-founding \isi{YY Software} funded through substantial venture capital to use the ERG for automated response to customer emails for e-commerce companies.  YY produced the first commercially viable software system using an HPSG implementation, processing email content in \ili{English} with the ERG and the PET parser \citep{callmeier00} which had been developed by Ulrich Callmeier at the DFKI, as well as in \ili{Japanese} with Jacy, further developed by Siegel\aimention{Melanie Siegel} and by Bender\aimention{Emily Bender}.  While technically capable, the product was not commercially successful enough to enable YY to survive the bursting of the dot-com bubble, and it closed down in 2003.  Flickinger returned to the LinGO project with a considerably more robust ERG, and soon picked up the translation application thread again, this time using the ERG for generation in the LOGON \ili{Norwegian}--\ili{English} \isi{machine translation} project \citep{Lon:Oep:Ber:04} based in Oslo.


\section{Research and teaching networks}

The first international conference on HPSG was held in 1993 in Columbus, Ohio, in conjunction with the Linguistic Society of America's Summer Institute.  The conference has been convened every year since then, with locations in Europe, Asia, and North America.  Two of these annual meetings have been held jointly with the annual Lexical Functional Grammar conference, in 2000 in Berkeley and in 2016 in Warsaw.  Proceedings of these conferences since 2000 are available on-line from CSLI Publications.\footnote{\url{http://csli-publications.stanford.edu/HPSG/}, 2021-01-15.}  Since 2003, HPSG researchers in Europe have frequently held a regional workshop in Bremen, Berlin, Frankfurt, or Paris, annually since 2012, to foster informal discussion of current work in HPSG.  These follow in the footsteps of European HPSG workshops starting with one on \ili{German} grammar, held in Saarbrücken in 1991, and including others in Edinburgh and Copenhagen in 1994, and in Tübingen in 1995.

In 1994, the HPSG mailing list was initiated,\footnote{Its archives can be found at \url{https://hpsg.hu-berlin.de/HPSG/MailingList}.} and from 1996 to 1998, the electronic newsletter, the HPSG Gazette,\footnote{\url{http://www.sfs.uni-tuebingen.de/~gazette}, 2021-01-15.} was distributed through the list, with its function then taken over by the HPSG mailing list.

Courses introducing HPSG to students became part of the curriculum during the late 1980s and early
1990s at universities in Osaka, Paris, Saarbrücken, Seoul, and Tübingen, along with Stanford
and OSU.  Additional courses came to be offered in Bochum, Bremen, Pittsburgh, Göttingen,
Heidelberg, Jena, Leuven, Potsdam, Seattle\added{, Berlin, Essex, Buffalo, and Austin}.\itdgreen{JP: Why aren’t Berlin, Buffalo, UT, or various places in the UK included?? I am sure I am forgetting many. I don’t really care, just wondering.}  Summer courses and workshops on HPSG have also been offered since the early 1990s at the LSA Summer Institute in the U.S., including a course by Sag and Pollard on binding and control in 1991 in Santa Cruz, and at the European Summer School in Logic, Language and Information (ESSLLI), including a course by Pollard in Saarbrücken in 1991 on HPSG, a workshop in Colchester in 1992 on HPSG, a workshop in Prague in 1996 on \ili{Romance} (along with two HPSG-related student papers at the first-ever ESSLLI student session), and courses in 1998 in Saarbrücken on \ili{Germanic} syntax, grammar engineering, and unification-based formalisms, in 2001 on HPSG syntax, in 2003 on linearization grammars, and more since.  Also in 2001, a \ili{Scandinavian} summer school on constraint-based grammar was held in Trondheim.

%Several HPSG textbooks have been published, including at least \citet{Borsley91a,Borsley96a-u}, \citet{SagW99a-u}, \citet*{SWB2003a}, \citet{MuellerLehrbuch1,MuellerGTBuch2,MuellerGT-Eng1}, \citet{Kim2016a-u}, and \citet{Levine2017a-u}.
Several HPSG textbooks have been published, including at least \citet{Borsley91a,Borsley96a-u}, \citet{SagW99a-u}, \citet*{SWB2003a}, \citet{MuellerLehrbuch1,MuellerGTBuch2,MuellerGT-Eng4}, \citet{Kim2016a-u}, and \citet{Levine2017a-u}.   

\section{Implementations and applications of HPSG}

The first implementation of a grammar in the HPSG framework emerged in the Hewlett-Packard Labs natural language project, for \ili{English}, with a lexical \isi{type hierarchy} \citep*{FPW85a}, a set of grammar rules that provided coverage of core syntactic phenomena including \isi{unbounded dependencies} and \isi{coordination}, and a semantic component called \isi{Natural Language Logic} \citep{LaubNerb1991}.  The corresponding parser for this grammar was implemented in Lisp \citep{PP85}, as part of a system called \isi{HP-NL} \citep{NerProud1987} which provided a natural language interface for \isi{querying relational databases}.  The grammar and parser were shelved when HP Labs terminated their natural language project in 1991, leading Sag and Flickinger to begin the LinGO project and development of the \ili{English} Resource Grammar at Stanford.

By this time, grammars in HPSG were being implemented in university research groups for several other languages, using a variety of parsers and engineering platforms for processing typed feature structure grammars.  Early platforms included the DFKI's DISCO system \citep{DISCO94} with a parser and graphical development tools, which evolved to the PAGE system; the ALE system \citep{Franz:90,CP96}, which evolved in Tübingen to TRALE \citep*{MPR2002a-u,Penn2004a-u}; and Ann Copestake's LKB\indexlkb \citep{Copestake2002a} which grew out of the ACQUILEX project.  Other early systems included ALEP within the Eurotra project \citep{SimpGron1994}, ConTroll at Tübingen \citep{GoetzMeurers1997}, CUF at IMS in Stuttgart \citep{DD93a-u}, CL-ONE at Edinburgh \citep{Manandhar1994}, TFS also at IMS \citep{Emele94a-u}, ProFIT at the University of Saarland \citep{Erbach95a}, Babel at Humboldt University in Berlin \citep{Babel}, and HDrug at Groningen \citep{NB97b-u}.  

Relatively early broad-coverage grammar implementations in HPSG, in addition to the {English Resource Grammar}\is{English Resource Grammar (ERG)} at Stanford \citep{erg}, included one for \ili{German} at the DFKI \citep{MK2000a} and one for \ili{Japanese} (Jacy: \citealt{Siegel2000a}), all used in the \verbmobil machine translation project; a separate \ili{German} grammar \citep{Babel,Mueller99a}; a \ili{Dutch} grammar in Groningen \citep*{BvNM2001a-u}; and a separate \ili{Japanese} grammar in Tokyo \citep{MNT2005a-u}.  Moderately large HPSG grammars were also developed during this period for \ili{Korean} \citep{KY2003a-u} and for \ili{Polish} \citep*{myk:etal:02}.  

In 1999, research groups at the DFKI, Stanford, and Tokyo set up a consortium called DELPH-IN (Initiative for Deep Linguistic Processing in HPSG), to foster broader development of both grammars and platform components, described in \citet*{OFTU2002a-ed}.  Over the next two decades, substantial DELPH-IN grammars were developed for \ili{Norwegian} \citep{HH2004a-u}, \ili{Portuguese} \citep{BrancoCosta2010}, and \ili{Spanish} \citep{MARIMON10.602}, along with moderate-coverage grammars for \ili{Bulgarian} \citep{Osenova2011}, \ili{Greek} \citep{KN2005a-u}, \ili{Hausa} \citep{Crysmann2012a-u}, \ili{Hebrew} \citep*{AHMW2015a-u}, \ili{Indonesian} \citep{MBS2015a-u}, \ili{Mandarin Chinese} \citep{FSB2015a-u}, \ili{Thai}, and \ili{Wambaya} \citep{Bender2008b-u}, all described at \url{http://delph-in.net}.  Several of these grammars are based on the \isi{Grammar Matrix} \citep*{BFO2002a-u}, a starter kit generalized from the ERG and Jacy for rapid prototyping of HPSG grammars, along with a much larger set of coursework grammars.\footnote{\url{http://moin.delph-in.net/MatrixTop}, 2021-01-15.}  Out of this work has grown the linguistically rich Grammar Matrix customization system \citep*{BDFPS2010a-u}, a set of libraries of phenomena enabling a grammar developer to complete a questionnaire about characteristics of a language to obtain a more effectively customized starting grammar.

Broad-coverage grammars developed in the \isi{TRALE} system \citep{MPR2002a-u,Penn2004a-u} include \ili{German} \citep{MuellerLehrbuch1}, \ili{Danish} \citep{MOeDanish}, and \ili{Persian} \citep{MuellerPersian}. Other TRALE grammars include
\ili{Mandarin Chinese} \citep{ML2013a}, \ili{Georgian} \citep{Abzianidze2011a-u}, \ili{Maltese} \citep{MuellerMalteseSketch}, \ili{Spanish} \citep{Machicao-y-Priemer2015a}, and \ili{Yiddish} \citep{MOe2011a}. Development of grammars in TRALE is supported by the \isi{Grammix} system \citep{MuellerGrammix}; \citet{MuellerCoreGram} provides a summary of this family of grammar implementations.

These grammars and systems have been used in a wide variety of applications, primarily as vehicles for research in computational linguistics, but also for some commercial software products.  Research applications already mentioned include \isi{database query} (HP Labs) and \isi{machine translation} (\verbmobil and LOGON), with additional applications developed for use in \isi{anthology search} \citep*{Sch:Kie:Spu:11}, \isi{grammar tutoring} in \ili{Norwegian} \citep*{Hellanetal:13}, \isi{ontology acquisition} \citep{Herb:Cope:06}, virtual \isi{robot control} \citep{packard2014uw}, visual \isi{question answering} \citep{DBLP:journals/corr/KuhnleC17}, and \isi{logic instruction} \citep{Flickinger:17}, among many others.  Commercial applications include \isi{e-commerce customer email response} (for YY Software), and \isi{grammar correction} in education (for \isi{Redbird Advanced Learning}, now part of \isi{McGraw-Hill Education}: \citealt*{Suppesetal:12}).  See \crossrefchaptert{cl} for further discussion.

For most practical applications, some approximate solution to the challenge of \isi{parse selection} (\isi{disambiguation}) must be provided, so developers of several of the DELPH-IN grammars, including the ERG, follow the approach of \citet*{OFTM2004a-u}, which uses a manually-annotated \isi{treebank} of sentences parsed by a grammar to train a \isi{statistical model} which is applied at run-time to identify the most likely analysis for each parsed sentence.  These treebanks can also serve as repositories of the analyses intended by the grammarian for the sentences of a corpus, and some resources, notably the \isi{Alpino Treebank} \citep*{BvNM2001a-u}, include analyses which the grammar may not yet be able to produce automatically.  

\section{Prospects}

As we noted early in this chapter, HPSG's origins are rooted in the desire simultaneously to address the theoretical concerns of linguists and the practical issues involved in building a useful natural language processing system.  In the decades since the birth of HPSG, the mainstream of work in both theoretical linguistics and NLP developed in ways that could not have been anticipated at the time.  NLP is now dominated by statistical methods, with almost all practical applications making use of machine learning technologies.  It is hard to see any influence of research by linguists in most NLP systems, though periodic workshops have helped to keep the conversation going.\footnote{For example, one on ``Building Linguistically Generalizable NLP Systems'' at the 2017 EMNLP conference in Copenhagen, and one on ``Relevance of Linguistic Structure in Neural NLP'' at the 2018 ACL conference in Melbourne.}  Mainstream grammatical theory, on the other hand, is now dominated by the \isi{Minimalist Program} (MP), which is too vaguely formulated for a rigorous comparison with HPSG.\footnote{Most work in MP is presented without precise definitions of the technical apparatus, but Edward Stabler and his collaborators have written a number of papers aimed at formalizing MP. See in particular \citet{CollStab2016}.\added{~\citet{Torr2019a-u} describes a large-scale implemented fragment in the framework of Minimalist Grammar. See \citew[177--180]{MuellerGT-Eng4} for a comparison of this fragment with HPSG. As Müller points out, many of the implementation techniques employed can be found in HPSG grammars, \eg discontinuous constituents and the \slasch-based approach to nonlocal dependencies.}}   Concern with computational implementation plays virtually no role in MP research; see \citet{MuellerGT-Eng1} for a discussion.

It might seem, therefore, that HPSG is further from the mainstream of both fields than it was at its inception, raising questions about how realistic the objectives of HPSG are.  We believe, however, that there are grounds for optimism.

With regard to implementations, there is no incompatibility between the use of HPSG and the machine learning methods of mainstream NLP.  Indeed, as noted above, HPSG-based systems that have been put to practical use have necessarily included components induced via statistical methods from annotated corpora.  Without such components, the systems cannot deal with the full variety of forms encountered in usage data.  On the other hand, existing NLP systems that rely solely on machine learning from corpora do not exhibit anything that can reasonably be called understanding of natural language.  Current technologies for machine translation, automatic summarization, and various other linguistic tasks fall far short of what humans do on these tasks, and are useful primarily as tools to speed up the tasks for the humans carrying them out.  Many NLP researchers are beginning to recognize that developing software that can plausibly be said to understand language will require representations of linguistic structure and meaning like those that are the stock in trade of linguists.  See \citet*{Bender2015LayersOI} for more discussion on sentence meaning.

Evidence for a renewed interest in linguistics among NLP researchers is the fact that major technology companies with natural language groups have recently begun (or in some cases, resumed) hiring linguists, and increasing numbers of new linguistics PhDs have taken jobs in the software industry.  

In the domain of theoretical linguistics, it is arguable that the distance between HPSG and the mainstream of grammatical research (that is, MP) has narrowed, given that both crucially incorporate ideas from Categorial Grammar (see \citealt{RetStab2004}, \citealt{BE95a}, and \citealt{MuellerUnifying} for comparisons between MP and CG\added{, for a general comparison of MP and HPSG see also \crossrefchapteralt{minimalism}}). Rather than trying to make that argument, however, we will point to connections that HPSG has made with other work in theoretical linguistics.  Perhaps the most obvious of these is the work of Peter Culicover and Ray Jackendoff on what they call {\em{\isi{Simpler Syntax}}}.  Their influential 2005 book with that title \citep{CJ2005a} argues for a theory of grammar that differs little in its architecture and motivations from HPSG. 

More interesting are the connections that have been forged between research in HPSG and work in \isi{Construction Grammar} (CxG).  \citet[\page 36]{Fillmore88a} characterizes the notion  of \emph{construction} as ``any syntactic pattern which is assigned one or more conventional functions in a language, together with whatever is linguistically conventionalized about its contribution to the meaning or use of structures containing it.''  Among the examples that construction grammarians have described at length are \emph{the Xer, the Yer} (as in \emph{the older I get, the longer I sleep}), \emph{X let alone Y} (as in \emph{I barely got up in time to eat lunch, let alone cook breakfast}), and \emph{What's X doing Y?} (as in \emph{What's this scratch doing in the table?}).  As noted above and in \crossrefchaptert[\page \pageref{page-HPSG-always-was-constructional}, \pageref{cxg:page-HPSG-inheritance}]{cxg}, HPSG has incorporated the notion of construction since at least the late 1990s.  

Nevertheless,\label{page-sbcg-start} work that labels itself CxG tends to look very different from HPSG.  This is in part because of the difference in their origins: many proponents of CxG come from the tradition of \isi{Cognitive Grammar} or typological studies, whereas HPSG's roots are in computational concerns. Hence, most of the CxG literature is not precise\is{formalization} enough to allow a straightforward comparison with HPSG, though the variants called Embodied Construction Grammar and Fluid Construction Grammar have more in common with HPSG; see \cites{MuellerFCG}[Sections~10.6.3--10.6.4]{MuellerGT-Eng4}\itd{Stefan: I added ref to GT book, since embodied CxG is only in there.} for a comparison.  In the last years of his life, Ivan Sag sought to unify CxG and HPSG through collaboration with construction grammarians from the University of California, Berkeley, particularly Charles Fillmore\aimention{Charles Fillmore}, Paul Kay\aimention{Paul Kay}, and Laura Michaelis\aimention{Laura Michaelis}.  They developed a theory called \emph{Sign-Based Construction Grammar} (SBCG), which would combine the insights of CxG with the explicitness of HPSG.  \citet[\page 70]{Sag2012a} wrote, ``To readers steeped in HPSG theory, SBCG will no doubt seem like a minor variant of constructional HPSG.''  Indeed, despite the name change, the main feature of SBCG that differs from HPSG is that it posits an inheritance hierarchy of constructs, which includes feature structure descriptions for such partially lexicalized multi-word expressions as \emph{Ved X's way PP}, instantiated in such VPs as \textit{ad-libbed his way through a largely secret meeting}.  While\label{evolution:page-sbcg-vs-lingo} this is a non-trivial extension to HPSG, there is no fundamental change to the technical machinery.  In fact, it has been a part of the LinGO implementation for many years.

That said, there is one important theoretical issue that divides HPSG and SBCG from much other work in CxG.  That issue is \isi{locality}.  To constrain the formal power of the theory, and to facilitate computational tractability, SBCG adopts what \citet[\page 150]{Sag2012a} calls ``Constructional Localism'' and describes it as follows:  ``Constructions license mother-daughter configurations without reference to embedding or embedded contexts.''  That is, like phrase structure rules, constructions must be characterized in terms of a mother node and its immediate daughters.  At first glance, this seems to rule out analyses of many of the examples of constructions provided in the CxG literature.  But  \citet[\page 150]{Sag2012a} goes on to say, ``Constructional Localism does not preclude an account of nonlocal dependencies in grammar, it simply requires that all such dependencies be locally encoded in signs in such a way that information about a distal element can be accessed locally at a higher level of structure.''

\citet[\page 35]{Fillmore88a} wrote:
\begin{quote}
Construction grammars differ from phrase-structure grammars which use \emph{complex symbols} and
allow the \emph{transmission of information} between lower and higher structural units, in that we
allow the direct representation of the required properties of subordinate constituents.  (Should it
turn out that there are completely general principles for predicting the kinds of information that
get transmitted upwards or downwards, this may not be a real difference.) \citep[\page 35]{Fillmore88a}
\end{quote}
SBCG is committed to the position alluded to in the parenthetical sentence in this quote, namely, that general principles of information transmission within sentences make it possible to insist on Constructional Localism.   See \crossrefchaptert{cxg} for a much more detailed discussion, and \citet{VanEynde2015a} for a review of the 2012 SBCG book.\label{page-sbcg-end}  

Finally, another point of convergence between work in HPSG and other work in both theoretical
linguistics and NLP is the increasing importance of corpus and experimental data. In the early years
of the HP NL project, the methodology was the same as that employed in almost all work in
theoretical syntax and semantics: the grammar was based entirely on examples invented by the
researchers.  At one point during the decade of the HP NL project, Flickinger, Pullum, and Wasow
compiled a list of sentences intended to exemplify many of the sentence types that they hoped the
system would eventually be able to analyze.  That list, 1328 sentences long, continues to be useful
as a test suite for the ERG, and is also used by various other NLP groups.  But it does not come
close to covering the variety of sentence forms that are found in corpora of speech and various
written genres.  As the goals of the HPSG implementations have broadened from database query to
dealing with ``language in the wild'', the use of corpora to test such systems and motivate
extensions to them has increased.  This parallels a development in other areas of linguistics, which
have also increasingly made use of large on-line corpora as sources of data and tests of their
theories.  This is a trend that we expect will continue.

Experimental data has been particularly important in the exploration of whether well-known
constraints on phenomena like extraction or ellipsis are really due to the grammar of natural
languages or the convergence of frequency, discourse factors, and aspects of human sentence
processing. \citet{HS2010a-u}, \citet{CD2019a-u}, and \citet{CP2020a-u}, for
example, have argued that many so-called island constraints are not grammatical in
nature. Similarly, \citet{SAHP2019a-u} claim that some parallelism
effects in \isi{Right-Node-Raising} are not grammatical in nature. Both lines of research lead to a
reduction of what grammars are responsible for and question the traditional division of labor
between the grammatical system, properties of the discourse within which utterances are embedded,
and processing considerations. We expect work along these lines to continue in the future (see also
\crossrefchapterw{processing} for the relation between HPSG and work in sentence processing).
 
In short, there are signs of convergence between work on HPSG and work in other areas, and it seems
plausible to think that the market for HPSG research will grow in the future.

\section*{Acknowledgements}

The work on this chapter by Flickinger was generously supported by a fellowship at the Oslo Center for Advanced Study at the \ili{Norwegian} Academy of Science and Letters.  The authors also thank several reviewers for their insightful and detailed comments on drafts of the chapter, including Emily \added{M.} Bender, Robert Borsley, Danièle Godard, Jean-Pierre Koenig and Stefan Müller.


{\sloppy
\printbibliography[heading=subbibliography,notkeyword=this]
}
\end{document}


%      <!-- Local IspellDict: en_US-w_accents -->
