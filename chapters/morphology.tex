\documentclass[output=paper
%	        ,collection
%	        ,collectionchapter
 	        ,biblatex
                ,babelshorthands
                ,newtxmath
                ,draftmode
                ,colorlinks, citecolor=brown
]{langscibook}

\IfFileExists{../localcommands.tex}{%hack to check whether this is being compiled as part of a collection or standalone
  \usepackage{../nomemoize}
  % add all extra packages you need to load to this file 

% the ISBN assigned to the digital edition
\usepackage[ISBN=9783961102556]{ean13isbn} 

\usepackage{graphicx}
\usepackage{tabularx}
\usepackage{amsmath} 

%\usepackage{tipa}      % Davis Koenig
\usepackage{xunicode} % Provide tipa macros (BC)

\usepackage{multicol}

% Berthold morphology
\usepackage{relsize}
%\usepackage{./styles/rtrees-bc} % forbidden forest 08.12.2019

% provides logo priniting commands
\usepackage{langsci-basic}

\usepackage{langsci-optional} 
% used to be in this package
\providecommand{\citegen}{}
\renewcommand{\citegen}[2][]{\citeauthor{#2}'s (\citeyear*[#1]{#2})}
\providecommand{\lsptoprule}{}
\renewcommand{\lsptoprule}{\midrule\toprule}
\providecommand{\lspbottomrule}{}
\renewcommand{\lspbottomrule}{\bottomrule\midrule}
\providecommand{\largerpage}{}
\renewcommand{\largerpage}[1][1]{\enlargethispage{#1\baselineskip}}

\usepackage{./styles/biblatex-series-number-checks}


\usepackage{langsci-lgr}

\newcommand{\MAS}{\textsc{m}\xspace} % \M is taken by somebody

%\usepackage{./styles/forest/forest}
\usepackage{langsci-forest-setup}

% is loaded in main etc.
% \usepackage{nomemoize} 
% \memoizeset{
%   memo filename prefix={chapters/hpsg-handbook.memo.dir/},
%   register=\todo{O{}+m},
%   prevent=\todo,
% }

\usepackage{tikz-cd}

\usepackage{./styles/tikz-grid}
\usetikzlibrary{shadows}


% removed with texlive 2020 06.05.2020
% %\usepackage{pgfplots} % for data/theory figure in minimalism.tex
% % fix some issue with Mod https://tex.stackexchange.com/a/330076
% \makeatletter
% \let\pgfmathModX=\pgfmathMod@
% \usepackage{pgfplots}%
% \let\pgfmathMod@=\pgfmathModX
% \makeatother

\usepackage{subcaption}

% Stefan Müller's styles
\usepackage{./styles/merkmalstruktur,./styles/makros.2020,./styles/my-xspace,./styles/article-ex,
./styles/eng-date}

\usepackage{varioref}
\newcommand\refORregion[2]{%
 \vrefpagenum\firstnum{#1}%
 \vrefpagenum\secondnum{#2}%
\ifthenelse{\equal\firstnum\secondnum}%
{\pageref{#1}}%
{\pageref{#1}--\pageref{#2}}%
}

% I am sick of fiddeling arround with babel. I want these shorthands also to work in commands I
% define. St.Mü. 13.08.2020
% e.g. with \iwithini
\usepackage{german}
\selectlanguage{USenglish}

\usepackage{./styles/abbrev}


% Has to be loaded late since otherwise footnotes will not work

%%%%%%%%%%%%%%%%%%%%%%%%%%%%%%%%%%%%%%%%%%%%%%%%%%%%
%%%                                              %%%
%%%           Examples                           %%%
%%%                                              %%%
%%%%%%%%%%%%%%%%%%%%%%%%%%%%%%%%%%%%%%%%%%%%%%%%%%%%
% remove the percentage signs in the following lines
% if your book makes use of linguistic examples
\usepackage{langsci-gb4e} 



% This introduces labels which makes hyperlinks work so that proofreading is easier.
%\makeatletter
%\newcommand{\mex}[1]{\ref{ex-\the\c@chapter-\the\numexpr\c@equation+#1}\relax}
%\newcommand{\eaautolabel}{\label{ex-\the\c@chapter-\the\numexpr\c@equation+1}}
%\makeatother

%\let\oldea\ea
%\def\ea{\oldea\eaautolabel}

%\let\oldeal\eal
%\def\eal{\oldeal\eaautolabel}


% Crossing out text
% uncomment when needed
%\usepackage{ulem}

\usepackage{./styles/additional-langsci-index-shortcuts}

% this is the completely redone avm package
\usepackage{langsci-avm}
\avmsetup{columnsep=.3ex,style=narrow}

\avmdefinecommand{phon}[phon]
  {
    attributes  = \itshape%,
%    delimfactor = 900,
%    delimfall   = 10pt
}

\avmdefinecommand{form}[form]
  {
    attributes  = \itshape%,
%    delimfactor = 900,
%    delimfall   = 10pt
}

% \set was already taken
\avmdefinecommand{avmset}[set]{ attributes=\itshape } % define a new \set command
\avmdefinecommand{list}[list]{ attributes=\itshape } % define a new \list command
   % Note: the label "list" will be output in whatever font is currently active.

% \avm{
% 	[subj  & \1 \\
% 	comps & \2 \- \list*(gap-ss) \\ % Produce a \list
% 	deps  & < \1 > \+ \2
% 	]
% }


\avmdefinecommand{nelist}[ne-list]{ attributes=\itshape } % define a new \nelist command
   % Note: the label "ne-list" will be output in whatever font is currently active.



% https://github.com/langsci/langsci-avm/issues/33#issuecomment-671201576
%\avmsetup{extraskip=0pt}

% if you have to use both langsci-avm and avm
% \usepackage{langsci-avm} % Load pkg with meaning A of conflicting cmd
% \let\lavm\avm % Send the conflicting command to an alternative
% \let\avm\undefined % Send the conflicting cmd to be \undefined
% \usepackage{avm} % Load pkg with meaning B for conf. cmd 

%\let\asort\type*

% remove this, once we really do without avm
%\usepackage{./styles/avm+}

% copied over from avm+.sty
% some relation operators:
%\newcommand{\append}[0]{\ensuremath{\oplus\hspace{.24em}}}
%\newcommand{\shuffle}[0]{\ensuremath{\bigcirc\hspace{.24em}}}

\newcommand{\append}[0]{\ensuremath{\oplus}\xspace}
\newcommand{\shuffle}[0]{\ensuremath{\bigcirc}\xspace}


% command to fontify relations in avms 
\newcommand{\rel}[1]{\texttt{#1}}
%\def\relfont{\slshape}%
%\def\relfont{\ttdefault}%


\let\idx\ibox
\let\avmbox\ibox

% command to fontify attributes in ordinary text
%\newcommand{\attrib}[1]{\textsc{#1}}


% some relation operators:
%\newcommand{\append}[0]{\ensuremath{\oplus\hspace{.24em}}}
%\newcommand{\shuffle}[0]{\ensuremath{\bigcirc\hspace{.24em}}}

\def\relfont{\slshape}%
%
% command to fontify relations in avms 
%\newcommand{\rel}[1]{{\relfont #1}}



% \renewcommand{\tpv}[1]{{\avmjvalfont\itshape #1}}

% % no small caps please
% \renewcommand{\phonshape}[0]{\normalfont\itshape}

% \regAvmFonts

\usepackage{theorem}

\newtheorem{mydefinition}{Def.}
\newtheorem{principle}{Principle}

{\theoremstyle{break}
%\newtheorem{schema}{Schema}
\newtheorem{mydefinition-break}[mydefinition]{Def.}
\newtheorem{principle-break}[principle]{Principle}
}


%% \newcommand{schema}[2]{
%% \begin{minipage}{\textwidth}
%% {\textbf{Schema~\theschema}}]\hspace{.5em}\textbf{(#1)}\\
%% #2
%% \end{minipage}}


% This avoids linebreaks in the Schema
\newcounter{schemacounter}
\makeatletter
\newenvironment{schema}[1][]
  {%
   \refstepcounter{schemacounter}%
   \par\bigskip\noindent
   \minipage{\linewidth}%
   \textbf{Schema~\theschemacounter\hspace{.5em} \ifx&#1&\else(#1)\fi}\par
  }{\endminipage\par\bigskip\@endparenv}%
\makeatother

%\usepackage{subfig}





% Davis Koenig Lexikon

\usepackage{tikz-qtree,tikz-qtree-compat} % Davis Koenig remove

\usepackage{shadow}



\usepackage[english]{isodate} % Andy Lücking
\usepackage[autostyle]{csquotes} % Andy
%\usepackage[autolanguage]{numprint}

%\defaultfontfeatures{
%    Path = /usr/local/texlive/2017/texmf-dist/fonts/opentype/public/fontawesome/ }

%% https://tex.stackexchange.com/a/316948/18561
%\defaultfontfeatures{Extension = .otf}% adds .otf to end of path when font loaded without ext parameter e.g. \newfontfamily{\FA}{FontAwesome} > \newfontfamily{\FA}{FontAwesome.otf}
%\usepackage{fontawesome} % Andy Lücking
\usepackage{pifont} % Andy Lücking -> hand

\usetikzlibrary{decorations.pathreplacing} % Andy Lücking
\usetikzlibrary{matrix} % Andy 
\usetikzlibrary{positioning} % Andy
\usepackage{tikz-3dplot} % Andy

% pragmatics
\usepackage{eqparbox} % Andy
\usepackage{enumitem} % Andy
\usepackage{longtable} % Andy
\usepackage{tabu} % Andy              needs to be loaded before hyperref as of texlive 2020

% tabu-fix
% to make "spread 0pt" work
% -----------------------------
\RequirePackage{etoolbox}
\makeatletter
\patchcmd
	\tabu@startpboxmeasure
	{\bgroup\begin{varwidth}}%
	{\bgroup
	 \iftabu@spread\color@begingroup\fi\begin{varwidth}}%
	{}{}
\def\@tabarray{\m@th\def\tabu@currentgrouptype
    {\currentgrouptype}\@ifnextchar[\@array{\@array[c]}}
%
%%% \pdfelapsedtime bug 2019-12-15
\patchcmd
	\tabu@message@etime
	{\the\pdfelapsedtime}%
	{\pdfelapsedtime}%
	{}{}
%
%
\makeatother
% -----------------------------


% Manfred's packages

%\usepackage{shadow}

\usepackage{tabularx}
\newcolumntype{L}[1]{>{\raggedright\arraybackslash}p{#1}} % linksbündig mit Breitenangabe


% Jong-Bok

%\usepackage{xytree}

\newcommand{\xytree}[2][dummy]{Let's do the tree!}

% seems evil, get rid of it
% defines \ex is incompatible with gb4e
%\usepackage{lingmacros}

% taken from lingmacros:
\makeatletter
% \evnup is used to line up the enumsentence number and an entry along
% the top.  It can take an argument to improve lining up.
\def\evnup{\@ifnextchar[{\@evnup}{\@evnup[0pt]}}

\def\@evnup[#1]#2{\setbox1=\hbox{#2}%
\dimen1=\ht1 \advance\dimen1 by -.5\baselineskip%
\advance\dimen1 by -#1%
\leavevmode\lower\dimen1\box1}
\makeatother


% YK -- CG chapter

%\usepackage{xspace}
\usepackage{bm}
\usepackage{ebproof}


% Antonio Branco, remove this
\usepackage{epsfig}

% now unicode
%\usepackage{alphabeta}





\usepackage{pst-node}


% fmr: additional packages
%\usepackage{amsthm}


% Ash and Steve: LFG
\usepackage{./styles/lfg/dalrymple}

\RequirePackage{graphics}
%\RequirePackage{./styles/lfg/trees}
%% \RequirePackage{avm}
%% \avmoptions{active}
%% \avmfont{\sc}
%% \avmvalfont{\sc}
\RequirePackage{./styles/lfg/lfgmacrosash}

\usepackage{./styles/lfg/glue}

%%%%%%%%%%%%%%%%%%%%%%%%%%%%%%
%% Markup
%%%%%%%%%%%%%%%%%%%%%%%%%%%%%%
\usepackage[normalem]{ulem} % For thinks like strikethrough, using \sout

% \newcommand{\high}[1]{\textbf{#1}} % highlighted text
\newcommand{\high}[1]{\textit{#1}} % highlighted text
%\newcommand{\term}[1]{\textit{#1}\/} % technical term
\newcommand{\qterm}[1]{``{#1}''} % technical term, quotes
%\newcommand{\trns}[1]{\strut `#1'} % translation in glossed example
\newcommand{\trnss}[1]{\strut \phantom{\sqz{}} `#1'} % translation in ungrammatical glossed example
\newcommand{\ttrns}[1]{(`#1')} % an in-text translation of a word
\newcommand{\LFGfeat}[1]{\mbox{\textsc{\MakeLowercase{#1}}}}     % feature name
%\newcommand{\val}[1]{\mbox{\textsc{\MakeLowercase{#1}}}}    % f-structure value
\newcommand{\featt}[1]{\mbox{\textsc{\MakeLowercase{#1}}}}     % feature name
\newcommand{\vall}[1]{\mbox{\textsc{\textup{\MakeLowercase{#1}}}}}    % f-structure value
\newcommand{\mg}[1]{\mbox{\textsc{\MakeLowercase{#1}}}}    % morphological gloss
%\newcommand{\word}[1]{\textit{#1}}       % mention of word
\providecommand{\kstar}[1]{{#1}\ensuremath{^*}}
\providecommand{\kplus}[1]{{#1}\ensuremath{^+}}
\newcommand{\template}[1]{@\textsc{\MakeLowercase{#1}}}
\newcommand{\templaten}[1]{\textsc{\MakeLowercase{#1}}}
\newcommand{\templatenn}[1]{\MakeUppercase{#1}}
\newcommand{\tempeq}{\ensuremath{=}}
\newcommand{\predval}[1]{\ensuremath{\langle}\textsc{#1}\ensuremath{\rangle}}
\newcommand{\predvall}[1]{{\rm `#1'}}
\newcommand{\lfgfst}[1]{\ensuremath{#1\,}}
\newcommand{\scare}[1]{``#1''} % scare quotes
\newcommand{\bracket}[1]{\ensuremath{\left\langle\mathit{#1}\right\rangle}}
\newcommand{\sectionw}[1][]{Section#1} % section word: for cap/non-cap
\newcommand{\tablew}[1][]{Table#1} % table word: for cap/non-cap
\newcommand{\lfgglue}{LFG+Glue}
\newcommand{\hpsgglue}{HPSG+Glue}
\newcommand{\gs}{GS}
%\newcommand{\func}[1]{\ensuremath{\mathbf{#1}}}
\newcommand{\func}[1]{\textbf{#1}}
\renewcommand{\glue}{Glue}
%\newcommand{\exr}[1]{(\ref{ex:#1}}
\newcommand{\exra}[1]{(\ref{ex:#1})}


%%%%%%%%%%%%%%%%%%%%%%%%%%%%%%
% Notation
%\newcommand{\xbar}[1]{$_{\mbox{\textsc{#1}$^{\raisebox{1ex}{}}$}}$}
\newcommand{\xprime}[2][]{\textup{\mbox{{#2}\ensuremath{^\prime_{\hspace*{-.0em}\mbox{\footnotesize\ensuremath{\mathit{#1}}}}}}}}
\providecommand{\xzero}[2][]{#2\ensuremath{^0_{\mbox{\footnotesize\ensuremath{\mathit{#1}}}}}}



\let\leftangle\langle
\let\rightangle\rangle

%\newcommand{\pslabel}[1]{}

% remove when finished
\usepackage{proofread}
  %add all your local new commands to this file

% The orchid-id is specified and then extracted by scripts for zenodo.
\newcommand{\orcid}[1]{} 

% do not show the chapter number. It is redundant, since most references to figures are within the
% same chapter.
\renewcommand{\thefigure}{\arabic{figure}}


% Don't do this at home. I do not like the smaller font for captions.
% I just removed loading the caption packege in langscibook.cls
%% \captionsetup{%
%% font={%
%% stretch=1%.8%
%% ,normalsize%,small%
%% },%
%% width=.8\textwidth
%% }

\makeatletter
\def\blx@maxline{77}
\makeatother


\let\citew\citet

\newcommand{\page}{}

\newcommand{\todostefan}[1]{\todo[color=orange!80]{\footnotesize #1}\xspace}
\newcommand{\todosatz}[1]{\todo[color=red!40]{\footnotesize #1}\xspace}

\newcommand{\inlinetodostefan}[1]{\todo[color=green!40,inline]{\footnotesize #1}\xspace}

\newcommand{\inlinetodoopt}[1]{\todo[color=green!40,inline]{\footnotesize #1}\xspace}
\newcommand{\inlinetodoobl}[1]{\todo[color=red!40,inline]{\footnotesize #1}\xspace}

\newcommand{\itd}[1]{\inlinetodoobl{#1}}
\newcommand{\itdobl}[1]{\inlinetodoobl{#1}}
\newcommand{\itdopt}[1]{\inlinetodoopt{#1}}

\newcommand{\itdsecond}[1]{}

\newcommand{\itddone}[1]{}
%\let\itddone\itdopt
\newcommand{\LATER}[1]{}



% A. Red: Simple typos, errors in the AVMs (only a couple) to take care of on the editorial side, no need to contact the authors
% B.: Green: Wording changes which do not necessarily require authors’ approval, but are not just typos/errors
% C.: Blue: Comments to the author that they don’t have to take care of, but after all, the authors might be interested to have the comments for future revisions. 
% D.: Purple: Comments to the editors about something we need to keep in mind or do. Nothing for you

\newcommand{\colorcodingexplanation}{\todo[color=green!40,inline]{%
Explanation of colors of bubbles and text:\\
A.: Red: Things that have to be fixed/commented upon.\\
B.: Green: optional comments\\
C.: Blue: Comments to the author that they don’t have to take care of, but after all, the authors
might be interested to have the comments for future revisions.\\
Explanation of colors of text:\\
Red: newly added material (crossreferences to other chapters and other references)\\
Orange: changed material, please check\\
Blue: suggestions for deletion\\
Please also check margin notes.
}}
% D.: Purple: Comments to the editors about something we need to keep in mind or do. Nothing for you


\newcommand{\itdgreen}[1]{\todo[color=green!40,inline]{\footnotesize #1}\xspace}
\newcommand{\itdblue}[1]{\todo[color=blue!40,inline]{\footnotesize #1}\xspace}

% for editing, remove later
\usepackage{xcolor}
\newcommand{\added}[1]{{\red #1}}
\newcommand{\addedthis}{\todostefan{added this}}

\newcommand{\changed}[1]{\textcolor{orange}{#1}}
\newcommand{\deleted}[1]{\textcolor{blue}{#1}}


% \newcommand{\addpages}{\todostefan{add pages}}
% %\newcommand{\iaddpages}{\inlinetodoobl{add pages}}
% \newcommand{\iaddpages}{\yel[add pages]{pages}\xspace}
% \newcommand{\addref}{\todostefan{add reference}}
% \newcommand{\inlineaddpages}{\inlinetodostefan{add pages}}
% \newcommand{\addglosses}{\todostefan{add glosses}}

\newcommand{\addpages}{\xspace}%np
\newcommand{\iaddpages}{\xspace}%islands und understudied languages
\newcommand{\addref}{\xspace}
\newcommand{\inlineaddpages}{\xspace}
% not used \newcommand{\addglosses}{}


%\newcommand{\spacebr}{\hphantom{[}}

\newcommand{\danishep}{\jambox{(\ili{Danish})}}
\newcommand{\english}{\jambox{(\ili{English})}}
\newcommand{\german}{\jambox{(\ili{German})}}
\newcommand{\yiddish}{\jambox{(\ili{Yiddish})}}
\newcommand{\welsh}{\jambox{(\ili{Welsh})}}

% Cite and cross-reference other chapters
\newcommand{\crossrefchaptert}[2][]{\citet*[#1]{chapters/#2}, Chapter~\ref{chap-#2} of this volume} 
\newcommand{\crossrefchapterp}[2][]{(\citealp*[#1]{chapters/#2}, Chapter~\ref{chap-#2} of this volume)}
\newcommand{\crossrefchapteralt}[2][]{\citealt*[#1]{chapters/#2}, Chapter~\ref{chap-#2} of this volume}
\newcommand{\crossrefchapteralp}[2][]{\citealp*[#1]{chapters/#2}, Chapter~\ref{chap-#2} of this volume}

\newcommand{\crossrefcitet}[2][]{\citet*[#1]{chapters/#2}} 
\newcommand{\crossrefcitep}[2][]{\citep*[#1]{chapters/#2}}
\newcommand{\crossrefcitealt}[2][]{\citealt*[#1]{chapters/#2}}
\newcommand{\crossrefcitealp}[2][]{\citealp*[#1]{chapters/#2}}


% example of optional argument:
% \crossrefchapterp[for something, see:]{name}
% gives: (for something, see: Author 2018, Chapter~X of this volume)



\let\crossrefchapterw\crossrefchaptert



% Davis Koenig

\let\ig=\textsc
\let\tc=\textcolor

% evolution, Flickinger, Pollard, Wasow

\let\citeNP\citet

% Adam P

%\newcommand{\toappear}{Forthcoming}
\newcommand{\pg}[1]{p.\,#1}
\renewcommand{\implies}{\rightarrow}

\newcommand*{\rref}[1]{(\ref{#1})}
\newcommand*{\aref}[1]{(\ref{#1}a)}
\newcommand*{\bref}[1]{(\ref{#1}b)}
\newcommand*{\cref}[1]{(\ref{#1}c)}

\newcommand{\msadam}{.}
\newcommand{\morsyn}[1]{\textsc{#1}}

\newcommand{\aux}{\textsc{aux}\xspace}

\newcommand{\nom}{\morsyn{nom}}
\newcommand{\acc}{\morsyn{acc}}
\newcommand{\dat}{\morsyn{dat}}
\newcommand{\gen}{\morsyn{gen}}
\newcommand{\ins}{\morsyn{ins}}
%\newcommand{\aploc}{\morsyn{loc}}
\newcommand{\voc}{\morsyn{voc}}
\newcommand{\ill}{\morsyn{ill}}
\renewcommand{\inf}{\morsyn{inf}}
\newcommand{\passprc}{\morsyn{passp}}

%\newcommand{\Nom}{\msadam\nom}
%\newcommand{\Acc}{\msadam\acc}
%\newcommand{\Dat}{\msadam\dat}
%\newcommand{\Gen}{\msadam\gen}
\newcommand{\Ins}{\msadam\ins}
\newcommand{\Loc}{\msadam\loc}
\newcommand{\Voc}{\msadam\voc}
\newcommand{\Ill}{\msadam\ill}
\newcommand{\PassP}{\msadam\passprc}

\newcommand{\Aux}{\textsc{aux}}

%\newcommand{\princ}[1]{\textnormal{\textsc{#1}}} % for constraint names
\newcommand{\princ}[1]{\textnormal{#1}} % for constraint names
\newcommand{\notion}[1]{\emph{#1}}
\renewcommand{\path}[1]{\textnormal{\textsc{#1}}}
\newcommand{\ftype}[1]{\textit{#1}}
\newcommand{\fftype}[1]{{\scriptsize\textit{#1}}}
\newcommand{\la}{$\langle$}
\newcommand{\ra}{$\rangle$}
%\newcommand{\argst}{\path{arg-st}}
\newcommand{\phtm}[1]{\setbox0=\hbox{#1}\hspace{\wd0}}
\newcommand{\prep}[1]{\setbox0=\hbox{#1}\hspace{-1\wd0}#1}


% Rui

\newcommand{\spc}[0]{\hspace{-1pt}\underline{\hspace{6pt}}\,}
\newcommand{\spcs}[0]{\hspace{-1pt}\underline{\hspace{6pt}}\,\,}
\newcommand{\bad}[1]{\leavevmode\llap{#1}}
\newcommand{\COMMENT}[1]{}


% Rui coordination
\newcommand{\subl}[1]{$_{\scriptstyle \textsc{#1}}$}



% Andy Lücking gesture.tex
\newcommand{\Pointing}{\ding{43}}
% Giotto: "Meeting of Joachim and Anne at the Golden Gate" - 1305-10 
\definecolor{GoldenGate1}{rgb}{.13,.09,.13} % Dress of woman in black
\definecolor{GoldenGate2}{rgb}{.94,.94,.91} % Bridge
\definecolor{GoldenGate3}{rgb}{.06,.09,.22} % Blue sky
\definecolor{GoldenGate4}{rgb}{.94,.91,.87} % Dress of woman with shawl
\definecolor{GoldenGate5}{rgb}{.52,.26,.26} % Joachim's robe
\definecolor{GoldenGate6}{rgb}{.65,.35,.16} % Anne's robe
\definecolor{GoldenGate7}{rgb}{.91,.84,.42} % Joachim's halo

\makeatletter
\newcommand{\@Depth}{1} % x-dimension, to front
\newcommand{\@Height}{1} % z-dimension, up
\newcommand{\@Width}{1} % y-dimension, rightwards
%\GGS{<x-start>}{<y-start>}{<z-top>}{<z-bottom>}{<Farbe>}{<x-width>}{<y-depth>}{<opacity>}
\newcommand{\GGS}[9][]{%
\coordinate (O) at (#2-1,#3-1,#5);
\coordinate (A) at (#2-1,#3-1+#7,#5);
\coordinate (B) at (#2-1,#3-1+#7,#4);
\coordinate (C) at (#2-1,#3-1,#4);
\coordinate (D) at (#2-1+#8,#3-1,#5);
\coordinate (E) at (#2-1+#8,#3-1+#7,#5);
\coordinate (F) at (#2-1+#8,#3-1+#7,#4);
\coordinate (G) at (#2-1+#8,#3-1,#4);
\draw[draw=black, fill=#6, fill opacity=#9] (D) -- (E) -- (F) -- (G) -- cycle;% Front
\draw[draw=black, fill=#6, fill opacity=#9] (C) -- (B) -- (F) -- (G) -- cycle;% Top
\draw[draw=black, fill=#6, fill opacity=#9] (A) -- (B) -- (F) -- (E) -- cycle;% Right
}
\makeatother


% pragmatics
\newcommand{\speaking}[1]{\eqparbox{name}{\textsc{\lowercase{#1}\space}}}
\newcommand{\alname}[1]{\eqparbox{name}{\textsc{\lowercase{#1}}}}
\newcommand{\HPSGTTR}{HPSG$_{\text{TTR}}$\xspace}

\newcommand{\ttrtype}[1]{\textit{#1}}
\newcommand{\avmel}{\q<\quad\q>} %% shortcut for empty lists in AVM
\newcommand{\ttrmerge}{\ensuremath{\wedge_{\textit{merge}}}}
\newcommand{\Cat}[2][0.1pt]{%
  \begin{scope}[y=#1,x=#1,yscale=-1, inner sep=0pt, outer sep=0pt]
   \path[fill=#2,line join=miter,line cap=butt,even odd rule,line width=0.8pt]
  (151.3490,307.2045) -- (264.3490,307.2045) .. controls (264.3490,291.1410) and (263.2021,287.9545) .. (236.5990,287.9545) .. controls (240.8490,275.2045) and (258.1242,244.3581) .. (267.7240,244.3581) .. controls (276.2171,244.3581) and (286.3490,244.8259) .. (286.3490,264.2045) .. controls (286.3490,286.2045) and (323.3717,321.6755) .. (332.3490,307.2045) .. controls (345.7277,285.6390) and (309.3490,292.2151) .. (309.3490,240.2046) .. controls (309.3490,169.0514) and (350.8742,179.1807) .. (350.8742,139.2046) .. controls (350.8742,119.2045) and (345.3490,116.5037) .. (345.3490,102.2045) .. controls (345.3490,83.3070) and (361.9972,84.4036) .. (358.7581,68.7349) .. controls (356.5206,57.9117) and (354.7696,49.2320) .. (353.4652,36.1439) .. controls (352.5396,26.8573) and (352.2445,16.9594) .. (342.5985,17.3574) .. controls (331.2650,17.8250) and (326.9655,37.7742) .. (309.3490,39.2045) .. controls (291.7685,40.6320) and (276.7783,24.2380) .. (269.9740,26.5795) .. controls (263.2271,28.9013) and (265.3490,47.2045) .. (269.3490,60.2045) .. controls (275.6359,80.6368) and (289.3490,107.2045) .. (264.3490,111.2045) .. controls (239.3490,115.2045) and (196.3490,119.2045) .. (165.3490,160.2046) .. controls (134.3490,201.2046) and (135.4934,249.3212) .. (123.3490,264.2045) .. controls (82.5907,314.1553) and (40.8239,293.6463) .. (40.8239,335.2045) .. controls (40.8239,353.8102) and (72.3490,367.2045) .. (77.3490,361.2045) .. controls (82.3490,355.2045) and (34.8638,337.3259) .. (87.9955,316.2045) .. controls (133.3871,298.1601) and   (137.4391,294.4766) .. (151.3490,307.2045) -- cycle;
\end{scope}%
}
%% leicht modifiziert nach Def. von Sebastian Nordhoff:
% \newcommand{\lueckingbox}[3]{\parbox[t][][t]{0.7cm}{\raggedright
%     \strut#1}\parbox[t][][t]{7.7cm}{\strut#2}\parbox[t][][t]{3cm}{\raggedright\strut#3}\bigskip\\}
\newcommand{\lueckingbox}[3]{\parbox[t][][t]{0.7cm}{\raggedright
    \strut\vspace*{-\baselineskip}\newline#1}\parbox[t][][t]{7.7cm}{\strut\vspace*{-\baselineskip}\newline#2}\parbox[t][][t]{3cm}{\raggedright\strut\vspace*{-\baselineskip}\newline#3}\bigskip\\}




% KdK
\newcommand{\smiley}{:)}

\renewbibmacro*{index:name}[5]{%
  \usebibmacro{index:entry}{#1}
    {\iffieldundef{usera}{}{\thefield{usera}\actualoperator}\mkbibindexname{#2}{#3}{#4}{#5}}}

% \newcommand{\noop}[1]{}

% chngcntr.sty otherwise gives error that these are already defined
%\let\counterwithin\relax
%\let\counterwithout\relax

% the space of a left bracket for glossings
\newcommand{\LB}{\hphantom{[}}

\newcommand{\LF}{\mbox{$[\![$}}

\newcommand{\RF}{\mbox{$]\!]_F$}}

\newcommand{\RT}{\mbox{$]\!]_T$}}





% Manfred's

\newcommand{\kommentar}[1]{}

\newcommand{\bsp}[1]{\emph{#1}}
\newcommand{\bspT}[2]{\bsp{#1} `#2'}
\newcommand{\bspTL}[3]{\bsp{#1} (lit.: #2) `#3'}

\newcommand{\noidi}{§}

\newcommand{\refer}[1]{(\ref{#1})}

%\newcommand{\avmtype}[1]{\multicolumn{2}{l}{\type{#1}}}
\newcommand{\attr}[1]{\textsc{#1}}

%\newcommand{\srdefault}{\mbox{\begin{tabular}{@{}c@{}}{\large <}\\[-1.5ex]$\sqcap$\end{tabular}}}
\newcommand{\srdefault}{$\stackrel{<}{\sqcap}$}


%% \newcommand{\myappcolumn}[2]{
%% \begin{minipage}[t]{#1}#2\end{minipage}
%% }

%% \newcommand{\appc}[1]{\myappcolumn{3.7cm}{#1}}


% Jong-Bok


% clean that up and do not use \def (killing other stuff defined before)
%\if 0
%\newcommand\DEL{\textsc{del}}
%\newcommand\del{\textsc{del}}

\newcommand\conn{\textsc{conn}}
\newcommand\CONN{\textsc{conn}}
\newcommand\CONJ{\textsc{conj}}
\newcommand\LITE{\textsc{lex}}
\newcommand\lite{\textsc{lex}}
\newcommand\HON{\textsc{hon}}

%\newcommand\CAUS{\textsc{caus}}
%\newcommand\PASS{\textsc{pass}}
\newcommand\NPST{\textsc{npst}}
%\newcommand\COND{\textsc{cond}}



\newcommand\hdlite{\textsc{head-lex construction}}
\newcommand\hdlight{\textsc{head-light} Schema}
\newcommand\NFORM{\textsc{nform}}

\newcommand\RELS{\textsc{rels}}
%\newcommand\TENSE{\textsc{tense}}


%\newcommand\ARG{\textsc{arg}}
\newcommand\ARGs{\textsc{arg0}}
\newcommand\ARGa{\textsc{arg}}
\newcommand\ARGb{\textsc{arg2}}
\newcommand\TPC{\textsc{top}}
%\newcommand\PROG{\textsc{prog}}

\newcommand\LIGHT{\textsc{light}\xspace}
\newcommand\pst{\textsc{pst}}
%\newcommand\PAST{\textsc{pst}}
%\newcommand\DAT{\textsc{dat}}
%\newcommand\CONJ{\textsc{conj}}
\newcommand\nominal{\textsc{nominal}}
\newcommand\NOMINAL{\textsc{nominal}}
\newcommand\VAL{\textsc{val}}
%\newcommand\val{\textsc{val}}
\newcommand\MODE{\textsc{mode}}
\newcommand\RESTR{\textsc{restr}}
\newcommand\SIT{\textsc{sit}}
\newcommand\ARG{\textsc{arg}}
\newcommand\RELN{\textsc{rel}}
%\newcommand\REL{\textsc{rel}}
%\newcommand\RELS{\textsc{rels}}
%\newcommand\arg-st{\textsc{arg-st}}
\newcommand\xdel{\textsc{xdel}}
\newcommand\zdel{\textsc{zdel}}
\newcommand\sug{\textsc{sug}}
%\newcommand\IMP{\textsc{imp}}
%\newcommand\conn{\textsc{conn}}
%\newcommand\CONJ{\textsc{conj}}
%\newcommand\HON{\textsc{hon}}
\newcommand\BN{\textsc{bn}}
\newcommand\bn{\textsc{bn}}
\newcommand\pres{\textsc{pres}}
\newcommand\PRES{\textsc{pres}}
\newcommand\prs{\textsc{pres}}
%\newcommand\PRS{\textsc{pres}}
\newcommand\agt{\textsc{agt}}
%\newcommand\DEL{\textsc{del}}
%\newcommand\PRED{\textsc{pred}}
\newcommand\AGENT{\textsc{agent}}
\newcommand\THEME{\textsc{theme}}
%\newcommand\AUX{\textsc{aux}}
%\newcommand\THEME{\textsc{theme}}
%\newcommand\PL{\textsc{pl}}
\newcommand\SRC{\textsc{src}}
\newcommand\src{\textsc{src}}
\newcommand{\FORMjb}{\textsc{form}}
\newcommand{\formjb}{\FORM}
\newcommand\GCASE{\textsc{gcase}}
\newcommand\gcase{\textsc{gcase}}
\newcommand\SCASE{\textsc{scase}}
\newcommand\PHON{\textsc{phon}}
%\newcommand\SS{\textsc{ss}}
\newcommand\SYN{\textsc{syn}}
%\newcommand\LOC{\textsc{loc}}
\newcommand\MOD{\textsc{mod}}
\newcommand\INV{\textsc{inv}}
%\newcommand\L{\textsc{l}}
%\newcommand\CASE{\textsc{case}}
\newcommand\SPR{\textsc{spr}}
\newcommand\COMPS{\textsc{comps}}
%\newcommand\comps{\textsc{comps}}
\newcommand\SEM{\textsc{sem}}
\newcommand\CONT{\textsc{cont}}
\newcommand\SUBCAT{\textsc{subcat}}
\newcommand\CAT{\textsc{cat}}
%\newcommand\C{\textsc{c}}
%\newcommand\SUBJ{\textsc{subj}}
\newcommand\subjjb{\textsc{subj}}
%\newcommand\SLASH{\textsc{slash}}
\newcommand\LOCAL{\textsc{local}}
%\newcommand\ARG-ST{\textsc{arg-st}}
%\newcommand\AGR{\textsc{agr}}
\newcommand\PER{\textsc{per}}
%\newcommand\NUM{\textsc{num}}
%\newcommand\IND{\textsc{ind}}
\newcommand\VFORM{\textsc{vform}}
\newcommand\PFORM{\textsc{pform}}
\newcommand\decl{\textsc{decl}}
%\newcommand\loc{\textsc{loc   }}
% \newcommand\   {\textsc{  }}

%\newcommand\NEG{\textsc{neg}}
\newcommand\FRAMES{\textsc{frames}}
%\newcommand\REFL{\textsc{refl}}

\newcommand\MKG{\textsc{mkg}}

%\newcommand\BN{\textsc{bn}}
\newcommand\HD{\textsc{hd}}
\newcommand\NP{\textsc{np}}
\newcommand\PF{\textsc{pf}}
%\newcommand\PL{\textsc{pl}}
\newcommand\PP{\textsc{pp}}
%\newcommand\SS{\textsc{ss}}
\newcommand\VF{\textsc{vf}}
\newcommand\VP{\textsc{vp}}
%\newcommand\bn{\textsc{bn}}
\newcommand\cl{\textsc{cl}}
%\newcommand\pl{\textsc{pl}}
\newcommand\Wh{\ital{Wh}}
%\newcommand\ng{\textsc{neg}}
\newcommand\wh{\ital{wh}}
%\newcommand\ACC{\textsc{acc}}
%\newcommand\AGR{\textsc{agr}}
\newcommand\AGT{\textsc{agt}}
\newcommand\ARC{\textsc{arc}}
%\newcommand\ARG{\textsc{arg}}
\newcommand\ARP{\textsc{arc}}
%\newcommand\AUX{\textsc{aux}}
%\newcommand\CAT{\textsc{cat}}
%\newcommand\COP{\textsc{cop}}
%\newcommand\DAT{\textsc{dat}}
\newcommand\NEWCOMMAND{\textsc{def}}
%\newcommand\DEL{\textsc{del}}
\newcommand\DOM{\textsc{dom}}
\newcommand\DTR{\textsc{dtr}}
%\newcommand\FUT{\textsc{fut}}
\newcommand\GAP{\textsc{gap}}
%\newcommand\GEN{\textsc{gen}}
%\newcommand\HON{\textsc{hon}}
%\newcommand\IMP{\textsc{imp}}
%\newcommand\IND{\textsc{ind}}
%\newcommand\INV{\textsc{inv}}
\newcommand\LEX{\textsc{lex}}
\newcommand\Lex{\textsc{lex}}
%\newcommand\LOC{\textsc{loc}}
%\newcommand\MOD{\textsc{mod}}
\newcommand\MRK{{\nr MRK}}
%\newcommand\NEG{\textsc{neg}}
\newcommand\NEW{\textsc{new}}
%\newcommand\NOM{\textsc{nom}}
%\newcommand\NUM{\textsc{num}}
%\newcommand\PER{\textsc{per}}
%\newcommand\PST{\textsc{pst}}
\newcommand\QUE{\textsc{que}}
%\newcommand\REL{\textsc{rel}}
\newcommand\SEL{\textsc{sel}}
%\newcommand\SEM{\textsc{sem}}
%\newcommand\SIT{\textsc{arg0}}
%\newcommand\SPR{\textsc{spr}}
%\newcommand\SRC{\textsc{src}}
\newcommand\SUG{\textsc{sug}}
%\newcommand\SYN{\textsc{syn}}
%\newcommand\TPC{\textsc{top}}
%\newcommand\VAL{\textsc{val}}
%\newcommand\acc{\textsc{acc}}
%\newcommand\agt{\textsc{agt}}
\newcommand\cop{\textsc{cop}}
%\newcommand\dat{\textsc{dat}}
\newcommand\foc{\textsc{focus}}
%\newcommand\FOC{\textsc{focus}}
\newcommand\fut{\textsc{fut}}
\newcommand\hon{\textsc{hon}}
\newcommand\imp{\textsc{imp}}
\newcommand\kes{\textsc{kes}}
%\newcommand\lex{\textsc{lex}}
%\newcommand\loc{\textsc{loc}}
\newcommand\mrk{{\nr MRK}}
%\newcommand\nom{\textsc{nom}}
%\newcommand\num{\textsc{num}}
\newcommand\plu{\textsc{plu}}
\newcommand\pne{\textsc{pne}}
%\newcommand\pst{\textsc{pst}}
\newcommand\pur{\textsc{pur}}
%\newcommand\que{\textsc{que}}
%\newcommand\src{\textsc{src}}
%\newcommand\sug{\textsc{sug}}
\newcommand\tpc{\textsc{top}}
%\newcommand\utt{\textsc{utt}}
%\newcommand\val{\textsc{val}}
%% \newcommand\LITE{\textsc{lex}}
%% \newcommand\PAST{\textsc{pst}}
%% \newcommand\POSP{\textsc{pos}}
%% \newcommand\PRS{\textsc{pres}}
%% \newcommand\mod{\textsc{mod}}%
%% \newcommand\newuse{{`kes'}}
%% \newcommand\posp{\textsc{pos}}
%% \newcommand\prs{\textsc{pres}}
%% \newcommand\psp{{\it en\/}}
%% \newcommand\skes{\textsc{kes}}
%% \newcommand\CASE{\textsc{case}}
%% \newcommand\CASE{\textsc{case}}
%% \newcommand\COMP{\textsc{comp}}
%% \newcommand\CONJ{\textsc{conj}}
%% \newcommand\CONN{\textsc{conn}}
%% \newcommand\CONT{\textsc{cont}}
%% \newcommand\DECL{\textsc{decl}}
%% \newcommand\FOCUS{\textsc{focus}}
%% %\newcommand\FORM{\textsc{form}} duplicate
%% \newcommand\FREL{\textsc{frel}}
%% \newcommand\GOAL{\textsc{goal}}
\newcommand\HEAD{\textsc{head}}
%% \newcommand\INDEX{\textsc{ind}}
%% \newcommand\INST{\textsc{inst}}
%% \newcommand\MODE{\textsc{mode}}
%% \newcommand\MOOD{\textsc{mood}}
%% \newcommand\NMLZ{\textsc{nmlz}}
%% \newcommand\PHON{\textsc{phon}}
%% \newcommand\PRED{\textsc{pred}}
%% %\newcommand\PRES{\textsc{pres}}
%% \newcommand\PROM{\textsc{prom}}
%% \newcommand\RELN{\textsc{pred}}
%% \newcommand\RELS{\textsc{rels}}
%% \newcommand\STEM{\textsc{stem}}
%% \newcommand\SUBJ{\textsc{subj}}
%% \newcommand\XARG{\textsc{xarg}}
%% \newcommand\bse{{\it bse\/}}
%% \newcommand\case{\textsc{case}}
%% \newcommand\caus{\textsc{caus}}
%% \newcommand\comp{\textsc{comp}}
%% \newcommand\conj{\textsc{conj}}
%% \newcommand\conn{\textsc{conn}}
%% \newcommand\decl{\textsc{decl}}
%% \newcommand\fin{{\it fin\/}}
%% %\newcommand\form{\textsc{form}}
%% \newcommand\gend{\textsc{gend}}
%% \newcommand\inf{{\it inf\/}}
%% \newcommand\mood{\textsc{mood}}
%% \newcommand\nmlz{\textsc{nmlz}}
%% \newcommand\pass{\textsc{pass}}
%% \newcommand\past{\textsc{past}}
%% \newcommand\perf{\textsc{perf}}
%% \newcommand\pln{{\it pln\/}}
%% \newcommand\pred{\textsc{pred}}


%% %\newcommand\pres{\textsc{pres}}
%% \newcommand\proc{\textsc{proc}}
%% \newcommand\nonfin{{\it nonfin\/}}
%% \newcommand\AGENT{\textsc{agent}}
%% \newcommand\CFORM{\textsc{cform}}
%% %\newcommand\COMPS{\textsc{comps}}
%% \newcommand\COORD{\textsc{coord}}
%% \newcommand\COUNT{\textsc{count}}
%% \newcommand\EXTRA{\textsc{extra}}
%% \newcommand\GCASE{\textsc{gcase}}
%% \newcommand\GIVEN{\textsc{given}}
%% \newcommand\LOCAL{\textsc{local}}
%% \newcommand\NFORM{\textsc{nform}}
%% \newcommand\PFORM{\textsc{pform}}
%% \newcommand\SCASE{\textsc{scase}}
%% \newcommand\SLASH{\textsc{slash}}
%% \newcommand\SLASH{\textsc{slash}}
%% \newcommand\THEME{\textsc{theme}}
%% \newcommand\TOPIC{\textsc{topic}}
%% \newcommand\VFORM{\textsc{vform}}
%% \newcommand\cause{\textsc{cause}}
%% %\newcommand\comps{\textsc{comps}}
%% \newcommand\gcase{\textsc{gcase}}
%% \newcommand\itkes{{\it kes\/}}
%% \newcommand\pass{{\it pass\/}}
%% \newcommand\vform{\textsc{vform}}
%% \newcommand\CCONT{\textsc{c-cont}}
%% \newcommand\GN{\textsc{given-new}}
%% \newcommand\INFO{\textsc{info-st}}
%% \newcommand\ARG-ST{\textsc{arg-st}}
%% \newcommand\SUBCAT{\textsc{subcat}}
%% \newcommand\SYNSEM{\textsc{synsem}}
%% \newcommand\VERBAL{\textsc{verbal}}
%% \newcommand\arg-st{\textsc{arg-st}}
%% \newcommand\plain{{\it plain}\/}
%% \newcommand\propos{\textsc{propos}}
%% \newcommand\ADVERBIAL{\textsc{advl}}
%% \newcommand\HIGHLIGHT{\textsc{prom}}
%% \newcommand\NOMINAL{\textsc{nominal}}

\newenvironment{myavm}{\begingroup\avmvskip{.1ex}
  \selectfont\begin{avm}}%
{\end{avm}\endgroup\medskip}
\newcommand\pfix{\vspace{-5pt}}


\newcommand{\jbsub}[1]{\lower4pt\hbox{\small #1}}
\newcommand{\jbssub}[1]{\lower4pt\hbox{\small #1}}
\newcommand\jbtr{\underbar{\ \ \ }\ }


%\fi

% cl

\newcommand{\delphin}{\textsc{delph-in}}


% YK -- CG chapter

\newcommand{\grey}[1]{\colorbox{mycolor}{#1}}
\definecolor{mycolor}{gray}{0.8}

\newcommand{\GQU}[2]{\raisebox{1.6ex}{\ensuremath{\rotatebox{180}{\textbf{#1}}_{\scalebox{.7}{\textbf{#2}}}}}}

\newcommand{\SetInfLen}{\setpremisesend{0pt}\setpremisesspace{10pt}\setnamespace{0pt}}

\newcommand{\pt}[1]{\ensuremath{\mathsf{#1}}}
\newcommand{\ptv}[1]{\ensuremath{\textsf{\textsl{#1}}}}

\newcommand{\sv}[1]{\ensuremath{\bm{\mathcal{#1}}}}
\newcommand{\sX}{\sv{X}}
\newcommand{\sF}{\sv{F}}
\newcommand{\sG}{\sv{G}}

\newcommand{\syncat}[1]{\textrm{#1}}
\newcommand{\syncatVar}[1]{\ensuremath{\mathit{#1}}}

\newcommand{\RuleName}[1]{\textrm{#1}}

\newcommand{\SemTyp}{\textsf{Sem}}

\newcommand{\E}{\ensuremath{\bm{\epsilon}}\xspace}

\newcommand{\greeka}{\upalpha}
\newcommand{\greekb}{\upbeta}
\newcommand{\greekd}{\updelta}
\newcommand{\greekp}{\upvarphi}
\newcommand{\greekr}{\uprho}
\newcommand{\greeks}{\upsigma}
\newcommand{\greekt}{\uptau}
\newcommand{\greeko}{\upomega}
\newcommand{\greekz}{\upzeta}

\newcommand{\Lemma}{\ensuremath{\hskip.5em\vdots\hskip.5em}\noLine}
\newcommand{\LemmaAlt}{\ensuremath{\hskip.5em\vdots\hskip.5em}}

\newcommand{\I}{\iota}

\newcommand{\sem}{\ensuremath}

\newcommand{\NoSem}{%
\renewcommand{\LexEnt}[3]{##1; \syncat{##3}}
\renewcommand{\LexEntTwoLine}[3]{\renewcommand{\arraystretch}{.8}%
\begin{array}[b]{l} ##1;  \\ \syncat{##3} \end{array}}
\renewcommand{\LexEntThreeLine}[3]{\renewcommand{\arraystretch}{.8}%
\begin{array}[b]{l} ##1; \\ \syncat{##3} \end{array}}}

\newcommand{\hypml}[2]{\left[\!\!#1\!\!\right]^{#2}}

%%%%for bussproof
\def\defaultHypSeparation{\hskip0.1in}
\def\ScoreOverhang{0pt}

\newcommand{\MultiLine}[1]{\renewcommand{\arraystretch}{.8}%
\ensuremath{\begin{array}[b]{l} #1 \end{array}}}

\newcommand{\MultiLineMod}[1]{%
\ensuremath{\begin{array}[t]{l} #1 \end{array}}}

\newcommand{\hypothesis}[2]{[ #1 ]^{#2}}

\newcommand{\LexEnt}[3]{#1; \ensuremath{#2}; \syncat{#3}}

\newcommand{\LexEntTwoLine}[3]{\renewcommand{\arraystretch}{.8}%
\begin{array}[b]{l} #1; \\ \ensuremath{#2};  \syncat{#3} \end{array}}

\newcommand{\LexEntThreeLine}[3]{\renewcommand{\arraystretch}{.8}%
\begin{array}[b]{l} #1; \\ \ensuremath{#2}; \\ \syncat{#3} \end{array}}

\newcommand{\LexEntFiveLine}[5]{\renewcommand{\arraystretch}{.8}%
\begin{array}{l} #1 \\ #2; \\ \ensuremath{#3} \\ \ensuremath{#4}; \\ \syncat{#5} \end{array}}

\newcommand{\LexEntFourLine}[4]{\renewcommand{\arraystretch}{.8}%
\begin{array}{l} \pt{#1} \\ \pt{#2}; \\ \syncat{#4} \end{array}}

\newcommand{\ManySomething}{\renewcommand{\arraystretch}{.8}%
\raisebox{-3mm}{\begin{array}[b]{c} \vdots \,\,\,\,\,\, \vdots \\
\vdots \end{array}}}

\newcommand{\lemma}[1]{\renewcommand{\arraystretch}{.8}%
\begin{array}[b]{c} \vdots \\ #1 \end{array}}

\newcommand{\lemmarev}[1]{\renewcommand{\arraystretch}{.8}%
\begin{array}[b]{c} #1 \\ \vdots \end{array}}

\newcommand{\p}{\ensuremath{\upvarphi}}

% clashes with soul package
\newcommand{\yusukest}{\textbf{\textsf{st}}}

\newcommand{\shortarrow}{\xspace\hskip-1.2ex\scalebox{.5}[1]{\ensuremath{\bm{\rightarrow}}}\hskip-.5ex\xspace}

\newcommand{\SemInt}[1]{\mbox{$[\![ \textrm{#1} ]\!]$}}

\newcommand{\HypSpace}{\hskip-.8ex}
\newcommand{\RaiseHeight}{\raisebox{2.2ex}}
\newcommand{\RaiseHeightLess}{\raisebox{1ex}}

\newcommand{\ThreeColHyp}[1]{\RaiseHeight{\Bigg[}\HypSpace#1\HypSpace\RaiseHeight{\Bigg]}}
\newcommand{\TwoColHyp}[1]{\RaiseHeightLess{\Big[}\HypSpace#1\HypSpace\RaiseHeightLess{\Big]}}

\newcommand{\LemmaShort}{\ensuremath{ \ \vdots} \ \noLine}
\newcommand{\LemmaShortAlt}{\ensuremath{ \ \vdots} \ }

\newcommand{\fail}{**}
\newcommand{\vs}{\raisebox{.05em}{\ensuremath{\upharpoonright}}}
\newcommand{\DerivSize}{\small}

% This is not needed, we just take unicode symbols
% The result of the code below came out wrong anyway.
% St. Mü. 10.06.2021
%
% \def\maru#1{{\ooalign{\hfil
%   \ifnum#1>999 \resizebox{.25\width}{\height}{#1}\else%
%   \ifnum#1>99 \resizebox{.33\width}{\height}{#1}\else%
%   \ifnum#1>9 \resizebox{.5\width}{\height}{#1}\else #1%
%   \fi\fi\fi%
% \/\hfil\crcr%
% \raise.167ex\hbox{\mathhexbox20D}}}}

\newenvironment{samepage2}%
 {\begin{flushleft}\begin{minipage}{\linewidth}}
 {\end{minipage}\end{flushleft}}

\newcommand{\cmt}[1]{\textsl{\textbf{[#1]}}}
\newcommand{\trns}[1]{\textbf{#1}\xspace}
\newcommand{\ptfont}{}
\newcommand{\gp}{\underline{\phantom{oo}}}
\newcommand{\mgcmt}{\marginnote}

\newcommand{\term}[1]{\emph{\isi{#1}}}

\newcommand{\citeposs}[1]{\citeauthor{#1}'s \citeyearpar{#1}}

% for standalone compilations Felix: This is in the class already
%\let\thetitle\@title
%\let\theauthor\@author 
\makeatletter
\newcommand{\togglepaper}[1][0]{ 
\bibliography{../Bibliographies/stmue,../localbibliography,
collection.bib}
  %% hyphenation points for line breaks
%% Normally, automatic hyphenation in LaTeX is very good
%% If a word is mis-hyphenated, add it to this file
%%
%% add information to TeX file before \begin{document} with:
%% %% hyphenation points for line breaks
%% Normally, automatic hyphenation in LaTeX is very good
%% If a word is mis-hyphenated, add it to this file
%%
%% add information to TeX file before \begin{document} with:
%% \include{localhyphenation}
\hyphenation{
A-la-hver-dzhie-va
ac-cu-sa-tive
anaph-o-ra
ana-phor
ana-phors
an-te-ced-ent
an-te-ced-ents
affri-ca-te
affri-ca-tes
ap-proach-es
Atha-bas-kan
Athe-nä-um
Be-schrei-bung
Bona-mi
Chi-che-ŵa
com-ple-ments
con-straints
Cope-sta-ke
Da-ge-stan
Dor-drecht
er-klä-ren-de
Flick-inger
Ginz-burg
Gro-ning-en
Has-pel-math
Jap-a-nese
Jon-a-than
Ka-tho-lie-ke
Ko-bon
krie-gen
Kroe-ger
Le-Sourd
moth-er
Mül-ler
Nie-mey-er
Ørs-nes
Par-a-digm
Prze-piór-kow-ski
phe-nom-e-non
re-nowned
Rie-he-mann
un-bound-ed
Ver-gleich
with-in
}

% listing within here does not have any effect for lfg.tex % 2020-05-14

% why has "erklärende" be listed here? I specified langid in bibtex item. Something is still not working with hyphenation.


% to do: check
%  Alahverdzhieva


% biblatex:

% This is a LaTeX frontend to TeX’s \hyphenation command which defines hy- phenation exceptions. The ⟨language⟩ must be a language name known to the babel/polyglossia packages. The ⟨text ⟩ is a whitespace-separated list of words. Hyphenation points are marked with a dash:

% \DefineHyphenationExceptions{american}{%
% hy-phen-ation ex-cep-tion }

\hyphenation{
A-la-hver-dzhie-va
ac-cu-sa-tive
anaph-o-ra
ana-phor
ana-phors
an-te-ced-ent
an-te-ced-ents
affri-ca-te
affri-ca-tes
ap-proach-es
Atha-bas-kan
Athe-nä-um
Be-schrei-bung
Bona-mi
Chi-che-ŵa
com-ple-ments
con-straints
Cope-sta-ke
Da-ge-stan
Dor-drecht
er-klä-ren-de
Flick-inger
Ginz-burg
Gro-ning-en
Has-pel-math
Jap-a-nese
Jon-a-than
Ka-tho-lie-ke
Ko-bon
krie-gen
Kroe-ger
Le-Sourd
moth-er
Mül-ler
Nie-mey-er
Ørs-nes
Par-a-digm
Prze-piór-kow-ski
phe-nom-e-non
re-nowned
Rie-he-mann
un-bound-ed
Ver-gleich
with-in
}

% listing within here does not have any effect for lfg.tex % 2020-05-14

% why has "erklärende" be listed here? I specified langid in bibtex item. Something is still not working with hyphenation.


% to do: check
%  Alahverdzhieva


% biblatex:

% This is a LaTeX frontend to TeX’s \hyphenation command which defines hy- phenation exceptions. The ⟨language⟩ must be a language name known to the babel/polyglossia packages. The ⟨text ⟩ is a whitespace-separated list of words. Hyphenation points are marked with a dash:

% \DefineHyphenationExceptions{american}{%
% hy-phen-ation ex-cep-tion }

  \memoizeset{
    memo filename prefix={hpsg-handbook.memo.dir/},
    % readonly
  }
  \papernote{\scriptsize\normalfont
    \@author.
    \titleTemp. 
    To appear in: 
    Stefan Müller, Anne Abeillé, Robert D. Borsley \& Jean-Pierre Koenig (eds.)
    HPSG Handbook
    Berlin: Language Science Press. [preliminary page numbering]
  }
  \pagenumbering{roman}
  \setcounter{chapter}{#1}
  \addtocounter{chapter}{-1}
}
\makeatother

\makeatletter
\newcommand{\togglepaperminimal}[1][0]{ 
  \bibliography{../Bibliographies/stmue,
                ../localbibliography,
collection.bib}
  %% hyphenation points for line breaks
%% Normally, automatic hyphenation in LaTeX is very good
%% If a word is mis-hyphenated, add it to this file
%%
%% add information to TeX file before \begin{document} with:
%% %% hyphenation points for line breaks
%% Normally, automatic hyphenation in LaTeX is very good
%% If a word is mis-hyphenated, add it to this file
%%
%% add information to TeX file before \begin{document} with:
%% \include{localhyphenation}
\hyphenation{
A-la-hver-dzhie-va
ac-cu-sa-tive
anaph-o-ra
ana-phor
ana-phors
an-te-ced-ent
an-te-ced-ents
affri-ca-te
affri-ca-tes
ap-proach-es
Atha-bas-kan
Athe-nä-um
Be-schrei-bung
Bona-mi
Chi-che-ŵa
com-ple-ments
con-straints
Cope-sta-ke
Da-ge-stan
Dor-drecht
er-klä-ren-de
Flick-inger
Ginz-burg
Gro-ning-en
Has-pel-math
Jap-a-nese
Jon-a-than
Ka-tho-lie-ke
Ko-bon
krie-gen
Kroe-ger
Le-Sourd
moth-er
Mül-ler
Nie-mey-er
Ørs-nes
Par-a-digm
Prze-piór-kow-ski
phe-nom-e-non
re-nowned
Rie-he-mann
un-bound-ed
Ver-gleich
with-in
}

% listing within here does not have any effect for lfg.tex % 2020-05-14

% why has "erklärende" be listed here? I specified langid in bibtex item. Something is still not working with hyphenation.


% to do: check
%  Alahverdzhieva


% biblatex:

% This is a LaTeX frontend to TeX’s \hyphenation command which defines hy- phenation exceptions. The ⟨language⟩ must be a language name known to the babel/polyglossia packages. The ⟨text ⟩ is a whitespace-separated list of words. Hyphenation points are marked with a dash:

% \DefineHyphenationExceptions{american}{%
% hy-phen-ation ex-cep-tion }

\hyphenation{
A-la-hver-dzhie-va
ac-cu-sa-tive
anaph-o-ra
ana-phor
ana-phors
an-te-ced-ent
an-te-ced-ents
affri-ca-te
affri-ca-tes
ap-proach-es
Atha-bas-kan
Athe-nä-um
Be-schrei-bung
Bona-mi
Chi-che-ŵa
com-ple-ments
con-straints
Cope-sta-ke
Da-ge-stan
Dor-drecht
er-klä-ren-de
Flick-inger
Ginz-burg
Gro-ning-en
Has-pel-math
Jap-a-nese
Jon-a-than
Ka-tho-lie-ke
Ko-bon
krie-gen
Kroe-ger
Le-Sourd
moth-er
Mül-ler
Nie-mey-er
Ørs-nes
Par-a-digm
Prze-piór-kow-ski
phe-nom-e-non
re-nowned
Rie-he-mann
un-bound-ed
Ver-gleich
with-in
}

% listing within here does not have any effect for lfg.tex % 2020-05-14

% why has "erklärende" be listed here? I specified langid in bibtex item. Something is still not working with hyphenation.


% to do: check
%  Alahverdzhieva


% biblatex:

% This is a LaTeX frontend to TeX’s \hyphenation command which defines hy- phenation exceptions. The ⟨language⟩ must be a language name known to the babel/polyglossia packages. The ⟨text ⟩ is a whitespace-separated list of words. Hyphenation points are marked with a dash:

% \DefineHyphenationExceptions{american}{%
% hy-phen-ation ex-cep-tion }

  \memoizeset{
    memo filename prefix={hpsg-handbook.memo.dir/},
    % readonly
  }
  \papernote{\scriptsize\normalfont
    \@author.
    \@title. 
    To appear in: 
    Stefan Müller, Anne Abeillé, Robert D. Borsley \& Jean-Pierre Koenig (eds.)
    HPSG Handbook
    Berlin: Language Science Press. [preliminary page numbering]
  }
  \pagenumbering{roman}
  \setcounter{chapter}{#1}
  \addtocounter{chapter}{-1}
}
\makeatother




% In case that year is not given, but pubstate. This mainly occurs for titles that are forthcoming, in press, etc.
\renewbibmacro*{addendum+pubstate}{% Thanks to https://tex.stackexchange.com/a/154367 for the idea
  \printfield{addendum}%
  \iffieldequalstr{labeldatesource}{pubstate}{}
  {\newunit\newblock\printfield{pubstate}}
}

\DeclareLabeldate{%
    \field{date}
    \field{year}
    \field{eventdate}
    \field{origdate}
    \field{urldate}
    \field{pubstate}
    \literal{nodate}
}

%\defbibheading{diachrony-sources}{\section*{Sources}} 

% if no langid is set, it is English:
% https://tex.stackexchange.com/a/279302
\DeclareSourcemap{
  \maps[datatype=bibtex]{
    \map{
      \step[fieldset=langid, fieldvalue={english}]
    }
  }
}


% for bibliographies
% biber/biblatex could use sortname field rather than messing around this way.
\newcommand{\SortNoop}[1]{}


% Doug Ball

\newcommand{\elist}{\q<\ \ \q>}

\newcommand{\esetDB}{\q\{\ \ \q\}}


\makeatletter

\newcommand{\nolistbreak}{%

  \let\oldpar\par\def\par{\oldpar\nobreak}% Any \par issues a \nobreak

  \@nobreaktrue% Don't break with first \item

}

\makeatother


% intermediate before Frank's trees are fixed
% This will be removed!!!!!
%\newcommand{\tree}[1]{} % ignore them blody trees
%\usepackage{tree-dvips}


\newcommand{\nodeconnect}[2]{}
\newcommand{\nodetriangle}[2]{}



% Doug relative clauses
%% I've compiled out almost all my private LaTeX command, but there are some
%% I found hard to get rid of. They are defined here.
%% There are few others which defined in places in the document where they have only
%% local effect (e.g. within figures); their names all end in DA, e.g. \MotherDA
%% There are a lot of \labels -- they are all of the form \label{sec:rc-...} or
%% \label{x:rc-...} or similar, so there should be no clashes.

% Subscripts -- scriptsize italic shape lowered by .25ex 
\newcommand{\subscr}[1]{\raisebox{-.5ex}{\protect{\scriptsize{\itshape #1\/}}}}
% A boxed subscript, for avm tags in normal text
\newcommand{\subtag}[1]{\subscr{\idx{#1}}}

%% Sets and tuples: I use \setof{} to get brackets that are upright, not slanted
%\newcommand{\setof}[1]{\ensuremath{\lbrace\,\mathit{#1}\,\rbrace}}
% 11.10.2019 EP: Doug requested replacement of existing \setof definition with the following:
%\newcommand{\setof}[1]{\begin{avm}\{\textcolor{red}{#1}\}\end{avm}}
% 31.1.2019 EP: Doug requested re-replacement of the above \textcolour version with the following:
\newcommand{\setof}[1]{\begin{avm}\{#1\}\end{avm}}

\newcommand{\tuple}[1]{\ensuremath{\left\langle\,\mbox{\textit{#1}}\,\right\rangle}}

% Single pile of stuff, optional arugment is psn (e.g. t or b)
% e.g. to put a over b over c in a centered column, top aligned, do:
%   \cPile[t]{a\\b\\c} 
\newcommand{\cPile}[2][]{%
  \begingroup%
  \renewcommand{\arraystretch}{.5}\begin{tabular}[#1]{@{}c@{}}#2\end{tabular}%
  \endgroup%
}

%% for linguistic examples in running text (`linguistic citation'):
\newcommand{\lic}[1]{\textit{#1}}

%% A gap marked by an underline, raised slightly
%% Default argument indicates how long the line should be:
\newcommand{\uGap}[1][3ex]{\raisebox{.25em}{\underline{\hspace{#1}}}\xspace}

%% \TnodeDA{XP}{avmcontents} -- in a Tree, put a node label next to an AVM
\newcommand{\TnodeDA}[2]{#1~\begin{avm}{#2}\end{avm}}

%% This allows tipa stuff to be put in \emph -- we need to change to cmr first.
%% It is used in the discussion of Arabic.
\newcommand{\emphtipa}[1]{{\fontfamily{cmr}\emph{\tipaencoding #1}}} 



 
 
\definecolor{lsDOIGray}{cmyk}{0,0,0,0.45}


% morphology.tex:
% Berthold

\newcommand{\dnode}[1]{\rnode{#1}{\fbox{#1}}}
\newcommand{\tnode}[1]{\rnode{#1}{\textit{#1}}}

\newcommand{\tl}[2]{#2}

\newcommand{\rrr}[3]{%
  \psframebox[linestyle=none]{%
    \avmoptions{center}
    \begin{avm}
      \[mud & \{ #1 \}\\
      ms & \{ #2 \}\\
      mph & \<  #3 \> \]
    \end{avm}
  }
}
\newcommand{\rr}[2]{%
  \psframebox[linestyle=none]{%
    \avmoptions{center}
    \begin{avm}
      \[mud & \{ #1 \}\\
      mph & \<  #2 \> \]
    \end{avm}
  }
}
 

% Frank Richter
\newtheorem{mydef}{Definition}

\long\def\set[#1\set=#2\set]%
{%
\left\{%
\tabcolsep 1pt%
\begin{tabular}{l}%
#1%
\end{tabular}%
\left|%
\tabcolsep 1pt%
\begin{tabular}{l}%
#2%
\end{tabular}%
\right.%
\right\}%
}

\newcommand{\einruck}{\\ \hspace*{1em}}


%\newcommand{\NatNum}{\mathrm{I\hspace{-.17em}N}}
\newcommand{\NatNum}{\mathbb{N}}
\newcommand{\Aug}[1]{\widehat{#1}}
%\newcommand{\its}{\mathrm{:}}
% Felix 14.02.2020
\DeclareMathOperator{\its}{:}

\newcommand{\sequence}[1]{\langle#1\rangle}

\newcommand{\INTERPRETATION}[2]{\sequence{#1\mathsf{U}#2,#1\mathsf{S}#2,#1\mathsf{A}#2,#1\mathsf{R}#2}}
\newcommand{\Interpretation}{\INTERPRETATION{}{}}

\newcommand{\Inte}{\mathsf{I}}
\newcommand{\Unive}{\mathsf{U}}
\newcommand{\Speci}{\mathsf{S}}
\newcommand{\Atti}{\mathsf{A}}
\newcommand{\Reli}{\mathsf{R}}
\newcommand{\ReliT}{\mathsf{RT}}

\newcommand{\VarInt}{\mathsf{G}}
\newcommand{\CInt}{\mathsf{C}}
\newcommand{\Tinte}{\mathsf{T}}
\newcommand{\Dinte}{\mathsf{D}}

% this was missing from ash's stuff.

%% \def \optrulenode#1{
%%   \setbox1\hbox{$\left(\hbox{\begin{tabular}{@{\strut}c@{\strut}}#1\end{tabular}}\right)$}
%%   \raisebox{1.9ex}{\raisebox{-\ht1}{\copy1}}}



\newcommand{\pslabel}[1]{}

\newcommand{\addpagesunless}{\todostefan{add pages unless you cite the
 work as such}}

% dg.tex
% framed boxes as used in dg.tex
% original idea from stackexchange, but modified by Saso
% http://tex.stackexchange.com/questions/230300/doing-something-like-psframebox-in-tikz#230306
\tikzset{
  frbox/.style={
    rounded corners,
    draw,
    thick,
    inner sep=5pt,
    anchor=base,
  },
}

% get rid of these morewrite messages:
% https://tex.stackexchange.com/questions/419489/suppressing-messages-to-standard-output-from-package-morewrites/419494#419494
\ExplSyntaxOn
\cs_set_protected:Npn \__morewrites_shipout_ii:
  {
    \__morewrites_before_shipout:
    \__morewrites_tex_shipout:w \tex_box:D \g__morewrites_shipout_box
    \edef\tmp{\interactionmode\the\interactionmode\space}\batchmode\__morewrites_after_shipout:\tmp
  }
\ExplSyntaxOff


% This is for places where authors used bold. I replace them by \emph
% but have the information where the bold was. St. Mü. 09.05.2020
\newcommand{\textbfemph}[1]{\emph{#1}}



% Felix 09.06.2020: copy code from the third line into localcommands.tex:
% https://github.com/langsci/langscibook#defined-environments-commands-etc
% Does not work with texlive 2020, is done with sed in Makefile
%\patchcmd{\mkbibindexname}{\ifdefvoid{#3}{}{\MakeCapital{#3} }}{\ifdefvoid{#3}{}{#3 }}{}{\AtEndDocument{\typeout{mkbibindexname could not be patched.}}}



\let\textnobf\textit
% instead of "in bold" write "in italics"
\newcommand{\bolddescriptionintext}{italics\xspace}

% Berthold
\newcommand{\mathplus}{+}
% \mbox{\normalfont +}}
\newcommand{\emdash}{--\xspace}
\newcommand{\emdashUS}{--\xspace}


% Stefan to get the space remvoed infront of the : in Bargmann NPN discussion
%\DeclareMathSymbol{:}{\mathord}{operators}{"3A}
% used {:\,} instead


% for cxg.tex needed for includonly to find the counter.
\newcounter{croftyears} 




% Needed for bibtex entry for Jackendoff's xbar syntax. Without it the bar would be off in itialics.

% https://tex.stackexchange.com/questions/95014/aligning-overline-to-italics-font/95079#95079
% \newbox\usefulbox

% \makeatletter
%     \def\getslant #1{\strip@pt\fontdimen1 #1}

%     \def\skoverline #1{\mathchoice
%      {{\setbox\usefulbox=\hbox{$\m@th\displaystyle #1$}%
%         \dimen@ \getslant\the\textfont\symletters \ht\usefulbox
%         \divide\dimen@ \tw@ 
%         \kern\dimen@ 
%         \overline{\kern-\dimen@ \box\usefulbox\kern\dimen@ }\kern-\dimen@ }}
%      {{\setbox\usefulbox=\hbox{$\m@th\textstyle #1$}%
%         \dimen@ \getslant\the\textfont\symletters \ht\usefulbox
%         \divide\dimen@ \tw@ 
%         \kern\dimen@ 
%         \overline{\kern-\dimen@ \box\usefulbox\kern\dimen@ }\kern-\dimen@ }}
%      {{\setbox\usefulbox=\hbox{$\m@th\scriptstyle #1$}%
%         \dimen@ \getslant\the\scriptfont\symletters \ht\usefulbox
%         \divide\dimen@ \tw@ 
%         \kern\dimen@ 
%         \overline{\kern-\dimen@ \box\usefulbox\kern\dimen@ }\kern-\dimen@ }}
%      {{\setbox\usefulbox=\hbox{$\m@th\scriptscriptstyle #1$}%
%         \dimen@ \getslant\the\scriptscriptfont\symletters \ht\usefulbox
%         \divide\dimen@ \tw@ 
%         \kern\dimen@ 
%         \overline{\kern-\dimen@ \box\usefulbox\kern\dimen@ }\kern-\dimen@ }}%
%      {}}
%     \makeatother




\newcommand{\acknowledgmentsEN}{Acknowledgements}
\newcommand{\acknowledgmentsUS}{Acknowledgments}

% to put two examples next to eachother
%\newcommand{\shortbox}[3][-.7]{
%    \parbox[t]{.4\textwidth}{
%      \vspace{#1\baselineskip} #2\strut~~ #3}%
%}

\newcommand{\twomulticolexamples}[2]{
\begin{tabular}[t]{@{}l@{~~}l@{\hspace{1em}}l@{~~}l@{}}
a. & \parbox[t]{.4\textwidth}{#1} & b. & \parbox[t]{.4\textwidth}{#2}\\
\end{tabular}
}




% This does a linebreak for \gll for long sentences leaving space for the language at the right
% margin.
% St.Mü. 17.06.2021
\newcommand{\longexampleandlanguage}[2]{%
\begin{tabularx}{\linewidth}[t]{@{}X@{}p{\widthof{(#2)}}@{}}%
\begin{minipage}[t]{\linewidth}%
#1%
\end{minipage} & (\ili{#2})%
\end{tabularx}}



\renewcommand{\indexccg}{\is{Categorial Grammar (CG)!Combinatorial \textasciitilde{} (CCG)}\xspace}
\newcommand{\indexccgstart}{\is{Categorial Grammar (CG)!Combinatorial \textasciitilde{} (CCG)|(}\xspace}
\newcommand{\indexccgend}{\is{Categorial Grammar (CG)!Combinatorial \textasciitilde{} (CCG)|)}\xspace}
\renewcommand{\indexmp}{\is{Minimalism}\xspace}


\newcommand{\gisu}{Giuseppe Varaschin\xspace}

\newcommand{\NPi}{NP$\mkern-1mu_i$\xspace}
\newcommand{\NPj}{NP$\mkern-1.5mu_j$\xspace}
  %% -*- coding:utf-8 -*-

%%%%%%%%%%%%%%%%%%%%%%%%%%%%%%%%%%%%%%%%%%%%%%%%%%%%%%%%%%%%
%
% gb4e

% fixes problem with to much vertical space between \zl and \eal due to the \nopagebreak
% command.
\makeatletter
\def\@exe[#1]{\ifnum \@xnumdepth >0%
                 \if@xrec\@exrecwarn\fi%
                 \if@noftnote\@exrecwarn\fi%
                 \@xnumdepth0\@listdepth0\@xrectrue%
                 \save@counters%
              \fi%
                 \advance\@xnumdepth \@ne \@@xsi%
                 \if@noftnote%
                        \begin{list}{(\thexnumi)}%
                        {\usecounter{xnumi}\@subex{#1}{\@gblabelsep}{0em}%
                        \setcounter{xnumi}{\value{equation}}}
% this is commented out here since it causes additional space between \zl and \eal 06.06.2020
%                        \nopagebreak}%
                 \else%
                        \begin{list}{(\roman{xnumi})}%
                        {\usecounter{xnumi}\@subex{(iiv)}{\@gblabelsep}{\footexindent}%
                        \setcounter{xnumi}{\value{fnx}}}%
                 \fi}
\makeatother

% the texlive 2020 langsci-gb4e adds a newline after \eas, the texlive 2017 version was OK.
% \makeatletter
% \def\eas{\ifnum\@xnumdepth=0\begin{exe}[(34)]\else\begin{xlist}[iv.]\fi\ex\begin{tabular}[t]{@{}p{.98\linewidth}@{}}}
% \makeatother



%%%%%%%%%%%%%%%%%%%%%%%%%%%%%%%%%%%%%%%%%%%%%%%%%%%%%%%%%%
%
% biblatex

% biblatex sets the option autolang=hyphens
%
% This disables language shorthands. To avoid this, the hyphens code can be redefined
%
% https://tex.stackexchange.com/a/548047/18561

\makeatletter
\def\hyphenrules#1{%
  \edef\bbl@tempf{#1}%
  \bbl@fixname\bbl@tempf
  \bbl@iflanguage\bbl@tempf{%
    \expandafter\bbl@patterns\expandafter{\bbl@tempf}%
    \expandafter\ifx\csname\bbl@tempf hyphenmins\endcsname\relax
      \set@hyphenmins\tw@\thr@@\relax
    \else
      \expandafter\expandafter\expandafter\set@hyphenmins
      \csname\bbl@tempf hyphenmins\endcsname\relax
    \fi}}
\makeatother


% the package defined \attop in a way that produced a box that has textwidth
%
\def\attop#1{\leavevmode\begin{minipage}[t]{.995\linewidth}\strut\vskip-\baselineskip\begin{minipage}[t]{.995\linewidth}#1\end{minipage}\end{minipage}}


%%%%%%%%%%%%%%%%%%%%%%%%%%%%%%%%%%%%%%%%%%%%%%%%%%%%%%%%%%%%%%%%%%%%


% Don't do this at home. I do not like the smaller font for captions.
% This does not work. Throw out package caption in langscibook
% \captionsetup{%
% font={%
% stretch=1%.8%
% ,normalsize%,small%
% },%
% width=\textwidth%.8\textwidth
% }
% \setcaphanging


  \togglepaper[22]
}{}



\author{Berthold Crysmann\affiliation{CNRS}
}

\title{Morphology} 
%\epigram{Change epigram in chapters/01.tex or remove it there }
\abstract{ This chapter provides an overview of work on morphology
  within HPSG. Following a brief discussion how morphology relates to
  the issue of lexical redundancy, and in particular horizontal
  redundancy, I map out the historical transition from meta-level
  lexical rules of derivational morphology and grammatical function
  change towards theories that are more tighly integrated with the
  hierarchical lexicon \citep{Riehemann98,Koenig99}. After a
  discussion of fundamental issues of inflectional morphology and the
  kind of models these favour, the chapter summarises previous HPSG
  approaches to the issue and finally provides an introduction to
  Information-based Morphology \citep{Crysmann:Bonami:2016}, a
  realisational model of morphology
  that systematically exploits HPSG-style underspecification in terms
  of multiple inheritance hierarchies. }





\begin{document}
\maketitle

\label{chap-morphology}



%\inlinetodostefan{15.05.2020: add page numbers}

\section{Introduction}
\label{morphology-sec:Intro}

Lexicalist approaches to grammar, such as HPSG, typically combine a
fairly general syntactic component with a rich and articulate
lexicon. While this makes for a highly principled syntactic
component \emdash e.g.\ the grammar fragment of \ili{English} presented in
\citet{Pollard94} contains only a handful of principles together with
six rather general phrase structure schemata \emdash, this decision places
quite a burden on the lexicon, an issue  known as lexical
redundancy.

Lexical redundancy comes in essentially two varieties: vertical redundancy
and horizontal redundancy. Vertical redundancy arises because many
lexical entries share a great number of syntactic and semantic
properties: e.g.\ in \ili{English} (and many other languages) there is a huge
class of strictly transitive verbs which display the same valency
specifications, the same semantic roles, and the same linking
patterns. From its outset, HPSG successfully eliminates vertical
redundancy by means of multiple inheritance networks over typed
feature structures \citep{Flickinger:Pollard:ea:85a}.

The problem of horizontal redundancy is associated with systematic
alternations in the lexicon: these include argument-structure
alternations, such as resultatives or the causative-inchoative
alternation, as well as classical instances of grammatical function
change, such as passives, applicatives or causatives. The crucial
difference with respect to vertical redundancy is that we are not
confronted with what is essentially a classificational problem \emdash
assigning lexical items to a more general class and inheriting its
properties \emdash, but rather with a relation between lexical items.
Morphological processes, both in word formation and inflection,
crucially involve this latter type of redundancy: for example, in the
case of deverbal adjectives in \textit{-able}, we find a substantial
number of derivations that show systematic changes in form, paired
with equally systematic changes in grammatical category, meaning, and
valency \citep{Riehemann98}. In inflection, change in morphosyntactic
properties, e.g.\ case or agreement marking, is often signalled by a
change in shape, which means the generalisation to be captured is about the
contrast of form and morphosyntactic properties between fully
inflected words.

Following \citet{Bresnan82}, the classical way to attack the issue of
horizontal redundancy in HPSG is by means of lexical rules
\citep{Flickinger87}. Early HPSG embraced Bresnan's original
conception of lexical rules as mappings between lexical items. To a
considerable extent\footnote{See also the work by
  \citet{Meurers02}, providing a formal
  description-level formalisation of lexical rules, as standardly used
  in HPSG.}, work on morphology and, in particular, derivational
morphology has led to a reconceptualisation of lexical rules within
HPSG: now, they are understood as partial descriptions of lexical
items that are fully integrated into the hierarchical lexicon
\citep{Koenig99}. As such, they are
amenable to the same underspecification techniques that are used to
generalise across classes of basic lexical items.

%\subsection*{Tour of the chapter}

%\bigskip\noindent Jib's nich.

\medskip

The chapter is structured as follows: in Section~\ref{sec:Deriv}, I
shall present the main developments towards an inheritance-based view
of derivational morphology within HPSG and provide pointers to
concrete work within HPSG and beyond that has grown out of these
efforts. In Section~\ref{sec:Infl}, I shall discuss inflectional
morphology, starting with an overview of the classical challenges
(Section \ref{sec:InflChallenges}) and assess how the different types
of inflectional theories \emdash Item-and-Arrangement (IA),
Item-and-Process (IP), and Word"=and"=Paradigm (WP) \emdash fare with
respect to these basic challenges
(Section~\ref{sec:InflTypology}). Against this backdrop, I shall
discuss previous work on inflection in HPSG
(Section~\ref{sec:InflHPSG}). Section~\ref{sec:IbM} will be devoted to
an introduction of Information-based Morphology, a recently developed
HPSG subtheory of inflectional morphology.



\section{Inheritance-based approaches to derivational morphology}
\label{sec:Deriv}\label{sec:derivational-morphology}

\subsection{\texorpdfstring{\citet{Krieger:Nerbonne:93}}{Krieger \& Nerbonne (1993)}}

Probably the first attempt at a more systematic treatment of
morphology is the approach by \citet{Krieger:Nerbonne:93}. They note
that meta-level lexical rules, as conceived of at the time, move the
description of lexical alternations, which are characteristic of
morphology, outside the scope of lexical inheritance
hierarchies. Consequently, they explore how morphology can be made
part of the lexicon. They observe that inflection and derivation
differ most crucially with respect to the finiteness of the domain:
while inflection is essentially finite (modulo case stacking;
\citealp{Sadler06,malouf:head-driven}), derivation need not be:
they cite repetitive prefixation in German as the decisive example
(\textit{Silbe} `syllable', \textit{Vor-silbe} `pre-syllable',
\textit{Vor-vor-silbe} `pre-pre-syllable', etc.). Consequently, they
propose  modelling derivation by means of morphological rule schemata,
which are underspecified descriptions of complex lexemes, and
integrating them as part of the lexical hierarchy. They adopt a
word-syntactic approach akin to \citet{Lieber92}, where affixes are
treated as signs that select the bases with which they combine. They
propose a number of principles that govern headedness,
subcategorisation, and semantic composition. What is special is that
all these principles are represented as types in the lexical type
hierarchy, cf.\ Chapter~\crossrefchaptert{lexicon}. Concrete
derivational rule schemata will then inherit from these
supertypes. What this amounts to is that different subclasses of
derivational processes may be subject to all or only a subset of these
principles. They briefly discuss conversion, i.e.\ zero derivation, and
suggest that this could be incorporated by means of unary rules.


% Recursion

\subsection{\texorpdfstring{\protect\citet{Riehemann98}}{Riehemann (1998)}}
\label{morphology:sec-Riehemann}

The work of \citet{Riehemann98} takes its starting point the
previous proposal laid out in \citet{Krieger:Nerbonne:93}, treating
derivational processes as partial descriptions of lexemes that are
organised in an inheritance type hierarchy and that relate a derived
lexeme to a morphological base.  Her approach, however, expands on the
previous proposal in two important respects. First, she argues against
a word-syntactic approach and suggests instead that only the
morphological base, a lexeme, should be considered a sign. Affixes or
modification of the base, if any, are syncategorematically introduced
by rule application. In contrast to the word-syntactic approach by
\citet{Krieger:Nerbonne:93}, Riehemann's conceptualisation of
derivation as unary rules integrated into the hierarchical lexicon
does not give any privileged status to concatenative word formation
processes: as a result, it generalises more easily to modificational
formations, conversion, and  (subtractive) back formations
(e.g.\ \textit{self-destruct} < \textit{self-destruction}). 

Second, she conducts a detailed empirical study of \textit{-bar}
`-able' affixation in \ili{German} and shows that besides regular
\textit{-bar} adjectives, which derive from transitive verbs and
introduce both modality and a  passivisation effect, there is a 
broader class of similar formations which adhere to some of the
properties, but not others.

\begin{figure}
  \centering
%  \includegraphics[scale=1.15]{figures/Riehemann-crop.pdf}
\oneline{
\begin{forest}
type hierarchy
[lexeme,s sep=3.5cm
	[structure,partition
		[complex
			[compound]
			[derived
				[compositionality,partition
					[compositional
						[syntype,partition
							[externalized,name=ex]
							[\ldots]
						]
						[semtype,partition
							[possibility,name=poss]
							[\ldots]
						]
					]
					[\ldots]
				]
				[dertype,partition
					[affixed,s sep=-.5
						[prefixed]
						[suffixed,s sep=1cm
							[\ldots]
							[bar-adj,name=bar
								[poss-bar-adj,name=possb
									[trans-bar-adj,name=trans
										[reg-bar-adj]
										[essbar]
										[\ldots]
									]
									[dative-bar-adj
										[untrennbar]
										[\ldots]
									]
									[prep-bar-adj
										[verfügbar]
										[\ldots]
									]
									[intr-bar-adj
										[brennbar]
										[\ldots]
									]
								]
								[fruchtbar]
								[\ldots]
							]
						]
						[\ldots]
					]
					[\ldots]
				]
			]
		]
		[simple]
	]
	[pos,partition
		[adjective,name=adj]
		[verb]
		[\ldots]
	]
]
{
\draw (ex.south) -- (trans.north);
\draw (poss.south) -- (possb.north);
\draw (adj.south) -- (bar.north);
}
\end{forest}
}  
  \caption{Type hierarchy of German
    \textit{-bar} derivation according to \citet[\page 64]{Riehemann98} }
  \label{fig:Riehemann}
\end{figure}


She concludes that multiple inheritance type hierarchies lend
themselves towards capturing the variety of the full empirical pattern
while at the same time providing the necessary abstraction in terms of
more general supertypes from which individual subclasses may inherit. 


% \newbox\poss
% \newbox\suffixed
% \newbox\externalised

% \setbox\poss=\hbox{
%   \avmoptions{active}
%   \begin{avm}
%     [\asort{possibility}
%     ss|l|cont|nuc|reln & $\diamond$]
%   \end{avm}
% }
% \setbox\suffixed=\hbox{
%   \avmoptions{active}
%   \begin{avm}
%     [\asort{suffixed}
%     ph & @1 $\oplus$ \normalfont\textit{list}\\
%     morph-b & <[ph & @1]>]
%   \end{avm}
% }
% \setbox\externalised=\hbox{
%   \avmoptions{active}
%   \begin{avm}
%     [\asort{externalised}
%     ss & [l|cat|val|subj &  <NP:@1>]\\
%     morph-b  & <[ss|l|cat|val|comps & <NP\[\type{acc}\]:@1,...>]>]
%   \end{avm}
% }

% \begin{exe}
%   \ex     \avmoptions{active}
%   \begin{avm}
%     [\asort{reg-bar-adj}\\
%     ph & @1 \mathplus \textit{bar}\\
%     morph-b & <[\asort{trans-verb}
%     ph & @1\\
%       ss|l & [cat|val|comps  <NP\[\type{acc}\]$_{@2}$> ~$\oplus$ @3]\\
%       cont|nuc & @4[act & \textit{nom-obj}\\
%       und & @2]]>\\
%     ss|loc & [cat & [head & \textit{adj}\\
%     val & [subj & <\normalfont NP$_{@2}$>\\
%     comps & @3]]\\
%     cont|nuc & [reln & $\diamond$\\
%     und & @2\\
%     soa-arg & @4]]]
%   \end{avm} \label{fig:riehemannBar}
% \end{exe}
\ea
% todo avm it only works with &
\avm[align=false,vectorsep=0ex]{
	[\type*{reg-bar-adj}
	phon    & \1 \+ \phonliste{ bar } \\
	morph-b &	<[\type*{trans-verb}
			  phon \,\1 \\
			  synsem|loc  [cat|comps & < NP ![\type{acc}]!:\ibox{2} > \+ \3 \\
			               cont|nuc  & \4 [act & index \\
			                               und & \5] ] ] > \\
	synsem|loc &	[cat	[head  & adj \\
				 subj  & < NP:\2$_{\5}$ > \\
				 comps & \3 ] \\
			 cont|nuc  [reln    & $\diamond$ \\
				    und     & \5 \\
				    soa-arg & \4 ] ] ]
} \label{fig:riehemannBar}
\z


Figure~\ref{fig:Riehemann} provides the extended hierarchy suggested
by \citet{Riehemann98}. The type for regular \textit{-bar} adjectives
given in (\ref{fig:riehemannBar}) is treated as a specific subtype
that inherits inter alia from more general supertypes that capture the
salient properties that characterise the regular formation,
e.g.\ \textit{anfechtbar} `contestable', but which also hold to some
extent for subregular \textit{-bar} adjectives, e.g.\ \textit{eßbar}
`edible'.\footnote{%
  The feature geometry and some further details have been adapted to
  the conventions used in this
  book. \added{For a version of Riehemann's lexical rule using the distinction between structural and
  lexical case \crossrefchapterp{case} see \citew{Mueller2003a}. Müller also deals with apparent
  bracketing-paradoxes in the analysis of particle verbs like \textit{anfechtbar} `contestable'.}
% \citet[\page 68]{Riehemann98} shares the whole semantic content between accusative object and
% subject. This is necessary since without such a sharing the semantic content of the accusative
% object could vary without any constraints. However, the value of the feature of \textsc{und} is a
% referential index not a complete semantic contribution of a noun phrase. Hence, the index \iboxb{5}
% rather than the whole content is shared between the subject of the adjective and the \textsc{und}
% feature. Since the content of the accusative object of the verb and the subject of the adjective are
% identical \iboxb{2}, the referential index of the adjective subject is also identical with the one of the
% accusative object of the verb.
}

  
One property that is almost trivial concerns suffixation of
\textit{-bar}, and it holds for the entire class. Suffixation is no
exclusive property of \textit{-bar} adjectives, so this property can
be abstracted out into the supertype \textit{suffixed} in
(\ref{fig:riehemannSuff}): the type \textit{bar-adj} in
Figure~\ref{fig:Riehemann} inherits this property and specifies the
concrete shape of the list appended to the morphological base.

\ea
	\avm{
		[\type*{suffixed}
		phon & \1 \+ list \\
		morph-b & <[phon & \1]> ]
	}\label{fig:riehemannSuff}
\z


\begin{sloppypar}
  A property which is common to most \textit{-bar} adjectives in
  \ili{German} is that they denote ``possibility'', as represented by the
  type constraint in (\ref{ex:possibility}). Exceptions include
  \textit{zahlbar} `payable', which denotes necessity instead. 
\end{sloppypar}

\begin{exe}

  \ex \label{ex:possibility}
  	\avm{
		[\type*{possibility}
  		synsem|loc|cont|nuc|reln & $\diamond$]
  	}
\end{exe}

Clearly more specific is the passivisation effect observed with
transitive bases. Clearly this does not apply in the same way to
verbal bases taking dative (\textit{entrinnbar} `escapable') or
prepositional complements (\textit{verfügbar} `available/""disposable')
instead of an accusative, and it does not apply at all to intransitive
bases (\textit{brennbar} `combustible').  

\begin{exe}
  \ex
  	\avm{
	  	[\type*{externalised}
	  	synsem & [loc|cat|subj &  < NP:\1 >]\\
	  	morph-b  & <[synsem|loc|cat|comps & < NP![\type{acc}]!:\1 , \ldots> ]> ]
	  }
\end{exe}

Regular \textit{-bar} adjectives (\ref{fig:riehemannBar}) inherit from all these supertypes, which
accounts for most of their properties, while at the same time the overall hierarchy of \textit{-bar} constructions
captures the relatedness of regular \textit{-bar} adjective to subregular formations. 


One aspect that Riehemann's approach does not capture as part of the
grammar is the productivity of the regular
pattern.
\itdobl{She does. This is what (\ref{fig:riehemannBar}) is about.
} \citet[\page 71]{Riehemann98} suggests that this could be accounted for
by extra-grammatical properties, such as lexical frequency. See also
Chapter~\crossrefchaptert{cxg} for further discussion.

\subsection{\texorpdfstring{\citet{Koenig99}}{Koenig (1999)}}

Koenig's work on lexical relations has made several important
contributions to our understanding of morphological processes within
the HPSG lexicon. Based on joint work with Dan Jurafsky
\citep{Koenig94}, he uses Online Type Construction to turn the
hierarchical lexicon, which is actually a static system into a
dynamic, generative device. This enables him in particular to make a
systematic distinction between open types for regular, productive
formations, and closed types for subregular and irregular ones.

\citet{Koenig99} takes issue with the early conception of lexical
rules as meta-level rules either deriving an expanded lexicon from a
base lexicon (generative lexical rules), or else establishing
relations between items within the lexicon (redundancy rules). He
argues on the basis of grammatical function change, such as the
\ili{English} passive, that systematic alternations are amenable to
underspecification in the hierarchical lexicon,  once
cross-classification between types can be performed dynamically.

Online Type Construction depends on a hierarchical lexicon that is
organised into an AND/OR network of conjunctive dimensions
(represented in boxed capitals) and disjunctive types (in
italics). While in a standard type hierarchy any two types that do not
have a common subtype are understood as incompatible, Online Type
Construction derives new subtypes by intersection of leaf types from
different dimensions. Leaf types within the same dimension are still
considered disjoint. Thus, dimensions define the range of inferrable
cross-classifications between types, without having to statically list
these types in the first place.

In Koenig's conception of the lexicon as a type underspecified
hierarchical lexicon (TUHL), the unexpanded lexicon is just a system of
types. Concrete lexical items, i.e.\ instances, are inferred from these
by means of Online Type Construction. 

Let us briefly consider a simple example for the active/passive
alternation: the minimal lexical type hierarchy in
Figure~\ref{fig:KoenigDyn} is organised into two
dimensions, one representing specific lexemes, the other specifying
active voice and passive voice linking patterns for lexemes. Concrete
lexical items are now derived by cross-classifying exactly one leaf
type from one dimension with exactly one leaf type from the other.

\begin{figure}
  \centering

  \footnotesize
%  \avmoptions{bottom}
%  \psset{angleA=-90,angleB=90,arm=0pt}
%
%  \resizebox{\textwidth}{!}{%
%    \tabcolsep.2\tabcolsep
%  \begin{tabular}{ccccccc}
%    \multicolumn{4}{c}{\tnode{lexeme}} \\[2ex]
%    \multicolumn{2}{c}{\dnode{ROOT}}  & \multicolumn{2}{c}{\dnode{VALENCE}}\\[2ex]
%    \multicolumn{2}{c}{\rnode{verbs} {
%    \begin{avm}
%      \[\asort{verbs}
%      cat & V\]
%    \end{avm}
%}}\\     \rnode{kill}{
%    \begin{avm}
%      \[\asort{kill-lxm} ph & kill\\
%      cont & \[reln & kill-rel\\
%      act & index\\
%      und & index\]\] 
%    \end{avm}
%}  & \rnode{resurrect}{\begin{avm}
%      \[\asort{resurrect-lxm} ph & resurrect\\
%      cont & \[reln & resurrect-rel\\
%      act & index\\
%      und & index\]\] 
%    \end{avm}
%}  & \rnode{trans}{\begin{avm}
%      \[\asort{trans-lxm}
%      cat & \[hd & v\\
%        subj & \< NP$_{\@x}$ \>\\
%      comps & \< NP$_{\@y}$ \>\]\\
%      cont & \[act & \@x\\
%      und & \@y\]
%      \]
%    \end{avm}
%            }
%        & 
%       \rnode{pass}{\begin{avm}
%        \[\asort{pass-lxm}
%          cat & \[hd & v\\
%            subj & \< NP$_{\@y}$ \>\\
%            comps & \< PP$_{\@x}$ \>\]\\
%          cont & \[act & \@x\\
%              und & \@y\]
%          \]
%         \end{avm}}
%    \\[6em] 
%             
%             
%    \rnode{kill-act}{\colorbox[gray]{1}{\textit{kill-lxm} $\wedge$ \textit{trans-lxm}}} &
%    \rnode{kill-pass}{\colorbox[gray]{1}{\textit{kill-lxm} $\wedge$ \textit{pass-lxm}}} &
%    \rnode{resurrect-act}{\colorbox[gray]{1}{\textit{resurrect-lxm} $\wedge$ \textit{trans-lxm}}} &
%    \rnode{resurrect-pass}{\colorbox[gray]{1}{\textit{resurrect-lxm} $\wedge$ \textit{pass-lxm}}}
%    
%        \psset{linewidth=.6pt,nodesep=2pt}                                                                                            
%    \ncdiag{lexeme}{ROOT}
%    \ncdiag{lexeme}{VALENCE}
%    \ncdiag{ROOT}{verbs}
%    \ncdiag{VALENCE}{trans}
%    \ncdiag{VALENCE}{pass}
%
%    \ncdiag[linestyle=dashed]{kill}{kill-act}
%    \ncdiag[linestyle=dashed]{resurrect}{resurrect-act}
%    \ncdiag[linestyle=dashed]{kill}{kill-pass}
%    \ncdiag[linestyle=dashed]{resurrect}{resurrect-pass}
%    \ncdiag[linestyle=dashed]{trans}{kill-act}
%    \ncdiag[linestyle=dashed]{trans}{resurrect-act}
%    \ncdiag[linestyle=dashed]{pass}{kill-pass}
%    \ncdiag[linestyle=dashed]{pass}{resurrect-pass}
%%    \ncdiag{pass}{rumored}
%
%    \ncdiag{verbs}{kill}
%    \ncdiag{verbs}{resurrect}
%    % \ncdiag{verbs}{hate}
% %   \ncdiag{verbs}{rumored}
%
%    
%    
%  \end{tabular}
% } 

\resizebox{\textwidth}{!}{%
	\begin{forest}
		type hierarchy
[lexeme
	[root,partition
		[%
		\avm{
			[\type*{verbs}
			cat & [hd & verb]]
		}
			[%
			\avm{
				[\type*{kill-lxm}
				ph & kill \\
				cont &	[\type*{kill-rel} \\
				act & index \\
				und & index] ]
			}, l sep+=1cm, name=a
				[kill-lxm $\land$ trans-lxm, tier=word, edge=dashed, name=1]
			]
			[%
			\avm{
				[\type*{resurrect-lxm}
				ph & resurrect \\
				cont &	[\type*{resurrect-rel} \\
				act & index \\
				und & index] ]
			}, l sep+=1cm, name=b
				[kill-lxm $\land$ pass-lxm, tier=word, no edge, name=2]
			]
		]
	]
	[valence,partition, l sep+=1cm
		[%
		\avm{
			[\type*{trans-lxm}
			cat &	[hd & verb \\
			subj & <NP$_{\tag{1}}$> \\
			comps & <NP$_{\tag{2}}$> ] \\
			cont &	[act & \tag{1} \\
			und & \tag{2} ] ]
		}, l sep+=1cm, name=c
			[resurrect-lxm $\land$ trans-lxm, tier=word, edge=dashed, name=3]
		]
		[%
		\avm{
			[\type*{pass-lxm}
			cat &	[hd & verb \\
			subj & <NP$_{\tag{2}}$> \\
			comps & <PP$_{\tag{1}}$> ] \\
			cont &	[act & \tag{1} \\
			und & \tag{2} ] ]
		}, l sep+=1cm, name=d
			[resurrect-lxm $\land$ pass-lxm, tier=word, edge=dashed, name=4]
		]
	]
]
{
\draw[dashed](a.south) to (2.north);
\draw[dashed](b.south) to (3.north);
\draw[dashed](b.south) to (4.north);
\draw[dashed](c.south) to (1.north);
\draw[dashed](d.south) to (2.north);
}
	\end{forest}
}

  \caption{Online type construction}
  \label{fig:KoenigDyn}
\end{figure}

An important aspect of this integration of alternations into the
hierarchical lexicon is that it becomes quite straightforward to deal
with lexical exceptions in a systematic way. The key to this is
pre-typing, as illustrated in Figure~\ref{fig:KoenigPre}: in \ili{English}, for instance, some transitive verbs, like
possessive \textit{have} fail to undergo passivisation. Rather than
marking these verbs diacritically with exception features, pre-typing to the
active pattern precludes their cross-classification with the passive
pattern, because leaf types within a dimension are disjoint and
pre-typing makes this type already a type in both dimensions.

\begin{figure}
  \centering



\resizebox{\textwidth}{!}{%
	\begin{forest}
		type hierarchy
[lexeme
	[root,partition, l sep+=1cm
		[%
		\avm{
			[\type*{verbs}
			cat & [hd & verb]]
		}, l sep+=2cm
			[%
			\avm{
                          [\type*{own-lxm}
                          ph & own \\
				cont &	[\type*{own-rel}
						act & index\\
						und & index] ]
			}, tier=word
			]
			[%
			\avm{
                          [\type*{have-lxm}
                          ph & have\\
				cont &	[\type*{have-rel}
						act & index\\
						und & index] ]
			}, tier=word, name=have
			]
		]
	]
	[valence,partition
		[%
		\avm{
			[\type*{trans}
			cat &	[hd & verb \\
			         subj & <NP$_{\tag{1}}$> \\
			         comps & <NP$_{\tag{2}}$> ] \\
			cont &	[act & \tag{1} \\
			und & \tag{2} ] ]
		}, name=trans
			[reg-trans, tier=word]
		]
		[%
		\avm{
			[\type*{pass}
			cat &	[hd & verb \\
			subj & <NP$_{\tag{2}}$> \\
			comps & <PP$_{\tag{1}}$> ] \\
			cont &	[act & \tag{1} \\
			und & \tag{2} ] ]
		}
			[reg-pass, tier=word]
		]
	]
]
{
\draw (trans.south) to (have.north);
}
\end{forest}
}

  \caption{Exceptions via pre-typing}
  \label{fig:KoenigPre}
\end{figure}



Online Type Construction successfully integrates systematic
alternations into type hierarchies. A crucial limitation is, however,
that Online Type Construction is confined to finite domains: by itself, it is suitable for
inflection and possibly quasi-inflectional, non-recursive processes as
grammatical function change, while a full treatment of derivational
processes will still require recursive rule types, which remain a
possibility in Koenig's general approach to derivational
morphology.\footnote{\citet{Blevins2003a} discusses the interaction
  between passives and impersonals in \ili{Baltic} and \ili{Slavic} languages and
  its relevance to some of the issues I just discussed. See
  \citet{ASU99a-u} for an account along these lines.  \citew[\page
  925--927]{MuellerUnifying} and \citew[Section~8.1]{MWArgSt} take a
  highly skeptical stance, arguing that interactions in grammatical
  function change depend on the possibility for one lexical rule to
  apply to the output of another, or, as in the case of \ili{Turkish}
  causatives, a rule may even apply more than
  once.}
  % \citet{Blevins2003a} discusses the interaction
  % between passives and impersonals in \ili{Baltic} and \ili{Slavic} languages and
  % shows that at least a three-fold distinction of argument structure
  % is needed in purely inheritance-based approaches. See
  % \citet{ASU99a-u} for an account along these lines.  \citew[\page
  % 925--927]{MuellerUnifying} and \citew[Section~8.1]{MWArgSt} take a
  % highly skeptical stance, arguing that interactions in grammatical
  % function change depend on the possibility for one lexical rule to
  % apply to the output of another, or, as in the case of \ili{Turkish}
  % causatives a rule may even apply more than
  % once. See also \citew[Section~7.5.2.2]{MuellerLehrbuch1} for  extensive
  % discussion fo this issue. Since \citet{Koenig99} already recognises
  % recursive lexeme types for derivational morphology, there is no
  % reason why lexical rules thus conceived should not be embraced for
  % grammatical function change as well, ultimately yielding an approach
  % that can scale up to more complex interactions.  }
% \footnote{%
  % It should be noted that the Online Type Construction approach did not go unchallenged. A
  % detailed discussion can be found in
  % \citew[Section~7.5.2.2]{MuellerLehrbuch1}. See also \citew[\page
  % 925--927]{MuellerUnifying}. Müller argues that interactions in
  % grammatical function change depend on the possibility for one
  % lexical rule to apply to the output of another. He discuses data
  % from \citet{Blevins2003a} showing that passives and impersonals
  % interact and that one would need at least a three-fold distinction
  % in inheritance-based approaches and hence miss a
  % generalization. Another example discussed by Müller is
  % causativization in \ili{Turkish}, which can be applied multiple times
  % . Since \citet{Koenig99} already recognises recursive lexeme types
  % for derivational morphology, there is no reason why lexical rules
  % thus conceived should not be embraced for grammatical function
  % change as well, ultimately yielding an approach that can scale up to
  % more complex interactions.  }

%\bigskip\noindent
The works of \citet{Riehemann98} and \cite{Koenig99} had considerable
impact on subsequent work on word formation, both within the framework 
of HPSG and beyond. Within HPSG, several studies of \ili{French} derivation
and compounding directly build on these proposals \citep[e.g.][]{Tribout10,Desmets09}. 
Outside, the development of Construction Morphology
\citep{Booij10} has largely been influenced by the HPSG work on word
formation within a hierarchical lexicon. 

\section{Inflection}
\label{sec:Infl}\label{sec:inflection}

\subsection{Classical challenges of inflectional systems}
\label{sec:InflChallenges}

Ever since \citet{Matthews72}, it has been recognised in morphological
theory that inflectional systems do not privilege one-to-one relations
between function and form, but must rather be conceived of as
many-to-many ($m:n$), in the general case. Thus, while
rule-by-rule compositionality can count as the success story
of syntax and semantics, this does not hold in the same way for inflection. 

Classical problems that illustrate the many-to-many nature of
inflection include cumulation, where a single form expresses multiple
morphosyntactic properties. An extreme example of cumulation is
contributed by the Latin verb \textit{am-o}
`love-1.\textsc{sg}.\textsc{prs}.\textsc{ind}.\textsc{av}', which
contrasts e.g.\ with forms \textit{amā-v-i}
`love-\textsc{prf}-\textsc{1.sg.av}', where perfective tense is
expressed by a discrete exponent \textit{-v}, or present subjunctive
\textit{am-ē-m} `love-\textsc{subj}-\textsc{1.sg.av}' where mood is
expressed by a marker of its own.

The mirror image of cumulation is extended (or multiple) exponence:
here, a single property is expressed by more than one exponent. This
is exemplified by \ili{German} circumfixal past participles, such as
\textit{ge-setz-t} `\textsc{ppp}-sit-\textsc{ppp}', which is
 marked by a prefix \textit{ge-} and a suffix \textit{-t},
jointly expressing the perfect/passive participial property. Another
case of multiple exponence is contributed by Nyanja, which marks
certain adjectives with a combination of two agreement markers, as
discussed on page \pageref{Nyanja} in
Section~\ref{sec:Panini}.  See
\citet{caballero_g-harris_a12} and \citet{Harris17} for a typological overview. 

Possibly more widely attested than pure multiple exponence is
overlapping exponence, i.e.\ the situation where two exponents both
express the same property, but at least one of them also expresses
some other property: e.g.\ many \ili{German} nouns form the dative plural by
suffixation of \textit{-n}, but plural marking is often signalled
additionally by stem modification (\textit{Umlaut}): while
\textit{Kutter-n} `tug(\textsc{m})-\textsc{dat.pl}' merely shows
cumulation of case and number, \textit{Mütter-n}
`mother(\textsc{f}).\textsc{pl}-\textsc{dat.pl}' exhibits plural
marking in both the inflectional ending and the fronting of the stem
vowel (cf.\ singular \textit{Mutter} `mother.\textsc{sg}').

An extremely wide-spread form of deviation from a one-to-one
correspondence between form and function is zero exponence, where some
morpho"=syntactic properties do not give rise to any exponence. In
\ili{English}, regular plural nouns are formed by suffixation of
\textit{-s}, as in \textit{jeep/jeeps}, but we also find cases, such
as \textit{sheep}, where no overt exponent of plural is
present. Likewise, the past tense of \ili{English} verbs is regularly
signalled by suffixation of \textit{-ed}, as with
\textit{flip/flipped} or British \ili{English} \textit{fit/fitted}, but
again, there are forms such as \textit{hit/hit} where past is not
overtly marked.  In \ili{German}, nouns inflect for four cases and two
numbers, yielding eight cells. However, in some paradigms very few
cells are actually overtly marked. The feminine noun \textit{Brezen}
`pretzel' does not take any inflectional markings. Similarly, one of
the most productive masculine/neuter paradigms, witnessed by
\textit{Rechner} `computer', only shows overt marking for two cells,
the genitive singular (\textit{Rechner-s}) and the dative plural
(\textit{Rechner-n}), all other forms being bare.



The many-to-many nature of inflectional morphology clearly has
repercussions as to how the system is organised.  One way to make
sense of inflection is in terms of paradigmatic opposition: while it
may be hard to figure out what exactly the meaning is of zero
case/number marking in \ili{German}, we can easily establish the meaning of
a form like \textit{Rechner} in opposition to the non-bare forms
\textit{Rechner-s} `computer-\textsc{gen.sg}' and \textit{Rechner-n}
`computer-\textsc{dat.pl}'. This is even more the case once we
consider different paradigms, i.e.\ different patterns of opposition:
the invariant form \textit{Brezen} `pretzel', for instance, has a wider
denotation than \textit{Rechner}, whereas \textit{Auto}
`car' has a narrower denotation,  standing in
opposition to more cells, cf.\ Table~\ref{tab:ParaSplit}(c).    

The recognition of paradigms has led to a number of works on
syncretism \citep[see, e.g.][]{Baerman05}, i.e.\ cases of systematic or
accidental identity of form across different cells of the
paradigm. Syncretism can give rise to splits of different types
\citep{Corbett15}: natural splits, where syncretic forms share some
(non-disjunctive) set of features, Pāṇinian splits, where syncretism
corresponds to some default form, and finally morphomic splits, where
syncretic forms neither form a natural class nor do they lend
themselves to be analysed as a default. 

\begin{table}
  \centering
  
  \begin{subfigure}{.45\textwidth}
    
    \centering
    \begin{tabular}{r|ll}
      \lsptoprule
      `granny' & \textsc{singular} & \textsc{plural}\\
      \midrule
      \textsc{nom} & Oma & Oma-s\\
      \textsc{gen} & Oma & Oma-s\\
      \textsc{dat} & Oma & Oma-s\\
      \textsc{acc} & Oma & Oma-s\\
      \lspbottomrule
    \end{tabular}

    \caption{Natural split}
  \end{subfigure}  
  \begin{subfigure}{.50\textwidth}
    
    \centering
    \begin{tabular}{r|ll}
      \lsptoprule
      `computer' & \textsc{singular} & \textsc{plural}\\
      \midrule
      \textsc{nom} & Rechner & Rechner\\
      \textsc{gen} & Rechner-s & Rechner\\
      \textsc{dat} & Rechner & Rechner-n\\
      \textsc{acc} & Rechner & Rechner\\
      \lspbottomrule
    \end{tabular}
    \caption{Pāṇinian split}
  \end{subfigure}

  \begin{subfigure}{.45\textwidth}
    \centering

    \begin{tabular}{r|ll}
        \lsptoprule
        `car' & \textsc{singular} & \textsc{plural}\\
        \midrule
        \textsc{nom} & Auto & Auto-s\\
        \textsc{gen} & Auto-s & Auto-s\\
        \textsc{dat} & Auto & Auto-s\\
        \textsc{acc} & Auto & Auto-s\\
        \lspbottomrule
      \end{tabular}
      \caption{Natural \& Pāṇinian split}
    \end{subfigure}
  \begin{subfigure}{.50\textwidth}

    \centering
    \begin{tabular}{r|ll}
      \lsptoprule
      `wall' & \textsc{singular} & \textsc{plural}\\
      \midrule
      \textsc{nom} & mur-s & mur\\
      \textsc{acc} & mur & mur-s\\
      \lspbottomrule
    \end{tabular}

    \caption{Morphomic split (Old French)}
  \end{subfigure}
  
  
  \caption{Paradigmatic splits}
  \label{tab:ParaSplit}
\end{table}

In Table~\ref{tab:ParaSplit}(a), we find a perfect alignment of
syncretic forms along the number dimension. By contrast,
Figure~\ref{tab:ParaSplit}(b) illustrates the case discussed above,
where two specific cells constitute overrides to a general default
pattern (here zero exponence). Default forms, however, need not
involve zero exponence: \ili{German} features a Pāṇinian split in another
paradigm where all forms are marked with \textit{-en}
(e.g.\ \textit{Mensch-en} `human(s)'), with the exception of the
nominative singular (\textit{Mensch} `human\textsc{.nom.sg}'), which
constitutes a zero override. Table~\ref{tab:ParaSplit}(c) illustrates
how a Pāṇinian split in the singular can combine with a natural split
between singular and plural. Finally, Table~\ref{tab:ParaSplit}(d)
illustrates what could be taken as a morphomic split, where there is no
natural alignment between form and function, and no clear way to
establish what is the default and what is the override (cf., however,
\citealt{Crysmann:Kihm:2018} for an analysis of the Old \ili{French}
declension system).

The patterns we have just seen have two clear implications for
morphological theory: first, many morphologists believe that
a version of Pāṇini's Principle, whereby more specific forms can block
more general ones, must be part of morphological theory, since
otherwise many generalisations will be lost. 
Second, the many-to-many nature of exponence has a direct impact on
the representation of inflectional meaning, which we will explore in
the next two subsections. 

\subsection{Typology of inflectional theories}
\label{sec:InflTypology}

Current morphological theories differ as to how they establish the
relation between a complex form and its parts and how this relation
determines the relation between form and function. The classical
morpheme-based view of morphology, where inflectional meaning is a
property of lexical elements, such as morphemes, constitutes the text
book case of what \citet{Hockett54} has dubbed the \isi{Item-and-Arrangement}
(IA) model.  The general criticism that has been raised against such
models is that they fail to recognise the paradigmatic structure of
inflectional morphology and furthermore need to make extensive appeal
to zero morphemes (see \citealp{Anderson92} for a systematic
criticism).

The alternative model \citet{Hockett54} discusses is the
\isi{Item-and-Process} (IP) model where inflectional meaning is introduced
syncategorematically by way of rule application. Such approaches
are less prone to have difficulties with non"=concatenative processes
like modification and zero exponence. However, IP approaches still do
not recognise the $m:n$ nature of inflectional morphology and are
therefore expected to have problems with e.g.\ multiple exponence. 


As a reaction to \citet{Matthews72}, new approaches to inflectional
morphology were developed taking the notion of paradigms much more
seriously. Theories, such as A-Morphous Morphology\is{morphology!A-Morphous} \citep{Anderson92}
or Paradigm Function Morphology\is{morphology!Paradigm Function} \citep{Stump01} have been classified
into the \isi{Word"=and"=Paradigm} (WP) category. Crucially, such models
locate inflection at the level of the word and rely on realisation
rules that associate the word's inflectional properties with exponents
that serve to express them. WP approaches contrast with IA in that
they do not recognise (classical) morphemes. They differ from IP in
that there is neither a notion of incrementality, i.e.\ that
inflectional rules must be information-increasing, nor that rules are
necessarily one-to-one correspondences between (alteration of) form
and meaning.


\subsection{HPSG approaches to inflection}
\label{sec:InflHPSG}

Over the years, several different proposals have been made regarding the
treatment of inflectional morphology in HPSG. From the point of view
of the underlying logic, there is no a priori expectation as to the type of model
(IA, IP, WP) that would be most compatible with HPSG's basic
assumptions. Indeed, every one of the three models have been proposed at
some point. However, the arguments against morpheme-based models put
forth by \citet{Matthews72}, \citet{Spencer91}, \citet{Anderson92} and
\citet{Stump01} have been taken quite seriously within the HPSG
community, such that there is a clear preference for IP or WP models over
IA, notable exceptions being \citet{van-eynde_f94} and, more
recently, \citet{Emerson15}.

One of the most common ways to express lexical alternations is by
means of (description-level) lexical rules. Morphophonological changes
effected by such a rule are typically captured by some (often
undefined) function on the phonology of the daughter. Since
morphological marking is tied directly to rule application, approaches
along these lines constitute an instance of an IP model of
morphology. Work on morphology in grammar implementation typically
follows this line: in platforms like the Linguistic Knowledge
Builder\indexlkb  (\citealp[LKB;][]{Copestake02}, see also \crossrefchapteralp[\page \pageref{cl:delphin}]{cl})
character unification serves to provide statements of
morphophonological changes that can be attached to (unary) lexical
rules. See \citet{Goodman10} for a proposal as to how requirements for
certain inflections and dependencies
between morphological rules, e.g.\ the parts of extended or overlapping
exponence, can be captured in a more systematic way,
and \citet{Crysmann:15:JLM,Crysmann:2017:JOMO} for implementations of
non-concatenative morphology. 


A notable exception to the function approach is the work of Olivier
Bonami \citep{bonami2015diversity,Bonami14d}: he argued for the incorporation of an
external formal model of morphology into HPSG, namely Paradigm Function
Morphology  \citep[=PFM;][]{Stump01}, and
showed specifically that the integration should be done at the level
of the word, rather than individual lexical rules, in order to reap
the benefits of a WP model over an IP model. In a similar vein,
\citet{Erjavec94} explores how a model such as PFM can be cast in
typed feature descriptions and observes that the only non-trivial
aspect of such an enterprise relates to Pāṇinian competition, which
requires a change to the underlying logic. See Section
\ref{sec:Panini} for detailed discussion.

In the area of cliticisation, several sketches of WP models have been
proposed: e.g.\ \citet{Miller97} provide an explication of the function
that realises the pronominal affix cluster, but the proposal was never
meant to scale up to a full formal theory of
inflection. \citet{crysmann_b03book} suggested a realisational,
morph-based model of inflection. While certainly more worked-out, the
approach was too tailored towards the treatment of clitic
clusters. 



\subsubsection{Word-based approaches}

\itdobl{This is the only subsubsection here. Such things should not exist. Add another one or remove
  this section heading.}

\paragraph*{\citet{Krieger:Nerbonne:93}}
As stated above, probably one of the first approach\-es to morphology
in HPSG was developed by \citet{Krieger:Nerbonne:93}. What they
propose is essentially an instance of a WP model, since they use
distributed disjunctions to directly represent entire paradigms,
matching exponents with the features they express. Most interestingly,
their approach to inflection contrasts quite starkly with their
work on derivation \citep{Krieger:Nerbonne:93}, which is essentially a
word-syntactic, i.e.\ morpheme-based, approach.


\begin{table}
  \centering
  \begin{tabular}{r|cc}
    \lsptoprule
    & \textsc{Singular} & \textsc{Plural}\\
    \midrule
    1 & -e & -n\\
    2 & -st & -t\\
    3 & -t & -n\\
    \lspbottomrule
  \end{tabular}
  \caption{Regular present indicative endings for  German verbs}
  \label{tab:GermanEndings}
\end{table}

(\ref{ex-DistrDisj}) represents an encoding of the present
indicative paradigm for \ili{German} (cf.\ the endings in Table~\ref{tab:GermanEndings}). The distributed disjunction, marked by ${\$
  1}$, associates each element in the disjunctive \textsc{ending}
value with the corresponding element in the disjunctive \textsc{agr}
value. 

% <<<<<<< HEAD

% \begin{figure}[htb]
%   \centering

%   \begin{avm}
%     \[morph & \[stem & \@2\\
%         ending & \@3 \{ $_{\$1}$  \normalfont ``e'', ``st'', ``t'', ``n'', ``t'',
%         ``n''\}\\
%         form & \normalfont \@2 \mathplus{}  \@3\]\\
%       synsem & \[local|head|agr & \{$_{\$1}$ \[per & 1\\num & sg \], \[per & 2\\
%         num & sg\], ... , \[per & 3\\num & pl\]\} \]\]
%   \end{avm}
  
%   \caption{Encoding paradigms by distributed disjunctions \citep{Krieger:Nerbonne:93}}
%   \label{fig:DistrDisj}
% \end{figure}
% =======
\ea
\label{ex-DistrDisj}
Encoding paradigms by distributed disjunctions \citep[105]{Krieger:Nerbonne:93}:\\
\avm{
	[morph &	[stem & \2 \\
				ending & \3 \{ $_{\$1}$  \normalfont ``e'', ``st'', ``t'', ``n'', ``t'', ``n''\} \\
				form & \2 $\mathplus $ \3] \\
	synsem &	[local|head|agr & \{$_{\$1}$	[per & 1 \\
												num & sg]
												,
												[per & 2 \\
        										num & sg]
        										, \ldots ,
        										[per & 3 \\
        										num & pl] \} ] ]
}
\z  


They further argue that partially regular formations, such as
\textit{sollen} `should', which has no ending in the first and third
singular can be captured by means of default inheritance, overriding
the \textsc{ending} value as in (\ref{ex-DistrDisjSollen}).  

\ea
\label{ex-DistrDisjSollen}

Partial irregularity by overriding default endings \citep[105]{Krieger:Nerbonne:93}:\\
\avm{
	[morph &	[ending & \{ $_{\$1}$ \normalfont ``'', ``st'', ``'', ``n'', ``t'', ``n''\} ] ]
}
\z

Suppletive forms, as for auxiliary \textit{sein} `be', will equally
inherit from (\ref{ex-DistrDisj}), yet override the form value,
cf.\ (\ref{ex-DistrDisjSupp}).  

\ea
\label{ex-DistrDisjSupp}
Suppletive verbs \citep[106]{Krieger:Nerbonne:93}:\\
\avm{
	[morph &	[form & \{ $_{\$1}$ \normalfont ``bin'', ``bist'', ``ist'', ``sind'', ``seid'', ``sind''\} ] ]
}
\z

The approach by \citet{Krieger:Nerbonne:93} has not been widely
adopted, partially because few versions of HPSG support default
inheritance and even fewer support distributed
disjunctions. \citet[176--178]{Koenig99} also argues against distributed
disjunctions on independent theoretical grounds, suggesting that the
approach will not scale up to morphologically more complex systems.  

\paragraph*{\citet{Koenig99}}

Similar to \citet{Krieger:Nerbonne:93}, \citet{Koenig99} pursues a
word-based approach to inflection, in contrast to the IP approach he
developed for derivation. He focuses on the distinction between
regular, subregular and irregular formations and explores how these
can be represented in a systematic way in lexical type hierarchies
using Online Type Construction.

He departs from the observation that words  inflect along a finite number
of different inflectional dimensions and that within each dimension,
pairings of exponents and morphosyntactic features stand in
paradigmatic opposition. Furthermore, neither completely uninflected
roots, nor partially derived words (e.g.\ lacking agreement
information) shall be able to function as lexical signs, so it is
necessary to enforce that inflection be applied. The AND/""OR
logic of dimensions and types he proposed appears to be very
well-suited to account for these properties. 

\begin{table}
\setlength{\tabcolsep}{.3em}
\centering
\begin{tabular}{llllll}
\lsptoprule 
 & \textsc{pos} & \textsc{neg}     &             & \textsc{pos} & \textsc{neg}\\
\midrule 
\textsc{1sg}       & ni-{ta}-tak-{a} & {si}-{ta}-tak-{a}    & \textsc{1pl} & tu-{ta}-tak-{a}     & {ha}-tu-{ta}-tak-{a}\\
\textsc{2sg}       & u-{ta}-tak-{a}  & {ha}-u-{ta}-tak-{a}  & \textsc{2pl} & m-{ta}-tak-{a}      & {ha}-m-{ta}-tak-{a}\\
\textsc{3sg.m/wa}  & a-{ta}-tak-{a}  & {ha}-a-{ta}-tak-{a}  & \textsc{3pl.m/wa} & wa-{ta}-tak-{a} & {ha}-wa-{ta}-tak-{a}\\ 
\textsc{3sg.ki/vi} & ki-{ta}-tak-{a} & {ha}-ki-{ta}-tak-{a} & \textsc{3pl.ki/vi} & vi-{ta}-tak-{a} & {ha}-vi-{ta}-tak-{a}\\
etc. & &\\
\lspbottomrule
\end{tabular}
\caption{Future forms of the Swahili verb  \textit{taka} `want'}
\label{tab:SwahiliPast}
\end{table}


For illustration, let us consider a subset of his analysis of \ili{Swahili}
verb inflection. As shown in Table~\ref{tab:SwahiliPast}, \ili{Swahili}
verbs (minimally) inflect for polarity, tense and subject
agreement.\footnote{The full paradigm recognises inflection for object
agreement and relatives, but this shall not concern us here, it being
sufficient that inflectional paradigms may be large but finite.} 


\begin{sloppypar}
  \citet[Section~5.5.2]{Koenig99} suggests that the inflectional
  morphology of \ili{Swahili} can be directly described at the word
  level. Accordingly, he proposes a type hierarchy of word-level
  inflectional constructions as given in
  Figure~\ref{fig:KoenigSwahili}.
\end{sloppypar}

\begin{figure}
  \centering
% \begin{forest}
% [\emph{verb-infl}
% 	[2ND-SLOT,draw
%     	[\emph{1sg},name=1sg
%     		[\emph{1sg-pos}]
%     	]
%     	[\emph{3pl}]
%    ]
% 	[1ST-SLOT,draw
% 		[\emph{neg}
% 			[\emph{1sg-neg},name=neg]
% 			[\emph{¬1sg-neg}]
% 		]
% 		[\emph{pos}]
% 	]
% 	[3RD-SLOT,draw
% 		[\emph{pst}
% 			[\emph{pos-pst}]
% 			[\emph{neg-pst}]
% 		]
% 		[\emph{fut}]
% 	]
% ]
% \draw (1sg.south) to (neg.north);
% \end{forest}
\begin{forest}
type hierarchy
[verb-infl
   [2nd-slot,partition
     [1sg
       [1sg-pos]
       [,identify=!NN]] % !N is the next leaf. !NN is the second next leaf. This one is
                        % identical to this node.
     [3pl]]
   [1st-slot,partition
     [neg
	[1sg-neg] % An alternative would be to draw an edge from here to the node
                  % that is the first daughter of the first daughter of the root node:, edge to=!r11
        [¬1sg-neg]]
     [pos]]
   [3rd-slot,partition
     [pst
	[pos-pst]
	[neg-pst]]
     [fut]]]
\end{forest}

  \caption{Koenig's (\citeyear[171]{Koenig99}) constructional approach to Swahili position
    classes}\label{fig:KoenigSwahili}
\end{figure}

As shown in Table~\ref{tab:SwahiliPast}, tensed verbs with plural
subjects take three prefixes in the negative and two in the positive, with
the exponent of negative preceding the exponent of subject agreement,
preceding in turn the exponent of tense. \citet{Koenig99} proposes
three dimensions of inflectional construction types that correspond to
the three positional prefix slots. Since dimensions are conjunctive, a
well-formed \ili{Swahili} word must inherit from exactly one type in each
dimension. As he states, the AND/""OR logic of dimensions and types is
the declarative analogue of the conjunctive rule blocks and
disjunctive rules in A-morphous Morphology \citep{Anderson92}.

Types in the dimensions are partial word-level descriptions of
(combinations of) prefixes. As shown by the sample types in (\ref{ex-KoenigNegPast}), these partial descriptions pair some
morphosyntactic properties (\textsc{$\mu$-feat}) with constraints on
the prefixes: the type \textit{$\neg$1sg-neg}, for instance, constrains the
first prefix slot to be \textit{ha-}, while leaving the other slots
underspecified. These will be further constrained by appropriate types
from the other two dimensions. Likewise, the type \textit{1sg-pos},
constrains slot~2 to be \textit{ni-}, but specifies the further
requirement that the verb be \textsc{[neg~$-$]}.    

\begin{exe}
\ex\label{ex-KoenigNegPast}
Sample types for \ili{Swahili}:
\begin{xlist}
\ex \textit{$\neg$1sg-neg}:\\
	\avm{
		[ph &	[aff &	[pref & < \normalfont ha, \ldots, \ldots > ] ] \\
		cat  &	[head &	[$\mu$-feat &	[neg & $\mathplus $] ] ]]
	}

\ex \textit{1sg-pos}:\\
	\avm{
		[ph &	[aff &	[pref & < \ldots, \normalfont ni, \ldots > ] ] \\
		cat  &	[head &	[$\mu$-feat &	[neg & $-$ \\
										subj-agr &	[per & 1 \\
    												num & sg] ] ] ] ]
	}
\ex \textit{1sg-neg}:\\
\avm{
	[ph &	[aff &	[pref & < \normalfont si, < >, \ldots > ] ]\\
	cat  &	[head &	[$\mu$-feat &	[neg & $\mathplus $ \\
									subj-agr &	[per & 1 \\
												num & sg] ] ] ] ]
}
\zl


Pre-linking of types finally permits a straightforward treatment of
cumulation across positional slots: e.g.\ the type \textit{1sg-neg}
simultaneously satisfies requirements for the first and second slot,
constraining one of the prefixes to be portmanteau \textit{si-}, the
other one to be empty. Thus, by adopting a constructional perspective
on inflectional morphology, \citet{Koenig99} can capture interactions
between different affix positions. There is, however, one important
limitation to a direct word-based perspective: situations where
exponents from the same set of markers may (repeatedly) co-occur
within a word cannot be captured without an intermediate level of
rules. Such a situation is found with subject and object agreement
markers in \ili{Swahili} \emdash so-called parallel position classes
\citep{Stump93,Crysmann:Bonami:2016} \emdash, as well as with exuberant
exponence in \ili{Batsbi} \citep{Harris09,Crysmann:2018:Batsbi}. We shall
come back to the issue in Section~\ref{sec:ConstrGen}. Finally, since
exponents are directly represented on an affix list under Koenig's
approach, position and shape cannot always be underspecified
independently of each other, which makes it more difficult to
capture variable morphotactics (see Section~\ref{sec:Mortax}).


An aspect of (inflectional) morphology that \citet{Koenig99} pays
particular attention to is the relation between regular, subregular
and irregular formations. He approaches the issue on two levels: the
level of knowledge representation and the level of knowledge use. 

At the representational level, regular formations, e.g.\ past tense
\textit{snored}, are said to be intensionally defined in terms of
regular rule types that license them: results of regular rule
application are consequently not listed in the lexicon. Rather, they are
constructed either by Online Type Construction or by rule application.
Irregular formations, by contrast, are fully listed, e.g.\ the past
tense form \textit{took} of a verb like \textit{take}. Most
interesting are subregular types, e.g.\ \textit{sing/sang/sung} or
\textit{ring/rang/rung}: like irregulars, class membership is
extensionally defined by enumeration, but the type hierarchy can still
be exploited to abstract out common properties.

With regular formations being defined in terms of productive schemata,
an important task is to preempt any subregular or irregular root from
undergoing the regular, productive pattern. \citet{Koenig99} discusses
three different approaches in depth: a feature-based approach, and two
ways of invoking Pāṇini's Principle. As for the former, he shows that
the costs associated with diacritic exception features is actually
minimal, i.e.\ it is sufficient to specify irregular and subregular
bases as \textsc{[irr~\mathplus{}]} and constrain the regular rule to
\textsc{[irr~$-$]}. Thus, use of such diacritics does not need to be
stated for the large and open class of regular, productive
bases. Despite the relatively harmless effects of the feature-based
approach, it should be kept in mind that this approach will not scale
up to a full treatment of Pāṇinian competition.\footnote{This is
  because first, every default/override pair would need to be
  stipulated, and second, if a paradigm has defaults in different
  dimension (e.g.\ a default tense, or a default agreement marking),
  each would need its own diacritic feature.}

\citet{Koenig99} proposes two variants of a morphological and/""or
lexical blocking theory. In essence, he builds on a previous
formulation by \citet{Andrews90} within LFG to define a notion of
morphological competition based on subsumption. Since competition is
between different realisations for the same morphological features, he
applies a restrictor on form-related features to then establish
competition in terms of unilateral subsumption ($\sqsubset$): i.e.\ a
rule-description that is more specific than some other rule (modulo
form-oriented features) will take precedence. I shall not go into the
details of Koenig's Blocking Principle here, since we shall come back
to a highly similar formulation of Pāṇinian competition in
Section~\ref{sec:Panini}.  \citet{Koenig99} discusses two different
ways this can be accomplished: one is a compilation approach where
complementation is used to make the more general type disjoint,
whereas the other relegates the problem to the area of knowledge
use. While the usage-based interpretation may appear preferable,
because it does not require expansion of the lexical type-hierarchy,
it leaves open the question why this kind of competition is mainly
restricted to lexical knowledge. On the other hand, the static
compilation approach requires prior expansion of the type
underspecified lexicon in order to give sound results under
restriction, a point made in \citet{crysmann_b03book}.

%\bigskip\noindent 
To summarise, several WP proposals have been made to
replace the IP model tacitly assumed by many HPSG syntacticians, which
merely attaches some morpho"=phonological function to a lexical rule.
Bonami \citep{Bonami08f,Bonami06,Bonami07e,Bonami11f} proposed 
directly ``plugging in'' a credible external framework, namely Paradigm
Function Morphology \citep{Stump01}, \citet{Koenig99} suggested a
word-based model. Neither approach has proven to be fully
satisfactory. Use of an external theory, such as PFM, begs
the question why we need a different formalism in order to implement a
theory of inflection, rather than exploiting the power of inheritance and
cross-classification in hierarchies of typed feature structure
descriptions. Word-based approaches suffer from problems of
scalability with morphotactically complex systems.  These issues led
to the development of Information-based Morphology
\citep{Crysmann:Bonami:2016}, which will be discussed in the next
section.


\section{Information-based Morphology}
\label{sec:IbM}

Information-based morphology \citep{Crysmann:Bonami:2016} is a theory
of inflectional morphology that systematically builds on HPSG-style
typed feature logic in order to implement an inferential-realisational
model of inflection. As the name suggests, in reference to
\citet{Pollard87}, it aims at complementing HPSG with a 
subtheory of inflection that systematically 
explores underspecification and cross-classification as the central
device for morphological generalisations.

IbM clearly builds on previous HPSG work on morphology and the
lexicon: Online Type Construction \citep{Koenig94} can be cited here
in the context of the underlying logic. Similarly, the decision to
represent morphotactics in terms of a flat lists of segmentable
exponents (=morphs) draws on previous work by
\citet{crysmann_b03book}. 

% Brief comparison between IbM and PFM/AM

\subsection{Architecture and principles}

The architecture of IbM is quite simple: essentially, words are
assumed to introduce a feature \textsc{infl} that encapsulates all
features relevant to inflection.\footnote{For the purposes of this
  chapter, I shall make the somewhat simplifying assumption that
  inflection is a property exclusively associated with words. However,
  \citet{Koenig:Michelson:2020} present compelling evidence from
  nominalisation in Oneida, showing that derivational processes in
  this language may target (partially) inflected bases, including
  nominalisation of aspectually inflected verbal stems, as well as
  incorporation of inflected and derived nominals into polysynthetic
  verbs. It therefore seems necessary to generalise the interface
  between lexical types and inflectional morphology in such a way that
realisational morphology can be applied to sub-word units within a
derivational chain. } At the top-level, these comprise \textsc{mph},\label{morphology:page-mph-feature} a partially
ordered list of exponents (\textit{morph}), a morphosyntactic (or
morphosemantic) property set \textsc{ms} associated with the word, and
finally \textsc{rr}, a set of realisation rules that establish the
correspondence between exponents and morphosyntactic properties.

\begin{exe}
  \ex
\avm{
	\emph{word} \impl
[infl &	[mph & \listOf{morph} \\
	 rr  & \avmset*(realisation-rule) \\
	 ms  & \setOf{msp} ] ]
}

\end{exe}

From the viewpoint of inflectional morphology, words can be regarded
as associations between a phonological shape (\textsc{ph}) and a
morphosyntactic property set (\textsc{ms}), the latter including, of
course, information pertaining to lexeme identity. This correspondence
can be described in a maximally holistic fashion, as shown in 
(\ref{fig:Word}), where a phonological form is paired with information
about lexemic identity (\textsc{lid}) and a morphosyntactic property (\textsc{tam}). Throughout this section, I shall use \ili{German}
(circumfixal) passive/past participle (\emph{ppp}) formation, as
witnessed by \textit{ge-\emph{setz}-t} `put', for illustration.

\begin{exe}
  \ex
	\avm{
		[ph & < \normalfont gesetzt> \\
		infl &	[ms & \{[lid & setzen], [tam & ppp]\}] ]
}
  
  \label{fig:Word}
\end{exe}

Since words in inflectional languages typically consist of multiple
segment\-able parts, realisational models provide means to index
position within a word: while in A-Morphous Morphology
\citep[=AM;][]{Anderson92} and Paradigm Function Morphology
\citep[=PFM;][]{Stump01} ordered rule blocks perform this function, IbM
uses a list of morphs (\textsc{mph}) to explicitly represent
exponents.
The sample word-level representation in (\ref{fig:WordMph})
illustrates the kind of information represented on the \textsc{mph}
list and the \textsc{ms} set. 
\begin{exe}
  \ex Structured association of form (\textsc{mph}) and function (\textsc{ms})   \label{fig:WordMph}

  \begin{xlist}
    \ex Word:

\avm{
	[mph &	<[ph &  !<ge>! \\
			pc & -1]
			,
			[ph &  !<setz>! \\
			pc & 0]
			,
			[ph &  !<t>! \\
			pc & 1]> \\
	ms &	\{[lid & setzen], [tam & ppp]\} ]
}
    \ex Abstraction of circumfixation ($1:n$):

\avm{
	[mph &	<[ph &  !<ge>! \\
			pc & -1]
			,
			[ph &  !<t>! \\
			pc & 1]
			, \ldots> \\
	ms &	\{[tam & ppp], \ldots\} ]
}
  \end{xlist}
\end{exe}
While elements of the \textsc{ms} set
are either inflectional features or lexemic properties, the latter
comprising e.g.\ information about the stem shape or inflection class
membership, \textsc{mph} is a list of structured elements (of type
\textit{mph}, cf.~(\ref{ex:mph})) consisting
of a phonological description (\textsc{ph}) paired with a position
class index (\textsc{pc}), which serves to establish linear order of
exponents. In some previous work on IbM, \textsc{mph} was assumed to
be a set, which is possible since order can be determined on the basis
of \textsc{pc} indices alone. More recently, however, it is assumed to
be a list, which is slightly redundant, yet permits much more
parsimonious descriptions of principles and rules.

\begin{exe}
  \ex \label{ex:mph} 
\avm{
	\emph{mph} \impl
[ph & \listOf{phon} \\
 pc & pos-class]
}
\end{exe}

The reification of position and shape as first-class citizens of
morphological representation is one of the central design decisions of
IbM: as a result, constraints on position and shape will be amenable
to the very same underspecification techniques as all other
morphological properties. As a consequence, IbM eliminates structure
from inflectional morphology, which clearly distinguishes this
approach from other inferential-realisational approaches, such as PFM
or AM, where order is derived from cascaded rule application.
Although IbM recognises a minimal structure in terms of segmentable
morphs, there is no hierarchy involved. AM and PFM, by contrast,
reject derived structure, to borrow a term from Tree Adjoining
Grammar, but this potential advantage is more than offset by their
abundant use of derivation structure.

By means of underspecification, i.e.\ partial descriptions, one can
easily abstract out realisation of the past participle property,
arriving at a direct word-based representation of circumfixal
realisation, as shown in (\ref{fig:WordMph}).  Yet, a direct
word-based description does not easily capture situations where the
same association between form and content is used more than once in
the same word, as is arguably the case for \ili{Swahili}
\citep{Stump93,Crysmann:Bonami:2016,Crysmann:Bonami:2017:HPSG} or \ili{Batsbi} \citep{Harris09,Crysmann:2018:Batsbi}.

By  introducing a level of realisation rules (\textit{rr}), reuse
of resources becomes possible. Rather than expressing the relation
between form and function directly at the word level, IbM assumes that
a word's description includes a specification of which rules license
the realisation between form and content, as shown in 
(\ref{fig:WordRR}).


\eas
\label{fig:WordRR}%
Association of form and function mediated by rule:\\
%\oneline{%
% This does not work since the distance in the small AVMs for ge, setzt, and t are too big.
% So, I added space manually.
%\avm[extraskip=\bigskipamount]{ todo avm
\avm{
	[mph & <\rnode{xx}{\tag{g}	[ph & <\normalfont ge> \\
								pc & -1]},
			\rnode{aa}{\tag{s}	[ph & <\normalfont setz> \\
								pc & 0]},
			\rnode{ab}{\tag{t}	[ph & <\normalfont t> \\
								pc & 1]}> \\\\\\
	rr &	\{[mph & <\rnode{ba}{\tag{s}	[ph & <\normalfont setz> \\
											pc & 0]}> \\
			mud & \{\rnode{ca}{\tag{l}	[lid & setzen]}\} ],
			[mph & <\rnode{yy}{\tag{g}	[ph &
                          <\normalfont ge> \\
										pc & -1]},
					\rnode{bb}{\tag{t}	[ph &
                                          <\normalfont t> \\
										pc & 1]}> \\
			mud & \{\rnode{cb}{\tag{p}	[tam & ppp]}\} ] \} \\\\\\
	ms & \{\rnode{da}{\tag{l}	[lid & setzen]},
			\rnode{db}{\tag{p}	[tam & ppp]}\} ]
}%
\psset{arrows=<->,linecolor=gray,linestyle=dotted,angleA=-90,angleB=90}%
\nccurve{xx}{yy}\nccurve{aa}{ba}\nccurve{ab}{bb}\nccurve{ca}{da}%
\nccurve{cb}{db}
%}%
\zs

Recognition of a level of realisation rules that mediate between parts
of form and parts of function slightly increases the complexity of
morphological descriptions beyond a simple pairing of form-related
\textsc{mph} lists and function-related \textsc{ms} sets. 

The crucial point about realisation rules is that they take care of
parts of the inflection of an entire word independently of other
realisation rules. Thus, in IbM, realisation rules are explicitly
defined in terms of the set of morphosyntactic features they express,
as opposed to contextually conditioning features. To that end, realisation
rules introduce a feature \textsc{mud} (Morphology Under Discussion),
in addition to \textsc{mph} and \textsc{ms}, in order to single out
the morphosyntactic features that are licensed by application of the
rule. Thus, \textsc{mud} specifies the subset of the morphosyntactic
property set \textsc{ms} that the rule serves to express, as detailed
in (\ref{ex:MudMS}). 

\begin{exe}
  \ex \label{ex:MudMS}
\avm{
\emph{realisation-rule} \impl
[mud & \1 \setOf{msp} \\
 ms  & \1 \cupAVM \setOf{msp} \\
 mph & \listOf{morph}]
}

\end{exe}

Realisation rules (members of set \textsc{rr}) pair a set of
morphological properties to be expressed, the morphology under
discussion (\textsc{mud}), with a list of morphs that realise them
(\textsc{mph}). Since \textsc{mud}, being a set, admits multiple
morphosyntactic properties, and since \textsc{mph}, being a list,
admits multiple exponents, realisation rules in fact establish $m:n$
relations between function and form: thus, the many-to-many nature of
inflectional morphology is captured at the most basic level. This very
property sets the present framework apart from cascaded rule models of
inferential-realisational morphology \citep{Anderson92,Stump01}, which
attain this property only indirectly as a system: rules in these
frameworks are $m:1$ correspondences between functions and form, but
since rules in different rule blocks may express the same functions,
the system as a whole can capture $m:n$ relations.

\ea
\label{ex-MCC}\label{morphology:morphology-well-formedness}
Morphological well-formedness:\\
% todo avm after a \tag, langsci-avm does not skip automatically #45.
\avm{
\emph{word} \impl
	[mph & \tag{e$_1$} \shuffle \ldots{} \shuffle \tag{e$_n$} \smallskip\\
	rr &	\{[mph & \tag{e$_1$} \\
			mud & \tag{m$_1$} \\
			ms & \0 ], \ldots,
			[mph & \tag{e$_n$} \\
			mud & \tag{m$_n$} \\
			ms & \0] \} \\
	ms & \tag{m$_1$} $\uplus \mbox{~\ldots~} \uplus$ \tag{m$_n$} ]
}
\z

Given two distinct levels of representation, the morphological word
and the rules that license it, it is of course necessary to define how
constraints contributed by realisation rules relate to the overall
morphological makeup of the word. Realisation rules per se only
provide recipes for matching morphosyntactic properties onto exponents
and vice versa. In order to describe well-formed words, it is
necessary to enforce that these recipes actually be applied. IbM
regulates the relation between word-level properties and realisation
rules by means of a rather straightforward principle, given in
(\ref{ex-MCC}): this very general principle of morphological
well-formedness ensures that the properties expressed by rules add up
to the word's property set, and that the rules' \textsc{mph} lists add
up to that of the word, such that no contribution of a rule may ever
be lost. This principle of general well-formedness in
(\ref{ex-MCC}) bears some resemblance to LFG's principles of
completeness and coherence \citep{bresnan_j82}, as well as to the
notion of ``Total Accountability'' proposed by \citet{Hockett47}. Since
$m:n$ relations are recognised at the most basic level,
i.e.\ morphological rules, mappings between the contributions of the
rules and the properties of the word can (and should) be $1:1$.  We
shall see below that this makes possible a formulation of morphological
well-formedness in terms of exhaustion of the morphosyntactic property
set.

In essence, a word's morphosyntactic property set (\textsc{ms}) will
correspond to the non-trivial set union (\isi{$\uplus$}) of the rules'
\textsc{mud} values: While standard set union (\isi{\cupAVM}) allows for the
situation that elements contributed by two sets may be collapsed,
non-trivial set union ($\uplus$) insists that the sets to be unioned
must be disjoint.  The entire morphosyntactic property set of the word
(\textsc{ms}) is visible on each realisation rule by way of structure
sharing (\ibox{0}).

Finally, a word's sequence of morphs, and hence its phonology, will
be obtained by shuffling (\isi{$\bigcirc$}) the rules' \textsc{mph} lists in
ascending order of position class (\textsc{pc}) indices (see
Chapter~\crossrefchaptert[\page \pageref{rel-shuffle}]{order} for a definition of the shuffle
relation, also known as
sequence union). This is ensured by the Morph
Ordering Principle given in (\ref{ex-MOP}), adapted from
\citet{Crysmann:Bonami:2016}.

\begin{exe}
  \ex\label{ex-MOP}
\isi{Morph Ordering Principle (MOP)}:
\begin{xlist}
\ex Concatenation:\\
\avm{
\emph{word} \impl
	[ph & \1 \+ \ldots\ \+ \tag{n} \\
	infl &	[mph &	<[ph & \1], \ldots, [ph & \tag{n}]> ] ]
}
\ex Order:\\
\avm{
\emph{word} \impl $\neg$
	([infl &	[mph & < \ldots\ [pc & \tag{m}], [pc & \tag{n}], \ldots> ] ] $\land$ \tag{m} $\geq$ \tag{n})
}
\zl

While the first clause in (\ref{ex-MOP}a) merely states that the
word's phonology is the concatenation of its constituent morphs, the
second clause (\ref{ex-MOP}b) ensures that the order implied by position class indices
(\textsc{pc}) is actually obeyed. \citet{bonami_o-crysmann_b13hpsg}
provide a formalisation of morph ordering using list constraints.


Given the very general nature of the well-formedness constraints and
particularly the commitment to monotonicity embodied by
(\ref{ex-MCC}), it is clear that most if not all of the actual
morphological analysis will take place at the level of realisation
rules.


\subsection{Realisation rules}

The fact that IbM, in contrast to PFM or AM, recognises $m:n$
relations between form and function at the most basic level of
organisation, i.e.\ realisation rules, means that morphological
generalisations can be expressed in a single place, name\-ly simply as
abstractions over rules. Rules in IbM are represented as descriptions
of typed feature structures organised in an inheritance hierarchy,
such that properties common to leaf types can be abstracted out into
more general supertypes. This vertical abstraction is illustrated in
Figure~\ref{fig:Vertical}. Using again \ili{German} past participles as an
example, the commonalities that regular circumfixal \textit{ge-...-t}
(as in \textit{gesetzt} `put') shares with subregular
\textit{ge-...-en} (as in \textit{geschrieben} `written') can be
generalised as the properties of a rule supertype from which the more
specific leaves inherit. Note that essentially all information except
choice of suffixal shape is associated with the supertype. This
includes the shared morphotactics of the suffix.

\begin{figure}
	\centering
%    \hspace*{-1em} 
\begin{forest}
%\psset{levelsep=8em,treesep=12em,linewidth=0pt,nodesep=0pt}
[%
\avm{
	[mud &	\{[tam & ppp]\} \\
	mph &	<[ph & ge \\
			pc & -1],
			[pc & 1]> ]
}
	[%
	\avm{
		[mph & < \ldots\ ,	[ph & t] > ]
	}]
	[%
	\avm{
		[mph & < \ldots\ ,	[ph & en] > ]
	}]
]
\end{forest}
	\caption{Vertical abstraction by inheritance}\label{fig:Vertical}
\end{figure}

In addition to vertical abstraction by means of standard monotonic
inheritance hierarchies, IbM draws on Online Type Construction
\citep{Koenig94}: using dynamic cross-classification, leaf types from
one dimension are distributed over the leaf types of another
dimension. This type of horizontal abstractions permits modelling of
systematic alternations, as illustrated once more with \ili{German} past
participle formation:

\begin{exe}
  \ex \label{ex:ppp}
  \begin{xlist}

    \ex \textit{ge}-setz-\textit{t} `put'
    \ex über-setz-\textit{t} `translated'
    \ex \textit{ge}-schrieb-\textit{en} `written'
    \ex über-schrieb-\textit{en} `overwritten'
  \end{xlist}
\end{exe}

In the more complete set of past participle formations shown in
(\ref{ex:ppp}), we find alternation not only between choice of suffix
shape (\textit{-t} vs.\ \textit{-en}), but also between presence
vs.\ absence of the prefixal part (\textit{ge-}).

\begin{figure}
  \centering
  \footnotesize
\begin{forest}
[%
\avm{
	[mud &	\{[tam & ppp]\} \\
	mph & < \ldots\  [pc & 1] > ]
}
	[pref,partition
		[%
		\avm{
			[mph & <[ph & ge \\
					pc & -1], ![ ]!> ]
		}
		]
		[%
		\avm{
			[mph & <![ ]!> ]
		}
		]
	]
	[suff,partition
		[%
		\avm{
			[mph & < \ldots  [ph & t]> ]
		}
		]
		[%
		\avm{
			[mph & < \ldots  [ph & en]> ]
		}
		]
	]
]
\end{forest}
  \caption{Horizontal abstraction by dynamic cross-classification}\label{fig:Horizontal}
\end{figure}

Figure~\ref{fig:Horizontal} shows how Online Type Construction provides
a means to generalise these patterns in a straightforward way: while the
common supertype still captures properties true of all four different
realisations \emdash namely the property to be expressed and the fact that it
involves at least a suffix \emdash, concrete prefixal and suffixal realisation
patterns are segregated into dimensions of their own (indicated by
\fbox{\textsc{pref}} and \fbox{\textsc{suff}}). Systematic cross-classification
(under unification) of types in \fbox{\textsc{pref}} with those in
\fbox{\textsc{suff}} yields the set of well-formed rule instances,
e.g.\ distributing the left-hand rule type in \fbox{\textsc{pref}} over the
types in \fbox{\textsc{suff}} yields the rules for \textit{ge-setz-t} and
\textit{ge-schrieb-en}, whereas distributing the right hand rule type
in \fbox{\textsc{pref}} gives us the rules for \textit{über-setz-t} and
\textit{über-schrieb-en}, which are characterised by the absence of
the participial prefix.

Having illustrated how the kind of dynamic cross-classification
offered by Online Type Construction is highly useful for the analysis
of systematic alternation in morphology, it seems necessary to lay out
in a more precise fashion its exact workings. In its original
formulation by \citet{Koenig94,Koenig99}, Online Type Construction was conceived as a
closure operation on underspecified lexical type hierarchies. IbM
merely redeploys their approach for the purposes of inflectional
morphology. Essentially, a minimal type hierarchy as in
Figure~\ref{fig:Horizontal} provides instructions on the set of
inferrable subtypes: according to \citet{Koenig94}, dimensions are
conjunctive and leaf types are disjunctive. Online Type Construction
dictates that any maximal subtype must inherit from exactly one leaf
type in each dimension. The maximal types of the hierarchy thus
expanded serve as the basis for rule instances, i.e.\ actual
rules.\footnote{\label{fn:OTC}There are two ways of conceptualising the status of
  Online Type Construction in grammar: under the dynamic view, hierarchies are
  underspecified and the full range of admissible types and therefore
  the range of instances is inferred online. Under the more
  conservative static view, the underspecified description is merely a
  convenient shortcut for the grammar writer. In either case,
  generalisations are preserved.   
   }



\subsection{Pāṇinian competition}
\label{sec:Panini}

In accordance with most theories of inflection
\citep{Prince93,Stump01,Anderson92,Noyer92,kiparsky_p85}, IbM embraces
a version of Morphological Blocking, also known as the Elsewhere Condition
\citep{kiparsky_p85} or Pāṇini's Principle. The basic intuition behind
Pāṇinian competition is that more specific rules can block the
application of more general rules, where the most unspecific rule will
count as a default. In terms of feature logic, the notion of
specificity corresponds to some version of the subsumption relation.

Competition between rules or lexical entries does not follow from the
logic standardly assumed within HPSG: if a rule can apply, it will
apply, no matter whether there are any more specific or more general
rules that could have applied as well (in fact, they would apply as
well). Thus, implementation of a notion of morphological blocking
necessitates a change to the logic.

As has been discussed already in \citet{Koenig99}, preemption based on
specificity of information can be either addressed statically (at
``compile-time'') as an issue of knowledge representation or
dynamically (at ``run-time'') as a question of knowledge
use. Independently of the choice between a static or dynamic version
of preemption, the main task is to provide a notion of competitor.  In
the interest of representing Pāṇinian inferences transparently in the
type hierarchy, IbM makes use of a closure operation on rule
instances, as detailed in (\ref{ex:PAN}), which is clearly inspired by
\citet{Koenig99} and \citet{Erjavec94}.\footnote{Alternatively, for 
a dynamic approach, it will be sufficient to use clause
(\ref{ex:PAN}a) and perform a topological sort on rule instances,
ordering  more specific rules before more general ones.}

\begin{samepage}
  \begin{exe}
    \ex \label{ex:PAN}\begin{description}
    \item[Pāṇinian Competition (PAN)] \mbox{~}
      \begin{xlist}
        \ex For any leaf type $t_1$[\textsc{mud} $\mu_1$,\textsc{ms}
        $\sigma$], $t_2$[\textsc{mud} $\mu_2$,\textsc{ms}
        $\sigma \wedge \tau$] is a morphological competitor, iff
        $\mu_1 \cup \mbox{\textit{set}} \sqsubseteq \mu_2 \cup \mbox{\textit{set}}$.
    
        \ex For any leaf type $t_1$ with competitor $t_2$, expand
        $t_1$'s \textsc{ms} $\sigma$ with the negation of $t_2$'s
        \textsc{ms} $\sigma \wedge \tau$: i.e.
        $\sigma \wedge \neg(\sigma \wedge \tau)$ which is equivalent
        to $\sigma \wedge
        \neg \tau$.
      \end{xlist}
    \end{description}
  \end{exe}
\end{samepage}

The first clause establishes competition, ensuring subsumption with
respect to both expressed features (\textsc{mud}) and conditioning
features (\textsc{ms} descriptions).\footnote{Since \textsc{mud} values
  can be of different cardinality, the subsumption  is checked
  on  open sets containing the original \textsc{mud} sets. } If the
condition in (\ref{ex:PAN}a) is met, the use
conditions of the more general rule are specialised in such a way (\ref{ex:PAN}b) as
to make the two rule descriptions fully disjoint.

% An example is needed here!

\begin{table}
\setlength{\tabcolsep}{.3em}
\centering
\begin{tabular}{llllll}
\lsptoprule 
 & \textsc{pos} & \textsc{neg}     &             & \textsc{pos} & \textsc{neg}\\
\midrule 
\textsc{1sg} & ni-{ta}-tak-{a} & \textbf{{si}}-{ta}-tak-{a}        & \textsc{1pl} & tu-{ta}-tak-{a}     & {ha}-tu-{ta}-tak-{a}\\
\textsc{2sg} & u-{ta}-tak-{a} & {ha}-u-{ta}-tak-{a}        & \textsc{2pl} & m-{ta}-tak-{a}      & {ha}-m-{ta}-tak-{a}\\
\textsc{3sg.m/wa} & a-{ta}-tak-{a} & {ha}-a-{ta}-tak-{a}    & \textsc{3pl.m/wa} & wa-{ta}-tak-{a} & {ha}-wa-{ta}-tak-{a}\\ 
\textsc{3sg.ki/vi} & ki-{ta}-tak-{a} & {ha}-ki-{ta}-tak-{a} & \textsc{3pl.ki/vi} & vi-{ta}-tak-{a} & {ha}-vi-{ta}-tak-{a}\\
etc. & &\\
\lspbottomrule
\end{tabular}
\caption{Future forms of the Swahili verb  \textit{taka} `want'}
\label{tab:SwahiliPortmanteau}
\end{table}

For concreteness, let us consider some examples from \ili{Swahili}: as shown
in Table~\ref{tab:SwahiliPortmanteau}, the negative in \ili{Swahili} is
typically formed by a prefix \textit{ha-}, preceding the equally
prefixal exponents of
subject agreement and tense (future \textit{ta-}). However, in the
negative first singular, discrete realisation of \textit{ha-} and
\textit{ni-} is blocked by the portmanteau \textit{si-}. Here, we have
a classical case of   Pāṇinian competition, where a rule that
expresses both negative and first person singular agreement preempts
application of the more general individual rules for negative or first
person singular.  



In the case of \textit{si}, we find the portmanteau in the same
surface position as the exponents it is in competition with.  However,
this need not be the case, nor indeed is preemption of this kind
limited to adjacency. Relative negative \textit{si-}, for instance, is
realised in a position following the subject agreement marker, yet
still, by virtue of expressing negative in the context of relative
marking, it blocks realisation of negative \textit{ha-} in
pre-agreement position. This constitutes a case of what
\citet{Noyer92} calls ``discontinuous bleeding''.

\begin{exe}
  \ex \label{ex:SwahiliNeg1sg}
  \begin{xlist}
    \ex[]{\gll {ha}- wa- ta- taka\\
      \textsc{neg} \textsc{sbj.pl.m/wa} \textsc{fut} want\\
      \glt `they will not want'}
    \ex[]{\gll watu wa- \textit{si-} o- soma\\
      people \textsc{sbj.pl.m/wa} \textsc{neg.rel} \textsc{rel.pl.m/wa} read\\
      \glt `people who don't read'}
    
    \ex[*]{\gll watu {ha-} wa- \textit{(si-)} o- soma\\
      people \textsc{neg} \textsc{sbj.pl.m/wa} \textsc{neg.rel} \textsc{rel.pl.m/wa} read\\
      \glt }
    
  \end{xlist}
\end{exe}

The relevant realisation rules for \textit{ha-}, \textit{ni-}, and the
two markers \textit{si-}, can be formulated quite straightforwardly as
in (\ref{ex:SwahiliPanini}a--d). For expository
purposes, I shall make explicit the fact that \textsc{mud} is necessarily
contained in \textsc{ms}.   

\begin{exe}
  \ex \label{ex:SwahiliPanini}
  \begin{minipage}[t]{0.4\linewidth} \begin{xlist}
      \exi{a.
      	\avm{
          [mud & \1 \{\type{neg}\} \\
          ms & \1 $~\cup$ set \\
          mph & <[ph & <\normalfont ha> \\
		  pc & $1$]> ]
		}}
    \end{xlist}
  \end{minipage} ~ \begin{minipage}[t]{0.4\linewidth} \begin{xlist}
        \exi{b.
        	\avm{
            	[mud & \1 \{[\type*{subj}
            				per & 1 \\
            				num & sg]\} \\
            	ms & \1 $~\cup$ set \\
            	mph & <[ph & <\normalfont ni> \\
            	pc & 2]> ]
			}}
    \end{xlist}
  \end{minipage}


\begin{minipage}[t]{0.4\linewidth} \begin{xlist}
      \exi{c.
      	\avm{
          [mud & \1	\{\type*{neg,}
          			[\type*{subj}
              		per & 1 \\
              		num & sg]\} \\
          mph & <[ph & <\normalfont si> \\
            	pc & $1..2$] > ]
		}}
    \end{xlist}
  \end{minipage} ~ \begin{minipage}[t]{0.5\linewidth} \begin{xlist}
      \exi{d.
      	\avm{
          [mud &  \1 \{\type{neg}\} \\
          ms & \1 \cupAVM \{\type{rel}\} $~\cup$ set\\
          mph & <[ph & <\normalfont si> \\
          		pc & $3$] > ]
		}}
    \end{xlist}
  \end{minipage}

\end{exe}

On the basis of the definition in (\ref{ex:PAN}a), portmanteau
\textit{si} in (\ref{ex:SwahiliPanini}c)\footnote{IbM uses the
  notation $m..n$ to represent spans of position classes. See
  \citet{bonami_o-crysmann_b13hpsg} for a proposal of how spans can be made explicit.} is a competitor for both
\textit{ni-} (\ref{ex:SwahiliPanini}b) and \textit{ha-}
(\ref{ex:SwahiliPanini}c), since the \textsc{mud} of portmanteau
\textit{si-} expands, i.e.~ is subsumed by each of the sets containing the \textsc{mud}
value of \textit{ni-} or \textit{ha-}. Moreover, the \textsc{ms} value
of portmanteau \textit{si-} is properly subsumed by \textit{ni-} (and
\textit{ha-}). Accordingly, the rule for \textit{ni-} will be expanded
as in (\ref{ex:PanExp}a).  Similarly, in a first iteration,
\textit{ha-} will be specialised as in (\ref{ex:PanExp}b). 

\begin{exe}
  \ex \label{ex:PanExp}
  \begin{minipage}[t]{0.44\linewidth}
    \begin{xlist}
      \exi{a.
		\avm{
          [mud & \1 \{[\type*{subj}
           			per & 1 \\
           			num & sg]\} \\
           ms & \1 \cupAVM set $\land \neg$ \{\type{neg}, \ldots\} \\
           mph & <[ph & <\normalfont ni> \\
              	pc & 2]> ]
		}}
    \end{xlist}
  \end{minipage}
  \begin{minipage}[t]{0.55\linewidth}
    \begin{xlist}
      \exi{b.
      	\avm{
          [mud & \1 \{\type{neg}\} \\
          ms & \1 \cupAVM  set $\land\neg$ \{[\type*{subj}
              					per & 1 \\
              					num & sg], \ldots \} \\
          mph & <[ph & <\normalfont ha> \\
             	 pc & 1]> ]
		}}
    \end{xlist}
  \end{minipage}
\end{exe}

However, \textit{ha-} (\ref{ex:SwahiliPanini}a) has another competitor,
namely negative relative \textit{si-} (\ref{ex:SwahiliPanini}d): while
in this case the \textsc{mud} values are equally informative, the
rules differ in terms of their \textsc{ms} descriptions, with
\textit{si-} being conditioned on relative and \textit{ha-} being
unconditioned.  Expansion by Pāṇinian competition will add another
existential constraint to (\ref{ex:PanExp}b). The fully expanded entry is given in
(\ref{ex:PanExpHa}).

\begin{exe}
  \ex \label{ex:PanExpHa}
\avm{
    [mud & \1 \{\type{neg}\} \\
    ms & \1 \cupAVM set $\land \neg$ \{[\type*{subj}
        				per & 1 \\
        				num & sg], \ldots\}
        				$\land \neg$ \{\type{rel}, \ldots\} \\
    mph & <[ph & <\normalfont ha> \\
        	pc & 1]> ]
}    
  
\end{exe}

% Zero exponence
A common case of default realisation is zero exponence: as illustrated
by the \ili{German} nominal paradigms in Table~\ref{tab:ParaSplit}, only a
small number of the cells feature overt exponents. For example in the paradigm
of \textit{Oma} `granny' (Table~\ref{tab:ParaSplit}a), singular number
is solely expressed by the significant absence of any
exponents. Particularly relevant to the case of default zero
realisation are the paradigms exhibiting a Pāṇinian split, e.g.\ that
of \textit{Rechner} `computer': here, only two cells are actually
marked with a specific exponent (genitive singular and dative plural),
all others are zero-marked and receive their interpretation by means
of paradigmatic contrast. In order to allow for the possibility of
zero realisation and to lend it the status of an ultimate default in
the absence of any overt realisation, realisational approaches such as
AM and PFM assume that every rule block returns an unmodified base,
unless preempted by a more specific rule. In PFM, this property is
ensured by the Identity Function Default (IFD)
\citep[][53]{Stump01}. Having a default principle, such as the IFD, is
economical in that it saves restating the identity function for every
rule block. On the downside, the IFD as a meta-level default 
will always be able to apply, possibly making an account of gaps in
paradigms more difficult. In IbM, zero exponence is captured by
providing a rule type that contributes an empty list of morphs, as
shown in Figure~\ref{fig:ifd} below. With an underspecified \textsc{mud}
value, such a rule type may act as a default realisation.

One assertion that has been made repeatedly in IbM work concerns
default zero exponence, the thesis being that there is need for only a
single instance. The current formulation of Pāṇini's principle works
as desired within an inflectional dimension, e.g.\ tense or polarity,
but not for a rule that has a fully underspecified \textsc{mud}
element, since such a rule would only be applicable if neither tense
nor polarity had a non-default value.  The rule for zero exponence
suggested in \citet{Crysmann:Bonami:2016}, for example, realises a
property (one underspecified element on \textsc{mud}) without
contributing any morph, as shown in Figure~\ref{fig:ifd}.



\begin{figure} 

  \begin{subfigure}{.3\textwidth}
\centering
    \avm{
      [mud & \{![ ]!\} \\
      mph & < >]
    }
  \caption{Simple type}
  \end{subfigure}
  ~
  \begin{subfigure}{.5\textwidth}
\centering
    \begin{forest}
    [%
    \avm{
    	[mud & \{![ ]!\} \\
      	mph & < >]
	}
	    [%
	    \avm{
	      [mud & \{\type{tns}\}]
		}
	    ]
	    [%
	    \avm{
	      [mud & \{\type{pol}\}]
		}
	    ]
    ]
  \end{forest}
  \caption{Simple type with more specific subtypes}
\end{subfigure}

\caption{Default zero realisation}
\label{fig:ifd}
\end{figure}

A simple solution is to provide subtypes of the ultimate default for
every inflectional dimension that witnesses zero exponence: the rule
type in Figure~\ref{fig:ifd}a, for
instance,  could be specialised by
adding appropriate subtypes, e.g.\ for tense and polarity, as in Figure~\ref{fig:ifd}b.
While this is slightly less general than what might have been hoped
for, the finer control that this move provides is independently
required to strike the right analytical balance between zero exponence
as a fallback strategy and the existence of defectiveness, i.e.\ gaps
in paradigms.

Having seen how Pāṇinian competition can be made explicit, we shall
briefly have a look at how this global principle interacts with
multiple and overlapping exponence. 

Let us start with overlapping exponence, which is much more common
than pure multiple exponence. As witnessed by the \ili{Swahili} examples in  
(\ref{ex:SwahiliNegFut}) and (\ref{ex:SwahiliNegPast}), the regular
exponent of negation combines with tense markers for past and
future. However, while the exponent for future is constant across
affirmative and negative (\ref{ex:SwahiliNegFut}), the negative past
marker \textit{ku-} in (\ref{ex:SwahiliNegPast}) displays overlapping
exponence. 

\begin{exe}
%  \avmoptions{top}
  \ex  \label{ex:SwahiliNegFut}
  \begin{xlist}
    \ex[]{\gll tu- {ta}- taka\\
      \textsc{1pl} \textsc{fut} want\\
      \glt `we will want'}
    \ex[]{ \gll {ha}- tu- {ta}- taka\\
      \textsc{neg} \textsc{1pl} \textsc{fut} want\\
      \glt `we will not want'}
  \end{xlist}
  \ex \label{ex:SwahiliNegPast}
  \begin{xlist}
    \ex[]{\gll tu- li- taka\\
      \textsc{1pl} \textsc{pst} want\\
      \glt `we wanted'}
    \ex[]{\gll *({ha}-) tu- {ku}- taka\\
      \textsc{neg} \textsc{1pl} \textsc{pst.neg} want\\
      \glt `we did not want'}
  \end{xlist}

\end{exe}

There are, in principle, two ways to picture cases of overlapping
exponence as in (\ref{ex:SwahiliNegPast}b): either \textit{ku-} is
regarded as cumulation of negative and past, or else it is an exponent
of past, allomorphically conditioned by the negative.  Following
\citet{Carstairs87}, IbM embraces a notion of inflectional allomorphy
by way of distinguishing between expression of a feature and
conditioning by some feature.

\begin{exe}
  \ex \label{fig:SwahiliFut}
  \begin{xlist}
    \ex 
\avm{
	[mud & \{\type{past}\}\\
        mph & < [ph & li\\
          pc & $3$]> ]
}    \ex
\avm{
      [mud & \{\type{past}\}\\
        ms & \{\type{neg}\} \cupAVM set\\
        mph & < [ph & ku\\
          pc & $3$]> ]
}
    
  \end{xlist}
\end{exe}

We can provide rules for the two past markers as given in
(\ref{fig:SwahiliFut}), where \textit{ku-} is additionally conditioned
on the presence of \textit{neg} in the morphosyntactic property set
(\textsc{ms}). While these two rules stand in Pāṇinian competition
with each other, rule (\ref{fig:SwahiliFut}b) is crucially no
competitor for the regular negative marker \textit{ha-}, since the
\textsc{mud} sets of (\ref{fig:SwahiliFut}b) and (\ref{ex:PanExp}a)
are actually disjoint. Thus, by embracing a distinction between
expression and conditioning, overlapping exponence behaves as expected
with respect to Pāṇini's principle.


Pure multiple exponence works somewhat differently from
overlapping exponence: in Nyanja \citep{Stump01,Crysmann:14:OUP},
class B adjectives, such as \textit{kulu} `large' in (\ref{ex:NyanjaQualConc}) take two class markers to
mark agreement with the head noun, one set of markers being the one
normally used with class A adjectives, such as \textit{bwino} `good' in (\ref{ex:NyanjaQual}), the other being attested with
verbs, such as \textit{kula} `grow' (\ref{ex:NyanjaConc}). Both sets distinguish the same
properties, i.e.\ nominal class.\label{Nyanja}\footnote{The examples in (\ref{ex:Nyanja}) are taken from \citew[6]{Stump01}.}

\eal
\label{ex:Nyanja}
\ex\label{ex:NyanjaQualConc}
\gll ci-pewa ca-ci-kulu\\
    \textsc{cl7}-hat(7/8) \textsc{qual7}-\textsc{conc7}-large\\
\glt `a large hat'
\ex
\label{ex:NyanjaQual} 
\gll ci-manga ca-bwino\\
    \textsc{cl7}-maize \textsc{qual7}-good\\
\glt `good maize'
\ex
\label{ex:NyanjaConc} 
\gll ci-lombo ci-kula.\\
    \textsc{cl7}-weed \textsc{conc7}-grow\\
\glt `A weed grows.'
\zl

\citet{Crysmann:14:OUP} shows that double inflection as in Nyanja
can be captured by composing rules of exponence for verbs and type A
adjectives to yield the complex rules for type B adjectives, as shown
in Figure~\ref{fig:Nyanja}. 

\begin{sidewaysfigure}
  \centering
\scalebox{.9}{
\begin{forest}for tree={l=2cm}
[\emph{realisation-rule}
	[morphotactics,partition, name=morph
		[%
		\avm{
			[
			mud & \{\type{agr}\}\\
			ms & \{ [\type*{pid}
			cat & verb], \ldots\}\\
			mph & < [pc & $-4$]> ]
		}, name= 1, tier=word]
		[%
		\avm{
			[mud & \{\type{agr}\}\\
			 ms & \{ [\type*{pid}
			cat & [\type*{adj}
			type & A]], \ldots\}\\
			mph & < [pc & --1]>]
		}, name=2, tier=word]
	]
	[exponence,partition
		[qual,partition, name=qual
			[%
			\avm{
				[mud & \{ [\type{agr}\\
						cl & 7]\}\\
				mph & < \ldots [ph & <\normalfont ca>\\
				pc & $-1 \lor -2$
				] \ldots >]
			}, tier=word]
			[\dots]
		]
		[conc,partition, name=conc
			[%
			\avm{
				[mud & \{ [\type{agr}
						\\cl & 7]\}\\
				mph & < \ldots [ph & <\normalfont ci>\\
				pc & $-1 \lor -4$
				] \ldots >]
			}, tier=word]
			[\dots]
		]
	]
]
%% sorry this is an ugly workaround but even on stackexchange, no one could help me :/
\node[below of=morph, node distance= 9cm](3){%
	\avm{
			[mud & \{\type{agr}\}\\
			ms & \{ [\type*{pid}
			cat & [\type*{adj}
			type & B]], \ldots\}\\
			mph & <[pc & $-2$], [pc & $-1$]>]
	}%
};
\draw (1.north) to (qual.south);
\draw (2.north) to (conc.south);
\draw (3.north) to (morph.south);
\end{forest}
}    
  \caption{Nyanja pre-prefixation \citep[210]{Crysmann:14:OUP}}\label{fig:Nyanja}
\end{sidewaysfigure}

The difference in treatment for overlapping and pure multiple
exponence of course raises the question whether or not the approaches
should be harmonised. The only way to do this would be to generalise
the Nyanja case to overlapping exponence, by way of treating all such
cases by means of composing rules. While possible in general, there is
a clear downside to such a move: as we saw in the discussion of
\ili{Swahili} above, there is not only a dependency between negative and
past tense, but also between negative \textit{si-} and relative
marking. As a result, one would end up organising negation, tense and
relative marking into a single cross-cutting multi-dimensional type
hierarchy. Inflectional allomorphy by contrast supports a much more
modularised perspective which greatly simplifies specification of the
grammar.  

\subsection{Morphotactics}
\label{sec:Mortax}

The treatment of morphotactically complex systems, as found in e.g.\
position class systems, was one of the major motivations behind the
development of IbM. With the aim of providing a formal model of
complex morph ordering that matches the parsimony of the traditional
descriptive template, \citet{Crysmann:Bonami:2016} discarded the
cascaded rule model adopted by e.g.\ PFM
\citep{Stump01}.\footnote{\citet{Crysmann12} was a conservative
  extension of PFM with reified position class indices, an approach
  that was rendered obsolete by subsequent work.} Instead, order is
directly represented as a property of exponents.

Taking as a starting point the classical challenges from
\citet{Stump93} \emdash portmanteau, ambifixal, reversed, and parallel
position classes \emdash, they developed an extended typology of variable
morphotactics, i.e.\ systems, which depart from the kind of rigid
ordering more commonly found in morphological systems.

\begin{table}[ht!]
\centering
    \begin{tabular}{lll}
      \lsptoprule
      & \textsc{present} & \textsc{future}\\
      \midrule
      1 & birsã-\textbf{tʃ\textsuperscript{h}a}-\emph{aũ} & 
                                                                birse-\emph{aũ}-\textbf{lā}\\
      \textsc{2.low} &
                 birsã-\textbf{tʃ\textsuperscript{h}a}-\emph{s} & 
                                                                      birse-\textbf{lā}-\emph{s}\\
      \textsc{2.mid} & 
                  birsã-\textbf{tʃ\textsuperscript{h}a} & 
                                                         birse-\textbf{lā}\\
      \textsc{3.low} & 
                  birsã-\textbf{tʃ\textsuperscript{h}a}-\emph{au} & 
                                                                        birse-\emph{au}-\textbf{lā}\\
      \textsc{3.mid} &
                 birsã-\textbf{tʃ\textsuperscript{h}a}-\emph{n} & 
                                                                      birse-\textbf{lā}-\emph{n}\\
      \lspbottomrule
    \end{tabular}
\caption{Masculine singular forms of the Nepali verb \textsc{birsanu} `forget'}
\label{tab:Nepali}
\end{table}

One of the most simple deviations from strict and invariable ordering
is misaligned placement: while exponents marking alternative values
for the same feature, and therefore stand in paradigmatic opposition,
tend to occur in the same position, this is not always the case, as
illustrated by the example from \ili{Nepali} in Table~\ref{tab:Nepali}.  
While the agreement markers (in italics) follow the tense marker (bold) in the present,
the relative order of tense and agreement marker differs from cell to
cell in the future (\textsc{low} and \textsc{mid} constitute
different levels in the system of honorifics).

\begin{figure}
  \centering
%\scalebox{.73}
\oneline{
\begin{forest}
[%
\avm{
	[mud & \1\\
	ms & \1 \cupAVM set]
}
	[\avm{[mud & \{ \type{tense} \}]}
		[%
		\avm{
			[mud & \{ \type{present} \}\\
			mph & < [ph & <\normalfont tʃ$^{\normalfont h}$a>\\
			pc & $1$] > ]
		}]
		[%
		\avm{
			[mud & \{\type{future}  \}\\
			mph & < [ph & <\normalfont lā >\\
			pc & $3$] > ]
		}]
	]
	[\avm{[mud & \{ \type{agr} \} ]}
		[\avm{[mph & < [pc & $2$] > ]}, l=4cm
			[%
			\avm{
				[mud & \{ [per & 1 ]\}\\
				mph & < [ph & <\normalfont aũ >] > ]
			}]
			[%
			\avm{
				[mud & \{ [per & 3\\
				hon & low ] \}\\
				mph & < [ph & <\normalfont au >] > ]
			}]
		]
		[\avm{[mph & < [pc & $4$] > ]}, l=4cm
			[%
			\avm{
				[mud & \{ [per & 2\\
				hon & low ]\}\\
				mph & < [ph  & <\normalfont s >] > ]
			}]
			[%
			\avm{
				[mud & \{ [per & 3\\
				hon & mid ] \}\\
				mph & <[ph & <\normalfont n >] > ]
			}]
		]
	]
]
\end{forest}
}
    \caption{Nepali tense and agreement marking}\label{fig:AnalysisNepali}
\end{figure}

If position class indices are part of the descriptive inventory, an
account of apparently reversed position classes \citep{Stump93}
becomes almost trivial, as shown in Figure~\ref{fig:AnalysisNepali}:
all it takes is to assign the present marker an index that precedes
all agreement markers and assign the future marker an index that
precedes some agreement markers, but not others.

A slightly more complex case is conditioned placement: in contrast to
misaligned placement, assignment of position does not just depend on
the properties expressed by the marker itself, but on some additional
property. An example of this is \ili{Swahili} ``ambifixal'' relative
marking, as shown in examples
(\ref{ex:SwahiliRel:suff})--(\ref{ex:SwahiliRel:pref}).\footnote{Conditioned
  placement is not only attested on alternate sides of the stem, as
  discussed for \ili{Swahili} in \citet{Stump93}, but also on the same
  side. See the discussion of mesoclisis in European Portuguese\il{Portuguese!European} in
  \citet{Crysmann:Bonami:2016}.} In the affirmative indefinite tense,
the relative marker is realised in a position after the stem, whereas
in all other cases it precedes it. 

\begin{exe}
  \ex\label{ex:SwahiliRel:suff}
  \begin{xlist}
    \ex\gll  a-soma-\textit{ye}\\
    \textsc{m/wa.sg}-read\textsc{-m/wa.rel}\\
    \glt `(person) who reads'
    \ex\gll a-ki-soma-\textit{cho}\\
    \textsc{m/wa.sg}-\textsc{ki/vi.o}-read-\textsc{ki/vi.rel}\\
    \glt `(book) which he reads'
  \end{xlist}
  \ex\label{ex:SwahiliRel:pref}
  \begin{xlist}
    \ex\gll  a-na-\textit{ye}-soma\\
    \textsc{m/wa.sg-pres-m/wa.rel}-read\\
    \glt `(person) who is reading'
    \ex\gll a-na-\textit{cho}-ki-soma\\
    \textsc{m/wa.sg-pres-ki/vi.rel-ki/vi.rel}-read\\
    \glt `(book) which he is reading'
  \end{xlist}
\end{exe}

Conditioned placement can be captured using a two-dimensional
hierarchy, as shown in Figure~\ref{fig:SwahiliRel}: the
\fbox{\textsc{morphotactics}} dimension on the left defines the conditions for
the corresponding placement constraints, whereas the \fbox{\textsc{exponence}}
dimension provides the constraints on the shape of the 16 relative
class markers that undergo the alternation.  Cross-classification by
means of Online Type Construction finally distributes the morphotactic
constraints over the rules of exponence.
  

\begin{figure}
  \centering
  
\oneline{
\begin{forest}
[%
\avm{
	[\type*{realisation-rule}
	mud & \1 set \\
	ms & \1 \cupAVM set \\
	mph & list]
}
	[morphotactics,partition
		[%
		\avm{
			[mud & \{\type{rel} \} \\
			mph & <[pc & $4$]> ]
		}
		]
		[%
		\avm{
			[mud & \{\type{rel} \} \\
			ms & \{\type{aff, def, \ldots}\} \\
			mph & <[pc & $7$]> ]
		}
		]
	]
	[exponence,partition
		[%
		\avm{
			[mud &	\{[\type*{rel}
					per & 3 \\
					num & sg \\
					cl & ki-vi ]\} \\
			mph &	<[ph & !<\normalfont cho>! ]> ]
		}
		]
		[%
		\avm{
			[mud &	\{[\type*{rel}
					per & 3 \\
					num & pl \\
					cl & ki-vi ]\} \\
			mph &	<[ph & !<\normalfont vyo>!]> ]
		} \ldots
		]
	]
]
\end{forest}
}
\caption{Swahili relative markers}\label{fig:SwahiliRel}
\end{figure}


The last basic type of variable morphotactics is free placement,
i.e.\ free permutation of a circumscribed number of
markers. This is attested e.g.\ in Chintang \citep{Bickel07} and in
Mari \citep{Luutonen97}.


\begin{table}
  \centering
  \begin{tabular}{llll}
    \lsptoprule
    & \textsc{absolute} & \multicolumn{2}{c}{\textsc{ 1pl possessed}}\\
    & & \textsc{poss} $\prec$ \textsc{case} &  \textsc{case} $\prec$ \textsc{poss} \\
    \midrule
    \textsc{nom} & 
    \tl{пӧрт}{pört} & 
    \multicolumn{2}{c}{\tl{пӧрт}{pört}-\textbf{\tl{на}{na}}}\\
    \textsc{gen} & 
    \tl{пӧрт}{pört}-\emph{\tl{ын}{ən}} & 
    \tl{пӧрт}{pört}-\textbf{\tl{на}{na}}-\emph{\tl{н}{n}}
    & *\\
    \textsc{acc} & 
    \tl{пӧрт}{pört}-\emph{\tl{ым}{əm}} & 
    \tl{пӧрт}{pört}-\textbf{\tl{на}{na}}-\emph{\tl{м}{m}}
    & *\\
    \textsc{dat} & 
    \tl{пӧрт}{pört}-\emph{\tl{лан}{lan}}  & 
    \tl{пӧрт}{pört}-\textbf{\tl{на}{na}}-\emph{\tl{лан}{lan}} & 
    \tl{пӧрт}{pört}-\emph{\tl{лан}{lan}}-\textbf{\tl{на}{na}}\\
    \textsc{lat} & 
    \tl{пӧрт}{pört}-\emph{\tl{еш}{eš}} &
    * &
    \tl{пӧрт}{pört}-\emph{\tl{еш}{eš}}-\textbf{\tl{на}{na}}\\
    \textsc{ill} & 
    \tl{пӧрт}{pört}-\emph{\tl{ышкӧ}{əš(kö)}} &
    * &
     \tl{пӧрт}{pört}-\emph{\tl{еш}{əškə}}-\textbf{\tl{на}{na}}\\
    % \sc iness & 
    % \tl{пӧрт}{pört}-\emph{\tl{ышкӧ}{əštö}} &
    % * &
    % \tl{пӧрт}{pört}-\emph{\tl{еш}{əštə}}-\textbf{\tl{на}{na}}\\
    % \sc comp & 
    % \tl{пӧрт}{pört}-\emph{\tl{ла}{la}}  & 
    % \tl{пӧрт}{pört}-\textbf{\tl{на}{na}}-\emph{\tl{ла}{la}} & 
    % \tl{пӧрт}{pört}-\emph{\tl{ла}{la}}-\textbf{\tl{на}{na}}\\
    % \sc comit & 
    % \tl{пӧрт}{pört}-\emph{\tl{ге}{ge}}  & 
    % \tl{пӧрт}{pört}-\textbf{\tl{на}{na}}-\emph{\tl{ге}{ge}} & 
    % *\\
    
    \lspbottomrule
  \end{tabular}
  
  \caption{Selected singular forms of the Mari noun \emph{pört} `house'}
  \label{tab:MariSingular}
\end{table}

While markers of core cases follow the possessive marker, and
exponents of the lative cases precede it, the dative marker permits
both relative orders. Free permutation appears to present a 
challenge for cascaded rule models, such as PFM, whereas an analysis is almost
trivial in IbM, as position can be underspecified. 

%\subsubsection*{Relative placement}
\medskip
Inflectional morphology does not provide much evidence for internal
structure. This is recognised in IbM by representing morphs on a flat
list with simple position class indices. While a simple indexing by
absolute position is often sufficient, there are cases where a more
sophisticated indexing scheme is called
for.

\citet{Crysmann:Bonami:2016} discuss placement of pronominal affix
clusters in \ili{Italian}. While placement is constant within the cluster of
pronominal affixes
itself, \textit{me-lo-} in the example below, as well as
between stem and tense and agreement affixes, the linearisation of the
cluster as a whole is variable, as shown by the alternation between
indicative and imperative in (\ref{ex:Italian}).

\begin{exe}
  \ex\label{ex:Italian}
  \begin{xlist}
    \ex\gll me- lo- da -te\\
    \textsc{dat.1sg} \textsc{acc.3sg.m} give[\textsc{prs}] \textsc{2pl}\\
    \glt `You give it to me.'
    \ex\gll da -te -me -lo!\\
    give[\textsc{imp}] \textsc{2pl} \textsc{dat.1sg} \textsc{acc.3sg.m}\\
    \glt `Give it to me!'
  \end{xlist}
\end{exe}


An important question raised by the \ili{Italian} facts is whether
morphotactics is in need of a more layered structure. If so, it will
certainly not be the kind of structure provided by stem-centric
cascaded rule approaches, like PFM, since it is the cluster that
alternates between pre-stem and post-stem position, not the individual
cluster members, which would yield mirroring.\footnote{See, however,
  \citet{Spencer05} for a variant of PFM that directly composes
  clusters.}

\citet{Crysmann:Bonami:2016} assume that it is the stem which is
mobile in \ili{Italian} and takes the exponents of tense and subject
agreement along. To implement this, they show that it is sufficient to
expose the positional index of the stem (the feature \textsc{stm-pc}
in Figure~\ref{fig:ItalianStem}), such that other markers can
be placed relative to this pivot (cf.~the agreement rule in Figure~\ref{fig:ItalianAff}).  

\begin{figure}\centering
  
%  \includegraphics[scale=.9]{figures/italian-stem-crop}

\begin{forest}
[\emph{realisation-rule}
	[%
	\avm{
		[mud &	\{[\type*{pid}
				stem & \0]\} \\
		mph &	<[stm-pc & \tag{s} \\
				pc & \tag{s}\\
				ph & \0]> ]
	}
		[%
		\avm{
			[ms & \{[\type{untensed, \ldots}]\} \\
			mph & <[pc & $1$]>]
		}
		]
		[%
		\avm{
			[ms & \{[\type{tensed, \ldots}]\} \\
			mph & <[pc & $9$]>]
		}
		]
	]
]
\end{forest}
  \caption{Partial hierarchy of Italian stem realisation rules}
  \label{fig:ItalianStem}
\end{figure}

\begin{figure}
  \centering
%  \includegraphics[scale=.84]{figures/italian-affix-crop}

\resizebox{\textwidth}{!}{
\begin{forest}
[\emph{realisation-rule}
	[%
	\avm{
		[mud & \{[\type{obj} ]\} \\
		mph & <[pc & $4$ ]> ]
	}
		[%
		\avm{
			[mud &	\{[per & $1$\\
					num & sg]\} \\
			mph & <[ph & \upshape me]> ]
		}
		]
		[\ldots]
	]
	[%
	\avm{
		[mud &	\{[\type*{dobj}
				per & $3$]\} \\
		mph & <[pc & $7$]> ]
	}
		[%
		\avm{
			[mud &	\{[num & sg \\
					gen & mas]\} \\
			mph & <[ph & \upshape lo]> ]
		}
		]
		[\ldots]
	]
	[%
	\avm{
		[mud & \{[\type{subj} ]\} \\
		mph &	<[stm-pc & \tag{s}\\
				pc & \tag{s} $\mathplus ~2$]> ]
	}
		[%
		\avm{
			[mud &	\{[per & $2$\\
					num & pl]\} \\
			mph & <[ph & \upshape te]> ]
		}
		]
		[\ldots]
	]
]
\end{forest}
}
  
  \caption{Partial hierarchy of Italian affixal realisation rules}
  \label{fig:ItalianAff}
\end{figure}



Compared to a layered structure, the pivot feature approach just
described appears to be more versatile, since it provides a suitable
solution to other cases of relative placement, such as second position
affixes. Sorani Kurdish\il{Kurdish!Sorani} endoclitic agreement markers surface after the
initial morph, be it the stem, or some prefixal marker
\citep{Samvelian07}. Thus, placement is relative to whatever happens
to be the first \textit{instantiated} position index.

\begin{table}
\centering
{\small
\begin{tabular}[t]{ccccccc|lr}
\lsptoprule
$1$ & $2$ & $3$ & $4$ & $5$ & $6$ & $7$ &\\
\textsc{neg} & & \textsc{ipfv} & & `send' & & \textsc{3pl}\\   
\midrule
    &&      && nard &=\textbf{jân} & im & ~ &{`they sent me'}\\
na & =\textbf{jân}     &  & & nard & & im & ~ &{`they did not send me'}\\
   & & da & =\textbf{jân} & nard & & im & ~ &{`they were sending me'}\\
na & =\textbf{jân}     &  da  & & nard & & im & ~ & {`they were not sending me'}\\
\lspbottomrule
\end{tabular}
}
\caption{Sorani Kurdish past person markers\label{table-kurdish}}
\label{tab:Sorani}
\end{table}

\citet{bonami_o-crysmann_b13hpsg} propose a pivot feature \textsc{1st-pc} that is
instantiated to the position class index of the first element on the
word's \textsc{mph} list and exposed on all other morphs by the
principle in (\ref{ex:stem-principle}).

\begin{exe}
  \ex\label{ex:stem-principle}
\avm{
\emph{word} \impl
	[infl &	[mph & <[pc & \1\\
        	1st-pc & \1\\
        	stm-pc & \tag{s} ]
        	,
      		[1st-pc & \1\\
        	stm-pc & \tag{s}] ,
    		\ldots ,
    		[1st-pc & \1\\
    		stm-pc &  \tag{s}]> ] ]
}
\end{exe}

What this principle does is distribute two critical position class
indices over every element of the \textsc{mph} list: one for the
position of the stem, in order to capture stem-relative vs.\ absolute placement as
in \ili{Italian}, the other for the lowest instantiated position class index,
to capture second position phenomena. 

The realisation rule for a second position clitic can then be
formulated  as
in (\ref{ex:jan}), determining its \textsc{pc} value relative to that of the
word's first morph. I use an arithmetic operator here as a convenient
shortcut, but note that indices are actually represented as lists 
underlyingly \citep{bonami-crysmann:2013}. 

\begin{exe}
  \ex \label{ex:jan}
  \avm{
    [mud &	\{[per & 3\\
        	num & pl ]\}\\
     mph &	<[ph & <\normalfont jân>\\
        	1st-pc & \1\\
     		pc & \1 $\mathplus ~1$ ]> ]
}
\end{exe}

For illustration, consider the two word forms \textit{nard=jân-im}
`they sent me' and \textit{da=jân-nard-im} `they were sending me' from
Table~\ref{tab:Sorani}. The first one consists of two positionally
fixed morphs, the stem in position 5 and the person ending in
position~7. According to (\ref{ex:stem-principle}), \textsc{1st-pc}
will be token identical to the \textsc{pc} of \textit{nard}, so
\textit{=jân} will be assigned a \textsc{pc} value of 6. The second
example \textit{da=jân-nard-im} has the additional progressive prefix
\textit{da-} in position~3, which is the lowest \textsc{pc} index of the word. Accordingly
\textit{=jân} is placed relative to the prefix \textit{da-}, in position~4.

To conclude the section, a more general remark is in order: as we have
seen, IbM uses explicit position indices to constrain morphotactic
position. In essence, these correspond to linear distribution classes,
where higher indices are realised to the right of lower indices
and no two morphs within a word may bear the same index, resulting in
competition for linear position. As a consequence, there is no static
notion of a slot: while morphs are ordered according to indices, there
is no requirement for indices to be consecutive. Thus, nothing much
needs to be said about empty slots, except that there happens to be no
morph in the word with that particular positional index. 


\subsection{Constructional vs.\ generative views}
\label{sec:ConstrGen}

IbM departs from previous, purely word-based approaches, such as
\citet{Blevins14} or, within HPSG, \citet[Section~5.2.2]{Koenig99} by recognising an
intermediate level of realisation rules that effects the actual $m:n$
relations between form and function. In this section, I shall discuss
how this facilitates partial generalisations over gestalt exponence,
provides for a better reuse of resources, as witnessed by parallel
inflection, and finally ensures a modular organisation of rules of
exponence.

\subsubsection{Gestalt exponence}

One of the strongest arguments for the word-based view and against a
generative rule-based approach comes from so-called gestalt exponence
in \ili{Estonian} \citep{Blevins05}. As shown in Table~\ref{tab:Estonian},
core cases in this language give rise to case/number paradigms where
(almost) all cells are properly distinguished by clearly segmentable
markers, yet there is no straightforward association between the
markers and the properties they express.   


\begin{table}
%  \centering
  \small
  %\setlength{\tabcolsep}{.7\tabcolsep}
\begin{tabular}{l|ll}
 \multicolumn{3}{c}{\textit{nokk} `beak'} \\
\lsptoprule
  & \textsc{sg} & \textsc{pl}  \\
  \midrule
  \textsc{nom} & nokk & nok-a-d \\
  \textsc{gen} & nok-a & nokk-a-de\\
  \textsc{part} & nokk-a & nokk-a-sid\\
\lspbottomrule
\end{tabular}
\hspace{2em}
\begin{tabular}{l|ll} 
\multicolumn{3}{c}{\textit{õpik} `workbook'}\\
\lsptoprule
  & \textsc{sg} & \textsc{pl}  \\
  \midrule
  \textsc{nom}  & õpik & õpik-u-d\\
  \textsc{gen}  & õpik-u & õpik-u-te\\
  \textsc{part} & õpik-u-t & õpik-u-id \\
\lspbottomrule
\end{tabular}\hfill
\begin{tabular}{l|ll} 
\multicolumn{3}{c}{\textit{seminar} `seminar'}\\
\lsptoprule
  & \textsc{sg} & \textsc{pl}\\
  \midrule
  \textsc{nom}  & seminar & seminar-i-d\\
  \textsc{gen}  & seminar-i & seminar-i-de\\
  \textsc{part} & seminar-i & seminar-i-sid \\
\lspbottomrule
\end{tabular}
\hfill
  \caption{Partial paradigms exemplifying three Estonian noun declensions (core cases; \citealp[287]{Blevins:Ackerman:Malouf:2016})}
\label{tab:Estonian}
\end{table}

The gestalt nature of \ili{Estonian} case/number marking can be schematised
as in Figure~\ref{fig:Matthews}. 

\begin{figure}
  \centering
  \begin{tabular}{lcc}
    \rnode{u1}{{`beak'}} & \rnode{u2}{\textsc{gen}} & \rnode{u3}{\textsc{pl}}\\[2ex]
    \rnode{l1}{nokk} & \rnode{l2}{-a} & \rnode{l3}{-de}
  \end{tabular}
      \psset{angleA=-90,angleB=90,arm=0pt,linewidth=.5pt,nodesep=2pt}

      \ncdiag{u1}{l1} \ncdiag{u1}{l2} \ncdiag{u2}{l1} \ncdiag{u2}{l3}
      \ncdiag{u3}{l1} \ncdiag{u3}{l2} \ncdiag{u3}{l3}

      \caption{\emph{m}:\emph{n} relations in Estonian}
      \label{fig:Matthews}
\end{figure}

While it is clear that this kind of complex association between form
and function requires a constructional perspective, it is far from
evident that (i) this association has to be made at the level of the word
rather than at the level of $m:n$ rules and
(ii) that this therefore requires word-to-word correspondences in the sense of
\citet{Blevins05,Blevins14}. To the contrary, the system depicted in
Table~\ref{tab:Estonian} displays partial generalisations that are
hard to capture in a system such as Blevins': e.g.\ theme vowels are
found in all cells except the nominative singular, only the nominative
singular is monomorphic, all plural forms are tri-morphic, to name
just a few.  

In IbM, $m:n$ correspondences are established at the level of
realisation rules, and these realisation rules are organised into
(cross-classifying) type
hierarchies. \citet{Crysmann:Bonami:2017:HPSG} argue that this makes
it possible to extract the kind of partial generalisation noted in the
previous paragraph and represent them in a three-dimensional type
hierarchy that specifies constraints on stem selection independently
of theme-vowel introduction and suffixation. Using pre-typing,
idiosyncratic aspects can be contained, while more regular aspects,
such as theme vowel and stem selection, are taken care of by Online
Type Construction.

Furthermore, encapsulating gestalt exponence as a subsystem of
realisation rules has the added advantage that it does not spill over
into the rest of the \ili{Estonian} inflection system, which, as a
\ili{Finno-Ugric} language,  is highly agglutinative.


While it is straightforward to implement constructional analyses
within IbM, involving complex $m:n$ relations between form and
function, non"=constructional analyses are actually preferred whenever
possible, generally yielding much more parsomonious descriptions.

\subsubsection{Reuse of resources}

Reuse of resources constitutes a particularly strong argument
against over"=generalising to the constructional, or word-based, view: parallel
position classes are a case at hand, as exemplified in \ili{Swahili}
\citep{Stump93,Crysmann:Bonami:2016} or Choc"-taw \citep{broadwell:2017}. 

\begin{table}[hbt]
  \centering
  \small
  \begin{tabular}{llllllllll}
    \lsptoprule
    \textsc{per} & \textsc{gen} & \multicolumn{2}{l}{\textsc{subject}} &
                                                                         \multicolumn{2}{l}{\textsc{object}}\\% & \multicolumn{2}{l}{\textsc{relative}}\\
                 & & \textsc{sg} & \textsc{pl} & \textsc{sg} & \textsc{pl}\\%  & \textsc{sg} & \textsc{pl}\\
    \midrule
    1	&       & ni & tu  & ni & tu\\%  &  \\
    2	&       & u	 & m   & ku & wa\\%   \\   
    3	& \textsc{m/wa}  & a	 & wa  & m  & wa\\%   & ye & o\\
                 & \textsc{m/mi}	& u  & i   & u  & i \\%  & o & yo \\
                 & \textsc{ki/vi}	& ki & vi  & ki & vi \\% & cho & vyo \\
                 & \textsc{ji/ma} & li & ya  & li & ya \\%  & lo & yo\\
                 & \textsc{n/n} & i    & zi  & i  & zi \\%  & yo & zo\\
                 & \textsc{u}     & u  & --- & u  & --- \\% & o & ---\\
                 & \textsc{u/n}   & u  & zi  & u  & zi  \\% & o & zo\\
                 & \textsc{ku}    & ku & --- & ku & --- \\% & ko & ---\\
    \lspbottomrule
\end{tabular}
  \caption{Swahili person markers \citep[143]{Stump93}} \label{ex:SwaPer} 

\end{table}

Consider the paradigms of \ili{Swahili} subject and object agreement markers
in Table~\ref{ex:SwaPer}: as one can easily establish, agreement
markers draw largely on the same set of shapes. Grammatical function
is disambiguated mainly by position, with subject agreement placed 
to the left of tense markers, and object agreement to the right. 

Under a constructional approach, such as the word-based analysis in
\citet{Koenig99}, the generalisation about identity of shapes is
essentially lost, which is due to the fact that under this view,
markers that can potentially combine must be introduced in different
cross-classifying dimensions, e.g.\ one for subject marking in slot 2,
the other for object marking in slot 5. Likewise, in order to
distribute shape constraints over subject and object agreement, they
must constitute yet another cross-cutting dimension, but there is
simply no way in this set-up to enforce that every shape constraint
must be evaluated twice.   

\begin{figure*}
  \centering
  \smaller %\smaller
\begin{forest}
[\emph{realisation-rule}
	[shape,partition
		[%
		\avm{
			[ mph & \{[ph &  <\normalfont ni>]\} \\
			mud &	\{[per & 1\\
					num & sg]\} ]
		}, name=1
			[%
			\avm{
				[ mph & \{[ph &  <\normalfont ni> \\
						pc & 2]\} \\
				mud &	\{[\type*{subj}
						per & 1\\
						num & sg]\} ]
			}, name=a, edge=dashed, tier=word, l=2.5cm
			]
		]
		[%
		\avm{
			[ mph & \{[ph &  <\normalfont wa>]\} \\
			mud &	\{[per & 3 \\
					num & pl]\} ]
		}, name=2
			[%
			\avm{
				[ mph &	\{[ph & <\normalfont wa> \\
						pc & 5]\} \\
				mud &	\{[\type*{obj}
						per & 3 \\
						num & pl]\} ]
			}, name=b, edge=dashed, tier=word, l=2.5cm
			]
		]
	]
	[position,partition
		[%
		\avm{
			[ mph & \{[pc & 2]\} \\
			mud & \{subj\} ]
		}, name=3
			[%
			\avm{
				[ mph & \{[ph &  <\normalfont wa>\\ pc & 2]\}\\
				mud & \{ [\type*{subj} per & 3\\ num & pl] \} ]
			}, name=c, edge=dashed, tier=word, l=2.5cm
			]
		]
		[%
		\avm{
			[ mph & \{[pc & 5]\}\\
			mud & \{ obj \} ]
		}, name=4
			[%
			\avm{
				[ mph & \{[ph &  <\normalfont ni>\\ pc & 5]\}\\
				mud & \{[\type*{obj} per & 1\\ num & sg]\} ]
			}, name=d, edge=dashed, tier=word, l=2.5cm
			]
		]
	]
]
\draw[dashed] (1.south) to (d.north);
\draw[dashed] (2.south) to (c.north);
\draw[dashed] (3.south) to (a.north);
\draw[dashed] (4.south) to (b.north);
\end{forest}
  
  \caption{Rule type hierarchy for Swahili parallel position classes \citep[356]{Crysmann:Bonami:2016}}
  \label{fig:Swahili}
\end{figure*}

However, once we move from word-based statements to realisation rules,
the problem simply vanishes, since we are not trying to solve the
problems of parallelism of exponence and combination at the same time.
As illustrated in Figure~\ref{fig:Swahili}, constraints about shape
can be straightforwardly distributed over realisation rules for
subject and object agreement (which are types), because their
combination is effectively factored out. Thus, by abstracting over
rules instead of words, generalisation regarding parallel sets of
exponents can be captured quite easily. Sharing of resources is in
fact a more general problem that tends to get overlooked by radically
word-based approaches such as \citet{Blevins14}.

\subsubsection{Modularity}

The final argument for combining constructional or holistic with
generative or atomistic views is that it provides for a divide and
conquer approach to complex inflectional systems.

\citet{diaz:koenig:michelson:19} discuss the pre-pronominal affix
cluster in \ili{Oneida}, an \ili{Iroquoian} language. \ili{Oneida} presents us with what
is probably the most complex morphotactic system that has been
described so far within IbM.

\ili{Oneida} is a highly polysynthetic language. According to
\citet{diaz:koenig:michelson:19}, the prefixal inflectional system alone
comprises seven  position classes in which up to eight
non-modal and three modal categories can be expressed
(cf.\ Table~\ref{tab:Oneida}). Given the number of categories and
position alone, it comes at no surprise that the system is
characterised by heavy competition. Adding to the complexity, several
markers undergo complex interactions, even between non-adjacent slots.
Finally, \ili{Oneida} pre-pronominal prefixes also display variable
morphotactics: the factual, for instance, appears in four different surface
positions, and the optative in three. Moreover, we find paradigmatic
misalignment (cf.\ the discussion of \ili{Nepali} above), with the
cislocative in a different surface position from the translocative.

\begin{table}
  \centering
\oneline{%
  \begin{tabular}{l|l|l|l|l|l|l|l}
    \lsptoprule
    1&2&3&4&5&6&7&8\\
    \midrule    Negative & Translocative & Dualic & Factual & Cislocative &
                                                                Factual
               & Pronominal & Stem\\

    Contrastive & Factual & & Optative & Repetitive & Optative &
                                                                 Factual
                 & \\
    Coincidental & & & Future & &  & Optative &  \\
    Partitive & & & & & & &\\
    \lspbottomrule
  \end{tabular}}
  \caption{Position classes of Oneida inflectional prefixes
    \citep[435]{diaz:koenig:michelson:19}}
  \label{tab:Oneida}
\end{table}

\citet{diaz:koenig:michelson:19} discuss three different types of
interaction within the system: (i) positional competition, exhibited
in slot 1 (negative, contrastive, coincidental, partitive) and slot 5
(cislocative, repetitive); (ii) borrowing, a particular case of
extended exponence exhibited in slot 2 (translocative borrowing vowels
from the future and factual); and (iii) sharing, witnessed by the
factual and the optative, which are distributed across different
positions. Cross-cutting these subsystems, we find a great level of
contextual inflectional allomorphy.

\citet{diaz:koenig:michelson:19} contain the complexity of the system
by building on several key notions, the first three of
which are integral parts of IbM: first, the fact that IbM recognises
$m:n$ relations at the rule level make it possible to approach the
\ili{Oneida} system in a more modular fashion, carving out four independent
subsystems for competition (slot 1 and slot 5), borrowing (slot 2), and
sharing (factual). Second, they draw on the distinction between
realisation (\textsc{mud}) and conditioning \textsc{ms} to abstract
out inflectional allomorphy. Third, they capture discontinuous
exponence of the factual and optative in terms of Koenig/Jurafsky
style cross-classification in order to derive complex discontinuous
rules.

The two innovative aspects of their analysis concern the treatment of
competition and an abstraction over morphosyntactic properties in
terms of syntagmatic classes. \ili{Oneida} resolves morphotactic competition
of semantically compatible features (slots 1 and 5) by means of a
markedness hierarchy: features that are outranked on this hierarchy
are optionally interpreted if the exponent of a higher feature is
present. For example the negative outranks the partitive, so if the negative
marker is present, it can be interpreted as negative or negative and
partitive. If, by contrast, the partitive marker is found, the
negative cannot be understood.  \citet{diaz:koenig:michelson:19}
approach this by modelling the ranking in terms of a type hierarchy
upon which realisation rules can draw. Their second innovation, i.e.\ 
the  segregation of morphosemantic properties according to
the positional properties of their exponents into e.g.\ inner or outer
types, has enabled them to give a much more concise representation of
allomorphy that can abstract over strata of positions. 

The combination of design properties of IbM with their two innovations
have permitted \citet{diaz:koenig:michelson:19} to provide an explicit
and surprisingly concise analysis of an extremely complex system: in
essence, their highly modular analysis (with only 36 rules) reduces the
number of allomorphs by a factor of ten.

\bigskip\noindent In sum, having $m:n$ relations at the most basic
level of realisation rules means that constructional views can be
implemented at any level of granularity, combining reuse and
recombination, as favoured by an atomistic (generative) view, with the
holistic (constructional) view necessitated by discontinuous or
gestalt exponence.  To quote \citet{diaz:koenig:michelson:19}, ``IbM's
approach to morphology [...] is something unification-based approaches
to syntax have stressed for the last forty-years or so''. In addition
to the model-theoretic aspect they capitalise on, the similarity of
IbM to current HPSG syntax also pertains to the fact that both integrate
lexicalist and constructional views.

\section{Conclusion}

This chapter has provided an overview of HPSG work in two core areas
of morphology, namely derivation and inflection. The focus of this
paper was biased to some degree towards inflection, for two reasons:
on the one hand, a handbook article that provides a more balanced
representation of derivational and inflectional work in
constraint-based grammar was published quite recently
\citep{Bonami15b}, while on the other, a comprehensive introduction
to recent developments within HPSG inflectional morphology was still
missing.

In the area of derivation and grammatical function change, a consensus
was reached relatively early, toward the end of the last century, with
the works of \citet{Riehemann98} and \cite{Koenig99}: within HPSG, it
is now clearly understood that lexical rules are description-level
devices organised into cross"=cutting inheritance type
hierarchies. One of the distinctive advantages of these approaches is
the possibility to capture regular, subregular, and irregular
formations using a single unified formal framework, namely partial
descriptions of typed feature structures. Beyond HPSG, these works have
influenced the development of Construction Morphology \citep{Booij10}.\footnote{See \crossrefchaptert{cxg} for a comparison of HPSG with Construction Grammar.}

Much more recently, a consensus model seems to have arrived for the
treatment of inflectional morphology. Information-based Morphology
\citep{Crysmann:Bonami:2016,Crysmann:14:OUP} builds on previous work
on inflectional morphology in HPSG (Bonami), Online Type Construction
\citep{Koenig99}, morph-based morphology \citep{crysmann_b03book}, and
finally unification-based approaches to Pāṇini's principle
\citep{Andrews90,Erjavec94,Koenig99} to provide an
inferential-realisational theory of morphology that exploits the same
logic as HPSG, namely typed feature structure inheritance networks to
capture linguistic generalisations. Furthermore, like its syntactic
parent, it permits to strike a balance between lexicalist and
constructional views. By recognising $m:n$ relations between function
and form at the most basic level, i.e.\ realisation rules,
morphological generalisations are uniformly captured in terms of
partial rule descriptions.


% Bring in m:n and new perspectives. 

{\sloppy
\printbibliography[heading=subbibliography,notkeyword=this]}

\end{document}

%      <!-- Local IspellDict: en_US-w_accents -->



%%% Local Variables:
%%% mode: latex
%%% TeX-master: "../main-morph"
%%% End:
