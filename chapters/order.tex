\documentclass[output=paper]{langsci/langscibook} 
\author{%
	Stefan Müller\affiliation{Humboldt-Universität zu Berlin}%
}
\title{Constituent order}

% \chapterDOI{} %will be filled in at production

%\epigram{Change epigram in chapters/03.tex or remove it there }
\abstract{}
\maketitle

\begin{document}
\label{chap:constituents}

\section{Introduction} 



\section{ID/LP format}

\citet*{GKPS85a}

\section{Flat and binary branching structures}

\argst (Chapter~\ref{chap-argumentstr}) as the underlying representation and determinant of basic order

valence features as sets \citep*{Gunji86a,HN89a,Pollard90a,EEU92a}

need for an extra list for Binding 

\citet{AMM2013a}

valence features as lists + append \citep{MuellerHPSGHandbook}


\section{Nonlocal dependencies}

Brief mention and pointers to Chapter~\ref{chap-udc}.

\section{Head movement vs.\ constructional approaches assuming flat structures}

\subsection{Head movement approaches}

English \citep{Borsley89} and German \textsc{double slash} 
(\citealp*[Section~4.7]{KW91a}; \citealp{Oliva92a}; \citealp*{Netter92};   
\citealp*{Kiss93}; \citealp*{Frank94}; \citealp*{Kiss95a}; \citealp{Feldhaus97},
\citealp{Meurers2000b}; \citealp{Mueller2005c}; \citealp{MuellerGS})

\subsection{Constructional approaches}

English Aux-System \citep{Fillmore99a,Sag2018a}

\section{Constituent order domains}

\subsection{A special representational layer for constituent order}

\citet{Reape94a,Kathol2001a,Mueller2004b}

Free constituent order languages \citet{DS99a}

\subsection{Partial compaction (extraposition)}

\subsection{Problems with order domains}

Partial verb phrase fronting requires partial constituents.

\citep{Kathol2001a,MuellerGS}

\subsection{Other usages of constiuent order domains}

reference to Chapter~\ref{chap-coordination} on coordination and Chapter~\ref{chap-ellipsis} on ellipsis.

comparison with Dependency Grammar (Chapter~\ref{chap-dg}

\section{Free constituent order languages without order domains}

\citet{Bender2008a}


 
\section*{Abbreviations}
\section*{Acknowledgements}

\printbibliography[heading=subbibliography,notkeyword=this] 
\end{document}



\if0
Bob:



It seems to me that whether or not constituent structures are confined to binary branching is quite important for constituent order. How far different constituent orders can be treated as a matter of alternative ordering of sisters depends on how much constituents are sisters. For example, the contrast between (1) and (2) might just show that PP sisters can appear in either order, but that is only possible if they are sisters.

 

(1) Kim talked to Lee about the weather.

(2) Kim talked about the weather to Lee.

 

I assume section 4 is concerned with verb-initial clauses. If so, perhaps the title should make that clear. For Minimalism head-movement is involved not just in verb-initial clauses but also VPs where the verb has two complements and nominal phrases where the noun precedes an attributive adjective, among other things.

 

Borsley (1989) was concerned with English, not Welsh. It made the point that you could have an analogue of verb-fronting for English auxiliary-initial sentences. Borsley (2006) discusses whether Welsh finite clauses involve some form of VP and argues that they do not and hence that they involve a flat structure.

 

Is 4.2 just about constructional approaches to English auxiliary-initial clauses (and not the English auxiliary system in general)? It’s not really clear. I think Pollard and Sag’s (1987) rule 3 and Pollard and Sag’s (1994) are essentially constructional approaches to verb-initial clauses. It is perhaps worth noting (at least briefly) that there has been a debate in versions of HPSG that distinguish between SUBJ and COMPS features about whether post-verbal subjects are realizations of the SUBJ feature like pre-verbal subjects or the COMPS feature like ordinary complements. The first view is adopted in Ginzburg and Sag (2000) and the second in Sag et al. (2003). I would see the first view as a constructional view (since it requires a special phrase type) and the second a lexical view (since it requires a lexical rule of some kind to derive appropriate lexical descriptions). Borsley (1989) argued that the second approach is appropriate for Welsh verb-initial clauses and Borsley (1995) argued that the first approach is right for Arabic verb-initial clauses.

 

Will you be saying much about extraposition phenomena? Chapter 1 currently has a brief reference to order domains, illustrating them with extraposition of relative clauses.

 

Will you say something about analyses of free constituent order languages with order domains?

 

 

REFERENCES

Borsley, R. D. (1989), An HPSG approach to Welsh, Journal of Linguistics 25, 333‑354.

Borsley, R. D. (1995), On some similarities and differences between Welsh and Syrian Arabic, Linguistics 33, 99-122.

Borsley R. D. (2006), On the nature of Welsh VSO clauses, Lingua 116, 462-490.

\fi


%      <!-- Local IspellDict: en_US-w_accents -->
