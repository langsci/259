%\addchap{Preface}
\section{Preface}
\begin{refsection}

%\largerpage[2]
Head-driven Phrase Structure Grammar (HPSG) is a declarative (or, as is often said,
constraint-based) monostratal approach to grammar which dates back to early 1985, when Carl Pollard
presented his Lectures on HPSG. It was developed initially in joint work by Pollard and Ivan Sag,
but many other people have made important contributions to its development over the decades. It
provides a framework for the formulation and implementation of natural language grammars which are
(i) linguistically motivated, (ii) formally explicit, and (iii) computationally tractable. From the
very beginning it has involved both theoretical and computational work seeking both to address the
theoretical concerns of linguists and the practical issues involved in building a useful natural
language processing system.

HPSG is an eclectic framework which has drawn ideas from the earlier Generalized Phrase Structure
Grammar (GPSG, \citealp{GKPS85a}), Categorial Grammar \citep{Ajdukiewicz35a-u}, and Lexical"=Functional
Grammar (LFG, \citealp{Bresnan82a-ed}), among others. It has naturally evolved over the decades. Thus, the construction"=based version of
HPSG, which emerged in the mid-1990s \citep{Sag97a,GSag2000a-u}, differs from earlier work
\citep{ps,ps2} in employing complex hierarchies of phrase types or
constructions. Similarly, the more recent Sign-Based Construction Grammar approach differs from
earlier versions of HPSG in making a distinction between signs and constructions and using it to make a
number of simplifications \citep{Sag2012a}.

Over the years, there have been groups of HPSG researchers in many locations engaged in both
descriptive and theoretical work and often in building HPSG-based computational systems. There have
also been various research and teaching networks, and an annual conference since 1993. The result of
this work is a rich and varied body of research focusing on a variety of languages and offering a
variety of insights. The present volume seeks to provide a picture of where HPSG is today. It begins
with a number of introductory chapters dealing with various general issues. These are followed by
chapters outlining HPSG ideas about some of the most important syntactic phenomena. Next are a
series of chapters on other levels of description, and then chapters on other areas of
linguistics. A final group of chapters considers the relation between HPSG and other theoretical
frameworks.

It should be noted that for various reasons not all areas of HPSG research are covered in the
handbook (e.g., phonology). So, the fact that a particular topic is not addressed in the handbook
should not be interpreted as an absence of research on the topic. Readers interested in such topics
can refer to the HPSG online bibliography maintained at the Humboldt Universität zu Berlin.\footnote{%
\url{https://hpsg.hu-berlin.de/HPSG-Bib/}, 2024-10-17.
}

All chapters were reviewed by one author and at least one of the editors. All chapters were reviewed
by Stefan Müller. Jean-Pierre Koenig and Stefan Müller did a final round of reading all papers and
checked for consistency and cross-linking between the chapters.


\section*{Open access}


Many authors of this handbook have previously been involved in several other handbook projects (some that cover various aspects of HPSG), and by now there are at least five handbook articles on HPSG available. But the editors felt that writing one authoritative resource describing the framework and being available free of charge to everybody was an important service to the linguistic community. We hence decided to publish the book open access with Language Science Press.

% militant version starts here: =:-)
%% The authors of this handbook were involved in many, many other handbook projects before. By now
%% there are at least five handbook articles on HPSG available.
%% % Detmar Bob Levine
%% % Stefan (HSK)
%% % Stefan (Artenvielfalt)
%% % Stefan & Felix
%% % Stefan & Antonio
%% % Adam Przepiórkowski and Anna Kupść  in journal
%% The editors felt that writing these handbook articles for commercial publishers who will hide them
%% behind paywalls is a waste of time. Established researchers do not need further handbook articles
%% that people cannot read. What is needed instead is one authoritative resource describing the framework
%% and being available free of charge to everybody. We hence decided to publish the book open access
%% with Language Science Press.

\section*{Open source}

%\largerpage
Since the book is freely available and no commercial interests stand in the way of openness, the \LaTeX\ source code of the book can be made available as well.
We put all relevant files on GitHub,\footnote{
\url{https://www.github.com/langsci/259}, 2024-10-17.
} and we hope that they may serve as a role model for future publications of HPSG papers.
Additionally, every single item in the bibliographies was checked by hand either by Stefan Müller or by one of his student assistants. 
We checked authors and editors; made sure first name information was complete; corrected page numbers; removed duplicate entries; added DOIs and URLs where appropriate; and added series and number information as applicable for books, book chapters, and journal issues.
The result is a resource containing 2623 bibliography entries.
These can be downloaded as a single readable PDF file or as \textsc{Bib}\TeX{} file from \url{https://github.com/langsci/hpsg-handbook-bib}.



%% -*- coding:utf-8 -*-
\section*{Foreword of the second edition}

\largerpage
The second edition comes with a lot of small improvements: the index has been improved, typos have
been fixed, reference were added, and ORCIDs were added to authors and are displayed on the title pages of the papers now.

%1 properties
The type in the example (\ref{ex:prop38})
on p.\,\pageref{ex:prop38} was changed from \type{phrase} to \type{example-type}. As noted by
Philipp Trapp in 2022, the presence of the feature \textsc{head-daughter} would entail that the type
of the AVM is \type{headed-phrase} and since the type implication in (\ref{ex:prop38}) applies to all
structures of type \type{phrase} this would mean that all linguistics objects of type \type{phrase} have to be
of type \type{headed-phrase}, which would result in contradictions for all subtypes of \emph{phrase}
that are not of type \type{headed-phrase}.

%2 evolution
The LILOG system is now mentioned in the chapter about the evolution of HPSG (Section~\ref{sec-LILOG}).

%4 lexicon
The argument realization principle in (\ref{wd-bouma}) on p.\,\pageref{wd-bouma} was fixed. It
contained too many brackets in the specification of the \depsl. Footnote~\ref{fn-subject-ARP} was
added to explain how constraints on the length of the \subjl can be enforced.

%10 order
The relation \texttt{synsems2signs} was explained by adding (\ref{ex-schema-hc-flat-synsem-sign}),
which is an expansion of (\ref{schema-hc-flat}) on p.\,\pageref{schema-hc-flat}. The relation that
is used in the chapter on complex predicates has the same name now (see
(\ref{CP-ex-head-complements-phrase}) on p.\,\pageref{CP-ex-head-complements-phrase}). The
footnote~\ref{fn-order-lexical-Uszkoreit} was added. It discusses a lexical account of constituent
order assuming a separate lexical item for each ordering variant.

%14 relative clauses
Relative pronouns are NPs. The respective representation in (\ref{x:rc-18}) on p.\,\pageref{x:rc-18}
was fixed. Valence features were added to the lexical item in (\ref{x:rc-17}).

%16 coordination
\emph{Mary} is of category NP rather than N in Figure~\ref{coordphr} on p.\,\pageref{coordphr}. 

%17 idioms
There was an NP to many in the \compsl in the lexical item in (\ref{le-idiomatic-spill}) on
p.\,\pageref{le-idiomatic-spill}.


%Diff-lists are now explained A bit of explanation and a reference was added to

% order: 04.01.22
% Gray -> gray
% der Frau -> dem Kind
% added \ref{ex-schema-hc-flat-synsem-sign}

% complex-predicates 04.01.22
% unified synsems2signs. The relation has the same name now in order.tex and complex-predicates.tex

% relative-clauses.tex 05.01.22
% \trace -> \trace{}
% glosses aligned in {x:rc-129}
% added language tag
% fixed index entry for Bavarian German


% 18.01.22 added language info for German examples

% 25.01.22 Footnote~\ref{fn-hf-schema} was missing. % in udc

% 03.02.22 Idioms: NP in (8) too much, REL bad feature name, ref to Krenn&Erbach added

% 08.02.22 Information structure: added page numbers for Bildhauer & Cook 2010
%          fixed layout issue with Head-Dislocation Schema for Catalan
% 09.02.22 Added sentence about diff-list and reference to copestake2002.
%
% 14.02.22 Added glosses to helfen in chapter on processing
%
% 30.03.22 Figure 4, Mary is NP not N
%
% 26.10.22 (38a) used to be phrase => but since the constraint referred to HD-DTR this would cause a
% conflict for unheaded phrases. Noticed by student Philipp Trapp.
% The left-hand daughter in (38b) must be SYNSEM X, noted by St.Mü.

% 01.11.22
% Daughter in head-filler-phrase must have SYNSEM|LOCAL instead of LOCAL. St. Mü.
%
% 22.11.22
% added comma in 14b in lexicon.tex
%
% 01.12.2022 index entries for \ominus
%
% 10.01.2023 Added comma in np.tex
%
% 17.01.2023 unified spelling of reduced-verb and basic-verb, added dot to example in
% complex-predicates-include.tex
%
% 24.01.2023  removed space udc.tex, added Section to reference of Ross67 regarding ATB
% added crossref to island chapter.
%
% 2023-08-23 Kim Sells appeared in 2015 not in 2014, we missed this despite the check.
%
% 2023-09-26 The LILOG system is now mentioned in evolution.tex
%
% 2023-11-21 SYNSEM|LOCAL in MORPH in (34) in lexicon.tex
%
% 2023-12-12 Fixed NP for relative pronouns rather than N'.
% added SPR and COMPS for lexical item for relative pronoun
% Changed PP [3] into [LOC [3]] in figure in relative clause chapter
% Fixed PP[4], which should have been [3] in footnote in relclause chapter.
%
% 31.01.2024 added Abbreviations for case.tex since illative is not in the Leipzig Glossing Rules.

The book was used at the LSA Linguistic Institute 2023 at the University of Massachusetts Amherst by Tony Davis and in various seminars at the Humboldt
Universität zu Berlin by Stefan Müller. We want to thank everybody who commented on the book.

~\medskip

\noindent
Berlin, Paris, Bangor, Buffalo, \today\hfill Stefan Müller, Anne Abeillé, Robert D. Borsley \& Jean-​Pierre Koenig


%      <!-- Local IspellDict: en_US-w_accents -->



\section{\acknowledgmentsUS}

We thank all the authors for their great contributions to the book, and for
reviewing chapters and chapter outlines of the other authors. We thank
Frank Richter, Bob Levine, and Roland Schäfer for discussion of points related to the handbook, and
Elizabeth Pankratz for extremely careful proofreading and help with typesetting issues. We also
thank Elisabeth Eberle and Luisa Kalvelage for doing bibliographies and typesetting trees of several
chapters and for converting a complicated chapter from Word into \LaTeX.

We thank Sebastian Nordhoff and Felix Kopecky for constant support regarding \LaTeX{} issues, both for
the book project overall and for individual authors. Felix implemented a new \LaTeX{} class for
typesetting AVMs, \texttt{langsci-avm}, which was used for typesetting this book. It is compatible with more
modern font management systems and with the \texttt{forest} package, which is used for most of the trees in this book.

We thank Sašo Živanović for writing and maintaining the \texttt{forest} package and for help
specifying particular styles with very advanced features. His package turned typesetting trees from a
nightmare into pure fun! To make the handling of this large book possible, Stefan Müller asked Sašo
for help with externalization of \texttt{forest} trees, which led to the development of
the \texttt{memoize} package. The HPSG handbook and other book projects by Stefan were an
ideal testing ground for externalization of \texttt{tikz} pictures. Stefan wants to thank
Sašo for the intense collaboration that led to a package of great value for everybody
living in the woods.


The book was used at the LSA Linguistic Institute 2023 at the University of Massachusetts Amherst by
Tony Davis and Jean-Pierre Koenig and in various seminars at the Humboldt
Universität zu Berlin by Stefan Müller. We want to thank the participants of these events for their
comments on the book, which were used to prepare the second edition.

~\medskip

\noindent
Berlin, Paris, Bangor, Buffalo, \today\hfill Stefan Müller, Anne Abeillé, Robert D. Borsley \& Jean-​Pierre Koenig


%~\medskip

%\noindent
%Berlin, Paris, Bangor, Buffalo, November 9, 2021\hfill Stefan Müller, Anne Abeillé, Robert D. Borsley \& Jean-​Pierre Koenig


\printbibliography[heading=subbibliography]

\section{Abbreviations and feature names used in the book}

\begin{longtable}{@{}p{3cm}p{9cm}@{}}
\feat{1st-pc} & first position class \\
\feat{accent} & accent \\
\feat{act(or)} & actor argument \\
\feat{addressee} & index for addressee \\
\feat{aff} & affixes \\
\feat{agr} & agreement \\
\feat{anaph} & anaphora \\
\feat{ancs} & anchors \\
\feat{antec} & antecedent referent markers \\
\feat{arg} & semantic argument of a relation \\
\feat{arg-st} & argument Structure \\
\feat{aux} & auxiliary verb (or not) \\
\feat{background} (\feat{backgr}) & background assumptions \\
\feat{bd} & boundary tone \\
\feat{bg} & background (in information structure) \\
\feat{body} & body (nuclear scope) of quantifier \\
\feat{case} & case \\
\feat{category} & syntactic category information \\
\feat{c-indices} (\feat{c-inds}) & contextual indices \\
\feat{cl} & inflectional class \\
\feat{clitic} (\feat{clts}) & clitics \\
\feat{conds} & predicative conditions \\
\feat{cluster} & cluster of phrases \\
\feat{coll} & collocation type \\
\feat{comps} & complements \\
\feat{concord} & concord information \\
\feat{content} (\feat{cont}) & lexical semantic content \\
\feat{context} (\feat{ctxt}) & contextual information \\
\feat{coord} & coordinator \\ 
\feat{correl} & correlative marker \\
\feat{det} & semantic determiner (a.k.a. quantifier force) \\
\feat{dsl} & double slash \\
\feat{deps} & dependents \\
\feat{dom} & order Domain \\
\feat{dr} & discourse referent \\
\feat{dte} & designated terminal element \\
\feat{dtrs} & daughters \\
\feat{econt} & external content \\
\feat{embed} & embedded (or not) \\
\feat{ending} & inflectional ending \\
\feat{exp} & experiencer \\
\feat{excont} (\feat{exc}) & external content (in LRS) \\
\feat{extra} & extraposed syntactic argument \\
\feat{fc} & focus-marked lexical item \\
\feat{fcompl} & functional complement \\
\feat{fig} & figure in a locative relation \\
\feat{first} & first member of a list \\
\feat{focus} & focus \\
\feat{form} & form of a lexeme \\
\feat{fpp} & focus projection potential \\
\feat{gend} & gender \\
\feat{given} & given information \\
\feat{grnd} & ground in a locative relation \\
\feat{ground} & ground (in information structure) \\
\feat{gtop} & global top \\
\feat{harg} & hole argument of handle constraints \\
\feat{hcons} & handle constraints (to establish relative scope in MRS) \\
\feat{head} (\feat{hd}) & head features\\
\feat{hd-dtr} & head-daughter \\
\feat{hook} & hook (relevant for scope relations in MRS) \\
\feat{ic} & inverted clause (or not) \\
\feat{icons} & individual constraints \\
\feat{icont} & internal content \\
\feat{i-form} & inflected form \\
\feat{index} (\feat{ind}) & semantic index \\
\feat{incont} (\feat{inc}) & internal content (in LRS) \\
\feat{infl} & inflectional features \\
\feat{info-struc} & information structure \\
\feat{inher} & inherited non-local features \\
\feat{inst} & instance (argument of an object category) \\
\feat{inv} & inverted verb (or not) \\
\feat{ip} & intonational phrase \\
\feat{key} & key semantic relation \\
\feat{lagr} & left conjunct agreement \\
\feat{larg} & label argument of handle constraints \\
\feat{lbl} & label of elementary predications \\
\feat{lex-dtr} & lexical daughter \\
\feat{lexeme} & lexeme identifier \\
\feat{lf} & logical form \\
\feat{lid} & lexical identifier \\
\feat{light} & light expressions (or not) \\
\feat{link} & link (in information structure) \\
\feat{listeme} & lexical identifier \\
\feat{liszt} & list of semantic relations \\
\feat{local} & syntactic and semantic information relevant in local contexts \\
\feat{l-periph} & left periphery \\
\feat{ltop} & local top \\
\feat{major} & major part of speech features  \\
\feat{major} & major or minor part of speech \\
\feat{main} & main semantic contribution of a lexeme \\
\feat{marking} (\feat{mrkg}) & marking \\
\feat{max-qud} & maximal question under discussion \\
\feat{mc} & main clause (or not) \\
\feat{$\mu$-feat} & morphological features \\
\feat{minor} & minor part of speech features \\
\feat{mkg} & information structure properties (marking) of lexical items \\
\feat{mod} & modified expression \\
\feat{modal-base} & modal modification of situation core \\
%\feat{mtr} & Mother \\
\feat{mood} & mood \\
\feat{morph} & morphology \\
\feat{morph-b} & morphological base \\
\feat{mp} & morphophonology \\
\feat{mph} & morphs \\
\feat{ms} & morphosyntactic (or morphosemantic) property set \\
\feat{mud} & morph under discussion \\
\feat{n} & nominal part of speech \\
\feat{neg} & negative expression \\
\feat{non-head-dtrs} (\feat{nh-dtrs}) & non-head daughters \\
\feat{nonlocal} & syntactic and semantic information relevant for non-local dependencies \\ 
\feat{nucl} (\feat{nuc}) & nucleus of a state of affairs  \\
\feat{numb} & number \\
\feat{params} & parameters (restricted variables) \\
\feat{pa} & pitch accent \\
\feat{parts} & list of meaningful expressions \\
\feat{pers} & person \\
\feat{pc} & position class \\
\feat{pform} & preposition form \\
\feat{phon} (\feat{ph}) & phonology \\
\feat{phon-string} & phonological string \\
\feat{php} & phonological phrase \\
\feat{pol} & polarity \\
\feat{pool} & pool of quantifiers to be retrieved \\
\feat{prd} & predicative (or not) \\
\feat{pred} & predicate \\
\feat{pref} & prefixes \\
\feat{pre-modifier} &  modifiers before the modified (or not) \\
\feat{prop} & proposition \\
\feat{quants} & list of quantifiers \\
\feat{qstore} & quantifier store \\
\feat{qud} & question under discussion \\
\feat{ques} & question \\ %Not sure what this does, p.396
\feat{ragr} & right conjunct agreement \\
\feat{realized} & realized syntactic argument \\
\feat{rel} & indices for relatives \\
\feat{rln} (\feat{reln}) & semantic relation \\
\feat{rels} & list or set of semantic relations \\
\feat{rest} & non-first members of a list \\
\feat{restr} & restriction of quantifier (in MRS) \\
\feat{restrictions} (\feat{restr}) & restrictions on index \\
\feat{retrieved} & retrieved quantifiers  \\
\feat{r-mark} & reference marker \\
\feat{root} & root clause or not \\
\feat{rr} & realizational Rules \\
\feat{sal-utt} & salient Utterance \\
\feat{select} (\feat{sel}) & selected expression \\
\feat{sit} & situation \\
\feat{sit-core} & situation core \\
\feat{slash} & set of locally unrealized arguments \\
\feat{soa} (\feat{soa-arg}) & state Of Affairs \\
\feat{speaker} & index for the Speaker \\
\feat{spec} & specified \\
\feat{spr} & specifier \\
\feat{status} & information structure status \\
\feat{stem} & stem phonology \\
\feat{stm-pc} & stem position class \\
\feat{store} & same as \feat{q-store} \\ %Check GS
\feat{struc-meaning} & structured meaning \\
\feat{subj-agr} & subject agreement \\
\feat{subcat} & subcategorization \\
\feat{synsem} & syntax/ Semantics features \\
\feat{subj} & subject \\
\feat{tail} & tail (in information structure) \\
\feat{tam} & tense, aspect, mood \\
\feat{tns} & tense \\
\feat{topic} & topic \\
\feat{tp} & topic-marked lexical item \\
\feat{und} & undergoer argument \\
\feat{ut} & phonological utterance \\
\feat{v} & verbal part of speech \\
\feat{val} & valence \\
\feat{var} & variable (bound by a quantifier) \\
\feat{vform} & verb form \\
\feat{weight} & expression weight \\
\feat{wh} & \emph{wh}-expression (for questions) \\
\feat{xarg} & extra-argument \\	
\end{longtable}




\end{refsection}

%      <!-- Local IspellDict: en_US-w_accents -->
