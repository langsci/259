%% -*- coding:utf-8 -*-

% !BIB TS-program = biber
% !TeX TS-program = xelatexmk

\documentclass[output=paper
%	        ,collection
%	        ,collectionchapter
 	        ,biblatex
                ,babelshorthands
                ,newtxmath
                ,draftmode
                ,colorlinks, citecolor=brown
]{langscibook}

\IfFileExists{../localcommands.tex}{%hack to check whether this is being compiled as part of a collection or standalone
  \usepackage{../nomemoize}
  % add all extra packages you need to load to this file 

% the ISBN assigned to the digital edition
\usepackage[ISBN=9783961102556]{ean13isbn} 

\usepackage{graphicx}
\usepackage{tabularx}
\usepackage{amsmath} 

%\usepackage{tipa}      % Davis Koenig
\usepackage{xunicode} % Provide tipa macros (BC)

\usepackage{multicol}

% Berthold morphology
\usepackage{relsize}
%\usepackage{./styles/rtrees-bc} % forbidden forest 08.12.2019

% provides logo priniting commands
\usepackage{langsci-basic}

\usepackage{langsci-optional} 
% used to be in this package
\providecommand{\citegen}{}
\renewcommand{\citegen}[2][]{\citeauthor{#2}'s (\citeyear*[#1]{#2})}
\providecommand{\lsptoprule}{}
\renewcommand{\lsptoprule}{\midrule\toprule}
\providecommand{\lspbottomrule}{}
\renewcommand{\lspbottomrule}{\bottomrule\midrule}
\providecommand{\largerpage}{}
\renewcommand{\largerpage}[1][1]{\enlargethispage{#1\baselineskip}}

\usepackage{./styles/biblatex-series-number-checks}


\usepackage{langsci-lgr}

\newcommand{\MAS}{\textsc{m}\xspace} % \M is taken by somebody

%\usepackage{./styles/forest/forest}
\usepackage{langsci-forest-setup}

% is loaded in main etc.
% \usepackage{nomemoize} 
% \memoizeset{
%   memo filename prefix={chapters/hpsg-handbook.memo.dir/},
%   register=\todo{O{}+m},
%   prevent=\todo,
% }

\usepackage{tikz-cd}

\usepackage{./styles/tikz-grid}
\usetikzlibrary{shadows}


% removed with texlive 2020 06.05.2020
% %\usepackage{pgfplots} % for data/theory figure in minimalism.tex
% % fix some issue with Mod https://tex.stackexchange.com/a/330076
% \makeatletter
% \let\pgfmathModX=\pgfmathMod@
% \usepackage{pgfplots}%
% \let\pgfmathMod@=\pgfmathModX
% \makeatother

\usepackage{subcaption}

% Stefan Müller's styles
\usepackage{./styles/merkmalstruktur,./styles/makros.2020,./styles/my-xspace,./styles/article-ex,
./styles/eng-date}

\usepackage{varioref}
\newcommand\refORregion[2]{%
 \vrefpagenum\firstnum{#1}%
 \vrefpagenum\secondnum{#2}%
\ifthenelse{\equal\firstnum\secondnum}%
{\pageref{#1}}%
{\pageref{#1}--\pageref{#2}}%
}

% I am sick of fiddeling arround with babel. I want these shorthands also to work in commands I
% define. St.Mü. 13.08.2020
% e.g. with \iwithini
\usepackage{german}
\selectlanguage{USenglish}

\usepackage{./styles/abbrev}


% Has to be loaded late since otherwise footnotes will not work

%%%%%%%%%%%%%%%%%%%%%%%%%%%%%%%%%%%%%%%%%%%%%%%%%%%%
%%%                                              %%%
%%%           Examples                           %%%
%%%                                              %%%
%%%%%%%%%%%%%%%%%%%%%%%%%%%%%%%%%%%%%%%%%%%%%%%%%%%%
% remove the percentage signs in the following lines
% if your book makes use of linguistic examples
\usepackage{langsci-gb4e} 



% This introduces labels which makes hyperlinks work so that proofreading is easier.
%\makeatletter
%\newcommand{\mex}[1]{\ref{ex-\the\c@chapter-\the\numexpr\c@equation+#1}\relax}
%\newcommand{\eaautolabel}{\label{ex-\the\c@chapter-\the\numexpr\c@equation+1}}
%\makeatother

%\let\oldea\ea
%\def\ea{\oldea\eaautolabel}

%\let\oldeal\eal
%\def\eal{\oldeal\eaautolabel}


% Crossing out text
% uncomment when needed
%\usepackage{ulem}

\usepackage{./styles/additional-langsci-index-shortcuts}

% this is the completely redone avm package
\usepackage{langsci-avm}
\avmsetup{columnsep=.3ex,style=narrow}

\avmdefinecommand{phon}[phon]
  {
    attributes  = \itshape%,
%    delimfactor = 900,
%    delimfall   = 10pt
}

\avmdefinecommand{form}[form]
  {
    attributes  = \itshape%,
%    delimfactor = 900,
%    delimfall   = 10pt
}

% \set was already taken
\avmdefinecommand{avmset}[set]{ attributes=\itshape } % define a new \set command
\avmdefinecommand{list}[list]{ attributes=\itshape } % define a new \list command
   % Note: the label "list" will be output in whatever font is currently active.

% \avm{
% 	[subj  & \1 \\
% 	comps & \2 \- \list*(gap-ss) \\ % Produce a \list
% 	deps  & < \1 > \+ \2
% 	]
% }


\avmdefinecommand{nelist}[ne-list]{ attributes=\itshape } % define a new \nelist command
   % Note: the label "ne-list" will be output in whatever font is currently active.



% https://github.com/langsci/langsci-avm/issues/33#issuecomment-671201576
%\avmsetup{extraskip=0pt}

% if you have to use both langsci-avm and avm
% \usepackage{langsci-avm} % Load pkg with meaning A of conflicting cmd
% \let\lavm\avm % Send the conflicting command to an alternative
% \let\avm\undefined % Send the conflicting cmd to be \undefined
% \usepackage{avm} % Load pkg with meaning B for conf. cmd 

%\let\asort\type*

% remove this, once we really do without avm
%\usepackage{./styles/avm+}

% copied over from avm+.sty
% some relation operators:
%\newcommand{\append}[0]{\ensuremath{\oplus\hspace{.24em}}}
%\newcommand{\shuffle}[0]{\ensuremath{\bigcirc\hspace{.24em}}}

\newcommand{\append}[0]{\ensuremath{\oplus}\xspace}
\newcommand{\shuffle}[0]{\ensuremath{\bigcirc}\xspace}


% command to fontify relations in avms 
\newcommand{\rel}[1]{\texttt{#1}}
%\def\relfont{\slshape}%
%\def\relfont{\ttdefault}%


\let\idx\ibox
\let\avmbox\ibox

% command to fontify attributes in ordinary text
%\newcommand{\attrib}[1]{\textsc{#1}}


% some relation operators:
%\newcommand{\append}[0]{\ensuremath{\oplus\hspace{.24em}}}
%\newcommand{\shuffle}[0]{\ensuremath{\bigcirc\hspace{.24em}}}

\def\relfont{\slshape}%
%
% command to fontify relations in avms 
%\newcommand{\rel}[1]{{\relfont #1}}



% \renewcommand{\tpv}[1]{{\avmjvalfont\itshape #1}}

% % no small caps please
% \renewcommand{\phonshape}[0]{\normalfont\itshape}

% \regAvmFonts

\usepackage{theorem}

\newtheorem{mydefinition}{Def.}
\newtheorem{principle}{Principle}

{\theoremstyle{break}
%\newtheorem{schema}{Schema}
\newtheorem{mydefinition-break}[mydefinition]{Def.}
\newtheorem{principle-break}[principle]{Principle}
}


%% \newcommand{schema}[2]{
%% \begin{minipage}{\textwidth}
%% {\textbf{Schema~\theschema}}]\hspace{.5em}\textbf{(#1)}\\
%% #2
%% \end{minipage}}


% This avoids linebreaks in the Schema
\newcounter{schemacounter}
\makeatletter
\newenvironment{schema}[1][]
  {%
   \refstepcounter{schemacounter}%
   \par\bigskip\noindent
   \minipage{\linewidth}%
   \textbf{Schema~\theschemacounter\hspace{.5em} \ifx&#1&\else(#1)\fi}\par
  }{\endminipage\par\bigskip\@endparenv}%
\makeatother

%\usepackage{subfig}





% Davis Koenig Lexikon

\usepackage{tikz-qtree,tikz-qtree-compat} % Davis Koenig remove

\usepackage{shadow}



\usepackage[english]{isodate} % Andy Lücking
\usepackage[autostyle]{csquotes} % Andy
%\usepackage[autolanguage]{numprint}

%\defaultfontfeatures{
%    Path = /usr/local/texlive/2017/texmf-dist/fonts/opentype/public/fontawesome/ }

%% https://tex.stackexchange.com/a/316948/18561
%\defaultfontfeatures{Extension = .otf}% adds .otf to end of path when font loaded without ext parameter e.g. \newfontfamily{\FA}{FontAwesome} > \newfontfamily{\FA}{FontAwesome.otf}
%\usepackage{fontawesome} % Andy Lücking
\usepackage{pifont} % Andy Lücking -> hand

\usetikzlibrary{decorations.pathreplacing} % Andy Lücking
\usetikzlibrary{matrix} % Andy 
\usetikzlibrary{positioning} % Andy
\usepackage{tikz-3dplot} % Andy

% pragmatics
\usepackage{eqparbox} % Andy
\usepackage{enumitem} % Andy
\usepackage{longtable} % Andy
\usepackage{tabu} % Andy              needs to be loaded before hyperref as of texlive 2020

% tabu-fix
% to make "spread 0pt" work
% -----------------------------
\RequirePackage{etoolbox}
\makeatletter
\patchcmd
	\tabu@startpboxmeasure
	{\bgroup\begin{varwidth}}%
	{\bgroup
	 \iftabu@spread\color@begingroup\fi\begin{varwidth}}%
	{}{}
\def\@tabarray{\m@th\def\tabu@currentgrouptype
    {\currentgrouptype}\@ifnextchar[\@array{\@array[c]}}
%
%%% \pdfelapsedtime bug 2019-12-15
\patchcmd
	\tabu@message@etime
	{\the\pdfelapsedtime}%
	{\pdfelapsedtime}%
	{}{}
%
%
\makeatother
% -----------------------------


% Manfred's packages

%\usepackage{shadow}

\usepackage{tabularx}
\newcolumntype{L}[1]{>{\raggedright\arraybackslash}p{#1}} % linksbündig mit Breitenangabe


% Jong-Bok

%\usepackage{xytree}

\newcommand{\xytree}[2][dummy]{Let's do the tree!}

% seems evil, get rid of it
% defines \ex is incompatible with gb4e
%\usepackage{lingmacros}

% taken from lingmacros:
\makeatletter
% \evnup is used to line up the enumsentence number and an entry along
% the top.  It can take an argument to improve lining up.
\def\evnup{\@ifnextchar[{\@evnup}{\@evnup[0pt]}}

\def\@evnup[#1]#2{\setbox1=\hbox{#2}%
\dimen1=\ht1 \advance\dimen1 by -.5\baselineskip%
\advance\dimen1 by -#1%
\leavevmode\lower\dimen1\box1}
\makeatother


% YK -- CG chapter

%\usepackage{xspace}
\usepackage{bm}
\usepackage{ebproof}


% Antonio Branco, remove this
\usepackage{epsfig}

% now unicode
%\usepackage{alphabeta}





\usepackage{pst-node}


% fmr: additional packages
%\usepackage{amsthm}


% Ash and Steve: LFG
\usepackage{./styles/lfg/dalrymple}

\RequirePackage{graphics}
%\RequirePackage{./styles/lfg/trees}
%% \RequirePackage{avm}
%% \avmoptions{active}
%% \avmfont{\sc}
%% \avmvalfont{\sc}
\RequirePackage{./styles/lfg/lfgmacrosash}

\usepackage{./styles/lfg/glue}

%%%%%%%%%%%%%%%%%%%%%%%%%%%%%%
%% Markup
%%%%%%%%%%%%%%%%%%%%%%%%%%%%%%
\usepackage[normalem]{ulem} % For thinks like strikethrough, using \sout

% \newcommand{\high}[1]{\textbf{#1}} % highlighted text
\newcommand{\high}[1]{\textit{#1}} % highlighted text
%\newcommand{\term}[1]{\textit{#1}\/} % technical term
\newcommand{\qterm}[1]{``{#1}''} % technical term, quotes
%\newcommand{\trns}[1]{\strut `#1'} % translation in glossed example
\newcommand{\trnss}[1]{\strut \phantom{\sqz{}} `#1'} % translation in ungrammatical glossed example
\newcommand{\ttrns}[1]{(`#1')} % an in-text translation of a word
\newcommand{\LFGfeat}[1]{\mbox{\textsc{\MakeLowercase{#1}}}}     % feature name
%\newcommand{\val}[1]{\mbox{\textsc{\MakeLowercase{#1}}}}    % f-structure value
\newcommand{\featt}[1]{\mbox{\textsc{\MakeLowercase{#1}}}}     % feature name
\newcommand{\vall}[1]{\mbox{\textsc{\textup{\MakeLowercase{#1}}}}}    % f-structure value
\newcommand{\mg}[1]{\mbox{\textsc{\MakeLowercase{#1}}}}    % morphological gloss
%\newcommand{\word}[1]{\textit{#1}}       % mention of word
\providecommand{\kstar}[1]{{#1}\ensuremath{^*}}
\providecommand{\kplus}[1]{{#1}\ensuremath{^+}}
\newcommand{\template}[1]{@\textsc{\MakeLowercase{#1}}}
\newcommand{\templaten}[1]{\textsc{\MakeLowercase{#1}}}
\newcommand{\templatenn}[1]{\MakeUppercase{#1}}
\newcommand{\tempeq}{\ensuremath{=}}
\newcommand{\predval}[1]{\ensuremath{\langle}\textsc{#1}\ensuremath{\rangle}}
\newcommand{\predvall}[1]{{\rm `#1'}}
\newcommand{\lfgfst}[1]{\ensuremath{#1\,}}
\newcommand{\scare}[1]{``#1''} % scare quotes
\newcommand{\bracket}[1]{\ensuremath{\left\langle\mathit{#1}\right\rangle}}
\newcommand{\sectionw}[1][]{Section#1} % section word: for cap/non-cap
\newcommand{\tablew}[1][]{Table#1} % table word: for cap/non-cap
\newcommand{\lfgglue}{LFG+Glue}
\newcommand{\hpsgglue}{HPSG+Glue}
\newcommand{\gs}{GS}
%\newcommand{\func}[1]{\ensuremath{\mathbf{#1}}}
\newcommand{\func}[1]{\textbf{#1}}
\renewcommand{\glue}{Glue}
%\newcommand{\exr}[1]{(\ref{ex:#1}}
\newcommand{\exra}[1]{(\ref{ex:#1})}


%%%%%%%%%%%%%%%%%%%%%%%%%%%%%%
% Notation
%\newcommand{\xbar}[1]{$_{\mbox{\textsc{#1}$^{\raisebox{1ex}{}}$}}$}
\newcommand{\xprime}[2][]{\textup{\mbox{{#2}\ensuremath{^\prime_{\hspace*{-.0em}\mbox{\footnotesize\ensuremath{\mathit{#1}}}}}}}}
\providecommand{\xzero}[2][]{#2\ensuremath{^0_{\mbox{\footnotesize\ensuremath{\mathit{#1}}}}}}



\let\leftangle\langle
\let\rightangle\rangle

%\newcommand{\pslabel}[1]{}

% remove when finished
\usepackage{proofread}
  %add all your local new commands to this file

% The orchid-id is specified and then extracted by scripts for zenodo.
\newcommand{\orcid}[1]{} 

% do not show the chapter number. It is redundant, since most references to figures are within the
% same chapter.
\renewcommand{\thefigure}{\arabic{figure}}


% Don't do this at home. I do not like the smaller font for captions.
% I just removed loading the caption packege in langscibook.cls
%% \captionsetup{%
%% font={%
%% stretch=1%.8%
%% ,normalsize%,small%
%% },%
%% width=.8\textwidth
%% }

\makeatletter
\def\blx@maxline{77}
\makeatother


\let\citew\citet

\newcommand{\page}{}

\newcommand{\todostefan}[1]{\todo[color=orange!80]{\footnotesize #1}\xspace}
\newcommand{\todosatz}[1]{\todo[color=red!40]{\footnotesize #1}\xspace}

\newcommand{\inlinetodostefan}[1]{\todo[color=green!40,inline]{\footnotesize #1}\xspace}

\newcommand{\inlinetodoopt}[1]{\todo[color=green!40,inline]{\footnotesize #1}\xspace}
\newcommand{\inlinetodoobl}[1]{\todo[color=red!40,inline]{\footnotesize #1}\xspace}

\newcommand{\itd}[1]{\inlinetodoobl{#1}}
\newcommand{\itdobl}[1]{\inlinetodoobl{#1}}
\newcommand{\itdopt}[1]{\inlinetodoopt{#1}}

\newcommand{\itdsecond}[1]{}

\newcommand{\itddone}[1]{}
%\let\itddone\itdopt
\newcommand{\LATER}[1]{}



% A. Red: Simple typos, errors in the AVMs (only a couple) to take care of on the editorial side, no need to contact the authors
% B.: Green: Wording changes which do not necessarily require authors’ approval, but are not just typos/errors
% C.: Blue: Comments to the author that they don’t have to take care of, but after all, the authors might be interested to have the comments for future revisions. 
% D.: Purple: Comments to the editors about something we need to keep in mind or do. Nothing for you

\newcommand{\colorcodingexplanation}{\todo[color=green!40,inline]{%
Explanation of colors of bubbles and text:\\
A.: Red: Things that have to be fixed/commented upon.\\
B.: Green: optional comments\\
C.: Blue: Comments to the author that they don’t have to take care of, but after all, the authors
might be interested to have the comments for future revisions.\\
Explanation of colors of text:\\
Red: newly added material (crossreferences to other chapters and other references)\\
Orange: changed material, please check\\
Blue: suggestions for deletion\\
Please also check margin notes.
}}
% D.: Purple: Comments to the editors about something we need to keep in mind or do. Nothing for you


\newcommand{\itdgreen}[1]{\todo[color=green!40,inline]{\footnotesize #1}\xspace}
\newcommand{\itdblue}[1]{\todo[color=blue!40,inline]{\footnotesize #1}\xspace}

% for editing, remove later
\usepackage{xcolor}
\newcommand{\added}[1]{{\red #1}}
\newcommand{\addedthis}{\todostefan{added this}}

\newcommand{\changed}[1]{\textcolor{orange}{#1}}
\newcommand{\deleted}[1]{\textcolor{blue}{#1}}


% \newcommand{\addpages}{\todostefan{add pages}}
% %\newcommand{\iaddpages}{\inlinetodoobl{add pages}}
% \newcommand{\iaddpages}{\yel[add pages]{pages}\xspace}
% \newcommand{\addref}{\todostefan{add reference}}
% \newcommand{\inlineaddpages}{\inlinetodostefan{add pages}}
% \newcommand{\addglosses}{\todostefan{add glosses}}

\newcommand{\addpages}{\xspace}%np
\newcommand{\iaddpages}{\xspace}%islands und understudied languages
\newcommand{\addref}{\xspace}
\newcommand{\inlineaddpages}{\xspace}
% not used \newcommand{\addglosses}{}


%\newcommand{\spacebr}{\hphantom{[}}

\newcommand{\danishep}{\jambox{(\ili{Danish})}}
\newcommand{\english}{\jambox{(\ili{English})}}
\newcommand{\german}{\jambox{(\ili{German})}}
\newcommand{\yiddish}{\jambox{(\ili{Yiddish})}}
\newcommand{\welsh}{\jambox{(\ili{Welsh})}}

% Cite and cross-reference other chapters
\newcommand{\crossrefchaptert}[2][]{\citet*[#1]{chapters/#2}, Chapter~\ref{chap-#2} of this volume} 
\newcommand{\crossrefchapterp}[2][]{(\citealp*[#1]{chapters/#2}, Chapter~\ref{chap-#2} of this volume)}
\newcommand{\crossrefchapteralt}[2][]{\citealt*[#1]{chapters/#2}, Chapter~\ref{chap-#2} of this volume}
\newcommand{\crossrefchapteralp}[2][]{\citealp*[#1]{chapters/#2}, Chapter~\ref{chap-#2} of this volume}

\newcommand{\crossrefcitet}[2][]{\citet*[#1]{chapters/#2}} 
\newcommand{\crossrefcitep}[2][]{\citep*[#1]{chapters/#2}}
\newcommand{\crossrefcitealt}[2][]{\citealt*[#1]{chapters/#2}}
\newcommand{\crossrefcitealp}[2][]{\citealp*[#1]{chapters/#2}}


% example of optional argument:
% \crossrefchapterp[for something, see:]{name}
% gives: (for something, see: Author 2018, Chapter~X of this volume)



\let\crossrefchapterw\crossrefchaptert



% Davis Koenig

\let\ig=\textsc
\let\tc=\textcolor

% evolution, Flickinger, Pollard, Wasow

\let\citeNP\citet

% Adam P

%\newcommand{\toappear}{Forthcoming}
\newcommand{\pg}[1]{p.\,#1}
\renewcommand{\implies}{\rightarrow}

\newcommand*{\rref}[1]{(\ref{#1})}
\newcommand*{\aref}[1]{(\ref{#1}a)}
\newcommand*{\bref}[1]{(\ref{#1}b)}
\newcommand*{\cref}[1]{(\ref{#1}c)}

\newcommand{\msadam}{.}
\newcommand{\morsyn}[1]{\textsc{#1}}

\newcommand{\aux}{\textsc{aux}\xspace}

\newcommand{\nom}{\morsyn{nom}}
\newcommand{\acc}{\morsyn{acc}}
\newcommand{\dat}{\morsyn{dat}}
\newcommand{\gen}{\morsyn{gen}}
\newcommand{\ins}{\morsyn{ins}}
%\newcommand{\aploc}{\morsyn{loc}}
\newcommand{\voc}{\morsyn{voc}}
\newcommand{\ill}{\morsyn{ill}}
\renewcommand{\inf}{\morsyn{inf}}
\newcommand{\passprc}{\morsyn{passp}}

%\newcommand{\Nom}{\msadam\nom}
%\newcommand{\Acc}{\msadam\acc}
%\newcommand{\Dat}{\msadam\dat}
%\newcommand{\Gen}{\msadam\gen}
\newcommand{\Ins}{\msadam\ins}
\newcommand{\Loc}{\msadam\loc}
\newcommand{\Voc}{\msadam\voc}
\newcommand{\Ill}{\msadam\ill}
\newcommand{\PassP}{\msadam\passprc}

\newcommand{\Aux}{\textsc{aux}}

%\newcommand{\princ}[1]{\textnormal{\textsc{#1}}} % for constraint names
\newcommand{\princ}[1]{\textnormal{#1}} % for constraint names
\newcommand{\notion}[1]{\emph{#1}}
\renewcommand{\path}[1]{\textnormal{\textsc{#1}}}
\newcommand{\ftype}[1]{\textit{#1}}
\newcommand{\fftype}[1]{{\scriptsize\textit{#1}}}
\newcommand{\la}{$\langle$}
\newcommand{\ra}{$\rangle$}
%\newcommand{\argst}{\path{arg-st}}
\newcommand{\phtm}[1]{\setbox0=\hbox{#1}\hspace{\wd0}}
\newcommand{\prep}[1]{\setbox0=\hbox{#1}\hspace{-1\wd0}#1}


% Rui

\newcommand{\spc}[0]{\hspace{-1pt}\underline{\hspace{6pt}}\,}
\newcommand{\spcs}[0]{\hspace{-1pt}\underline{\hspace{6pt}}\,\,}
\newcommand{\bad}[1]{\leavevmode\llap{#1}}
\newcommand{\COMMENT}[1]{}


% Rui coordination
\newcommand{\subl}[1]{$_{\scriptstyle \textsc{#1}}$}



% Andy Lücking gesture.tex
\newcommand{\Pointing}{\ding{43}}
% Giotto: "Meeting of Joachim and Anne at the Golden Gate" - 1305-10 
\definecolor{GoldenGate1}{rgb}{.13,.09,.13} % Dress of woman in black
\definecolor{GoldenGate2}{rgb}{.94,.94,.91} % Bridge
\definecolor{GoldenGate3}{rgb}{.06,.09,.22} % Blue sky
\definecolor{GoldenGate4}{rgb}{.94,.91,.87} % Dress of woman with shawl
\definecolor{GoldenGate5}{rgb}{.52,.26,.26} % Joachim's robe
\definecolor{GoldenGate6}{rgb}{.65,.35,.16} % Anne's robe
\definecolor{GoldenGate7}{rgb}{.91,.84,.42} % Joachim's halo

\makeatletter
\newcommand{\@Depth}{1} % x-dimension, to front
\newcommand{\@Height}{1} % z-dimension, up
\newcommand{\@Width}{1} % y-dimension, rightwards
%\GGS{<x-start>}{<y-start>}{<z-top>}{<z-bottom>}{<Farbe>}{<x-width>}{<y-depth>}{<opacity>}
\newcommand{\GGS}[9][]{%
\coordinate (O) at (#2-1,#3-1,#5);
\coordinate (A) at (#2-1,#3-1+#7,#5);
\coordinate (B) at (#2-1,#3-1+#7,#4);
\coordinate (C) at (#2-1,#3-1,#4);
\coordinate (D) at (#2-1+#8,#3-1,#5);
\coordinate (E) at (#2-1+#8,#3-1+#7,#5);
\coordinate (F) at (#2-1+#8,#3-1+#7,#4);
\coordinate (G) at (#2-1+#8,#3-1,#4);
\draw[draw=black, fill=#6, fill opacity=#9] (D) -- (E) -- (F) -- (G) -- cycle;% Front
\draw[draw=black, fill=#6, fill opacity=#9] (C) -- (B) -- (F) -- (G) -- cycle;% Top
\draw[draw=black, fill=#6, fill opacity=#9] (A) -- (B) -- (F) -- (E) -- cycle;% Right
}
\makeatother


% pragmatics
\newcommand{\speaking}[1]{\eqparbox{name}{\textsc{\lowercase{#1}\space}}}
\newcommand{\alname}[1]{\eqparbox{name}{\textsc{\lowercase{#1}}}}
\newcommand{\HPSGTTR}{HPSG$_{\text{TTR}}$\xspace}

\newcommand{\ttrtype}[1]{\textit{#1}}
\newcommand{\avmel}{\q<\quad\q>} %% shortcut for empty lists in AVM
\newcommand{\ttrmerge}{\ensuremath{\wedge_{\textit{merge}}}}
\newcommand{\Cat}[2][0.1pt]{%
  \begin{scope}[y=#1,x=#1,yscale=-1, inner sep=0pt, outer sep=0pt]
   \path[fill=#2,line join=miter,line cap=butt,even odd rule,line width=0.8pt]
  (151.3490,307.2045) -- (264.3490,307.2045) .. controls (264.3490,291.1410) and (263.2021,287.9545) .. (236.5990,287.9545) .. controls (240.8490,275.2045) and (258.1242,244.3581) .. (267.7240,244.3581) .. controls (276.2171,244.3581) and (286.3490,244.8259) .. (286.3490,264.2045) .. controls (286.3490,286.2045) and (323.3717,321.6755) .. (332.3490,307.2045) .. controls (345.7277,285.6390) and (309.3490,292.2151) .. (309.3490,240.2046) .. controls (309.3490,169.0514) and (350.8742,179.1807) .. (350.8742,139.2046) .. controls (350.8742,119.2045) and (345.3490,116.5037) .. (345.3490,102.2045) .. controls (345.3490,83.3070) and (361.9972,84.4036) .. (358.7581,68.7349) .. controls (356.5206,57.9117) and (354.7696,49.2320) .. (353.4652,36.1439) .. controls (352.5396,26.8573) and (352.2445,16.9594) .. (342.5985,17.3574) .. controls (331.2650,17.8250) and (326.9655,37.7742) .. (309.3490,39.2045) .. controls (291.7685,40.6320) and (276.7783,24.2380) .. (269.9740,26.5795) .. controls (263.2271,28.9013) and (265.3490,47.2045) .. (269.3490,60.2045) .. controls (275.6359,80.6368) and (289.3490,107.2045) .. (264.3490,111.2045) .. controls (239.3490,115.2045) and (196.3490,119.2045) .. (165.3490,160.2046) .. controls (134.3490,201.2046) and (135.4934,249.3212) .. (123.3490,264.2045) .. controls (82.5907,314.1553) and (40.8239,293.6463) .. (40.8239,335.2045) .. controls (40.8239,353.8102) and (72.3490,367.2045) .. (77.3490,361.2045) .. controls (82.3490,355.2045) and (34.8638,337.3259) .. (87.9955,316.2045) .. controls (133.3871,298.1601) and   (137.4391,294.4766) .. (151.3490,307.2045) -- cycle;
\end{scope}%
}
%% leicht modifiziert nach Def. von Sebastian Nordhoff:
% \newcommand{\lueckingbox}[3]{\parbox[t][][t]{0.7cm}{\raggedright
%     \strut#1}\parbox[t][][t]{7.7cm}{\strut#2}\parbox[t][][t]{3cm}{\raggedright\strut#3}\bigskip\\}
\newcommand{\lueckingbox}[3]{\parbox[t][][t]{0.7cm}{\raggedright
    \strut\vspace*{-\baselineskip}\newline#1}\parbox[t][][t]{7.7cm}{\strut\vspace*{-\baselineskip}\newline#2}\parbox[t][][t]{3cm}{\raggedright\strut\vspace*{-\baselineskip}\newline#3}\bigskip\\}




% KdK
\newcommand{\smiley}{:)}

\renewbibmacro*{index:name}[5]{%
  \usebibmacro{index:entry}{#1}
    {\iffieldundef{usera}{}{\thefield{usera}\actualoperator}\mkbibindexname{#2}{#3}{#4}{#5}}}

% \newcommand{\noop}[1]{}

% chngcntr.sty otherwise gives error that these are already defined
%\let\counterwithin\relax
%\let\counterwithout\relax

% the space of a left bracket for glossings
\newcommand{\LB}{\hphantom{[}}

\newcommand{\LF}{\mbox{$[\![$}}

\newcommand{\RF}{\mbox{$]\!]_F$}}

\newcommand{\RT}{\mbox{$]\!]_T$}}





% Manfred's

\newcommand{\kommentar}[1]{}

\newcommand{\bsp}[1]{\emph{#1}}
\newcommand{\bspT}[2]{\bsp{#1} `#2'}
\newcommand{\bspTL}[3]{\bsp{#1} (lit.: #2) `#3'}

\newcommand{\noidi}{§}

\newcommand{\refer}[1]{(\ref{#1})}

%\newcommand{\avmtype}[1]{\multicolumn{2}{l}{\type{#1}}}
\newcommand{\attr}[1]{\textsc{#1}}

%\newcommand{\srdefault}{\mbox{\begin{tabular}{@{}c@{}}{\large <}\\[-1.5ex]$\sqcap$\end{tabular}}}
\newcommand{\srdefault}{$\stackrel{<}{\sqcap}$}


%% \newcommand{\myappcolumn}[2]{
%% \begin{minipage}[t]{#1}#2\end{minipage}
%% }

%% \newcommand{\appc}[1]{\myappcolumn{3.7cm}{#1}}


% Jong-Bok


% clean that up and do not use \def (killing other stuff defined before)
%\if 0
%\newcommand\DEL{\textsc{del}}
%\newcommand\del{\textsc{del}}

\newcommand\conn{\textsc{conn}}
\newcommand\CONN{\textsc{conn}}
\newcommand\CONJ{\textsc{conj}}
\newcommand\LITE{\textsc{lex}}
\newcommand\lite{\textsc{lex}}
\newcommand\HON{\textsc{hon}}

%\newcommand\CAUS{\textsc{caus}}
%\newcommand\PASS{\textsc{pass}}
\newcommand\NPST{\textsc{npst}}
%\newcommand\COND{\textsc{cond}}



\newcommand\hdlite{\textsc{head-lex construction}}
\newcommand\hdlight{\textsc{head-light} Schema}
\newcommand\NFORM{\textsc{nform}}

\newcommand\RELS{\textsc{rels}}
%\newcommand\TENSE{\textsc{tense}}


%\newcommand\ARG{\textsc{arg}}
\newcommand\ARGs{\textsc{arg0}}
\newcommand\ARGa{\textsc{arg}}
\newcommand\ARGb{\textsc{arg2}}
\newcommand\TPC{\textsc{top}}
%\newcommand\PROG{\textsc{prog}}

\newcommand\LIGHT{\textsc{light}\xspace}
\newcommand\pst{\textsc{pst}}
%\newcommand\PAST{\textsc{pst}}
%\newcommand\DAT{\textsc{dat}}
%\newcommand\CONJ{\textsc{conj}}
\newcommand\nominal{\textsc{nominal}}
\newcommand\NOMINAL{\textsc{nominal}}
\newcommand\VAL{\textsc{val}}
%\newcommand\val{\textsc{val}}
\newcommand\MODE{\textsc{mode}}
\newcommand\RESTR{\textsc{restr}}
\newcommand\SIT{\textsc{sit}}
\newcommand\ARG{\textsc{arg}}
\newcommand\RELN{\textsc{rel}}
%\newcommand\REL{\textsc{rel}}
%\newcommand\RELS{\textsc{rels}}
%\newcommand\arg-st{\textsc{arg-st}}
\newcommand\xdel{\textsc{xdel}}
\newcommand\zdel{\textsc{zdel}}
\newcommand\sug{\textsc{sug}}
%\newcommand\IMP{\textsc{imp}}
%\newcommand\conn{\textsc{conn}}
%\newcommand\CONJ{\textsc{conj}}
%\newcommand\HON{\textsc{hon}}
\newcommand\BN{\textsc{bn}}
\newcommand\bn{\textsc{bn}}
\newcommand\pres{\textsc{pres}}
\newcommand\PRES{\textsc{pres}}
\newcommand\prs{\textsc{pres}}
%\newcommand\PRS{\textsc{pres}}
\newcommand\agt{\textsc{agt}}
%\newcommand\DEL{\textsc{del}}
%\newcommand\PRED{\textsc{pred}}
\newcommand\AGENT{\textsc{agent}}
\newcommand\THEME{\textsc{theme}}
%\newcommand\AUX{\textsc{aux}}
%\newcommand\THEME{\textsc{theme}}
%\newcommand\PL{\textsc{pl}}
\newcommand\SRC{\textsc{src}}
\newcommand\src{\textsc{src}}
\newcommand{\FORMjb}{\textsc{form}}
\newcommand{\formjb}{\FORM}
\newcommand\GCASE{\textsc{gcase}}
\newcommand\gcase{\textsc{gcase}}
\newcommand\SCASE{\textsc{scase}}
\newcommand\PHON{\textsc{phon}}
%\newcommand\SS{\textsc{ss}}
\newcommand\SYN{\textsc{syn}}
%\newcommand\LOC{\textsc{loc}}
\newcommand\MOD{\textsc{mod}}
\newcommand\INV{\textsc{inv}}
%\newcommand\L{\textsc{l}}
%\newcommand\CASE{\textsc{case}}
\newcommand\SPR{\textsc{spr}}
\newcommand\COMPS{\textsc{comps}}
%\newcommand\comps{\textsc{comps}}
\newcommand\SEM{\textsc{sem}}
\newcommand\CONT{\textsc{cont}}
\newcommand\SUBCAT{\textsc{subcat}}
\newcommand\CAT{\textsc{cat}}
%\newcommand\C{\textsc{c}}
%\newcommand\SUBJ{\textsc{subj}}
\newcommand\subjjb{\textsc{subj}}
%\newcommand\SLASH{\textsc{slash}}
\newcommand\LOCAL{\textsc{local}}
%\newcommand\ARG-ST{\textsc{arg-st}}
%\newcommand\AGR{\textsc{agr}}
\newcommand\PER{\textsc{per}}
%\newcommand\NUM{\textsc{num}}
%\newcommand\IND{\textsc{ind}}
\newcommand\VFORM{\textsc{vform}}
\newcommand\PFORM{\textsc{pform}}
\newcommand\decl{\textsc{decl}}
%\newcommand\loc{\textsc{loc   }}
% \newcommand\   {\textsc{  }}

%\newcommand\NEG{\textsc{neg}}
\newcommand\FRAMES{\textsc{frames}}
%\newcommand\REFL{\textsc{refl}}

\newcommand\MKG{\textsc{mkg}}

%\newcommand\BN{\textsc{bn}}
\newcommand\HD{\textsc{hd}}
\newcommand\NP{\textsc{np}}
\newcommand\PF{\textsc{pf}}
%\newcommand\PL{\textsc{pl}}
\newcommand\PP{\textsc{pp}}
%\newcommand\SS{\textsc{ss}}
\newcommand\VF{\textsc{vf}}
\newcommand\VP{\textsc{vp}}
%\newcommand\bn{\textsc{bn}}
\newcommand\cl{\textsc{cl}}
%\newcommand\pl{\textsc{pl}}
\newcommand\Wh{\ital{Wh}}
%\newcommand\ng{\textsc{neg}}
\newcommand\wh{\ital{wh}}
%\newcommand\ACC{\textsc{acc}}
%\newcommand\AGR{\textsc{agr}}
\newcommand\AGT{\textsc{agt}}
\newcommand\ARC{\textsc{arc}}
%\newcommand\ARG{\textsc{arg}}
\newcommand\ARP{\textsc{arc}}
%\newcommand\AUX{\textsc{aux}}
%\newcommand\CAT{\textsc{cat}}
%\newcommand\COP{\textsc{cop}}
%\newcommand\DAT{\textsc{dat}}
\newcommand\NEWCOMMAND{\textsc{def}}
%\newcommand\DEL{\textsc{del}}
\newcommand\DOM{\textsc{dom}}
\newcommand\DTR{\textsc{dtr}}
%\newcommand\FUT{\textsc{fut}}
\newcommand\GAP{\textsc{gap}}
%\newcommand\GEN{\textsc{gen}}
%\newcommand\HON{\textsc{hon}}
%\newcommand\IMP{\textsc{imp}}
%\newcommand\IND{\textsc{ind}}
%\newcommand\INV{\textsc{inv}}
\newcommand\LEX{\textsc{lex}}
\newcommand\Lex{\textsc{lex}}
%\newcommand\LOC{\textsc{loc}}
%\newcommand\MOD{\textsc{mod}}
\newcommand\MRK{{\nr MRK}}
%\newcommand\NEG{\textsc{neg}}
\newcommand\NEW{\textsc{new}}
%\newcommand\NOM{\textsc{nom}}
%\newcommand\NUM{\textsc{num}}
%\newcommand\PER{\textsc{per}}
%\newcommand\PST{\textsc{pst}}
\newcommand\QUE{\textsc{que}}
%\newcommand\REL{\textsc{rel}}
\newcommand\SEL{\textsc{sel}}
%\newcommand\SEM{\textsc{sem}}
%\newcommand\SIT{\textsc{arg0}}
%\newcommand\SPR{\textsc{spr}}
%\newcommand\SRC{\textsc{src}}
\newcommand\SUG{\textsc{sug}}
%\newcommand\SYN{\textsc{syn}}
%\newcommand\TPC{\textsc{top}}
%\newcommand\VAL{\textsc{val}}
%\newcommand\acc{\textsc{acc}}
%\newcommand\agt{\textsc{agt}}
\newcommand\cop{\textsc{cop}}
%\newcommand\dat{\textsc{dat}}
\newcommand\foc{\textsc{focus}}
%\newcommand\FOC{\textsc{focus}}
\newcommand\fut{\textsc{fut}}
\newcommand\hon{\textsc{hon}}
\newcommand\imp{\textsc{imp}}
\newcommand\kes{\textsc{kes}}
%\newcommand\lex{\textsc{lex}}
%\newcommand\loc{\textsc{loc}}
\newcommand\mrk{{\nr MRK}}
%\newcommand\nom{\textsc{nom}}
%\newcommand\num{\textsc{num}}
\newcommand\plu{\textsc{plu}}
\newcommand\pne{\textsc{pne}}
%\newcommand\pst{\textsc{pst}}
\newcommand\pur{\textsc{pur}}
%\newcommand\que{\textsc{que}}
%\newcommand\src{\textsc{src}}
%\newcommand\sug{\textsc{sug}}
\newcommand\tpc{\textsc{top}}
%\newcommand\utt{\textsc{utt}}
%\newcommand\val{\textsc{val}}
%% \newcommand\LITE{\textsc{lex}}
%% \newcommand\PAST{\textsc{pst}}
%% \newcommand\POSP{\textsc{pos}}
%% \newcommand\PRS{\textsc{pres}}
%% \newcommand\mod{\textsc{mod}}%
%% \newcommand\newuse{{`kes'}}
%% \newcommand\posp{\textsc{pos}}
%% \newcommand\prs{\textsc{pres}}
%% \newcommand\psp{{\it en\/}}
%% \newcommand\skes{\textsc{kes}}
%% \newcommand\CASE{\textsc{case}}
%% \newcommand\CASE{\textsc{case}}
%% \newcommand\COMP{\textsc{comp}}
%% \newcommand\CONJ{\textsc{conj}}
%% \newcommand\CONN{\textsc{conn}}
%% \newcommand\CONT{\textsc{cont}}
%% \newcommand\DECL{\textsc{decl}}
%% \newcommand\FOCUS{\textsc{focus}}
%% %\newcommand\FORM{\textsc{form}} duplicate
%% \newcommand\FREL{\textsc{frel}}
%% \newcommand\GOAL{\textsc{goal}}
\newcommand\HEAD{\textsc{head}}
%% \newcommand\INDEX{\textsc{ind}}
%% \newcommand\INST{\textsc{inst}}
%% \newcommand\MODE{\textsc{mode}}
%% \newcommand\MOOD{\textsc{mood}}
%% \newcommand\NMLZ{\textsc{nmlz}}
%% \newcommand\PHON{\textsc{phon}}
%% \newcommand\PRED{\textsc{pred}}
%% %\newcommand\PRES{\textsc{pres}}
%% \newcommand\PROM{\textsc{prom}}
%% \newcommand\RELN{\textsc{pred}}
%% \newcommand\RELS{\textsc{rels}}
%% \newcommand\STEM{\textsc{stem}}
%% \newcommand\SUBJ{\textsc{subj}}
%% \newcommand\XARG{\textsc{xarg}}
%% \newcommand\bse{{\it bse\/}}
%% \newcommand\case{\textsc{case}}
%% \newcommand\caus{\textsc{caus}}
%% \newcommand\comp{\textsc{comp}}
%% \newcommand\conj{\textsc{conj}}
%% \newcommand\conn{\textsc{conn}}
%% \newcommand\decl{\textsc{decl}}
%% \newcommand\fin{{\it fin\/}}
%% %\newcommand\form{\textsc{form}}
%% \newcommand\gend{\textsc{gend}}
%% \newcommand\inf{{\it inf\/}}
%% \newcommand\mood{\textsc{mood}}
%% \newcommand\nmlz{\textsc{nmlz}}
%% \newcommand\pass{\textsc{pass}}
%% \newcommand\past{\textsc{past}}
%% \newcommand\perf{\textsc{perf}}
%% \newcommand\pln{{\it pln\/}}
%% \newcommand\pred{\textsc{pred}}


%% %\newcommand\pres{\textsc{pres}}
%% \newcommand\proc{\textsc{proc}}
%% \newcommand\nonfin{{\it nonfin\/}}
%% \newcommand\AGENT{\textsc{agent}}
%% \newcommand\CFORM{\textsc{cform}}
%% %\newcommand\COMPS{\textsc{comps}}
%% \newcommand\COORD{\textsc{coord}}
%% \newcommand\COUNT{\textsc{count}}
%% \newcommand\EXTRA{\textsc{extra}}
%% \newcommand\GCASE{\textsc{gcase}}
%% \newcommand\GIVEN{\textsc{given}}
%% \newcommand\LOCAL{\textsc{local}}
%% \newcommand\NFORM{\textsc{nform}}
%% \newcommand\PFORM{\textsc{pform}}
%% \newcommand\SCASE{\textsc{scase}}
%% \newcommand\SLASH{\textsc{slash}}
%% \newcommand\SLASH{\textsc{slash}}
%% \newcommand\THEME{\textsc{theme}}
%% \newcommand\TOPIC{\textsc{topic}}
%% \newcommand\VFORM{\textsc{vform}}
%% \newcommand\cause{\textsc{cause}}
%% %\newcommand\comps{\textsc{comps}}
%% \newcommand\gcase{\textsc{gcase}}
%% \newcommand\itkes{{\it kes\/}}
%% \newcommand\pass{{\it pass\/}}
%% \newcommand\vform{\textsc{vform}}
%% \newcommand\CCONT{\textsc{c-cont}}
%% \newcommand\GN{\textsc{given-new}}
%% \newcommand\INFO{\textsc{info-st}}
%% \newcommand\ARG-ST{\textsc{arg-st}}
%% \newcommand\SUBCAT{\textsc{subcat}}
%% \newcommand\SYNSEM{\textsc{synsem}}
%% \newcommand\VERBAL{\textsc{verbal}}
%% \newcommand\arg-st{\textsc{arg-st}}
%% \newcommand\plain{{\it plain}\/}
%% \newcommand\propos{\textsc{propos}}
%% \newcommand\ADVERBIAL{\textsc{advl}}
%% \newcommand\HIGHLIGHT{\textsc{prom}}
%% \newcommand\NOMINAL{\textsc{nominal}}

\newenvironment{myavm}{\begingroup\avmvskip{.1ex}
  \selectfont\begin{avm}}%
{\end{avm}\endgroup\medskip}
\newcommand\pfix{\vspace{-5pt}}


\newcommand{\jbsub}[1]{\lower4pt\hbox{\small #1}}
\newcommand{\jbssub}[1]{\lower4pt\hbox{\small #1}}
\newcommand\jbtr{\underbar{\ \ \ }\ }


%\fi

% cl

\newcommand{\delphin}{\textsc{delph-in}}


% YK -- CG chapter

\newcommand{\grey}[1]{\colorbox{mycolor}{#1}}
\definecolor{mycolor}{gray}{0.8}

\newcommand{\GQU}[2]{\raisebox{1.6ex}{\ensuremath{\rotatebox{180}{\textbf{#1}}_{\scalebox{.7}{\textbf{#2}}}}}}

\newcommand{\SetInfLen}{\setpremisesend{0pt}\setpremisesspace{10pt}\setnamespace{0pt}}

\newcommand{\pt}[1]{\ensuremath{\mathsf{#1}}}
\newcommand{\ptv}[1]{\ensuremath{\textsf{\textsl{#1}}}}

\newcommand{\sv}[1]{\ensuremath{\bm{\mathcal{#1}}}}
\newcommand{\sX}{\sv{X}}
\newcommand{\sF}{\sv{F}}
\newcommand{\sG}{\sv{G}}

\newcommand{\syncat}[1]{\textrm{#1}}
\newcommand{\syncatVar}[1]{\ensuremath{\mathit{#1}}}

\newcommand{\RuleName}[1]{\textrm{#1}}

\newcommand{\SemTyp}{\textsf{Sem}}

\newcommand{\E}{\ensuremath{\bm{\epsilon}}\xspace}

\newcommand{\greeka}{\upalpha}
\newcommand{\greekb}{\upbeta}
\newcommand{\greekd}{\updelta}
\newcommand{\greekp}{\upvarphi}
\newcommand{\greekr}{\uprho}
\newcommand{\greeks}{\upsigma}
\newcommand{\greekt}{\uptau}
\newcommand{\greeko}{\upomega}
\newcommand{\greekz}{\upzeta}

\newcommand{\Lemma}{\ensuremath{\hskip.5em\vdots\hskip.5em}\noLine}
\newcommand{\LemmaAlt}{\ensuremath{\hskip.5em\vdots\hskip.5em}}

\newcommand{\I}{\iota}

\newcommand{\sem}{\ensuremath}

\newcommand{\NoSem}{%
\renewcommand{\LexEnt}[3]{##1; \syncat{##3}}
\renewcommand{\LexEntTwoLine}[3]{\renewcommand{\arraystretch}{.8}%
\begin{array}[b]{l} ##1;  \\ \syncat{##3} \end{array}}
\renewcommand{\LexEntThreeLine}[3]{\renewcommand{\arraystretch}{.8}%
\begin{array}[b]{l} ##1; \\ \syncat{##3} \end{array}}}

\newcommand{\hypml}[2]{\left[\!\!#1\!\!\right]^{#2}}

%%%%for bussproof
\def\defaultHypSeparation{\hskip0.1in}
\def\ScoreOverhang{0pt}

\newcommand{\MultiLine}[1]{\renewcommand{\arraystretch}{.8}%
\ensuremath{\begin{array}[b]{l} #1 \end{array}}}

\newcommand{\MultiLineMod}[1]{%
\ensuremath{\begin{array}[t]{l} #1 \end{array}}}

\newcommand{\hypothesis}[2]{[ #1 ]^{#2}}

\newcommand{\LexEnt}[3]{#1; \ensuremath{#2}; \syncat{#3}}

\newcommand{\LexEntTwoLine}[3]{\renewcommand{\arraystretch}{.8}%
\begin{array}[b]{l} #1; \\ \ensuremath{#2};  \syncat{#3} \end{array}}

\newcommand{\LexEntThreeLine}[3]{\renewcommand{\arraystretch}{.8}%
\begin{array}[b]{l} #1; \\ \ensuremath{#2}; \\ \syncat{#3} \end{array}}

\newcommand{\LexEntFiveLine}[5]{\renewcommand{\arraystretch}{.8}%
\begin{array}{l} #1 \\ #2; \\ \ensuremath{#3} \\ \ensuremath{#4}; \\ \syncat{#5} \end{array}}

\newcommand{\LexEntFourLine}[4]{\renewcommand{\arraystretch}{.8}%
\begin{array}{l} \pt{#1} \\ \pt{#2}; \\ \syncat{#4} \end{array}}

\newcommand{\ManySomething}{\renewcommand{\arraystretch}{.8}%
\raisebox{-3mm}{\begin{array}[b]{c} \vdots \,\,\,\,\,\, \vdots \\
\vdots \end{array}}}

\newcommand{\lemma}[1]{\renewcommand{\arraystretch}{.8}%
\begin{array}[b]{c} \vdots \\ #1 \end{array}}

\newcommand{\lemmarev}[1]{\renewcommand{\arraystretch}{.8}%
\begin{array}[b]{c} #1 \\ \vdots \end{array}}

\newcommand{\p}{\ensuremath{\upvarphi}}

% clashes with soul package
\newcommand{\yusukest}{\textbf{\textsf{st}}}

\newcommand{\shortarrow}{\xspace\hskip-1.2ex\scalebox{.5}[1]{\ensuremath{\bm{\rightarrow}}}\hskip-.5ex\xspace}

\newcommand{\SemInt}[1]{\mbox{$[\![ \textrm{#1} ]\!]$}}

\newcommand{\HypSpace}{\hskip-.8ex}
\newcommand{\RaiseHeight}{\raisebox{2.2ex}}
\newcommand{\RaiseHeightLess}{\raisebox{1ex}}

\newcommand{\ThreeColHyp}[1]{\RaiseHeight{\Bigg[}\HypSpace#1\HypSpace\RaiseHeight{\Bigg]}}
\newcommand{\TwoColHyp}[1]{\RaiseHeightLess{\Big[}\HypSpace#1\HypSpace\RaiseHeightLess{\Big]}}

\newcommand{\LemmaShort}{\ensuremath{ \ \vdots} \ \noLine}
\newcommand{\LemmaShortAlt}{\ensuremath{ \ \vdots} \ }

\newcommand{\fail}{**}
\newcommand{\vs}{\raisebox{.05em}{\ensuremath{\upharpoonright}}}
\newcommand{\DerivSize}{\small}

% This is not needed, we just take unicode symbols
% The result of the code below came out wrong anyway.
% St. Mü. 10.06.2021
%
% \def\maru#1{{\ooalign{\hfil
%   \ifnum#1>999 \resizebox{.25\width}{\height}{#1}\else%
%   \ifnum#1>99 \resizebox{.33\width}{\height}{#1}\else%
%   \ifnum#1>9 \resizebox{.5\width}{\height}{#1}\else #1%
%   \fi\fi\fi%
% \/\hfil\crcr%
% \raise.167ex\hbox{\mathhexbox20D}}}}

\newenvironment{samepage2}%
 {\begin{flushleft}\begin{minipage}{\linewidth}}
 {\end{minipage}\end{flushleft}}

\newcommand{\cmt}[1]{\textsl{\textbf{[#1]}}}
\newcommand{\trns}[1]{\textbf{#1}\xspace}
\newcommand{\ptfont}{}
\newcommand{\gp}{\underline{\phantom{oo}}}
\newcommand{\mgcmt}{\marginnote}

\newcommand{\term}[1]{\emph{\isi{#1}}}

\newcommand{\citeposs}[1]{\citeauthor{#1}'s \citeyearpar{#1}}

% for standalone compilations Felix: This is in the class already
%\let\thetitle\@title
%\let\theauthor\@author 
\makeatletter
\newcommand{\togglepaper}[1][0]{ 
\bibliography{../Bibliographies/stmue,../localbibliography,
collection.bib}
  %% hyphenation points for line breaks
%% Normally, automatic hyphenation in LaTeX is very good
%% If a word is mis-hyphenated, add it to this file
%%
%% add information to TeX file before \begin{document} with:
%% %% hyphenation points for line breaks
%% Normally, automatic hyphenation in LaTeX is very good
%% If a word is mis-hyphenated, add it to this file
%%
%% add information to TeX file before \begin{document} with:
%% \include{localhyphenation}
\hyphenation{
A-la-hver-dzhie-va
ac-cu-sa-tive
anaph-o-ra
ana-phor
ana-phors
an-te-ced-ent
an-te-ced-ents
affri-ca-te
affri-ca-tes
ap-proach-es
Atha-bas-kan
Athe-nä-um
Be-schrei-bung
Bona-mi
Chi-che-ŵa
com-ple-ments
con-straints
Cope-sta-ke
Da-ge-stan
Dor-drecht
er-klä-ren-de
Flick-inger
Ginz-burg
Gro-ning-en
Has-pel-math
Jap-a-nese
Jon-a-than
Ka-tho-lie-ke
Ko-bon
krie-gen
Kroe-ger
Le-Sourd
moth-er
Mül-ler
Nie-mey-er
Ørs-nes
Par-a-digm
Prze-piór-kow-ski
phe-nom-e-non
re-nowned
Rie-he-mann
un-bound-ed
Ver-gleich
with-in
}

% listing within here does not have any effect for lfg.tex % 2020-05-14

% why has "erklärende" be listed here? I specified langid in bibtex item. Something is still not working with hyphenation.


% to do: check
%  Alahverdzhieva


% biblatex:

% This is a LaTeX frontend to TeX’s \hyphenation command which defines hy- phenation exceptions. The ⟨language⟩ must be a language name known to the babel/polyglossia packages. The ⟨text ⟩ is a whitespace-separated list of words. Hyphenation points are marked with a dash:

% \DefineHyphenationExceptions{american}{%
% hy-phen-ation ex-cep-tion }

\hyphenation{
A-la-hver-dzhie-va
ac-cu-sa-tive
anaph-o-ra
ana-phor
ana-phors
an-te-ced-ent
an-te-ced-ents
affri-ca-te
affri-ca-tes
ap-proach-es
Atha-bas-kan
Athe-nä-um
Be-schrei-bung
Bona-mi
Chi-che-ŵa
com-ple-ments
con-straints
Cope-sta-ke
Da-ge-stan
Dor-drecht
er-klä-ren-de
Flick-inger
Ginz-burg
Gro-ning-en
Has-pel-math
Jap-a-nese
Jon-a-than
Ka-tho-lie-ke
Ko-bon
krie-gen
Kroe-ger
Le-Sourd
moth-er
Mül-ler
Nie-mey-er
Ørs-nes
Par-a-digm
Prze-piór-kow-ski
phe-nom-e-non
re-nowned
Rie-he-mann
un-bound-ed
Ver-gleich
with-in
}

% listing within here does not have any effect for lfg.tex % 2020-05-14

% why has "erklärende" be listed here? I specified langid in bibtex item. Something is still not working with hyphenation.


% to do: check
%  Alahverdzhieva


% biblatex:

% This is a LaTeX frontend to TeX’s \hyphenation command which defines hy- phenation exceptions. The ⟨language⟩ must be a language name known to the babel/polyglossia packages. The ⟨text ⟩ is a whitespace-separated list of words. Hyphenation points are marked with a dash:

% \DefineHyphenationExceptions{american}{%
% hy-phen-ation ex-cep-tion }

  \memoizeset{
    memo filename prefix={hpsg-handbook.memo.dir/},
    % readonly
  }
  \papernote{\scriptsize\normalfont
    \@author.
    \titleTemp. 
    To appear in: 
    Stefan Müller, Anne Abeillé, Robert D. Borsley \& Jean-Pierre Koenig (eds.)
    HPSG Handbook
    Berlin: Language Science Press. [preliminary page numbering]
  }
  \pagenumbering{roman}
  \setcounter{chapter}{#1}
  \addtocounter{chapter}{-1}
}
\makeatother

\makeatletter
\newcommand{\togglepaperminimal}[1][0]{ 
  \bibliography{../Bibliographies/stmue,
                ../localbibliography,
collection.bib}
  %% hyphenation points for line breaks
%% Normally, automatic hyphenation in LaTeX is very good
%% If a word is mis-hyphenated, add it to this file
%%
%% add information to TeX file before \begin{document} with:
%% %% hyphenation points for line breaks
%% Normally, automatic hyphenation in LaTeX is very good
%% If a word is mis-hyphenated, add it to this file
%%
%% add information to TeX file before \begin{document} with:
%% \include{localhyphenation}
\hyphenation{
A-la-hver-dzhie-va
ac-cu-sa-tive
anaph-o-ra
ana-phor
ana-phors
an-te-ced-ent
an-te-ced-ents
affri-ca-te
affri-ca-tes
ap-proach-es
Atha-bas-kan
Athe-nä-um
Be-schrei-bung
Bona-mi
Chi-che-ŵa
com-ple-ments
con-straints
Cope-sta-ke
Da-ge-stan
Dor-drecht
er-klä-ren-de
Flick-inger
Ginz-burg
Gro-ning-en
Has-pel-math
Jap-a-nese
Jon-a-than
Ka-tho-lie-ke
Ko-bon
krie-gen
Kroe-ger
Le-Sourd
moth-er
Mül-ler
Nie-mey-er
Ørs-nes
Par-a-digm
Prze-piór-kow-ski
phe-nom-e-non
re-nowned
Rie-he-mann
un-bound-ed
Ver-gleich
with-in
}

% listing within here does not have any effect for lfg.tex % 2020-05-14

% why has "erklärende" be listed here? I specified langid in bibtex item. Something is still not working with hyphenation.


% to do: check
%  Alahverdzhieva


% biblatex:

% This is a LaTeX frontend to TeX’s \hyphenation command which defines hy- phenation exceptions. The ⟨language⟩ must be a language name known to the babel/polyglossia packages. The ⟨text ⟩ is a whitespace-separated list of words. Hyphenation points are marked with a dash:

% \DefineHyphenationExceptions{american}{%
% hy-phen-ation ex-cep-tion }

\hyphenation{
A-la-hver-dzhie-va
ac-cu-sa-tive
anaph-o-ra
ana-phor
ana-phors
an-te-ced-ent
an-te-ced-ents
affri-ca-te
affri-ca-tes
ap-proach-es
Atha-bas-kan
Athe-nä-um
Be-schrei-bung
Bona-mi
Chi-che-ŵa
com-ple-ments
con-straints
Cope-sta-ke
Da-ge-stan
Dor-drecht
er-klä-ren-de
Flick-inger
Ginz-burg
Gro-ning-en
Has-pel-math
Jap-a-nese
Jon-a-than
Ka-tho-lie-ke
Ko-bon
krie-gen
Kroe-ger
Le-Sourd
moth-er
Mül-ler
Nie-mey-er
Ørs-nes
Par-a-digm
Prze-piór-kow-ski
phe-nom-e-non
re-nowned
Rie-he-mann
un-bound-ed
Ver-gleich
with-in
}

% listing within here does not have any effect for lfg.tex % 2020-05-14

% why has "erklärende" be listed here? I specified langid in bibtex item. Something is still not working with hyphenation.


% to do: check
%  Alahverdzhieva


% biblatex:

% This is a LaTeX frontend to TeX’s \hyphenation command which defines hy- phenation exceptions. The ⟨language⟩ must be a language name known to the babel/polyglossia packages. The ⟨text ⟩ is a whitespace-separated list of words. Hyphenation points are marked with a dash:

% \DefineHyphenationExceptions{american}{%
% hy-phen-ation ex-cep-tion }

  \memoizeset{
    memo filename prefix={hpsg-handbook.memo.dir/},
    % readonly
  }
  \papernote{\scriptsize\normalfont
    \@author.
    \@title. 
    To appear in: 
    Stefan Müller, Anne Abeillé, Robert D. Borsley \& Jean-Pierre Koenig (eds.)
    HPSG Handbook
    Berlin: Language Science Press. [preliminary page numbering]
  }
  \pagenumbering{roman}
  \setcounter{chapter}{#1}
  \addtocounter{chapter}{-1}
}
\makeatother




% In case that year is not given, but pubstate. This mainly occurs for titles that are forthcoming, in press, etc.
\renewbibmacro*{addendum+pubstate}{% Thanks to https://tex.stackexchange.com/a/154367 for the idea
  \printfield{addendum}%
  \iffieldequalstr{labeldatesource}{pubstate}{}
  {\newunit\newblock\printfield{pubstate}}
}

\DeclareLabeldate{%
    \field{date}
    \field{year}
    \field{eventdate}
    \field{origdate}
    \field{urldate}
    \field{pubstate}
    \literal{nodate}
}

%\defbibheading{diachrony-sources}{\section*{Sources}} 

% if no langid is set, it is English:
% https://tex.stackexchange.com/a/279302
\DeclareSourcemap{
  \maps[datatype=bibtex]{
    \map{
      \step[fieldset=langid, fieldvalue={english}]
    }
  }
}


% for bibliographies
% biber/biblatex could use sortname field rather than messing around this way.
\newcommand{\SortNoop}[1]{}


% Doug Ball

\newcommand{\elist}{\q<\ \ \q>}

\newcommand{\esetDB}{\q\{\ \ \q\}}


\makeatletter

\newcommand{\nolistbreak}{%

  \let\oldpar\par\def\par{\oldpar\nobreak}% Any \par issues a \nobreak

  \@nobreaktrue% Don't break with first \item

}

\makeatother


% intermediate before Frank's trees are fixed
% This will be removed!!!!!
%\newcommand{\tree}[1]{} % ignore them blody trees
%\usepackage{tree-dvips}


\newcommand{\nodeconnect}[2]{}
\newcommand{\nodetriangle}[2]{}



% Doug relative clauses
%% I've compiled out almost all my private LaTeX command, but there are some
%% I found hard to get rid of. They are defined here.
%% There are few others which defined in places in the document where they have only
%% local effect (e.g. within figures); their names all end in DA, e.g. \MotherDA
%% There are a lot of \labels -- they are all of the form \label{sec:rc-...} or
%% \label{x:rc-...} or similar, so there should be no clashes.

% Subscripts -- scriptsize italic shape lowered by .25ex 
\newcommand{\subscr}[1]{\raisebox{-.5ex}{\protect{\scriptsize{\itshape #1\/}}}}
% A boxed subscript, for avm tags in normal text
\newcommand{\subtag}[1]{\subscr{\idx{#1}}}

%% Sets and tuples: I use \setof{} to get brackets that are upright, not slanted
%\newcommand{\setof}[1]{\ensuremath{\lbrace\,\mathit{#1}\,\rbrace}}
% 11.10.2019 EP: Doug requested replacement of existing \setof definition with the following:
%\newcommand{\setof}[1]{\begin{avm}\{\textcolor{red}{#1}\}\end{avm}}
% 31.1.2019 EP: Doug requested re-replacement of the above \textcolour version with the following:
\newcommand{\setof}[1]{\begin{avm}\{#1\}\end{avm}}

\newcommand{\tuple}[1]{\ensuremath{\left\langle\,\mbox{\textit{#1}}\,\right\rangle}}

% Single pile of stuff, optional arugment is psn (e.g. t or b)
% e.g. to put a over b over c in a centered column, top aligned, do:
%   \cPile[t]{a\\b\\c} 
\newcommand{\cPile}[2][]{%
  \begingroup%
  \renewcommand{\arraystretch}{.5}\begin{tabular}[#1]{@{}c@{}}#2\end{tabular}%
  \endgroup%
}

%% for linguistic examples in running text (`linguistic citation'):
\newcommand{\lic}[1]{\textit{#1}}

%% A gap marked by an underline, raised slightly
%% Default argument indicates how long the line should be:
\newcommand{\uGap}[1][3ex]{\raisebox{.25em}{\underline{\hspace{#1}}}\xspace}

%% \TnodeDA{XP}{avmcontents} -- in a Tree, put a node label next to an AVM
\newcommand{\TnodeDA}[2]{#1~\begin{avm}{#2}\end{avm}}

%% This allows tipa stuff to be put in \emph -- we need to change to cmr first.
%% It is used in the discussion of Arabic.
\newcommand{\emphtipa}[1]{{\fontfamily{cmr}\emph{\tipaencoding #1}}} 



 
 
\definecolor{lsDOIGray}{cmyk}{0,0,0,0.45}


% morphology.tex:
% Berthold

\newcommand{\dnode}[1]{\rnode{#1}{\fbox{#1}}}
\newcommand{\tnode}[1]{\rnode{#1}{\textit{#1}}}

\newcommand{\tl}[2]{#2}

\newcommand{\rrr}[3]{%
  \psframebox[linestyle=none]{%
    \avmoptions{center}
    \begin{avm}
      \[mud & \{ #1 \}\\
      ms & \{ #2 \}\\
      mph & \<  #3 \> \]
    \end{avm}
  }
}
\newcommand{\rr}[2]{%
  \psframebox[linestyle=none]{%
    \avmoptions{center}
    \begin{avm}
      \[mud & \{ #1 \}\\
      mph & \<  #2 \> \]
    \end{avm}
  }
}
 

% Frank Richter
\newtheorem{mydef}{Definition}

\long\def\set[#1\set=#2\set]%
{%
\left\{%
\tabcolsep 1pt%
\begin{tabular}{l}%
#1%
\end{tabular}%
\left|%
\tabcolsep 1pt%
\begin{tabular}{l}%
#2%
\end{tabular}%
\right.%
\right\}%
}

\newcommand{\einruck}{\\ \hspace*{1em}}


%\newcommand{\NatNum}{\mathrm{I\hspace{-.17em}N}}
\newcommand{\NatNum}{\mathbb{N}}
\newcommand{\Aug}[1]{\widehat{#1}}
%\newcommand{\its}{\mathrm{:}}
% Felix 14.02.2020
\DeclareMathOperator{\its}{:}

\newcommand{\sequence}[1]{\langle#1\rangle}

\newcommand{\INTERPRETATION}[2]{\sequence{#1\mathsf{U}#2,#1\mathsf{S}#2,#1\mathsf{A}#2,#1\mathsf{R}#2}}
\newcommand{\Interpretation}{\INTERPRETATION{}{}}

\newcommand{\Inte}{\mathsf{I}}
\newcommand{\Unive}{\mathsf{U}}
\newcommand{\Speci}{\mathsf{S}}
\newcommand{\Atti}{\mathsf{A}}
\newcommand{\Reli}{\mathsf{R}}
\newcommand{\ReliT}{\mathsf{RT}}

\newcommand{\VarInt}{\mathsf{G}}
\newcommand{\CInt}{\mathsf{C}}
\newcommand{\Tinte}{\mathsf{T}}
\newcommand{\Dinte}{\mathsf{D}}

% this was missing from ash's stuff.

%% \def \optrulenode#1{
%%   \setbox1\hbox{$\left(\hbox{\begin{tabular}{@{\strut}c@{\strut}}#1\end{tabular}}\right)$}
%%   \raisebox{1.9ex}{\raisebox{-\ht1}{\copy1}}}



\newcommand{\pslabel}[1]{}

\newcommand{\addpagesunless}{\todostefan{add pages unless you cite the
 work as such}}

% dg.tex
% framed boxes as used in dg.tex
% original idea from stackexchange, but modified by Saso
% http://tex.stackexchange.com/questions/230300/doing-something-like-psframebox-in-tikz#230306
\tikzset{
  frbox/.style={
    rounded corners,
    draw,
    thick,
    inner sep=5pt,
    anchor=base,
  },
}

% get rid of these morewrite messages:
% https://tex.stackexchange.com/questions/419489/suppressing-messages-to-standard-output-from-package-morewrites/419494#419494
\ExplSyntaxOn
\cs_set_protected:Npn \__morewrites_shipout_ii:
  {
    \__morewrites_before_shipout:
    \__morewrites_tex_shipout:w \tex_box:D \g__morewrites_shipout_box
    \edef\tmp{\interactionmode\the\interactionmode\space}\batchmode\__morewrites_after_shipout:\tmp
  }
\ExplSyntaxOff


% This is for places where authors used bold. I replace them by \emph
% but have the information where the bold was. St. Mü. 09.05.2020
\newcommand{\textbfemph}[1]{\emph{#1}}



% Felix 09.06.2020: copy code from the third line into localcommands.tex:
% https://github.com/langsci/langscibook#defined-environments-commands-etc
% Does not work with texlive 2020, is done with sed in Makefile
%\patchcmd{\mkbibindexname}{\ifdefvoid{#3}{}{\MakeCapital{#3} }}{\ifdefvoid{#3}{}{#3 }}{}{\AtEndDocument{\typeout{mkbibindexname could not be patched.}}}



\let\textnobf\textit
% instead of "in bold" write "in italics"
\newcommand{\bolddescriptionintext}{italics\xspace}

% Berthold
\newcommand{\mathplus}{+}
% \mbox{\normalfont +}}
\newcommand{\emdash}{--\xspace}
\newcommand{\emdashUS}{--\xspace}


% Stefan to get the space remvoed infront of the : in Bargmann NPN discussion
%\DeclareMathSymbol{:}{\mathord}{operators}{"3A}
% used {:\,} instead


% for cxg.tex needed for includonly to find the counter.
\newcounter{croftyears} 




% Needed for bibtex entry for Jackendoff's xbar syntax. Without it the bar would be off in itialics.

% https://tex.stackexchange.com/questions/95014/aligning-overline-to-italics-font/95079#95079
% \newbox\usefulbox

% \makeatletter
%     \def\getslant #1{\strip@pt\fontdimen1 #1}

%     \def\skoverline #1{\mathchoice
%      {{\setbox\usefulbox=\hbox{$\m@th\displaystyle #1$}%
%         \dimen@ \getslant\the\textfont\symletters \ht\usefulbox
%         \divide\dimen@ \tw@ 
%         \kern\dimen@ 
%         \overline{\kern-\dimen@ \box\usefulbox\kern\dimen@ }\kern-\dimen@ }}
%      {{\setbox\usefulbox=\hbox{$\m@th\textstyle #1$}%
%         \dimen@ \getslant\the\textfont\symletters \ht\usefulbox
%         \divide\dimen@ \tw@ 
%         \kern\dimen@ 
%         \overline{\kern-\dimen@ \box\usefulbox\kern\dimen@ }\kern-\dimen@ }}
%      {{\setbox\usefulbox=\hbox{$\m@th\scriptstyle #1$}%
%         \dimen@ \getslant\the\scriptfont\symletters \ht\usefulbox
%         \divide\dimen@ \tw@ 
%         \kern\dimen@ 
%         \overline{\kern-\dimen@ \box\usefulbox\kern\dimen@ }\kern-\dimen@ }}
%      {{\setbox\usefulbox=\hbox{$\m@th\scriptscriptstyle #1$}%
%         \dimen@ \getslant\the\scriptscriptfont\symletters \ht\usefulbox
%         \divide\dimen@ \tw@ 
%         \kern\dimen@ 
%         \overline{\kern-\dimen@ \box\usefulbox\kern\dimen@ }\kern-\dimen@ }}%
%      {}}
%     \makeatother




\newcommand{\acknowledgmentsEN}{Acknowledgements}
\newcommand{\acknowledgmentsUS}{Acknowledgments}

% to put two examples next to eachother
%\newcommand{\shortbox}[3][-.7]{
%    \parbox[t]{.4\textwidth}{
%      \vspace{#1\baselineskip} #2\strut~~ #3}%
%}

\newcommand{\twomulticolexamples}[2]{
\begin{tabular}[t]{@{}l@{~~}l@{\hspace{1em}}l@{~~}l@{}}
a. & \parbox[t]{.4\textwidth}{#1} & b. & \parbox[t]{.4\textwidth}{#2}\\
\end{tabular}
}




% This does a linebreak for \gll for long sentences leaving space for the language at the right
% margin.
% St.Mü. 17.06.2021
\newcommand{\longexampleandlanguage}[2]{%
\begin{tabularx}{\linewidth}[t]{@{}X@{}p{\widthof{(#2)}}@{}}%
\begin{minipage}[t]{\linewidth}%
#1%
\end{minipage} & (\ili{#2})%
\end{tabularx}}



\renewcommand{\indexccg}{\is{Categorial Grammar (CG)!Combinatorial \textasciitilde{} (CCG)}\xspace}
\newcommand{\indexccgstart}{\is{Categorial Grammar (CG)!Combinatorial \textasciitilde{} (CCG)|(}\xspace}
\newcommand{\indexccgend}{\is{Categorial Grammar (CG)!Combinatorial \textasciitilde{} (CCG)|)}\xspace}
\renewcommand{\indexmp}{\is{Minimalism}\xspace}


\newcommand{\gisu}{Giuseppe Varaschin\xspace}

\newcommand{\NPi}{NP$\mkern-1mu_i$\xspace}
\newcommand{\NPj}{NP$\mkern-1.5mu_j$\xspace}
  %% -*- coding:utf-8 -*-

%%%%%%%%%%%%%%%%%%%%%%%%%%%%%%%%%%%%%%%%%%%%%%%%%%%%%%%%%%%%
%
% gb4e

% fixes problem with to much vertical space between \zl and \eal due to the \nopagebreak
% command.
\makeatletter
\def\@exe[#1]{\ifnum \@xnumdepth >0%
                 \if@xrec\@exrecwarn\fi%
                 \if@noftnote\@exrecwarn\fi%
                 \@xnumdepth0\@listdepth0\@xrectrue%
                 \save@counters%
              \fi%
                 \advance\@xnumdepth \@ne \@@xsi%
                 \if@noftnote%
                        \begin{list}{(\thexnumi)}%
                        {\usecounter{xnumi}\@subex{#1}{\@gblabelsep}{0em}%
                        \setcounter{xnumi}{\value{equation}}}
% this is commented out here since it causes additional space between \zl and \eal 06.06.2020
%                        \nopagebreak}%
                 \else%
                        \begin{list}{(\roman{xnumi})}%
                        {\usecounter{xnumi}\@subex{(iiv)}{\@gblabelsep}{\footexindent}%
                        \setcounter{xnumi}{\value{fnx}}}%
                 \fi}
\makeatother

% the texlive 2020 langsci-gb4e adds a newline after \eas, the texlive 2017 version was OK.
% \makeatletter
% \def\eas{\ifnum\@xnumdepth=0\begin{exe}[(34)]\else\begin{xlist}[iv.]\fi\ex\begin{tabular}[t]{@{}p{.98\linewidth}@{}}}
% \makeatother



%%%%%%%%%%%%%%%%%%%%%%%%%%%%%%%%%%%%%%%%%%%%%%%%%%%%%%%%%%
%
% biblatex

% biblatex sets the option autolang=hyphens
%
% This disables language shorthands. To avoid this, the hyphens code can be redefined
%
% https://tex.stackexchange.com/a/548047/18561

\makeatletter
\def\hyphenrules#1{%
  \edef\bbl@tempf{#1}%
  \bbl@fixname\bbl@tempf
  \bbl@iflanguage\bbl@tempf{%
    \expandafter\bbl@patterns\expandafter{\bbl@tempf}%
    \expandafter\ifx\csname\bbl@tempf hyphenmins\endcsname\relax
      \set@hyphenmins\tw@\thr@@\relax
    \else
      \expandafter\expandafter\expandafter\set@hyphenmins
      \csname\bbl@tempf hyphenmins\endcsname\relax
    \fi}}
\makeatother


% the package defined \attop in a way that produced a box that has textwidth
%
\def\attop#1{\leavevmode\begin{minipage}[t]{.995\linewidth}\strut\vskip-\baselineskip\begin{minipage}[t]{.995\linewidth}#1\end{minipage}\end{minipage}}


%%%%%%%%%%%%%%%%%%%%%%%%%%%%%%%%%%%%%%%%%%%%%%%%%%%%%%%%%%%%%%%%%%%%


% Don't do this at home. I do not like the smaller font for captions.
% This does not work. Throw out package caption in langscibook
% \captionsetup{%
% font={%
% stretch=1%.8%
% ,normalsize%,small%
% },%
% width=\textwidth%.8\textwidth
% }
% \setcaphanging


  \togglepaper[20]
}{}



\title{Anaphoric binding} 
\author{%
Stefan Müller\affiliation{Humboldt-Universität zu Berlin}
}
% \chapterDOI{} %will be filled in at production

% \epigram{}

\abstract{
This chapter is an introduction into the Binding Theory assumed within HPSG. While it was inspired
by work on Government \& Binding (GB), a key insight of HPSG's Binding Theory is that contrary to GB's
Binding Theory, reference to tree structures
alone is not sufficient and that reference to the syntactic level of argument structure is
required. Since the argument structure is tightly related to the thematic structure, HPSG's Binding
Theory is a mix of aspects of thematic binding theories and entirely configurational theories. This
chapter discusses the advantages of this new view and the development to a strongly lexical binding
theory as a result of shortcomings of earlier approaches. The chapter also addresses so-called
exempt anaphors, that is, anaphors not bound inside of the clause or another local domain. 
}


\begin{document}
\maketitle
\label{chap-binding}

%\if0
\section{Introduction} 

%\itdopt{Add (ref to dalrymple on scandinavian and slavic, butt on indian…)}

\itdopt{Cite: In Pursuit of Condition C: (non-)coreference in grammar, discourse and processing}

% \itddone{Anne: This is supposed to explain why HPSG develops its own binding theory\\
% Stefan: I saw the general setup differently. I did not want to confront the MGG people but rather
% say what Binding Theory has to account for. And of course we have to have a Binding Theory. We will see on the way, where
% other theories have problems.\\
% %
% If you present ``classical'' syntactic theory, you should also point out its weakness: c command does not account for:
% LDD: a. Himself, John likes
% Ditransitives: b. Mary talked to John about himself.
% c. *Mary talked to himself about John.
% This is in section2 (fig 4), should be in section 1

% Please say something about crossling variation; many lg have long distance refl, not just Chinese
% and Portuguese (ref to dalrymple on scandinavian and slavic, butt on indian…) and the fact that
% non-reflexive pers pro cannot have a local antecedent is very English specific (ok in Romance:
% Paul-i pense à lui-i/j (p thinks about him(self)), Paolo-i parla di lui-i/j (p talks about
% him(self), etc) (in Romance only weak =clitic pers pro are constrained, not strong ones)
% }

Binding Theories deal with questions of semantic identity and \isi{agreement} of coreferring
items. For example, the reflexives in (\mex{1}) must corefer, and agree in \isi{gender}, with a
coargument:
\itdobl{Steve:  Coreference is not binding. Order of explanation perhaps should be: what is binding; relation of binding to coreference; pronouns agree with their antecedents; how BT deals with the distribution of binding relations.}
\eal
\label{ex-binding-reflexives}
\ex[]{
Peter$_i$ thinks that Mary$_j$ likes herself$_{*i/j/*k}$.
}
\ex[*]{
Peter$_i$ thinks that Mary$_j$ likes himself$_{*i/*j/*k}$.
}
\ex[*]{
Mary$_i$ thinks that Peter$_j$ likes herself$_{*i/*j/*k}$.
}
\ex[]{
Mary$_i$ thinks that Peter$_j$ likes himself$_{*i/j/*k}$.
}
\zl
The indices show what bindings are possible and which ones are ruled out. For example, in
(\mex{0}a), \emph{herself} cannot refer to \emph{Peter}, it can refer to \emph{Mary} and it cannot
refer to some discourse referent that is not mentioned in the sentence (indicated by the index
$k$). Coreference of \emph{himself} and \emph{Mary} is ruled out in (\mex{0}b), since \emph{himself}
has an incompatible gender.

At first look it may seem possible to account for the binding relations of reflexives on the
\itdobl{Steve: I think it would be better to first present the theory, then note that it works for raised arguments and note that this contrasts with semantic theories.}
semantic level \citep{BP80a}: it seems to be the case that reflexives and their antecedents have to be semantic
arguments of the same predicate.\footnote{%
  See \citet{Riezler95a} for a way to formalize this in HPSG. See \citew{RR93a} for an approach to
  Binding mixing constraints on the semantic and syntactic level.% p. 678
} For examples like (\mex{0}), this makes the right predictions,
\itdobl{Steve: No antecedent for ‘this’, i.e. no predictive theory has been provided; instead there is a statement of what ‘seems to be the case’.}
since the reflexive is the undergoer of \emph{likes} and the only possible antecedent is the actor of
\emph{likes}.\footnote{
  See \citew{Dowty91a} and \citew{VanValin99a-u} on semantic roles. Dowty suggested role labels like
  proto-agent and proto-patient and VanValin the labels actor and undergoer. We use the latter
  here. See also \crossrefchapterw[Section~\ref{arg-st:sec-hpsg-approaches-to-linking}]{arg-st} on
  actor and undergoer and linking in HPSG.
} However, there are raising predicates like \emph{believe} not assigning semantic roles
to their objects but nevertheless allowing coreference of the raised element and the subject of
\emph{believe} \citep[\page 128]{MS98a}:\footnote{%
 See \citew[Chapter~3.5]{ps2} and \crossrefchapterw{control-raising} on raising. See also \citew[\page 679]{RR93a} on Binding
 Theory and raising.%
}
\itdobl{G: Thematic approaches should be distinguished from semantic approaches like Bach \& Partee’s.
You’re really talking about thematic approaches here. The basic references for that are
\citet{Jackendoff72a-u}, which you cite below, \citew{Wilkins1988a-u} and \citew{Williams1994a-u}. What you mention here isn't really a big problem for semantic approaches of the Bach \& Partee variety. They can assign a lambda term as the meaning of “believe” in such a way that the raised object winds up being one of its arguments, as is often done in CG:$
\lambda x[\lambda P[\lambda z[believe(P(x))(z)]]]$
} 

\ea
John$_i$ believes himself$_i$ to be a descendant of Beethoven.
\z
The fact that \emph{believes} does not assign a semantic role to its object is confirmed by the
possibility to embed predicates with an expletive subject under \emph{believe}:\footnote{%
The example is due to \citet[\page 137]{ps2}. See the sources cited above for further discussion.
}
\ea
Kim believed there to be some misunderstanding about these issues.
\z
So, it really is the clause or -- to be more precise -- some syntactically defined local domain in
which reflexive pronouns have to be bound provided \changed{the structure is such that an
  appropriate antecedent could be available in principle}.\footnote{%
Another argument against a binding theory relying exclusively on semantics involves different binding
behavior in active and \isi{passive} sentences: since the semantic contribution is the same for active
and passive sentences, the difference in binding options cannot be explained in semantics-based
approaches. Binding and passive is discussed more thoroughly in Section~\ref{binding-sec-passive}. 
For a general discussion of
thematic approaches to binding see \citew[Section~8]{PS92a} and \citew[Section~6.8.2]{ps2}.
}
\itdobl{Steve: This doesn't seem quite right. The rule is not always conditioned on the availability of an appropriate antecedent, as shown by 1b. Maybe better to replace “provided an appropriate antecedent is available” with “(with some exemptions discussed in Section XX)”.
}
In cases like (\mex{1}), no antecedent is available within the clause and in such situations a
reflexive may be bound by an element outside the clause.
\eanoraggedright
John$_i$ was going to get even with Mary. That picture of himself$_i$
in the paper would really annoy her, as would the other stunts he had planned.\footnote{
        \citew[\page 270]{ps2}.
}
\z
Reflexives without an element that could function as a binder in a certain local domain are regarded
as exempt from Binding Theory. Section~\ref{sec-excempt-anaphors} deals with so-called exempt anaphors in more detail.

Personal pronouns cannot \changed{bind} an antecedent within the same domain of locality in English:
% Clause is too strong: \emph{Mary's mother likes her.}
\eal
\ex[]{
Peter$_i$ thinks that Mary$_j$ likes her$_{*i/*j/k}$.
}
\ex[]{
Peter$_i$ thinks that Mary$_j$ likes him$_{i/*j/k}$.
}
\ex[]{
Mary$_i$ thinks that Peter$_j$ likes her$_{i/*j/k}$.
}
\ex[]{
Mary$_i$ thinks that Peter$_j$ likes him$_{*i/*j/k}$.
}
\zl
As the examples show, the pronouns \emph{her} and \emph{him} cannot be coreferent with the subject
of \emph{likes}. If a speaker wants to express coreference, he or she has to use a reflexive pronoun
as in (\ref{ex-binding-reflexives}). 

Interestingly, the binding of pronouns is less restricted than that of reflexives, but this does
not mean that anything goes. For example, a pronoun cannot bind a full referential NP if the NP is
embedded in a complement clause and the pronoun is in the matrix clause:
\eal
\label{ex-he-thinks-that-Peter}
\ex[]{
He$_{*i/*j/k}$ thinks that Mary$_i$ likes Peter$_j$.
}
\ex[]{
He$_{*i/*j/k}$ thinks that Peter$_i$ likes Mary$_j$.
}
\zl  

The sentences discussed so far can be assigned a structure like the one in Figure~\ref{fig-binding-gb}.
\begin{figure}
\begin{forest}
sm edges without translation
[S
  [NP [John$_i$\\John$_i$\\he$_{*i/*j/k}$]]
  [VP
    [V [thinks\\thinks\\thinks]]
    [CP 
      [C [that\\that\\that]]
      [S
        [NP [Paul$_j$\\Paul$_j$\\Mary$_i$]]
        [VP
         [V [likes\\likes\\likes]]
         [NP [him$_{i/*j/k}$\\himself$_{*i/j/*k}$\\Peter$_j$]]]]]]]
\end{forest}
\itddone{Anne: Would be easier if you labeled the NPs as NP1, NP2, NP3.\\
Stefan: not necessary since not referenced in the text. But I added the indices to the words.}
\caption{\label{fig-binding-gb}Tree configuration of examples for binding}
\end{figure}
\textcites[Section~3.2]{Chomsky81a}[Section~3]{Chomsky86a} suggested that tree"=configurational properties play a role in
accounting for binding facts. He uses the notion of c(onstituent)-command going back to
work by \citet{Reinhart76a-u}. \isi{c-command} is a relation that holds between nodes in a
tree. According to one definition, a node c-commands its sisters and the constituents of its sisters.%
%% Y is said to c-command another node Z, iff Y and Z
%% are sisters or if a sister of Y dominates Z.
\footnote{\label{fn-c-command-GB}%
``Node A c(onstituent)-commands node B if neither A nor B dominates the other and the first
  branching node which dominates A dominates B.'' \citet[\page 32]{Reinhart76a-u}

\citet{Chomsky86a} uses another definition that allows one to go up to the next maximal projection
dominating A. As of 2020-02-25 the \ili{English} and \ili{German} Wikipedia pages for c-command have two
conflicting definitions of c-command. The \ili{English} version follows \citet[\page 168]{SKS2013a-u}, whose
definition excludes c-command between sisters: ``Node X c-commands node Y if a sister of X dominates
Y.''
\itdopt{G: This is asymmetric c-command, which is really important in Kayne (1994).}
}

To take an example, the NP node of \emph{John} c-commands all other nodes dominated by S. The V of
\emph{thinks} c-commands everything within the CP including the CP node, the C of \emph{that}
c-commands all nodes in S including also S and so on. The CP c-commands the \emph{think}-V, and the
\emph{likes him}-VP c-commands the \emph{Paul}-NP. Per definition, a Y binds Z just in case Y and Z
are coindexed and Y c-commands Z. One precondition for being coindexed (in \ili{English}) is that the
person, number, and gender features of the involved items are compatible. Coindexing can be
established with all kinds of nominal expressions including quantified ones and negated NPs like
\emph{no animal} (see \citealp[\page 128--129]{BP80a}).
\itdobl{Steve: ‘coindexed’ needs to be introduced first, and its relation to coreference needs to be explained.}

Now, the goal is to find restrictions that ensure that English reflexives are bound locally, personal
pronouns are not bound locally and that referential expressions like proper names and full NPs are
not bound by other expressions (anaphors, personal pronouns or fully referential expressions).
\itdobl{Steve: Both ‘bound’ and ‘locally’ need to be introduced, defined, explained. Also ‘anaphors’ and ‘fully referential expressions’. Are pronouns ‘fully referential’?}
 The conditions that were developed for
GB's Binding Theory are complex. They also account for the binding of traces that are the result of
moving elements by transformations (\citealp{Chomsky81a}, but given up in
\citealp{Chomsky86a}).\label{page-traces-binding} While it is elegant to subsume filler-gap relations (and other relations
between moved items and their traces) under a general Binding Theory, proponents of HPSG think that
coindexed semantic indices and filler-gap
\itdobl{Steve:  This discussion seems out of place. It would be better to introduce and describe HPSG binding theory first, on its own terms, and then compare it to GB and motivate it.}
%% \inlinetodostefan{Bob:
%% For GB binding theiry was concerned not just with A'-movement traces but also with A-movement traces, hence not just with filler-gap relations.  (1) and (2) were assumed to be bad for the sae reason as (3)
%% %
%% 1) *Kim seems [t is clever]
%% 2) *Kim is believed [t is clever]
%% 3) *Kim believes [himself is clever]
%% %
%% And of course all reflexives and reciprocals were assumed to be subject to binding theory, a position rejected in HPSG.
%% }
dependencies are crucially different.\footnote{\label{binding-fn-percolation-of-indices}%
The HPSG treatment of relative and interrogative pronouns in respective clauses is special, but this
is due to their special distribution: they have to be part of a phrase that is initial in the
relative or interrogative clause. 
\itddone{Anne: What do you mean? rel pro are anaphoric, not interrog ones;
what about demonstratives and indefinites?
Stefan: I hope it is clear now.}
See \crossrefchapterw{relative-clauses} on relative
clauses in HPSG. %Relative clauses are the topic of Section~\ref{binding-sec-relative-pronouns}.
\citew[Section~7.2.3]{Bredenkamp96a} was an early suggestion to model binding relations of personal pronouns and anaphors by
the same means as filler-gap dependencies. We will discuss approaches relying on HPSG's general apparatus
for nonlocal dependencies without assuming that the phenomena are of the same kind in Section~\ref{sec-bt-nonlocal}. 
} The places of occurrence of traces (if they are assumed at all)
are restricted by other components of the theory. For an overview of the treatment of nonlocal
dependencies in HPSG see \crossrefchapterw{udc}.

We will not go into the details of the Binding Theory in Mainstream Generative Grammar
(MGG)\footnote{
We follow \citet[\page 3]{CJ2005a} in using the term \emph{Mainstream Generative Grammar} when
referring to work in Government \& Binding\indexgb \citep{Chomsky81a} or Minimalism \citep{Chomsky95a-u}.}, but we
give a verbatim description of the ABC of Binding Theory (ignoring movement). Chomsky distinguishes between
so-called R-expressions (referential expressions like proper nouns or full NPs/DPs), personal
pronouns and reflexives and reciprocals. The latter two are subsumed under the term \emph{anaphor}. 
Principle A says that an anaphor must be bound in a certain local domain. Principle B says that a
pronoun must not be bound in a certain local domain and Principle C says that a referential
expression must not be bound by another item at all.

Some researchers questioned whether syntactic principles like Chomsky's Principle C and the
respective HPSG variant should be formulated at all and it was suggested to leave an account of the
unavailability of bindings like the binding of \emph{he} to full NPs in
(\ref{ex-he-thinks-that-Peter}) to \isi{pragmatics} (\citealp[\page 302]{Bolinger79a-u}; \citealp[\page 227--228]{Bresnan2001a};
\citealp*[\page 44]{BMS2001a}). \citet[Section~6]{Walker2011a} discussed the claims in detail
and showed why Principle~C is needed and how data that was considered problematic for syntactic
binding theories can be explained in a configurational binding theory in HPSG. Hence the following
discussion includes a discussion of Principle~C in its various variants.
\itdopt{G: Technically, Reinhart's theory is also an attempt to reduce Principle C to pragmatics -- see especially Reinhart 1983: 166-167. The difference is that, unlike Bolinger, Kuno and others, her pragmatic account crucially involves a comparison between Principle C violations and variable binding configurations, which she takes to be syntactically defined (i.e. by c-command). So (Pragmatic) Principle C winds up being sensitive to syntax.\\
Walker's attempt to avoid the counter-examples to (Syntactic) Principle C focuses on a too narrow set of cases (mostly when-clauses). In many cases, the conclusion that Principle C is violated is really unavoidable. See the Varaschin, Culicover and Winkler paper I sent you.}

\section{A non-configurational Binding Theory}
\label{binding:sec-a-non-configurational-binding-theory}

As was noted above, \ili{English} pronouns and reflexives have to agree with their antecedents in
gender.
\itdopt{G: This is complicated by singular \emph{they}:
(a) I know someone who thinks they are the greatest thing since sliced bread.
(b) Explained briefly, narcissism is when someone admires themselves to the point that they don't care about anyone else. (from the internet)
Bob Levine has an HPSG analysis for singular \emph{they} and other cases where one finds mismatches
between syntactic and semantic agreement: \citew{Levine2010a-u}.
Overlapping reference with plurals also poses a problem for traditional theories based on syntactic indexing:
(c) * We voted for me.
Is bad but 
(d) I’m really proud of ourselves. 
(e) I talk about ourselves. 
Are much better (and attested online). Note that, on the face of it, (b) and (d)-(e) actually come out as a violations of HPSG's Principle A.
In Brazilian Portuguese, counter-examples really abound, since we have a pronominal form \emph{a gente} which is syntactically (i.e. for the purposes of concord) 3sg but is semantically 1pl. This form can be locally bound by a genunine 1pl form. 
(d) Nós vimos a-gente na TV ontem.
pron.1pl  saw pron.3sg on-the TV yesterday
These facts suggests we might want to dissociate binding from agreement.} 
In addition there is agreement in person and number. This is modeled by assuming that
referential units come with a referential index in their semantic representation.\footnote{
 There is also resolved agreement in case of (conjoined or split) antecedents with different
 gender/""person:
\eal
\ex I told John that we should leave.
\ex Tom told Mary that they should leave. \citep[\page 396]{Bresnan82c}
% The second example is by Bresnan.
\zl
See \crossrefchapterw[Section~\ref{coordination:sec-agreement-with-coordinate-phrases}]{coordination}
for more on conjoined antecedents. Anaphoric agreement is also discussed in Chapter~\ref{chap-agreement}
\citep[Section~\ref{agreement:sec-anaphotic-agreement}]{chapters/agreement}. The approach discussed in
Section~\ref{sec-bt-nonlocal} is powerful enough to 
introduce additional indices for binding that are not related to individual nodes in a tree like the
NP node for \emph{Paul} and \emph{Mary} but represent the set of the combined indices for Paul and Mary.
} (On referential
indices and coindexation vs.\ coreference see \citealp[Section~6.3]{BP80a}.) 
The following makeup for the semantic contribution of nominal objects is assumed:
\ea
Representation of semantic information contributed by nominal objects adapted from \citet[\page
  248]{ps2}:\\
\onems[nom-obj]{
  index  \ms[index]{
          per & per\\
          num & num\\
          gen & gen\\
          }\\
  restrictions \type{set of restrictions}\\
}
\z
Every nominal object comes with a referential index with person, number and gender information and a
set of restrictions. In the case of pronouns, the set of restrictions is the empty set, but for
nouns like \emph{house}, the set of restrictions would contain something like \relation{house}$(x)$
where $x$ is the referential index of the noun \emph{house}. Nominal objects can be of various
types. The types are ordered hierarchically in the inheritance hierarchy given in Figure~\ref{bt-fig-hierarchy-nominal-types}.
\begin{figure}
\centering
\begin{forest}
type hierarchy
[nom-obj
  [pron
    [ana
      [refl]
      [recp]]
    [ppro]]
  [npro]]
\end{forest}
\caption{Type hierarchy of nominal objects}\label{bt-fig-hierarchy-nominal-types}
\end{figure}
Nominal objects (\type{nom-obj}) can either be pronouns (\type{pron}) or non-pronouns
(\type{npro}). Pronouns can be anaphors (\type{ana}) or personal pronouns (\type{ppro}) and anaphors
are devided into reflexives (\type{refl}) and reciprocals (\type{recp}).


HPSG's Binding Theory differs from GB's Binding Theory in referring less to tree structures but
rather to the notion of \isi{obliqueness} of arguments of a head. The syntactic arguments of a head are represented
in a list called the argument structure list \crossrefchapterp{arg-st}. The list is the value of the feature \argst. The
\argst elements are descriptions of arguments of a head containing syntactic and semantic properties
of the selected arguments but not their daughters. So they are not complete signs but \type{synsem}
objects. See \crossrefchaptert{properties} for more on the general setup of HPSG
theories. The list elements are ordered with respect to their obliqueness, the least oblique element
being the first element \citep[\page 266]{PS92a}:\footnote{
  While \citet[\page 120]{ps} use \citegen[]{KC77a} version of the Obliqueness Hierarchy in (i), they avoid the
  terms \emph{direct object} and \emph{indirect object} in \citew[\page 266, 280]{PS92a} and
  \citew[\page 24]{ps2}.
\ea
\label{def-obliqueness-hierarchy-ps87}
\oneline{%
\is{object!indirect}\is{object!direct}\is{subject}%
\begin{tabular}[t]{@{}l@{\hspace{1ex}}l@{\hspace{1ex}}l@{\hspace{1ex}}l@{\hspace{1ex}}l@{\hspace{1ex}}l@{}}
SUBJECT $>$ & DIRECT $>$ & INDIRECT $>$ & OBLIQUES $>$ & GENITIVES $>$  & OBJECTS OF\\
            & OBJECT     & OBJECT       &              &                & COMPARISON 
\end{tabular}%
}
\z
%last
}
\ea
\label{def-obliqueness-hierarchy}
%\oneline{%
\is{object!second}\is{object!primary}\is{subject}%
\begin{tabular}[t]{@{}l@{\hspace{1ex}}l@{\hspace{1ex}}l@{\hspace{1ex}}l@{}}
SUBJECT $>$ & PRIMARY $>$ & SECONDARY    $>$ & OTHER COMPLEMENTS\\
             & OBJECT      & OBJECT        &\\
\end{tabular}%
%}
\z
This order was suggested by \citet[\page 66]{KC77a}. It corresponds to the level of syntactic activity of grammatical functions\is{grammatical function}. Elements
higher in this hierarchy are less oblique and can participate more easily in syntactic constructions, like for instance,
reductions in coordinated structures\is{coordination} \citep[\page 15]{Klein85},
topic drop\is{topic drop} \citep{Fries88b},
non-matching free relative clauses\is{relative clause!free} 
\parencites[Section~3]{Bausewein90}[\page 195]{Pittner95b}[\page 60--62]{Mueller99b}, 
passive\is{passive} and relativization\is{relativization} \citep[\page 96, 68]{KC77a}, and
depictive predication\is{predicate!secondary!depictive} \citep[Section~2]{Mueller2008a}.
In addition, \citet{Pullum77a} and \citet[\page 174]{ps} argued that this hierarchy plays a role in
constituent order\is{scrambling}\is{serialization}.
% (but see Section~\ref{sec-argst-order}.)
And, of course, it was claimed to play an important role in Binding Theory\is{Binding Theory} 
(Grewendorf, \citeyear[\page 176]{Grewendorf83a}; \citeyear[\page 160]{Grewendorf85a}; \citeyear[\page 60]{Grewendorf88a}; \citealp[Chapter~6]{ps2}).

The \argstl plays an important role for linking syntax to semantics \crossrefchapterp{arg-st}. For example, the index of the
subject and the object of the verb \emph{like} are linked to the respective semantic roles in the
representation of the verb:\footnote{%
  NP\ind{1} is an abbreviation for a feature description of a nominal phrase with the index \ibox{1}. The feature description in (\mex{1}) is also an
  abbreviation. Path information leading to \cont is omitted, since it is irrelevant for the present
  discussion.
%
}
\eas
\label{ex-like}
\emph{like}:\\
\avm{
[arg-st < NP\ind{1}, NP\ind{2} >\\
 cont   [\type*{like}
         actor & \1\\
         undergoer & \2 ] ]
}
\itddone{Anne: this cant be right: like is not an agentive verb
the subject of like is EXP, its object is theme.
Stefan: Yes, according to \citew{VanValin99a-u} this is correct. Acto and undergoer are like
proto-agent and proto-patient.}
\zs 
A lot more can be said about linking in HPSG and the interested reader is referred to
\citew{Wechsler95a-u}, \citew{Davis2001a-u}, \citew{DK2000b-u}, and \crossrefchapterw{arg-st} for this.

After these introductory remarks, we can now turn to the details of HPSG's Binding Theory:
Figure~\ref{fig-binding-argst} shows a version of Figure~\ref{fig-binding-gb} including \argst
information.
\begin{figure}
\begin{forest}
sm edges without translation
[S
  [\ibox{1} NP$_i$ [John\\John\\he]]
  [VP
    [V \sliste{ \ibox{1} NP$_i$, \ibox{2} CP } [thinks\\thinks\\thinks]]
    [\ibox{2} CP 
      [C [that\\that\\that]]
      [S, s sep+=1ex
        [\ibox{3} NP$_j$ [Paul\\Paul\\Mary]]
        [VP
         [V \sliste{ \ibox{3} NP$_j$, \ibox{4} NP$_k$ } [likes\\likes\\likes]]
         [\ibox{4} NP$_k$ [him\\himself\\Peter]]]]]]]
\end{forest}

\caption{\label{fig-binding-argst}Tree configuration of examples for binding with \argst lists}
\end{figure}
The main points of HPSG's Binding Theory can be discussed with respect to this simple figure:
(non-exempt) anaphors have to be bound locally. The definition of the domain of locality is rather simple. One
does not have to refer to tree configurations, since all arguments of a head are represented locally
in a list. Simplifying a bit, reflexives and reciprocals must be bound to elements preceding them in
the \argstl (but see Section~\ref{sec-excempt-anaphors} for so-called exempt anaphors) and a pronoun like
\emph{him} must not be bound by a preceding element in the same \argstl.

To be able to specify the conditions on binding of anaphors, personal pronouns and non-pronouns, some further
definitions are necessary. The following definitions are definitions of \emph{local o-command}, \emph{o-command},
and \emph{o-bind}. The terms are reminiscent of \emph{c-command} and so on but we have an ``o''
rather than a ``c'' here, which is supposed to indicate the important role of the \isi{obliqueness}
hierarchy. The definitions are as follows:

\eanoraggedright
\label{def-local-o-command-initial-version}\label{def-local-o-command}
Let Y and Z be \type{synsem} objects with distinct \localvs, Y referential. Then Y \emph{locally
o-commands}\is{o-command!local} Z just in case Y is less oblique than Z.
\z
\itdobl{Steve: Clarify that ‘X is less oblique than Y’ requires that X and Y are on the same arg-st list.}

\eanoraggedright
\label{def-o-command}
Let Y and Z be \type{synsem} objects with distinct \localvs, Y referential. Then Y \emph{o-commands}\is{o-command} Z just
in case Y locally o-commands X dominating Z.
\z

\eanoraggedright
\label{def-o-bind}
Y (\emph{locally}) \emph{o-binds}\is{o-bind} Z just in case Y and Z are coindexed and Y (locally) o"=commands Z. If Z
is not (locally) o-bound, then it is said to be (\emph{locally}) \emph{o"=free}\is{o-free}.
\z

\noindent
(\ref{def-local-o-command-initial-version}) says that an \argst element locally o-commands any other \argst element
further to the right of it. The condition of non-identity of the two elements under consideration in
(\ref{def-local-o-command-initial-version}) and (\ref{def-o-command}) is necessary to deal with cases of \isi{raising}, in
which one element may appear in various different \argstls. It is also needed to rule out unwanted
command relations in the case of nonlocal dependencies since the local value of a filler is shared
with its gap. See %Section~\ref{sec-binding-raising} below and 
\crossrefchaptert{control-raising} for discussion of raising in HPSG and \crossrefchapterw{udc} on
unbounded dependencies in HPSG. The condition that Y has
to be referential excludes expletive pronouns\is{pronoun!expletive} like \emph{it} in \emph{it rains} from entering
o-command relations. Such expletives are part of \argst and valence lists, but they are entirely
irrelevant for Binding Theory, which is the reason for their exclusion in the
definition. \citet[\page 258]{ps2} discuss the following examples going back to observations by
\citet[\page 65]{FH83a-u} and \citet[\page 95]{Kuno87a-u}:

\eal
\ex They$_i$ made sure that it was clear to each other$_i$ that this needed to be done immediately.
% ps2 does not have "immediately", but Kuno has. 
\ex They$_i$ made sure that it wouldn't bother each other$_i$ to invite their respective friends to dinner.
\zl
According to \citet[Section~3.6]{ps2}, the \emph{it} is an expletive. They assume that
extrapositions\is{extraposition} with \emph{it} are accounted for by a lexical\is{lexical rule!extraposition} rule that introduces an expletive and a
\emph{that} clause or an infinitival verb phrase into the valence list of the respective predicates
(see also \crossrefchapteralt[Section~\ref{prop:sec-lex-rules}]{properties}). Since the \emph{it} is
not referential it is not a possible antecedent for the anaphors in sentences like (\mex{0}) and
hence a Binding Theory built on the definitions in
(\ref{def-local-o-command-initial-version}) and (\ref{def-o-command}) will make the right
predictions.\footnote{
But see the discussion of (\ref{ex-it-nothers-myself-that}) below.
} 

The definition of o-command uses the relations of \emph{locally o-command} and \emph{dominate}. With respect to
Figure~\ref{fig-binding-argst}, we can say that NP$_i$ o-commands all nodes below the CP node since
NP$_i$ locally o-commands the CP and the CP node dominates everything below it. So NP$_i$ o-commands
C, NP$_j$, VP, V, and NP$_k$.

The definition of \emph{o-bind} in (\ref{def-o-bind}) says that two elements have to be coindexed
and there has to be a (local) o-command relation between them. The indices include person, number
and gender information (in \ili{English}), so that \emph{Mary} can bind \emph{herself} but not
\emph{themselves} or \emph{himself}. With these definitions, the binding principles can now be stated
as follows:

\begin{principle-break}[HPSG Binding Theory (preliminary)]
\begin{description}
\item [Principle A\is{principle!Binding!Principle A}] A locally o-commanded anaphor must be locally o-bound.
\item [Principle B\is{principle!Binding!Principle B}] A personal pronoun must be locally o-free.
\item [Principle C\is{principle!Binding!Principle C}] A nonpronoun must be o-free.
\end{description}
\end{principle-break}

\noindent
Principle A accounts for the ungrammaticality of sentences like (\mex{1}):
\eal
\ex[*]{
Mary$_i$ likes himself$_j$.
}
\ex[]{
\emph{likes}:\\
\argst \sliste{ NP$_i$, NP[\type{ana}]$_j$ }
}
\zl
Since both \emph{Mary} and \emph{himself} are members of the \argstl of \emph{likes}, there is an NP
that locally o-commands \emph{himself}. Therefore there should be a local o-binder. But since the
indices are incompatible because of incompatible gender values, \emph{Mary} cannot o-bind
\emph{himself}, \emph{himself} is locally o-free and hence in conflict to Principle A.

Similarly, the binding in (\mex{1}) is excluded, since \emph{Mary} locally o-binds the pronoun \emph{her}
and hence Principle B is violated.
\eal
\ex[]{
Mary$_i$ likes her$_{*i}$.
}
\ex[]{
\emph{likes}:\\
\argst \sliste{ NP$_i$, NP[\type{ppro}]$_{*i}$ }
}
\zl

\noindent
Finally, Principle C accounts for the ungrammaticality of (\mex{1}):
\eal
\ex He$_i$ thinks that Mary likes Peter$_{*i}$.
\ex \emph{thinks}:\\
\argst \sliste{ NP$_i$, CP }
\ex \emph{likes}:\\
\argst \sliste{ NP, NP[\type{npro}]$_{*i}$ }
\zl
Since \emph{he} and \emph{Peter} are coindexed and since \emph{he} o-commands \emph{Peter},
\emph{he} also o-binds \emph{Peter}. According to Principle C, this is forbidden and hence bindings
like the one in (\mex{0}) are ruled out.

\subsection{Ditransitives}

For ditransitives, there are three elements on the \argstl: the subject, the primary object and the
secondary object. If the secondary object is a reflexive, Principle~A requires this reflexive to be
coindexed with either the primary object or the subject. Hence, the bindings in (\mex{1}) are
predicted to be possible and the ones in (\mex{2}) are out:
\eal
\ex[]{
John$_i$ showed Mary$_j$ herself$_j$.\\
\argst \sliste{ NP$_i$, NP$_j$, NP[\type{ana}]$_j$ }
}
\ex[]{
John$_i$ showed Mary$_j$ himself$_i$.\\
\argst \sliste{ NP$_i$, NP$_j$, NP[\type{ana}]$_i$ }
}
\zl
\eal
\ex[*]{
John$_i$ showed herself$_j$ Mary$_j$.\\
\argst \sliste{ NP$_i$, NP[\type{ana}]$_j$, NP$_j$ }
}
\ex[*]{
I$_i$ showed you$_j$ herself$_k$.\\
\argst \sliste{ NP$_i$, NP$_j$, NP[\type{ana}]$_k$ }
}
\zl
Note that configuration-based Binding Theories like the one entertained in GB and Minimalism require
the primary object to c-command the secondary object but not vice versa. This results in theories
that have to assume certain branchings and in some cases even auxiliary nodes
\citep[Section~4.4]{Adger2003a}. In HPSG the branching that is assumed does not depend on binding
facts and indeed, ternary branching VPs \citep[\page 40]{ps2} and binary branching VPs have been assumed (see
\crossrefchapteralp[Section~\ref{sec-binary-flat}]{order} for discussion).

The list-based Binding Theory outlined above seems very simple. So far we explained binding relations between
coarguments of a head where the coarguments are NPs or pronouns. But there are also prepositional\label{binding:page-prepositional-objects-start}
objects, which have an internal structure with the referential NPs embedded within a
PP. \citet[\page 246, 255]{ps2} discuss examples like (\mex{1}): 
\eal
\label{ex-john-depends-on-him}
\ex{
John$_i$ depends [on him$_{*i}$].
%\\
%\argst \sliste{ NP$_i$, PP[\type{on}, \type{ppro}]$_{*i}$ }
}
\ex{
\label{ex-mary-talked-to-john-about-himself}
Mary talked [to John$_j$] [about himself$_j$].
}
\zl
As noted by \citet[\page 137, Section~6.5.6]{BP80a}, \citet[\page 226]{Chomsky81a}, and \citet[\page
246]{ps2}, examples like the second one are a problem for the GB Binding Theory since \emph{John} is
inside the PP and does not c-command \emph{himself}. 
% \inlinetodostefan{Bob: It is not obvious for GB why this is not like the following:
% Mary showed John's picture to himself.\\
% Stefan: I do not understand this comment. Do you think \emph{John} is embedded under \emph{'s} and hence
% this is also not c-commanding? So this means Pollard \& Sag were wrong here? If so, I would remove
% the passage. Or do you say, there is another problem for GB's Binding Theory? I also added the reference to \citet{BP80a}. }
\begin{figure}
\begin{forest}
sm edges without translation
[S
  [\ibox{1} NP$_i$ [Mary]]
  [VP
    [V \sliste{ \ibox{1}, \ibox{2}, \ibox{3} } [talked]]
    [\ibox{2} PP$_j$
       [P [to]]
       [NP$_j$ [John]]]
    [\ibox{3} PP$_j$
       [P [about]]
       [NP$_j$ [himself]]]]]
\end{forest}
\caption{Binding within prepositional objects poses a challenge for GB's Binding Theory}
\end{figure}
Examples involving case-marking
prepositions are no problem for HPSG however, since it is assumed that the semantic content of
propositions is identified with the semantic content of the NP they are selecting. Hence, the PP
\emph{to John} has the same referential index as the NP \emph{John} and the PP \emph{about himself}
has the same index as \emph{himself}. The \argstl of \emph{talked} is shown in (\mex{1}):
\ea
\emph{talked}:\\
\argst \sliste{ NP$_i$, PP[\type{to}]$_j$, PP[\type{about}, \type{ana}]$_j$ }
\z
The Binding Theory applies as it would apply to ditransitive verbs. Since the first PP is less
oblique than the second one, it can bind an anaphor in the second one. The same is true for the
example in (\mex{-1}a) and the lexical item for \emph{depend} with the \argst in (\mex{1}):
\ea
\emph{depend}:\\
\argst \sliste{ NP$_i$, PP[\type{on}, \type{ppro}]$_{*i}$ }
\z
Since the subject is less oblique than the PP object, it locally o-commands
the PP and even though the pronoun \emph{him} is embedded in a PP and not a direct argument of the verb,
the pronoun cannot be bound by \emph{John}. An anaphor would be possible within the PP object though.
Of course the subject NP can bind NPs within both PPs: both \emph{to herself} and \emph{about
  herself} would be possible as well.\label{binding:page-prepositional-objects-end}
\itdopt{G: Perhaps this would be a good place to mention the fact that HPSG Binding Theory does a fair job (better than the MGG one) in accounting for binding into locative PPs:
(a) John saw a snake near himself / him.
Since "near" has its own ARG-STR, nothing o-commands the pronoun -- so "himself" is exempt. This was a big problem for the MGG BT, as Chomsky (1981: section 5.2) recognized. A good reference is \citew{Hestvik1991a-u}.
}


\subsection{Binding and nonlocal dependencies}

\itddone{Stefan: Drop this section? 
Anne: this should be called LDD and put in section 2.

The part on LDD should not be called ‘reconstruction’ (that is Chomskyan propaganda); instead you should mention these data as counter ex to configurational theories in section 1, and give the hpsg solution in section 2: with arg-st of likes:
Himself, John likes. arg-st <NPi, NP[gap, ana]i>

The issue with the reflexive as a complement of a noun should go in section 4.2

}

Examples like (\mex{1}) are covered by HPSG's Binding Theory since \emph{himself} is fronted via
HPSG's nonlocal mechanism (see \crossrefchapteralp{udc}) and there is a connection between the
fronted element and the missing object.
\eal
\ex Himself$_i$, Trump$_i$ really admires \trace.
\ex \emph{admire}:\\
    \argst \sliste{ NP$_i$, NP[\type{gap}, \type{ana}]$_i$ }
\zl
\itdopt{G: A variant with a personal pronoun is also good, I'm told (see Pesetsky 1995: 240):
(a) Him, Trump really admires \_.
And so is (b), from \citew{Postal2006a-u}:
(b) Himself$_{i}$, Felix$_{i}$ claimed that ordinary people would never understand \_ .
It seems that the distinction between personal pronouns and anaphors is neutralized in topic/focus
position. In R\&R’s theory, pronouns in focus/topic positions are exempt from Binding
Conditions. This is not so in HPSG. I think R\&R are right, and there is probably a straightforward
way of building this into an HPSG account. 
}
Therefore, the \localv of \emph{himself} is identified with the \localv of the object in the \argstl of
\emph{admires} and since the object is local to the subject of admire, it has to be bound by the
subject. But there is more to say about binding and nonlocal dependencies in HPSG:
\citet[\page 265]{ps2} point out an interesting consequence of the treatment of nonlocal
dependencies in HPSG: since nonlocal dependencies are introduced by traces that are lexical elements
\itddone{Anne: this is old fashioned, skip it since we don not have traces now\\
Stefan: It helps readers to related to things.}
rather then by deriving one structure from another one as is common in Transformational Grammar,
there is no way to reconstruct some phrase with all its internal structure into the position of the
trace. Since traces do not have daughters, $\__j$ in (\mex{1}a) has the same local properties (part
of speech, case, referential index) as \emph{which of Claire's$_i$ friends} without having its
internal structure. 
\ea
\label{ex-which-of-clairs-friends}
%\ex 
I wonder [which of Claire's$_i$ friends]$_j$ [we should let her$_i$ invite $\__j$ to the party]?
% \ex\label{ex-which-picture-of-herself}
% {}[Which picture of herself$_i$]$_j$ does Mary$_i$ think John likes $\__j$?
% discussion would be too complicated since reconstruction would also involve exempt anaphors
\z
\itdopt{G: This is what is called anti-reconstruction. Perhaps you can put some pointers to this
  literature here. The following are particularly relevant: \citew{Heycock1995a-u,Fischer2002a-u,
    Lasnik2003a-u,Adger2017a-u}.
HPSG’s treatment of UDs, unlike the copy theory, predicts anti-reconstruction (i.e, the absence of Principle C effects under movement) to be the norm in cases where the r-expression is inside the moved constituent but is not itself identified with the moved constituent. I think this is right. The appearance of reconstruction in cases like 
(a) Whose claim that John is a jerk does he believe.
is best explained in pragmatic terms, along the lines of Lasnik (2003). 
}

Since extracted elements are not reconstructed into the position where they would be usually
located, (\mex{0}) is not related to (\mex{1}):
\ea
We should let her$_{*i}$ invite which of Claire's$_i$ friends to the party?
\z
\emph{Claire} would be o-bound by \emph{her} in (\mex{0}) and this would be a violation of
Principle~C, but since traces do not have daughters, no problem arises in (\ref{ex-which-of-clairs-friends}). 

Some of the more recent theories of nonlocal dependencies even do without traces
\citep*{BMS2001a}. These are discussed in more detail in \crossrefchapterw{udc}. For the treatment
of binding data it does not matter whether there is a trace or not: traceless accounts of extraction
assume that members of the \argstl, which contains all arguments, are not mapped onto the valence lists. So for
the lexical item in (\ref{ex-like}), one would assume the two variants in (\mex{2}) that play a role in the
analysis of the sentences in (\mex{1}):
\eal
\ex I like bagels.
\ex Bagels, I like.
\zl
\ea
%\label{ex-like}
\begin{tabular}[t]{@{}l@{~}l@{\hspace{1cm}}l@{~}l@{}}
a. & \emph{like} without extraction: & b. & \emph{like} with extraction:\\
   & \avm{
[
subj   & < \1 NP >\\
comps  & < \2 NP > \\
arg-st & < \1 NP, \2 NP > ]
} & &
\avm{
[ subj   & < \1 NP >\\
  comps  & <  > \\
  arg-st & < \1 NP, NP[\type{gap}] > ]
}
\end{tabular}
\z
\type{gap} stands for a special type that is used to indicate that a certain argument is a gap
rather than an overtly realized element. Gaps pass up their nonlocal information to the mother node,
which is indicated by a slash in the figures in Figure~\ref{fig-trace-based-and-traceless}. The
traceless analysis does not differ from the trace-based approach as far as the make up of the \argstl
is concerned. In a trace-based analysis the trace is an argument of the verb and hence the
description of the accusative object is identified with the description in the \compsl and this
element is identical to the second element of the \argstl. This means that we can talk about the
same \argst configurations for both types of theories and can abstract away from the concrete
realization of extraction.
\begin{figure}
\hfill
\scalebox{.9}{%
\begin{forest}
sm edges
[S
  [\ibox{1}\,NP [Bagels]]
  [S/NP
    [\ibox{2}\,NP [I]]
    [VP/NP\feattab{\subj \sliste{ \ibox{2} },\\
                   \comps \eliste,\\
                   \argst \sliste{ \ibox{2}, \ibox{1} }} [like]]]]
\end{forest}}
\hfill
\scalebox{.9}{%
\begin{forest}
sm edges
[S
  [\ibox{1}\,NP [Bagels]]
  [S/NP
    [\ibox{2}\,NP [I]]
    [VP/NP 
     [V\feattab{\subj \sliste{ \ibox{2} },\\
                \comps \sliste{ \ibox{1} },\\
                \argst \sliste{ \ibox{2}, \ibox{1} }} [like]]
     [\ibox{1}\,NP/NP [\trace]]]]]
\end{forest}}
\hfill{}
\caption{Traceless and trace-based analyses of fronting.}\label{fig-trace-based-and-traceless}
\end{figure}
\citegen[\page 265]{ps2} analysis of (\ref{ex-which-of-clairs-friends}) works in both worlds: in the
traceless analysis there is no element that could have daughters and in the trace-based analysis
there is a trace but since traces are simple lexical items in HPSG without internal structure \citep[164]{ps2} there
is nothing like the ``reconstruction'' known from GB.\footnote{
\citet[Section~20.2]{Mueller99a} discussed the examples in (i), which seem to be problematic for
theories in which the internal structure of extracted material plays no role:
\eal
\ex {}[Karl$_i$'s friend]$_j$, he$_{*i}$ knows $\__j$.
% EP: binding is out
\ex 
\gll Karls$_i$ Freund kennt er$_{*i}$.\\
     Karl's    friend knows he\\
\glt `He knows Karl's friend.'
\zl
\itdopt{G: I think the problem with (i) is unrelated to Principle C. It is just pragmatically odd (if X is a friend of Karl, then it is expected that Karl knows X. So it does not make sense to construe Karl knowing X as focus.) If we eliminate this pragmatic oddity, keeping the basic structure intact, the Principle C effect largely disappears.
(i’) Karl doesn’t want to invite his parents to his dissertation defense. Karl’s friends, on the other hand, he’ll definitely invite. 
I would imagine the same thing goes on in German.}
%\itdopt{Anne: Is (\mex{0} really out?}
% checked ?
According to the definition of o-command, \emph{he} locally o-commands the object of
\emph{knows}. This object is a gap. Therefore the local properties of \emph{Karl's
  friend} are in relation to \emph{he} but since gaps/traces do not have daughters, there is no o-command
relation between \emph{he} and \emph{Karl}, hence \emph{Karl} is o-free and Principle C is not
violated. Hence there is no explanation for the impossibility to bind \emph{Karl} to \emph{he}. In
order to fix this, the definition of dominance could be changed so that GB's notion of
reconstruction would be mimicked \citep[\page 409--410]{Mueller99a}. According to Müller's definition a trace or gap
would ``dominate'' the daughters of its filler. While this would account for cases like (\mex{0}),
the account of (\ref{ex-which-of-clairs-friends}) would be lost.
\itdopt{G: Ultimately, I think HPSG’s failure to predict reconstruction effects is a virtue. Off the top of my head, I can’t think of any putative instance of reconstruction that doesn’t get a better explanation in terms of pragmatics (as in the cases of Principle C reconstruction) or exemption (as in the cases of, picture NPs reflexives, topic/focus reflexives, etc.). }
}

\subsection{Exempt anaphors}
\label{sec-excempt-anaphors}\label{binding:sec-excempt-anaphors}

\itddone{Anne: this Section is a bit misleading, exempt anaphors are often used in indirect speech,
  Pollard and Sag refer to attested examples found by A Zribi Hertz,
  (\ref{ex-it-nothers-myself-that}) would be ok in such a context.\\
Stefan: I checked with natives. They say it is totally out.}

The statement of Principle A has interesting consequences: if an anaphor is not locally o-commanded,
Principle A does not say anything about requirements for binding. This means that anaphors that are
initial in an \argstl may be bound outside of their local environment. The following example by
\citet[\page 270]{ps2} shows that a reflexive can even be bound to an antecedent outside of the sentence:

\eanoraggedright
John$_i$ was going to get even with Mary. That picture of himself$_i$
in the paper would really annoy her, as would the other stunts he had planned.\footnote{
        \citew[\page 270]{ps2}.
}
\z

\noindent
A further example are NPs within adverbial PPs. Since there is nothing in the PP \emph{around
  himself} that is less oblique than the reflexive, the principles governing the distribution of
reflexives do not apply and hence both a pronoun and an anaphor is possible:

\eal
\label{ex-john-wrapped-a-blanket-around-him}
\ex John$_i$ wrapped a blanket around him$_i$.
\ex John$_i$ wrapped a blanket around himself$_i$.
\zl
\itdopt{G: \citegen{Golde1999a-u} OSU dissertation under Carl discusses the issue in more depth than P\&S 1994. I can send you the file if you don’t have it. 
Also, \citew{Charnavel2020a-u} is a more updated reference which goes over all of the cases and proposes an interesting typology of different types of point of view roles.
Her general theoretical framework is completely misguided, I think, but I doubt anyone knows more
about this point of view stuff than her. 
}

\noindent
Which of the pronouns is used is said to depend on the \emph{point of view} of the speaker
(%\citealp{Kuroda65a-u}
\citealp{Kuroda1973a-u}, for further discussion and a list of references see \citealp[\page 270]{ps2}).

%% Unfortunately the situation is different in languages like \ili{German}. The binding of a pronoun to the subject is strictly ungrammatical:
%% \ea
%% John$_i$ wickelte die Decke um ihn$_i$.
%% \z

The exemptness of anaphors seems to cause a problem since the Binding Theory does not rule out sentences like (\mex{1}):
% Fanselow86a:349 Einander arbeitet.
\ea[*]{
Himself sleeps.
}
\z
\itdopt{G: There is still the broader issue of why we don’t find subject anaphors in languages that
  do subject verb agreement — I.e. the Anaphor Agreement Effect. Stipulating that ``himself'' is
  accusative doesn't explain this. 
}

This is not a real problem for languages like \ili{English}, since such sentences are ruled out because \emph{sleeps}
requires an NP in the nominative and \emph{himself} is accusative \parencites[\page 388]{Brame77}[\page 262]{ps2}.
\citet[Section~20.4.6]{Mueller99a} pointed out that \ili{German} has subjectless verbs like
\emph{dürsten} `be thirsty' that govern an accusative:
\eal
\ex[]{
\gll Den        Mann friert.\\
     the.\acc{} man  cold.is\\
\glt `The man is cold.'
}
\ex[*]{
\gll Einander         friert.\footnotemark\\
     eachother.\acc{} cold.is\\
}
\footnotetext{
\citew[\page 349]{Fanselow86a}.
}
\ex[]{
\gll Den Mann dürstet.\\
     the.\acc{} man thirsts\\
\glt `The man is thirsty.'
}
\ex[*]{
\label{duersten}
\gll Sich dürstet.\\
     \textsc{self}.\acc{} thirst\\
}
\zl
However, as \citet[\page 158, 161]{Kiss2012a} -- discussing his own data and referring to
\citet[\page 131]{Frey93a} -- pointed out, anaphors are not exempt in German. So, examples like
(\mex{0}b) and (\mex{0}d) are correctly ruled out by a general ban on unbound anaphors in German.

The contrast in (\mex{1}) seems to be problematic. The analysis suggested by \citet[\page 149]{ps2}
assumes that an \isi{extraposition} \emph{it} is inserted into the \argstl and the clause is appended to
this list:
\eal
\ex[]{
That Sandy snores bothers me.\\
\emph{bother}: \argst: \sliste{ S, NP }
}
\ex[]{
It bothers me that Sandy snores.\\
\emph{bother}: \argst: \sliste{ NP[\type{it}], NP[\type{ppro}], S }
}
\ex[*]{
\label{ex-it-nothers-myself-that}
It bothers myself that Sandy snores.\\
\emph{bother}: \argst: \sliste{ NP[\type{it}], NP[\type{ana}], S }
}
\zl
\itdopt{G: In my thesis, I argue that the acceptability of long-distance reflexives within expletive clauses depends on the nature of the predicates. Roughly, if the predicate containing the expletive in its ARG-STR is one where a reflexive interpretation is in principle possible, the reflexive must be bound locally. 
This is the case of “bother” in (31b), because (a) is fine. 
(a) John bothers himself 
But it is not the case with a predicate like “clear”: 
(b) They made sure that it was clear to themselves what needed to be done
(c) ??John is clear to himself 
Non-local binding is fine in (b) because (c) semantically odd. 
}
According to \citet[\page 149]{ps2} the \emph{it} in (\mex{0}b--c) is non-referential. Hence there is
nothing that o-commands the accusative object and hence anaphors would be exempt in the object
position and sentences like (\mex{0}c) were predicted to be grammatical. This seems to argue for an
analysis that treats the \isi{extraposition} \emph{it} as a referential element \citep[\page 215, 232]{Mueller99a}.


% \subsection{Binding and adjuncts}

% We already discussed an example with an adverbial argument in
% (\ref{ex-john-wrapped-a-blanket-around-him}). In (\ref{ex-john-wrapped-a-blanket-around-him}), the
% prepositional phrase was an argument, but one with a preposition that contributes its own semantics
% and does not simply share the index with the embedded noun as in (\ref{ex-john-depends-on-him}). Here, we want to briefly
% discuss adjuncts. The classical example from the literature is given in (\mex{1}):
% \ea
% John$_i$ saw a snake near him$_i$/himself$_i$.
% \z
% Some authors claim that the variant with \emph{himself} is ungrammatical, while others state that both are fine.
% \citet[\page 57]{Buering2005a-u} provides attested data like (\mex{1}) showing that reflexives may
% appear in adjuncts.

% \ea
% The seductress$_i$ must be careful not to cast this spell near herself$_i$.
% \z

% The possibility of reflexives in adjuncts as in (\mex{0}) is predicted by \citeauthor{PS92a} Binding
% Theory: \emph{herself} is not o-commanded and hence exempt from binding constraints.

% Note that the general setting leaves some options for the treatment of adjuncts with respect to the
% \argstl. It has been argued that some adjuncts rather behave like complements and hence should be
% added to the \argstl \citep{BMS2001a}.\addref

% Adjuncts will be discussed further in Section~\ref{sec-totally-non-configurational-BT} below.

% \itddone{Anne: There should be a subsection on adjuncts
% Start with adjunct to verbs which can be analysed as complements (see bouma et al 2001 and UDC chapter)
% Bresnan 1982 also noticed that adjuncts are more free than cplts:
% John saw a snake near him/ himself
% Ex (\ref{ex-he-knows-the-woman-loved-by-John}) could be in a fn since it is an open issue.
% }

%\subsection{Subject bound anaphors}
%\citet{AGS1998a}

\subsection{Inalienable possession NPs}

\citet{Koenig1999b} examines examples like (\mex{1}) in which a definite noun phrase is interpreted
as a body part of some other argument of the involved verb. Koenig discusses French data, but a
parallel construction exists in \ili{German} as well.\footnote{
See also \crossrefchapterw[\page \pageref{ex-herz-augen}]{idioms} for discussion of body parts in
the context of idioms.
}

\ea
\label{ex-marc-avance-le-pied}
\gll Marc$_i$ a avancé le pied$_i$.\\
     Marc     has advanced the foot\\\hfill{(French)}
\glt `Marc$_i$ moved his$_i$ foot forward.'
\z

Koenig argues that these inalinable possession NPs should be interpreted by making recourse to the
same mechanism as used in Binding Theory rather than argument linking (see
\crossrefchapteralt{arg-st} on linking).\footnote{
\added{Steve Wechsler (p.c., 2021) pointed out to me that in the light of \citegen{Schwarz2019a-u} theory
of weak definites a reanalysis of the phenomena discussed in this section may be possible. I leave this for further research.}
} In addition to what Binding
Theory predicts, he defines the concept of Active Zone to be able to further restrict possible
candidates for the possessor. He also formulates restrictions that have to hold on semantic roles
filled by the possessor and the body part. While a discussion of all this would take us too far
away, I want to discuss Koenig's lexical\is{lexical rule!for body part nouns|(} rule for possessive nouns. He assumes a lexical rule
similar to what is given in (\mex{1}):

\vbox{
\ea
Lexical rule for body part nouns adapted from \citet[256]{Koenig1999b}:\\
\avm{
[ synsem & [ local & [ cat  & [ head  & noun\\
                                spr   & < \1 > \\
                                comps & < >\\
                                arg-st & < \1 NP\ind{2} > ]\\
                       cont & [ index & \3\\
                                restriction & \{ [\type*{inal-poss-rel}
                                                  possessor & \2\\
                                                  possessed & \3 ] \} ] ] ] ]} $\mapsto$\\
\flushright
\avm{
[\type*{body-part-noun}
 synsem & [ local & [ cat & [ spr   & < Det > \\
                              comps & < > \\
                              arg-st & < Det, \1 NP![\type{refl} $\wedge$ \type{s-ana}]! > ] ] ] ]
}
\z
}

\noindent
The lexical rule maps a body-part noun selecting for a possessive NP \iboxb{1} via its \spr feature onto a
body part noun selecting for a definite article. Since by convention features not mentioned in the
lexical rule are taken over from the input, the output has the same \cont value as the input. The
output has the specification that the element on the \argst is of type \type{refl} and
\type{s-ana}. Pronominal elements of this type behave like reflexive pronouns and have to be bound
in the subject domain.
\eal
\ex 
\label{le-body-part-noun-possessive}
Body part noun with possessive pronoun:\\
\avm{
[ cat  & [ head  & noun\\
                                spr   & < \1 > \\
                                comps & < >\\
                                arg-st & < \1 NP\ind{2} > ]\\
                       cont & [ index & \3\\
                                restr & \{ [\type*{inal-poss-rel}
                                                  possessor & \2\\
                                                  possessed & \3 ] \} ] ]
}
\ex 
\label{le-body-part-noun-determiner}
Body part noun with definite determiner:\\*
\avm{
[ cat  & [ head  & noun\\
                                spr   & < Det > \\
                                comps & < >\\
                                arg-st & < Det, NP![\type{refl} $\wedge$ \type{s-ana}]!\ind{2} > ]\\
                       cont & [ index & \3\\
                                restr & \{ [\type*{inal-poss-rel}
                                                  possessor & \2\\
                                                  possessed & \3 ] \} ] ]
}
\zl 
The two lexical items can be used to analyze (\mex{1}) and (\ref{ex-marc-avance-le-pied}), respectively.
\ea
\gll Marc$_i$ a   avancé   son$_i$ pied.\\
     Marc     has advanced his     foot\\
\glt `Marc$_i$ moved his$_i$ foot forward.'
\z
While in (\mex{0}) a possessive pronoun is selected by the body part noun in
(\ref{le-body-part-noun-possessive}), this is not the case in the analysis of
(\ref{ex-marc-avance-le-pied}). But in terms of binding the situation is similar: in both sentences
there is an initial element in the \argst that is linked to the \text{possessor} role of the
noun. The possessive pronoun has of course a pronominal index and the NP in the \argst in
(\ref{le-body-part-noun-determiner}) has a pronominal index as well, since this is what was
specified in the lexical rule. So \citeauthor{Koenig1999b}'s approach can account for the data
without assuming any additional structure or additional empty pronominal elements.
\is{lexical rule!for body part nouns|)}


\subsection{Long-distance reflexives}

A lot of work on binding in various frameworks deals with English and how to formulate the ABC of
Binding Theory. However, work by \citet{Dalrymple93a} shows convincingly that there is considerable
crosslinguistic variation. Following Dalrymple, researchers working in HPSG suggested various types of pronominal
elements that have to be bound in various domains \citep{AGS1998a,Koenig1999b,XPS94a-u,PX98a,BM99a,Dalrymple93a,Hellan2005a}.
Those working on languages that have so-called long"=distance
reflexives\label{page-long-distance-reflexives} as for example \ili{Mandarin Chinese},
\ili{Portuguese}, and \ili{Norwegian}
\citep*{XPS94a-u,PX98a,BM99a,Dalrymple93a,Hellan2005a} suggested a fourth binding principle.\footnote{
For discussion of some of these languages and further examples from other languages and an analysis
in LFG\indexlfg see \citew{Dalrymple93a}.
} In such languages, there are pronouns that \added{must be bound, but} may be bound locally or non-locally.
Such pronouns are called Z-pronouns and the binding principle responsible for them is
Principle~Z \citep[\page 171]{BM99a}. Adding Principle~Z to the preliminary version of HPSG's
Binding Theory we get:
\begin{principle-break}[HPSG Binding Theory]
\begin{description}
\item [Principle A\is{principle!Binding!Principle A}] A locally o-commanded anaphor must be locally o-bound.
\item [Principle B\is{principle!Binding!Principle B}] A personal pronoun must be locally o-free.
\item [Principle C\is{principle!Binding!Principle C}] A nonpronoun must be o-free.
\item [Principle Z\is{principle!Binding!Principle Z}] An o-commanded anaphor must be o-bound.
\end{description}
\end{principle-break}
Principle~Z is like Principle~A but with the requirement that anaphors must be o-bound rather than
locally o-bound. The requirement to be o-bound includes the option of being locally o-bound but
nonlocal o-binding is possible as well. 

\itddone{Anne: what’s this doing there? should be in intro= traditional view. The last part with figure 5 should be in section 1 (how different is it from gov and binding theory?)\\
Stefan: It is different from GB theory in that we have a principle Z. I like this since such higher
order symmetries are interesting. It has nothing to do with the traditional view. It is a property
of the HPSG approach and therefore I find it interesting.}
When the symmetries between the various principles are further explored, the intriguing observation
that emerges with respect to the empirical generalisations in the principles above is that they
instantiate a square of logical oppositions\is{square of opposition} shown in
Figure~\ref{bindingSquareOpposition} \parencites[Section~11.4]{BM99a}[\page 227]{Branco2006a-u}.
\begin{figure}
\centerline{\includegraphics[width=18pc]{figures/bindingSquareOpposition.pdf}}
\caption{The Binding square of opposition}\label{bindingSquareOpposition}
\itdobl{G: This makes the BT look very nice and elegant. But how does this work exactly? The HPSG versions of Principle A and Z do not say that “x is (locally) bound”, since this would exclude the possibility of exempt anaphors. They say "o-commanded anaphors must be (locally) o-bound". This is not captured in 20.6. 
\\
Once we build in exemption into our account, Principle A and B are no longer contradictory (which is why we find many contexts in which anaphors and pronominals are not in complementary distribution).
\\
As Dalrymple (1993) shows, there is a lot of cross linguistic variation, not only with respect to binding domains, but also regarding the types of antecedents particular forms can take. It is hard to see how to make this compatible with Figure 20.6.}
\end{figure}
\is{square of duality}
\LATER{\inlinetodoobl{redraw figure}}
There are two pairs of \emph{contradictory} constraints, which are formed
by the two diagonals, (Principles A, B) and (C, Z). One pair of \emph{contrary}
constraints (they can be both false but cannot be both true) is given
by the upper horizontal edge (A, C).  One pair of \emph{compatible}
constraints (they can be both true but cannot be both false) is given
by the lower horizontal edge (Z, B). Finally two pairs of
\emph{subcontrary} constraints (the first coordinate implies the second,
but not vice-versa) are obtained by the vertical edges, (A, Z) and (C, B).






\section{A totally non-configurational binding theory}
\label{sec-totally-non-configurational-BT}

The initial definition of o-command contains the notion of dominance and hence makes reference to
tree structures. \citet[\page 279]{ps2} pointed out that the binding of \emph{John} by \emph{he} in
(\mex{1}a) is correctly ruled out since \emph{he} o-commands the trace of \emph{John} and hence
Principle~C is violated. But since they follow GPSG in assuming that \ili{English} has no subject traces
\citep[Chapter~4.4]{ps2}, this account would not work for (\mex{1}b). 
\eal
\label{ex-subject-object-extraction-traceless}
\ex John$_{*i}$, he$_i$ said you like \trace$_i$.
\ex John$_{*i}$, he$_i$ claimed left.
\zl
\itddone{Anne: Discussing Pollard and Sag’s view on extraction seems obsolete, under current analyses, is there is a pb with (25b)?
Arg-st of claimed <NPi,S[SUBJ<NP[gap]i>]>.\\
Stefan: As I showed above the amalgamation analysis is parallel to trace-based analyses. Hence, it
seems that all problems carry over to an amalgamation analysis. Pollard \& Sag used lexical rules to
get rid of the subject. This could have been done for the traceless analysis as well, but I think Binding Theory
was not discussed any longer.
}

\noindent
Later work in HPSG abolished traces alltogether (\citealp*{BMS2001a}; \crossrefchapteralp{udc} but
see \citew[Section~4.9]{MuellerCurrentApproaches} for a trace-based approach and Müller \citeyear{Mueller2004e}; \citeyear[Chapter~19]{MuellerGT-Eng4} on empty elements in general) and hence Binding Theory cannot rely on
dominance any longer. This section deals with the revised version of Binding Theory not making
reference to dominance: the revised non-configurational variant of o-command suggested by \citet[\page 279]{ps2} has the
form in (\mex{1}):\footnote{%
  I replaced ``subcategorized by'' by reference to the \argstl.%
}
\eanoraggedright
\label{def-non-configurational-o-command}
Let Y and Z be \type{synsem} objects with distinct \localvs, Y referential. Then Y o-commands Z just in case either:
\begin{enumerate}[label=\roman*.]
\item Y is less oblique than Z; or
\item Y o-commands some X that has Z on its \argstl; or
\item Y o-commands some X that is a projection of Z (\ie the \headvs of X and Z are token-identical).
\end{enumerate}
\z
\itdobl{Steve: Already this theory doesn’t work, even for the cases P\&S discussed. I noticed the flaw in the early 90s and pointed it out to Carl and Ivan before the book appeared; they agreed (I still remember Carl’s email reply to me that began with the sentence “Shit!”). but it was too late to fix because it had already gone to press. The problem is “a projection of Z (i.e., the HEAD values of X and Z are token-identical)”-- that the definition of the ‘projection’ relation inadvertantly includes filler-gap relations, since their HEAD values are token-identical.}
The o-command relation can be explained with respect to Figure~\ref{fig-explanation-o-command}.
%While it would have been sufficient to require that Z is not raised in the definition of local o-command this seems
%to be too strong in the general definition of o-command.
\begin{figure}
\begin{forest}
sm edges without translation
[A
  [B [John]]
  [C{[\head \ibox{1}]}
    [D\feattab{\head \ibox{1},\\\argst \sliste{ B, E }} [thinks]]
    [E{[\head \ibox{2}]}
      [F\feattab{\head \ibox{2},\\\argst \sliste{ G }} [that]]
      [G{[\head \ibox{3}]}
        [H [Peter]]
        [I{[\head \ibox{3}]} 
          [J\feattab{\head \ibox{3},\\ \argst \sliste{ H, K }} [likes]]
          [K [him]]]]]]]
\end{forest}
\caption{Tree for explanation of the o-command relation}\label{fig-explanation-o-command}
\end{figure}

According to the definition, B o-commands E by clause i, since B and E are in the \argstl of
\emph{thinks} and B is less oblique than E. B o-commands F, since it o-commands E and E
is a projection of F (clause iii). B also o-commands G, since B o-commands F and F has G on its
\argstl (clause ii). Since B o-commands G, it also o-commands J, since G is a projection of J
(clause iii). And because of all this B also o-commands H and K, since B o-commands J and both H and
K are members of the \argstl of J (clause ii). 

This recursive definition of o-command is really impressive and it can account for binding phenomena
\itddone{Anne: ``This recursive definition of o-command is really impressive and it can account for
  binding phenomena in approaches that do not have empty nodes for traces in the tree structures''
  What do you mean? the reader is lost, explain, put here the examples in section 3.\\
Stefan: This refers to (\ref{ex-subject-object-extraction-traceless}). I guess it is ok now since it
is closer.
}
in approaches that do not have empty nodes for traces in the tree structures, but there are still open
issues.\footnote{%
\itddone{Anne: fn is useless.\\
Stefan: It makes a point comparing the frameworks. Wh do you think it is useless?}
Note that the label \emph{totally non-configurational Binding Theory} seems to suggest that
dominance relations do not play a role at all and hence this version of Binding Theory could be appropriate for
HPSG flavors like \sbcg that do not have daughters in linguistic signs (see \citew{Sag2012a} and
\crossrefchaptert[Section~\ref{sec-sbcg}]{cxg} for discussion). But this is not the case. The definition of o-command in
(\ref{def-non-configurational-o-command}) contains the notion of projection. While this notion can
be formalized with respect to a complex linguistic sign having daughters in Constructional HPSG as
assumed in this volume, this is impossible in SBCG and one would have to refer to the derivation
tree, which is something external to the linguistic signs licensed by a SBCG theory. See also footnote~\ref{binding-fn-percolation-of-indices}.%
}

As was pointed out by \citet[\page 490]{HL96a}, \citet[Sect~20.4.1]{Mueller99a} and \citet{Walker2011a}, adjuncts pose a challenge for the
non-configurational Binding Theory. For example, a referential NP can be part of an adjunct and
since adjuncts are usually not part of \argstls they would not be covered by the definition of
o-command given above. \emph{John} is part of the reduced relative clause modifying \emph{woman} in
(\mex{1}).
\ea
\label{ex-he-knows-the-woman-loved-by-John}
He$_{*i}$ knows the woman loved by John$_i$.
\z
\itdopt{G: It’s interesting that the problem only arises for Principle C and (presumably) Principle Z. All the more reasons to try to reduce both of these principles to performance factors. }
Since the relative clause does not appear on any \argstl, \emph{he} does not o-command \emph{John}
and hence there is no Principle~C violation and the binding should be fine.

Several authors suggested including adjuncts into \argstls of verbs
% Adam: at least some adjuncts put on Arg-ST. Could also be DEPS if case is assigned on DEPS
(\citealp[\page 168]{Chung98}; \citealp[\page 240]{Prze99}; \citealp*[\page 60]{MSI99a}), but this would
result in conflicts with Binding Theory if applied to the nominal domain \citep[Section~20.4.1.]{Mueller99a}. The reason is that nominal modifiers have a semantic contribution that
contains an index that is identical to the index of the modified noun.\footnote{%
See \crossrefchapterw[Section~\ref{sec:rc-pollard--sag}]{relative-clauses} and
\citew{Mueller99b} on relative clauses. \citet{Sag97a} suggests an approach to relative clauses in
which a special schema is assumed that combines the modified noun with a verbal projection. This
approach does not have the problem mentioned here. However, prenominal adjuncts would remain
problematic as the following example (based on \citealt[\page 412]{Mueller99a}) shows:
\ea
\gll Sie$_{*i}$ kennt das Kim$_i$ begeisternde Buch.\\
     she        knows the Kim     enthusing    book\\
\glt `She knows the book enthusing Kim.'
\z
The adjectival participle behaves like a normal adjectival modifier. For Principle~C to make the
right predictions, there should be a command relation between \emph{er} and the parts of the
prenominal modifier. See also
\crossrefchapterw[\pageref{relative-clauses:fn-page-to-be-read}]{relative-clauses}. PP adjuncts
within nominal structures are a further instance of problematic examples.%
} If there are several such modifiers, we get a conflict since we have several coindexed non-pronominal indices on the same
\argstl, which would violate Principle~C. 

There are two possible solutions that come to mind. The first one is pretty ad
hoc: one can assume two different features for different purposes. There could be the normal index
for establishing coindexation between heads and adjuncts and heads and arguments and there could be
a further index for binding. Adjectives would then have a referential index for establishing
coindexation with nouns and an additional index referring to a state, which would be irrelevant for the
binding principles.
% Way out: stipulate two sets of indices one for binding one for linking the semantics of
% heads/adjuncts/arguments ...

The second solution to the adjunct problem might be seen in defining o"=command with respect to the \depsl. The \depsl is a list
of dependents that is the concatenation of the \argstl and a list of adjuncts that are introduced on
this list \citep*[\page 12]{BMS2001a}. Binding would be specified with respect to \argst and dominance with
respect to \deps (which includes everything on \argst). The lexical introduction of adjuncts has
been criticized because of scope issues by \citet[\page 153]{LH2006a} and there are also problems related to
binding: \citet[\page 490]{HL96a} pointed out that there are differences when it comes to the
interpretation of pronouns in examples like (\mex{1}a,b) and (\mex{1}c,d):
\eal
\ex They$_i$ went into the city without anyone noticing the twins$_{*i/j}$.
\ex They$_i$ went into the city without the twins$_{*i/j}$ being noticed.
\ex You can't say anything to them$_i$ without the twins$_{i/j}$ being offended.
\ex You can't say anything about them$_i$ without Terry criticizing the twins$_{i/j}$ mercilessly.
\zl
While the subject pronoun cannot be coreferential with \emph{the twins} inside the adjunct, the object pronoun in
(\mex{0}c,d) can.
\itdopt{G: I’m not sure I understand what you mean. Why would we need to refer to the position of the adjunct here? The position of the adjunct patently doesn’t matter (there is no contrast between (39a) and (39b) or (39c) and (39d)). The only thing that matters is the position of the pronoun. Why can’t we refer to that in terms of the differences between subjects and objects in the DEPS list?}
If we just register adjuncts on the \depsl, we are unable to refer to their
position in the tree and hence we cannot express any statement needed to cover the differences in
(\mex{0}). Note that this is crucially different for elements on the \argstl in \ili{English}, since the \argst of a lexical item
basically determines the trees it can appear in in \ili{English}: the first element appears to the left of
the verb as the subject and all other elements to the right of the verb as complements. However,
this is just an artifact of the rather strict syntactic system of \ili{English}, this is not the case for
languages with freer constituent order like \ili{German}, which causes problems for Binding Theories not
taking the linearization of elements into account (see \citealp[\page 140]{Grewendorf85a} and
\citealp[\page 12]{Riezler95a} for crucial examples).
%to be discussed below 
%(see Section~\ref{bt-sec-obliqueness-and-constituent-order}).

There is another issue related to the totally non-configurational version of the Binding Theory: in 1994, HPSG was strictly
head-driven. There were rather few schemata and most of them were headed. Since then more and more
constructional schemata were suggested that do not necessarily have a head. For example, relative
clauses were analyzed involving an empty relativizer (\citealp[Chapter~5]{ps2}; \crossrefchapteralp[Section~\ref{sec:rc-pollard--sag}]{relative-clauses}). One way to eliminate this empty element from
grammars is to assume a headless schema that combines the relative phrase and the clause from which
it is extracted directly \parencites[Section~2.7]{Mueller99b}[522]{Sag:10b}[\page 345]{MuellerCurrentApproaches}. In addition there were proposals to analyze free
relative clauses\is{relative clause!free} in a way in which the relative phrase is the head
\citep[\page 383]{WK2003a}. So, if \emph{whoever} is the head of \emph{whoever is loved by John}, the whole
relative clause is not a projection of \emph{loved}. Furthermore, \emph{is loved by John} is not an argument
of \emph{whoever} and hence there is no appropriate connection between the involved elements, which
means that the arguments of \emph{loved} will not be found by the definition of o-command in
(\ref{def-non-configurational-o-command}). This means that \emph{John} is not o-commanded by
\emph{he}, which predicts that the binding in (\mex{1}) is possible, but it is not. 
\ea
He$_{*i}$ knows whoever is loved by John$_i$.
\z
Further examples of phenomena that are treated using unheaded constructions are serial verbs in
\ili{Mandarin Chinese}: \citet{ML2009a} argue that VPs are combined to form a new complex VP with a
meaning determined by the combination. None of the combined VPs contributes a head. No VP selects
for another VP. 

%% They are unheaded but the CAT values are shared.
%% and coordination
%% \crossrefchapterp[Section~\ref{coord-sec-headedness}]{coordination}. If coordinated structures are
%% unheaded, as often assumed in HPSG analyses, \emph{John} would not be o-commanded by \emph{he} in
%% (\mex{1}):
%% \ea
%% He$_{*i}$ knows that Mary loves John$_i$ and that Sandy loves Kim.
%% \z


There seems to be no way of accounting for such cases without the notion of
dominance (but see Section~\ref{sec-bt-nonlocal} for a lexical solution). For those insisting on
grammars without empty elements, the solution would be a fusion of the definition given in
(\ref{def-non-configurational-o-command}) with the initial definition involving dominance in
(\ref{def-o-command}). \citet{HL95b} suggested such a fusion. This is their definition of vc-command:
\eanoraggedright
\label{def-vc-command-HL}
v(alence-based) c-command:\\
Let α be an element on a valence list that is the value of the valence feature γ and α$'$ the \dtrs element whose \synsemv is structure-shared
with α. Then if the constituent that would be formed by α$'$ and one or more elements β has a null
list as its value for γ, α vc-commands β and all its descendants.
\z
Rewritten in more understandable prose this definition means that if we have some constituent α$'$
then its counterpart in the valence list vc-commands all siblings of α$'$ and their
\changed{descendants} provided the valence list on which α$'$ is selected is empty at the next higher node. We have two
valence lists that are relevant in the verbal domain: \subj (some authors use \spr instead) and
\comps. The \compsl is empty at the VP node and the \subjl ist empty at the S node. So, the
definition in (\mex{0}) makes statements about two nodes in Figure~\ref{fig-vc-command-HL}: the
lower VP node and the S node. 
\begin{figure}
\begin{forest}
sm edges without translation
[S\feattab{\subj \eliste,\\
           \comps \eliste }
  [\ibox{1} NP [they]]
  [VP\feattab{\subj \sliste{ \ibox{1} },\\
              \comps \eliste }
    [VP\feattab{\subj \sliste{ \ibox{1} },\\
                \comps \eliste}
      [V\feattab{ \subj \sliste{ \ibox{1} },\\
                  \comps \sliste{ \ibox{2} },\\
                  \argst \sliste{ \ibox{1}, \ibox{2} }}
         [bought]]
      [\ibox{2} NP
        [the car,roof]]]
    [{PP}
      [without anybody noticing the twins,roof]]]]
\end{forest}
\caption{Example tree for explaining vc-command: the subject vc-commands the adjunct because it is
  in the valence list of the upper-most VP and this VP dominates the adjunct PP}\label{fig-vc-command-HL}
\end{figure}
For Figure~\ref{fig-vc-command-HL}, this entails that the object NP \emph{the car} vc-commands
\emph{bought} since \emph{the car} is an immediate daughter of the first projection with empty
\compsl. The NP \emph{they} vc-commands the VP \emph{bought the car without anybody noticing the
  twins}, since both are immediately dominated by the node with the empty \subjl.

The proposal by \citeauthor{HL95b} was criticized by \citet[\page 235]{Walker2011a}, who argued that the modal
component \emph{would be formed} in the definition is not formalizable and suggested the following revision:
\eanoraggedright
Let α, β, γ be \type{synsem} objects, and β$'$ and γ$'$ signs such that β$'$: [\synsem β] and γ$'$: [\synsem γ]. Then α vc-commands β iff
\begin{enumerate}[label=\roman*.]
\item γ$'$: [ \textsc{ss|loc|cat|subj} \sliste{ α } ] and γ$'$ dominates β$'$, or 
\item α locally o-commands γ and γ$'$ dominates β$'$.
\end{enumerate}
\z
Principle~C is then revised as follows:
\ea
Principle C: A non-pronominal must neither be bound under o-command nor under a vc-command relation.
\z
Walker uses the tree in Figure~\ref{fig-vc-command} to explain her definition of
vc-command.
%\ea
%It was herself that Mary loved.
%\z
\begin{figure}
\begin{forest}
sm edges without translation
[S
  [\ibox{1}\,NP [they]]
  [{VP[\subj \sliste{ \ibox{1} = α } ]}, name=gamma
    [{VP[\subj \sliste{ \ibox{1} } ]}
      [V\feattab{ \subj \sliste{ \ibox{1} },\\
                  \comps \sliste{ \ibox{2} },\\
                  \argst \sliste{ \ibox{1}, \ibox{2} }}
         [bought]]
      [\ibox{2}\,NP
        [the car,roof]]]
    [PP, name=beta
      [without anybody noticing the twins,roof]]]]
\node [right=2ex] at (beta)
    {
         = β$'$
    };
\node [right=8ex] at (gamma)
    {
         = γ$'$
    };
\end{forest}
\caption{Example tree for explaining vc-command: the subject vc-commands the adjunct because it is
  in the valence list of the upper-most VP and this VP dominates the adjunct PP}\label{fig-vc-command}
\end{figure}
The second clause in the definition of vc-command is the same as before: it is based on local
o-command and domination. What is new is the first clause. Because of this clause the subject
vc-commands the adjunct since the subject \ibox{1} is in the \subjl of the top-most VP (α) and
this top-most VP (γ$'$) dominates the adjunct PP (β′). 

% \itddone{Anne: this may be ‘interesting’ but it is not for a handbook, it is for a paper.\\
% Stefan: I disagree. This is something fundamental, one has to talk about.}
% There is an interesting puzzle\label{page-binding-as-filter} here as far as the formal foundations of HPSG are concerned
% \crossrefchapterp{formal-background}: usually binding theories are defined with respect to some tree
% structures. So the structures are assumed to exist and then there are constraints put onto them to
% rule out certain bindings. The definition of \citeauthor{HL95b} contains a modal component talking
% about structures that would be licensed. Walker criticizes this and formulates a definition that
% does without this part. However, by doing so the problem does not go away. If HPSG grammars are seen
% as a set of constraints describing models of linguistic objects, there would not be a linguistic
% object for *\emph{Mary likes himself.} and hence one could not say that \emph{Mary} o-commands
% \emph{himself}. Hence, there is a problem, whether one names it in the definition or not. It seems
% to be necessary to conceptualize binding conditions as something external to the core theory of
% HPSG: a filter that is applied on top of everything else as is common in more
% implementation-oriented approaches to HPSG and in the generate and test model of GB.
% \itddone{Anne: The discussion about formal foundation starting p883 should be skiped, a handbook is for promoting solid published work, not for raising open issues, page 884 is not understandable;
% And by the way I disagree, Mary likes himself would exist as a linguistic object that obeys some but
% not all of the constraints, see pullum and scholz for gradient grammaticality;\\
% Stefan: This is not a valid argument. You claim that we work on a structure that violates constraint
% X and then say we rule it out because of X. This does not seem to work.}
%%
%% Ist nur inefficnet aber nciht schlimm, weil Prinzip C das negiert.
%%
%% The drawback of Walker's definition is that there are two ways something within an object NP can be
%% vc-commanded. The reason is that the second clause applies to to the object \ibox{2} and the first
%% one applies as well since the \subjl of the lower VP matches the description in clause i. as
%% well. This means that there are spurious ambiguities in the analysis of 

Apart from the elimination of the modal component in the defintion of vc-command, there is a further
difference between \citeauthor{HL95b}'s and \citeauthor{Walker2011a}'s definition: the former
applies to Specifier-Head structures, in which the singleton element of the \sprl is saturated. We
will return to this in Section~\ref{sec-nominal-heads-as-binders}. Note also that the definition of
\citeauthor{HL95b} includes the sibling VP among the items commanded by the subject, while Walker's
definition includes elements dominated by this VP only.\footnote{% 
  The situation is similar to the different versions of c-command in MGG. See footnote~\ref{fn-c-command-GB}.
}
This difference will also matter in Section~\ref{sec-nominal-heads-as-binders}.


\citeauthor{HL95b}'s examples involve a subject-object asymmetry. Interestingly, a similar
subject-object asymmetry seems to exist in \ili{German}, as \citet[\page 148]{Grewendorf85a} pointed
out. The following example is based on his example:
%% \eal
%% \ex[]{
%% \gll In Marias$_i$ Wohnung erwartete sie$_i$ ein Lustmolch.\\
%%      in Maria's    flat     waits     her     a  lecher\\
%% \glt `A lecher waits for Maria in her flat.'
%% }
%% \ex[*]{
%% \gll Ein Lustmolch erwartete sie$_i$ in Marias$_i$ Wohnung.\\
%%      a   lecher    waits     her     in Maria's    flat\\
%% }
%% \zl
%% \eal
%% \ex[*]{
%% \gll In Marias$_i$ Wohnung erwartete sie$_i$ einen Lustmolch.\\
%%      in Maria's    flat    waits     she     a.\acc{} lecher\\
%% \glt Intended: `Maria waits for a lecher in her flat.'
%% }
%% \ex[*]{
%% \gll Sie$_i$ erwartete in Marias$_i$ Wohnung einen Lustmolch.\\
%%      she     waits     in Maria's   flat     a\acc{} lecher\\
%% \glt Intended: `Maria waits for a lecher in her flat.'
%% }
%% \zl
% \eal
% \ex[]{
% \gll In Marias$_i$ Wohnung erwartete sie$_i$ ein Lustmolch.\\
%      in Maria's    flat     waits     her.\acc{}    a.\nom{}  lecher\\
% \glt `A lecher waits for Maria in her flat.'
% }
% \ex[*]{
% \gll In Marias$_i$ Wohnung erwartete sie$_i$ einen Lustmolch.\\
%      in Maria's    flat    waits     she.\nom{}     a.\acc{} lecher\\
% \glt Intended: `Maria waits for a lecher in her flat.'
% }
% \zl
\eal
\ex[]{
\gll In Marias$_i$ Wohnung erwartete sie$_i$ ein Blumenstrauß.\\
     in Maria's    flat     waits     her.\acc{}    a.\nom{}  bouquet\\
\glt `A bouquet waits for Maria in her flat.'
}
\ex[*]{
\gll In Marias$_i$ Wohnung erwartete sie$_i$ einen Blumenstrauß.\\
     in Maria's    flat    waits     she.\nom{}     a.\acc{} bouquet\\
\glt Intended: `Maria waits for a bouquet in her flat.'
}
\zl
\itdopt{G: This is also reported in English. 
(a) In John’s apartment, he smokes pot.
But simply adding more descriptive material to the fronted PP makes the sentence better, as Reinhart (1983: 80) noted.
(A) In John’s newly renovated apartment on 5th avenue, he smokes pot. 
Peter, Susanne and I also point out that if we change where the nuclear accent of the PP falls, coreference is also possible:
(b) In Ben's CARport, he smokes pot
(b) is a lot better than (a).
Again, whatever is going on here is probably unrelated to the actual syntactic structure. There are more examples in the paper I sent you.}
While the fronted adjunct can bind the object in (\mex{0}a), binding the subject in (\mex{0}b) is
ruled out. \citeauthor{Walker2011a}'s proposals for \ili{English} would not help in such examples, since all arguments of
finite verbs are represented in one valence list in grammars of \ili{German}. Hence the highest domain in
which vc-command is defined (taking \citeauthor{HL95b}'s definition) is the full clause since \comps
would be empty at this level. There is the additional problem that the adjunct is fronted in a
non-local dependency (\ili{German} is a V2 language, see \cites[Chapter~2.4]{Erdmann1886a}[\page 69, \page 77]{Paul1919a}[Section~3]{MuellerGermanHandbook}) and that the arguments are scrambled in
(\mex{0}a). There is no VP node in the analysis of (\mex{0}a) that is commonly assumed in HPSG
grammars of \ili{German} and it is unclear how a reconstruction of the fronted adjunct into a certain
position could help explaining the differences in (\mex{0}).



%\subsection{Conclusion}

\itddone{‘conclusion’ is not clear=> should just be a sentence pointing to section 10\\
Stefan: Done}
Concluding this section, it seems that a totally non-configurational Binding Theory seems to be
impossible because of adjuncts and the combination of configurational and non-configurational parts
seems appropriate. 
%The subject of the embedded verb should not be included among the local domain of
%VP embedding verbs.

Section~\ref{sec-bt-nonlocal} discusses an alternative approach that collects indices in lists. This can be done in a
way that can be used to get the adjunct bindings right.
 





\section{Binding and passive: \texorpdfstring{\argst}{ARG-ST} lists with internal structure}
\label{binding-sec-passive}

\itddone{Section should come earlier\\
Stefan: Complete rearrangement of sections. Moved stuff to Section~\ref{sec-bt-nonlocal}, deleted Sections and put
this section earlier.}

\itddone{Anne: Section 9 should be called Binding and passive and should come before
Explain what (64) does for you with an example
*Himself disappointed John
John was disappointed by himself\\
Explain that  the by phrase of the passive is a complement, the prep is selected by the passive part, it does not assign a thematic role to the NP, the passive part does (see Abeillé and Godard on French and Romance passive, Sag, Wasow, Bender on English passive)}

\citet{MS98a} discuss binding in passive clauses. They suggest that the passive is analyzed as a
lexical rule demoting the subject argument and adding an optional PP.\footnote{
  See also \citew[\page 114, 116]{MS98a}, \citew{Mueller2003e}, \citew{MOe2013a}, and \citew[512]{Blevins2003a} for lexical rule-based analyses of the
%[\page 439]{AG2002b-u} just a footnote
%\itdopt{Which paper shall I cite on French? There is no rule just a footnote ...}
  passive in 
%\ili{French}, 
\ili{English}, \ili{German}, \ili{Danish}, Balto-Finnic\il{Finnic!Balto-}, and Balto-Slavic\il{Slavic!Balto}. \crossrefchaptert[Section~\ref{arg-st:sec-passives}]{arg-st} give an overview.
} 
\ea
Lexical rule for the passive in English:\\
% MS copy CONT redundantly and have a dangling \ibox{1}
\ms{
arg-st & \sliste{ NP$_i$, \ibox{1}, \ldots } \\
%cont   & \ibox{3}\\
}
$\mapsto$
\ms{
arg-st & \sliste{ \ibox{1}, \ldots } \upshape ( $\oplus$ \sliste{ PP[\type{by}]$_i$ })\\
%cont   & \ibox{3}\\
}
\z
The lexical rule applies to a verb with at least two arguments: the NP$_i$ and \ibox{1}. It licenses
the lexical item for the participle. The \argstl of the participle does not contain the subject NP
any longer but instead a PP object is appended to the list that is coindexed with the same argument $i$. 
The lexical rule does not show the \contv in the input and the output. A notational convention
regarding lexical rules is that values of features that are not mentioned are taken over unchanged
from the input. For our example, this means that linking is not affected. The index of the initial
element in the input $i$ was linked to a certain role and this index -- now associated with the PP
-- is linked to the same semantic role in the output. The PP does not assign a role. It just
functions like one of the prepositional objects discussed on
page~\pageref{binding:page-prepositional-objects-start}--\pageref{binding:page-prepositional-objects-end}
above. The examples in (\mex{1}) illustrate:

{\jamwidth=5cm
\eal
\ex[]{
John disappointed himself.\\
\argst \sliste{ NP$_i$, NP$_j$ }            \jambox{disappoint(i,j)}
}
\ex[]{
John was disappointed by himself.\\
\argst \sliste{ NP$_j$, PP[\type{by}]$_i$ } \jambox{disappoint(i,j)}
}
\zl
(\mex{0}a) shows the linking to the arguments of the finite verb \emph{disappoint}; the subject is
linked to the first argument and the object to the second. In the passive case in (\mex{0}b), the
logical object is realized as the subject but still linked to the second argument of
\emph{disappoint}. The former subject, now a realized as a PP, is linked to the first argument. 

The passive example in (\mex{0}b) would -- if one would just put the reflexive in subject position
-- correspond to (\mex{1}):
\ea[*]{
Himself disappointed John.\\
\argst \sliste{ NP$_j$, NP$_i$ }   \jambox{disappoint(i,j)}
}
\z
}

\noindent
Of course (\mex{0}) is ungrammatical because of the case of the reflexive pronoun: it is accusative
and hence cannot function as subject \citep[\page 388]{Brame77}. But the example would also be bad for binding reasons: the
reflexive cannot bind a more oblique argument. In any case, the discussion shows that a purely
semantic theory of binding would not work since the semantic representation in the examples above is
the same. It is the obliqueness of arguments that differs and that makes different binding options available.

So, the lexical rule-based approach to passive makes the right predictions as far as the English
data is concerned, but \citet{Perlmutter1984} argued that more complex representations are necessary to capture the
fact that some languages allow binding to the logical subject of the passivized verb. He discusses
examples from \ili{Russian}. While usually the reflexive has to be bound by the subject as in
(\mex{1}a), the antecedent can be either the subject or the logical subject in passives like (\mex{1}b):

\eal
\label{binding:russian-pass}
\ex  
\gll Boris$_i$    mne      rasskazal anekdot o sebe$_i$.\\
     Boris.\nom{} me.\dat{} told      joke    about \self\\
\glt `Boris told me a joke about himself.'
\itddone{Anne: But sebia is a long distance anaphor isn’t it?\\
Stefan: Yes, I say so in the text. It is MS's example.}
\ex
\gll Eta kniga byla kuplena Borisom$_{i}$ dlja sebja$_{i}$.  \\
     this book.\nom{} was bought Boris.\textsc{instr} for \self  \\
\glt `This book was bought by Boris for himself.'
\zl
In order to capture the binding facts, \citet{MS98a} suggest that passives of verbs like
\emph{kupitch} `buy' have the following representation at least in \ili{Russian}.
\ea
\emph{kuplena} `bought':\\
\ms{
arg-st & \sliste{ NP[\type{nom}]$_j$, \sliste{ NP[\type{instr}]$_i$, PRO$_j$, PP$_k$ } }\\[1mm]
cont   & \ms[buying]{
         actor       & i\\
         undergoer   & j\\
         beneficiary & k\\
        }
}
\z
The \argstl is not a simple list like the list for \ili{English} but it is nested. The complete \argstl of
the lexeme \emph{kupitch} `buy' is contained in the \argstl of the passive. The logical subject is
realized in the instrumental and the logical object is stated as PRO$_j$ on the embedded \argst but
as full NP in the nominative on the top-most \argstl. This setup makes it possible to account for
the fact that a long-distance reflexive (see p.\,\pageref{page-long-distance-reflexives}) like the
reflexive in the PP may refer to one of the two subjects: the nominative NP in the upper \argstl and
the NP in the instrumental in the embedded list. The PRO element is kept as a reflex of the
argument structure of the lexeme. Such PRO elements also play a role in binding phenomena in languages
like \ili{Chi-Mwi:ni} also discussed by \citeauthor{MS98a}.

In order to facilitate distributing the elements of such nested \argstls to valence features like
\subj and \comps, \citet[\page 124, 140]{MS98a} use a complex relational constraint that basically flattens the
nested \argst{}s again and removes all occurrences of PRO. An alternative would be to keep the
\argstl for linking, case assignment, and scope and use additional lists related to the \argstl for
binding. Such lists can contain PRO indices and additional indices for complex coordinations (see
Section~\ref{binding:sec-locality}). We discuss an approach assuming additional lists in Section~\ref{sec-bt-nonlocal}.
%\fi



\section{Austronesian: Disentangeling \argst and grammatical functions}
\label{binding:toba-batak}

So far we have discussed binding for \ili{English} with some occasional reference to \ili{Mandarin Chinese},
\ili{Portuguese} and \ili{German}. The question is whether Binding Theory is universal, that is, whether it is a
set of constraints holding for all languages or whether language specific solutions are necessary,
maybe involving a general machinery for establishing such solutions. In this section, we discuss
approaches suggested for \ili{Austronesian} languages.

\citet{MS98a} discuss data from \ili{Toba Batak}, a Western Austronesian\il{Austronesian!Western} language. They assume that the
\argst elements are in the order actor and undergoer, but since \ili{Toba Batak} has two ways to realize
arguments, the so-called \emph{active voice}\is{voice!active} and the \emph{objective voice}\is{voice!objective} either of the arguments
can be the subject. 
\eal
\ex
\gll Mang-ida        si Ria si Torus\\
     \textsc{av}-see \textsc{pm} Ria \textsc{pm} Torus\\
\glt `Torus sees/saw Ria.'
\ex\label{ex-toba-batak-objective-voice}
\gll Di-ida          si Torus si Ria\\
     \textsc{ov}-see \textsc{pm} Torus \textsc{pm} Ria\\
\glt `Torus sees/saw Ria.'
\zl
\citeauthor{MS98a} argue that the verb and the adjacent NP form a VP which is combined with the final NP
to yield a full clause. They furthermore argue that neither sentence in (\mex{0}) is a passive or anti-passive
variant of the other. Instead they suggest that the two variants are simply due to different
mappings from argument structure (\argst) to surface valence (\subj and \comps). They provide the
following lexical items:
\ea
\begin{tabular}[t]{@{}l@{~}ll@{~}l}
a. & \emph{mang-ida} `\textsc{av}-see': & b. & \emph{di-ida} `\textsc{ov}-see':\\
   & \ms{
phon & \phonliste{ mang-ida }\\
subj & \sliste{ \ibox{1} }\\
comps & \sliste{ \ibox{2} }\\
arg-st & \sliste{ \ibox{1} NP$_i$, \ibox{2} NP$_j$ }\\[1mm]
cont & \ms[seeing]{
       actor & i\\
       undergoer & j\\
       }
} & & \ms{
phon & \phonliste{ di-ida }\\
subj & \sliste{ \ibox{2} }\\
comps & \sliste{ \ibox{1} }\\
arg-st & \sliste{ \ibox{1} NP$_i$, \ibox{2} NP$_j$ }\\[1mm]
cont & \ms[seeing]{
       actor & i\\
       undergoer & j\\
       }
}
\end{tabular}
\z
The analysis of (\ref{ex-toba-batak-objective-voice}) is given in Figure~\ref{fig-toba-batak-objective-voice}.
Since the second argument, the logical object and undergoer is mapped to \subj in (\mex{0}b), it is combined
with the verb last.

\begin{figure}
\begin{forest}
sm edges
[\ms{ head & \ibox{3} \upshape V\\
      subj & \eliste\\
      comps & \eliste\\
      cont & \ibox{4} \ms[seeing]{
                      actor & i\\
                      undergoer & j\\
                      } }
  [\ms{ head & \ibox{3} \upshape V\\
        subj  & \sliste{ \ibox{2} }\\
        comps & \eliste\\
        cont  & \ibox{4} }
    [\ms{ head & \ibox{3} \upshape V\\
          subj  & \sliste{ \ibox{2} }\\
          comps & \sliste{ \ibox{1} }\\
          arg-st & \sliste{ \ibox{1}$_i$, \ibox{2}$_j$ }\\
          cont  & \ibox{4} } [di-ida;\textsc{ov}-see]]
    [\ibox{1} \ms{ head & \upshape N\\
                   spr & \eliste\\
                   comps & \eliste } [si Tours;\textsc{pm} Torus,roof]]]
    [\ibox{2} \ms{ head & \upshape N\\
                   spr & \eliste\\
                   comps & \eliste } [si Ria;\textsc{pm} Ria,roof]]]
\end{forest}
\caption{Analysis of Toba Batak example in objective voice according to \citet[\page 120]{MS98a}}\label{fig-toba-batak-objective-voice}
\end{figure}

But since binding is taken care of at the \argstl and this list is not affected by voice
differences, this account correctly predicts that the binding patterns do not change independent of
the realization of arguments: as the following examples show, it is always the logical subject, the
actor (the initial element on the \argstl) that binds the non-initial one.
\eal
\ex[]{
\gll [Mang-ida        diri-na$_i$] si John$_i$.\\
     \spacebr{}\textsc{av}-saw self-his \textsc{pm} John\\
\glt `John saw himself.'
}
\ex[*]{
\gll [Mang-ida si John$_i$] diri-na$_{*i}$.\\
     \spacebr{}\textsc{av}-saw \textsc{pm} John self-his\\
\glt Intended: `John saw himself.' with \emph{himself} as the (logical) subject
}
\zl
\eal
\ex[*]{
\gll [Di-ida          diri-na$_i$] si John$_i$\\
     \spacebr{}\textsc{ov}-saw self-his \textsc{pm} John\\
\glt Intended: `John saw himself.' with \emph{himself} as the (logical) subject
}
\ex[]{
\gll [Di-ida          si John$_i$] diri-na$_i$\\
     \spacebr{}\textsc{ov}-saw \textsc{pm} John self-his\\
\glt `John saw himself.'
}
\zl

\citet[\page 121]{MS98a} point out that theories relying on tree configurations will have to assume
rather complex tree structures for one of the patterns to establish the required c-command
relations. This is unnecessary for \argst-based binding theories.

\citet{WA98a-u} discuss similar data from \ili{Balinese} and provide a parallel analysis. This
analysis is also discussed in \crossrefchapterw[Section~\ref{arg-st-sec-ergativity}]{arg-st}. 
\citet{Wechsler99a-u} compares GB analyses with \argst-based HPSG analyses.
\itdobl{Steve: hmm, not to blow my own horn (and Wayan’s), but I think the Balinese work had considerably
  more theoretical importance and interest than what this paragraph suggests. It was based on
  Manning and Sag, as you say, but it went much further and looked at ditransitives and interactions
  with raising. As a result we were able to identify a deep theoretical difference between HPSG and
  GB (which I wrote about in Wechsler 1999), even when they seemed to be parallel, and show that the
  HPSG account extends without problem or stipulation while the seemingly parallel GB account led to
  an insoluble contradiction. Lisa Travis reacted to Wechsler and Arka by proposing a fundamental
  change to GB binding theory, apparently just to get Austronesian to work. David Pesetsky told me
  he considered Wechsler 1999 to be one of the strongest arguments for lexicalism he had ever
  seen. It has to do with formal differences between chains and reentrancies.}


The conclusion to be drawn from this section is that obliqueness should not be defined in terms of
grammatical functions as was done in (\ref{def-obliqueness-hierarchy}) above but rather with
reference to a thematic hierarchy as suggested by \citet{Jackendoff72a-u}.
% Yes, the whole book
\itdobl{Steve: That was not exactly our conclusion about Balinese. It was a little more interesting than that. ARG-ST for us is related to the thematic hierarchy, but it’s grammaticalized; for example it includes raised items, which receive no thematic role.}
This would not make a difference for languages like \ili{English}, but for languages like
\ili{Toba Batak} and \ili{Balinese} the arguments may be mapped to different grammatical functions
depending on the voice the verb is realized in. \added{Since \citet{WA98a-u} keep the \argst
representation, their theory not just refers to a thematic hierarchy but also provides explanations for
examples including raising.}





\section{Explicit constructions of lists with possible antecedents}
\label{sec-bt-nonlocal}

We mentioned on p.\,\pageref{page-traces-binding} that HPSG sees binding as crucially different from nonlocal
dependencies: while in GB the relation between a trace and its filler was seen as similar to pronoun
binding. This section explains how the general mechanism for nonlocal dependencies (see
\crossrefchapteralt{udc}) can be used to account for binding data and in which way this solves or
avoids problems of earlier approaches based on o-command. 

The idea to use the nonlocal mechanism was first suggested by
\citet[Section~7.2.3]{Bredenkamp96a}. He did not work out his proposal in detail (see 
p.\,104--105). He used the \slashf for percolation of binding information, which probably would
result in conflicts with true non-local dependencies. \citet{Hellan2005a} developed an account using
special nonlocal features for binding information. Both Bredenkamp and Hellan assume that the
binding information is bound off in certain structures as it is common in the treatment of nonlocal
dependencies in HPSG. In what follows we will look into \citegen{Branco2002a} account. Branco also
uses the nonlocal machinery of HPSG but in a novel way without something like a filler-head
schema. Before we look into the details, we want to discuss two phenomena that have not been
accounted for so far and that are problematic for a Binding Theory based on o-command: first, there
is nothing that rules out nominal heads as binders and second there are problems with
coordinations. Both problems can be solved if there is a bit more control of which indices are
involved in binding relations in which local environment.


\subsection{Nominal heads as binders}
\label{sec-nominal-heads-as-binders}

\itddone{Anne: Turn this into a footnote and delete section. This section should be skiped and the
  problematic ex put in section 10 where a solution to them is given.\\
Stefan: it is too long for a footnote.}

\citegen{ps2} definition of o-command has an interesting consequence: it does not say anything about possible
binding relations between heads and their dependents. What is regulated is the binding relations
between co-arguments and referential objects dominated by a more oblique coargument. As
\citet[\page 419]{Mueller99a} pointed out, bindings like the one in (\mex{1}) are not ruled out by
the Binding Theory of \citet[Chapter~6]{ps2}: 
\ea
his$_{*i}$ father$_i$
\z
The possessive pronoun is selected via \spr and hence a dependent of \emph{father}
(\citealt{MuellerHeadless,MyPM2021a}; \crossrefchapteralt[\page \pageref{knjiga-avm}]{agreement}), but the noun does not appear in any \argstl (assuming an NP
analysis, see also \crossrefchapteralt{np} for discussion). The consequence is that Principle~B and~C do not apply and the o-command-based Binding Theory just
does not have anything to say about (\mex{0}). This problem can be fixed by assuming \citegen{HL95b}
version of Principle~C together with their definition of vc-command in (\ref{def-vc-command-HL}).
This would also cover cases like (\mex{1}):
\ea
his$_{*i}$ father of John$_i$
\z

What is not accounted for so far is \citegen[\page 344]{Fanselow86a} examples in (\mex{2}):
\eal
\ex[*]{
\gll die Freunde$_i$ voneinander$_i$\\
     the friends     of.each.other\\
}
\ex[]{
der Besitzer$_i$ seines$_{*i}$ Botes\\
the owner        of.his       boat\\
}
\zl
These examples would be covered by an \iwithinic\is{i-within-i-Condition@\emph{i}-within"=\emph{i}"=Condition|(} as suggested by
\citet[\page 212]{Chomsky81a}. Chomsky's condition basically rules out configurations like the one
in (\mex{1}):
\ea
( \ldots{} x$_i$ \ldots{} )$_i$
\z
\citet[\page 244]{ps2} discuss the \iwithinic in their discussion of GB's Binding Theory but do not
assume anything like this in their papers. Nor was anything of this kind adopted anywhere else in
th discussion of binding. Having such a constraint could be a good solution, but as
\citet[\page 343]{Fanselow86a} working in GB pointed out, such a condition would also rule out cases like his examples in (\mex{1}):
\eal
\ex
\gll die sich$_i$ treue Frau$_i$\\
     the \self{} faithful woman\\
\glt `the woman who is faithful to herself'
\ex 
\gll die einander$_i$ verachtenden Männer$_i$\\
     the each.other   despising    men\\
\glt `the men who despise each other'
\zl
\ili{German} allows for complex prenominal adjectival phrases. The subject of the respective adjectives or
adjectival participles are coindexed with the noun that is modified. Since the reflexive and
reciprocal in (\mex{0}) are coindexed with the non-expressed subject and since this subject is
coindexed with the modified noun \citep[Section~3.2.7]{Mueller2002b}, a general \iwithinic cannot be formulated for HPSG
grammars of \ili{German}. The problem also applies to \ili{English}, although \ili{English} does not have complex
prenominal adjectival modifiers. Relative clauses basically produce a similar configuration:
\ea
the woman$_i$ seeing herself$_i$ in the mirror
\z
\itdopt{G: Culicover (1997: 71) cites the following:
(a) One finds [many books about themselves$_{i}$ ]$_{i}$  on Borges’s literary output.
}
The non-expressed subject in (\mex{0}) is the antecedent for \emph{herself} and since this element
is coindexed with the antecedent noun of the relative clause, we have a parallel situation.

\citet[\page 229, Fn.\,63]{Chomsky81a} notes that his formulation of the \iwithini rules out relative
clauses and suggests a revision. However, the revised version would not rule out the examples above
either, so it does not seem to be of much help.
\itdopt{G: It is often hard to see what i-within-i violations would even mean: e.g. his$_{i}$ father$_{i}$ is supposed to be a person X s.t. X is a father of X. At least pragmatically, we tend to view "father" as an irreflexive predicate. Could the i-within-i amount to a semantic/pragmatic constraint if some sort?}

In a version of the Binding Theory that is based on command relations in tree configurations, some
special constraint seems to be needed that rules out binding by and to the head of nominal 
constructions unless this binding is established by adnominal modifiers directly. The approach to
binding discussed below accounts for i-within-i problems by explicitly collecting
indices that are possible antecedents and excluding the unwanted indices in this collection. But
before we look into the details, we want to discuss another area that is problematic for
tree-configurational approaches in general, not just for the HPSG approach based on o-command.
\is{i-within-i-Condition@\emph{i}-within"=\emph{i}"=Condition|)}

%This failure is similar to the failure in the
%definition of c-command that can be found in \citew[\page 168]{SKS2013a-u}


%% \ea
%% Karl heiratet eine nur sich$_i$ selbst liebende Frau$_i$.
%% \z


\subsection{Binding and coordination: Questions of locality}
\label{binding:sec-locality}

\itddone{Section 7 should be called Binding and coordination (it is because the coord phrase is unheaded), and put as a subsection inside section 10}

\citet[Section~20.4.7]{Mueller99a} pointed out that examples like (\mex{1}) involving anaphors
within coordinations are problematic for the HPSG Binding Theory:
\ea
\label{ex-sich-und-seine-familie}
\gll Wir beschreiben ihm$_{i}$ [sich$_{i}$       und seine Familie].\\
     we describe     him      \spacebr{}\self{} and his family\\
\glt `We describe him and his family to him.'
\z
Since \emph{sich} `\textsc{self}' is not local to \emph{ihm} `him' and since reflexives are not exempt in \ili{German}
\citep[\page 158--159]{Kiss2012a}, \emph{ihn} `him' would be expected as the only option for a pronominal element within
the coordination.

\citet[\page 112]{Fanselow87a} discussed such examples in the context of a GB-style Binding
Theory. See also \citew[\page 420]{Mueller99a} for attested examples.
% Such coordination examples are used since weight plays a role in ordering constituents and
% putting the reflexives into a coordination makes the examples more natural.
% %Reis73:522
% %Hans_i läßt es zwischen Emma_j und sich_i / ihm_*i zum Streit kommen.
% %
% The example may still seem a bit artificial but there are attested examples from newspapers \citep[\page 420]{Mueller99a}:
% \eal
% \ex Die Erneuerung war ausschließlich auf Druck von außen zustande gekommen.
% \gll Sie verdankte sich keineswegs dem Bedürfnis, vor sich und der Öffentlichkeit Rechenschaft
% abzulegen.\footnotemark\\
%      she to.be.due.to \self{} not.at.all the desire before \self{} and the public account to.give\\
% \footnotetext{
%         Taz UNIspezial WS 94/95, 10.15.94, p.\,16
% }
% \glt `This was not due to the the desire to give account to oneself and the public.'
% \ex 
% \gll Martin Walser versucht, sich und die Nation zu verstehen.\footnotemark\\
%      Martin Walser tries     \self{} and the nation to understand\\
% \footnotetext{
%         taz, 12.10.98, p.\,1
% }
% \glt `Martin Walser tries to understand himself and the nation.'
% \zl
Such sentences pose a challenge for the way locality is defined as part of the definition of local
o-command. Local o-command requires that the commander and the commanded phrase are members of the
same \argstl (\ref{def-local-o-command}), but the result of coordinating two NPs is usually a complex NP with a plural index:
\ea
\gll Der Mann und die Frau kennen / * kennt das Kind.\\
     the man  and the woman know {} {} knows the child\\
\glt `The man and the woman know the child.'
\z
The NP \emph{der Mann und die Frau} `the man and the woman' is an argument of \emph{kennen} `to know'. The index of
\emph{der Mann und die Frau} `the man and the woman' is local with respect to \emph{das Kind} `the child'. The indices of
\emph{der Mann} `the man' and \emph{die Frau} `the woman' are embedded in the complex NP.

For the same reason \emph{sich} is not local to \emph{ihm} in (\mex{-1}). This means that the anaphor is not locally o-commanded in any
of the sentences and hence Binding Theory does not say anything about the binding of the reflexive
in these sentences: the anaphors are exempt.
\itdopt{G: So is this is a problem for the HPSG BT or for the analysis that says German does not have exempt anaphors? I think it is the latter. 
%
Maybe we can say German has exempt anaphors only in a subset of the cases where English does.
%
There is really a lot of cross-linguistic variation with respect to the contexts in which anaphors can be exempt. English is actually quite restrictive (more so than Japanese, Icelandic, Chinese and Brazilian Portuguese). It wouldn’t surprise me if  German was even more restrictive: only anaphors which are not themselves syntactic arguments on an ARG-STR list are exempt. }

For the same reason, \emph{ihn} `him' is not local to \emph{er} `he' in (\mex{1}b) and hence the
binding of \emph{ihn} `him' to \emph{er} `he', which should be excluded by Principle~B, is not ruled out.
\eal
\label{ex-sorgt-fuer-sich-und-seine-familie}
\ex
\gll Er$_{i}$ sorgt nur für [sich$_{i}$ und seine Familie].\\
     he      cares only for \spacebr{}\self{} and his family\\
\glt `He cares for himself and his family only.'
\ex 
\gll Er$_{i}$ sorgt nur für [ihn$_{*i}$ und seine Familie].\\
     he      cares only for \spacebr{}him and his family\\
\zl
%\inlinetodostefan{Bob: You could presumably propose a similar deletion-based account within a version of HPSG that has order domains. But it would presumably facce the samme objections as a transformational account.}
If one assumed transformational theories of coordination deriving (\mex{0}) from (\mex{1}) (see for example
\citealp[\page 303]{WC80a-u} and \citealp[\page 61, 67]{Kayne94a-u} for proposals to derive verb
coordination from VP coordination plus deletion), the problem would be solved, but as has been
pointed out frequently in the literature such transformation-based theories of coordinations have
many problems \parencites[\page 102]{BV72}[\page 192--193]{Jackendoff77a}[\page143]{Dowty79a}[\page
  104--105]{denBesten83a}{Klein85}{Eisenberg94a}[\page 471]{Borsley2005a} and nobody ever assumed
something parallel in HPSG (see \crossrefchaptert{coordination} on coordination in HPSG).
\ea
\gll Er$_{i}$ sorgt nur für sich     und er$_{i}$ sorgt nur für seine Familie.\\
     he      cares only for \self{} and he       cares only for his family\\
\z

\citet{RR93a} develop a Binding Theory that works at the level of syntactic or semantic
predicates. Discussing the examples in (\mex{1}) they argue that the semantic representation is
(\mex{2}) and hence their semantic restrictions on reflexive predicates apply.
\eal
\ex[]{
The queen invited both Max and herself to our party.
}
\ex[*]{
The queen$_1$ invited both Max and her$_1$ to our party.
}
\zl
\ea
the queen ($\lambda$ x (x invited Max \& x invited x))
\z
Such an approach solves the problem for coordinations with \emph{both \ldots{} and \ldots} having a
distributive reading. \citet[\page 677]{RR93a} explicitly discuss coordinations with a collective
reading. Since we have a collective reading in examples like
(\ref{ex-sorgt-fuer-sich-und-seine-familie}), examples like (\ref{ex-sorgt-fuer-sich-und-seine-familie}) continue to pose a problem. There are
however ways to cope with such data: one is to assume a construction-based account to binding
domains. The details of an account that makes this possible will be discussed in the following subsection.
\itdopt{G: Weird, but cool! So even under a collective non-distributive reading (62b) is out?
R\&R’s analysis seems to work well for English. (a) seems fine under a collective reading:
(a) John only cares for [him and his family]
When (a) is interpreted collectively, no reflexive semantic predicate is formed. So R\&R's Condition B is not violated.}


\subsection{The list-threading approach to binding}


The discussion of early HPSG approaches to binding revealed a number of problems. The proposals are
based on tree configurations and on command relations. This is basically the conceptual inheritance
of the GB Binding Theory, of course with a lot of improvements. The general problem seems to be that
the command relations are defined in a uniform way, without taking into account special configurations such as coordinate structures.

Now, there is a more recent approach to binding that looks technical at first, but it is the
solution to the problems caused by an approach assuming one command relation that is supposed
to work for all structures in all languages. \citet{Branco2002a} suggested an approach that collects
indices that are available for binding in certain binding domains.\footnote{%
For a much more detailed overview of Branco's approach see \citew{Branco2021a}.
} Since the way in which indices
relevant for binding are collected can be specified with reference to specific constructions the
problems mentioned so far can be circumvented. 

\citet{Branco2002a} argues that sentences with wrong bindings of pronouns and/or reflexives are not
syntactically ill-formed but semantically deviant. For the representation of his Binding Theory, he
assumes Underspecified Discourse Representation Theory (UDRT, \citealp{Reyle93b-u,FR95a-u}) as the
underlying formalism for semantics (see also
\crossrefchapteralp[Section~\ref{semantics:sec-semantic-underspecification}]{semantics}). 

Similar to the notion assumed in Minimal Recursion Semantics (MRS, \citealp*{CFPS2005a}) there is an
attribute for distinguished labels that indicate the upper (\textsc{l-max}) and lower
(\textsc{l-min}) bound for quantifier scope, there is a set of subordination condition for
quantifier scope (the \textsc{hcons} set in MRS), a list of semantic conditions (the \textsc{rels}
set in MRS). In addition, \citeauthor{Branco2002a} suggests a feature \textsc{anaph(ora)} for handling
the Binding Theory constraints. Information about the anaphoric potential of nominals is represented
here. There is a reference marker represented under \textsc{r(eference)-mark(er)} and there is a list
of reference markers under \textsc{antec(edents)}. The list is set up in a way so that it contains
the antecedent candidates of a nominal element. Furthermore, \citeauthor{Branco2002a} adds special
lists containing antecedents for special types of anaphora. The lists are named after the binding
principles that were already discussed in previous sections: \textsc{list-a} contains all reference
markers of elements that locally o-command a certain nominal expression \emph{n} ordered with
respect to their obliqueness, \textsc{list-z} contains all o-commanders also including everything
from \textsc{list-a}. The elements in \textsc{list-z} may come from various
embedded clauses and are also ordered with respect to their obliqueness. The list \textsc{list-u}
contains all the reference markers in the discourse context including those not linguistically
introduced. The list \textsc{list-lu} is an auxiliary list that will be explained below.
\ea
\avm{
[loc|cont & [\type*{udrs}
              ls & [l-max & \1\\
                    l-min & \1 ]\\
	      subord & \{ \ldots \}\\
	      conds  & \{ \ldots \}\\
              anaph  & [r-mark & refm\\
                        antec  & \listOf{refm} ] ]\\
  nonloc|bind & [\type*{bind}
                 list-a  & \listOf{refm} \\
		 list-z  & \listOf{refm} \\
		 list-u  & \listOf{refm} \\
		 list-lu & \listOf{refm} ] ]
}
\z
The lists containing possible antecedents for various nominal elements are represented under
\textsc{nonlocal} as the value of a newly introduced feature \attrib{bind}. These binding lists differ from
other \textsc{nonlocal} features in that nothing is ever removed from them (on unbounded dependencies
and \textsc{nonlocal} features in general see \crossrefchapterw{udc}). Before we provide the principles that
determine the list values, we explain an example:
Figure~\ref{fig-every-student-thought-she-saw-herself} shows the relevant aspects of the analysis of (\mex{1}):
\ea
Every student thought that she saw herself.
\z
\begin{sidewaysfigure}
\centering
\resizebox{.8\textwidth}{!}{%
\begin{forest}
%sm edges
[\avm{
   [ list-a  & < > \\
     list-z  & < > \\
     list-u  & < \1, \2, \3, \4, \5 >\\
     list-lu & < \1, \2, \3, \4, \5 > ] }
   [\avm{
     [ \ldots cont|conds & < \ldots, [ arg-r & \1 ], \ldots > \\
       \ldots |binding   & [ list-a  & < > \\
                             list-z  & < > \\
                             list-u  & < \1, \2, \3, \4, \5 >\\
                             list-lu & < \1 > ] ]}
     [ctx]]
   [\avm{
     [ list-a  & < \2, \3 > \\
       list-z  & < \2, \3 > \\
       list-u  & < \1, \2, \3, \4, \5 >\\
       list-lu & < \2, \3, \4, \5 > ]} 
     [\avm{
       [ \ldots anaphora & [ r-mark & \3 \\
                             var    & \2 \\ ]\\
         \ldots |binding   & [ list-a  & < \2, \3 > \\
                               list-z  & < \2, \3 > \\
                               list-u  & < \1, \2, \3, \4, \5 >\\
                               list-lu & < \2, \3 > ] ]} 
       [every student, roof]]
      [\avm{
          [ list-a  & < \2, \3 > \\
            list-z  & < \2, \3 > \\
            list-u  & < \1, \2, \3, \4, \5 >\\
            list-lu & < \4, \5 > ]} 
        [thought]
        [e
          [that]
          [e
            [\avm{
              [ \ldots anaphora & [ r-mark & \4 \\
                                    antec  & < \1, \2, \3, \5 > \\ ]\\
                \ldots |binding & [ list-a  & < \4, \5 > \\
                                    list-z  & < \2, \3, \4, \5 > \\
                                    list-u  & < \1, \2, \3, \4, \5 >\\
                                    list-lu & < \4 > ] ]}
              [she]]
            [\avm{
                [ list-a  & < \4, \5 > \\
                  list-z  & < \2, \3, \4, \5 > \\
                  list-u  & < \1, \2, \3, \4, \5 >\\
                  list-lu & < \5 > ] }
              [saw]
              [\avm{
                [ \ldots anaphora & [ r-mark & \5 \\
                                      antec  & < \4 > \\ ]\\
                  \ldots |binding & [ list-a  & < \4, \5 > \\
                                      list-z  & < \2, \3, \4, \5 > \\
                                      list-u  & < \1, \2, \3, \4, \5 >\\
                                      list-lu & < \5 > ] ]}
                [herself]]]]]]]]
\end{forest}}
\caption{Partial grammatical representation of \emph{Every student thought that she saw herself}.}\label{fig-every-student-thought-she-saw-herself}
\end{sidewaysfigure}
The noun phrase \emph{every student} introduces the reference marker (\textsc{r-mark}) \ibox{3} for e-type anaphora
\citep{Evans80a-u} and as value of \textsc{var} the value used for bound-variable anaphora
interpretations \citep{Reinhart83a-u}. This is \ibox{2} in the example. The pronouns \emph{she} and \emph{herself} introduce the
reference markers \ibox{4} and \ibox{5} respectively. All these reference markers are added to the
book keeping list \textsc{list-lu} of the respective lexical items: \emph{she} has \ibox{4} in its
\listlu and \emph{herself} has \ibox{5} in this list. The noun phrase \emph{every student} has both
the variable \ibox{2} and the reference marker \iboxb{3} in the \listlu. As can be seen by looking at the
individual nodes in Figure~\ref{fig-every-student-thought-she-saw-herself}, the elements of \listlu
in daughters are collected at the mother node. The element \emph{ctx} is an empty element that
stands for the non-linguistic context. It is combined with one or more sentences to form a text
fragment (see also \crossrefchapterw{pragmatics} for discourse models and HPSG). The \textsc{conds}
list of the \emph{ctx} element contains semantic relations that hold of the world and all reference
markers contained in these relations are also added to the \listlul. In the example this is just
\ibox{1}. The example shows just one sentence that is combined with the empty head, but in principle
there can be arbitrarily many sentences. The \listlul at the top node contains all reference markers contained in
all sentences and the non-linguistic context. 

The top node of Figure~\ref{fig-every-student-thought-she-saw-herself} is licensed by a schema that
also identifies the \listu value with the \listlu value. The \listu value is shared between mothers
and their daughters and since \listlu is a collection of all referential markers in the tree and
this collection is shared with \listu at the top node, it is ensured that all nodes have an \listu
value that contains all reference markers available in the whole discourse. In our example, all
\listu values are \sliste{ \ibox{1}, \ibox{2}, \ibox{3}, \ibox{4}, \ibox{5} }.

\lista values are determined with respect to the argument structures of governing heads. So the
\lista value of \emph{thought} is \sliste{ \ibox{2}, \ibox{3} } and the one of \emph{saw} is
\sliste{ \ibox{4}, \ibox{5} }. The \lista values of NP or PP arguments are identical to the ones of the head,
hence \emph{she} and \emph{herself} have the same \lista value as \emph{saw} and \emph{every
  student} has the same \lista value as \emph{thought}. Apart from this the \lista value is
projected along the head path in non-nominal and non-prepositional projections. For further cases
see \citet[\page 77]{Branco2002a}.

The value of \listz is determined as follows \citep[\page 77]{Branco2002a}: for all sentences combined with the context element,
the \listz value is identified with the \lista value. Therefore the \listz value of \emph{every
  student thought that she saw herself} is \sliste{ \ibox{2}, \ibox{3} }: the \lista value is
projected from \emph{thought} and then identified with the \listz value. In sentential daughters
that are not at the top-level, the \listz value is the concatenation of the \listz value of the
mother and the \lista value of the sentential daughter. In other non-filler daughters of a sign, the
\listz value is structure shared with the \listz value of the sign. For example, \emph{she} and
\emph{saw} and \emph{herself} have the same \listz value, namely \sliste{ \ibox{2}, \ibox{3},
  \ibox{4}, \ibox{5} }.

\citet[\page 78]{Branco2002a} provides the following lexical item for a pronoun:
\eas
Parts of the \synsemv for \emph{she}:\\
\avm{
[loc|cont & [ ls & [l-max & \1\\
                    l-min & \1 ]\\
              subord & \{ \} \\
	      conds  & \{ [label & \1\\
                           dref  & \2 ] \}\\
              anaph & [r-mark & \2\\
                       antec  & \texttt{principleB}(\4,\3,\2)] ]\\
  nonloc|bind & [list-a  & \3 \\
		 list-z  & \listOf{refm} \\
		 list-u  & \4 \\
		 list-lu & < \2 > \\] ] }
\zs
The interesting thing about the analysis is that all information that is needed to determine
possible binders of the pronoun are available in the lexical item of the pronoun. The relational
constraint \texttt{principleB} takes as input the \listal \ibox{3}, the \listu list \ibox{4} and the reference marker
of the pronoun under consideration \iboxb{2}. The result of the application of \texttt{principleB} is the list
of reference markers that does not contain elements locally o-commanding the pronoun, since all
o-commanders of the reference marker \ibox{2}, which are contained in the \lista are removed from \listu (the
list of all reference markers in the complete discourse). In the case of \emph{she} in our example,
\texttt{principleB} returns the complete discourse \sliste{ \ibox{1}, \ibox{2}, \ibox{3}, \ibox{4},
  \ibox{5} } minus all reference markers of elements less oblique than \ibox{4}, which is the empty
list, minus \ibox{4} since the pronoun is not a possible antecedent of itself. So, the list of
possible antecedents of \emph{she} is \sliste{ \ibox{1}, \ibox{2}, \ibox{3}, \ibox{5} }. This list
contains \ibox{5} as a possible binder, which is of course unwanted. According to
\citet[\page 84]{Branco2002a}, \emph{herself} as a binder of \emph{she} is ruled out, since
\emph{she} binds \emph{herself}.

The \synsemv for \emph{herself} is shown in (\mex{1}):
\eas
Parts of the \synsemv for \emph{herself}:\\
\avm{
[loc|cont & [ ls & [l-max & \1\\
                    l-min & \1 ]\\
	      subord & \{ \} \\
	      conds  & \{ [label & \1\\
                           dref  & \2 ] \}\\
              anaph & [r-mark & \2\\
                       antec  & \texttt{principleA}(\3,\2)] ]\\
  nonloc|bind & [list-a  & \3 \\
		 list-z  & \listOf{refm} \\
		 list-u  & \listOf{refm} \\
		 list-lu & < \2 > \\] ]
}
\zs
\lista contains the reference markers of locally o-commanding phrases \iboxb{3}. Together with the
reference marker of \emph{herself} \iboxb{2}, \ibox{3} is the input to the relational constraint
\texttt{principleA}. This constraint returns a list containing all possible binders for \ibox{2},
that is, all elements of \ibox{3} that are less oblique than \ibox{2}. If there is no such element,
the returned list is the empty list and the anaphor is exempt (see Section~\ref{sec-excempt-anaphors}).

%{\sloppypar
The example discussed here involves a personal pronoun and a reflexive. The antecedents were
determined by the relational constraints \texttt{principleB} and \texttt{prin\-cipleA}. Further
relational constraints are assumed for long-distance reflexives (\texttt{principleZ}) and normal
referential NPs (\texttt{principleC}). \texttt{principleC} is part of the specification of the
specifier used in non-lexical anaphoric nominals \citep[\page 79]{Branco2002a}.
%}

The setting up of the \lista and \listu list is flexible enough to take care of problems that are
unsolvable in the standard HPSG approach (and in GB approaches). For example, the \listul of a noun
phrase can be set up in such a way that the reference marker of the whole NP, which is introduced by
the specifier is not contained in the \listul of the \nbar that is combined with it. As pointed out
by \citet[\page 76]{Branco2002a}, this solves \emph{i-within-i} puzzles, which were discussed in Section~\ref{sec-nominal-heads-as-binders}.

Note also that this flexibility in determining the lists of possible local antecedents on a
construction specific basis makes it possible for the first time to account for puzzling data like
the coordination data discussed in Section~\ref{binding:sec-locality}. If the coordination analysis
standardly assumed in HPSG (see \crossrefchapterw{coordination}) is on the right track, a special
rule for licensing coordination is needed and this rule can also incorporate the proper
specification of binding domains with respect to coordination.


Summing up, it can be said that the lexical, list-based solution discussed in this last section
provides flexibility in defining binding domains and can cope with the \iwithini problem and
problems of locality. 

\itdopt{G: Branco’s proposal is really interesting. It is sort of similar in spirit to Dalryimple’s LFG approach which encodes binding requirements in terms of functional uncertainty equations specified in the lexical entries for pronouns. These equations impose constraints on the types of antecedents pronouns can have as well as on the ``path'' between the pronouns their antecedents.}


\section{Conclusion}

We discussed several approaches to Binding Theory in HPSG. It was shown that the
valence-based approach referring to the \argstl of lexical items has advantages over proposals
exclusively referring to tree configurations. Since tree configurations play a minor role in HPSG's
Binding Theory, binding data does not force syntacticians to assume structures branching in a
certain way. This sets HPSG apart from theories like Government \& Binding and Minimalism, in which
empty nodes are assumed for sentences with ditransitive verbs in order to account for binding facts
\crossrefchapterp[\pageref{minimalism:page-binding-branching-start}--\pageref{minimalism:page-binding-branching-end}]{minimalism}. 

A further highlight is the treatment of so-called exempt anaphors, that is, anaphors that are not
commanded by a possible antecedent. \citet{PS92a}
% several sections
argued that these anaphors should not be regarded
as constrained by the Binding Theory and hence that binding by antecedents outside of the clause or
the projection are possible.

Finally, we discussed a lexical approach to binding making all the relevant binding information
available locally within lexical items of pronouns. This approach is flexible enough to deal with
problematic aspects like the \iwithini situations and locality problems in coordinated structures.

\section*{Abbreviations}

\begin{tabularx}{.99\textwidth}{@{}lX}
\textsc{av} & agentive voice\\
\textsc{ov} & objective voice\\
\textsc{pm} & pivot marker\\
\end{tabularx}

\section{Todo}

% \citet{AGS1998a,PX98a,PX2001a-u,Riezler95a,XMcF98a-u,XPS94a-u,Buering2005a-u}

\citew{Levine2010a-u,Golde1999a-u,VaraschinCulicoverWinkler2021a-u}

\section*{\acknowledgmentsUS}

%The research reported in this chapter was partially supported by the 
%Research Infrastructure for the Science and Technology of Language (\mbox{PORTULAN CLARIN}).

We thank Anne Abeillé and Bob Borsley for detailed comments on earlier versions of the paper and Bob Levine for discussion and Giuseppe Varaschin for comments.
% and for making \citet{HL95b}
% available to me. 


% \section{Unintegrated material}


% Note that percolating binding information seems to be the only way
% to account for binding data for HPSG variants assuming that linguistic objects do not have internal
% structure, \eg \sbcg. See \crossrefchaptert[Section~\ref{cxg:sec-sbcg}]{cxg} for discussion.

%A variant of Binding Theory that uses the HPSG mechanism for nonlocal dependencies in an innovative
%way is discussed in Section~\ref{sec-bt-nonlocal}.%



{\sloppy
\printbibliography[heading=subbibliography,notkeyword=this]
}
\end{document}


%      <!-- Local IspellDict: en_US-w_accents -->


\if 0

Chung98a:

beide Objekte sind gleich oblique

Local P-command: X locally p-commands Y iff either
(i) X locally obliqueness-commands (locally o-commands) Y, or
(ii) X and Y are equally oblique and X linearly precedes Y.





BMS2001: 44 adjuncts vs. complements. Binding cannot take place on ARG-ST

*I told themi about [the twins’]i birthday.
b. I only get themi presents on [the twins’]i birthday.


Changes: Removed 




Problem with raising verbs removed.

\fi
