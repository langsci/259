%% -*- coding:utf-8 -*-

% !BIB TS-program = biber
% !TeX TS-program = xelatexmk

\documentclass[output=paper
                ,modfonts
                ,nonflat
	        ,collection
	        ,collectionchapter
	        ,collectiontoclongg
 	        ,biblatex
                ,babelshorthands
                ,newtxmath
                ,draftmode
                ,colorlinks, citecolor=brown
]{./langsci/langscibook}

\IfFileExists{../localcommands.tex}{%hack to check whether this is being compiled as part of a collection or standalone
  % add all extra packages you need to load to this file 

\usepackage{graphicx}
\usepackage{tabularx}
\usepackage{amsmath} 
\usepackage{multicol}
\usepackage{lipsum}
%%%%%%%%%%%%%%%%%%%%%%%%%%%%%%%%%%%%%%%%%%%%%%%%%%%%
%%%                                              %%%
%%%           Examples                           %%%
%%%                                              %%%
%%%%%%%%%%%%%%%%%%%%%%%%%%%%%%%%%%%%%%%%%%%%%%%%%%%%
% remove the percentage signs in the following lines
% if your book makes use of linguistic examples


\usepackage{./langsci/styles/langsci-gb4e} 
\usepackage{./langsci/styles/langsci-optional} 
\usepackage{./langsci/styles/langsci-lgr}
\usepackage{./langsci/styles/langsci-forest-setup}
\usepackage{morewrites}



% Stefan Müller's styles
\usepackage{./styles/merkmalstruktur,./styles/abbrev,./styles/makros.2e,./styles/my-xspace,./styles/article-ex,
./styles/eng-date}

\usepackage{./langsci/styles/jambox}

% Crossing out text
% uncomment when needed
%\usepackage{ulem}

\usepackage{./styles/additional-langsci-index-shortcuts}

\usepackage{./styles/avm+}

\renewcommand{\tpv}[1]{{\avmjvalfont\itshape #1}}

\regAvmFonts

\usepackage{theorem}

\newtheorem{mydefinition}{Def.}
\newtheorem{principle}{Principle}

{\theoremstyle{break}
\newtheorem{schema}{Schema}
\newtheorem{mydefinition-break}[mydefinition]{Def.}
\newtheorem{principle-break}[principle]{Principle}
}

\usepackage{subfig}

  %add all your local new commands to this file

\makeatletter
\def\blx@maxline{77}
\makeatother
  \tikzset{external/prefix={hpsg-handbook.for.dir/}}
  \forestset{external/master dir={./}}
  \tikzexternalize
  %% hyphenation points for line breaks
%% Normally, automatic hyphenation in LaTeX is very good
%% If a word is mis-hyphenated, add it to this file
%%
%% add information to TeX file before \begin{document} with:
%% %% hyphenation points for line breaks
%% Normally, automatic hyphenation in LaTeX is very good
%% If a word is mis-hyphenated, add it to this file
%%
%% add information to TeX file before \begin{document} with:
%% %% hyphenation points for line breaks
%% Normally, automatic hyphenation in LaTeX is very good
%% If a word is mis-hyphenated, add it to this file
%%
%% add information to TeX file before \begin{document} with:
%% \include{localhyphenation}
\hyphenation{
A-la-hver-dzhie-va
anaph-o-ra
affri-ca-te
affri-ca-tes
Atha-bas-kan
com-ple-ments
Da-ge-stan
Dor-drecht
er-klä-ren-de
Ginz-burg
Gro-ning-en
Jon-a-than
Ka-tho-lie-ke
Ko-bon
krie-gen
Le-Sourd
moth-er
Mül-ler
Nie-mey-er
Prze-piór-kow-ski
phe-nom-e-non
re-nowned
Rie-he-mann
un-bound-ed
}

% why has "erklärende" be listed here? I specified langid in bibtex item. Something is still not working with hyphenation.


% to do: check
%  Alahverdzhieva

\hyphenation{
A-la-hver-dzhie-va
anaph-o-ra
affri-ca-te
affri-ca-tes
Atha-bas-kan
com-ple-ments
Da-ge-stan
Dor-drecht
er-klä-ren-de
Ginz-burg
Gro-ning-en
Jon-a-than
Ka-tho-lie-ke
Ko-bon
krie-gen
Le-Sourd
moth-er
Mül-ler
Nie-mey-er
Prze-piór-kow-ski
phe-nom-e-non
re-nowned
Rie-he-mann
un-bound-ed
}

% why has "erklärende" be listed here? I specified langid in bibtex item. Something is still not working with hyphenation.


% to do: check
%  Alahverdzhieva

\hyphenation{
A-la-hver-dzhie-va
anaph-o-ra
affri-ca-te
affri-ca-tes
Atha-bas-kan
com-ple-ments
Da-ge-stan
Dor-drecht
er-klä-ren-de
Ginz-burg
Gro-ning-en
Jon-a-than
Ka-tho-lie-ke
Ko-bon
krie-gen
Le-Sourd
moth-er
Mül-ler
Nie-mey-er
Prze-piór-kow-ski
phe-nom-e-non
re-nowned
Rie-he-mann
un-bound-ed
}

% why has "erklärende" be listed here? I specified langid in bibtex item. Something is still not working with hyphenation.


% to do: check
%  Alahverdzhieva

  \bibliography{../Bibliographies/stmue,
                ../localbibliography,
../Bibliographies/formal-background,
../Bibliographies/understudied-languages,
../Bibliographies/phonology,
../Bibliographies/case,
../Bibliographies/evolution,
../Bibliographies/agreement,
../Bibliographies/lexicon,
../Bibliographies/np,
../Bibliographies/negation,
../Bibliographies/argst,
../Bibliographies/binding,
../Bibliographies/complex-predicates,
../Bibliographies/coordination,
../Bibliographies/relative-clauses,
../Bibliographies/udc,
../Bibliographies/processing,
../Bibliographies/cl,
../Bibliographies/dg,
../Bibliographies/islands,
../Bibliographies/diachronic,
../Bibliographies/gesture,
../Bibliographies/semantics,
../Bibliographies/pragmatics,
../Bibliographies/information-structure,
../Bibliographies/idioms,
../Bibliographies/cg,
../Bibliographies/udc}

  \togglepaper[14]
}{}



\title{Anaphoric Binding} 
\author{%
Stefan Müller\affiliation{Humboldt-Universität zu Berlin} \lastand António Branco\affiliation{University of Lisbon}
}
% \chapterDOI{} %will be filled in at production

% \epigram{}

\abstract{
This chapter is an introduction into the Binding Theory assumed within HPSG. While it was inspired
by work on Government \& Binding in the beginning, it turned out that reference to tree structures
alone is not sufficient and that reference to the syntactic level of argument structure is
required. Since the argument structure is tightly related to the thematic structure, HPSG's Binding
Theory is a mix of aspects of thematic binding theories and entirely configurational theories. This
discusses both the advantages of this new view and open issues.
}


\begin{document}
\maketitle
\label{chap-binding}


\section{Introduction} 

Binding Theories deal with questions of \isi{coreference} and \isi{agreement} of coreferring
items. For example, the reflexives in (\mex{1}) have to refer to the referent of an NP being a
coargument of the reflexive and they have to have the same \isi{gender} as the NP they are coreferent with:
\eal
\label{ex-binding-reflexives}
\ex[]{
Peter$_i$ thinks that Mary$_j$ likes herself$_{*i/j/*k}$.
}
\ex[*]{
Peter$_i$ thinks that Mary$_j$ likes himself$_{*i/*j/*k}$.
}
\ex[*]{
Mary$_i$ thinks that Peter$_j$ likes herself$_{*i/*j/*k}$.
}
\ex[]{
Mary$_i$ thinks that Peter$_j$ likes himself$_{*i/j/*k}$.
}
\zl
The indices show what bindings are possible and which ones are ruled out. For example, in
(\mex{0}a), \emph{herself} cannot refer to \emph{Peter}, it can refer to \emph{Mary} and it cannot
refer to some discourse referent that is not mentioned in the sentence (indicated by the index
$k$). Coreference of \emph{himself} and \emph{Mary} is ruled out in (\mex{0}b), since \emph{himself}
has an incompatible gender.

At first look it may seem possible to account for the binding relations of reflexives on the
semantic level: it seems to be the case that reflexives and their antecedents have to be semantic
arguments of the same predicate.\footnote{%
  See \citet{Riezler95a} for a way to formalize this in HPSG.
} For examples like (\mex{0}), this makes the right predictions,
since the reflexive is the undergoer of \emph{likes} and the only possible antecedent is the actor of
\emph{likes}. However, there are raising predicates like \emph{believe} not assigning semantic roles
to their objects but nevertheless allowing coreference of the raised element and the subject of
\emph{believe}:\footnote{%
 See \citealp[Chapter~3.5]{ps2} and \crossrefchapteralp{control-raising} on raising. The example
 (\mex{1}) is taken from \citew[\page 128]{MS98a}.%
}
\ea
John$_i$ believes himself$_i$ to be a descendant of Beethoven.
\z
The fact that \emph{believes} does not assign a semantic role to its object is confirmed by the
possibility to embed predicates with an expletive subject under \emph{believe}:\footnote{%
The example is due to \citet[\page 137]{ps2}. See the sources cited above for further discussion.
}
\ea
Kim believed there to be some misunderstanding about these issues.
\z
So, it really is the clause or -- to be more precise -- some syntactically defined local domain in
which reflexive pronouns have to be bound.\footnote{%
Another argument against a binding theory relying exclusively on semantics is different binding
behavior in active and \isi{passive} sentences: since the semantic contribution is the same for active
and passive sentences, the difference in binding options cannot be explained in semantics-based
approaches. For more on passive see Section~\ref{binding-sec-passive}. For a general discussion of
thematic approaches to binding see \citew[Section~8]{PS92a} and \citew[Section~6.8.2]{ps2}.
}

Personal pronouns can not refer to an antecedent within the same clause:
\eal
\ex[]{
Peter$_i$ thinks that Mary$_j$ likes her$_{*i/*j/k}$.
}
\ex[]{
Peter$_i$ thinks that Mary$_j$ likes him$_{i/*j/k}$.
}
\ex[]{
Mary$_i$ thinks that Peter$_j$ likes her$_{i/*j/k}$.
}
\ex[]{
Mary$_i$ thinks that Peter$_j$ likes him$_{*i/*j/k}$.
}
\zl
As the examples show, the pronouns \emph{her} and \emph{him} cannot be coreferent with the subject
of \emph{likes}. If a speaker wants to express coreference, he or she has to use a reflexive pronoun
as in (\ref{ex-binding-reflexives}). 

Interestingly, the binding of pronouns is less restricted than that of reflexives, but this does
not mean that anything goes. For example, a pronoun cannot bind a full referential NP if the NP is
embedded in a clause and the pronoun is in the matrix clause:
\eal
\label{ex-he-thinks-that-Peter}
\ex[]{
He$_{*i/*j/k}$ thinks that Mary$_i$ likes Peter$_j$.
}
\ex[]{
He$_{*i/*j/k}$ thinks that Peter$_i$ likes Mary$_j$.
}
\zl  

The sentences discussed so far can be assigned a structure like the one in Figure~\ref{fig-binding-gb}.
\begin{figure}
\begin{forest}
sm edges without translation
[S
  [NP [John\\John\\he]]
  [VP
    [V [thinks\\thinks\\thinks]]
    [CP 
      [C [that\\that\\that]]
      [S
        [NP [Paul\\Paul\\Mary]]
        [VP
         [V [likes\\likes\\likes]]
         [NP [him\\himself\\Peter]]]]]]]
\end{forest}

\caption{\label{fig-binding-gb}Tree configuration of examples for binding}
\end{figure}
\citet{Chomsky81a,Chomsky86a} suggested that tree-configurational properties play a role in
accounting for binding facts. He uses the notion of c(onstituent)-command going back to
work by \citet{Reinhart76a-u}. \isi{c-command} is a relation that holds between nodes in a
tree. According to one definition, a node c-commands its sisters and the constituents of its sisters.%
%% Y is said to c-command another node Z, iff Y and Z
%% are sisters or if a sister of Y dominates Z.
\footnote{\label{fn-c-command-GB}%
``Node A c(onstituent)-commands node B if neither A nor B dominates the other and the first
  branching node which dominates A dominates B.'' \citet[\page 32]{Reinhart76a-u}

\citet{Chomsky86a} uses another definition that allows one to go up to the next maximal projection
dominating A. As of 2020-02-25 the English and German Wikipedia pages for c-command have two
conflicting definitions of c-command. The English version follows \citet[\page 168]{SKS2013a-u}, whose
definition excludes c-command between sisters: ``Node X c-commands node Y if a sister of X dominates Y.''
}

To take an example, the NP node of
\emph{John} c-commands all other nodes dominated by S. The V of \emph{thinks} c-commands everything
within the CP including the CP node, the C of \emph{that} c-commands all nodes in S including also S
and so on. The CP c-commands the \emph{think}-V, and the \emph{likes him}-VP c-commands the
\emph{Paul}-NP. Per definition, a Y binds Z just in case Y and Z are coindexed and Y c-commands
Z. One precondition for being coindexed (in English) is that the person, number, and gender features
of the involved items are compatible.

Now, the goal is to find restrictions that ensure that reflexives are bound locally, personal
pronouns are not bound locally and that referential expressions like proper names and full NPs are
not bound by other expressions (anaphors, pronouns or fully referential expressions). The conditions that were developed for
GB's Binding Theory are complex. They also account for the binding of traces that are the result of
moving elements by transformations. While it is elegant to subsume filler-gap relations (and other
relations between moved items and their traces) under a general Binding Theory, proponents of HPSG think that coreferential semantic indices and
filler-gap
%% \inlinetodostefan{Bob:
%% For GB binding theiry was concerned not just with A'-movement traces but also with A-movement traces, hence not just with filler-gap relations.  (1) and (2) were assumed to be bad for the sae reason as (3)
%% %
%% 1) *Kim seems [t is clever]
%% 2) *Kim is believed [t is clever]
%% 3) *Kim believes [himself is clever]
%% %
%% And of course all reflexives and reciprocals were assumed to be subject to binding theory, a position rejected in HPSG.
%% }
dependencies are crucially different.\footnote{\label{binding-fn-percolation-of-indices}%
The treatment of relative and interrogative pronouns is special, but they have a special
distribution that has to be accounted for. Relative clauses are the topic of Section~\ref{binding-sec-relative-pronouns}.
\citet[Section~7.2.3]{Bredenkamp96a} was an early suggestion to model binding relations of personal pronouns and anaphors by
the same means as filler-gap dependencies (see \crossrefchaptert{udc} for an overview of the
mechanisms for dealing with unbounded dependencies\is{unbounded dependency}). Bredenkamp did not work out his proposal in detail (see
p.\,104--105). He used the \slashf for percolation of binding information, which probably would result in
conflicts with true non-local dependencies. A possible way to fix this seems to be the introduction of
special nonlocal features for binding as was suggested by \citet{Hellan2005a}. Bredenkamp relies on the use of the \subjf for binding
subject-related pronouns, but the \subjf is not used for finite verbs in German. Usually all
arguments of finite verbs are selected via one list: \comps (see \crossrefchapteralp[Section~\ref{sec-svo-sov}]{order} for details). Furthermore the
subject of non-finite verbs in control\is{control} constructions is usually not realized and hence there is no
subj-head schema that could take care of binding off the respective nonlocal dependency. Bredenkamp
did not formulate disjointness constraints, but \citet{Hellan2005a} developed a way to do this. Like
Bredenkamp's approach Hellan's approach has problems with binding by implicit subjects in control
constructions since his schemata for combining the subject with a VP containing a reflexive are not
applicable because of the lack of an overt subject (see also \crossrefchapteralp{control-raising} on
the treatment of control in HPSG). The only way out of this dilemma seems to be to assume a unary
branching projection discharging the pronoun information and binding it to the unexpressed
subject. This entails that pronoun binding has a reflex in syntactic structure (an additional unary
projection\is{projection!unary}), in our opinion an unwanted consequence of this proposal. 

Note that percolating binding information seems to be the only way
to account for binding data for HPSG variants assuming that linguistic objects do not have internal
structure, \eg \sbcg. See \crossrefchaptert[Section~\ref{sec-sbcg}]{cxg} for discussion.

A variant of Binding Theory that uses the HPSG mechanism for nonlocal dependencies in an innovative
way is discussed in Section~\ref{sec-bt-nonlocal}.%
} The places of occurrence of gaps (if they are assumed at all)
is restricted by other components of the theory. For an overview of the treatment of nonlocal
dependencies in HPSG see \crossrefchapterw{udc}.

We will not go into the details of the Binding Theory in Mainstream Generative Grammar
(MGG)\footnote{
We follow \citet[\page 3]{CJ2005a} in using the term \emph{Mainstream Generative Grammar} when
referring to work in Government \& Binding \citep{Chomsky81a} or Minimalism \citep{Chomsky95a-u}.}, but we
give a verbatim description of the ABC of Binding Theory (ignoring movement). Chomsky distinguishes between
so-called R-expressions (referential expressions like proper nouns or full NPs/DPs), personal
pronouns and reflexives and reciprocals. The latter two are subsumed under the term \emph{anaphor}. 
Principle A says that an anaphor must be bound in a certain local domain. Principle B says that a
pronoun must not be bound in a certain local domain and Principle C says that a referential
expression must not be bound by another item at all.

Some researchers questioned whether syntactic principles like Chomsky's Principle C and the
respective HPSG variant should be formulated at all and it was suggested to leave an account of the
unavailability of bindings like the binding of \emph{he} to full NPs in
(\ref{ex-he-thinks-that-Peter}) to \isi{pragmatics} (\citealp[\page 302]{Bolinger79a-u}; \citealp[\page 227--228]{Bresnan2001a};
\citealp*[\page 44]{BMS2001a}). \citet[Section~6]{Walker2011a} discussed the claims in detail
and showed why Principle~C is needed and how data that was considered problematic for syntactic
binding theories can be explained in a configurational binding theory in HPSG. Hence the following
discussion includes a discussion of Principle~C in its various variants.

\section{A non-configural Binding Theory}

As was noted above, English pronouns and reflexives have to agree with their antecedents in
gender. In addition there is agreement in person and number. This is modeled by assuming that
referential units come with a referential index in their semantic representation. The following makeup for the semantic contribution of nominal objects is assumed:
\eas
Representation of semantic information contributed by nominal objects adapted from \citet[\page
  248]{ps2}:\\
\onems[nom-obj]{
  index  \ms[index]{
          per & per\\
          num & num\\
          gen & gen\\
          }\\
  restrictions \type{set of restrictions}\\
}
\zs
Every nominal object comes with a referential index with person, number and gender information and a
set of restrictions. In the case of pronouns, the set of restrictions is the empty set, but for
nouns like \emph{house}, the set of restrictions would contain something like \relation{house}$(x)$
where $x$ is the referential index of the noun \emph{house}. Nominal objects can be of various
types. The types are ordered hierarchically in the inheritance hierarchy given in Figure~\ref{bt-fig-hierarchy-nominal-types}.
\begin{figure}
\centering
\begin{forest}
typehierarchy
[nom-obj
  [pron
    [ana
      [refl]
      [recp]]
    [ppro]]
  [npro]]
\end{forest}
\caption{Type hierarchy of nominal objects}\label{bt-fig-hierarchy-nominal-types}
\end{figure}
Nominal objects (\type{nom-obj}) can either be pronouns (\type{pron}) or non-pronouns
(\type{npro}). Pronouns can be anaphors (\type{ana}) or personal pronouns (\type{ppro}) and anaphors
are devided into reflexives (\type{refl}) and reciprocals (\type{recp}).


HPSG's Binding Theory differs from GB's Binding Theory in referring less to tree structures but
rather to the notion of obliqueness of arguments of a head. The arguments of a head are represented
in a list called the argument structure list. The list is the value of the feature \argst. The
\argst elements are descriptions of arguments of a head containing syntactic and semantic properties
of the selected arguments but not their daughters. So they are not complete signs but \type{synsem}
objects. See \crossrefchaptert{properties} for more on the general setup of HPSG
theories. The list elements are ordered with respect to their obliqueness, the least oblique element
being the first element:\footnote{
  While \citet[\page 120]{ps} use \citegen[]{KC77a} version of the Obliqueness Hierarchy in (i), they avoid the
  terms \emph{direct object} and \emph{indirect object} in \citew[\page 266, 280]{PS92a} and \citew{ps2}.
\ea
\label{def-obliqueness-hierarchy-ps87}
\oneline{%
\is{object!indirect}\is{object!direct}\is{subject}%
\begin{tabular}[t]{@{}l@{\hspace{1ex}}l@{\hspace{1ex}}l@{\hspace{1ex}}l@{\hspace{1ex}}l@{\hspace{1ex}}l@{}}
SUBJECT $>$ & DIRECT $>$ & INDIRECT $>$ & OBLIQUES $>$ & GENITIVES $>$  & OBJECTS OF\\
             & OBJECT      & OBJECT        &               &                 & COMPARISON 
\end{tabular}%
}
\zlast
}
\ea
\label{def-obliqueness-hierarchy}
%\oneline{%
\is{object!second}\is{object!primary}\is{subject}%
\begin{tabular}[t]{@{}l@{\hspace{1ex}}l@{\hspace{1ex}}l@{\hspace{1ex}}l@{}}
SUBJECT $>$ & PRIMARY $>$ & SECONDARY    $>$ & OTHER COMPLEMENTS\\
             & OBJECT      & OBJECT        &\\
\end{tabular}%
%}
\z
This order was suggested by \citet{KC77a}. It corresponds to the level of syntactic activity of grammatical functions\is{grammatical function}. Elements
higher in this hierarchy are less oblique and can participate more easily in syntactic constructions, like for instance,
reductions in coordinated structures\is{coordination} \citep[\page 15]{Klein85},
topic drop\is{topic drop}\is{Vorfeldellipse@{\it Vorfeldellipse}} \citep{Fries88b},
non-matching free relative clauses\is{relative clause!free} 
\parencites[Section~3]{Bausewein90}[\page 195]{Pittner95b}[\page 60--62]{Mueller99b}, 
passive\is{passive} and relativization\is{relativization} \citep{KC77a}, and
depictive predicates\is{predicate!depictive secondary} \citep[Section~2]{Mueller2008a}.
In addition, \citet{Pullum77a} argued that this hierarchy plays a role in constituent order\is{scrambling}\is{serialization} (but see Section~\ref{sec-argst-order}.)
And, of course, it was claimed to play an important role in Binding Theory\is{Binding Theory} 
(Grewendorf, \citeyear[\page 176]{Grewendorf83a}; \citeyear[\page 160]{Grewendorf85a}; \citeyear[\page 60]{Grewendorf88a}; \citealp[Chapter~6]{ps2}).

The \argstl plays an important role for linking syntax to semantics. For example, the index of the
subject and the object of the verb \emph{like} are linked to the respective semantic roles in the
representation of the verb:\footnote{%
  NP\ind{1} is an abbreviation for a feature description of a nominal phrase with the index \ibox{1}
  (see \crossrefchaptert{properties}). The feature description in (\mex{1}) is also an
  abbreviation. Path information leading to \cont is omitted since it is irrelevant for the present discussion.%
}
\eas
\emph{like}:\\
\ms{
arg-st & \sliste{ NP\ind{1}, NP\ind{2} }\\[2mm]
cont & \ms{ ind  & event\\
            rels & \liste{ \ms[like]{
                             actor & \ibox{1}\\
                             undergoer & \ibox{2}\\
                           } }\\
          }
}
\zs 
A lot more can be said about linking in HPSG and the interested reader is referred to
\crossrefchaptert{arg-st} for this.

After these introductory remarks, we can now turn to the details of HPSG's Binding Theory:
Figure~\ref{fig-binding-argst} shows a version of Figure~\ref{fig-binding-gb} including \argst
information.
\begin{figure}
\begin{forest}
sm edges without translation
[S
  [\ibox{1} NP$_i$ [John\\John\\he]]
  [VP
    [V \sliste{ \ibox{1} NP$_i$, \ibox{2} CP } [thinks\\thinks\\thinks]]
    [\ibox{2} CP 
      [C [that\\that\\that]]
      [S
        [\ibox{3} NP$_j$ [Paul\\Paul\\Mary]]
        [VP
         [V \sliste{ \ibox{3} NP$_j$, \ibox{4} NP$_k$ } [likes\\likes\\likes]]
         [\ibox{4} NP$_k$ [him\\himself\\Peter]]]]]]]
\end{forest}

\caption{\label{fig-binding-argst}Tree configuration of examples for binding with \argst lists}
\end{figure}
The main points of HPSG's Binding Theory can be discussed with respect to this simple figure:
anaphors have to be bound locally. The definition of the domain of locality is rather simple. One
does not have to refer to tree configurations, since all arguments of a head are represented locally
in a list. Simplifying a bit, reflexives and reciprocals must be bound to elements preceding them in
the \argstl and a pronoun like \emph{him} must not be bound by a preceding element in the same \argstl.

To be able to specify the conditions on binding of anaphors, pronouns and non-pronouns some further
definitions are necessary. The following definitions are definitions of local \emph{o-command}, \emph{o-command}
and \emph{o-bind}. The terms are reminiscent of \emph{c-command} and so on but we have an ``o''
rather than a ``c'' here, which is supposed to indicate the important role of the obliqueness
hierarchy. The definitions are as follows:

\ea
\label{def-local-o-command-initial-version}
Let Y and Z be \type{synsem} objects with distinct \localvs, Y referential. Then Y \emph{locally
o-commands}\is{o-command!local} Z just in case Y is less oblique than Z.
\z

\ea
\label{def-o-command}
Let Y and Z be \type{synsem} objects with distinct \localvs, Y referential. Then Y \emph{o-commands}\is{o-command} Z just
in case Y locally o-commands X dominating Z.
\z

\ea
\label{def-o-bind}
Y (\emph{locally}) \emph{o-binds}\is{o-bind} Z just in case Y and Z are coindexed and Y (locally) o"=commands Z. If Z
is not (locally) o-bound, then it is said to be (\emph{locally}) \emph{o"=free}\is{o-free}.
\z

\noindent
(\ref{def-local-o-command-initial-version}) says that an \argst element locally o-commands any other \argst element
further to the right of it. The condition of non-identity of the two elements under consideration in
(\ref{def-local-o-command-initial-version}) and (\ref{def-o-command}) is necessary to deal with cases of \isi{raising}, in
which one element may appear in various different \argstls. See Section~\ref{sec-binding-raising} below
and \crossrefchaptert{control-raising} for discussion of raising in HPSG. The condition that Y has
to be referential excludes expletive pronouns\is{pronoun!expletive} like \emph{it} in \emph{it rains} from entering
o-command relations. Such expletives are part of \argst and valence lists, but they are entirely
irrelevant for Binding Theory, which is the reason for their exclusion in the
definition. \citet[\page 258]{ps2} discuss the following examples going back to observations by
\citet{FH83a-u} and \citet[\page 95]{Kuno87a-u}:\todostefan{check pages}

\eal
\ex They$_i$ made sure that it was clear to each other$_i$ that this need to be done.
\ex They$_i$ made sure that it wouldn't bother each other$_i$ to invite their respective friends to dinner.
\zl
According to \citet[Section~3.6]{ps2}, the \emph{it} is an expletive. They assume that extrapositions with
\emph{it} are accounted for by a lexical rule that introduces an expletive and a \emph{that} clause
or an infinitival verb phrase into the valence list of the respective predicates. Since the
\emph{it} is not referential it is not a possible antecedent for the anaphors in sentences like (\mex{0}) and hence
a Binding Theory build on top of the definitions in (\ref{def-local-o-command-initial-version}) and (\ref{def-o-command}) will make the right predictions.

The definition of o-command uses the relations of locally o-command and dominate. With respect to
Figure~\ref{fig-binding-argst}, we can say that NP$_i$ o-commands all nodes below the CP node since
NP$_i$ locally o-commands the CP and the CP node dominates everything below it. So NP$_i$ o-commands
C, NP$_j$, VP, V, and NP$_k$.

The definition of \emph{o-bind} in (\ref{def-o-bind}) says that two elements have to be coindexed
and there has to be a (local) o-command relation between them. The indices include person, number
and gender information (in English), so that \emph{Mary} can bind \emph{herself} but not
\emph{themselves} or \emph{himself}. With these definitions, the binding principles can now be stated
as follows:

\begin{principle-break}[HPSG Binding Theory]
\begin{description}
\item [\isi{Principle A}] A locally o-commanded anaphor must be locally o-bound.
\item [\isi{Principle B}] A personal pronoun must be locally o-free.
\item [\isi{Principle C}] A nonpronoun must be o-free.
\end{description}
\end{principle-break}

\noindent
Principle A accounts for the ungrammaticality of sentences like (\mex{1}):
\ea[*]{
Mary likes himself.
}
\z
Since both \emph{Mary} and \emph{himself} are members of the \argstl of \emph{likes}, there is an NP
that locally o-commands \emph{himself}. Therefore there should be a local o-binder. But since the
indices are incompatible because of incompatible gender values, \emph{Mary} cannot o-bind
\emph{himself}, \emph{himslef} is locally o-free and hence in conflict to Principle A.

Similarly, the binding in (\mex{1}) is excluded, since \emph{Mary} locally o-binds the pronoun \emph{her}
and hence Principle B is violated.
\ea[]{
Mary$_i$ likes her$_{*i}$.
}
\z

\noindent
Finally, Principle C accounts for the ungrammaticality of (\mex{1}):
\ea
He$_i$ thinks that Mary likes Peter$_{*i}$.
\z
Since \emph{he} and \emph{Peter} are coindexed and since \emph{he} o-commands \emph{Peter},
\emph{he} also o-binds \emph{Peter}. According to Principle C, this is forbidden and hence bindings
like the one in (\mex{0}) are ruled out.

For ditransitives, there are three elements on the \argstl: the subject, the primary object and the
secondary object. If the secondary object is a reflexive, Principle~A requires this reflexive to be
coindexed with either the primary object or the subject. Hence, the bindings in (\mex{1}) are
predicted to be possible and the ones in (\mex{2}) are out:
\eal
\ex[]{
John showed Mary$_i$ herself$_i$.
}
\ex[]{
John$_i$ showed Mary himself$_i$.
}
\zl
\eal
\ex[*]{
John showed herself Mary.
}
\ex[*]{
I showed you herself.
}
\zl
Note that configuration-based Binding Theories like the one entertained in GB and Minimalism require
the objects to asymmetrically c-command each other, that is, the primary object c-commands the secondary object but
not vice versa. This results in theories that have to assume certain branchings and in some cases
even auxiliary nodes \citep[Section~4.4]{Adger2003a}. In HPSG the branching that is assumed does not depend on binding facts and
indeed, ternary branching VPs \citep[\page 40]{ps2} and binary branching VPs have been assumed (see
\crossrefchapteralp[Section~\ref{sec-binary-flat}]{order} for discussion).

The list-based Binding Theory outlined above seems very simple. So far we explained binding relations between
coarguments of a head where the coarguments are NPs or pronouns. But there are also prepositional
objects, which have an internal structure with the referential NPs embedded within a
PP. \citet[\page 246, 255]{ps2} discuss examples like (\mex{1}): 
\eal
\ex{
John$_i$ depends [on him$_{*i}$].
}
\ex{
\label{ex-mary-talked-to-john-about-himself}
Mary talked [to John$_i$] [about himself$_i$].
}
\zl
As noted by \citet[\page 246]{ps2}, the second example is a problem for the GB Binding Theory since
\emph{John} is inside the PP and does not c-command \emph{himself}.
\inlinetodostefan{Bob: It is not obvious for GB why this is not like the following:
Mary showed John's picture to himself.}
\begin{figure}
\begin{forest}
sm edges without translation
[S
  [\ibox{1} NP [Mary]]
  [VP
    [V \sliste{ \ibox{1}, \ibox{2}, \ibox{3} } [talked]]
    [\ibox{2} PP$_i$
       [P [to]]
       [NP$_i$ [John]]]
    [\ibox{3} PP$_i$
       [P [about]]
       [NP$_i$ [himself]]]]]
\end{forest}
\caption{Binding within prepositional objects poses a challenge for GB's Binding Theory}
\end{figure}
Examples involving case-marking
prepositions are no problem for HPSG however, since it is assumed that the semantic content of
propositions is identified with the semantic content of the NP they are selecting. Hence, the PP
\emph{to John} has the same referential index as the NP \emph{John} and the PP \emph{about himself}
has the same index as \emph{himself}. The \argstl of \emph{talked} is shown in (\mex{1}):
\ea
\sliste{ NP, PP, PP }
\z
The Binding Theory applies as it would apply to ditransitive verbs. Since the first PP is less
oblique than the second one, it can bind an anaphor in the second one. The same is true for the
example in (\mex{-1}a): since the subject is less oblique than the PP object it locally o-commands
it and even though the pronoun \emph{him} is embedded in a PP and not a direct argument of the verb
the pronoun cannot be bound by \emph{him}. An anaphor would be possible within the PP object though.
Of course the subject NP can bind NPs within both PPs: both \emph{to herself} and \emph{about
  herself} would be possible as well.


\section{Reconstruction}

Examples like (\mex{1}) are covered by HPSG's Binding Theory since \emph{himself} is fronted via
HPSG's nonlocal mechanism (see \crossrefchapteralp{udc}) and there is a connection between the
fronted element and the missing object.
\eal
\ex Himself$_i$, Trump$_i$ really admires \trace.
\ex \sliste{ \textnormal{Trump, himself} }
\zl
Therefore, the \synsemv of \emph{himself} is identified with the object in the \argstl of
\emph{admires} and since the object is local to the subject of admire, it has to be bound by the
subject. But there is more to say about reconstruction and nonlocal dependencies in HPSG:
\citet[\page 265]{ps2} point out an interesting consequence of the treatment of nonlocal
dependencies in HPSG: since nonlocal dependencies are introduced by traces that are lexical elements
rather then by deriving one structure from another one as is common in Transformational Grammar,
there is no way to reconstruct some phrase with all its internal structure into the position of the
trace. Since traces do not have daughters, $\__j$ in (\mex{1}a) has the same local properties (part
of speech, case, referential index) as \emph{which of Claire's$_i$ friends} without having its internal structure.\footnote{
  Some of the more recent theories of nonlocal dependencies even do without traces. See
  \crossrefchaptert{udc} for details.
}
\eal
\label{ex-which-of-clairs-friends}
\ex I wonder [which of Claire's$_i$ friends]$_j$ [we should let her$_i$ invite $\__j$ to the party]?
\ex {}[Which picture of herself$_i$]$_j$ does Mary$_i$ think John likes $\__j$?
\zl
Since extracted elements are not reconstructed into the position where they would be usually
located, (\mex{0}a) is not related to (\mex{1}):
\inlinetodostefan{Bob: Is the important point that filler and gap only share a local object and not a synsem object so that there is no sense in which the filler occupies the position of the gap?
Presumably things are different in SCBG in which a sign is shared in an unbounded dependency.}
\ea
We should let her$_i$ invite which of Claire's$_i$ friends to the party.
\z
\emph{Claire} would be o-bound by \emph{her} in (\mex{0}) and this would be a violation of Principle~C, but since traces do not have daughters,
no problem arises.

This is an interesting feature of the Binding Theory introduced so far, but as
\citet[Section~20.2]{Mueller99a} pointed out, it makes wrong predictions as far as German and English are
concerned. (\mex{1}) is the English example:
\ea
{}[Karl$_i$'s friend]$_j$, he$_{*i}$ knows $\__j$.
\z
According to the definition of o-command, \emph{he} locally o-commands the object of
\emph{knows}. This object is realized as a trace. Therefore the local properties of \emph{Karl's
  friend} are in relation to \emph{he} but since traces do not have daughters, there is no o-command
relation between \emph{he} and \emph{Karl}, hence \emph{Karl} is o-free and Principle C is not
violated. Hence there is no explanation for the impossibility to bind \emph{Karl} to \emph{he}. In
order to fix this, some notion of reconstruction would have to be introduced in HPSG's BT but then
the account of (\ref{ex-which-of-clairs-friends}) would be lost.



\section{A totally non-configurational binding theory}

The initial definition of o-command contains the notion of domination and hence makes reference to
tree structures. \citet[\page 279]{ps2} pointed out that the binding of \emph{John} by \emph{he} in
(\mex{1}a) is correctly ruled out since \emph{he} o-commands the trace of \emph{John} and hence
Principle~C is violated. But since they follow GPSG in assuming that English has no subject traces
\citep[Chapter~4.4]{ps2}, this account would not work for (\mex{1}b). 
\eal
\ex John$_{*i}$, he$_i$ said you like \trace$_i$.
\ex John$_{*i}$, he$_i$ claimed left.
\zl
Later work in HPSG abolished traces alltogether (\citealp*{BMS2001a}; \crossrefchapteralp{udc} but
see Müller \citeyear{Mueller2004e}; \citeyear[Chapter~19]{MuellerGT-Eng1} on empty elements in general) and hence BT cannot rely on
dominance any longer. This section deals with the revised version of Binding Theory not making
reference to dominance. I will first discuss the revision of local o-command involving one
additional aspect having to do with Control Theory and then the revision of o-command without the
notion of dominance.

\subsection{Local o-command including subjects of embedded verbs}

\citet[Section~6.8.3]{ps2} revise the definition of local o-command in a way that
includes the subject of embedded verb phrases in order to combine \isi{control} and binding
theory.\footnote{%
  See also \citet{Branco2007a} for arguments that subjects of controlled verbs should be treated as
  reflexives to be bound in the domain of the control verb.%
} 
\ea
Let Y and Z be \type{synsem} objects with distinct \localvs, Y referential. Then Y locally
o-commands Z just in case either:
\begin{enumerate}[label=\roman*.]
\item Y is less oblique than Z; or
\item Y locally o-commands some X that subcategorizes for Z.
\end{enumerate}
\z
The second clause was intended to apply to verbs selecting VPs or predicative phrases. At the time
HPSG's BT was developed, all valents of a head were represented in a list called
\subcatl. \citet[Chapter~9]{ps2} changed this and introduced the \spr feature for specifiers in NPs,
the \subjf for the representation of subjects and the \compsf for complements. The intention behind
the second clause of the definition is to include the subject of an embedded verb into the local
domain of control verbs. So the natural reformulation of (\mex{0}) is (\mex{1}):
\ea
\label{def-local-o-command}
Let Y and Z be \type{synsem} objects with distinct \localvs, Y referential. Then Y locally
o-commands Z just in case either:
\begin{enumerate}[label=\roman*.]
\item Y is less oblique than Z; or
\item Y locally o-commands some X having Z as its \subjv.
\end{enumerate}
\z
\citet[\page 303]{ps2} assume the following valence for a subject control verb like \emph{promise}:
\ea
\sliste{ NP$_i$(, NP), VP[\subj \sliste{ NP:\type{refl}$_i$ } ] }
\z
According to the definition of local o-command, the subject NP of the embedded verb is local to the
subject NP and the object NP of the matrix verb and because of Principle~A it has to be bound to the
subject. Enforcing this binding is strictly speaking unnecessary since \citet[Chapter~7]{ps2} developed a
Control Theory taking care of this. But this type of organization has a nice side effect: it
explains \isi{Visser's Generalization} according to which subject control verbs do not
passivize. Since passive suppresses the first argument (bearing structural case) and turning it into
an optional \emph{by} phrase one would get the following representation:
\ea
\sliste{ NP$_i$, VP[\subj \sliste{ NP:\type{refl}$_i$ } ], PP[\type{by}]$_i$ }
\z
The coindexing between the downstairs subject and the \emph{by}-PP is due to Control Theory and the
coindexing between the NP$_i$ and the downstairs subject is enforced by Principle~A of the BT. The
result is that three items are coindexed in (\mex{0}), which leads to all sorts of binding conflicts
and hence examples like (\mex{1}) are ruled out \citep[\page 305]{ps2}:
\eal
\judgewidth{?*}
\ex[*]{
Kim$_i$ was promised to leave by Sandy$_i$/Kim$_i$.
}
\ex[*]{
John$_i$ was promised to leave by him$_i$.
}
\ex[?*]{
John$_i$ was promised to leave by himslef$_i$.
}
\zl

Interestingly the situation seems to be parallel in German. Sentences like the following are rather
strange:
\ea
\gll Klaus        wurde versprochen, früher abzufahren.\\
     Klaus.\dat{} was   promised     earlier to.leave\\
\glt Intended: `Somebody promised Klaus to leave early.'
\z 
The sentence is bad if the meaning is that somebody promises that he or she will leave early. It
improves if Klaus will leave early together with the one who made the promise. And, as \citet[\page 129]{Mueller2002b}
showed passivization of subject control verbs is possible in German:
\eal
% Es sei aber von beiden Seiten für nächstes Jahr versprochen worden, den Kinderfasching im RNZ ganz neu und vielleicht auch schöner
% und stilvoller zu feiern. 
% M99/902.07778 Mannheimer Morgen, 05.02.1999, Lokales; Kinderfasching im nächsten Jahr
\ex 
\gll Wie oft schon wurde von der Stadtverwaltung versprochen, Abhilfe zu schaffen.\footnotemark\\
     like often yet was  by the council          promised     remedy  to manage\\
\footnotetext{
        Mannheimer Morgen, 13.07.1999, Leserbriefe; Keine Abhilfe.%%M99/907.45513 Mannheimer Morgen, 13.07.1999, Leserbriefe; Keine Abhilfe
}
%Selbst gebastelte Übersetzung
\glt `As often, the council promised to resolve the matter.'
\ex Erneut wird versprochen, das auf eine Dekade angesetzte Investitionsprogramm mit einem Volumen von 630 Billionen Yen (10,5
Billionen DM) vorfristig zu erfüllen, [\ldots]\footnote{
    Süddeutsche Zeitung, 28.06.1995, p.\,28.% %U95/JUN.42005 Süddeutsche Zeitung, 028.06.1995, S. 28, Ressort: WIRTSCHAFT; Japans flaues Konjunkturprogramm
}
%Selbst gebastelte Übersetzung
\glt `Again, one promised to complete the investment program planned for one decade with the total amount 
of 630 trillion Yen before the agreed date.'
%wofür es allerdings noch sehr wenig Anzeichen gibt. 
%
% Nun sind solche Reformversprechen fast so alt wie die Stiftung Pro Helvetia selbst, geradezu rituell wird bei passender Gelegenheit
% versprochen, die unübersichtlichen Strukturen zu verschlanken, Leerläufe zu vermeiden, die Zusammenarbeit des Bundes mit anderen
% Förderern zu intensivieren und, wie sich die bundesrätliche Botschaft sehr höflich ausdrückt, "die langsamen und komplizierten
% Entscheidungswege von Pro Helvetia" zu straffen. 
% E99/MAI.12678 Züricher Tagesanzeiger, 14.05.1999, S. 5, Ressort: Schweiz; Mehr Geld, weniger Umstände
\zl
All of these examples are fine with \emph{uns} `us' as the object of \emph{versprechen} `to promise'
as well. So, if one assumes that a central phenomenon like control is handled in parallel ways in English
and German, the controlled subject should not be treated as a reflexive local to the arguments of
control verbs.

%% A further problem for including the downstairs subject into the local o"=command relation is posed by
%% subject control verbs with an object. If one assumes that the subject of the embedded verb is a
%% reflexive that is local to the arguments of the control verb, a reflexive object of the control verb
%% causes chaos: 
%% \ea
%% Peter$_i$ promised himself$_i$ PRO$_i$ to eat an ice cream after the work out.
%% \z
%% The reflexive 

A further problem for including the downstairs subject into the local o"=command relation is posed by
raising-to-subject verbs with an object. \emph{believe} is an example:
\eal
\ex Donald$_i$ believes himself$_i$ to be a liar.
\ex \sliste{ \ibox{1} NP$_i$, NP$_i$, VP[ \subj \sliste{ \ibox{1} } } 
\zl
The \argstl of \emph{believe} contains the subject NP (\emph{Donald}), the NP of the reflexive
(\emph{himself}) and the VP (\emph{to be a liar}). Since the
subject of the VP is regarded as local to \emph{Donald} and \emph{himself}, we have a situation in
which \emph{Donald} locally o"=commands \emph{himself} and \emph{himself} o"=commands
\emph{Donald} (in the \subjl of the embedded VP). Hence the second \emph{Donald} is bound by \emph{himself}, which is a violation of
Principle~C. 
Note that \citet[\page 253]{ps2} took measures to deal with raising by excluding two
elements with identical \localvs in the definition of o-command. So the two Donald's would not cause problems, but the example in
(\mex{0}a) involves one intermediate NP, namely \emph{himself}. The definition of local o-command
could be fixed by requiring that Z is not raised. This requirement is more general than the
requirement to have a \localv different from the one of Y since it includes the case in which Z is
raised and different from Y.\footnote{%
  Note that the general statement that something is not raised is much more involved than requiring
  that the \localvs of two elements are distinct, since the requirement that something is not raised
  requires a check of all other elements on the higher \argstl for non-identity. Some approaches to
  valence mark the status of arguments with respect to raising explicitly by assuming a boolean
  feature \feat{raised} \citep{Prze99,Meurers99b}. This would simplify things considerably.%
} But since the inclusion of the downstairs subject would
make wrong predictions for German anyway and its merits for English short passives are unclear
\citep[\page 306]{ps2}, I suggest leaving the definition of local o-command the
way it originally was, that is, without the reference to subjects of embedded verbs. 

\subsection{o-command}

The revised non-configurational variant of o-command suggested by \citet[\page 279]{ps2} has the
form in (\mex{1}):\footnote{%
  I replaced subcategorized by reference to the \argstl.%
}
\ea
\label{def-non-configurational-o-command}
Let Y and Z be \type{synsem} objects with distinct \localvs, Y referential. Then Y o-commands Z just in case either:
\begin{enumerate}[label=\roman*.]
\item Y is less oblique than Z; or
\item Y o-commands some X that has Z on its \argstl; or
\item Y o-commands some X that is a projection of Z (\ie the \headvs of X and Z are token-identical).
\end{enumerate}
\z
This definition of course has the problem pointed out in the previous subsection. 
\inlinetodostefan{
Check whether there are examples in which one needs both an upstairs and a downstairs \argst.
}
%While it would have been sufficient to require that Z is not raised in the definition of local o-command this seems
%to be too strong in the general definition of o-command.
\begin{figure}
\begin{forest}
sm edges without translation
[A
  [B [John]]
  [C{[\head \ibox{1}]}
    [D\feattab{\head \ibox{1},\\\argst \sliste{ B, E }} [thinks]]
    [E{[\head \ibox{2}]}
      [F\feattab{\head \ibox{2},\\\argst \sliste{ G }} [that]]
      [G{[\head \ibox{3}]}
        [H [Peter]]
        [I{[\head \ibox{3}]} 
          [J\feattab{\head \ibox{3},\\ \argst \sliste{ H, K }} [likes]]
          [K [him]]]]]]]
\end{forest}
\caption{Tree for explanation of the o-command relation}
\end{figure}

According to the definition, B o-commands E by clause i. B o-commands F, since it o-commands E and E
is a projection of F (clause iii). B also o-commands G, since B o-commands F and F has G on its
\argstl (clause ii). Since B o-commands G, it also o-commands J, since G is a projection of J
(clause iii). And because of all this B also o-commands H and K, since B o-commands J and both H and
K are members of the \argstl of J (clause ii). 

This recursive definition of o-command is really impressive and it can account for binding phenomena
in approaches that do not have empty nodes for traces in the tree structures, but there are still open
issues.\footnote{%
Note that the label \emph{totally non-configurational Binding Theory} seems to suggest that
dominance relations do not play a role at all and hence this version of BT could be appropriate for
HPSG flavors like \sbcg that do not have daughters in linguistic signs (see \citew{Sag2012a} and
\crossrefchaptert[Section~\ref{sec-sbcg}]{cxg} for discussion). But this is not the case. The definition of o-command in
(\ref{def-non-configurational-o-command}) contains the notion of projection. While this notion can
be formalized with respect to a complex linguistic sign having daughters in Constructional HPSG as
assumed in this volume, this is impossible in SBCG and one would have to refer to the derivation
tree, which is something external to the linguistic signs licensed by a SBCG theory. See also footnote~\ref{binding-fn-percolation-of-indices}.%
}

As was pointed out by \citet[\page 490]{HL96a}, \citet[Section~20.4.1]{Mueller99a} and \citet{Walker2011a}, adjuncts pose a challenge for the
non-configurational Binding Theory. For example, a referential NP can be part of an adjunct and
since adjuncts are usually not part of \argstls they would not be covered by the definition of
o-command given above. \emph{John} is part of the reduced relative clause modifying \emph{woman} in
(\mex{1}).
\ea
He$_{*i}$ knows the woman loved by John$_i$.
\z
Since the relative clause does not appear on any \argstl, \emph{he} does not o-command \emph{John}
and hence there is no Principle~C violation and the binding should be fine.

Several authors suggested including adjuncts into \argstls of verbs
% Adam: at least some adjuncts put on Arg-ST. Could also be DEPS if case is assigned on DEPS
(\citealp[\page 168]{Chung98}; \citealp[\page 240]{Prze99}; \citealp*[\page 60]{MSI99a}), but this would
result in conflicts with BT if applied to the nominal domain \citep[Section~20.4.1.]{Mueller99a}. The reason is that nominal modifiers have a semantic contribution that
contains an index that is identical to the index of the modified noun.\footnote{%
See \crossrefchapterw[Section~\ref{sec:rc-pollard--sag}]{relative-clauses} and
\citew{Mueller99b} on relative clauses. \citet{Sag97a} suggests an approach to relative clauses in
which a special schema is assumed that combines the modified noun with a verbal projection. This
approach does not have the problem mentioned here. However, prenominal adjuncts would remain
problematic as \citegen{Mueller99a} example in (i) shows:
\ea
\gll Er$_{*i}$ kennt die Karl$_i$ betrügende Frau.\\
     he knows the Karl betraying woman\\
\glt `He knows the woman betraying Karl.'
\z
The adjectival participle behaves like a normal adjectival modifier. For Principle~C to make the
right predictions, there should be a command relation between \emph{er} and the parts of the
prenominal modifier. PP adjuncts within nominal structures are a further instance of problematic examples.%
} If there are several such modifiers, we get a conflict since we have several coindexed non-pronominal indices on the same
\argstl, which would violate Principle~C. 

There are two possible solutions that come to mind. The first one is pretty ad
hoc: one can assume two different features for different purposes. There could be the normal index
for establishing coindexation between heads and adjuncts and heads and arguments and there could be
a further index for binding. Adjectives would then have a referential index for establishing
coindexation with nouns and an additional index that is a state, which would be irrelevant for the
binding principles.
% Way out: stipulate two sets of indices one for binding one for linking the semantics of
% heads/adjuncts/arguments ...

The second solution to the adjunct problem might be seen in defining o"=command with respect to the \depsl. The \depsl is a list
of dependents that is the concatenation of the \argstl and a list of adjuncts that are introduced on
this list \citep*{BMS2001a}. Binding would be specified with respect to \argst and dominance with
respect to \deps (which includes everything on \argst). The lexical introduction of adjuncts has
been criticized because of scope issues by \citet{LH2006a} and there are also problems related to
binding: \citet[\page 490]{HL96a} pointed out that there are differences when it comes to the
interpretation of pronouns in examples like (\mex{1}a,b) and (\mex{1}c,d):
\eal
\ex They$_i$ went into the city without anyone noticing the twins$_{*i/j}$.
\ex They$_i$ went into the city without the twins$_{*i/j}$ being noticed.
\ex You can't say anything to them$_i$ without the twins$_{i/j}$ being offended.
\ex You can't say anything about them, without Terry criticizing the twins$_{i/j}$ mercilessly.
\zl
While the subject pronoun cannot be coreferential with \emph{the twins} inside the adjunct, the object pronoun in
(\mex{0}c,d) can. If we just register adjuncts on the \depsl, we are unable to refer to their
position in the tree and hence we cannot express any statement needed to cover the differences in
(\mex{0}). Note that this is crucially different for elements on the \argstl in English, since the \argst of a lexical item
basically determines the trees it can appear in in English: the first element appears to the left of
the verb as the subject and all other elements to the right of the verb as complements. However,
this is just an artifact of the rather strict syntactic system of English, this is not the case for
languages with freer constituent order like German, which causes problems to be discussed below (see Section~\ref{bt-sec-obliqueness-and-constituent-order}).

There is another issue related to the totally non-configurational version of the BT: in 1994, HPSG was strictly
head-driven. There were rather few schemata and most of them were headed. Since then more and more
constructional schemata were suggested that do not necessarily have a head. For example, relative
clauses were analyzed involving an empty relativizer (\citealp[Chapter~5]{ps2}; \crossrefchapteralp[Section~\ref{sec:rc-pollard--sag}]{relative-clauses}). One way to eliminate this empty element from
grammars is to assume a headless schema that combines the relative phrase and the clause from which
it is extracted directly \citep[Section~2.7]{Mueller99b}. In addition there were proposals to analyze free
relative clauses\is{relative clause!free} in a way in which the relative phrase is the head
\citep[\page 383]{WK2003a}. So, if \emph{whoever} is the head of \emph{whoever is loved by John}, the whole
relative clause is not a projection of \emph{loved} and hence the arguments of \emph{loved} will not
be found by the definition of o-command in (\ref{def-non-configurational-o-command}). This means that \emph{John} is not o-commanded by
\emph{he}, which predicts that the binding in (\mex{1}) is possible, but it is not.
\ea
He$_{*i}$ knows whoever is loved by John$_i$.
\z
Further examples of phenomena that are treated using unheaded constructions are serial verbs in
Mandarin Chinese: \citet{ML2009a} argue that VPs are combined to form a new complex VP with a
meaning determined by the combination. None of the combined VPs contributes a head. No VP selects
for another VP. 

%% They are unheaded but the CAT values are shared.
%% and coordination
%% \crossrefchapterp[Section~\ref{coord-sec-headedness}]{coordination}. If coordinated structures are
%% unheaded, as often assumed in HPSG analyses, \emph{John} would not be o-commanded by \emph{he} in
%% (\mex{1}):
%% \ea
%% He$_{*i}$ knows that Mary loves John$_i$ and that Sandy loves Kim.
%% \z


I think there is no way of accounting for such cases without the notion of
dominance. For those insisting on grammars without empty elements, the solution would be a fusion of
the definition given in (\ref{def-non-configurational-o-command}) with the initial definition involving
dominance in (\ref{def-o-command}). \citet{HL95b} suggested such a fusion. This is their definition
of vc-command:
\ea
\label{def-vc-command-HL}
v(alence-based) c-command:\\
Let α be an element on a valence list that is the value of the valence feature γ and α' the \dtrs element whose \synsemv is structure-shared
with α. Then if the constituent that would be formed by α' and one or more elements β has a null
list as its value for γ, α vc-commands β and all its descendants.
\z
Rewritten in more understandable prose this definition means that if we have some constituent α'
then its counterpart in the valence list vc-commands all siblings of α' and their decedents
provided the valence list on which α' is selected is empty at the next higher node. We have two
valence lists that are relevant in the verbal domain: \subj (some authors use \spr instead) and
\comps. The \compsl is empty at the VP node and the \subjl ist empty at the S node. So, the
definition in (\mex{0}) makes statements about two nodes in Figure~\ref{fig-vc-command-HL}: the
lower VP node and the S node. 
\begin{figure}
\begin{forest}
sm edges without translation
[S\feattab{\subj \eliste,\\
           \comps \eliste }
  [\ibox{1} NP [they]]
  [VP\feattab{\subj \sliste{ \ibox{1} },\\
              \comps \eliste }
    [VP\feattab{\subj \sliste{ \ibox{1} },\\
                \comps \eliste}
      [V\feattab{ \subj \sliste{ \ibox{1} },\\
                  \comps \sliste{ \ibox{2} },\\
                  \argst \sliste{ \ibox{1}, \ibox{2} }}
         [bought]]
      [\ibox{2} NP
        [the car,roof]]]
    [{PP}
      [without anybody noticing the twins,roof]]]]
\end{forest}
\caption{Example train for explaining vc-command: The subject vc-commands the adjunct because it is
  in the valence list of the upper-most VP and this VP dominates the adjunct PP}\label{fig-vc-command-HL}
\end{figure}
For Figure~\ref{fig-vc-command-HL}, this entails that the object NP \emph{the car} vc-commands
\emph{bought} since \emph{the car} is an immediate daughter of the first projection with empty
\compsl. The NP \emph{they} vc-commands the VP \emph{bought the car without anybody noticing the
  twins}, since both are immediately dominated by the node with the empty \subjl.

The proposal by \citeauthor{HL95b} was criticized by \citet[\page 235]{Walker2011a}, who argued that the modal
component \emph{would be formed} in the definition is not formalizable and suggested the following revision:
\ea
Let α, β, γ be \type{synsem} objects, and β' and γ' signs such that β': [\synsem β] and γ': [\synsem γ]. Then α vc-commands β iff
\begin{enumerate}[label=\roman*.]
\item γ': [ \textsc{ss|loc|cat|subj} \sliste{ α } ] and γ' dominates β', or 
\item α locally o-commands γ and γ' dominates β'.
\end{enumerate}
\z
Principle~C is then revised as follows:
\ea
Principle C: A non-pronominal must neither be bound under o-command nor under a vc-command relation.
\z
Walker uses the tree in Figure~\ref{fig-vc-command} to explain her definition of
vc-command.\todostefan{The annotations should be extra not part of the node. Especially for the PP.}
%\ea
%It was herself that Mary loved.
%\z
\begin{figure}
\begin{forest}
sm edges without translation
[S
  [\ibox{1} NP [they]]
  [{VP[\subj \sliste{ \ibox{1} } ] = γ'}
    [{VP[\subj \sliste{ \ibox{1} = α } ]}
      [V\feattab{ \subj \sliste{ \ibox{1} },\\
                  \comps \sliste{ \ibox{2} },\\
                  \argst \sliste{ \ibox{1}, \ibox{2} }}
         [bought]]
      [\ibox{2} NP
        [the car,roof]]]
    [{PP = β'}
      [without anybody noticing the twins,roof]]]]
\end{forest}
\caption{Example train for explaining vc-command: The subject vc-commands the adjunct because it is
  in the valence list of the upper-most VP and this VP dominates the adjunct PP}\label{fig-vc-command}
\end{figure}
The second clause in the definition of vc-command is the same as before: it is based on local
o-command and domination. What is new is the first clause. Because of this clause the subject
vc-commands the adjunct since the subject \ibox{1} is in the \subjl of the top-most VP (α) and
this top-most VP (γ') dominates the adjunct PP (β′). 

There is an interesting puzzle here as far as the formal foundations of HPSG are concerned
\crossrefchapterp{formal-background}: usually binding theories are defined with respect to some tree
structures. So the structures are assumed to exist and then there are constraints put onto them to
rule out certain bindings. The definition of \citeauthor{HL95b} contains a modal component talking
about structures that would be licensed. Walker criticizes this and formulates a definition that
does without this part. However, by doing so the problem does not go away. If HPSG grammars are seen
as a set of constraints describing models of linguistic objects, there would not be a linguistic
object for *~\emph{Mary likes himself.} and hence one could not say that \emph{Mary} o-commands
\emph{himself}. Hence, there is a problem, whether one names it in the definition or not. It seems
to be necessary to conceptualize binding conditions as something external to the core theory of
HPSG: a filter that is applied on top of everything else as is common in more
implementation-oriented approaches to HPSG in in the generate and test model of GB.

%% Ist nur inefficnet aber nciht schlimm, weil Prinzip C das negiert.
%%
%% The drawback of Walker's definition is that there are two ways something within an object NP can be
%% vc-commanded. The reason is that the second clause applies to to the object \ibox{2} and the first
%% one applies as well since the \subjl of the lower VP matches the description in clause i. as
%% well. This means that there are spurious ambiguities in the analysis of 

There is a further difference between \citeauthor{HL95b}'s and \citeauthor{Walker2011a}'s
definition: the former applies to Specifier-Head structures, in which the singleton element of the
\sprl is saturated. We will return to this in Section~\ref{sec-nominal-heads-as-binders}. Note also
that the definition of \citeauthor{HL95b} includes the sibling VP among the items commanded by the
subject, while Walker's definition includes elements dominated by this VP only.\footnote{%
  The situation is similar to the different versions of c-command in MGG. See footnote~\ref{fn-c-command-GB}.
}
This difference will also matter in Section~\ref{sec-nominal-heads-as-binders}.


\citeauthor{HL95b}'s examples involve a subject-object asymmetry. Interestingly, a similar subject-object asymmetry seems to exist in German, as \citet[\page
  148]{Grewendorf85a} pointed out:
%% \eal
%% \ex[]{
%% \gll In Marias$_i$ Wohnung erwartete sie$_i$ ein Lustmolch.\\
%%      in Maria's    flat     waits     her     a  lecher\\
%% \glt `A lecher waits for Maria in her flat.'
%% }
%% \ex[*]{
%% \gll Ein Lustmolch erwartete sie$_i$ in Marias$_i$ Wohnung.\\
%%      a   lecher    waits     her     in Maria's    flat\\
%% }
%% \zl
%% \eal
%% \ex[*]{
%% \gll In Marias$_i$ Wohnung erwartete sie$_i$ einen Lustmolch.\\
%%      in Maria's    flat    waits     she     a.\acc{} lecher\\
%% \glt Intended: `Maria waits for a lecher in her flat.'
%% }
%% \ex[*]{
%% \gll Sie$_i$ erwartete in Marias$_i$ Wohnung einen Lustmolch.\\
%%      she     waits     in Maria's   flat     a\acc{} lecher\\
%% \glt Intended: `Maria waits for a lecher in her flat.'
%% }
%% \zl
\eal
\ex[]{
\gll In Marias$_i$ Wohnung erwartete sie$_i$ ein Lustmolch.\\
     in Maria's    flat     waits     her.\acc{}    a.\nom{}  lecher\\
\glt `A lecher waits for Maria in her flat.'
}
\ex[*]{
\gll In Marias$_i$ Wohnung erwartete sie$_i$ einen Lustmolch.\\
     in Maria's    flat    waits     she.\nom{}     a.\acc{} lecher\\
\glt Intended: `Maria waits for a lecher in her flat.'
}
\zl
While the fronted adjunct can bind the object in (\mex{0}a), binding the subject in (\mex{0}b) is
ruled out. \citeauthor{Walker2011a}'s proposals for English would not help in such examples, since all arguments of
finite verbs are represented in one valence list in grammars of German. Hence the highest domain in
which vc-command is defined (taking \citeauthor{HL95b}'s definition) is the full clause since \comps
would be empty at this level. There is the additional problem that the adjunct is fronted in a
non-local dependency (German is a V2 language) and that the arguments are scrambled in
(\mex{0}a). There is no VP node in the analysis of (\mex{0}a) that is commonly assumed in HPSG
grammars of German and it is unclear how a reconstruction of the fronted adjunct into a certain
position could help explaining the differences in (\mex{0}).


\subsection{Conclusion}

Concluding this section, it seems that a totally non-configurational Binding Theory seems to be
impossible because of adjuncts and the combination of configurational and non-configurational parts
seems appropriate. The subject of the embedded verb should not be included among the local domain of
VP embedding verbs.

\inlinetodostefan{
check raising issues.

provide final definitions.
}


\section{Exempt anaphors}

\inlinetodostefan{Bob: I think it would be good to mention earlier that unlike GB binding theory HPSG binding theory does not try to deal with all anaphors.

I also think you might have sone more examples of exempt anaphors since this is quite important in Pollard and Sag.}

The statement of Principle A has interesting consequences: if an anaphor is not locally o-commanded,
Principle A does not say anything about requirements for binding. This means that anaphors that are
initial in an \argstl may be bound outside of their local environment. The following example by
\citet[\page 270]{ps2} shows that a reflexive can even be bound to an antecedence outside of the sentence:

\ea
John$_i$ was going to get even with Mary. That picture of himself$_i$
in the paper would really annoy her, as would the other stunts he had planned.\footnote{
        \citep[\page 270]{ps2}
}
\z

A further example are NPs within adjunct PPs. Since there is nothing in the PP \emph{around
  himself} that is less oblique than the reflexive, the principles governing the distribution of
reflexives do not apply and hence both a pronoun and an anaphor is possible:

\eal
\ex John$_i$ wrapped a blanket around him$_i$.
\ex John$_i$ wrapped a blanket around himself$_i$.
\zl

\noindent
Which of the pronouns is used is said to depend on the \emph{point of view} of the speaker
(\citealp{Kuroda65a}, for further discussin and a list of references see \citealp[\page 270]{ps2}).

%% Unfortunately the situation is different in languages like German. The binding of a pronoun to the subject is strictly ungrammatical:
%% \ea
%% John$_i$ wickelte die Decke um ihn$_i$.
%% \z

The exemptness of anaphors seems to cause a problem since the Binding Theory does not rule out sentencens like (\mex{1}):
% Fanselow86a:349 Einander arbeitet.
\ea[*]{
Himself sleeps.
}
\z
This is not a real problem for languages like English, since such sentences are ruled out because \emph{sleeps}
requires an NP in the nominative and \emph{himself} is accusative \parencites[\page 388]{Brame77}[\page 262]{ps2}.
However, as \citet[Section~20.4.6]{Mueller99a} pointed out, German does have subjectless verbs like \emph{dürsten} `be thirsty' and \emph{grauen} `to dread' and here the problem is real:
\eal
\ex[]{
\gll Den        Mann friert.\\
     the.\acc{} man  cold.is\\
\glt `The man is cold.'
}
\ex[*]{
\gll Einander         friert.\footnotemark\\
     eachother.\acc{} cold.is\\
}
\footnotetext{
\citep[\page 349]{Fanselow86a}
}
\ex[]{
\gll Den Mann dürstet.\\
     the.\acc{} man thirsts\\
\glt `The man is thirsty.'
}
\ex[*]{
\label{duersten}
\gll Sich dürstet.\\
     \textsc{self}.\acc{} thirst\\
}
\zl
Note that subjectless verbs usually can be used with an expletive subject:
\eal
\ex[]{
\gll weil es den Mann friert\\
     because \expl{} the.\acc{} man cold.is\\
}
\ex[]{
\gll weil es den Mann dürstet\\
     because \expl{} the.\acc{} man thirsts\\
\glt `because the man is thirsty'
}
\zl
This does not help to explain these examples away since expletives are non-referential and hence
they do not o-command any other item. 

This line of thought leads to English examples that are problematic: the analysis of extraposition
with expletive \emph{it} results in a similar \argstl:
\eal
\ex[]{
It bothers me that Sandy snores.
}
\ex[*]{
It bothers myself that Sandy snores.
}
\zl
According to \citet[\page 149]{ps2} the \emph{it} in (\mex{0}) is non-referential. Hence there is
nothing that o-commands the accusative object and hence anaphors would be exempt in the object
position and sentences like (\mex{0}b) are predicted to be grammatical. 


\section{Nominal heads as binders}
\label{sec-nominal-heads-as-binders}

The definition of o-command has an interesting consequence: it does not say anything about possible
binding relations between heads and their dependents. What is regulated is the binding relations
between co-arguments and referential objects dominated by a more oblique coargument. As
\citet[\page 419]{Mueller99a} pointed out, bindings like the one in (\mex{1}) are not ruled out by
the Binding Theory of \citet[Chapter~6]{ps2}: 
\ea
his$_{*i}$ father$_i$
\z
The possessive pronoun is selected via \spr and hence a dependent of \emph{father}
\citep{MuellerHeadless,MyPM2020a}, but the noun does not appear in any \argstl (assuming an NP
analysis). The consequence is that Principle~B and~C do not apply and the o-command-based Binding Theory just
does not have anything to say about (\mex{0}). This problem can be fixed by assuming \citegen{HL95b}
version of Principle~C together with their definition of vc-command in (\ref{def-vc-command-HL}).
This would also cover cases like (\mex{1}):
\ea
his$_{*i}$ father of John$_i$
\z

What is not accounted for so far is \citegen[\page 344]{Fanselow86a} examples in (\mex{2}):
\eal
\ex[*]{
\gll die Freunde$_i$ voneinander$_i$\\
     the friends     of.each.other\\
}
\ex[]{
der Besitzer$_i$ seines$_{*i}$ Botes\\
the owner        of.his       boat\\
}
\zl
These examples would be covered by an \iwithini\is{i-within-i-Condition@\emph{i}-within"=\emph{i}"=Condition|(} as suggested by
\citet[\page 212]{Chomsky81a}. Chomsky's condition basically rules out configurations like the one
in (\mex{1}):
\ea
( \ldots{} x$_i$ \ldots{} )$_i$
\z
\citet[\page 244]{ps2} discuss the \iwithini in their discussion of GB's Binding Theory but do not
assume anything like this in their papers. Nor was anything of this kind adopted anywhere else in
th discussion of binding. Having such a constraint could be a good solution, but as
\citet[]{Fanselow86a} working in GB pointed out, such a condition would also rule out cases like his examples in (\mex{1}):
\eal
\ex
\gll die sich$_i$ treue Frau$_i$\\
     the \self{} faithful woman\\
\glt `the woman who is faithful to herself'
\ex 
\gll die einander$_i$ verachtenden Männer$_i$\\
     the each.other   despising    men\\
\glt `the men who despise each other'
\zl
German allows for complex prenominal adjectival phrases. The subject of the respective adjectives or
adjectival participles are coindexed with the noun that is modified. Since the reflexive and
reciprocal in (\mex{0}) are coindexed with the non-expressed subject and since this subject is
coindexed with the modified noun \citep[Section~3.2.7]{Mueller2002b}, a general \iwithini cannot be formulated for HPSG
grammars of German. The problem also applies to English, although English does not have complex
prenominal adjectival modifiers. Relative clauses basically produce a similar configuration:
\ea
the woman$_i$ seeing herself$_i$ in the mirror
\z
The non-expressed subject in (\mex{0}) is the antecedent for \emph{herself} and since this element
is coindexed with the antecedent noun of the relative clause, we have a parallel situation.

\citet[\page 229, Fn.\,63]{Chomsky81a} notes that his formulation of the \iwithini rules out relative
clauses and suggests a revision. However, the revised version would not rule out the examples above
either, so it does not seem to be of much help.

So some special constraint seems to be needed that rules out binding by and to the head of nominal
constructions unless this binding is established by adnominal modifiers directly.\is{i-within-i-Condition@\emph{i}-within"=\emph{i}"=Condition|)}

%This failure is similar to the failure in the
%definition of c-command that can be found in \citew[\page 168]{SKS2013a-u}


%% \ea
%% Karl heiratet eine nur sich$_i$ selbst liebende Frau$_i$.
%% \z





\section{A general Binding Theory with reference to obliqueness or language-specific binding conditions}


\subsection{Obliqueness and constituent order}
\label{bt-sec-obliqueness-and-constituent-order}

As was explained above the order of the elements in the \argstl is seen as crucial for the
determination of possible bindings and reflexivization. Anaphors may refer to elements further to
the left on the \argstl. If one assumes a nom, acc, dat order on the \argstl, \citegen[\page 58]{Grewendorf88a}
binding examples in (\ref{bsp-der-arzt-zeigte-den-dem}) are correctly predicted.
\eal
\label{bsp-der-arzt-zeigte-den-dem}
\ex
\label{bsp-der-arzt-zeigte-den} 
\gll Der Arzt   zeigte den        Patienten$_j$ sich$_j$ / ihm$_{*j}$ im Spiegel.\\
     the doctor showed the.\acc{} patient       \self.\dat{}    {} him.\dat{}       in.the mirror\\
\glt `The doctor showed the patient himself in the mirror.'
\ex
\label{bsp-der-arzt-zeigte-dem} 
\gll Der Arzt zeigte dem Patienten$_j$ ihn$_j$ / sich$_{*j}$ im Spiegel.\\
     the doctor showed the.\dat{} patient     him.\acc{}    {} \self.\acc{}       in.the mirror\\
\zl
But, as \citet[\page 184]{Eisenberg86} points out, bindings like those in (\mex{1}) exist as well:
\eal
\label{ex-dat-acc-verbs}
\ex 
\gll Ich empfehle ihm$_j$ sich$_j$.\\
     I   recommend him.\dat{} \self.\acc\\ 
\ex 
\gll Du ersparst ihm$_j$ sich$_j$.\\
     you spare him.\dat{}  \self.\acc\\
\ex 
\gll Du verleidest ihm$_j$ sich$_j$.\\
     you put.off him.\dat{}  \self.\acc\\
\glt `You put him off himself.'
\zl
The examples in (\mex{0}) show that datives may bind accusatives. As (\mex{1}) shows,
\emph{empfehlen} `recommend' allows for passivization, so the accusative object is a direct object in
the sense of the obliqueness hierarchy and should be seen as less oblique than the dative.
\ea
\gll Dieser Stoff wurde ihm empfohlen.\\
     this.\nom{} cloth was him.\dat{} recommended\\
\glt `This cloth was recommended to him.'
\z
It is an open issue how this situation can be resolved. One way is to make binding principles verb
(class) dependent and independent of the obliqueness hierarchy or to assume that it is verb (class)
dependent and that verbs like those in (\ref{ex-dat-acc-verbs}) have a different order of elements
in the \argstl. Of course this could have consequences for other parts of the grammar relying on the
order of elements in the \argstl (see Section~\ref{sec-argst-order} for further discussion).

Note also that (\ref{bsp-der-arzt-zeigte-dem}) causes a Principle C violation. Since accusative
(direct object) is less oblique than dative (indirect object), \emph{ihn} `him' (locally) o-binds \emph{dem
  Patienten} `the patient', which is prohibited by Principle C. This seems to indicate that linear
order plays a role in binding. Since the order on the \argstl determines the constituent order in
English, similar problems do not arise in English. In a configurational Binding Theory involving
movement and c-command, the dative would be higher in the tree and hence c-command the accusative
but due to the analysis of scrambling in HPSG \crossrefchapterp{order} this is not the
case. \citet{Mueller2004b} discusses various alternative analyses of constituent
order. Chapter~\ref{chap-order} of this book presents the one that is commonly assumed: there is a
fixed order of elements of the \argstl and heads may be combined with their arguments in any
order. The alternative would be to assume multiple lexical items with different \argstls, each
corresponding to one possible ordering of the arguments \citep{Uszkoreit86b}. This would fix the problem with
(\ref{bsp-der-arzt-zeigte-dem}) but it would cause new problems since subjects may be ordered after
objects without changing binding behavior (see also \citealp[\page 13]{Riezler95a}):\footnote{%
  This shows that statements of Principle~A incorporating precedence cannot be universally true: ``A
  bound anaphor must be bound by a co-argument that precedes it.'' (\citealp[Section~9]{ArkaWechsler96a-u} for
  \ili{Balinese}, \citealp[\page 44]{AMM2017a-u} for \ili{Moro}).
% AMM2017 cite \citet{Bresnan2001a} with Malayalam, Korean, Balinese, Japanese, and other languages ). 
Rather than incorporating linear precedence into
  Principle~A, this constraint should be stated separately on a language by language basis.%
}
\eal
\ex
\gll dass der Mann sich vorstellt\\
     that the.\nom{} man  \self.\acc{} introduces\\
\glt `that the man introduces himself'
\ex
\gll dass sich der Mann vorstellt\\
     that \self.\acc{} the.\nom{} man  introduces\\
\glt `that the man introduces himself'
\zl
%% If the antecedent is the subject, both orders are possible \citep[\page 13]{Riezler95a}:
%% \eal
%% \ex
%% \gll Sich$_i$    sollte der Hans$_i$ wieder einmal   rasieren.\\
%%      himself should the Hans again  one.time shave\\
%% \glt `Himself, Hans should shave again.'
%% \ex
%% \gll daß sich$_i$ der Hans$_i$ wieder einmal rasieren sollte\\
%%      that himself the Hans again one.time shave should\\
%% \glt `that Hans1 should shave himself1 again'
%% \zl
The reflexive has to be bound to the subject independent of the relative order of subject and
accusative object. If constituent order were connected to the order of elements in the \argstl, one
would have to reverse the order of elements to be able to analyze sentences like (\mex{0}b), but
then \emph{sich} would be exempt and \emph{der Mann} would be bound (Principle C violation).

\citet[\page 12]{Riezler95a} discusses the data in (\mex{1}):
\eal
\ex
\gll Auf der Faschingsparty haben wir ihm$_i$ {sich$_i$ selbst} vorgestellt.\\
     at the carnival.party have we him.\dat{} \self.\acc{}      presented\\
\glt `At the carnival party, we presented to him himself.'
\ex
\gll Auf der Faschingsparty haben wir ihn$_i$ {sich$_i$ selbst} vorgestellt.\\
     at  the carnival.party have we him\acc{} \self.\dat{}   presented\\
\glt `At the carnival party, we presented him to himself.'
\zl
Examples like these seem to suggest that the two objects have the same obliqueness since the
accusative can bind the dative and vice versa.\todostefan{reference} So, one could assume that these
elements are not ordered on the \argstl.
% No the binding in Moro is strict.
%\citet{AMM2017a-u} suggested an account for double
%object constructions in \ili{Moro} in which they assumed that the objects are in a set and hence not
%ordered with respect to obliqueness. 
While this would solve the problem of (\mex{0}), it would
introduce another problem: reordering of arguments was accounted for by discharging arguments in an
arbitrary order from an ordered list. Now, if the data structure from which elements are
canceled is not ordered there is too much freedom and the result is \isi{spurious ambiguities}. 

As \citet[\page 140]{Grewendorf85a} notes the reflexive must not precede its antecedent if the
antecedent is an object:
\eal
\ex 
\gll Peter überließ sich$_{*i}$ die Schwester$_i$.\\
     Peter left     \self.\dat{} the.\acc{} sister\\
\glt Intended: `Peter left the sister to himself.'
\ex 
\gll Peter überließ die Schwester$_i$ sich$_i$.\\
     Peter left     the.\acc{} sister \self.\dat{}\\
\glt `Peter left the sister to himself.'
\zl 

%% geht eigentlich doch. Hans ist aber wohl präferiert, weil näher.
%% \citet[\page 148]{Grewendorf85a}:
%% \ea
%% Mit sich$_{*i}$ konfrontierte Hans den Studenten$_i$.
%% \z

%% \citet[\page 148]{Grewendorf85a}:
%% \eal
%% \ex * Hans spricht über sie$_i$ mit Maria$_i$.
%% \ex Hans spricht über sie$_i$ mit ihr$_i$.
%% \ex * Hans spricht mit ihr$_i$ über Maria$_i$.
%% \ex Hans spricht mit ihr$_i$ über sie$_i$.
%% \zl

\inlinetodostefan{
summary}



\subsection{Binding and prepositional objects}

We already discussed the English examples in (\ref{ex-mary-talked-to-john-about-himself}) with two
prepositional objects and showed that the second PP can contain a reflexive referring to a preceding
PP and that HPSG's Binding Theory explains this nicely. However, the situation in German is
different, as the following data from \citet[\page 58]{Grewendorf88a} show:
\eal
\ex 
\gll Ich sprach mit Maria$_i$ über sie$_i$ / *sich.\\
     I talked   with Maria about her {} \self\\
\glt `I talked with Maria about herself.'
\ex 
\gll Ich beklagte mich bei Maria$_i$ über sie$_i$ / *sich.\\
     I complained myself at Maria    about her {} \self\\
\glt `I complained to Maria about herself.'
\zl


The conclusion of the discussion in the previous two subsections is that a Binding Theory that is entirely based on
obliqueness seems to be not possible and that language-specific binding rules referring to specific
situation involving case and part of speech are necessary (see \citew[Chapter~6]{Grewendorf88a} for such rules
for German in GB).

\subsection{Disentangeling \argst and grammatical functions}

\citet{MS98a} discuss data from \ili{Toba Batak}, \ili{Western Austronesian} language. They assume that the
\argst elements are in the order actor and undergoer, but since Toba Batak has two ways to realize
arguments, the so-called \emph{active voice}\is{voice!active} and the \emph{objective voice}\is{voice!objective} either of the arguments
can be the subject. 
\eal
\ex
\gll Mang-ida        si Ria si Torus\\
     \textsc{av}-see \textsc{pm} Ria \textsc{pm} Torus\\
\glt `Torus sees/saw Ria.'
\ex\label{ex-toba-batak-objective-voice}
\gll Di-ida          si Torus si Ria\\
     \textsc{ov}-see \textsc{pm} Torus \textsc{pm} Ria\\
\glt `Torus sees/saw Ria.'
\zl
\citeauthor{MS98a} argue that verb and the adjacent NP form a VP which is combined with the final NP
to yield a full clause. They furthermore argue that neither sentence in (\mex{0}) is a passive or anti-passive
variant of the other. Instead they suggest that the two variants are simply due to different
mappings from argument structure (\argst) to surface valence (\subj and \comps). They provide the
following lexical items:
\ea
\begin{tabular}[t]{@{}l@{~}ll@{~}l}
a. & \emph{mang-ida} `\textsc{av}-see': & b. & \emph{di-ida} `\textsc{ov}-see':\\
   & \ms{
phon & \phonliste{ mang-ida }\\
subj & \sliste{ \ibox{1} }\\
comps & \sliste{ \ibox{2} }\\
arg-st & \sliste{ \ibox{1} NP$_i$, \ibox{2} NP$_j$ }\\[1mm]
cont & \ms[seeing]{
       actor & i\\
       undergoer & j\\
       }
} & & \ms{
phon & \phonliste{ di-ida }\\
subj & \sliste{ \ibox{2} }\\
comps & \sliste{ \ibox{1} }\\
arg-st & \sliste{ \ibox{1} NP$_i$, \ibox{2} NP$_j$ }\\[1mm]
cont & \ms[seeing]{
       actor & i\\
       undergoer & j\\
       }
}
\end{tabular}
\z
The analysis of (\ref{ex-toba-batak-objective-voice}) is given in Figure~\ref{fig-toba-batak-objective-voice}.
Since the second argument, the logical object and undergoer is mapped to \subj in (\mex{0}b), it is combined
with the verb last.

\begin{figure}
\begin{forest}
sm edges
[\ms{ head & \ibox{3} \upshape V\\
      subj & \eliste\\
      comps & \eliste\\
      cont & \ibox{4} \ms[seeing]{
                      actor & i\\
                      undergoer & j\\
                      } }
  [\ms{ head & \ibox{3} \upshape V\\
        subj  & \sliste{ \ibox{2} }\\
        comps & \eliste\\
        cont  & \ibox{4} }
    [\ms{ head & \ibox{3} \upshape V\\
          subj  & \sliste{ \ibox{2} }\\
          comps & \sliste{ \ibox{1} }\\
          arg-st & \sliste{ \ibox{1}$_i$, \ibox{2}$_j$ }\\
          cont  & \ibox{4} } [di-ida;\textsc{ov}-see]]
    [\ibox{1} \ms{ head & \upshape N\\
                   spr & \eliste\\
                   comps & \eliste } [si Tours;\textsc{pm} Torus,roof]]]
    [\ibox{2} \ms{ head & \upshape N\\
                   spr & \eliste\\
                   comps & \eliste } [si Ria;\textsc{pm} Ria,roof]]]
\end{forest}
\caption{Analysis of Toba Batak example in objective voice according to \citet[\page 120]{MS98a}}\label{fig-toba-batak-objective-voice}
\end{figure}

But since binding is taken care of at the \argstl and this list is not affected by voice
differences, this account correctly predicts that the binding patterns do not change independent of
the realization of arguments: as the following examples show, it is always the logical subject, the
actor (the initial element on the \argstl) that binds the non-initial one.
\eal
\ex[]{
\gll [Mang-ida        diri-na$_i$] si John$_i$.\\
     \spacebr{}\textsc{av}-saw self-his \textsc{pm} John\\
\glt `John saw himself.'
}
\ex[*]{
\gll [Mang-ida si John$_i$] diri-na$_{*i}$.\\
     \spacebr{}\textsc{av}-saw \textsc{pm} John self-his\\
\glt Intended: `John saw himself.' with \emph{himself} as the (logical) subject
}
\zl
\eal
\ex[*]{
\gll [Di-ida          diri-na$_i$] si John$_i$\\
     \spacebr{}\textsc{ov}-saw self-his \textsc{pm} John\\
\glt Intended: `John saw himself.' with \emph{himself} as the logical subject
}
\ex[]{
\gll [Di-ida          si John$_i$] diri-na$_i$\\
     \spacebr{}\textsc{ov}-saw \textsc{pm} John self-his\\
\glt `John saw himself.'
}
\zl

\citet[\page 121]{MS98a} point out that theories relying on tree configurations will have to assume
rather complex tree structures for one of the patterns to establish the required c-command
relations. This is unnecessary for \argst-based binding theories.

\citet{WA98a-u} discuss similar data from Balinese and provide a parallel analysis. See also
\citet{Wechsler99a-u} for a comparison of GB analyses with \argst-based HPSG analyses.

The conclusion to be drawn from this section is that obliqueness should not be defined in terms of
grammatical functions as was done in (\ref{def-obliqueness-hierarchy}) above but rather with reference to a thematic hierarchy
as suggested by \citet{Jackendoff72a-u}.\todostefan{pages} This would not make a difference for languages like English, but for languages like
\ili{Toba Batak} and \ili{Balinese} the arguments may be mapped to different grammatical functions
depending on the voice the verb is realized in.

\inlinetodostefan{
Say that binding cannot happen on a thematic level because of raising.
%\ea
Peter believes himself to win.
%\z
believe(Peter,win(himself))
%
I show Peter to himself.
%
Peter was shown to myself by me.
Peter was shown to me by me.
%
Passive active differences cannot be covered.
%
weil Hans sich nicht lachen sah
sehen(Hans,nicht(lachen(sich)))
}

\section{Relative pronouns}
\label{binding-sec-relative-pronouns}

While relative pronouns share many properties with personal pronouns, the syntax of relative and
interrogative clauses places additional demands on the distribution of the respective pronouns. For
languages like German a relative clause consists of a relative phrase and a sentence in
which this relative phrase is missing. The relative phrase has to contain a relative word. Such
relative clauses exist in English as well, but English has various types of relative clauses without
relative phrases. Since we are interested in pronouns here, we consider the cases with relative
pronouns. The simplest case is of course a relative clause with just a relative pronoun in initial position but as
\citet[\page 109]{Ross67a} pointed out the relative pronoun may be deeply embedded.
\eal
\ex\label{Beispiel-Minister}
Here's the minister [[in [the middle [of [whose sermon]]]] the dog barked].\footnote{%
\citew[\page 212]{ps2}.
}
\ex Reports [the height of the lettering on the covers of which] the government prescribes should be
abolished.\label{Ross-reports}\footnote{%
\citew[\page 109]{Ross67a}.\nocite{Ross86a-u}
}
\zl
\citet[Chapter~5]{ps2} accounted for the distribution of relative pronouns by using a GPSG-style percolation
mechanism that is used to percolate the information about the referential index of the relative
pronoun to the maximal level of the relative phrase. The percolation of index information makes it
possible to keep the normal syntax of relative phrases, which is a big advantage for HPSG in
comparison to functor-based approaches like Categorial Grammar (see \citew{Pollard88a} for
discussion) and Dependency Grammar (see \crossrefchaptert[Section~\ref{dg-sec-complex-dependency}]{dg} on Pied Piping in Dependency Grammar).
This index information of the relative pronoun is then identified with
the referential index of the noun that is modified. By identifying the indices it is ensured that
relative pronouns and their antecedent match in gender and number in languages like German.
\eal
\ex
\gll der Junge, der über das Buch gesprochen hat\\
     the boy    who about the book spoken has\\
\glt `the boy who spoke about the book'
\ex
\gll die Frau, die über das Buch gesprochen hat\\
     the woman who about the book spoken has\\
\glt `the woman who spoke about the book'
\ex
\gll die Jungen, die über das Buch gesprochen haben\\
     the boys    who about the book spoken have\\
\glt `the boys who spoke about the book'
\zl
While there is a morphological distinction in the pronouns in German, the same effect can be
observed in English, although indirectly:\footnote{%
  I thank Bob Borsley for pointing this out ot me.
}
\eal
\ex the woman [who talked about herself / * himself]
\ex the men [who talked about themselevs / * himself]
\zl
The personal pronoun \emph{who} is coindexed with the antecedent noun and due to the binding
principle responsible for anaphors, \emph{herself} and \emph{themselves} are coindexed with
\emph{who} and only bindings with agreeing anaphors are possible. \emph{himself} is ruled out in
(\mex{0}). In (\mex{0}a) because it has the wrong gender and in (\mex{0}b) because of wrong number.

The coindexing of antecedent noun and relative pronoun is parallel to the
treatment of personal pronouns but the explicit sharing of the relative pronoun index with the index
of the modified noun excludes bindings to antecedents outside of the clause, which is possible with
personal pronouns.
% Adjuncts are not part of o-command
%% \footnote{%
%% \citet[\page 210, 220]{ps2} assume that relative pronouns are of type \type{npro}. This would rule
%% out all NPs with relative clauses since the noun has a non-pronominal index binding the
%% non-pronominal index of the relative pronoun, which would be a violation of Principle C.%
%% } 
For further discussion of relative clauses see \citew{Sag97a},
\crossrefchaptert{relative-clauses} and \citew{Mueller99b}.

\if0 %\directorscut
\section{Coindexing vs.\ coreference and pragmatic binding}

% Bob: This sees necessary for identity statements such as "I am the reviewer". The two NPs are coreferentail but differ in person.

\citet[\page 75]{ps2} distinguish between coindexing and coreference. They explicitly mention this
possibility in the discussion of examples like the ones in (\mex{1}):\footnote{
  The following sentence is an attested example of a sentence in which one would expect a reflexive
  rather than a pronoun:
        \ea
        I saw me but I thought I was my dad! (J.\,K.\,Rowling \emph{Harry Potter and the Prisoner of Azbakan}, London: Bloomsbury, 1999, p.\,301)
        \z
  The sentence is uttered as part of a description of a time travel. Harry Potter traveled back three hours in
  time and could see the visitor from the future (himself).
}
\eal
\ex It isn't true that nobody voted for John$_{i}$. JOHN$_{j}$ voted for him$_{i}$. 
        (in a context where both uses of {\em John\/} refer to the same person)
\ex He$_{i}$ [pointing to Richard Nixon] voted for Nixon$_{j}$.
\zl

The definition of o-binding requires coindexing. Indices include person, number and gender
information. Since the anaphors in (\mex{1}) are objects and the subjects o-command them locally,
Principle A requires the subject to bind the anaphor and since this is impossible in (\mex{1}b) due
to gender mismatches, the sentence is ungrammatical.
\eal
\ex[]{
John knows himself.
}
\ex[*]{
John knows herself.
}
\zl
However, the inclusion of coindexing into the definition of o-binding causes a problem since nothing
rules out a coreference like the one in (\mex{1}):
\ea
\label{bsp-john-likes-her}
John$_{i'}$ likes her$_{*i'}$.
\z
The primes show coreference rather than coindexation. According to the definition, \emph{John}
does not o-bind \emph{her} since the two NPs cannot be coindexed. Hence, \emph{her} is o-free and
Principle B is not violated even though the coreference in (\mex{0}) is impossible. One could now
stipulate that coreference is excluded in case of gender mismatches, but the problem is more general
and not restricted to gender. \citet[\page 197]{Eisenberg94a} discusses a German example with number mismatches:
\ea
\gll Auf der Brücke stand ein Paar. Sie stritten sich heftig.\\
     on the bridge stood a.\sg{} couple   they.\pl{} argued  self fiercely\\
\glt `There was a couple on the bridge. They argued fiercely.'
\z
The pronoun \emph{sie} `they' refers to \emph{ein Paar} although \emph{sie} is plural and \emph{ein
  Paar} is singular. HPSG's Binding Theory does not have anything to say about coreferences in
texts, but it is easy to create similar examples with pronoun binding within a single sentence:
\eal
\ex 
\gll Das Paar$_{i'}$ behauptet, dass sie$_{i'}$ sich lieben.\\
     the couple.\sg{} claims that they.\pl{} self love.\pl\\
\ex 
\gll Das Paar$_{i}$ behauptet, dass es$_{i}$ sich liebt.\\
     the couple.\sg{} claims that it.\sg{} self love.\sg\\
\zl
The two NPs cannot be coindexed in (\mex{0}a) since the number of the two NPs is
different.\footnote{
  The number is also shown by the agreeing verbs. One way to model agreement is to assume that the
  verb selects a subject with an index with person and number features corresponding to the
  agreement features of the verb \citep[Section~2.4.2]{ps2}. See \citew{WZ2003a} and \crossrefchapterw{agreement} on
  agreement. The alternative would be to have separate purely syntactic agreement features for
  NPs. \emph{das Paar} would be singular as far as agreement is concerned but could have a
  referential index that can be singular or plural.%
}
\citet[Section~9.2.1]{WZ2003a} discuss similar mismatches in \ili{Serbo-Croatian}. They call the phenomenon
\emph{pragmatic agreement}. They formulate rules for female/male pronouns referring to neuter
antecedents, but the examples discussed here involve number differences. As we will see below a
simple rule concerning all pronouns will not work for German since there are some forms of
``pragmatic agreement'' that work for relative and personal pronouns while others work for personal
pronouns only and strict agreement is required for relative pronouns.

As in (\ref{bsp-john-likes-her}), \emph{das Paar} does not o-bind \emph{sie} in (\mex{1}), since it cannot be coindexed:
\ea
\gll Das Paar$_{i'}$ kennt sie$_{*i'}$.\\
     the couple.\sg{} knows they.\pl\\
\glt `The couple knows them.'
\z
Hence, the coreference in (\mex{0}) does not violate any binding principles but it should be excluded
by something like Principle B requiring that a pronoun must not be bound by/coreferential with
something local.

Similarly Principle C does not apply in (\mex{1}):
\ea
\gll Sie$_{*i'}$ behaupten, dass das Paar$_{i'}$ sich liebt.\\
     they.\pl{} claim      that the couple.\sg{} self loves\\
\glt `They claim that the couple loves each other.'
\z

One could assume that \emph{Paar} is underspecified with respect to number, but this would require
an approach to agreement that does not refer to the index.
\eal
\ex[]{
\gll Das Paar schläft.\\
     the couple.\sg{} sleeps.\sg\\
\glt `The couple sleeps.'
}
\ex[*]{
\gll Das Paar schlafen.\\
     the couple.\sg{} sleep.\pl\\
\glt Intended: `The couple sleeps.'
}
\zl
Similarly, the match of the relative pronoun and the noun it refers to is usually established by
sharing the index (\citealp[Chapter~5]{ps2}; \citealp{Sag97a}; \citealp[Chapter~11]{MuellerLehrbuch1};
\crossrefchapteralp{relative-clauses}). While \citet[\page 417--418]{Mueller99a} has data for neuter nouns in German
like \emph{Mädchen} `girl' and \emph{Weib} archaic for `woman', showing that the relative pronoun may be both the neuter
relative pronoun \emph{das} and the female pronoun \emph{die}, the plural relative pronoun with
coreference to \emph{Paar} is strictly ungrammatical:\is{number|)}%
\eal
\ex[]{
\gll das Paar, das sich liebt,\\
     the couple.\sg{} that.\sg{} self loves\\
}
\ex[*]{
\label{ex-paar-die-sich-lieben}
\gll das Paar, die sich lieben,\\
     the couple.\sg{} that.\pl{} self love\\
}
\zl

Furthermore, as pointed out by \citet[\page 418]{Mueller99a}, underspecifying the number feature would run into problems with sentences like
(\mex{1}) containing two different pronouns bound by the same NP:
\ea
\label{ex-paar-es-sie}
\gll Das Paar$_{i}$    behauptet, dass es$_{i}$ sich    liebt und dass sie$_{i}$   sich nie streiten.\\
     the couple.\sg{} claims     that it.\sg{} \self{} loves and that they.\pl{} \self{} never argue\\
\glt `The couple claims that they love each other and that they never argue with each other.'
\z
Since the number value of \emph{Paar} is underspecified, both \emph{es} and \emph{sie} would be
compatible with the index of \emph{Paar}, but identifying the index of \emph{Paar} with one pronoun
makes it incompatible with the other one. 

\inlinetodostefan{ subsumption test}

Now, HPSG developed some new and interesting techniques to cope with conflicting demands in
coordinate structures (see \citealp[\page 207]{LHC2001a-u} and
\crossrefchapteralp[Figure~\ref{qwsa}]{coordination}) and these seem to be applicable here as well:
Figure~\ref{fig-type-number} shows a type hierarchy that has a common subtype of \type{sg} and
\type{pl}. 
\begin{figure}
\centering
\begin{forest}
     [\type{number}
        [\type{sg}
          [\type{strict-sg} ]
          [\type{sg-pl}, name = sgpl ]] 
        [\type{pl}, name=pl,
          [\type{strict-pl} ] ]]
\draw  (pl.south) --(sgpl.north);
\end{forest}
\caption{Type hierarchy making the types \type{sg} and \type{pl} compatible}\label{fig-type-number}
\end{figure}
(\ref{ex-paar-es-sie}) can be analyzed now since \emph{Paar} can be specified to be of type
\type{sg} and this would be compatible with \emph{es} (\type{sg}) and \emph{sie} (\type{pl}): the
result of identifying all indices would result in an index with number value \type{sg-pl}. 

While this general compatibility of singular and plural looks frightening at first sight, one can avoid collapsing all occurrences of
singular and plural into \type{sg-pl} by specifying the number value of linguistic objects that are
strictly singular by assigning the type \type{strict-sg} to them. So \emph{Haus} `house' would have
the number value \type{strict-sg} and would be incompatible with \type{pl}. 

This technique can also be used for relative pronouns: for example, \emph{die} could be specified as \type{strict-pl}
and hence would be incompatible with words like \emph{Paar} `couple', which are \type{sg}. This
would correctly rule out (\ref{ex-paar-die-sich-lieben}), but the
problem with such a solution would be that nouns can be combined with relative clauses and referred
to with pronouns within one sentence:
\ea
\label{ex-paar-das-es-sie}
\gll Das Paar$_{i}$,   das        sich    küsst, behauptet, dass es$_{i}$  sich    liebt und dass sie$_{i}$ sich nie streiten.\\
     the couple.\sg{} that.\sg{} \self{} kisses  claims     that it.\sg{} \self{} loves and that they.\pl{} \self{} never argue\\
\glt `The couple claims that they love each other and that they never argue with each other.'
\z

To make things even more complicated, there are other cases in which relative pronouns can differ in gender.
\citet[\page 417--418]{Mueller99a} provides examples for neuter nouns in German
like \emph{Mädchen} `girl' and \emph{Weib} archaic for `woman', showing that the relative pronoun may be both the neuter
relative pronoun \emph{das} and the female pronoun \emph{die}. The examples in (\mex{1}) show that
the relative pronoun can be \emph{die} (feminine = sexus) and the examples in (\mex{2}) show that it
may be \emph{das} corresponding to the neuter gender of \emph{Mädchen} and \emph{Weib}.
\eal
\ex 
\gll Jenes \emph{Mädchen} ist es, das vertriebene, \emph{die} du gewählt hast.\footnotemark\\
    this.\neu{} girl.\neu{} is it that.\neu{} banished who.\fem{} you choosen have\\
\glt `This is the girl, the banished one, who you chose.' 
\footnotetext{
        Goethe, \emph{Hermann und Dorothea}, Berliner Ausgabe, volume 3, p.\,609, Berlin: Aufbau, 1960. See also \citew[\page 139]{Jung67a}.
}
\ex 
\gll und auch Zeset, sein vollwüchsiges          \emph{Weib}, \emph{die} Dienst tat bei Mut-em-enet, stellte er an, daß sie auf sie einwirke im Sinne seiner Gehässigkeit,\footnotemark\\
     and also Zeset his.\neu{} full.grown.\neu{} wife.\neu{}       who.\fem{} service did at Mut-em-enet  hired   he \partic{} that she on him influence in.the sense his malevolence\\
\footnotetext{
        Thomas Mann, \emph{Joseph und seine Brüder}, 1948, SFV 1960, volume 4/5, p.\,980.
      }
\glt `and he also hired his full-grown wife Zeset, who worked as a maiden at  Mut-em-enet, so that she influenced her in the sense of his malevolence'
\zl
\eal
\ex 
\gll Das    \emph{Mädchen}, \emph{das} Rosen und andere Blumen herumtrug, bot ihm \emph{ihren} Korb dar, [\ldots]\footnotemark\\
     the.{} girl.\neu{}     who.\neu{} roses and other  flowers around.carried offered him her.\fem{} basket \partic\\
\footnotetext{
        Goethe, \emph{Wilhelm Meisters Lehrjahre}, Hamburger Ausgabe, volume 7, p.\,90.
      }
\glt `The girl carrying roses and other flowers around offered her basket to him.'
\ex 
\gll Nun sitz ich hier, wie ein altes \emph{Weib}, \emph{das} \emph{ihr} Holz von Zäunen stoppelt und \emph{ihr} Brot an den Türen, [\ldots]\footnotemark\\
% um ihr hinsterbendes, freudloses Dasein noch einen Augenblick zu verlängern und zu erleichtern.\footnotemark\\
     now sit  I here    like an.\neu{} old.\neu{} woman.\neu{} who.\neu{} her.\fem{} wood of fences gleans and her.\fem{} bread at the doors\\
\footnotetext{
        Goethe, \emph{Die Leiden des jungen Werther}, Hamburger Ausgabe, volume 6, p.\,99.
      }\label{bsp-goethe-altes-weib}
\glt `I sit here now like an old woman gleaning her wood at fences and her bread at the doors [\ldots]'
\zl
The examples in (\mex{0}) additionally have possessive pronouns with female gender. The conclusion is
that ``simple'' underspecification approaches do not work and that additional indices are needed for
pragmatic binding of nouns like \emph{Paar} `couple' and \emph{Mädchen} `girl' but that relative
pronoun binding does not work uniformly for these cases and some special treatment seems to be
required in addition.

A further, more technical problem with type-based solutions as the one suggested in Figure~\ref{fig-type-number} is
that underspecified types basically stand for a disjunction of maximally specific types: \type{sg}
stands for \type{strict-sg} $\vee$ \type{sg-pl}. Since models of linguistic objects have to be
maximally specific \crossrefchapterp{formal-background}, this means that nouns like \emph{Paar} `couple' would have two models. In
situation in which there is no pronoun to decide between \type{strict-sg} and \type{sg-pl}, our
theory would license two models for every phrase the noun occurs in, a clearly unwanted result.

Concluding this section on pragmatic binding, a solution for cases like those in German and
Serbo-Croatian seems to require multiple indices (a pragmatic one in addition to what is usually
assumed) and binding principles that allow reference to the pragmatic index as an alternative to
reference to the normal one.
\fi 

\section{Raising and o-command}
\label{sec-binding-raising}

\inlinetodostefan{Maybe drop that section}

A further problem has to do with predicate complex constructions in languages like
German. Researcher working on SOV languages like German, Dutch or Korean assume that the verbs form
a verbal complex. The arguments of the embedded verb are attracted by the governing verb. This
technique was developed in the framework of Categorial Grammar and taken over to HPSG by
\citet{HN89a,HN94a}. See also
\crossrefchapterw{complex-predicates}. Figure~\ref{fig-verbal-complex-German} shows the analysis of
the following example:

\ea
\gll dass der Sheriff den Dieb  sich überlassen wird\\
     that the sheriff the thief self leave      will\\
\glt `The sheriff will leave the thief to himself.'
\z


\begin{figure}
\begin{forest}
sm edges
[CP
  [C [dass;that]]
  [S
     [\ibox{1} NP [der Sheriff;the sheriff]]
     [V$'$
       [\ibox{2} NP [den Dieb;the thief]]
       [V$'$
         [\ibox{3} NP [sich;self]]
         [V
           [\ibox{4} V \sliste{ \ibox{1}, \ibox{2}, \ibox{3} } [überlassen;leave]]
           [V \sliste{ \ibox{1}, \ibox{2}, \ibox{3}, \ibox{4} } [wird;will]]]]]]]
%% [S
%%   [\ibox{1} NP [Kim]]
%%   [VP
%%     [V \sliste{ \ibox{1}, \ibox{2}, \ibox{3} } [believes]]
%%     [\ibox{2} NP [her]]
%%     [\ibox{3} VP
%%       [V [to]]
%%       [VP
%%         [V \sliste{ \ibox{2}, \ibox{4} } [like]]
%%         [\ibox{4} NP [Sandy]]]]]]
\end{forest}
\caption{Analysis of a German sentence with a verbal complex}\label{fig-verbal-complex-German}
\end{figure}
The verb \emph{überlassen} `to leave' is ditransitive and takes a nominative \iboxb{1}, a dative \iboxb{2}, and an
accusative argument \iboxb{3}. A verb selecting another verb for verbal complex formation takes over
the argument of the embedded verb. The auxiliary \emph{wird} `will' selects \emph{überlassen} `to
leave' \iboxb{4} and the arguments of \emph{überlassen} (\ibox{1}, \ibox{2}, \ibox{3}). The \argstl
of \emph{wird} contains \emph{den Dieb} and \emph{sich} and hence \emph{den Dieb} locally o-binds
\emph{sich}, but \emph{sich} also binds \emph{den Dieb} since \emph{sich} \iboxb{3} is
less-oblique than the verbal complement \ibox{4} and \ibox{4} selects for \emph{den Dieb} \iboxb{2}. For the latter reason, Principle C is violated.  



\citet[\page 33]{Kiss95a}:
\ea
Der Junge$_i$ ließ das Mädchen$_j$ das Boot für sich$_{i/j}$ reparieren.
\z

\eal
\ex Peter ließ sich den Mann helfen.
\ex Peter ließ sich von dem Mann helfen.
\zl

\citet{NB97a}


\citet[\page 9]{BMS2001a}:
As observed by van Noord and Bouma (1996), complement inheritance is incompatible with the notion of ‘local domain’ that is crucial for binding theory. If binding applies to argument structure, however, and complement inheritance is defined for the valence feature COMPS only, the problem is avoided.



\section{Locality}

\citet[Section~20.4.7]{Mueller99a} pointed out that examples like (\mex{1}) involving anaphors
within coordinations are problematic for the HPSG Binding Theory:
\ea
\label{ex-sich-und-seine-familie}
\gll Wir beschreiben ihm$_{i}$ [sich$_{i}$       und seine Familie].\\
     we describe     him      \spacebr{}\self{} and his family\\
\glt `We describe him and his family to him.'
\z
\citet[\page 112]{Fanselow87a} discussed such examples in the context of a GB-style Binding
Theory. Such coordination examples are used since weight plays a role in ordering constituents and
putting the reflexives into a coordination makes the examples more natural.
%Reis73:522
%Hans_i läßt es zwischen Emma_j und sich_i / ihm_*i zum Streit kommen.
%
The example may still seem a bit artificial but there are attested examples from newspapers \citep[\page 420]{Mueller99a}:
\eal
\ex Die Erneuerung war ausschließlich auf Druck von außen zustande gekommen.
\gll Sie verdankte sich keineswegs dem Bedürfnis, vor sich und der Öffentlichkeit Rechenschaft
abzulegen.\footnotemark\\
     she to.be.due.to \self{} not.at.all the desire before \self{} and the public account to.give\\
\footnotetext{
        Taz UNIspezial WS 94/95, 10.15.94, p.\,16
}
\glt `This was not due to the the desire to give account to oneself and the public.'
\ex 
\gll Martin Walser versucht, sich und die Nation zu verstehen.\footnotemark\\
     Martin Walser tries     \self{} and the nation to understand\\
\footnotetext{
        taz, 12.10.98, p.\,1
}
\glt `Martin Walser tries to understand himself and the nation.'
\zl
These sentences pose a challenge for the way locality is defined as part of the definition of local
o-command. Local o-command requires that the commander and the commanded phrase are members of the
same \argstl (\ref{def-local-o-command}), but the result of coordinating two NPs is usually a complex NP with a plural index:
\ea
\gll Der Mann und die Frau kennen / * kennt das Kind.\\
     the man  and the woman know {} {} knows the child\\
\glt `The man and the woman know the child.'
\z
The NP \emph{der Mann und die Frau} `the man and the woman' is an argument of \emph{kennen} `to know'. The index of
\emph{der Mann und die Frau} `the man and the woman' is local with respect to \emph{das Kind} `the child'. The indices of
\emph{der Mann} `the man' and \emph{die Frau} `the woman' are embedded in the complex NP.

For the same reason \emph{sich} is not local to the subject of \emph{abzulegen} in (\mex{-1}) and
not local to \emph{ihm} in (\mex{-2}). This means that the anaphor is not locally o-commanded in any
of the sentences and hence Binding Theory does not say anything about the binding of the reflexive
in these sentences: the anaphors are exempt.

For the same reason, \emph{ihn} `him' is not local to \emph{er} `he' in (\mex{1}b) and hence the
binding of \emph{ihn} `him' to \emph{er} `he', which should be excluded by Principle~B, i not ruled out.
\eal
\ex
\gll Er$_{i}$ sorgt nur für [sich$_{i}$ und seine Familie].\\
     he      cares only for \spacebr{}\self{} and his family\\
\glt `He cares for himself and his family only.'
\ex 
\gll Er$_{i}$ sorgt nur für [ihn$_{*i}$ und seine Familie].\\
     he      cares only for \spacebr{}him and his family\\
\zl
%\inlinetodostefan{Bob: You could presumably propose a similar deletion-based account within a version of HPSG that has order domains. But it would presumably facce the samme objections as a transformational account.}
If one assumed transformational theories of coordination deriving (\mex{0}) from (\mex{1}) (see for example
\citealp[\page 303]{WC80a-u} and \citealp[\page 61, 67]{Kayne94a-u} for proposals to derive verb
coordination from VP coordination plus deletion), the problem would be solved, but as has been
pointed out frequently in the literature such transformation-based theories of coordinations have
many problems \parencites[\page 102]{BV72}[\page 192--193]{Jackendoff77a}[\page143]{Dowty79a}[\page
  104--105]{denBesten83a}{Klein85}{Eisenberg94a}[\page 471]{Borsley2005a} and nobody ever assumed
something parallel in HPSG (see \crossrefchaptert{coordination} on coordination in HPSG).
\ea
\gll Er$_{i}$ sorgt nur für sich     und er$_{i}$ sorgt nur für seine Familie.\\
     he      cares only for \self{} and he       cares only for his family\\
\z


\if0 %directorscut
\section{Linking, order, scope and binding}
\label{sec-argst-order}

While \citet{KC77a} showed that the obliqueness hierarchy is relevant for activeness of grammatical
functions crosslinguistically, it is an open question whether this hierarchy should be assumed to
hold for all lexemes in all languages and if so, whether it plays a role in the same phenomena
universally. As was discussed by \crossrefchaptert{arg-st}, the \argstl plays an important role in
linking theories. In an analysis of (\mex{1}a), \emph{the dog} is the direct object, while \emph{dem
  Hund} bearing dative case is the indirect object in (\mex{1}b): 
\eal
\ex The elephant gave the dog a ball.
\ex
\gll Der Elephant gab  dem Hund einen Ball.\\
     the.\nom{} elephant gave the.\dat{} dog  a.\acc{} ball\\\jambox*{(German)}
\zl
While \citet{Mueller99a} ordered arguments in valence lists  according to the obliqueness hierarchy in (\ref{def-obliqueness-hierarchy}),
\citet{MuellerCoreGram} decided to keep the \argstls and hence also the linking patterns constant
across languages. \citet{MuellerGermanic} analyzes the Germanic languages with \argstls having the
same order of elements and linking patterns, the differences are captured by a different distribution
of lexical and structural case and a different mapping from \argst to \spr and \comps. See
\crossrefchaptert{case} for more on case assignment and \argst in HPSG.

\citet{Kiss2005a} develops an account of quantifier scope determination for German arguing that scope
is determined with respect to an unmarked order.\footnote{
  See \citew{Hoehle82a} for a definition of \emph{normal order}.
}
The unmarked order is nominative, dative, accusative for most German verbs \citep{Hoehle82a}. This does not correspond
to universal tendencies, according to which the direct object precedes the secondary object
\citep{Pullum77a}. Kiss uses the \argstl to represent the unmarked constitutent order. The consequence is that
German seems to require a nom, dat, acc order for (uniform) linking, constituent order, and scope
and nom, acc, dat for binding. If this really is the case, one seems to need two separate lists
to be able to represent both orders.

\fi


\section{\texorpdfstring{\argst}{ARG-ST} lists with internal structure}
\label{binding-sec-passive}

\citet{MS98a} discuss binding in passive clauses. They suggest that the passive is analyzed as a
lexical rule demoting the subject argument and adding an optional PP. If this lexical rule involves the \argstl, this means
that the former object is the initial argument on the \argstl and that reflexives must be bound by
this element rather than by the logical subject of the passivized verb which is optionally expressed
in the \emph{by}-PP.
\ea
\ms{
arg-st & \sliste{ \ibox{1}$_i$, \ibox{2}, \ldots } \\
cont   & \ibox{3}\\
}
$\mapsto$
\ms{
arg-st & \sliste{ \ibox{2}, \ldots } \upshape ( $\oplus$ \sliste{ PP[\type{by}]$_i$ })\\
cont   & \ibox{3}\\
}
\z
However, \citet{Perlmutter1984} argued that more complex representations are necessary to capture the
fact that some languages allow binding to the logical subject of the passivized verb. He discusses
examples from \ili{Russian}. While usually the reflexive has to be bound by the subject as in
(\mex{1}a), the antecedent can be either the subject or the logical subject in passives like (\mex{1}a):

\eal
\label{binding:russian-pass}
\ex  
\gll Boris$_i$    mne      rasskazal anekdot o sebe$_i$.\\
     Boris.\nom{} me.\dat{} told      joke    about \self\\
\glt `Boris told me a joke about himself.'
\ex
\gll Eta kniga byla kuplena Borisom$_{i}$ dlja sebja$_{i}$.  \\
     this book.\nom{} was bought Boris.\textsc{instr} for \self  \\
\glt `This book was bought by Boris for himself.'
\zl
In order to capture the binding facts, \citet{MS98a} suggest that passives of verbs like
\emph{kupitch} `buy' have the following representation at least in Russian.
\ea
\emph{kuplena} `bought':\\
\ms{
arg-st & \sliste{ NP[\type{nom}]$_j$, \sliste{ NP[\type{instr}]$_i$, PRO$_j$, PP$_k$ } }\\[1mm]
cont   & \ms[buying]{
         actor       & i\\
         undergoer   & j\\
         beneficiary & k\\
        }
}
\z
The \argstl is not a simple list like the list for English but it is nested. The complete \argstl of
the lexeme \emph{kupitch} `buy' is contained in the \argstl of the passive. The logical subject is
realized in the instrumental and the logical object is stated as PRO$_j$ on the embedded \argst but
as full NP in the nominative on the top-most \argstl. This setup makes it possible to account for
the fact that a long distance anaphor\footnote{%
Long distance anaphors must obey the following binding principle:
\ea
Principle Z: A locally o-commanded long distance anaphor most be o-bound.
\z
This means that it either can be bound locally or by a suitable binder higher up in the structure.%
}\todostefan{reference} like the reflexive in the PP may refer to one of the two
subjects: the nominative NP in the upper \argstl and the NP in the instrumental in the embedded
result. The PRO element is kept as a reflex of the argument structure of the lexeme. Such PRO
elements play a role in binding phenomena in languages like \ili{Chi-Mwi:ni} also discussed by
\citeauthor{MS98a}.

In order to facilitate distributing the elements of such nested \argstls to valence features like
\subj and \comps, \citet[\page 124, 140]{MS98a} use a complex relational constraint that basically flattens the
nested \argst{}s again and removes all occurrences of PRO.

\section{Explicit constructions of lists with possible antecedents}
\label{sec-bt-nonlocal}

\citet{Branco2002a-u}



\section{Conclusion}

This chapter discussed several approaches to Binding Theory in HPSG. It is shown that approaches
based on an obliqueness order are not sufficient for some languages, that language-specific binding
rules are needed and that order (\eg scrambling in German) plays a role for binding which is not accounted for if obliqueness
is made the sole criterion for legitimate bindings. 
\inlinetodostefan{
elaborate, give explicit rules for German
}
Examples from so-called pragmatic binding show
that HPSG's Binding Theory has to be extended to include rules that account for bindings with
pronouns with different number features and different gender features.

It was shown that the right combination of non-configurational aspects (order of elements on a list
with respect to a thematic hierarchy, allowing for expletives and raised arguments in the list and
accounting for arguments not realized overtly) with configurational aspects (dominance relations
including dominance of adjuncts) results in a novel and unique approach to many problems that do not
have straightforward alternative analyses.

\section*{Abbreviations}

\begin{tabularx}{.99\textwidth}{@{}lX}
\textsc{av} & Agentive Voice\\
\textsc{ov} & Objective Voice\\
\textsc{pm} & pivot marker\\
\end{tabularx}

\section{Todo}

\citet{AGS1998a,Branco2002a-u,Branco:Marrafa:99,Pollard2005a,PX98a,PX2001a-u,Riezler95a,XMcF98a-u,Xue:Pollard:ea:94,Buering2005a-u}


\section*{Acknowledgements}

%The research reported in this chapter was partially supported by the 
%Research Infrastructure for the Science and Technology of Language (\mbox{PORTULAN CLARIN}).

I thank Anne Abeillé for discussion and Bob Levine for discussion and for making \citet{HL95b}
available to me. Thanks go to Bob Borsley for detailed comments of an earlier version of the paper.


{\sloppy
\printbibliography[heading=subbibliography,notkeyword=this]
}
\end{document}


%      <!-- Local IspellDict: en_US-w_accents -->


\if 0

Chung98a:

beide Objekte sind gleich oblique

Local P-command: X locally p-commands Y iff either
(i) X locally obliqueness-commands (locally o-commands) Y, or
(ii) X and Y are equally oblique and X linearly precedes Y.





BMS2001: 44 adjuncts vs. complements. Binding cannot take place on ARG-ST

*I told themi about [the twins’]i birthday.
b. I only get themi presents on [the twins’]i birthday.



\fi
