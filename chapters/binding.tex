% !BIB TS-program = biber
% !TeX TS-program = xelatexmk

\documentclass[output=paper
                ,modfonts
                ,nonflat
	        ,collection
	        ,collectionchapter
	        ,collectiontoclongg
 	        ,biblatex
                ,babelshorthands
                ,newtxmath
                ,draftmode
                ,colorlinks, citecolor=brown
]{./langsci/langscibook}

\IfFileExists{../localcommands.tex}{%hack to check whether this is being compiled as part of a collection or standalone
  % add all extra packages you need to load to this file 

\usepackage{graphicx}
\usepackage{tabularx}
\usepackage{amsmath} 
\usepackage{multicol}
\usepackage{lipsum}
%%%%%%%%%%%%%%%%%%%%%%%%%%%%%%%%%%%%%%%%%%%%%%%%%%%%
%%%                                              %%%
%%%           Examples                           %%%
%%%                                              %%%
%%%%%%%%%%%%%%%%%%%%%%%%%%%%%%%%%%%%%%%%%%%%%%%%%%%%
% remove the percentage signs in the following lines
% if your book makes use of linguistic examples


\usepackage{./langsci/styles/langsci-gb4e} 
\usepackage{./langsci/styles/langsci-optional} 
\usepackage{./langsci/styles/langsci-lgr}
\usepackage{./langsci/styles/langsci-forest-setup}
\usepackage{morewrites}



% Stefan Müller's styles
\usepackage{./styles/merkmalstruktur,./styles/abbrev,./styles/makros.2e,./styles/my-xspace,./styles/article-ex,
./styles/eng-date}

\usepackage{./langsci/styles/jambox}

% Crossing out text
% uncomment when needed
%\usepackage{ulem}

\usepackage{./styles/additional-langsci-index-shortcuts}

\usepackage{./styles/avm+}

\renewcommand{\tpv}[1]{{\avmjvalfont\itshape #1}}

\regAvmFonts

\usepackage{theorem}

\newtheorem{mydefinition}{Def.}
\newtheorem{principle}{Principle}

{\theoremstyle{break}
\newtheorem{schema}{Schema}
\newtheorem{mydefinition-break}[mydefinition]{Def.}
\newtheorem{principle-break}[principle]{Principle}
}

\usepackage{subfig}

  %add all your local new commands to this file

\makeatletter
\def\blx@maxline{77}
\makeatother
  %% hyphenation points for line breaks
%% Normally, automatic hyphenation in LaTeX is very good
%% If a word is mis-hyphenated, add it to this file
%%
%% add information to TeX file before \begin{document} with:
%% %% hyphenation points for line breaks
%% Normally, automatic hyphenation in LaTeX is very good
%% If a word is mis-hyphenated, add it to this file
%%
%% add information to TeX file before \begin{document} with:
%% %% hyphenation points for line breaks
%% Normally, automatic hyphenation in LaTeX is very good
%% If a word is mis-hyphenated, add it to this file
%%
%% add information to TeX file before \begin{document} with:
%% \include{localhyphenation}
\hyphenation{
A-la-hver-dzhie-va
anaph-o-ra
affri-ca-te
affri-ca-tes
Atha-bas-kan
com-ple-ments
Da-ge-stan
Dor-drecht
er-klä-ren-de
Ginz-burg
Gro-ning-en
Jon-a-than
Ka-tho-lie-ke
Ko-bon
krie-gen
Le-Sourd
moth-er
Mül-ler
Nie-mey-er
Prze-piór-kow-ski
phe-nom-e-non
re-nowned
Rie-he-mann
un-bound-ed
}

% why has "erklärende" be listed here? I specified langid in bibtex item. Something is still not working with hyphenation.


% to do: check
%  Alahverdzhieva

\hyphenation{
A-la-hver-dzhie-va
anaph-o-ra
affri-ca-te
affri-ca-tes
Atha-bas-kan
com-ple-ments
Da-ge-stan
Dor-drecht
er-klä-ren-de
Ginz-burg
Gro-ning-en
Jon-a-than
Ka-tho-lie-ke
Ko-bon
krie-gen
Le-Sourd
moth-er
Mül-ler
Nie-mey-er
Prze-piór-kow-ski
phe-nom-e-non
re-nowned
Rie-he-mann
un-bound-ed
}

% why has "erklärende" be listed here? I specified langid in bibtex item. Something is still not working with hyphenation.


% to do: check
%  Alahverdzhieva

\hyphenation{
A-la-hver-dzhie-va
anaph-o-ra
affri-ca-te
affri-ca-tes
Atha-bas-kan
com-ple-ments
Da-ge-stan
Dor-drecht
er-klä-ren-de
Ginz-burg
Gro-ning-en
Jon-a-than
Ka-tho-lie-ke
Ko-bon
krie-gen
Le-Sourd
moth-er
Mül-ler
Nie-mey-er
Prze-piór-kow-ski
phe-nom-e-non
re-nowned
Rie-he-mann
un-bound-ed
}

% why has "erklärende" be listed here? I specified langid in bibtex item. Something is still not working with hyphenation.


% to do: check
%  Alahverdzhieva

  \bibliography{../Bibliographies/stmue,
                ../localbibliography,
../Bibliographies/formal-background,
../Bibliographies/understudied-languages,
../Bibliographies/phonology,
../Bibliographies/case,
../Bibliographies/evolution,
../Bibliographies/agreement,
../Bibliographies/lexicon,
../Bibliographies/np,
../Bibliographies/negation,
../Bibliographies/argst,
../Bibliographies/binding,
../Bibliographies/complex-predicates,
../Bibliographies/coordination,
../Bibliographies/relative-clauses,
../Bibliographies/udc,
../Bibliographies/processing,
../Bibliographies/cl,
../Bibliographies/dg,
../Bibliographies/islands,
../Bibliographies/diachronic,
../Bibliographies/gesture,
../Bibliographies/semantics,
../Bibliographies/pragmatics,
../Bibliographies/information-structure,
../Bibliographies/idioms,
../Bibliographies/cg,
../Bibliographies/udc}

  \togglepaper[14]
}{}



\title{Anaphoric Binding} 
\author{%
Stefan Müller\affiliation{Humboldt-Universität zu Berlin} \lastand António Branco\affiliation{University of Lisbon}
}
% \chapterDOI{} %will be filled in at production

% \epigram{}

\abstract{
This chapter is an introduction into the Binding Theory assumed within HPSG. While it was inspired
by work on Government \& Binding in the beginning, it turned out that reference to tree structures
are not necessary and that relations that are required for interpreting the reference of personal pronouns
and reflexives can be established with respect to lexical properties of heads namely the argument
structure list, a list containing descriptions of arguments of a head.
}


\begin{document}
\maketitle

\label{chap-binding}

\section{Introduction} 

Binding Theories deal with questions of coreference and correspondence of forms. For example, the
reflexives in (\mex{1}) have to refer to the referent the NP in the same clause refers to and they
have to have the same gender as the NP they are coreferent with:
\eal
\ex[]{
Peter$_i$ thinks that Mary$_j$ likes herself$_{*i/j/*k}$.
}
\ex[*]{
Peter$_i$ thinks that Mary$_j$ likes himself$_{*i/*j/*k}$.
}
\ex[*]{
Mary$_i$ thinks that Peter$_j$ likes herself$_{*i/*j/*k}$.
}
\ex[]{
Mary$_i$ thinks that Peter$_j$ likes himself$_{*i/j/*k}$.
}
\zl
The indices show what bindings are possible and which ones are ruled out. For example, in
(\mex{0}a), \emph{herself} cannot refer to \emph{Peter}, it can refer to \emph{Mary} and it cannot
refer to some discourse referent that is not mentioned in the sentence. Coreference of
\emph{himself} and \emph{Mary} is ruled out in (\mex{0}b) since \emph{himself} has an incompatible gender.

Personal pronouns can not refer to an antecedent within the same clause:
\eal
\ex[]{
Peter$_i$ thinks Mary$_j$ that likes her$_{*i/*j/k}$.
}
\ex[]{
Peter$_i$ thinks Mary$_j$ that likes him$_{i/*j/k}$.
}
\ex[]{
Mary$_i$ thinks Peter$_j$ that likes her$_{i/*j/k}$.
}
\ex[]{
Mary$_i$ thinks Peter$_j$ that likes him$_{*i/*j/k}$.
}
\zl
As the examples show, the pronouns \emph{her} and \emph{him} cannot be coreferent with the subject
of \emph{likes}. If a speaker wants to express coreference he or she has to use a reflexive pronoun
as in (\mex{-1}). 

Interestingly, the binding of pronouns is less restricted than the one of reflexives, but this does
not mean that anything goes. For example, a pronoun cannot bind a full referential NP if the NP is
embedded in a clause and the pronoun is in the matrix clause:
\eal
\ex[]{
He$_{*i/*j/k}$ thinks that Mary$_i$ likes Peter$_j$.
}
\ex[]{
He$_{*i/*j/k}$ thinks that Peter$_i$ likes Mary$_j$.
}
\zl  

The sentences discussed so far can be assigned a structure like the one in Figure~\ref{fig-binding-gb}.
\begin{figure}
\begin{forest}
sm edges without translation
[S
  [NP [John\\John\\he]]
  [VP
    [V [thinks\\thinks\\thinks]]
    [CP 
      [C [that\\that\\that]]
      [S
        [NP [Paul\\Paul\\Mary]]
        [VP
         [V [likes\\likes\\likes]]
         [NP [him\\himself\\Peter]]]]]]]
\end{forest}

\caption{\label{fig-binding-gb}Tree configuration of examples for binding}
\end{figure}
\citet{Chomsky81a,Chomsky86a} suggested accounting for the facts by referring to the hierarchical
structure in Figure~\ref{fig-binding-gb}. He uses the notion of c(onstituent)-command going back to
work by \citegen{Reinhart76a-u}. \isi{c-command} is a relation that holds between nodes in a
tree. Accoring to one definition, a node Y is said to c-command another node Z, Y and Z
are sisters or if a sister of Y dominates Z.\footnote{
``Node A c(onstituent)-commands node B if neither A nor B dominates the other and the first
  branching node which dominates A dominates B.'' \citet[\page 32]{Reinhart76a-u}

\citet{Chomsky86a} uses another definition that allows one to go up to the next maximal projection
dominating A. As of 25/02/2020 the English and German Wikipedia pages for c-command have two
conflicting definitions of c-command. The English version follows \citet{SKS2013a-u}, whose
definition excludes c-command between sisters: ``Node X c-commands node Y if a sister of X dominates Y.''
}\todostefan{add page number}

To take an example, the NP node of
\emph{John} c-commands all other nodes dominated by S. The V of \emph{thinks} c-commands everything
within the CP including the CP node, the C of \emph{that} c-commands all nodes in S including also S
and so on. The CP c-commands the \emph{think}-V, and the \emph{likes him}-VP c-commands the
\emph{Paul}-NP. Per definition, a Y binds Z just in case Y and Z are coindexed and Y c-commands
Z. One precondition for being coindexed (in English) is that the person, number, and gender features
of the involved items are compatible.

Now, the goal is to find restrictions that ensure that reflexives are bound locally, personal
pronouns are not bound locally and that referential expressions like proper names and full NPs do
not refer to pronouns or fully referential expressions. The conditions that were developed for
Binding Theory are complex. They also account for the binding of traces that are the result of
moving elements by transformations. While it is elegant to subsume the filler-gap relations under a
general Binding Theory, proponents of HPSG think that coreferential semantic indices and filler-gap
dependencies are crucially different. The places of occurrence of gaps (if they are assumed at all)
is restricted by other components of the theory. For an overview of the treatment of nonlocal
dependencies in HPSG see \crossrefchapterw{udc}.

We will not go into the details of the Binding Theory in Mainstream Generative Grammar
(MGG)\footnote{
We follow \citet[\page 3]{CJ2005a} in using the term \emph{Mainstream Generative Grammar} when
referring to work in Government \& Binding \citep{Chomsky81a} or Minimalism \citep{Chomsky95a-u}.}, but we
give a verbatim description of the ABC of Binding Theory for overt elements. Chomsky distinguishes between
so-called R-expressions (referential expressions like proper nouns or full NPs/DPs), personal
pronouns and reflexives and reciprocals. The latter two are subsumed under the term anaphor. 
Principle A says that an anaphor must be bound within the least maximal projection containing a
subject. Principle B says that a pronoun that is governed by some element G has to be A-free in the
least maximal projection M containing G and a subject. Principle C says that a referential
expression Z heading its own chain has to be A-free in the domain of the head of the chain of Z.


\section{A non-configural Binding Theory}


\ea
Let Y and Z be \type{synsem} objects with distinct \localvs, Y referential. Then Y locally
o-commands Z just in case Y is less oblique than Z.
\z

\ea
Let Y and Z be \type{synsem} objects with distinct \localvs, Y referential. Then Y o-commands Z just
in case Y locally o-commands X dominating Z.
\z

\ea
Y (\emph{locally}) \emph{o-binds} Z just in case Y and Z are coindexed and Y (locally) o-commands Z. If Z
is not (locally) o-bound, then it is said to be (\emph{locally}) \emph{o-free}.
\z

\begin{principle-break}[HPSG-Bindungsprinzipien]
\begin{description}
\item [Prinzip A] Eine lokal o-kom\-man\-dierte Anapher mu"s lokal o-gebunden sein.
\item [Prinzip B] Ein Personalpronomen mu"s lokal o-frei sein.
\item [Prinzip C] Ein Nicht-Pronomen mu"s o-frei sein.
\end{description}
\end{principle-break}


\section{Reconstruction}


\section{Matters of order in the \argstl}

\section{Raising and o-command}


A further problem has to do with predicate complex constructions in languages like
German. Researcher working on SOV languages like German, Dutch or Korean assume that the verbs form
a verbal complex. The arguments of the embedded verb are attracted by the governing verb. This
technique was developed in the framework of Categorial Grammar and taken over to HPSG by
\citet{HN89a,HN94a}. See also
\crossrefchapterw{complex-predicates}. Figure~\ref{fig-verbal-complex-German} shows the analysis of
the following example:

\ea
\gll dass der Sheriff den Dieb  sich überlassen wird\\
     that the sheriff the thief self leave      will\\
\glt `The sheriff will leave the thief to himself.'
\z


\begin{figure}
\begin{forest}
sm edges
[CP
  [C [dass;that]]
  [S
     [\ibox{1} NP [der Sheriff;the sheriff]]
     [V$'$
       [\ibox{2} NP [den Dieb;the thief]]
       [V$'$
         [\ibox{3} NP [sich;self]]
         [V
           [\ibox{4} V \sliste{ \ibox{1}, \ibox{2}, \ibox{3} } [überlassen;leave]]
           [V \sliste{ \ibox{1}, \ibox{2}, \ibox{3}, \ibox{4} } [wird;will]]]]]]]
%% [S
%%   [\ibox{1} NP [Kim]]
%%   [VP
%%     [V \sliste{ \ibox{1}, \ibox{2}, \ibox{3} } [believes]]
%%     [\ibox{2} NP [her]]
%%     [\ibox{3} VP
%%       [V [to]]
%%       [VP
%%         [V \sliste{ \ibox{2}, \ibox{4} } [like]]
%%         [\ibox{4} NP [Sandy]]]]]]
\end{forest}
\caption{Analysis of a German sentence with a verbal complex}\label{fig-verbal-complex-German}
\end{figure}
The verb \emph{überlassen} `to leave' is ditransitive and takes a nominative \iboxb{1}, a dative \iboxb{2}, and an
accusative argument \iboxb{3}. A verb selecting another verb for verbal complex formation takes over
the argument of the embedded verb. The auxiliary \emph{wird} `will' selects \emph{überlassen} `to
leave' \iboxb{4} and the arguments of \emph{überlassen} (\ibox{1}, \ibox{2}, \ibox{3}). The \argstl
of \emph{wird} contains \emph{den Dieb} and \emph{sich} and hence \emph{den Dieb} locally o-binds
\emph{sich}, but \emph{sich} also binds \emph{den Dieb} since \emph{sich} \iboxb{3} is
less-oblique than the verbal complement \ibox{4} and \ibox{4} selects for \emph{den Dieb} \iboxb{2}. For the latter reason, Principle C is violated.  



Kiss95a:33
Der Junge ließ das Mädchen das Boot für sich reparieren.


\section*{Abbreviations}


\section*{Acknowledgements}

The research reported in this chapter was partially supported by the 
Research Infrastructure for the Science and Technology of Language (\mbox{PORTULAN CLARIN}).


{\sloppy
\printbibliography[heading=subbibliography,notkeyword=this]
}
\end{document}


%      <!-- Local IspellDict: en_US-w_accents -->
