%% -*- coding:utf-8 -*-

\documentclass[output=paper
%	        ,collection
%	        ,collectionchapter
 	        ,biblatex
                ,babelshorthands
                ,newtxmath
                ,draftmode
                ,colorlinks, citecolor=brown
]{langscibook}

\IfFileExists{../localcommands.tex}{%hack to check whether this is being compiled as part of a collection or standalone
  \usepackage{../nomemoize}
  % add all extra packages you need to load to this file 

% the ISBN assigned to the digital edition
\usepackage[ISBN=9783961102556]{ean13isbn} 

\usepackage{graphicx}
\usepackage{tabularx}
\usepackage{amsmath} 

%\usepackage{tipa}      % Davis Koenig
\usepackage{xunicode} % Provide tipa macros (BC)

\usepackage{multicol}

% Berthold morphology
\usepackage{relsize}
%\usepackage{./styles/rtrees-bc} % forbidden forest 08.12.2019

% provides logo priniting commands
\usepackage{langsci-basic}

\usepackage{langsci-optional} 
% used to be in this package
\providecommand{\citegen}{}
\renewcommand{\citegen}[2][]{\citeauthor{#2}'s (\citeyear*[#1]{#2})}
\providecommand{\lsptoprule}{}
\renewcommand{\lsptoprule}{\midrule\toprule}
\providecommand{\lspbottomrule}{}
\renewcommand{\lspbottomrule}{\bottomrule\midrule}
\providecommand{\largerpage}{}
\renewcommand{\largerpage}[1][1]{\enlargethispage{#1\baselineskip}}

\usepackage{./styles/biblatex-series-number-checks}


\usepackage{langsci-lgr}

\newcommand{\MAS}{\textsc{m}\xspace} % \M is taken by somebody

%\usepackage{./styles/forest/forest}
\usepackage{langsci-forest-setup}

% is loaded in main etc.
% \usepackage{nomemoize} 
% \memoizeset{
%   memo filename prefix={chapters/hpsg-handbook.memo.dir/},
%   register=\todo{O{}+m},
%   prevent=\todo,
% }

\usepackage{tikz-cd}

\usepackage{./styles/tikz-grid}
\usetikzlibrary{shadows}


% removed with texlive 2020 06.05.2020
% %\usepackage{pgfplots} % for data/theory figure in minimalism.tex
% % fix some issue with Mod https://tex.stackexchange.com/a/330076
% \makeatletter
% \let\pgfmathModX=\pgfmathMod@
% \usepackage{pgfplots}%
% \let\pgfmathMod@=\pgfmathModX
% \makeatother

\usepackage{subcaption}

% Stefan Müller's styles
\usepackage{./styles/merkmalstruktur,./styles/makros.2020,./styles/my-xspace,./styles/article-ex,
./styles/eng-date}

\usepackage{varioref}
\newcommand\refORregion[2]{%
 \vrefpagenum\firstnum{#1}%
 \vrefpagenum\secondnum{#2}%
\ifthenelse{\equal\firstnum\secondnum}%
{\pageref{#1}}%
{\pageref{#1}--\pageref{#2}}%
}

% I am sick of fiddeling arround with babel. I want these shorthands also to work in commands I
% define. St.Mü. 13.08.2020
% e.g. with \iwithini
\usepackage{german}
\selectlanguage{USenglish}

\usepackage{./styles/abbrev}


% Has to be loaded late since otherwise footnotes will not work

%%%%%%%%%%%%%%%%%%%%%%%%%%%%%%%%%%%%%%%%%%%%%%%%%%%%
%%%                                              %%%
%%%           Examples                           %%%
%%%                                              %%%
%%%%%%%%%%%%%%%%%%%%%%%%%%%%%%%%%%%%%%%%%%%%%%%%%%%%
% remove the percentage signs in the following lines
% if your book makes use of linguistic examples
\usepackage{langsci-gb4e} 



% This introduces labels which makes hyperlinks work so that proofreading is easier.
%\makeatletter
%\newcommand{\mex}[1]{\ref{ex-\the\c@chapter-\the\numexpr\c@equation+#1}\relax}
%\newcommand{\eaautolabel}{\label{ex-\the\c@chapter-\the\numexpr\c@equation+1}}
%\makeatother

%\let\oldea\ea
%\def\ea{\oldea\eaautolabel}

%\let\oldeal\eal
%\def\eal{\oldeal\eaautolabel}


% Crossing out text
% uncomment when needed
%\usepackage{ulem}

\usepackage{./styles/additional-langsci-index-shortcuts}

% this is the completely redone avm package
\usepackage{langsci-avm}
\avmsetup{columnsep=.3ex,style=narrow}

\avmdefinecommand{phon}[phon]
  {
    attributes  = \itshape%,
%    delimfactor = 900,
%    delimfall   = 10pt
}

\avmdefinecommand{form}[form]
  {
    attributes  = \itshape%,
%    delimfactor = 900,
%    delimfall   = 10pt
}

% \set was already taken
\avmdefinecommand{avmset}[set]{ attributes=\itshape } % define a new \set command
\avmdefinecommand{list}[list]{ attributes=\itshape } % define a new \list command
   % Note: the label "list" will be output in whatever font is currently active.

% \avm{
% 	[subj  & \1 \\
% 	comps & \2 \- \list*(gap-ss) \\ % Produce a \list
% 	deps  & < \1 > \+ \2
% 	]
% }


\avmdefinecommand{nelist}[ne-list]{ attributes=\itshape } % define a new \nelist command
   % Note: the label "ne-list" will be output in whatever font is currently active.



% https://github.com/langsci/langsci-avm/issues/33#issuecomment-671201576
%\avmsetup{extraskip=0pt}

% if you have to use both langsci-avm and avm
% \usepackage{langsci-avm} % Load pkg with meaning A of conflicting cmd
% \let\lavm\avm % Send the conflicting command to an alternative
% \let\avm\undefined % Send the conflicting cmd to be \undefined
% \usepackage{avm} % Load pkg with meaning B for conf. cmd 

%\let\asort\type*

% remove this, once we really do without avm
%\usepackage{./styles/avm+}

% copied over from avm+.sty
% some relation operators:
%\newcommand{\append}[0]{\ensuremath{\oplus\hspace{.24em}}}
%\newcommand{\shuffle}[0]{\ensuremath{\bigcirc\hspace{.24em}}}

\newcommand{\append}[0]{\ensuremath{\oplus}\xspace}
\newcommand{\shuffle}[0]{\ensuremath{\bigcirc}\xspace}


% command to fontify relations in avms 
\newcommand{\rel}[1]{\texttt{#1}}
%\def\relfont{\slshape}%
%\def\relfont{\ttdefault}%


\let\idx\ibox
\let\avmbox\ibox

% command to fontify attributes in ordinary text
%\newcommand{\attrib}[1]{\textsc{#1}}


% some relation operators:
%\newcommand{\append}[0]{\ensuremath{\oplus\hspace{.24em}}}
%\newcommand{\shuffle}[0]{\ensuremath{\bigcirc\hspace{.24em}}}

\def\relfont{\slshape}%
%
% command to fontify relations in avms 
%\newcommand{\rel}[1]{{\relfont #1}}



% \renewcommand{\tpv}[1]{{\avmjvalfont\itshape #1}}

% % no small caps please
% \renewcommand{\phonshape}[0]{\normalfont\itshape}

% \regAvmFonts

\usepackage{theorem}

\newtheorem{mydefinition}{Def.}
\newtheorem{principle}{Principle}

{\theoremstyle{break}
%\newtheorem{schema}{Schema}
\newtheorem{mydefinition-break}[mydefinition]{Def.}
\newtheorem{principle-break}[principle]{Principle}
}


%% \newcommand{schema}[2]{
%% \begin{minipage}{\textwidth}
%% {\textbf{Schema~\theschema}}]\hspace{.5em}\textbf{(#1)}\\
%% #2
%% \end{minipage}}


% This avoids linebreaks in the Schema
\newcounter{schemacounter}
\makeatletter
\newenvironment{schema}[1][]
  {%
   \refstepcounter{schemacounter}%
   \par\bigskip\noindent
   \minipage{\linewidth}%
   \textbf{Schema~\theschemacounter\hspace{.5em} \ifx&#1&\else(#1)\fi}\par
  }{\endminipage\par\bigskip\@endparenv}%
\makeatother

%\usepackage{subfig}





% Davis Koenig Lexikon

\usepackage{tikz-qtree,tikz-qtree-compat} % Davis Koenig remove

\usepackage{shadow}



\usepackage[english]{isodate} % Andy Lücking
\usepackage[autostyle]{csquotes} % Andy
%\usepackage[autolanguage]{numprint}

%\defaultfontfeatures{
%    Path = /usr/local/texlive/2017/texmf-dist/fonts/opentype/public/fontawesome/ }

%% https://tex.stackexchange.com/a/316948/18561
%\defaultfontfeatures{Extension = .otf}% adds .otf to end of path when font loaded without ext parameter e.g. \newfontfamily{\FA}{FontAwesome} > \newfontfamily{\FA}{FontAwesome.otf}
%\usepackage{fontawesome} % Andy Lücking
\usepackage{pifont} % Andy Lücking -> hand

\usetikzlibrary{decorations.pathreplacing} % Andy Lücking
\usetikzlibrary{matrix} % Andy 
\usetikzlibrary{positioning} % Andy
\usepackage{tikz-3dplot} % Andy

% pragmatics
\usepackage{eqparbox} % Andy
\usepackage{enumitem} % Andy
\usepackage{longtable} % Andy
\usepackage{tabu} % Andy              needs to be loaded before hyperref as of texlive 2020

% tabu-fix
% to make "spread 0pt" work
% -----------------------------
\RequirePackage{etoolbox}
\makeatletter
\patchcmd
	\tabu@startpboxmeasure
	{\bgroup\begin{varwidth}}%
	{\bgroup
	 \iftabu@spread\color@begingroup\fi\begin{varwidth}}%
	{}{}
\def\@tabarray{\m@th\def\tabu@currentgrouptype
    {\currentgrouptype}\@ifnextchar[\@array{\@array[c]}}
%
%%% \pdfelapsedtime bug 2019-12-15
\patchcmd
	\tabu@message@etime
	{\the\pdfelapsedtime}%
	{\pdfelapsedtime}%
	{}{}
%
%
\makeatother
% -----------------------------


% Manfred's packages

%\usepackage{shadow}

\usepackage{tabularx}
\newcolumntype{L}[1]{>{\raggedright\arraybackslash}p{#1}} % linksbündig mit Breitenangabe


% Jong-Bok

%\usepackage{xytree}

\newcommand{\xytree}[2][dummy]{Let's do the tree!}

% seems evil, get rid of it
% defines \ex is incompatible with gb4e
%\usepackage{lingmacros}

% taken from lingmacros:
\makeatletter
% \evnup is used to line up the enumsentence number and an entry along
% the top.  It can take an argument to improve lining up.
\def\evnup{\@ifnextchar[{\@evnup}{\@evnup[0pt]}}

\def\@evnup[#1]#2{\setbox1=\hbox{#2}%
\dimen1=\ht1 \advance\dimen1 by -.5\baselineskip%
\advance\dimen1 by -#1%
\leavevmode\lower\dimen1\box1}
\makeatother


% YK -- CG chapter

%\usepackage{xspace}
\usepackage{bm}
\usepackage{ebproof}


% Antonio Branco, remove this
\usepackage{epsfig}

% now unicode
%\usepackage{alphabeta}





\usepackage{pst-node}


% fmr: additional packages
%\usepackage{amsthm}


% Ash and Steve: LFG
\usepackage{./styles/lfg/dalrymple}

\RequirePackage{graphics}
%\RequirePackage{./styles/lfg/trees}
%% \RequirePackage{avm}
%% \avmoptions{active}
%% \avmfont{\sc}
%% \avmvalfont{\sc}
\RequirePackage{./styles/lfg/lfgmacrosash}

\usepackage{./styles/lfg/glue}

%%%%%%%%%%%%%%%%%%%%%%%%%%%%%%
%% Markup
%%%%%%%%%%%%%%%%%%%%%%%%%%%%%%
\usepackage[normalem]{ulem} % For thinks like strikethrough, using \sout

% \newcommand{\high}[1]{\textbf{#1}} % highlighted text
\newcommand{\high}[1]{\textit{#1}} % highlighted text
%\newcommand{\term}[1]{\textit{#1}\/} % technical term
\newcommand{\qterm}[1]{``{#1}''} % technical term, quotes
%\newcommand{\trns}[1]{\strut `#1'} % translation in glossed example
\newcommand{\trnss}[1]{\strut \phantom{\sqz{}} `#1'} % translation in ungrammatical glossed example
\newcommand{\ttrns}[1]{(`#1')} % an in-text translation of a word
\newcommand{\LFGfeat}[1]{\mbox{\textsc{\MakeLowercase{#1}}}}     % feature name
%\newcommand{\val}[1]{\mbox{\textsc{\MakeLowercase{#1}}}}    % f-structure value
\newcommand{\featt}[1]{\mbox{\textsc{\MakeLowercase{#1}}}}     % feature name
\newcommand{\vall}[1]{\mbox{\textsc{\textup{\MakeLowercase{#1}}}}}    % f-structure value
\newcommand{\mg}[1]{\mbox{\textsc{\MakeLowercase{#1}}}}    % morphological gloss
%\newcommand{\word}[1]{\textit{#1}}       % mention of word
\providecommand{\kstar}[1]{{#1}\ensuremath{^*}}
\providecommand{\kplus}[1]{{#1}\ensuremath{^+}}
\newcommand{\template}[1]{@\textsc{\MakeLowercase{#1}}}
\newcommand{\templaten}[1]{\textsc{\MakeLowercase{#1}}}
\newcommand{\templatenn}[1]{\MakeUppercase{#1}}
\newcommand{\tempeq}{\ensuremath{=}}
\newcommand{\predval}[1]{\ensuremath{\langle}\textsc{#1}\ensuremath{\rangle}}
\newcommand{\predvall}[1]{{\rm `#1'}}
\newcommand{\lfgfst}[1]{\ensuremath{#1\,}}
\newcommand{\scare}[1]{``#1''} % scare quotes
\newcommand{\bracket}[1]{\ensuremath{\left\langle\mathit{#1}\right\rangle}}
\newcommand{\sectionw}[1][]{Section#1} % section word: for cap/non-cap
\newcommand{\tablew}[1][]{Table#1} % table word: for cap/non-cap
\newcommand{\lfgglue}{LFG+Glue}
\newcommand{\hpsgglue}{HPSG+Glue}
\newcommand{\gs}{GS}
%\newcommand{\func}[1]{\ensuremath{\mathbf{#1}}}
\newcommand{\func}[1]{\textbf{#1}}
\renewcommand{\glue}{Glue}
%\newcommand{\exr}[1]{(\ref{ex:#1}}
\newcommand{\exra}[1]{(\ref{ex:#1})}


%%%%%%%%%%%%%%%%%%%%%%%%%%%%%%
% Notation
%\newcommand{\xbar}[1]{$_{\mbox{\textsc{#1}$^{\raisebox{1ex}{}}$}}$}
\newcommand{\xprime}[2][]{\textup{\mbox{{#2}\ensuremath{^\prime_{\hspace*{-.0em}\mbox{\footnotesize\ensuremath{\mathit{#1}}}}}}}}
\providecommand{\xzero}[2][]{#2\ensuremath{^0_{\mbox{\footnotesize\ensuremath{\mathit{#1}}}}}}



\let\leftangle\langle
\let\rightangle\rangle

%\newcommand{\pslabel}[1]{}

% remove when finished
\usepackage{proofread}
  %add all your local new commands to this file

% The orchid-id is specified and then extracted by scripts for zenodo.
\newcommand{\orcid}[1]{} 

% do not show the chapter number. It is redundant, since most references to figures are within the
% same chapter.
\renewcommand{\thefigure}{\arabic{figure}}


% Don't do this at home. I do not like the smaller font for captions.
% I just removed loading the caption packege in langscibook.cls
%% \captionsetup{%
%% font={%
%% stretch=1%.8%
%% ,normalsize%,small%
%% },%
%% width=.8\textwidth
%% }

\makeatletter
\def\blx@maxline{77}
\makeatother


\let\citew\citet

\newcommand{\page}{}

\newcommand{\todostefan}[1]{\todo[color=orange!80]{\footnotesize #1}\xspace}
\newcommand{\todosatz}[1]{\todo[color=red!40]{\footnotesize #1}\xspace}

\newcommand{\inlinetodostefan}[1]{\todo[color=green!40,inline]{\footnotesize #1}\xspace}

\newcommand{\inlinetodoopt}[1]{\todo[color=green!40,inline]{\footnotesize #1}\xspace}
\newcommand{\inlinetodoobl}[1]{\todo[color=red!40,inline]{\footnotesize #1}\xspace}

\newcommand{\itd}[1]{\inlinetodoobl{#1}}
\newcommand{\itdobl}[1]{\inlinetodoobl{#1}}
\newcommand{\itdopt}[1]{\inlinetodoopt{#1}}

\newcommand{\itdsecond}[1]{}

\newcommand{\itddone}[1]{}
%\let\itddone\itdopt
\newcommand{\LATER}[1]{}



% A. Red: Simple typos, errors in the AVMs (only a couple) to take care of on the editorial side, no need to contact the authors
% B.: Green: Wording changes which do not necessarily require authors’ approval, but are not just typos/errors
% C.: Blue: Comments to the author that they don’t have to take care of, but after all, the authors might be interested to have the comments for future revisions. 
% D.: Purple: Comments to the editors about something we need to keep in mind or do. Nothing for you

\newcommand{\colorcodingexplanation}{\todo[color=green!40,inline]{%
Explanation of colors of bubbles and text:\\
A.: Red: Things that have to be fixed/commented upon.\\
B.: Green: optional comments\\
C.: Blue: Comments to the author that they don’t have to take care of, but after all, the authors
might be interested to have the comments for future revisions.\\
Explanation of colors of text:\\
Red: newly added material (crossreferences to other chapters and other references)\\
Orange: changed material, please check\\
Blue: suggestions for deletion\\
Please also check margin notes.
}}
% D.: Purple: Comments to the editors about something we need to keep in mind or do. Nothing for you


\newcommand{\itdgreen}[1]{\todo[color=green!40,inline]{\footnotesize #1}\xspace}
\newcommand{\itdblue}[1]{\todo[color=blue!40,inline]{\footnotesize #1}\xspace}

% for editing, remove later
\usepackage{xcolor}
\newcommand{\added}[1]{{\red #1}}
\newcommand{\addedthis}{\todostefan{added this}}

\newcommand{\changed}[1]{\textcolor{orange}{#1}}
\newcommand{\deleted}[1]{\textcolor{blue}{#1}}


% \newcommand{\addpages}{\todostefan{add pages}}
% %\newcommand{\iaddpages}{\inlinetodoobl{add pages}}
% \newcommand{\iaddpages}{\yel[add pages]{pages}\xspace}
% \newcommand{\addref}{\todostefan{add reference}}
% \newcommand{\inlineaddpages}{\inlinetodostefan{add pages}}
% \newcommand{\addglosses}{\todostefan{add glosses}}

\newcommand{\addpages}{\xspace}%np
\newcommand{\iaddpages}{\xspace}%islands und understudied languages
\newcommand{\addref}{\xspace}
\newcommand{\inlineaddpages}{\xspace}
% not used \newcommand{\addglosses}{}


%\newcommand{\spacebr}{\hphantom{[}}

\newcommand{\danishep}{\jambox{(\ili{Danish})}}
\newcommand{\english}{\jambox{(\ili{English})}}
\newcommand{\german}{\jambox{(\ili{German})}}
\newcommand{\yiddish}{\jambox{(\ili{Yiddish})}}
\newcommand{\welsh}{\jambox{(\ili{Welsh})}}

% Cite and cross-reference other chapters
\newcommand{\crossrefchaptert}[2][]{\citet*[#1]{chapters/#2}, Chapter~\ref{chap-#2} of this volume} 
\newcommand{\crossrefchapterp}[2][]{(\citealp*[#1]{chapters/#2}, Chapter~\ref{chap-#2} of this volume)}
\newcommand{\crossrefchapteralt}[2][]{\citealt*[#1]{chapters/#2}, Chapter~\ref{chap-#2} of this volume}
\newcommand{\crossrefchapteralp}[2][]{\citealp*[#1]{chapters/#2}, Chapter~\ref{chap-#2} of this volume}

\newcommand{\crossrefcitet}[2][]{\citet*[#1]{chapters/#2}} 
\newcommand{\crossrefcitep}[2][]{\citep*[#1]{chapters/#2}}
\newcommand{\crossrefcitealt}[2][]{\citealt*[#1]{chapters/#2}}
\newcommand{\crossrefcitealp}[2][]{\citealp*[#1]{chapters/#2}}


% example of optional argument:
% \crossrefchapterp[for something, see:]{name}
% gives: (for something, see: Author 2018, Chapter~X of this volume)



\let\crossrefchapterw\crossrefchaptert



% Davis Koenig

\let\ig=\textsc
\let\tc=\textcolor

% evolution, Flickinger, Pollard, Wasow

\let\citeNP\citet

% Adam P

%\newcommand{\toappear}{Forthcoming}
\newcommand{\pg}[1]{p.\,#1}
\renewcommand{\implies}{\rightarrow}

\newcommand*{\rref}[1]{(\ref{#1})}
\newcommand*{\aref}[1]{(\ref{#1}a)}
\newcommand*{\bref}[1]{(\ref{#1}b)}
\newcommand*{\cref}[1]{(\ref{#1}c)}

\newcommand{\msadam}{.}
\newcommand{\morsyn}[1]{\textsc{#1}}

\newcommand{\aux}{\textsc{aux}\xspace}

\newcommand{\nom}{\morsyn{nom}}
\newcommand{\acc}{\morsyn{acc}}
\newcommand{\dat}{\morsyn{dat}}
\newcommand{\gen}{\morsyn{gen}}
\newcommand{\ins}{\morsyn{ins}}
%\newcommand{\aploc}{\morsyn{loc}}
\newcommand{\voc}{\morsyn{voc}}
\newcommand{\ill}{\morsyn{ill}}
\renewcommand{\inf}{\morsyn{inf}}
\newcommand{\passprc}{\morsyn{passp}}

%\newcommand{\Nom}{\msadam\nom}
%\newcommand{\Acc}{\msadam\acc}
%\newcommand{\Dat}{\msadam\dat}
%\newcommand{\Gen}{\msadam\gen}
\newcommand{\Ins}{\msadam\ins}
\newcommand{\Loc}{\msadam\loc}
\newcommand{\Voc}{\msadam\voc}
\newcommand{\Ill}{\msadam\ill}
\newcommand{\PassP}{\msadam\passprc}

\newcommand{\Aux}{\textsc{aux}}

%\newcommand{\princ}[1]{\textnormal{\textsc{#1}}} % for constraint names
\newcommand{\princ}[1]{\textnormal{#1}} % for constraint names
\newcommand{\notion}[1]{\emph{#1}}
\renewcommand{\path}[1]{\textnormal{\textsc{#1}}}
\newcommand{\ftype}[1]{\textit{#1}}
\newcommand{\fftype}[1]{{\scriptsize\textit{#1}}}
\newcommand{\la}{$\langle$}
\newcommand{\ra}{$\rangle$}
%\newcommand{\argst}{\path{arg-st}}
\newcommand{\phtm}[1]{\setbox0=\hbox{#1}\hspace{\wd0}}
\newcommand{\prep}[1]{\setbox0=\hbox{#1}\hspace{-1\wd0}#1}


% Rui

\newcommand{\spc}[0]{\hspace{-1pt}\underline{\hspace{6pt}}\,}
\newcommand{\spcs}[0]{\hspace{-1pt}\underline{\hspace{6pt}}\,\,}
\newcommand{\bad}[1]{\leavevmode\llap{#1}}
\newcommand{\COMMENT}[1]{}


% Rui coordination
\newcommand{\subl}[1]{$_{\scriptstyle \textsc{#1}}$}



% Andy Lücking gesture.tex
\newcommand{\Pointing}{\ding{43}}
% Giotto: "Meeting of Joachim and Anne at the Golden Gate" - 1305-10 
\definecolor{GoldenGate1}{rgb}{.13,.09,.13} % Dress of woman in black
\definecolor{GoldenGate2}{rgb}{.94,.94,.91} % Bridge
\definecolor{GoldenGate3}{rgb}{.06,.09,.22} % Blue sky
\definecolor{GoldenGate4}{rgb}{.94,.91,.87} % Dress of woman with shawl
\definecolor{GoldenGate5}{rgb}{.52,.26,.26} % Joachim's robe
\definecolor{GoldenGate6}{rgb}{.65,.35,.16} % Anne's robe
\definecolor{GoldenGate7}{rgb}{.91,.84,.42} % Joachim's halo

\makeatletter
\newcommand{\@Depth}{1} % x-dimension, to front
\newcommand{\@Height}{1} % z-dimension, up
\newcommand{\@Width}{1} % y-dimension, rightwards
%\GGS{<x-start>}{<y-start>}{<z-top>}{<z-bottom>}{<Farbe>}{<x-width>}{<y-depth>}{<opacity>}
\newcommand{\GGS}[9][]{%
\coordinate (O) at (#2-1,#3-1,#5);
\coordinate (A) at (#2-1,#3-1+#7,#5);
\coordinate (B) at (#2-1,#3-1+#7,#4);
\coordinate (C) at (#2-1,#3-1,#4);
\coordinate (D) at (#2-1+#8,#3-1,#5);
\coordinate (E) at (#2-1+#8,#3-1+#7,#5);
\coordinate (F) at (#2-1+#8,#3-1+#7,#4);
\coordinate (G) at (#2-1+#8,#3-1,#4);
\draw[draw=black, fill=#6, fill opacity=#9] (D) -- (E) -- (F) -- (G) -- cycle;% Front
\draw[draw=black, fill=#6, fill opacity=#9] (C) -- (B) -- (F) -- (G) -- cycle;% Top
\draw[draw=black, fill=#6, fill opacity=#9] (A) -- (B) -- (F) -- (E) -- cycle;% Right
}
\makeatother


% pragmatics
\newcommand{\speaking}[1]{\eqparbox{name}{\textsc{\lowercase{#1}\space}}}
\newcommand{\alname}[1]{\eqparbox{name}{\textsc{\lowercase{#1}}}}
\newcommand{\HPSGTTR}{HPSG$_{\text{TTR}}$\xspace}

\newcommand{\ttrtype}[1]{\textit{#1}}
\newcommand{\avmel}{\q<\quad\q>} %% shortcut for empty lists in AVM
\newcommand{\ttrmerge}{\ensuremath{\wedge_{\textit{merge}}}}
\newcommand{\Cat}[2][0.1pt]{%
  \begin{scope}[y=#1,x=#1,yscale=-1, inner sep=0pt, outer sep=0pt]
   \path[fill=#2,line join=miter,line cap=butt,even odd rule,line width=0.8pt]
  (151.3490,307.2045) -- (264.3490,307.2045) .. controls (264.3490,291.1410) and (263.2021,287.9545) .. (236.5990,287.9545) .. controls (240.8490,275.2045) and (258.1242,244.3581) .. (267.7240,244.3581) .. controls (276.2171,244.3581) and (286.3490,244.8259) .. (286.3490,264.2045) .. controls (286.3490,286.2045) and (323.3717,321.6755) .. (332.3490,307.2045) .. controls (345.7277,285.6390) and (309.3490,292.2151) .. (309.3490,240.2046) .. controls (309.3490,169.0514) and (350.8742,179.1807) .. (350.8742,139.2046) .. controls (350.8742,119.2045) and (345.3490,116.5037) .. (345.3490,102.2045) .. controls (345.3490,83.3070) and (361.9972,84.4036) .. (358.7581,68.7349) .. controls (356.5206,57.9117) and (354.7696,49.2320) .. (353.4652,36.1439) .. controls (352.5396,26.8573) and (352.2445,16.9594) .. (342.5985,17.3574) .. controls (331.2650,17.8250) and (326.9655,37.7742) .. (309.3490,39.2045) .. controls (291.7685,40.6320) and (276.7783,24.2380) .. (269.9740,26.5795) .. controls (263.2271,28.9013) and (265.3490,47.2045) .. (269.3490,60.2045) .. controls (275.6359,80.6368) and (289.3490,107.2045) .. (264.3490,111.2045) .. controls (239.3490,115.2045) and (196.3490,119.2045) .. (165.3490,160.2046) .. controls (134.3490,201.2046) and (135.4934,249.3212) .. (123.3490,264.2045) .. controls (82.5907,314.1553) and (40.8239,293.6463) .. (40.8239,335.2045) .. controls (40.8239,353.8102) and (72.3490,367.2045) .. (77.3490,361.2045) .. controls (82.3490,355.2045) and (34.8638,337.3259) .. (87.9955,316.2045) .. controls (133.3871,298.1601) and   (137.4391,294.4766) .. (151.3490,307.2045) -- cycle;
\end{scope}%
}
%% leicht modifiziert nach Def. von Sebastian Nordhoff:
% \newcommand{\lueckingbox}[3]{\parbox[t][][t]{0.7cm}{\raggedright
%     \strut#1}\parbox[t][][t]{7.7cm}{\strut#2}\parbox[t][][t]{3cm}{\raggedright\strut#3}\bigskip\\}
\newcommand{\lueckingbox}[3]{\parbox[t][][t]{0.7cm}{\raggedright
    \strut\vspace*{-\baselineskip}\newline#1}\parbox[t][][t]{7.7cm}{\strut\vspace*{-\baselineskip}\newline#2}\parbox[t][][t]{3cm}{\raggedright\strut\vspace*{-\baselineskip}\newline#3}\bigskip\\}




% KdK
\newcommand{\smiley}{:)}

\renewbibmacro*{index:name}[5]{%
  \usebibmacro{index:entry}{#1}
    {\iffieldundef{usera}{}{\thefield{usera}\actualoperator}\mkbibindexname{#2}{#3}{#4}{#5}}}

% \newcommand{\noop}[1]{}

% chngcntr.sty otherwise gives error that these are already defined
%\let\counterwithin\relax
%\let\counterwithout\relax

% the space of a left bracket for glossings
\newcommand{\LB}{\hphantom{[}}

\newcommand{\LF}{\mbox{$[\![$}}

\newcommand{\RF}{\mbox{$]\!]_F$}}

\newcommand{\RT}{\mbox{$]\!]_T$}}





% Manfred's

\newcommand{\kommentar}[1]{}

\newcommand{\bsp}[1]{\emph{#1}}
\newcommand{\bspT}[2]{\bsp{#1} `#2'}
\newcommand{\bspTL}[3]{\bsp{#1} (lit.: #2) `#3'}

\newcommand{\noidi}{§}

\newcommand{\refer}[1]{(\ref{#1})}

%\newcommand{\avmtype}[1]{\multicolumn{2}{l}{\type{#1}}}
\newcommand{\attr}[1]{\textsc{#1}}

%\newcommand{\srdefault}{\mbox{\begin{tabular}{@{}c@{}}{\large <}\\[-1.5ex]$\sqcap$\end{tabular}}}
\newcommand{\srdefault}{$\stackrel{<}{\sqcap}$}


%% \newcommand{\myappcolumn}[2]{
%% \begin{minipage}[t]{#1}#2\end{minipage}
%% }

%% \newcommand{\appc}[1]{\myappcolumn{3.7cm}{#1}}


% Jong-Bok


% clean that up and do not use \def (killing other stuff defined before)
%\if 0
%\newcommand\DEL{\textsc{del}}
%\newcommand\del{\textsc{del}}

\newcommand\conn{\textsc{conn}}
\newcommand\CONN{\textsc{conn}}
\newcommand\CONJ{\textsc{conj}}
\newcommand\LITE{\textsc{lex}}
\newcommand\lite{\textsc{lex}}
\newcommand\HON{\textsc{hon}}

%\newcommand\CAUS{\textsc{caus}}
%\newcommand\PASS{\textsc{pass}}
\newcommand\NPST{\textsc{npst}}
%\newcommand\COND{\textsc{cond}}



\newcommand\hdlite{\textsc{head-lex construction}}
\newcommand\hdlight{\textsc{head-light} Schema}
\newcommand\NFORM{\textsc{nform}}

\newcommand\RELS{\textsc{rels}}
%\newcommand\TENSE{\textsc{tense}}


%\newcommand\ARG{\textsc{arg}}
\newcommand\ARGs{\textsc{arg0}}
\newcommand\ARGa{\textsc{arg}}
\newcommand\ARGb{\textsc{arg2}}
\newcommand\TPC{\textsc{top}}
%\newcommand\PROG{\textsc{prog}}

\newcommand\LIGHT{\textsc{light}\xspace}
\newcommand\pst{\textsc{pst}}
%\newcommand\PAST{\textsc{pst}}
%\newcommand\DAT{\textsc{dat}}
%\newcommand\CONJ{\textsc{conj}}
\newcommand\nominal{\textsc{nominal}}
\newcommand\NOMINAL{\textsc{nominal}}
\newcommand\VAL{\textsc{val}}
%\newcommand\val{\textsc{val}}
\newcommand\MODE{\textsc{mode}}
\newcommand\RESTR{\textsc{restr}}
\newcommand\SIT{\textsc{sit}}
\newcommand\ARG{\textsc{arg}}
\newcommand\RELN{\textsc{rel}}
%\newcommand\REL{\textsc{rel}}
%\newcommand\RELS{\textsc{rels}}
%\newcommand\arg-st{\textsc{arg-st}}
\newcommand\xdel{\textsc{xdel}}
\newcommand\zdel{\textsc{zdel}}
\newcommand\sug{\textsc{sug}}
%\newcommand\IMP{\textsc{imp}}
%\newcommand\conn{\textsc{conn}}
%\newcommand\CONJ{\textsc{conj}}
%\newcommand\HON{\textsc{hon}}
\newcommand\BN{\textsc{bn}}
\newcommand\bn{\textsc{bn}}
\newcommand\pres{\textsc{pres}}
\newcommand\PRES{\textsc{pres}}
\newcommand\prs{\textsc{pres}}
%\newcommand\PRS{\textsc{pres}}
\newcommand\agt{\textsc{agt}}
%\newcommand\DEL{\textsc{del}}
%\newcommand\PRED{\textsc{pred}}
\newcommand\AGENT{\textsc{agent}}
\newcommand\THEME{\textsc{theme}}
%\newcommand\AUX{\textsc{aux}}
%\newcommand\THEME{\textsc{theme}}
%\newcommand\PL{\textsc{pl}}
\newcommand\SRC{\textsc{src}}
\newcommand\src{\textsc{src}}
\newcommand{\FORMjb}{\textsc{form}}
\newcommand{\formjb}{\FORM}
\newcommand\GCASE{\textsc{gcase}}
\newcommand\gcase{\textsc{gcase}}
\newcommand\SCASE{\textsc{scase}}
\newcommand\PHON{\textsc{phon}}
%\newcommand\SS{\textsc{ss}}
\newcommand\SYN{\textsc{syn}}
%\newcommand\LOC{\textsc{loc}}
\newcommand\MOD{\textsc{mod}}
\newcommand\INV{\textsc{inv}}
%\newcommand\L{\textsc{l}}
%\newcommand\CASE{\textsc{case}}
\newcommand\SPR{\textsc{spr}}
\newcommand\COMPS{\textsc{comps}}
%\newcommand\comps{\textsc{comps}}
\newcommand\SEM{\textsc{sem}}
\newcommand\CONT{\textsc{cont}}
\newcommand\SUBCAT{\textsc{subcat}}
\newcommand\CAT{\textsc{cat}}
%\newcommand\C{\textsc{c}}
%\newcommand\SUBJ{\textsc{subj}}
\newcommand\subjjb{\textsc{subj}}
%\newcommand\SLASH{\textsc{slash}}
\newcommand\LOCAL{\textsc{local}}
%\newcommand\ARG-ST{\textsc{arg-st}}
%\newcommand\AGR{\textsc{agr}}
\newcommand\PER{\textsc{per}}
%\newcommand\NUM{\textsc{num}}
%\newcommand\IND{\textsc{ind}}
\newcommand\VFORM{\textsc{vform}}
\newcommand\PFORM{\textsc{pform}}
\newcommand\decl{\textsc{decl}}
%\newcommand\loc{\textsc{loc   }}
% \newcommand\   {\textsc{  }}

%\newcommand\NEG{\textsc{neg}}
\newcommand\FRAMES{\textsc{frames}}
%\newcommand\REFL{\textsc{refl}}

\newcommand\MKG{\textsc{mkg}}

%\newcommand\BN{\textsc{bn}}
\newcommand\HD{\textsc{hd}}
\newcommand\NP{\textsc{np}}
\newcommand\PF{\textsc{pf}}
%\newcommand\PL{\textsc{pl}}
\newcommand\PP{\textsc{pp}}
%\newcommand\SS{\textsc{ss}}
\newcommand\VF{\textsc{vf}}
\newcommand\VP{\textsc{vp}}
%\newcommand\bn{\textsc{bn}}
\newcommand\cl{\textsc{cl}}
%\newcommand\pl{\textsc{pl}}
\newcommand\Wh{\ital{Wh}}
%\newcommand\ng{\textsc{neg}}
\newcommand\wh{\ital{wh}}
%\newcommand\ACC{\textsc{acc}}
%\newcommand\AGR{\textsc{agr}}
\newcommand\AGT{\textsc{agt}}
\newcommand\ARC{\textsc{arc}}
%\newcommand\ARG{\textsc{arg}}
\newcommand\ARP{\textsc{arc}}
%\newcommand\AUX{\textsc{aux}}
%\newcommand\CAT{\textsc{cat}}
%\newcommand\COP{\textsc{cop}}
%\newcommand\DAT{\textsc{dat}}
\newcommand\NEWCOMMAND{\textsc{def}}
%\newcommand\DEL{\textsc{del}}
\newcommand\DOM{\textsc{dom}}
\newcommand\DTR{\textsc{dtr}}
%\newcommand\FUT{\textsc{fut}}
\newcommand\GAP{\textsc{gap}}
%\newcommand\GEN{\textsc{gen}}
%\newcommand\HON{\textsc{hon}}
%\newcommand\IMP{\textsc{imp}}
%\newcommand\IND{\textsc{ind}}
%\newcommand\INV{\textsc{inv}}
\newcommand\LEX{\textsc{lex}}
\newcommand\Lex{\textsc{lex}}
%\newcommand\LOC{\textsc{loc}}
%\newcommand\MOD{\textsc{mod}}
\newcommand\MRK{{\nr MRK}}
%\newcommand\NEG{\textsc{neg}}
\newcommand\NEW{\textsc{new}}
%\newcommand\NOM{\textsc{nom}}
%\newcommand\NUM{\textsc{num}}
%\newcommand\PER{\textsc{per}}
%\newcommand\PST{\textsc{pst}}
\newcommand\QUE{\textsc{que}}
%\newcommand\REL{\textsc{rel}}
\newcommand\SEL{\textsc{sel}}
%\newcommand\SEM{\textsc{sem}}
%\newcommand\SIT{\textsc{arg0}}
%\newcommand\SPR{\textsc{spr}}
%\newcommand\SRC{\textsc{src}}
\newcommand\SUG{\textsc{sug}}
%\newcommand\SYN{\textsc{syn}}
%\newcommand\TPC{\textsc{top}}
%\newcommand\VAL{\textsc{val}}
%\newcommand\acc{\textsc{acc}}
%\newcommand\agt{\textsc{agt}}
\newcommand\cop{\textsc{cop}}
%\newcommand\dat{\textsc{dat}}
\newcommand\foc{\textsc{focus}}
%\newcommand\FOC{\textsc{focus}}
\newcommand\fut{\textsc{fut}}
\newcommand\hon{\textsc{hon}}
\newcommand\imp{\textsc{imp}}
\newcommand\kes{\textsc{kes}}
%\newcommand\lex{\textsc{lex}}
%\newcommand\loc{\textsc{loc}}
\newcommand\mrk{{\nr MRK}}
%\newcommand\nom{\textsc{nom}}
%\newcommand\num{\textsc{num}}
\newcommand\plu{\textsc{plu}}
\newcommand\pne{\textsc{pne}}
%\newcommand\pst{\textsc{pst}}
\newcommand\pur{\textsc{pur}}
%\newcommand\que{\textsc{que}}
%\newcommand\src{\textsc{src}}
%\newcommand\sug{\textsc{sug}}
\newcommand\tpc{\textsc{top}}
%\newcommand\utt{\textsc{utt}}
%\newcommand\val{\textsc{val}}
%% \newcommand\LITE{\textsc{lex}}
%% \newcommand\PAST{\textsc{pst}}
%% \newcommand\POSP{\textsc{pos}}
%% \newcommand\PRS{\textsc{pres}}
%% \newcommand\mod{\textsc{mod}}%
%% \newcommand\newuse{{`kes'}}
%% \newcommand\posp{\textsc{pos}}
%% \newcommand\prs{\textsc{pres}}
%% \newcommand\psp{{\it en\/}}
%% \newcommand\skes{\textsc{kes}}
%% \newcommand\CASE{\textsc{case}}
%% \newcommand\CASE{\textsc{case}}
%% \newcommand\COMP{\textsc{comp}}
%% \newcommand\CONJ{\textsc{conj}}
%% \newcommand\CONN{\textsc{conn}}
%% \newcommand\CONT{\textsc{cont}}
%% \newcommand\DECL{\textsc{decl}}
%% \newcommand\FOCUS{\textsc{focus}}
%% %\newcommand\FORM{\textsc{form}} duplicate
%% \newcommand\FREL{\textsc{frel}}
%% \newcommand\GOAL{\textsc{goal}}
\newcommand\HEAD{\textsc{head}}
%% \newcommand\INDEX{\textsc{ind}}
%% \newcommand\INST{\textsc{inst}}
%% \newcommand\MODE{\textsc{mode}}
%% \newcommand\MOOD{\textsc{mood}}
%% \newcommand\NMLZ{\textsc{nmlz}}
%% \newcommand\PHON{\textsc{phon}}
%% \newcommand\PRED{\textsc{pred}}
%% %\newcommand\PRES{\textsc{pres}}
%% \newcommand\PROM{\textsc{prom}}
%% \newcommand\RELN{\textsc{pred}}
%% \newcommand\RELS{\textsc{rels}}
%% \newcommand\STEM{\textsc{stem}}
%% \newcommand\SUBJ{\textsc{subj}}
%% \newcommand\XARG{\textsc{xarg}}
%% \newcommand\bse{{\it bse\/}}
%% \newcommand\case{\textsc{case}}
%% \newcommand\caus{\textsc{caus}}
%% \newcommand\comp{\textsc{comp}}
%% \newcommand\conj{\textsc{conj}}
%% \newcommand\conn{\textsc{conn}}
%% \newcommand\decl{\textsc{decl}}
%% \newcommand\fin{{\it fin\/}}
%% %\newcommand\form{\textsc{form}}
%% \newcommand\gend{\textsc{gend}}
%% \newcommand\inf{{\it inf\/}}
%% \newcommand\mood{\textsc{mood}}
%% \newcommand\nmlz{\textsc{nmlz}}
%% \newcommand\pass{\textsc{pass}}
%% \newcommand\past{\textsc{past}}
%% \newcommand\perf{\textsc{perf}}
%% \newcommand\pln{{\it pln\/}}
%% \newcommand\pred{\textsc{pred}}


%% %\newcommand\pres{\textsc{pres}}
%% \newcommand\proc{\textsc{proc}}
%% \newcommand\nonfin{{\it nonfin\/}}
%% \newcommand\AGENT{\textsc{agent}}
%% \newcommand\CFORM{\textsc{cform}}
%% %\newcommand\COMPS{\textsc{comps}}
%% \newcommand\COORD{\textsc{coord}}
%% \newcommand\COUNT{\textsc{count}}
%% \newcommand\EXTRA{\textsc{extra}}
%% \newcommand\GCASE{\textsc{gcase}}
%% \newcommand\GIVEN{\textsc{given}}
%% \newcommand\LOCAL{\textsc{local}}
%% \newcommand\NFORM{\textsc{nform}}
%% \newcommand\PFORM{\textsc{pform}}
%% \newcommand\SCASE{\textsc{scase}}
%% \newcommand\SLASH{\textsc{slash}}
%% \newcommand\SLASH{\textsc{slash}}
%% \newcommand\THEME{\textsc{theme}}
%% \newcommand\TOPIC{\textsc{topic}}
%% \newcommand\VFORM{\textsc{vform}}
%% \newcommand\cause{\textsc{cause}}
%% %\newcommand\comps{\textsc{comps}}
%% \newcommand\gcase{\textsc{gcase}}
%% \newcommand\itkes{{\it kes\/}}
%% \newcommand\pass{{\it pass\/}}
%% \newcommand\vform{\textsc{vform}}
%% \newcommand\CCONT{\textsc{c-cont}}
%% \newcommand\GN{\textsc{given-new}}
%% \newcommand\INFO{\textsc{info-st}}
%% \newcommand\ARG-ST{\textsc{arg-st}}
%% \newcommand\SUBCAT{\textsc{subcat}}
%% \newcommand\SYNSEM{\textsc{synsem}}
%% \newcommand\VERBAL{\textsc{verbal}}
%% \newcommand\arg-st{\textsc{arg-st}}
%% \newcommand\plain{{\it plain}\/}
%% \newcommand\propos{\textsc{propos}}
%% \newcommand\ADVERBIAL{\textsc{advl}}
%% \newcommand\HIGHLIGHT{\textsc{prom}}
%% \newcommand\NOMINAL{\textsc{nominal}}

\newenvironment{myavm}{\begingroup\avmvskip{.1ex}
  \selectfont\begin{avm}}%
{\end{avm}\endgroup\medskip}
\newcommand\pfix{\vspace{-5pt}}


\newcommand{\jbsub}[1]{\lower4pt\hbox{\small #1}}
\newcommand{\jbssub}[1]{\lower4pt\hbox{\small #1}}
\newcommand\jbtr{\underbar{\ \ \ }\ }


%\fi

% cl

\newcommand{\delphin}{\textsc{delph-in}}


% YK -- CG chapter

\newcommand{\grey}[1]{\colorbox{mycolor}{#1}}
\definecolor{mycolor}{gray}{0.8}

\newcommand{\GQU}[2]{\raisebox{1.6ex}{\ensuremath{\rotatebox{180}{\textbf{#1}}_{\scalebox{.7}{\textbf{#2}}}}}}

\newcommand{\SetInfLen}{\setpremisesend{0pt}\setpremisesspace{10pt}\setnamespace{0pt}}

\newcommand{\pt}[1]{\ensuremath{\mathsf{#1}}}
\newcommand{\ptv}[1]{\ensuremath{\textsf{\textsl{#1}}}}

\newcommand{\sv}[1]{\ensuremath{\bm{\mathcal{#1}}}}
\newcommand{\sX}{\sv{X}}
\newcommand{\sF}{\sv{F}}
\newcommand{\sG}{\sv{G}}

\newcommand{\syncat}[1]{\textrm{#1}}
\newcommand{\syncatVar}[1]{\ensuremath{\mathit{#1}}}

\newcommand{\RuleName}[1]{\textrm{#1}}

\newcommand{\SemTyp}{\textsf{Sem}}

\newcommand{\E}{\ensuremath{\bm{\epsilon}}\xspace}

\newcommand{\greeka}{\upalpha}
\newcommand{\greekb}{\upbeta}
\newcommand{\greekd}{\updelta}
\newcommand{\greekp}{\upvarphi}
\newcommand{\greekr}{\uprho}
\newcommand{\greeks}{\upsigma}
\newcommand{\greekt}{\uptau}
\newcommand{\greeko}{\upomega}
\newcommand{\greekz}{\upzeta}

\newcommand{\Lemma}{\ensuremath{\hskip.5em\vdots\hskip.5em}\noLine}
\newcommand{\LemmaAlt}{\ensuremath{\hskip.5em\vdots\hskip.5em}}

\newcommand{\I}{\iota}

\newcommand{\sem}{\ensuremath}

\newcommand{\NoSem}{%
\renewcommand{\LexEnt}[3]{##1; \syncat{##3}}
\renewcommand{\LexEntTwoLine}[3]{\renewcommand{\arraystretch}{.8}%
\begin{array}[b]{l} ##1;  \\ \syncat{##3} \end{array}}
\renewcommand{\LexEntThreeLine}[3]{\renewcommand{\arraystretch}{.8}%
\begin{array}[b]{l} ##1; \\ \syncat{##3} \end{array}}}

\newcommand{\hypml}[2]{\left[\!\!#1\!\!\right]^{#2}}

%%%%for bussproof
\def\defaultHypSeparation{\hskip0.1in}
\def\ScoreOverhang{0pt}

\newcommand{\MultiLine}[1]{\renewcommand{\arraystretch}{.8}%
\ensuremath{\begin{array}[b]{l} #1 \end{array}}}

\newcommand{\MultiLineMod}[1]{%
\ensuremath{\begin{array}[t]{l} #1 \end{array}}}

\newcommand{\hypothesis}[2]{[ #1 ]^{#2}}

\newcommand{\LexEnt}[3]{#1; \ensuremath{#2}; \syncat{#3}}

\newcommand{\LexEntTwoLine}[3]{\renewcommand{\arraystretch}{.8}%
\begin{array}[b]{l} #1; \\ \ensuremath{#2};  \syncat{#3} \end{array}}

\newcommand{\LexEntThreeLine}[3]{\renewcommand{\arraystretch}{.8}%
\begin{array}[b]{l} #1; \\ \ensuremath{#2}; \\ \syncat{#3} \end{array}}

\newcommand{\LexEntFiveLine}[5]{\renewcommand{\arraystretch}{.8}%
\begin{array}{l} #1 \\ #2; \\ \ensuremath{#3} \\ \ensuremath{#4}; \\ \syncat{#5} \end{array}}

\newcommand{\LexEntFourLine}[4]{\renewcommand{\arraystretch}{.8}%
\begin{array}{l} \pt{#1} \\ \pt{#2}; \\ \syncat{#4} \end{array}}

\newcommand{\ManySomething}{\renewcommand{\arraystretch}{.8}%
\raisebox{-3mm}{\begin{array}[b]{c} \vdots \,\,\,\,\,\, \vdots \\
\vdots \end{array}}}

\newcommand{\lemma}[1]{\renewcommand{\arraystretch}{.8}%
\begin{array}[b]{c} \vdots \\ #1 \end{array}}

\newcommand{\lemmarev}[1]{\renewcommand{\arraystretch}{.8}%
\begin{array}[b]{c} #1 \\ \vdots \end{array}}

\newcommand{\p}{\ensuremath{\upvarphi}}

% clashes with soul package
\newcommand{\yusukest}{\textbf{\textsf{st}}}

\newcommand{\shortarrow}{\xspace\hskip-1.2ex\scalebox{.5}[1]{\ensuremath{\bm{\rightarrow}}}\hskip-.5ex\xspace}

\newcommand{\SemInt}[1]{\mbox{$[\![ \textrm{#1} ]\!]$}}

\newcommand{\HypSpace}{\hskip-.8ex}
\newcommand{\RaiseHeight}{\raisebox{2.2ex}}
\newcommand{\RaiseHeightLess}{\raisebox{1ex}}

\newcommand{\ThreeColHyp}[1]{\RaiseHeight{\Bigg[}\HypSpace#1\HypSpace\RaiseHeight{\Bigg]}}
\newcommand{\TwoColHyp}[1]{\RaiseHeightLess{\Big[}\HypSpace#1\HypSpace\RaiseHeightLess{\Big]}}

\newcommand{\LemmaShort}{\ensuremath{ \ \vdots} \ \noLine}
\newcommand{\LemmaShortAlt}{\ensuremath{ \ \vdots} \ }

\newcommand{\fail}{**}
\newcommand{\vs}{\raisebox{.05em}{\ensuremath{\upharpoonright}}}
\newcommand{\DerivSize}{\small}

% This is not needed, we just take unicode symbols
% The result of the code below came out wrong anyway.
% St. Mü. 10.06.2021
%
% \def\maru#1{{\ooalign{\hfil
%   \ifnum#1>999 \resizebox{.25\width}{\height}{#1}\else%
%   \ifnum#1>99 \resizebox{.33\width}{\height}{#1}\else%
%   \ifnum#1>9 \resizebox{.5\width}{\height}{#1}\else #1%
%   \fi\fi\fi%
% \/\hfil\crcr%
% \raise.167ex\hbox{\mathhexbox20D}}}}

\newenvironment{samepage2}%
 {\begin{flushleft}\begin{minipage}{\linewidth}}
 {\end{minipage}\end{flushleft}}

\newcommand{\cmt}[1]{\textsl{\textbf{[#1]}}}
\newcommand{\trns}[1]{\textbf{#1}\xspace}
\newcommand{\ptfont}{}
\newcommand{\gp}{\underline{\phantom{oo}}}
\newcommand{\mgcmt}{\marginnote}

\newcommand{\term}[1]{\emph{\isi{#1}}}

\newcommand{\citeposs}[1]{\citeauthor{#1}'s \citeyearpar{#1}}

% for standalone compilations Felix: This is in the class already
%\let\thetitle\@title
%\let\theauthor\@author 
\makeatletter
\newcommand{\togglepaper}[1][0]{ 
\bibliography{../Bibliographies/stmue,../localbibliography,
collection.bib}
  %% hyphenation points for line breaks
%% Normally, automatic hyphenation in LaTeX is very good
%% If a word is mis-hyphenated, add it to this file
%%
%% add information to TeX file before \begin{document} with:
%% %% hyphenation points for line breaks
%% Normally, automatic hyphenation in LaTeX is very good
%% If a word is mis-hyphenated, add it to this file
%%
%% add information to TeX file before \begin{document} with:
%% \include{localhyphenation}
\hyphenation{
A-la-hver-dzhie-va
ac-cu-sa-tive
anaph-o-ra
ana-phor
ana-phors
an-te-ced-ent
an-te-ced-ents
affri-ca-te
affri-ca-tes
ap-proach-es
Atha-bas-kan
Athe-nä-um
Be-schrei-bung
Bona-mi
Chi-che-ŵa
com-ple-ments
con-straints
Cope-sta-ke
Da-ge-stan
Dor-drecht
er-klä-ren-de
Flick-inger
Ginz-burg
Gro-ning-en
Has-pel-math
Jap-a-nese
Jon-a-than
Ka-tho-lie-ke
Ko-bon
krie-gen
Kroe-ger
Le-Sourd
moth-er
Mül-ler
Nie-mey-er
Ørs-nes
Par-a-digm
Prze-piór-kow-ski
phe-nom-e-non
re-nowned
Rie-he-mann
un-bound-ed
Ver-gleich
with-in
}

% listing within here does not have any effect for lfg.tex % 2020-05-14

% why has "erklärende" be listed here? I specified langid in bibtex item. Something is still not working with hyphenation.


% to do: check
%  Alahverdzhieva


% biblatex:

% This is a LaTeX frontend to TeX’s \hyphenation command which defines hy- phenation exceptions. The ⟨language⟩ must be a language name known to the babel/polyglossia packages. The ⟨text ⟩ is a whitespace-separated list of words. Hyphenation points are marked with a dash:

% \DefineHyphenationExceptions{american}{%
% hy-phen-ation ex-cep-tion }

\hyphenation{
A-la-hver-dzhie-va
ac-cu-sa-tive
anaph-o-ra
ana-phor
ana-phors
an-te-ced-ent
an-te-ced-ents
affri-ca-te
affri-ca-tes
ap-proach-es
Atha-bas-kan
Athe-nä-um
Be-schrei-bung
Bona-mi
Chi-che-ŵa
com-ple-ments
con-straints
Cope-sta-ke
Da-ge-stan
Dor-drecht
er-klä-ren-de
Flick-inger
Ginz-burg
Gro-ning-en
Has-pel-math
Jap-a-nese
Jon-a-than
Ka-tho-lie-ke
Ko-bon
krie-gen
Kroe-ger
Le-Sourd
moth-er
Mül-ler
Nie-mey-er
Ørs-nes
Par-a-digm
Prze-piór-kow-ski
phe-nom-e-non
re-nowned
Rie-he-mann
un-bound-ed
Ver-gleich
with-in
}

% listing within here does not have any effect for lfg.tex % 2020-05-14

% why has "erklärende" be listed here? I specified langid in bibtex item. Something is still not working with hyphenation.


% to do: check
%  Alahverdzhieva


% biblatex:

% This is a LaTeX frontend to TeX’s \hyphenation command which defines hy- phenation exceptions. The ⟨language⟩ must be a language name known to the babel/polyglossia packages. The ⟨text ⟩ is a whitespace-separated list of words. Hyphenation points are marked with a dash:

% \DefineHyphenationExceptions{american}{%
% hy-phen-ation ex-cep-tion }

  \memoizeset{
    memo filename prefix={hpsg-handbook.memo.dir/},
    % readonly
  }
  \papernote{\scriptsize\normalfont
    \@author.
    \titleTemp. 
    To appear in: 
    Stefan Müller, Anne Abeillé, Robert D. Borsley \& Jean-Pierre Koenig (eds.)
    HPSG Handbook
    Berlin: Language Science Press. [preliminary page numbering]
  }
  \pagenumbering{roman}
  \setcounter{chapter}{#1}
  \addtocounter{chapter}{-1}
}
\makeatother

\makeatletter
\newcommand{\togglepaperminimal}[1][0]{ 
  \bibliography{../Bibliographies/stmue,
                ../localbibliography,
collection.bib}
  %% hyphenation points for line breaks
%% Normally, automatic hyphenation in LaTeX is very good
%% If a word is mis-hyphenated, add it to this file
%%
%% add information to TeX file before \begin{document} with:
%% %% hyphenation points for line breaks
%% Normally, automatic hyphenation in LaTeX is very good
%% If a word is mis-hyphenated, add it to this file
%%
%% add information to TeX file before \begin{document} with:
%% \include{localhyphenation}
\hyphenation{
A-la-hver-dzhie-va
ac-cu-sa-tive
anaph-o-ra
ana-phor
ana-phors
an-te-ced-ent
an-te-ced-ents
affri-ca-te
affri-ca-tes
ap-proach-es
Atha-bas-kan
Athe-nä-um
Be-schrei-bung
Bona-mi
Chi-che-ŵa
com-ple-ments
con-straints
Cope-sta-ke
Da-ge-stan
Dor-drecht
er-klä-ren-de
Flick-inger
Ginz-burg
Gro-ning-en
Has-pel-math
Jap-a-nese
Jon-a-than
Ka-tho-lie-ke
Ko-bon
krie-gen
Kroe-ger
Le-Sourd
moth-er
Mül-ler
Nie-mey-er
Ørs-nes
Par-a-digm
Prze-piór-kow-ski
phe-nom-e-non
re-nowned
Rie-he-mann
un-bound-ed
Ver-gleich
with-in
}

% listing within here does not have any effect for lfg.tex % 2020-05-14

% why has "erklärende" be listed here? I specified langid in bibtex item. Something is still not working with hyphenation.


% to do: check
%  Alahverdzhieva


% biblatex:

% This is a LaTeX frontend to TeX’s \hyphenation command which defines hy- phenation exceptions. The ⟨language⟩ must be a language name known to the babel/polyglossia packages. The ⟨text ⟩ is a whitespace-separated list of words. Hyphenation points are marked with a dash:

% \DefineHyphenationExceptions{american}{%
% hy-phen-ation ex-cep-tion }

\hyphenation{
A-la-hver-dzhie-va
ac-cu-sa-tive
anaph-o-ra
ana-phor
ana-phors
an-te-ced-ent
an-te-ced-ents
affri-ca-te
affri-ca-tes
ap-proach-es
Atha-bas-kan
Athe-nä-um
Be-schrei-bung
Bona-mi
Chi-che-ŵa
com-ple-ments
con-straints
Cope-sta-ke
Da-ge-stan
Dor-drecht
er-klä-ren-de
Flick-inger
Ginz-burg
Gro-ning-en
Has-pel-math
Jap-a-nese
Jon-a-than
Ka-tho-lie-ke
Ko-bon
krie-gen
Kroe-ger
Le-Sourd
moth-er
Mül-ler
Nie-mey-er
Ørs-nes
Par-a-digm
Prze-piór-kow-ski
phe-nom-e-non
re-nowned
Rie-he-mann
un-bound-ed
Ver-gleich
with-in
}

% listing within here does not have any effect for lfg.tex % 2020-05-14

% why has "erklärende" be listed here? I specified langid in bibtex item. Something is still not working with hyphenation.


% to do: check
%  Alahverdzhieva


% biblatex:

% This is a LaTeX frontend to TeX’s \hyphenation command which defines hy- phenation exceptions. The ⟨language⟩ must be a language name known to the babel/polyglossia packages. The ⟨text ⟩ is a whitespace-separated list of words. Hyphenation points are marked with a dash:

% \DefineHyphenationExceptions{american}{%
% hy-phen-ation ex-cep-tion }

  \memoizeset{
    memo filename prefix={hpsg-handbook.memo.dir/},
    % readonly
  }
  \papernote{\scriptsize\normalfont
    \@author.
    \@title. 
    To appear in: 
    Stefan Müller, Anne Abeillé, Robert D. Borsley \& Jean-Pierre Koenig (eds.)
    HPSG Handbook
    Berlin: Language Science Press. [preliminary page numbering]
  }
  \pagenumbering{roman}
  \setcounter{chapter}{#1}
  \addtocounter{chapter}{-1}
}
\makeatother




% In case that year is not given, but pubstate. This mainly occurs for titles that are forthcoming, in press, etc.
\renewbibmacro*{addendum+pubstate}{% Thanks to https://tex.stackexchange.com/a/154367 for the idea
  \printfield{addendum}%
  \iffieldequalstr{labeldatesource}{pubstate}{}
  {\newunit\newblock\printfield{pubstate}}
}

\DeclareLabeldate{%
    \field{date}
    \field{year}
    \field{eventdate}
    \field{origdate}
    \field{urldate}
    \field{pubstate}
    \literal{nodate}
}

%\defbibheading{diachrony-sources}{\section*{Sources}} 

% if no langid is set, it is English:
% https://tex.stackexchange.com/a/279302
\DeclareSourcemap{
  \maps[datatype=bibtex]{
    \map{
      \step[fieldset=langid, fieldvalue={english}]
    }
  }
}


% for bibliographies
% biber/biblatex could use sortname field rather than messing around this way.
\newcommand{\SortNoop}[1]{}


% Doug Ball

\newcommand{\elist}{\q<\ \ \q>}

\newcommand{\esetDB}{\q\{\ \ \q\}}


\makeatletter

\newcommand{\nolistbreak}{%

  \let\oldpar\par\def\par{\oldpar\nobreak}% Any \par issues a \nobreak

  \@nobreaktrue% Don't break with first \item

}

\makeatother


% intermediate before Frank's trees are fixed
% This will be removed!!!!!
%\newcommand{\tree}[1]{} % ignore them blody trees
%\usepackage{tree-dvips}


\newcommand{\nodeconnect}[2]{}
\newcommand{\nodetriangle}[2]{}



% Doug relative clauses
%% I've compiled out almost all my private LaTeX command, but there are some
%% I found hard to get rid of. They are defined here.
%% There are few others which defined in places in the document where they have only
%% local effect (e.g. within figures); their names all end in DA, e.g. \MotherDA
%% There are a lot of \labels -- they are all of the form \label{sec:rc-...} or
%% \label{x:rc-...} or similar, so there should be no clashes.

% Subscripts -- scriptsize italic shape lowered by .25ex 
\newcommand{\subscr}[1]{\raisebox{-.5ex}{\protect{\scriptsize{\itshape #1\/}}}}
% A boxed subscript, for avm tags in normal text
\newcommand{\subtag}[1]{\subscr{\idx{#1}}}

%% Sets and tuples: I use \setof{} to get brackets that are upright, not slanted
%\newcommand{\setof}[1]{\ensuremath{\lbrace\,\mathit{#1}\,\rbrace}}
% 11.10.2019 EP: Doug requested replacement of existing \setof definition with the following:
%\newcommand{\setof}[1]{\begin{avm}\{\textcolor{red}{#1}\}\end{avm}}
% 31.1.2019 EP: Doug requested re-replacement of the above \textcolour version with the following:
\newcommand{\setof}[1]{\begin{avm}\{#1\}\end{avm}}

\newcommand{\tuple}[1]{\ensuremath{\left\langle\,\mbox{\textit{#1}}\,\right\rangle}}

% Single pile of stuff, optional arugment is psn (e.g. t or b)
% e.g. to put a over b over c in a centered column, top aligned, do:
%   \cPile[t]{a\\b\\c} 
\newcommand{\cPile}[2][]{%
  \begingroup%
  \renewcommand{\arraystretch}{.5}\begin{tabular}[#1]{@{}c@{}}#2\end{tabular}%
  \endgroup%
}

%% for linguistic examples in running text (`linguistic citation'):
\newcommand{\lic}[1]{\textit{#1}}

%% A gap marked by an underline, raised slightly
%% Default argument indicates how long the line should be:
\newcommand{\uGap}[1][3ex]{\raisebox{.25em}{\underline{\hspace{#1}}}\xspace}

%% \TnodeDA{XP}{avmcontents} -- in a Tree, put a node label next to an AVM
\newcommand{\TnodeDA}[2]{#1~\begin{avm}{#2}\end{avm}}

%% This allows tipa stuff to be put in \emph -- we need to change to cmr first.
%% It is used in the discussion of Arabic.
\newcommand{\emphtipa}[1]{{\fontfamily{cmr}\emph{\tipaencoding #1}}} 



 
 
\definecolor{lsDOIGray}{cmyk}{0,0,0,0.45}


% morphology.tex:
% Berthold

\newcommand{\dnode}[1]{\rnode{#1}{\fbox{#1}}}
\newcommand{\tnode}[1]{\rnode{#1}{\textit{#1}}}

\newcommand{\tl}[2]{#2}

\newcommand{\rrr}[3]{%
  \psframebox[linestyle=none]{%
    \avmoptions{center}
    \begin{avm}
      \[mud & \{ #1 \}\\
      ms & \{ #2 \}\\
      mph & \<  #3 \> \]
    \end{avm}
  }
}
\newcommand{\rr}[2]{%
  \psframebox[linestyle=none]{%
    \avmoptions{center}
    \begin{avm}
      \[mud & \{ #1 \}\\
      mph & \<  #2 \> \]
    \end{avm}
  }
}
 

% Frank Richter
\newtheorem{mydef}{Definition}

\long\def\set[#1\set=#2\set]%
{%
\left\{%
\tabcolsep 1pt%
\begin{tabular}{l}%
#1%
\end{tabular}%
\left|%
\tabcolsep 1pt%
\begin{tabular}{l}%
#2%
\end{tabular}%
\right.%
\right\}%
}

\newcommand{\einruck}{\\ \hspace*{1em}}


%\newcommand{\NatNum}{\mathrm{I\hspace{-.17em}N}}
\newcommand{\NatNum}{\mathbb{N}}
\newcommand{\Aug}[1]{\widehat{#1}}
%\newcommand{\its}{\mathrm{:}}
% Felix 14.02.2020
\DeclareMathOperator{\its}{:}

\newcommand{\sequence}[1]{\langle#1\rangle}

\newcommand{\INTERPRETATION}[2]{\sequence{#1\mathsf{U}#2,#1\mathsf{S}#2,#1\mathsf{A}#2,#1\mathsf{R}#2}}
\newcommand{\Interpretation}{\INTERPRETATION{}{}}

\newcommand{\Inte}{\mathsf{I}}
\newcommand{\Unive}{\mathsf{U}}
\newcommand{\Speci}{\mathsf{S}}
\newcommand{\Atti}{\mathsf{A}}
\newcommand{\Reli}{\mathsf{R}}
\newcommand{\ReliT}{\mathsf{RT}}

\newcommand{\VarInt}{\mathsf{G}}
\newcommand{\CInt}{\mathsf{C}}
\newcommand{\Tinte}{\mathsf{T}}
\newcommand{\Dinte}{\mathsf{D}}

% this was missing from ash's stuff.

%% \def \optrulenode#1{
%%   \setbox1\hbox{$\left(\hbox{\begin{tabular}{@{\strut}c@{\strut}}#1\end{tabular}}\right)$}
%%   \raisebox{1.9ex}{\raisebox{-\ht1}{\copy1}}}



\newcommand{\pslabel}[1]{}

\newcommand{\addpagesunless}{\todostefan{add pages unless you cite the
 work as such}}

% dg.tex
% framed boxes as used in dg.tex
% original idea from stackexchange, but modified by Saso
% http://tex.stackexchange.com/questions/230300/doing-something-like-psframebox-in-tikz#230306
\tikzset{
  frbox/.style={
    rounded corners,
    draw,
    thick,
    inner sep=5pt,
    anchor=base,
  },
}

% get rid of these morewrite messages:
% https://tex.stackexchange.com/questions/419489/suppressing-messages-to-standard-output-from-package-morewrites/419494#419494
\ExplSyntaxOn
\cs_set_protected:Npn \__morewrites_shipout_ii:
  {
    \__morewrites_before_shipout:
    \__morewrites_tex_shipout:w \tex_box:D \g__morewrites_shipout_box
    \edef\tmp{\interactionmode\the\interactionmode\space}\batchmode\__morewrites_after_shipout:\tmp
  }
\ExplSyntaxOff


% This is for places where authors used bold. I replace them by \emph
% but have the information where the bold was. St. Mü. 09.05.2020
\newcommand{\textbfemph}[1]{\emph{#1}}



% Felix 09.06.2020: copy code from the third line into localcommands.tex:
% https://github.com/langsci/langscibook#defined-environments-commands-etc
% Does not work with texlive 2020, is done with sed in Makefile
%\patchcmd{\mkbibindexname}{\ifdefvoid{#3}{}{\MakeCapital{#3} }}{\ifdefvoid{#3}{}{#3 }}{}{\AtEndDocument{\typeout{mkbibindexname could not be patched.}}}



\let\textnobf\textit
% instead of "in bold" write "in italics"
\newcommand{\bolddescriptionintext}{italics\xspace}

% Berthold
\newcommand{\mathplus}{+}
% \mbox{\normalfont +}}
\newcommand{\emdash}{--\xspace}
\newcommand{\emdashUS}{--\xspace}


% Stefan to get the space remvoed infront of the : in Bargmann NPN discussion
%\DeclareMathSymbol{:}{\mathord}{operators}{"3A}
% used {:\,} instead


% for cxg.tex needed for includonly to find the counter.
\newcounter{croftyears} 




% Needed for bibtex entry for Jackendoff's xbar syntax. Without it the bar would be off in itialics.

% https://tex.stackexchange.com/questions/95014/aligning-overline-to-italics-font/95079#95079
% \newbox\usefulbox

% \makeatletter
%     \def\getslant #1{\strip@pt\fontdimen1 #1}

%     \def\skoverline #1{\mathchoice
%      {{\setbox\usefulbox=\hbox{$\m@th\displaystyle #1$}%
%         \dimen@ \getslant\the\textfont\symletters \ht\usefulbox
%         \divide\dimen@ \tw@ 
%         \kern\dimen@ 
%         \overline{\kern-\dimen@ \box\usefulbox\kern\dimen@ }\kern-\dimen@ }}
%      {{\setbox\usefulbox=\hbox{$\m@th\textstyle #1$}%
%         \dimen@ \getslant\the\textfont\symletters \ht\usefulbox
%         \divide\dimen@ \tw@ 
%         \kern\dimen@ 
%         \overline{\kern-\dimen@ \box\usefulbox\kern\dimen@ }\kern-\dimen@ }}
%      {{\setbox\usefulbox=\hbox{$\m@th\scriptstyle #1$}%
%         \dimen@ \getslant\the\scriptfont\symletters \ht\usefulbox
%         \divide\dimen@ \tw@ 
%         \kern\dimen@ 
%         \overline{\kern-\dimen@ \box\usefulbox\kern\dimen@ }\kern-\dimen@ }}
%      {{\setbox\usefulbox=\hbox{$\m@th\scriptscriptstyle #1$}%
%         \dimen@ \getslant\the\scriptscriptfont\symletters \ht\usefulbox
%         \divide\dimen@ \tw@ 
%         \kern\dimen@ 
%         \overline{\kern-\dimen@ \box\usefulbox\kern\dimen@ }\kern-\dimen@ }}%
%      {}}
%     \makeatother




\newcommand{\acknowledgmentsEN}{Acknowledgements}
\newcommand{\acknowledgmentsUS}{Acknowledgments}

% to put two examples next to eachother
%\newcommand{\shortbox}[3][-.7]{
%    \parbox[t]{.4\textwidth}{
%      \vspace{#1\baselineskip} #2\strut~~ #3}%
%}

\newcommand{\twomulticolexamples}[2]{
\begin{tabular}[t]{@{}l@{~~}l@{\hspace{1em}}l@{~~}l@{}}
a. & \parbox[t]{.4\textwidth}{#1} & b. & \parbox[t]{.4\textwidth}{#2}\\
\end{tabular}
}




% This does a linebreak for \gll for long sentences leaving space for the language at the right
% margin.
% St.Mü. 17.06.2021
\newcommand{\longexampleandlanguage}[2]{%
\begin{tabularx}{\linewidth}[t]{@{}X@{}p{\widthof{(#2)}}@{}}%
\begin{minipage}[t]{\linewidth}%
#1%
\end{minipage} & (\ili{#2})%
\end{tabularx}}



\renewcommand{\indexccg}{\is{Categorial Grammar (CG)!Combinatorial \textasciitilde{} (CCG)}\xspace}
\newcommand{\indexccgstart}{\is{Categorial Grammar (CG)!Combinatorial \textasciitilde{} (CCG)|(}\xspace}
\newcommand{\indexccgend}{\is{Categorial Grammar (CG)!Combinatorial \textasciitilde{} (CCG)|)}\xspace}
\renewcommand{\indexmp}{\is{Minimalism}\xspace}


\newcommand{\gisu}{Giuseppe Varaschin\xspace}

\newcommand{\NPi}{NP$\mkern-1mu_i$\xspace}
\newcommand{\NPj}{NP$\mkern-1.5mu_j$\xspace}
  %% -*- coding:utf-8 -*-

%%%%%%%%%%%%%%%%%%%%%%%%%%%%%%%%%%%%%%%%%%%%%%%%%%%%%%%%%%%%
%
% gb4e

% fixes problem with to much vertical space between \zl and \eal due to the \nopagebreak
% command.
\makeatletter
\def\@exe[#1]{\ifnum \@xnumdepth >0%
                 \if@xrec\@exrecwarn\fi%
                 \if@noftnote\@exrecwarn\fi%
                 \@xnumdepth0\@listdepth0\@xrectrue%
                 \save@counters%
              \fi%
                 \advance\@xnumdepth \@ne \@@xsi%
                 \if@noftnote%
                        \begin{list}{(\thexnumi)}%
                        {\usecounter{xnumi}\@subex{#1}{\@gblabelsep}{0em}%
                        \setcounter{xnumi}{\value{equation}}}
% this is commented out here since it causes additional space between \zl and \eal 06.06.2020
%                        \nopagebreak}%
                 \else%
                        \begin{list}{(\roman{xnumi})}%
                        {\usecounter{xnumi}\@subex{(iiv)}{\@gblabelsep}{\footexindent}%
                        \setcounter{xnumi}{\value{fnx}}}%
                 \fi}
\makeatother

% the texlive 2020 langsci-gb4e adds a newline after \eas, the texlive 2017 version was OK.
% \makeatletter
% \def\eas{\ifnum\@xnumdepth=0\begin{exe}[(34)]\else\begin{xlist}[iv.]\fi\ex\begin{tabular}[t]{@{}p{.98\linewidth}@{}}}
% \makeatother



%%%%%%%%%%%%%%%%%%%%%%%%%%%%%%%%%%%%%%%%%%%%%%%%%%%%%%%%%%
%
% biblatex

% biblatex sets the option autolang=hyphens
%
% This disables language shorthands. To avoid this, the hyphens code can be redefined
%
% https://tex.stackexchange.com/a/548047/18561

\makeatletter
\def\hyphenrules#1{%
  \edef\bbl@tempf{#1}%
  \bbl@fixname\bbl@tempf
  \bbl@iflanguage\bbl@tempf{%
    \expandafter\bbl@patterns\expandafter{\bbl@tempf}%
    \expandafter\ifx\csname\bbl@tempf hyphenmins\endcsname\relax
      \set@hyphenmins\tw@\thr@@\relax
    \else
      \expandafter\expandafter\expandafter\set@hyphenmins
      \csname\bbl@tempf hyphenmins\endcsname\relax
    \fi}}
\makeatother


% the package defined \attop in a way that produced a box that has textwidth
%
\def\attop#1{\leavevmode\begin{minipage}[t]{.995\linewidth}\strut\vskip-\baselineskip\begin{minipage}[t]{.995\linewidth}#1\end{minipage}\end{minipage}}


%%%%%%%%%%%%%%%%%%%%%%%%%%%%%%%%%%%%%%%%%%%%%%%%%%%%%%%%%%%%%%%%%%%%


% Don't do this at home. I do not like the smaller font for captions.
% This does not work. Throw out package caption in langscibook
% \captionsetup{%
% font={%
% stretch=1%.8%
% ,normalsize%,small%
% },%
% width=\textwidth%.8\textwidth
% }
% \setcaphanging


  \togglepaper[9]
}{}

\author{%
	Stephen Wechsler\affiliation{The University of Texas}%
	\and Jean-Pierre Koenig\affiliation{University at Buffalo}%
	\lastand Anthony Davis\affiliation{Southern Oregon University}%
}
\title{Argument structure and linking}

% \chapterDOI{} %will be filled in at production

%\epigram{}%We are up against one of the great sources of philosophical bewilderment: we must try to find a substance for a substantive.}
\abstract{In this chapter, we discuss the nature and purpose of argument structure in HPSG, focusing
  on the problems that theories of argument structure are intended to solve, including: (1) the
  relationship between semantic arguments of predicates and their syntactic realizations, (2) the
  fact that lexical items can occur in more than one syntactic frame (so-called valence or diathesis
  alternations), and (3) argument structure as the locus of binding principles. We also discuss
  cases where the argument structure of a verb includes more \add{elements} than predicted from the meaning of the
  verb, as well as rationales for a lexical approach to argument structure.} 


\begin{document}
\maketitle
\label{chap:argumentstr}\label{chap-argumentstr}\label{chap-arg-st}

\colorcodingexplanation

\section{Introduction}


For a verb or other predicator to compose with the phrases or pronominal affixes expressing its
semantic arguments, the grammar must specify the mapping between the semantic participant
roles\is{participant roles|see{thematic role, semantic role, proto-role}} and syntactic dependents
of that verb.  For example, the grammar of \ili{English} indicates that the subject of \word{eat}
fills the eater role and the object of \word{eat}  fills the role of the thing eaten.  In HPSG, this
mapping is usually broken down into two simpler mappings by positing an intermediate representation
called \argst (argument structure).  The first mapping connects the participant roles within the
semantic \content with the elements of the value of the \argst feature; here we will call the theory
of this mapping \emph{linking theory} (see Section~\ref{linking-sec}).  The second mapping connects
those \argst list elements to the elements of the \changed{valence} lists, namely \comps
(complements) and \subj \changed{(subject; or \spr for specifier)}; we will refer to this second mapping as
\emph{argument realization} (see Section~\ref{sec:arg-st}).\footnote{Some linguists, such as
  \citet{LevinandRappaport2005}, use the term ``argument realization'' more broadly, to encompass
  linking as well.}  
These two mappings\is{mapping|see{linking, argument realization}} are
illustrated with the simplified lexical sign for the verb \word{eat} in (\ref{eat}) (for ease of
presentation, we use a standard predicate-calculus representation of the value of \content in
(\ref{eat}) rather than the attribute-value representation we introduce later on).  
%JPK 1-13-20 Just explained the notation 


\ea
\label{eat}
Lexical sign for the verb \word{eat}:\\
\avm{
[\phon < eat > \\
%TD 17 July: maybe /i:t/, or orth rather than phon
subj   & < \1 > \\ 
comps  & < \2 > \\ 
arg-st & < \1\,NP$_i$ , \2\,NP$_j$ > \\	
content & eat!(i, j)!] 
}
\z

	
\noindent
In (\ref{eat}), ``NP'' abbreviates a feature \rep{structure}{description} representing syntactic and semantic information about a nominal phrase.  The variables $i$ and 
$j$ are the referential indices for the eater and eaten arguments, respectively, of the \textit{eat} relation.  The semantic information in 
NP$_i$ semantically restricts the value or referent of $i$. 

The \argst feature plays an important  
role in HPSG grammatical theory.  In addition to regulating the mapping from semantic arguments to
grammatical relations, \argst is the locus of the theories of anaphoric binding and other construal
relations such as control and raising.  (This chapter focuses on the function of \argst  in semantic
mapping, with some discussion of binding and other construal relations only insofar as they interact
with that mapping.  A more detailed look at binding is presented in
\crossrefchapterw{binding}. \added{Control and raising is the topic of Chapter~\ref{chap-control-raising} \citep{chapters/control-raising}}.)   

In HPSG, verb diathesis alternations, voice alternations, and derivational processes such as category conversions are all captured within the lexicon (see Section~\ref{alternations} and \crossrefchapteralt{lexicon}).  The different variants of a word are grammatically related either through lexical rules or by means of the lexical type hierarchy.  HPSG grammars explicitly capture paradigmatic relations between word variants, making HPSG a \textit{lexical approach to argument structure}, in the sense of \citet{MWArgSt}.
This fundamental property of lexicalist theories contrasts with many transformational approaches, where such relationships are treated as syntagmatically related through operations on phrasal structures representing sentences and other syntactic constituents.  Arguments for the lexical approach are reviewed in Section~\ref{lexicalapproach}.  

Within the HPSG framework presented here, we will formulate and address a number of empirical and theoretical questions: 

\begin{itemize}
\item We know that a verb's meaning influences its valence requirements (via the \argst list, on this theory). 
 What are the principles governing the mapping from \content to \argst?  Are some aspects of \argst idiosyncratically stipulated for individual verbs?  Which aspects of the semantic \content  bear on the value of \argst, and which aspects do not?  (For example, what is the role of modality?)  
\item How are argument alternations defined with respect to our formal system?  For each alternation
  we may ask which of the following it involves: a shuffling of the \argst list;  a change in the
  mapping from \argst to \changed{the valence lists}; or  a change in the \content, with a concomitant change in the \argst?  
\end{itemize}

\noindent
These questions will be addressed below in the course of presenting the theory.  We begin by
considering \argst itself (Section~\ref{sec:arg-st}), followed by the mapping from \argst to
\changed{valence lists} (Section~\ref{argst-valence-sec}), and the mapping from \content to \argst
(Section~\ref{linking-sec}).\is{mapping|see{linking, argument realization}} The remaining sections
address further issues relating to argument structure: the nature of argument alternations,
extending the \argst attribute to include additional elements, whether \argst is a universal feature
of languages, and a comparison of the lexicalist view of argument structure presented here with
phrasal approaches. 


\section{The representation of argument structure in HPSG}

\label{sec:arg-st}

In the earliest versions of HPSG, the selection of dependent phrases was specified in the \subcat feature\isfeat{subcat} of the head word (\citealt{pollard+sag:1987}, \citealt[Chapters~1--8]{pollard+sag:1994}).  The value of \subcat is a list of items, each of which corresponds to the \synsem value of a complement or subject.  The following are \subcat features for an intransitive verb, a transitive verb, and a transitive verb with obligatory PP complement:


\begin{exe} 
\ex \label{subcats}
\begin{xlist}
\ex \word{laugh}: $[$\subcat \sliste{ NP }$]$
\ex \word{eat}:   $[$\subcat \sliste{ NP, NP }$]$
\ex \word{put}:   $[$\subcat \sliste{ NP, NP, PP }$]$
\end{xlist}
\end{exe}

\noindent
Phrase structure rules in the form of immediate dominance schemata\is{immediate dominance schema}
identify a certain daughter node as the head daughter (\textsc{head-dtr}) and others, including
subjects, as complement daughters (\textsc{comp-dtrs}).  In keeping with the \emph{Subcategorization
  Principle},\is{Subcategorization Principle} here paraphrased from \citew[34]{pollard+sag:1994},
list items are effectively ``cancelled'' from the \subcat list as complement phrases, including the
subject, are joined with the selecting head:

\begin{exe}
\ex Subcategorization Principle: In a headed phrase, the \subcat value of the \headdtr (head daughter) is the concatenation of the phrase's \subcat list with the list of \synsem values of the \compsdtrs (complement daughters).
\end{exe}

\noindent
Phrasal positions are distinguished by their saturation level: ``VP'' is defined as a verbal projection whose \subcat list contains a single item, corresponding to the subject, and ``S'' is defined as a verbal projection whose  \subcat list is empty. 

\is{ARG-ST!as locus of binding}
The ``subject'' of a verb, a distinguished dependent with respect to
construal processes such as binding, control, and raising, was then defined as the first item in the
\subcat list, hence the last item with which the verb combines.  However, defining ``subject'' as
the last item to combine with the head proved inadequate \citep[Chapter~9]{pollard+sag:1994}.  There
are many cases where the dependent displaying subject properties need not be the last item added to
the head projection.  For example, in \ili{German} the subject is a nominal in nominative case
\citep{Reis82}, but the language allows subjectless clauses containing only a dative or genitive
non-subject NP.  If that oblique\is{oblique argument} NP is the only NP dependent to combine with
the verb, then it is \emph{ipso facto} the last NP to combine, yet such obliques lack the construal
properties of subjects in \ili{German}.

Consequently, the \subcat list was split into two valence lists, a \subj list of length zero or one
for subjects, and a \comps list for complements.  Nonetheless, certain grammatical phenomena, such
as binding and other construal processes, must still be defined on a single list comprising both
subject and complements \citep{Manning+Sag:1999}. Additionally, some syntactic arguments are
unexpressed or realized by affixal pronouns, rather than as subject or complement phrases.  The new
list containing all the syntactic arguments of a predicator was named \textsc{arg-st} (argument
structure).

In clauses without implicit or affixal arguments, the \textsc{arg-st} is the concatenation of
\feat{subj} and \feat{comps} respectively.  For example, the \subcat list for \word{put} in
(\ref{subcats}c) is replaced with the following:

\begin{exe} 
	\label{put}
\ex
\avm{
[\phon < put > \\
subj   & < \1 > \\ 
comps  & < \2, \3 >  \\ 
arg-st & < \1\,NP, \2\,NP, \3 PP > ] 
}
\end{exe}

\noindent
The idealization according to which \argst is the concatenation of \subj and \comps is canonized as
the \emph{Argument Realization Principle} (ARP)\is{Argument Realization Principle}
\citep[494]{SWB2003a}.  Systematic exceptions to the ARP, that is, dissociations between
\textsc{valence} and \argst, are discussed in Section~\ref{argst-sec} below.

A predicator's \textsc{valence} lists indicate its requirements for syntactic concatenation with
phrasal dependents (Section~\ref{argst-valence-sec}).  \argst, meanwhile, provides syntactic
information about the expression of semantic roles and is related, via linking theory, to the
lexical semantics of the word (Section~\ref{argst-sec}).  The \argst list contains specifications
for the union of the verb's local phrasal dependents (the subject and complements, whether they are
semantic arguments, raised phrases, or expletives) and its arguments that are not realized locally,
whether they are unbounded dependents, affixes, or unexpressed
arguments.\is{feature!valence@\textsc{valence}!as distinct from \argst}\is{feature!arg-st@\argst!as
  distinct from \textsc{valence}}

Figure~\ref{fig:over} provides a schematic representation of %``typical'' 
linking and argument realization in HPSG,  illustrated with the verb \textit{donate}, as in
\textit{Mary donated her books to the library}.   Linking principles govern the mapping of
participant roles in a predicator's \content to %direct  
\is{mapping|see{linking, argument realization}} elements of the \argst list.   Argument realization
is shown in this figure only for mapping to \changed{valence}, which represents locally realized phrasal
dependents; affixal and null arguments are not depicted (but are discussed below). 
%Here, the semantic roles are just arbitrary labels, but we discuss in Section (\ref{linking-sec}) how they can be systematically related to lexical entailments of predicators.
The \argst and \changed{valence} lists in this figure contain only arguments linked to participant
roles,\is{participant roles|see{thematic role, semantic role, proto-role}}  but in
Section~\ref{sec:extended-arg-st} we discuss proposals for extending \argst to include additional
elements. In Section~\ref{argst-valence-sec}, we examine cases where the relationship between \argst
and \changed{valence} violates the ARP.

\begin{figure}\footnotesize
\begin{tabular}{@{}p{2.7cm}p{4.8cm}p{4cm}@{}}
	Semantics (\feat{content}) & \avm{
 [\type*{donate-rel}
 donor & i \\
 recipient & j \\
 theme & k ]
 } & Semantic relation denoted by a verb\\
  & \Large{\begin{tikzcd}
 {} \arrow[leftrightarrow]{d}{\,\,\,\text{\textit{Linking Principles}}}
               \\ 
               {}
\end{tikzcd} } &  \\
& \\
	Argument Structure\newline (\feat{arg-st}) & 
	\avm{
	[arg-st & < \1\,NP$_i$, \2\,NP$_k$, \3 PP![\type{to}]$_j$'! > ]	
 } & List of syntactic arguments of\newline the verb \\
 & \Large{\begin{tikzcd}
 {} \arrow[leftrightarrow]{d}{\,\,\,\text{\textit{Argument Realization Principles}}}
               \\ 
               {}
\end{tikzcd} } & \\
	Syntax (\feat{subj}/\feat{comps}) & \avm{
 [subj & < \1\,NP$_i$ > \\
 comps & < \2\,NP$_k$, \3 PP![\type{to}]$_j$! > ] 	
 }   
 & Lists of locally realized phrasal dependents
\end{tabular}
\caption{\label{fig:over}Linking and argument realization in HPSG, illustrated with the verb
  \textit{donate}} 
\itd{How can one change the type of the arrows?}
\end{figure}



\section{Argument realization: The mapping between \argst and \changed{valence} lists}
\label{argst-valence-sec}

\subsection{Variation in the expression of arguments}
\label{express-sec}

The \changed{valence features \subj and \comps are} responsible for composing a verb with its
\deleted{phrasal} dependents,
\itdopt{Stefan: You talk about phrasal dependents frequently. Dependents do not have to be
  phrasal. Adverbs can be used without projecting. Same is ture for pronouns. Complex formation
  requires lexical dependents as well. I suggest just dropping the ``phrasal'' throughout the
  paper. I mark the occurances in blue for deletion.}
but this is just one of the ways that semantic arguments of a verb are expressed in natural language.  Semantic arguments can be expressed in various linguistic forms: as local syntactic dependents (\subj and \comps), as affixes, or displaced in unbounded dependency constructions (\textsc{slash}). 

Affixal arguments can be illustrated with the first person singular \ili{Spanish} verb \textit{hablo} `speak.\textsc{1sg}', as in (\ref{hablo}).


\begin{exe} 
\ex	\label{hablo}
\begin{xlist}
\ex 		\gll Habl-o espa\~{n}ol.  \\
		speak-\textsc{1sg} \ili{Spanish}  \\
		\glt `I speak \ili{Spanish}.'
\ex \textit{hablo} `speak.\textsc{1sg}': \\*
\avm{
[\phon < ablo > \\
 subj & < > \\ 
 comps & < \2 >  \\ 
 arg-st & < NP:[\type*{ppro}
                index & [pers & 1st \\
	                 num  & sg] ], \2\,NP > ]
}
\end{xlist}
\end{exe}

\noindent
The \textit{-o} suffix contributes the first person singular pronominal subject content to the verb
form (the morphological process is not shown here; see \crossrefchapteralt{morphology}).  The
pronominal subject appears on the \argst list and hence is subject to the binding theory.  But it
does not appear in \subj, if no subject NP appears in construction with the verb.

A lexical sign whose \argst list is just the concatenation of its \subj and \comps lists conforms to
the Argument Realization Principle\is{Argument Realization Principle} (ARP); such signs are called
\word{canonical signs}\is{canonical sign} by \citet{Boumaetal2001}.  Non-canonical signs, which
violate the ARP, have been approached in two ways.  In one approach, a lexical rule takes as input a
canonical entry and derives a non-canonical one by removing items from the \changed{valence} lists, while adding
an affix or designating an item as an unbounded dependent by placement on the \textsc{slash}
list.\is{feature!valence@\textsc{valence}!as distinct from
  \argst}\is{feature!arg-st@\textsc{arg-st}!as distinct from \textsc{valence}} In the other
approach, a feature of each \argst list item specifies whether the item is subject to the ARP (hence mapped to
a \changed{valence} list), or ignored by it (hence expressed in some other way).  See \changed{\crossrefchapterw{lexicon}
for more detail on the lexicon} and \citet{MillerandSag1997} for a treatment of \ili{French} clitics as affixes.

A final case to consider is null anaphora, in which a semantic argument is simply left unexpressed
and receives a definite pronoun-like interpretation.  \ili{Japanese} \textit{mi-} `see' is
transitive but the object NP can be omitted as in (\ref{jap}).

\ea
\label{jap}
\gll Naoki-ga       mi-ta.  \\
     Naoki-\ig{nom} see-\ig{pst}  \\
\glt `Naoki saw it/him/her/*himself.'
\z

\noindent
Null anaphors of this kind typically arise in discourse contexts similar to those that license
ordinary weak pronouns, and the unexpressed object often has the obviation effects characteristic of
overt pronouns,\is{obviation} as shown in (\ref{jap}).  HPSG eschews the use of silent formatives
like ``small \textit{pro}'' when there is no evidence for such items, such as local interactions with
the phrase structure.  Instead, null anaphors of this kind are present in \argst but absent from
\changed{valence} lists.  \argst is directly linked to the semantic \content and is the locus of Binding Theory,
so the presence of a syntactic argument on the \argst list but not a \changed{valence} list accounts for null
anaphora.\is{ARG-ST} To account for obviation, the \argst list item, when unexpressed, receives the
binding feature of ordinary (non-reflexive) pronouns, usually \textit{ppro}.  This language-specific
option can be captured in a general way by \changed{valence} and \argst defaults in the lexical hierarchy for
verbs.

\subsection{The syntax of  \texorpdfstring{\argst}{ARG-ST} and its relation to  \changed{valence lists}}
\label{argst-sec}


The ordering of members of the \argst list represents a preliminary syntactic structuring of the set
of argument roles.  In that sense, \argst functions as an interface between the lexical semantics of
the verb and the expressions of dependents as described in Section~\ref{argst-valence-sec}.  Its
role thus bears some relation to the initial stratum in Relational Grammar
\citep{PerlmutterandPostal1984},\is{Relational Grammar} \textit{argument structure} (including
intrinsic classifications) in LFG Lexical Mapping Theory\is{Lexical Functional Grammar}
\citep{Bresnan+etal:2015}, macroroles in Role and Reference Grammar \citep{VanValinandLapolla1997},
D-structure in Government and Binding Theory, and the Merge positions of arguments in
Minimalism,\is{Minimalism} assuming in the last two cases the \changed{\isi{Uniformity of Theta Assignment Hypothesis}}
\citep[\added{46}]{Baker1988} or something similar.  However, it also differs from all of those in important
ways.


Semantic constraints on \argst are explored in Section~\ref{linking-sec} below.  But \argst represents not only
% is  structured not only by    
semantic distinctions between the arguments, but also  %by 
syntactic  ones.  Specifically, the list ordering represents relative syntactic \term{obliqueness}
of arguments.   The least oblique argument is the subject (\subj), followed by the complements
(\comps).  Following \citet{Manning1996}, term arguments\is{argument!term} (direct arguments, i.e., subjects and
objects) are  assumed to be less oblique than ``oblique'' arguments (adpositional and oblique case
marked phrases), followed finally by predicate and clausal complements.  The transitive ordering
relation on the \argst list is called \textit{o-command} (obliqueness
command):\is{o-command|see{anaphoric binding}}\is{anaphora|(} the list item that corresponds to the
subject o-commands those corresponding to complements; a list item corresponding to an object
o-commands \is{o-command|see{anaphoric binding}} those corresponding to any obliques; and so on (see
\crossrefchapteralt{binding} for details). 
 
% TD 18 Jan. 20: I think the following paragraph and examples could just be deleted.  They presuppose the non-canonical linking analysis we're skeptical of later, and they don't really seem to usefully illustrate or advocate anything.
% \is{alternation}
%Voice alternations like the passive \is{passive}  are sometimes defined on the \argst list and illustrate the ordering of terms before obliques on the \argst list.   Passivization, in some analyses that go back to \citet{pollard+sag:1987}, alters the syntactic properties of \argst list items: the initial item of the active, normally mapped to \subj of the active, is an oblique (\textit{by} phrase) or unexpressed argument in the passive.  Given that terms precede obliques in the list order, any term arguments must o-command the passive oblique, so passive effectively reorders the initial item in \argst to a list position following any terms.   
%JPK 1-15-20 I changed the wording in the paragraph above to reflect our more complex story about the passive

%\begin{exe}
%\ex \label{passive}
%\begin{xlist}
%\ex Susan gave Mary a book.
%\ex Mary was given a book by Susan.
% \end{xlist}
% \end{exe}

%\begin{exe}
%\ex \label{pasargst}
%\begin{xlist}
%\ex \textit{give} (active): $[ \textsc{arg-st}  < \textsc{np}_i, \textsc{np}_j, \textsc{np}_k > ]$
%\ex \textit{given} (passive): $[ \textsc{arg-st}  < \textsc{np}_j, \textsc{np}_k, \textsc{pp}[by]_i > ]$
% \end{xlist}
% \end{exe}

Relative obliqueness conditions a number of syntactic processes and phenomena, including anaphoric
binding.  The o-command relation replaces the c"=command in the Principles A, B, and C of Chomsky's
\citeyearpar{Chomsky:1981} configurational theory of binding.  For example, HPSG's Principle B
states that an ordinary pronoun cannot be o-commanded by its coargument antecedent, which accounts
for the pronoun obviation observed in the \ili{English} sentence \textit{Naoki$_i$ saw
  him$_{*i/j}$}, and also accounts for obviation in the \ili{Japanese} sentence (\ref{jap}) above.

Relative obliqueness also conditions the \isi{accessibility hierarchy} of
\citet{KeenanandComrie1977}, according to which a language allowing relativization of some type of
dependent also allows relativization of any \added{dependent} less oblique than it.  Hence if a language has relative
clauses at all, it has subject relatives; if it allows obliques to relativize, then it also allows
subject and object relatives; and so on. Similar implicational universals apply to verb agreement
with subjects, objects, and obliques\is{oblique argument} \citep{greenberg:1966}.\is{anaphora|)} 

Returning now to argument realization, we saw above that the rules for the selection of the subject
from among the verb's arguments are also stated in terms of the \argst list.  In a canonical\
realization the subject is the first list item, o-commanding all of its coarguments.
% TD 10 Jan. 20: added the following to transition to the next section
In various non-canonical circumstances, such as those we noted above, o-command relations do not
correspond to ordering on the valence lists, and this can be reflected in phenomena such as
anaphoric binding.  In the following section we examine another kind of non-canonical relationship
between \argst and \changed{valence} in more detail: syntactic ergativity, exemplified by \ili{Balinese}.

\subsection{Syntactic ergativity}
\label{ergativity}\label{arg-st-sec-ergativity}

\is{ergativity|(} The autonomy of \argst from the \changed{valence} lists is further illustrated by
cross"=linguistic variation in the mapping between them.  As just noted, in \ili{English} and many
other languages, the initial item in \argst maps to the subject.  However, languages with so-called
\emph{syntactically ergative} clauses have been analyzed as following a different mapping rule.
Crucially, the \argst ordering in those languages is still supported by independent evidence from
properties such as binding and NP versus PP categorial status of arguments.  \ili{Balinese}
(\ili{Austronesian}), as analyzed by \citet{Wechsler+Arka:1998}, is such a language.  In the
morphologically unmarked, and most common voice, called \emph{Objective Voice}\is{voice!objective}
(\feat{ov}), the subject is any term \textit{except} the \argst-initial one.

\il{Balinese|(}\ili{Balinese} canonically has SVO order, regardless of the verb's voice form
\citep{Artawa1994, Wechsler+Arka:1998}.  The preverbal NPs in (\ref{bal1}) are the surface subjects
and the postverbal ones are complements.  When the verb appears in the unmarked objective voice
(\feat{ov}), a non-initial term is the subject, as in (\ref{bal1-a}).  But verbs in the
\emph{Agentive Voice}\is{voice!agentive} (\feat{av}) select as their subject the \argst{}-initial
item, as in (\ref{bal1-b}).

\begin{exe}
	\ex\label{bal1}
\begin{xlist}
\ex \label{bal1-a}	
\gll Bawi adol ida.  \\
     pig \textsc{ov}.sell \textsc{3sg}   \\
\glt `He/She sold a pig.'
\ex\label{bal1-b}
\gll Ida ng-adol bawi.  \\
     \textsc{3sg} \textsc{av}-sell pig   \\
\glt `He/She sold a pig.'
\end{xlist}
\end{exe} 

\noindent
A ditransitive verb,\is{ditransitive} such as the benefactive applied form of \textit{beli} `buy' in
(\ref{bal2}), has three term arguments on its \argst list.  The subject can be either term that is
non-initial in \argst{}:

\eal
\label{bal2}
\ex
\gll Potlote             ento beli-ang=a                      I            Wayan.  \\
     pencil.\textsc{def} that \textsc{ov}.buy-\textsc{appl}=3 \textsc{art} Wayan   \\
\glt `(S)he bought Wayan the pencil.'
\ex
\gll I            Wayan beli-ang=a                      potlote             ento.   \\
     \textsc{art} Wayan \textsc{ov}.buy-\textsc{appl}=3 pencil.\textsc{def} that   \\
\glt `(S)he bought Wayan the pencil.'
\zl

\noindent
Wechsler and Arka argue that \ili{Balinese} voice alternations\is{alternation} do not affect \argst
list order.  Thus the agent argument can bind a coargument reflexive pronoun (but not vice versa),
regardless of whether the verb is in OV or AV form:

\eal
\label{bal3}
\ex
\gll Ida          ny-ingakin      ragan idane. \\
     \textsc{3sg} \textsc{av}-see \textsc{self}\\
\glt ‘(S)he saw himself/herself\add{.}’
\ex
\gll {Ragan idane} cingakin        ida. \\
     \textsc{self} \textsc{ov}.see \textsc{3sg} \\
\glt ‘(S)he saw himself/herself\add{.}’
\zl

\noindent
The `seer' argument o-commands\is{o-command|see{anaphoric binding}} the `seen', with the AV versus
OV voice forms regulating subject selection:\is{anaphora}

\begin{exe} 
	\label{avsee}
\ex	Agentive Voice form of `see': \\
\avm{
[\phon < nyinkagin > \\
 subj   & < \1 > \\ 
 comps  & < \2 > \\ 
 arg-st & < \1\,NP$_i$, \2\,NP$_j$ > \\
 content & [\type*{see-rel}
	    seer & i \\
	    seen & j ] ]
}
\end{exe}

\begin{exe} 
	\label{ovsee}
\ex	Objective Voice form of `see': \\
\avm{
[\phon < cinkagin > \\
 subj & < \2 > \\ 
 comps & < \1 > \\ 
 arg-st & < \1\,NP$_i$, \2\,NP$_j$ > \\
content &	[\type*{see-rel}
			seer & i \\
			seen & j ] ]
}
\end{exe}

\noindent
Languages like \ili{Balinese} illustrate the autonomy of \argst.  Although the agent binds the
patient in both (\ref{bal3}a) and (\ref{bal3}b), the binding conditions cannot be stated directly on
the\is{thematic hierarchy|see{thematic roles}} thematic hierarchy.  For example, in HPSG a raised
argument appears on the \argst list of the raising verb, even though that verb assigns no thematic
role\is{thematic role|see{semantic role, participant role, thematic hierarchy}} to that list item.
But a raised subject can bind a coargument reflexive in \ili{Balinese} (this is comparable to
\ili{English} \textit{John seems to himself to be ugly}).  Anaphoric binding in \ili{Balinese}
raising constructions thus behaves as predicted by the \argst based theory \citep{Wechsler1999}.  In
conclusion, neither \changed{valence lists} nor \content provides the right representation for defining binding
conditions, but \argst fits the bill.


Syntactically ergative languages besides \ili{Balinese} that have been analyzed as using an
alternative mapping between \argst and \changed{valence} include \ili{Tagalog}, Inuit, some Mayan languages,
Chukchi, \ili{Toba Batak}, Tsimshian languages, and Nad{\"e}b \citep{Manning1996,Manning+Sag:1999}.

Interestingly, while the GB/Minimalist configurational binding theory may be defined on analogues of
\changed{the valence lists} or \content, those theories lack any analogue of \argst.  This leads to special problems for
such theories in accounting for binding in many \ili{Austronesian} languages like \ili{Balinese}.
In transformational theories since \citet{Chomsky:1981}, anaphoric binding conditions are usually
stated with respect to the A-positions (argument positions).  A-positions are analogous to HPSG
\changed{valence} list items, with relative c"=command in the configurational structure corresponding to relative
list ordering in HPSG, in the simplest cases.  Meanwhile, to account for data similar to
(\ref{bal3}), where agents asymmetrically bind patients, \ili{Austronesian} languages like
\ili{Balinese} were said to define binding on the ``thematic structure'' encoded in d-structure or
Merge positions, where agents asymmetrically c"=command patients regardless of their surface
positions \citep{Guilfoyle+etal:1992}.  But the interaction with raising shows that neither of those
levels is appropriate as the locus of binding theory \citep{Wechsler1999}.\footnote{To account for
  (\ref{bal3}b) under the configurational binding theory, the subject position must be an A-bar
  position, but to account for binding by a raised subject, it must be an A-position.  See
  \citet{Wechsler1999}. }
%
\il{Balinese|)}
\is{ergativity|)}

\subsection{Symmetrical objects}
\is{symmetrical object}

We have thus far tacitly assumed a total ordering of elements on the \argst list, but
\citet*{AMM2013a,Ackermanetal2017} propose a partial ordering \is{ARG-ST!partial ordering
  on} for certain so-called \emph{symmetrical object languages}.  In \ili{Moro} (Kordofanian), the
two term complements of a ditransitive\is{ditransitive} verb have exactly the same object
properties.  Relative linear order of the theme and goal arguments is free, as shown by the two
translations of (\ref{moro}) (from \citealt[9]{Ackermanetal2017}; \feat{cl} `noun class'; \feat{sm}
`subject marker).


\begin{exe}
	\ex\label{moro}
\gll   \'{e}-g-a-nat\textipa{S}-\'{o} \'{o}r\'{a}\textipa{N}  \textipa{N}e\textipa{R}\'{a}  \\
        1\textsc{sg.sm-cl}g-\textsc{main}-give-\textsc{pfv}    \textsc{cl}g.man \textsc{cl}\textipa{N}.girl \\
\glt `I gave the girl to the man.’ / `I gave the man to the girl.’
\end{exe} 

\noindent
More generally, the two objects have identical object properties with respect to occurrence in
post-predicate position, case marking, realization by an object marker, and ability to undergo
passivization \citep[9]{Ackermanetal2017}.

\citet{Ackermanetal2017} propose that the two objects are unordered on the \argst list.  This allows
for two different mappings to the \comps list, as shown here:

\begin{exe} 
\ex		\label{moro-avm1}
\begin{xlist}
\ex Goal argument as primary object: \\
\avm{
[subj & < \1 > \\ 
 comps & < \2, \3 > \\ 
arg-st & <  \1\,NP$_i$, \{ \2\,NP$_j$, \3\,NP$_k$ \} > \\
content &	[ \type*{give-rel}
			agent & i \\
			goal & j \\
			theme & k ] ]
}
\ex Theme argument as primary object: \\
\avm{
[subj & < \1 > \\ 
 comps & < \3, \2 > \\ 
arg-st & < \1\,NP$_i$, \{ \2\,NP$_j$, \3\,NP$_k$ \} > \\
content &	[\type*{give-rel}
			agent & i \\
			goal & j \\
			theme & k ] ]
}
\end{xlist}
\end{exe}

\noindent
The primary object properties, which are associated with the initial term argument of \comps, can go
with either the goal or theme argument.

To summarize this section, while the relationship between \argst, \subj, and \comps lists was
originally conceived as a straightforward one, enabling binding principles to maintain their simple
form by defining \argst as the concatenation of the other two, the relationship was soon loosened.
%Non-canonical 
Looser relationships between \argst and the \changed{valence} lists are invoked in accounts of several core
syntactic phenomena.  Arguments not realized overtly in their canonical positions due to extraction,
cliticization, or pro-drop (null anaphora)\is{anaphor!null|see{obviation}} appear on \argst but not
in any \changed{valence} list.  Accounts of syntactic ergativity in HPSG involve variations in the mapping
between \argst and \changed{valence} lists; in particular, the element of \subj is not, in such languages, the
first element of \argst.  Modifications of \argst play a role in some treatments of passivization,
where its expected first element is suppressed, and in languages with multiple, symmetric objects,
where a partial rather than total ordering of \argst elements has been postulated (see
Section~\ref{passives} for details on the analysis of passives in HPSG).  Thus \argst has now
acquired an autonomous % independent 
status within HPSG, and is not merely a predictable rearrangement of information present in the
valence lists.   
%present elsewhere in lexical entries.


\section{Linking: the mapping between semantics and \argst}
\label{linking-sec}
\is{ARG-ST!as locus of linking|see{linking}|(}

\subsection{HPSG approaches to linking}
\label{arg-st:sec-hpsg-approaches-to-linking}

The term \textit{linking} refers to the mapping specified in a lexical entry between participant
roles\is{participant roles|see{thematic role, semantic role, proto-role}} in the semantics and their
syntactic representations on the \argst list.\itdopt{Stefan: Maybe misleading since it also contains
  semantic information.} Early HPSG grammars stipulated the linking of each
verb: semantic \content values with predicator-specific attributes like \textsc{devourer} and
\textsc{devoured} were mapped to the subject and object, respectively, of the verb \textit{devour}.
But linking follows general patterns across verbs, and across languages; e.g., if one argument of a
transitive verb in active voice has an agentive role, it will map to the subject, not the object,
except in syntactically ergative languages described in Section~\ref{ergativity} above, and in those
languages the linking is just as regularly reversed.  Those early HPSG grammars did not capture the
regularities across verbs.
%Thus these early accounts were regarded as unsatisfying, because they lead to purely stipulative accounts of linking, specified verb by verb.  
%  ALERT: ok to remove this sentence?  I'd rather not be so negative.  also I actually think the 'unsatisfying' account is right.  :)  i.e. I think innovation involves a clustering algorithm that leads to patterns; no grammar rules are needed to explain those patterns.  
% TD 28 Jan 20: I have reworded the sentence to accommodate that viewpoint, but I think it should remain in some form as an explanation of why more general constraints were sought. 
%JPK 1-29-20 I agree with Tony's rewording.
%SW 1-29-20  How about the new sentence above?  It still says why more general constraints were sought, but doesn't involve a statement that I disagree with, in the sense that "purely stipulative" is a pejorative in our field.  I don't agree with the view that these regularities should be captured in the competence grammar, so to me those 'stipulations' are good.  I'm not insisting you agree with me on that, just that we say things in a neutral way.  

To capture those regularities, HPSG researchers beginning with \citet{Wechsler1995b} and
\citet{Davis1996} formulated linking principles stated on more general semantic properties that hold
across verbs.

Within the history of linguistics, there have been three general approaches to modeling the
lexico-semantic side of linking: thematic role types\is{thematic role|see{semantic role, participant
    role, thematic hierarchy}} (P\={a}\d{n}ini ca.\ 400 B.C., \citealt{Fillmore1968}); lexical
decomposition\is{lexical decomposition} \citep{FoleyandvanValin1984,RappaportandLevin1998}; and the
proto-roles approach\is{proto-role|see{thematic role, semantic role, participant role}}
\citep{Dowty1991}.  In developing linking theories within the HPSG framework, \citet{Wechsler1995b}
and \citet{Davis1996} employed a kind of lexical decomposition that also incorporated some elements
of the proto-roles approach.  The reasons for preferring this over the alternatives are discussed in
Section~\ref{thetaroles} below.

\is{linking!and meaning|(} 
Wechsler's (\citeyear{Wechsler1995b}) linking theory constrains the
relative order of pairs of arguments on the \argst list according to semantic relations entailed
between them.  For example, his \emph{notion rule}\is{Notion Rule}\is{experiencer predicates} states
that if one participant in an event is entailed to have a mental notion of another, then the first
must precede the second on the \argst list.  The \textit{conceive-pred} type is defined by the
following type declaration (based on \citealt[127]{Wechsler1995b}, with formal details adjusted for
consistency with current usage):

\begin{exe}
\ex\label{conceive}
\textit{conceive-pred}:\\  
\avm{
	[arg-st &  < NP$_i$, NP$_j$ > \\
	content &	[\type*{conceive-rel}  
				conceiver & i \\
				conceived & j ] ]
}
\end{exe}

This accounts for a host of linking facts in verbs as varied as \word{like}, \word{enjoy},
\word{invent}, \word{claim}, and \word{murder}, assuming these verbs belong to the type
\textit{conceive-pred}.  It explains the well-known contrast between experiencer-subject \word{fear}
and experiencer"=object \word{frighten} verbs: \word{fear} entails that its subject has some notion
of its object, so \word{The tourists feared the lumberjacks} entails that the tourists are aware of
the lumberjacks.  But the object of \word{frighten} need not have a notion of its subject: in
\word{The lumberjacks frightened the tourists (by cutting down a large tree that crashed right in
  front of them)}, the tourists may not be aware of the lumberjacks' existence.

Two other linking rules appear in \citet{Wechsler1995b}.  One states that ``affected themes'', that
is, participants that are entailed to undergo a change, map to the object, rather than subject, of a
transitive verb.  Another states that when stative transitive verbs entail a part-whole relation
between the two participants, the whole maps to the subject and the part to the object: for example,
\textit{X includes Y} and \textit{X contains Y} each entail that \textit{Y} is a part of \textit{X}.

These linking constraints do not rely on a total ordering of thematic roles, nor on an exhaustive
asssignment of thematic role types to every semantic role in a predicator. Instead, a small set of
partial orderings of semantic roles, based on lexical entailments,\is{entailments!lexical} suffices
to account for the linking patterns of a wide range of verbs.  This insight was adopted in a
slightly different guise in work by \citet{Davis1996}, \citet{Davis2001}, and
\citet{DavisandKoenig2000b}, who develop a more elaborated representation of lexical semantics, with
which simple linking constraints can be stated.  The essence of this approach is to posit a small
number of dyadic semantic relations such as \textit{act-und-rel} (actor-undergoer relation) with
attributes \feat{act(or)}\isfeat{act(or)} and \feat{und(ergoer)}\isfeat{under(goer)} that serve as
intermediaries between semantic roles and syntactic arguments (akin to the notion of Generalized
Semantic Roles discussed in \citealt{VanValin1999}).

What are the truth conditions of \textit{act-und-rel}?  Following \citet{Fillmore1977},
\citet{Dowty1991}, and \citet{Wechsler1995b}, \citeauthor{DavisandKoenig2000b} note that many of the pertinent
lexical entailments come in related pairs.  For instance, one of Dowty's entailments is that one
participant causally affects another, and of course the other is entailed to be causally affected.
Another involves the entailments in Wechsler's notion rule (\ref{conceive}); one participant is
entailed to have a notion of another.  These entailments of paired participant types characterize
classes of verbs (or other predicators), and can then be naturally represented as dyadic relations
in \feat{content}.  Collecting those entailments, we arrive at a disjunctive statement of truth
conditions:

\eanoraggedright
\label{def-act-und-rel}
\textit{act-und-rel}($x,y$) is true iff $x$ causes a change in $y$, or $x$ has a notion of $y$, or \ldots
\z

\noindent
We can designate the $x$ participant in the pair as the value of \feat{actor} (or \feat{act}) and
$y$ as the value of \feat{undergoer} (or \feat{und}), in a relation of type \type{act-und-rel}.
Semantic arguments that are \feat{actor} or \feat{undergoer} will then bear at least one of the
entailments characteristic of \feat{actor}s or \feat{undergoer}s
\citep[72]{DavisandKoenig2000b}. This then simplifies the statement of linking constraints for all
of these paired participant types.  \citet{Davis1996} and \citet{KoenigandDavis2001} argue that this
obviates counting the relative number of proto-agent and proto-patient entailments, which is what
\citet{Dowty1991} had advocated.

The linking constraints (\ref{act-vb-linking}) and (\ref{und-vb-linking}) state that 
a verb whose semantic \content is of type \emph{act-und-rel} will be constrained to link the
\feat{act} participant to the the first element of the verb's \argst list (its subject), and the
\feat{und} participant to the second element of the verb's \argst list (this is analogous to
Wechsler's constraints based on partial orderings).\itdobl{The way it is formalized now, the undergoer can go to any of the arguments.
}  The attribute \feat{key}\isfeat{key} selects
one predication as relevant for linking, among a set of predications included in a lexical item's
\feat{content}; we furnish more details below. 
%JPK 1-13-2020 I added a sentence to briefly defined KEY.
% TD 13 Jan. 20: modified a bit for clarity

These linking constraints can be viewed as parts of the definition of lexical types, as in
\citet{Davis2001}, where each of the constraints in
(\ref{act-vb-linking})--(\ref{emb-act-vb-linking}) defines a particular class of lexemes (or
words).\footnote{Alternatively, (\ref{act-vb-linking}) (and other linking constraints) can be recast
  as implicational constraints on lexemes or words \citep{KoenigandDavis2003}.
  (\ref{act-vb-linking-alt}) is an implicational constraint indicating that a word whose semantic
  content includes an \textsc{actor} role must map that role to the initial item in the \argst list.

\begin{exe}
\ex\label{act-vb-linking-alt}
\avm{
[content|key &	[act & \1 ] ]} \impl \avm{[arg-st & < NP$_{\1}$, \ldots > ]
}
\end{exe} 
}   

\begin{exe}
	\ex\label{act-vb-linking}
	\avm{
		[\punk{content|key}{[act & \1 ]} \\
		 arg-st & < NP$_{\1}$, \ldots > ]
	}
\end{exe}

\begin{exe}
	\ex\label{und-vb-linking}
	\avm{
		[\punk{content|key}{[und & \2 ]} \\
		arg-st & < \ldots, NP$_{\2}$, \ldots > ]
	}
\end{exe}

\begin{exe}
	\ex\label{emb-act-vb-linking}
	\avm{
		[content|key &	[\type*{cause-possess-rel} 
						soa & [act & \3 ] ] \\
		arg-st & < \type{synsem} > \+ < NP$_{\3}$, \ldots > ]
	}
\end{exe}


\noindent
The first constraint, in (\ref{act-vb-linking}), links the value of \feat{act} (when not embedded
within another attribute) to the first element of \argst.  The second, in (\ref{und-vb-linking}),
merely links the value of \feat{und} (again, when not embedded within another attribute) to some NP
on \argst.  Given this understanding of how the values of \feat{act} and \feat{und} are determined,
these constraints cover the linking patterns of a wide range of transitive verbs: \word{throw}
(\feat{act} causes motion of \feat{und}), \word{slice} (\feat{act} causes change of state in
\feat{und}), \word{frighten} (\feat{act} causes emotion in \feat{und}), \word{imagine} (\feat{act}
has a notion of \feat{und}), \word{traverse} (\feat{act} ``measures out'' \feat{und} as an
incremental theme), and \word{outnumber} (\feat{act} is superior to \feat{und} on a scale).

The third constraint, in (\ref{emb-act-vb-linking}), links the value of an \feat{act} attribute
embedded within a \feat{soa} (state of affairs)\is{feature!soa@\textsc{soa} (State of Affairs)}
attribute to an NP that is second on \argst.  This constraint accounts for the linking of the
(primary) object of ditransitives.\is{ditransitive} In \ili{English}, these verbs (\word{give},
\word{hand}, \word{send}, \word{earn}, \word{owe}, etc.) involve (prospective) causing of possession
\citep{Pinker1989,Goldberg1995}, and the possessor is represented as the value of the embedded
\feat{act} in (\ref{emb-act-vb-linking}).  There could be additional constraints of a similar form
in languages with a wider range of ditransitive constructions; conversely, such a constraint might
be absent in languages that lack ditransitives entirely.  As mentioned earlier in this section, the
range of subcategorization options varies somewhat from one language to another.

The \feat{key} attribute in (\ref{act-vb-linking})--(\ref{emb-act-vb-linking}) also requires further
explanation.\isfeat{key} The formulation of linking constraints here employs the architecture used
in \citet{KoenigandDavis2006}, in which the semantics represented in \content values is expressed as
a set of \emph{elementary predications}, formalized within Minimal Recursion Semantics
\citep{Copestakeetal2001,Copestakeetal2005}.  Each elementary predication is a simple relation, but
the relationships among them may be left unspecified.  For linking, one of the elementary
predications is designated the \feat{key}, and it serves as the locus of linking.  This allows us to
% sidestep thorny questions of lexical semantics and linking, such as 
indicate the linking of participants that play multiple roles in the denoted situation. 
%, or whether all aspects of a particpant's involvement in an situation type are properly represented in \content.
The \feat{key} selects one relation as the ``focal point,'' and the other elementary predications
are then irrelevant as far as linking is concerned. The choice of \feat{key} then becomes an issue
demanding consideration; we will see in the discussion of argument alternations\is{alternation} in
Section~\ref{alternations} how this choice might account for some alternation phenomena. 
\is{linking!and meaning|)}

%Note too that these linking constraints are treated as constraints on classes in the lexical hierarchy
These linking constraints apply to word classes in the lexical hierarchy (see
\crossrefchapteralt{lexicon}).  One consequence of this fact merits brief mention.  Constraint
(\ref{und-vb-linking}), which links the value of \feat{und} to some NP on \argst, is a specification
of one class of verbs.  Not all verbs (and certainly not all other predicators, such as
nominalizations) with a \content value containing an \feat{und} value realize it as an NP.  Verbs
obeying this constraint include the transitive verbs noted above, and intransitive ``unaccusative''
verbs such as \word{fall} and \word{persist}.  But some verbs with both \feat{act} and \feat{und}
attributes in their \content are intransitive, such as \word{impinge (on)}, \word{prevail (on)}, and
\word{tinker (with)}.  Interactions with other constraints, such as the requirement that verbs (in
\ili{English}, at least) have an NP subject, determine the range of observed linking patterns.

These linking constraints also assume that the proto-role\is{proto-role|see{thematic role, semantic
    role, participant role}} attributes \feat{actor}, \feat{undergoer}, and \feat{soa} are
appropriately matched to entailments, as described above.  Other formulations are possible, such as
that of \citet{KoenigandDavis2003}, where the participant roles\is{participant roles|see{thematic
    role, semantic role, proto-role}} pertinent to each lexical entailment are represented in
\content by corresponding, distinct attributes.

In addition to the linking constraints, there may be some very general well-formedness conditions on
linking. We rarely find verbs that obligatorily map  one semantic role to two  distinct members of
the \argst list, both expressed overtly.  A verb meaning `eat', but  with that disallowed property,
could appear in a ditransitive sentence like  (\ref{rah-a}), with the meaning that Pat ate dinner,
and his dinner was a large steak.

\begin{exe}
\ex[*]{\label{rah-a} 
Pat ate dinner a large steak.
}
\end{exe}

\noindent
Typically, semantic arguments map to at most one (overtly expressed) \argst list item \citep[262--268]{Davis2001}.

% TD 10 Jan. 20: added the following:
% JPK 13 Jan.2020: slightly reworded
Having set out some general principles of linking and their implementation in HPSG, we now briefly discuss linking of oblique arguments.\is{oblique argument}
We also return in the remainder of this section to issues relating to lexical semantic representations as they pertain to linking.
To what extent are the elements of \argst determined by lexical semantics?
Do HPSG lexical semantic representations require thematic roles?
And how does other information in these representations, such as modality and modifier scope, affect linking?
\is{ARG-ST!as locus of linking|see{linking}|)}

\subsection{Linking oblique arguments}
\is{oblique argument}
%We have not yet discussed 
In this section we discuss linking of oblique arguments, that is, PPs and oblique case marked NPs.
In some instances, a verb's selection of a particular preposition appears at least partly arbitrary;
it is hard to explain why \ili{English} speakers accept \word{hanker after} and \word{yearn for}, but  not \word{*yearn after}.
In these cases, the choice of preposition may be stipulated by the individual lexical entry.
But as \citet{Gawron1986} and \citet{Wechsler1995} have shown, many prepositions selected by a verb have semantic content.
\word{For} in the above-mentioned cases, and in \word{look for}, \word{wait for}, and \word{aim for}, is surely not a lexical accident.
And in  cases like \word{cut with}, \word{with} is used in an instrumental sense, denoting a \type{use-rel} relation, as with verbs that either allow (\word{eat}) or require (\word{cut}) an instrument \citep{KoenigandDavis2006}.
\changed{\citet{Davis1996,Davis2001} adopts} the position of Gawron and Wechsler in his treatment of linking to PPs.
As an example of this kind of account, the linking type in (\ref{with-linking}) characterizes a verb selecting a \word{with}-PP. 
The PP argument is linked from the \rels list\isfeat{rels} rather than the \feat{key}.\isfeat{key}

\begin{exe}\ex\label{with-linking}
\avm{
[content &	[key & \1 \\
			rels & < \1, \2	[\type*{use-rel} \\
                                    act & a \\
                                    und & u  \\
                                    soa & s ], 
			\ldots > ] \\
arg-st & < \ldots, PP$_{with}$:\2\added{,} \ldots > ]
}
\itdopt{no sharing of a, u, s?}
\end{exe} 


Apart from the details of individual linking constraints, we have endeavored here to describe how linking can be modeled in HPSG using the same kinds of constraints used ubiquitously in the framework.
Within the hierarchical lexicon (see \crossrefchapteralt{lexicon}), constraints between semantically defined classes and syntactically defined ones can furnish an account of linking patterns, and there is no resort to additional mechanisms such as a thematic hierarchy\is{thematic hierarchy|see{thematic roles}} or numerical comparison of entailments.\is{entailments!lexical}



\subsection{To what extent does meaning predict linking?}
\is{linking!and meaning|(}

The framework outlined above allows us to address the following question: how much of linking is strictly determined by semantic factors, and how much is left open to 
lexically arbitrary subcategorization specifications, or perhaps subject to other factors?

Subcategorization --~the position and nature of \argst elements, in HPSG terms~-- is evidently driven to a great extent by semantics,
but debate continues about how much, and which components of semantics are involved.
%Taking a viewpoint on one end of the decompositional spectrum, there are many elements of lexical semantics that could appear on \argst.
%For example, denominal verbs like \word{pocket}, \word{chisel}, and \word{saddle}; verbs of sound emission like \word{laugh} and \word{rumble}, and verbs with entailed participants like \word{spit}, \word{bleed}, and \word{urinate} all involve participant types that typically are syntactically unrealized.
%Reviving syntactic representations from Generative Semantics, some  Minimalist accounts have advocated exactly that (see the chapter on HPSG and Minimalism for more discussion) and so do linking theories based on the kind of conceptual structure developed in \citet{Jackendoff1990}, to some extent .
%Within HPSG too, there has been an increasing tendency to include ``non-core'' semantic participants--- adjuncts, adverbials, secondary predications, and implicit arguments such as null anaphors--- on the \argst list.
%We will examine this in more detail in Section~\ref{extended-arg-st}.
%Even with regard to core, syntactically realized, arguments, we can ask how closely subcategorization mirrors lexical semantics.
%Here as well, 
Views have ranged from the strict, highly constrained relationship in which lexical semantics essentially determines syntactic argument structure to a looser one in which some elements of subcategorization may be stipulated.
Among  the first camp are those who espouse the \isi{Uniformity of Theta Assignment Hypothesis}  proposed in \citet[46]{Baker1988}\add{,} which maintains that ``identical thematic relationships between items are represented by identical structural relationships'' in the syntax (see also \citealt{Baker1997}).
With regard to the source of\is{alternation!diathesis} diathesis alternations,
\citet[12--13]{Levin1993} notes that ``studies of these properties suggest that argument structures
might in turn be derivable to a large extent from the meaning of words'', and accordingly ``pursues
the hypothesis of semantic determinism seriously to see just how far it can be taken''. 

Others, including \citet[Section~5.3]{pollard+sag:1987} and \citet[Section~5.1]{Davis2001}, have
expressed caution, pointing out cases where subcategorization and diathesis
alternations\is{alternation!diathesis} seem to be at least partly arbitrary. \citet[ex.\ 214--215]{pollard+sag:1987} note contrasts like these:

\eal
\ex    Sandy spared/*deprived Kim a second helping.
\ex    Sandy *spared/deprived Kim of a second helping.
%\citep[ex.\ 214--215]{pollard+sag:1987}
\zl

\noindent
And \citet[ex.\ 5.4]{Davis2001} provides these pairs of semantically similar verbs with differing subcategorization requirements:

\eal
\label{ard-subcat-ex}
\ex    Few passengers waited for/awaited the train.
\ex    Homer opted for/chose a chocolate frosted donut.
\ex    The music grated on/irritated the critics.
%\citep[ex.\ 5.4]{Davis2001}
\zl

\noindent
Other cases where argument structure seems not to mirror semantics precisely include raising
constructions, in which one of a verb's direct arguments bears no semantic role to it at all.
Similarly, overt expletive arguments cannot be seen as deriving from some participant role in a
predicator's semantics.  Like the examples above, these phenomena suggest that some aspects of
subcategorization are specified independently of semantics.

Another point against strict semantic determination of argument structure comes from cross"=linguistic observations of subcategorization possibilities.
It is evident, for example, that not all languages display the same range of direct argument mappings.
Some lack ditransitive\is{ditransitive} constructions entirely (Halkomelem), some allow them across a limited semantic range (\ili{English}), some quite generally (\ili{Georgian}), and a few permit tritransitives (Kinyarwanda and \ili{Moro}).
\citet{Gerdts1992} surveys about twenty languages and describes consistent patterns like these.
The range of phenomena such as causative and applicative formation in a language is constrained by what she terms its ``relational profile''; this includes, in HPSG terms, the number of direct NP arguments permitted on its \argst lists.
Again, it is unclear that underlying semantic differences across languages in the semantics of verbs meaning \word{give} or \word{write} would be responsible for these general patterns.

%Summarizing, there is much evidence tempting us to derive the contents of \argst solely from lexical semantics.
%If this ultimately proves feasible, then \argst serves more as a convenient interface notion with little possibility of independently expressing strictly syntactic aspects of subcategorization.
%This view, however satisfying it might be, does not accord with our current best understanding of the syntactic and semantic evidence.
%In the following sections we delve into some of the nuances that make linking more than a simple rendering of lexical semantics.
%We begin by noting a point on which HPSG accounts of linking differ from many others --- the absence of traditional thematic roles.
%ALERT:   I don't understand this paragraph, so I provisionally commented it out.   
% TD 28 Jan. 20: OK. I'm not strongly attached to this paragraph and I suspect it's there partly as a transition to what's next.  If you don't feel that's needed, let's junk it.
%JPK 1-29-20 OK with junking the paragraph. 

\is{linking!and meaning|)}
\subsection{HPSG and thematic roles}
\label{thetaroles}

\is{thematic role|see{semantic role, participant role, thematic hierarchy}|(}

The \textsc{arg-st} list constitutes the syntactic side of the mapping between semantic roles and syntactic dependents.  As \argst is merely an ordered list of arguments, without any semantic ``labels'', it contains no counterparts to thematic role types, such as \textsc{agent}, \textsc{patient}, \textsc{theme}, or \textsc{goal}.  Thematic roles like these, however, have been a mainstay of linking in Generative Grammar since \citet{Fillmore1968} and have antecedents going back to P\={a}\d{n}ini.
Ranking them in a\is{thematic hierarchy|see{thematic roles}} \emph{thematic hierarchy}, and labeling each of a predicator's semantic roles with a unique thematic role, then yields an ordering of roles analogous to the ordering on the \argst list.  Indeed, it would not be difficult to import this kind of system into HPSG, as a means of determining the order of elements on the \argst list.  However, HPSG researchers have generally avoided using a thematic hierarchy, for reasons we now briefly set out.

\citet{Fillmore1968} and many others thereafter have posited a small set of disjoint thematic roles,
with each semantic role of a predicator assigned exactly one thematic role.
\itdopt{What is ``semantic role''? I always thought it is the same as ``thematic role''? Do you mean
  ``each semantic argument''?}
Thematic hierarchies depend on these properties for a consistent linking theory, but they do not hold up well to formal scrutiny.
\citet{Jackendoff1987} and \citet{Dowty1991} note (from somewhat different perspectives) that
numerous verbs have arguments not easily assigned a thematic role from the typically posited
inventory (e.g., the objects of \word{risk}, \word{blame}, and \word{avoid}), that more than one
argument might sensibly be assigned the same role (e.g., the subjects and objects of
\word{resemble}, \word{border}, and some alternants of commercial transaction verbs), and that
multiple roles can be sensibly assigned to a single argument (the subjects of verbs of volitional
motion\itdopt{add example ``like \emph{add some verb}''} are like both an \textsc{agent} and a \textsc{theme}).
In addition, consensus on the inventory of thematic roles has proven elusive, and some, notoriously \textsc{theme}, have resisted clear definition.
Work in formal semantics, including \citet{LadusawandDowty1988}, \citet{Dowty1989}, \citet{Landman2000}, and \citet{Schein2002}, casts doubt on the prospects of assigning formally defined thematic roles to all of a predicator's arguments, at least in a manner that would allow them to play a crucial part in linking.
Thematic role types seem to pose problems, and there are alternatives that avoid those problems.  As \citet{Carlson1998} notes about thematic roles: ``It is easy to conceive of how to write a lexicon, a syntax, a morphology, a semantics, or a pragmatics without them''.

%
%Moreover, it is not clear how an ordered hierarchy of thematic roles could be modeled elegantly within HPSG's typed feature structure formalism, though it is certainly not impossible to achieve in some fashion.  Equally unclear is how to compute and compare total numbers of proto-agent and proto-patient entailments for each of a predicator's semantic roles, as \citet{Dowty1991} suggests, because no mechanisms are available for numerical calculations. As a constraint-based formalism, HPSG has led researchers to seek other ways of incorporating insights from these approaches into the framework.
\is{thematic role|see{semantic role, participant role, thematic hierarchy}|)}

\itdopt{Stefan: While working on other chapters, I tried to link here for actor and undergoer. I
  read this section for this and wonder what it is supposed to tell me. What is HPSG's stand on
  thematic roles. JP told me to use ACTOR and UNDERGOER. So I expected here an explanation why these
  generalized rules are the way to go.\\
At least add a sentence leading into the next section.
}

\subsection{\content decomposition and \argst}
\is{lexical decomposition|(}

%ok, I see.  I agree, this should be discussed.  So, how does this sound as a list of criteria for inclusion in the decomposition?  Include something in the decomposition (in CONTENT) only if needed in order to:
%
%-- account for sublexical scope, e.g. result states (the Dowty 1979 stuff)
%-- account in a Pinker / Davis & Koenig way for argument alternations  
%-- account for linking in our system
%
%Basically this is everything the grammar needs to see.  Anything else?     
%
%So I'll say this, and I'll briefly contrast it with Hale & Keyser, Jackendoff.  Sound ok?  

Instead of thematic role types, lexical decomposition is typically used in HPSG to model the semantic side of the linking relation.  The word meaning represented by the \content value is decomposed into elementary predications that share arguments, as described in Section~\ref{linking-sec}\itdobl{we are in section~\ref{linking-sec}} above.  Lexical decompositions cannot be directly observed, but the decompositions are justified indirectly by the roles they play in the grammar.  Decompositions play a role in at least the following processes:

\begin{itemize}
\item  \textit{Linking.}  As described in Section~\ref{linking-sec},\itdobl{we are in section~\ref{linking-sec}} linking constraints are stated on semantic relations like \textit{act-und-rel} (actor-undergoer relation), so those relations must be called out in the \content \rep{field}{value}.
\item \textit{Sublexical scope.}  Certain modifiers can scope over a part of the situation denoted by a verb \citep{Dowty:1979a}.  
\end{itemize}

Consider sentence (\ref{again}).

\begin{exe}
\ex \label{again} John sold the car, and then he bought it again.
\end{exe}

\noindent
In this sentence, the adverb \textit{again} either adds the presupposition that John bought it before, or, in the more probable interpretation, it adds the presupposition that \textit{the result of buying the car} obtained previously.  The result of buying a car is owning it, so this sentence presupposes that John previously owned the car. Thus the decomposition of the verb \textit{buy} includes a \textit{possess-rel} (possession relation) holding between the buyer and the goods.  This is available for modification by adverbials like \textit{again}.

\begin{itemize}
\item \textit{Argument alternations.}  Some argument alternations can be modeled as the highlighting of different portions of a single lexical decomposition.  See Section~\ref{alternations}.\is{alternation}
\end{itemize}  

\noindent
In general, sublexical decompositions are included in the \content \rep{field}{value} only insofar as they are visible to the grammar for processes like these.  

% TD 19 Jan 20: I eedited this paragraph a bit
The \argst list lies on the syntax side of linking.  Just as the roles and predicates within \content must be motivated by (linguistic) semantic considerations, the presence of elements on \argst is primarily motivated by their syntactic visibility.  Many \argst list items are obviously justified, being explicitly expressed as subject and complement phrases, or as affixal pronouns.  In addition, certain implicit arguments should appear on \argst if, for instance, they are subject to the binding theory constraints that apply to \argst, as discussed in Section~\ref{express-sec} above.

Some implicit arguments can also participate in the syntax, for example, by acting as controllers of
adjunct clauses.  This could plausibly be viewed as evidence that such arguments are present on the
\argst list.  \ili{English} rationale clauses,\is{rational clause} like the infinitival phrase in
(\ref{mka}), are controlled\is{controlled argument} by the agent argument in the clause, \textit{the
  hunter} in this example.  The implicit agent of a short passive\is{passive} can likewise control
the rationale clause as shown in (\ref{mkb}).  But control is not possible in the middle
construction (\ref{mkc}) even though loading a gun requires some agent.  This contrast was observed
by \citet{KeyserandRoeper1984} and confirmed in experimental work by \citet{MaunerandKoenig2000}.

\begin{exe}
\ex\label{mk}
\begin{xlist}
\ex[]{
\label{mka}
The shotgun was loaded quietly by the hunter to avoid the possibility of frightening off the deer.
}
\ex[]{\label{mkb}
The shotgun was loaded quietly to avoid the possibility of frightening off the deer.
}
\ex[*]{
\label{mkc}
The shotgun had loaded quietly to avoid the possibility of frightening off the deer.
}
\end{xlist}
\end{exe}



\noindent
If the syntax of control  is specified such that the controller of the rationale clause is an (agent) argument on the \argst list of the verb, then this contrast is captured by assuming that the agent appears on the \argst list of the passive verb but not the middle.


\subsection{Modal transparency}
\is{modal transparency|(}
Another observation concerning lexical entailments\is{entailments!lexical} and linking was developed by \citet{KoenigandDavis2001}, who point out that linking appears to ignore modal elements of lexical semantics, even when those elements invalidate entailments (expanding on an observation implicit in \citealt{Goldberg1995}).
For instance, there are various \ili{English} verbs that display linking patterns like the ditransitive\is{ditransitive} verbs of possession transfer \word{give} and \word{hand}, but which denote situations in which the transfer need not, or does not, take place.
Thus, \word{offer} describes a situation where the transferrer is willing to effect the transfer, \word{owe} one in which the transferor should effect the transfer but has not yet, \word{promise} describes a situation where the transferor commits to effect the transfer, and \word{deny} one in which the transferor does not effect the contemplated transfer. 
\citeauthor{KoenigandDavis2001} argue that modal elements should be clearly separated in \content
values from the representations of predicators and their arguments.  (\ref{promise-sem}) exemplifies
this factoring out of sublexical modal information from core situational information. 
\begin{exe}
\ex\label{promise-sem} The lexical semantic representation of \word{promise} 
\citep[101]{KoenigandDavis2001} \\
\avm{
[\type*{promise-sem $\land$ cause-possess-sem}
sit-core &	\3 [\type*{cause-possess-rel}
			act & \1 \\
			und & \2 \\
			soa & [sit-core &	\5 [\type*{have-rel}
								act & \2 \\
								und & \4 ] ] ] \\
modal-base &	<[\type*{deontic-mb $\land$ condit-satis-mb}
				soa & \3 ]> ]
}
\itdopt{\ibox{1}, \ibox{4}, \ibox{5} are dangling.}
\end{exe}
This pattern of linking functioning independently of sublexical modal information applies not only to these ditransitive cases, but also to verbs involving possession (cf.\ \word{own} and \word{obtain} vs.\ \word{lack}, \word{covet}, and \word{lose}), perception (\word{see} vs.\ \word{ignore} and \word{overlook}), and carrying out an action (\word{manage} vs.\ \word{fail} and \word{try}).  Whatever the role of lexical entailments in linking, then, the modal components should be factored out, since the entailments that determine, e.g., the ditransitive linking patterns of verbs like \word{give} and \word{hand} do not hold for \word{offer}, \word{owe}, or \word{deny}, which display the same linking patterns. The constraints in (\ref{act-vb-linking})--(\ref{emb-act-vb-linking}) need only be minimally altered to target the value of \feat{sit-core}, representing the ``situational core'' of a relation.

 
% TD 10 Jan. 20: addded the following, including the summary subsection
%JPK 13 Jan 20: rewordinf the very end of the sentence to make it crisper.
This kind of semantic decomposition preserves the simplicity of linking constraints, while representing the differences between verbs that straightforwardly entail the relation between the arguments in the situational core and verbs for which those entailments do not hold, because their meaning contains a modal component restricting those entailments to a subset of possible worlds.
\is{modal transparency|)}

\is{lexical decomposition|)}

\subsection{Summary of linking}

In this section we have examined HPSG approaches to linking.  HPSG constrains the mapping between
participant roles\is{participant roles|see{thematic role, semantic role, proto-role}} in
\feat{content} and their syntactic \rep{realizations}{representation} on \argst\itdopt{
realization to me sounds as if it is actually realized somewhere in the tree as a daughter or an
affix.
} based on entailments of the semantic
relations in \feat{content}.  These constraints do not require a set of thematic roles arranged in a
hierarchy.  Nor do they require a numerical comparison of entailments holding for each participant
role, which has been an influential alternative to a thematic hierarchy.  Rather, they reference the
types of relations wthin a lexical entry's \feat{content}, and the subcategorization requirements of
its \argst.  Information from both is necessary because, although semantics is a strong determinant
of argument realization, independent stipulations of subcategorization appear to be needed, too.
Finally, we have examined the role of modal information in lexical semantics, which seem not to
interact much with linking, and described mechanisms proposed within HPSG that separate this
information from the situational core that drives linking.

In the remainder of this chapter, we will examine the relationship of argument structure to argument
alternations, including passives, as well as broader questions concerning the addition of other
elements like modifiers to \argst, the universality of \argst across languages, and whether \argst
is best regarded as solely a lexical attribute or one that should also apply to phrases or
constructions.


\section{The semantics and linking of argument alternations}
\label{alternations}
\is{alternation|(}
A single verb can often alternate between
%occur 
%in varied syntactic contexts, 
 various alternative patterns of dependent phrases, situations called either 
%which are termed 
\emph{argument alternations},
\emph{valence alternations}, or \emph{diathesis alternations}.
\citet{Levin1993} lists around 50 kinds of alternations in \ili{English}, and \ili{English} is not untypical in this regard.

%TD 14 Jan 20: I suggest removing the following, which is not referenced later and has nothing in particular to do with HPSG
%and there are still more, including the alternation illustrated in (\ref{chair}).  
%
%\begin{exe}
%	\ex\label{chair}
%	\begin{xlist}
%	\ex\label{chair-a} I found that the chair was comfortable
%	\ex\label{chair-b} I found the chair to be comfortable
%	\ex\label{chair-c} I found the chair comfortable	
%	\end{xlist}
%\end{exe}

How has argument structure in HPSG been used to account for alternations?
Many alternations exhibit (often subtle) meaning differences between the two 
alternants.
%Most alternations have been associated with subtle differences between the meanings of the two alternants, though they may be suble.
%Consequently, alternants with different argument structures and syntactic contexts owe this difference to different \feat{content} values.
We first discuss alternations due to these differences in meaning, showing how their differing \argst lists arise from differences in \feat{content}.  
We then examine some alternations where meaning differences are less apparent.
Although the \feat{content} values of the two alternants in such cases may not differ, we can analyze the alternation in terms of a different choice of \feat{key} predicate in each.
Lastly, we consider active-passive voice alternations, which are distinct from other alternations in important ways.
% TD 28 Jan 20: what was the motivation for commenting out the following?
%It is doubtful that meaning differences between passives and corresponding actives can furnish an explanation for their syntactic differences.
%Indeed, because passivization applies so productively in many languages, to verbs that range so widely in meaning, it would be difficult to model it as a semantic alternation or as a change of \feat{key} predicate.
%Accounts of passivization in HPSG have therefore resorted to two kinds of non-canonical lexical entries.
%One relaxes linking constraints, so that the active form's subject is not the first element of the passive form's \argst.
%The other relaxes the mapping between elements of \argst and elements of the \changed{valence} list, in a manner similar to the treatment of \ili{Balinese} voice alternations sketched in Section~\ref{ergativity}.

\subsection{Meaning-based argument alternations}

One well-studied alternation, the locative alternation, is exemplified by the two uses of \word{spray} in (\ref{spray}).

\begin{exe}
\ex \label{spray}
\begin{xlist}
\ex \label{spraya} \textit{spray$_{\text{loc}}$}: Joan sprayed the paint onto the statue.
\ex \label{sprayb} \textit{spray$_{\text{with}}$}: Joan sprayed the statue with paint.
\end{xlist}
\end{exe}

\noindent
It is typically assumed that these two different uses of \textit{spray} in (\ref{spray}) have
slightly different meanings, with the statue being in some sense more affected in the \word{with}
alternant.  This exemplifies the ``holistic'' effect of direct objecthood,\is{holistic effect of
  direct objecthood} which we will return to.  Here, we will examine how semantic differences
between alternants relate to their linking patterns.  The semantic side of linking has often been
devised with an eye to syntax (e.g., \citealt{Pinker1989}, and see \citealt{KoenigandDavis2006} for
more examples).  There is a risk of stipulation here, without independent evidence for these
semantic differences.  In the case of locative alternations, though, the meaning difference between
(\ref{spraya}) and (\ref{sprayb}) is easily stated (and Pinker's intuition seems correct), as
(\ref{sprayb}) entails (\ref{spraya}), but not conversely.  Informally, (\ref{spraya}) describes a
particular kind of caused motion situation, while (\ref{sprayb}) describes a situation in which
this kind of caused motion additionally results in a caused change of state.  The difference is
depicted in the two structures in (\ref{spray-sem}).

\begin{exe}\ex\label{spray-sem}
\begin{xlist}
\ex \label{spray-sema} \textsc{cause (Joan, go (paint, to (statue)))}
\ex \label{spray-semb} \textsc{act-on (Joan, statue, by (cause (Joan, go (paint, to (statue)))))}
\end{xlist}
\end{exe}

This description of the semantic difference between sentences (\ref{spraya}) and (\ref{sprayb}) provides a strong basis for predicting their different argument structures.
But we still need to explain how linking principles give rise to this difference.
Pinker's account rests on semantic structures like (\ref{spray-sem}), in which depth of embedding reflects sequence of causation, with ordering on \argst stemming from depth of semantic embedding, a strategy adopted in \citet{Davis1996} and \citet{Davis2001}.
This is one reasonable alternative, although the resulting complexity of some of the semantic representations raises valid questions about what independent evidence supports them.
An alternative appears in \citet{KoenigandDavis2006}, who borrow from \isi{Minimal Recursion Semantics} (see \crossrefchapteralt{semantics} for an introduction to MRS).
MRS ``flattens'' semantic relations, rather than embedding them in one another, so the configuration of these \emph{elementary predications}\is{elementary predication} with respect to one another is of less import.
It posits a \feat{relations} (or \rels)\isfeat{rels} attribute that collects a set of elementary predications, each representing some part of the predicator's semantics.
In Koenig and Davis' analysis, a \feat{key} attribute specifies a particular member of \rels as the relevant one for linking (of direct syntactic arguments).\isfeat{key}
In the case of (\ref{sprayb}), the \feat{key} is the caused change of state description.
These MRS-style representations of the two alternants of \word{spray}, with different \feat{key} values, are shown in (\ref{spray-on}) and (\ref{spray-with}).

\begin{exe}
\ex\label{spray-on}
\avm{
[key & \5 \\
rels &	<\5 [\type*{spray-ch-of-loc-rel}
		act & \1 \\
		und & \4 \\
		soa &	[\type*{ch-of-loc-rel}
				figure & \4 ] ] > ]
}
\itdopt{Is the idea to have this parallel to the next example? Otherwise renumber.}
\end{exe}

\begin{exe}\ex\label{spray-with}
\avm{
[key &	\3 [\type*{spray-ch-of-st-rel}
		act & \1 \\
		und & \2  \\
		soa &	[\type*{ch-of-st-rel}
				und & \2 ] 
		] \\
rels & < \3,	[\type*{use-rel}
				act & \1 \\
				und & \4  \\
				soa & \3 ], 
				[\type*{spray-ch-of-loc-rel}
				act & \1 \\
				und & \4 \\
				soa &	[\type*{ch-of-loc-rel} 
						figure & \4 ] ] > ]
}
\end{exe}                  
                  
Generalizing from this example, one possible characterization of valence alternations, implicit in \citet{KoenigandDavis2006}, is as systematic relations between two sets of lexical entries in which the \rels of any pair of related entries are in a subset/superset relation (a weaker version of that definition would merely require an overlap between the \rels values of the two entries). 
Consider another case; (\ref{caus-inch}) illustrates the causative-inchoative alternation, where the intransitive alternant describes only the change of state, while the transitive one ascribes a explicit causing agent.

\begin{exe}
\ex\label{caus-inch}
\begin{xlist}
	\ex\label{caus-inch-a}John broke the window.
	\ex\label{caus-inch-b}The window broke.
\end{xlist}	
\end{exe}

\noindent
Under an MRS representation, the change of state relation is a separate member of \rels; it is also included in the \rels of the transitive alternant, which contains a cause relation as well.
Again, the \rels value of one member of each pair of related entries is a subset of the \rels value of the other.

Many other alternations involve one argument shifting from direct to oblique.\is{oblique argument}
Some \ili{English} examples include conative, locative preposition drop, and \word{with} preposition drop alternations, as shown in (\ref{altex}):

\begin{exe}\ex\label{altex}
\begin{xlist}
\ex \label{altexa} Rover clawed (at) Spot. 
\ex \label{altexb} Bill hiked (along/on) the Appalachian Trail.
\ex \label{altexc} Burns debated (with) Smithers.
\end{xlist}
\end{exe}

\noindent
The direct object argument in (\ref{altexa}) is interpreted as more ``affected'' than its oblique counterparts:  if Rover clawed Spot, we infer that Spot was subjected to direct contact with Rover's claws and may have been injured by them, while if Rover merely clawed \word{at} Spot, no such inference can be made.
Similarly, to say that one has hiked the Appalachian Trail as in the transitive variant of (\ref{altexb}) suggests that one has hiked its entire length, while the prepositional variants merely suggest one hiked along some portion of it.  In still other cases like (\ref{altexc}), the two variants seem to differ very little in meaning.  

%How might these varying intuitions related to ``affectedness'' relate to lexical semantic representations like those in the (\ref{spray-on}) and (\ref{spray-with})?  

\citet{Beavers2010} observes the following generalization over direct--oblique alternations:  the direct variant entails the oblique one, and can have an additional entailment that the oblique variant lacks.  
His \emph{Morphosyntactic Alignment Principle}\is{Morphosyntactic Alignment Principle} states this generalization in terms of ``L-thematic roles'', which are defined as sets of entailments associated with individual thematic roles:\is{thematic role|see{semantic role, participant role, thematic hierarchy}}   

\begin{exe}
\ex\label{beavers-map}
When participant $x$ may be realized as either a direct or \isi{oblique argument} of verb V, it bears L-thematic role $R$ as a direct argument and L-thematic role $Q\subseteq_{M}R$ as an oblique.
\citep[848]{Beavers2010}
\end {exe}

\noindent
Here, $Q$ and $R$ are roles, defined as sets of individual entailments, and  $Q\subseteq_{M}R$ means that set $Q$ is a subset of $R$ that is minimally different from $R$, differing in at most one entailment.
%a ``minimally weaker'' role than $R$; in other words, there is no role
%$P$ in the predicator such that $Q⊂P⊂R$.
Thus, the substantive claim is essentially that the MAP rules out ``verbs where the alternating participant has \textsc{MORE} lexical entailments as an oblique than the corresponding object realization'' \citep[849]{Beavers2010}.
% \citet{beavers:2005} presents this generalization in the HPSG framework, using MRS semantic representations similar to those employed above.  
% TD 31 Jan. 20: I've now looked at Beavers (2005) -- thanks for tracking that down -- and I think there are some substantive differences worth noting briefly here, so I've revived some of the stuff that was commented out below, putting it here in a revised form.  
The notion of a stronger role in Beavers' analysis has a rough analogue in terms of whether a particular elementary predication is present in the semantics of a particular alternant.
\citet{Beavers2005} describes a version of the Morphosyntactic Alignment Principle implemented in HPSG, which posits a separate \feat{roles} attribute within \feat{content}, containing a list of labeled roles.
The roles are ordered on the \feat{roles} list, determined at least partly by direction of causality, although this is not fully worked out.
Each role can be regarded as a bundle of entailments.
The bundle of entailments varies slightly between different alternants of verbs like those in (\ref{altex}), and the Morphosyntactic Alignment Principle comes into play, comparing the sets of entailments constituting each role.
Assessing which of two roles is stronger, according to this principle, requires some additional mechanisms within HPSG that are not spelled out.

Beavers notes the resemblance between his account and numerical comparison approaches such as those of \citet{Dowty1991} and \citet{AckermanandMoore2001}.
He points out that the direct object bears an additional entailment in each alternant. However, the specific entailment involved depends on the verb; the entailments involved in each of the examples in (\ref{altex}), for instance, are all different.  Thus, comparing the numbers of entailments holding for a verb's arguments in each alternant is crucial.

%The entailments Beavers employs differ somewhat from those we have discussed here, involving quantized change, nonquantized change, potential for change (where change can refer to change in location, possession, state, or something more abstract), furnishing the clear ordering by strength that is central to his proposal.
%But they do resemble entailments of semantic relations we have represented as elementary predications, such as incremental theme, change of state, and possession, along with the modal effects described in \citet{KoenigandDavis2001}.
%Thus the notion of a stronger role in Beavers' analysis has a rough analog in terms of whether a particular elementary predication is present in the semantics of a particular alternant.
%And only if an elementary predication is present, can it be designated as the \feat{key}, and its roles linked directly.
%For example, in (\ref{spray-on}), there is nothing representing affectedness of the location, while in (\ref{spray-with}), there is, and it is designated as the \feat{key}.
%As noted earlier, the semantics in (\ref{spray-with}) represents this additional entailment borne by the location argument.
%However, we are not aware of any simple, general way to represent Beavers' MAP within the EP-based model of \citet{KoenigandDavis2006}.  
%ALERT:  Why not?  can't you associate the transitive V with an extra entailment?  Then when you replace the object with a PP that entailment goes away.  
% TD 28 Jan 20: You might be able to do that, if there's a consistent set of entailments that go with direct objecthood.  But the constraint as presented is about numbers of entailments.  Let's think about it a bit.
% JPK 1-29-20 The problem is that there is no consistent entailment associated with objecthood according to John's approach. It is a meta-semantic statement: whatever the meaning of the "input", strenghten it (clearly, a simplified description). I can't see how to model that in the way HPSG is supposed to work. We maybe could stipulate an additional semantic type strengthened_affectedness that would be vacuous for some objects and have a different model-theoretic effect for different classes of verbs alternating between obliques and direct objects, but that would be a hack. Am I missing something?
%Indeed, there is an aspect of Beavers' view that seems more in accord with numerical comparison approaches such as those of \citet{Dowty1991} and \citet{AckermanandMoore2001}, in that role strength is determined by the number of entailments that hold of it relative to others.  

\subsection{Relationships between alternants}

Having outlined the semantic basis of the different linking patterns of alternating verbs, we briefly take up two other issues.
First is the question of how the alternants are related to one another.
Second is how \feat{key} selection has been used to account not just for alternants of the same verb, but for (nearly) synonymous verbs whose semantics contain the same set of elementary predications.

The hypothesis pursued in \citet{Davis1996} and \citet{Davis2001}  is that 
most alternations are the consequence of classes of lexical entries having
two related meanings. This follows researchers such as \citet{Pinker1989} and \citet{Levin1993} in modeling subcategorization alternations as underlyingly meaning alternations. 
This change in meaning is crucial to the \citet{KoenigandDavis2006} \feat{key} shifts as well. In some cases, the value of the \rels attribute of the two valence alternates differ (as in the two alternates of \word{spray} in the so-called \word{spray/load} alternation we discussed earlier).
In some cases, the alternation might be different construals of the same event for some verbs, but not others, as \citet{RappaportandLevin2008} claim for the \ili{English} ditransitive\is{ditransitive} alternations, which adds the meaning of transfer for verbs like \word{send}, but not for verbs like \word{promise}; a \feat{key} change would be involved (with the addition of a \type{cause-possess-rel}) for the first verb only. But \feat{key} shifts and diathesis alternations do not always involve a change in meaning. The same elementary predications can be present in the \feat{content} values of two alternants, with each alternant designating a different elementary predication as the \feat{key}.\isfeat{key}

Koenig and Davis\itdopt{Here it really makes a difference whether we have \& or ``and''. Is it the
  sum of your works or is it \citeauthor{KoenigandDavis2006} as in \citew{KoenigandDavis2006}?} propose this not only for cases in which there is no obvious meaning difference between two alternants of a single verb, but also for different verbs that appear to be truth-conditionally equivalent.
% TD 31 Jan 20: revisions made throughout this paragraph to checnge from buy/sell to substitute/replace
The verbs \word{substitute} and \word{replace} are one such pair.
The two sentences in (\ref{subrep}) illustrate this equivalence.

\begin{exe}
\ex\label{subrep}
\begin{xlist}
\ex\label{subrepa}They substituted an LED for the burnt-out incandescent bulb.
\ex\label{subrepb}They replaced the burnt-out incandescent bulb with an LED.
\end{xlist}
\end{exe}

\noindent
These two verbs denote a type of event in which a new entity takes the place of an old one, through (typically intentional) causal action.
%Koenig \& Davis suggest that both the removal of the old entity and the placement of the new one are represented as elementary predications in the \feat{content} values of these verbs.
Koenig and Davis decompose both verb meanings into two simpler actions of removal and placement: `x removes y (from g)' and `x places z (at g)', each represented as an elementary predication in the \feat{content} values of these verbs.  


% one famous example being the verbs of commercial exchange \word{buy} and \word{sell} (but see \citet[387-388]{VanValin1999}, \citet[20]{LevinandRappaport2005}, and \citet{Wechsler:2005} for arguments that \word{buy} and \word{sell} are not equivalent). 
% ??   if they were equivalent then 'John bought it' and 'John sold it' would have the same meaning, which nobody believes.  Not sure what you mean here.  ALERT
% Koenig and Davis argue that a commercial event involves two reciprocal actions, a transfer
%an exchange [exchange = 'an act of giving one thing and receiving another (especially of the same type or value) in return.'  So a sale involves one exchange, not two.] of goods (which involves giving goods and obtaining goods) and 
%an exchange a transfer
% of money (which involves giving money and obtaining money). Individual verbs might select  one or the other these four relations, thus accounting for the differences in subject and object selection. As shown in (\ref{buy-lex}) and (\ref{sell-lex}), each of these verbs contains four elementary predications: one \type{exch-give-rel} and one \type{obtain-rel} for the transfer of goods, and one of each for the counter-transfer of money. 
%ALERT: there are 2, not 4 EPs in the AVM for 'sell', right?  Also I don't understand this analysis.  why are there 4 EPs for 'buy'?   The seller doesn't have to own the item he sells; cp. salesmen.  Nor does the buyer necessarily come to own it; cp. purchasing agents.  See my little v paper.   More generally, what is the basis for including all this structure?  a word usually has an infinite set of lexical entailments, so accounting for entailments can't be sufficient.  
% TD 28 Jan 20: I think we could avoid some of these issues (which obscure the main point here anyway) by using an example other than 'buy'/'sell', such as 'substitute' (X for Y)/'replace' (Y with X). I agree that 'sell' differs in meaning from 'buy' (a plausible suggestion is that it means ``cause to buy'), so it's not a great example to use, even though historically it has been sometimes held up as one.. 
%JPK 1-29-20 Fine with the substitution (so to speak!). For whatever it is worth, I disagree witht the claim that it's clear that "sell" and "buy" are not truth conditionally equivalent. I could imagine making the claim that BOX (sell <-> buy) and that examples typically cited that suggest it's false are different uses of "buy" (different "micro-meanings"). But that's for another day.  
%SW 1-29-20.  You think that 'JP bought a horse' and 'JP sold a horse' have the same meaning?  
%JPK 1-31-20 No I mean that "Somebody sold a horse to JP" and "JP bought a horse from somebody" have the same meaning. At least, one can make the claim that they mutually entail each other, and some people's view that's having the same meaning :).
%\word{Buy} designates the \type{obtain-rel} representing the transfer of goods as the \feat{key}, while \word{sell} designates the \type{exch-give-rel} representing the transfer of goods as the \feat{key}.  Other verbs, such as \word{pay} or \word{charge}, choose elementary predications representing the counter-transfer as the \feat{key}.  
% The relevant portions of the entries for \word{buy} and \word{sell} in (\ref{buy-lex}) and (\ref{sell-lex}) below illustrate: critically, the \feat{key} relation for \word{buy} is not the same as that for \word{sell}.


%\begin{exe}
%\ex\label{buy-lex}
%A representation of the relevant parts of the lexical entry for \word{buy}: \\
%{\avmoptions{center}
%\begin{avm}\[content & \[key &\@7 \\
%                         relations & \<\@7\[act & \@1 (buyer) \\
%                          und & \@2 \\
%                          soa & \@5\[\asort{$\textit{possess-rel}$} 
%                                    act & \@1 \\
%                                    und & \@2 (goods)\]\],
%                        \[act & \@1 (buyer) \\
%                          und & \@4 \\
%                    soa & \@6\[\asort{$\textit{possess-rel}$}
%                                    act & \@3 (seller) \\ 
%                                    und & \@4 (money)\]\] \\
%                              \[\asort{$\textit{obtain-rel}$}
%                   act & \@3 (seller) \\
%                          und & \@4 \\
%                    soa & \@6\[\asort{$\textit{possess-rel}$} 
%                                    act & \@3 (seller) \\ 
%                                    und & \@4 (money)\]\], 
%\@8\[\asort{$\textit{exch-give-rel}$}
%                    act & \@3 (seller) \\
%                          und & \@2 \\
%                    soa & \@5\[\asort{$\textit{possess-rel}$} 
%                                    act & \@1 \\ 
%                                    und & \@2 (goods)\]\]
%                                                         \> \] \\
%             arg-st & \<NP:\@1, NP:\@2, PP(from):\@8\>\]
%\end{avm}}	
%\end{exe}

%\begin{exe}
%\ex\label{sell-lex}	
%A representation of the relevant parts of the lexical entry for \word{sell}: \\
%{\avmoptions{center}
%\begin{avm}\[content & \[key &\@7 \\
%                         relations & \<\@7\[act & \@1 (seller) \\
%                                          und & \@2 \\
%                    soa & \@6\[\avmspan{\textit{possess-rel}} \\
%                                    act & \@3(buyer) \\ 
%                                    und & \@2 (goods)\]\],
%                             \[act & \@1 (seller) \\
%                          und & \@4 \\
%                    soa & \@6\[\avmspan{\textit{possess-rel}} \\
%                                    act & \@1 \\ 
%                                    und & \@4 (money)\]\]\> \] \\
%             arg-st & \<NP:\@1, NP:\@2, PP(to):\@7\>\]
%\end{avm}}
%\end{exe}



%\pagebreak

% TD 31 Jan. 20: We might consider simplifying and clarifying these AVMs a bit.  We might dispense with all the SIT-CORE and modal aspects that aren't pertinent to the point we're making here, and distinguish the removal and placement relations with different names.

\begin{exe}
\ex \label{fig:place} Representation of `x places y (at g)': \\*
\avm{
\tag{a} $=$
[\type*{place-rel}
act & \1 (\normalfont x) \\
und & \2 (\normalfont y) \\
soa &	[\type*{motion-sem} \\
		sit-core &	[\type*{motion-rel} \\
					fig & \2 \\
					grnd & g ]\\
		modal-base & < > ] ]
}
\end{exe}


\begin{exe}
\ex \label{fig:remove} Representation of `x removes z (from g)' \\
\avm{
\tag{b} $=$
[\type*{remove-rel} \\
act & \1 (\normalfont x) \\
und & \3 (\normalfont z) \\
soa &	[\type*{motion-sem} \\
		sit-core & \4 [\type*{motion-rel} \\
			       fig & \3 \\
			       grnd & g ]\\
		modal-base & <[\type*{neg-rel} \\
			      soa & \4] > ] ]
}
\end{exe}

\noindent
Either one can be selected as the \feat{key}.
In the lexical entry of \word{replace}, the removal predication is the value of \feat{key}, while in the lexical entry of \word{substitute}, the placement of the new object is the value of \feat{key}.
(\ref{fig:replace}) and (\ref{fig:subst}) show the \feat{content} values of these two verbs under this account, where \fbox{a} and \fbox{b} abbreviate the structures in (\ref{fig:place}) and (\ref{fig:remove}).
In both cases, the same linking constraints apply between the  \feat{key} and the \argst list, but the two verbs have different argument realizations because their \feat{key} values differ, even though their semantics are equivalent.


\begin{exe}
\ex \label{fig:replace} \textsc{content} value of \word{substitute for}: \\
\avm{
[key &  \tag{a} \\
rels & < \tag{a}, \tag{b} > ]
}
\end{exe}


\begin{exe}
\ex \label{fig:subst} \textsc{content} value of \word{replace with}: \\
\avm{
[key &  \tag{b} \\
rels & < \tag{a}, \tag{b} > ]
}
\end{exe}

As a final example of the effect of alternations on fine-grained aspects of verb meaning, we consider the source-final product alternation exemplified in (\ref{carve}), where the direct object can be either the final product or the material source of the final product. 

\begin{exe}
\ex\label{carve}
\begin{xlist}
	\ex\label{carve-a} Kim made/carved/sculpted/crafted a toy (out of the wood).
	\ex\label{carve-b} Kim made/carved/sculpted/crafted the wood into a toy.
\end{xlist}
\end{exe}

\noindent
Davis proposes that the (\ref{carve-a}) sentences involve an alternation between the two meanings represented in (\ref{carve-sem}), each associated with a distinct entry. We adapt \citet{Davis2001} to make it consistent with \citet{KoenigandDavis2006} and also treat the alternation as an alternation of \emph{entries} with distinct meanings. 
%Note that in the alternation described in (\ref{carve-sem}), we use, informally, a double-headed arrow. 
Lexical rules are a frequent analytical tool used to model to alternations between two related meanings of a single entry illustrated in (\ref{carve-sem}). One of the potential drawbacks of a lexical rule approach to valence alternations is that it requires selecting one alternant as basic and the other as derived. This is not always an easy decision, as \citet{Goldberg1995}\addpages or \citet{LevinandRappaport1994} have pointed out (e.g., is the inchoative or the causative basic?). Sometimes, morphology provides a clue, although in different languages the clues may point in different directions.  \ili{French}, and other \ili{Romance} languages, use a ``reflexive'' clitic as a detransitivizing affix.  In \ili{English}, though, there is no obvious ``basic'' form or directionality. It is to avoid committing ourselves to a directionality in the relation between the semantic contents described in (\ref{carve-sem}) that we eschew treating it as a lexical rule.

% TD 10 Jan. 20: the arrow in this figure is pretty ugly; can we change it, perhaps to the sort of arrows that are in Figure~\ref{fig:over}?  We might consider orienting it vertically as well, rather than horizontally
%JPK 1-13-20: I removed the arrows and reworded the last sentence in the preceding paragraph.



\begin{exe}
	\ex\label{carve-sem}
	\begin{xlist}
	\ex \label{carve-sem-a}
\avm{
[content &	[key &	\3[\type*{affect-incr-th-rel} 
					act & \1 \\
					soa & \2 \upshape !(final product)! ] \\
rels & < \3, \4 > ] ]
}
\ex \label{carve-sem-b}
\avm{
[content	[key & \4	[\type*{affect-incr-th-rel}
						act & \1 \\
						und & \upshape !(source material)! \\
						soa &	[\type*{affect-incr-th-rel} 
								act & \1 \\
								soa & \2 \upshape !(final product)! ] ] \\
			rels & < \3, \4 > ] ]		
}
\end{xlist}
\itdopt{Stefan: Does this assume cross-AVM structure sharing? Otherwise there are many unlinked
  tags.
What are the things in brackets? Annotations? Or types? Right now they are not in italics. Which may
be better if they are annotations. Generally I am uneasy about things that could be confused with
actual formal inventory. Everything that can be confused will be confused. Add a footnote aboout the
things in brackets?}
\end{exe}
 
% TD 14 Jan. 20: new subsection for passives
\subsection{The problem of passives}
\label{passives}\label{arg-st:sec-passives}
\is{passive(}
\is{alternation!voice|see{passive}}
% TD 14 Jan. 20: revised this introductory material a bit, to fit with the new stuff
Although most diathesis alternations can be modeled as alternations in meaning and as \feat{key} shifts, some arguably cannot. 
One prominent example is the active/passive alternation.
Other widely attested constructions, such as raising constructions, similarly involve no change in meaning, but we will examine only passives here. 

The semantics of actives and corresponding long passives, as in (\ref{act-pass}), are practically identical and the difference between the two alternants is pragmatic in nature. 

\begin{exe}
	\ex\label{act-pass}
	\begin{xlist}
		\ex\label{act-pass-a}Fido dug a couple of holes.
		\ex\label{act-pass-b}A couple of holes were dug by Fido.
	\end{xlist}
\end{exe}

%We might regard the relationship between actives and passives as a degenerate case of the subset relationship between \rels attributes, where both the \rels and \feat{key} values of the two entries are identical.
% TD 16 Jan. 20: I don't quite follow this; is it referencing Beavers's stuff?  Maybe we explain a bit more, or just delete this first sentence.
%SW: Same here.  I deleted it.  
%But we must still account for the demotion of the active subject in the corresponding passive, and, in the context of this chapter, what the \argst list of a passive looks like.
In this section, we outline two possible approaches to 
%this question.
the passive.  
Both of them treat the crucial characteristic of passivization as subject demotion (see \citealt{Blevins2003} for a thorough exposition of this characterization), rather than object advancement, as proposed, e.g., in \isi{Relational Grammar} \citep{PerlmutterandPostal1983b}.
As we will see, there are various options for implementing this general idea of demotion within HPSG.

% TD 14 Jan. 20: new material inserted here
The first approach, which goes back to \citet[\page 215]{pollard+sag:1987}, assumes that passivization targets the first member of a \feat{subcat} list and either removes it or optionally puts it last on the list, but as a PP.
This approach is illustrated in (\ref{pass-lr}), a possible formulation of a lexical rule for transitive verbs adapted to a theory that replaces \feat{subcat} with \argst, as discussed in \citet[67]{Manning+Sag:1999}. 
See \citew{Mueller2003e} for a  more refined formulation of the passive lexical rule for \ili{German} that accounts for impersonal passives, and \citet{Blevins2003} for a similar analysis.
The first NP is demoted and either does not appear on the output's \argst or is a PP coindexed with the input's first NP's index (see \crossrefchapteralt{binding} 
%and \crossrefchapteralt{diachronic} 
for scholars who assume the latter view within this handbook).\itdopt{Stefan: What exactly are you
  referring to here?}
Linking in passives thus violates the constraints in (\ref{act-vb-linking})--(\ref{emb-act-vb-linking}), specifically (\ref{act-vb-linking}), which links the value of \feat{act} to the first element of \argst. (We use one possible feature-based representation for lexical rules to help comparing approaches to passives.) See \citet{Meurers2001} and \crossrefchaptert{lexicon} for a discussion of various approaches to lexical rules.
Note that we use the attribute \feat{lex-dtr} rather than the \feat{in(put)} attribute used in the
representation of lexical rules in \citet[76]{Meurers2001} to avoid any procedural
implications;\itdopt{nothing is procedural in Meurers' setting. So this is slightly misleading.} nothing substantial hinges on this labeling change.

%\avmsetup{columnsep=.3ex}

 \ea
\label{pass-lr}
 %% {\avmoptions{center}\begin{avm}
 
 %% \[\asort{passive-verb}
 %% arg-st & \@2\textup{(}$\oplus$\<PP[\type{by}$]$$_{i}$\>\textup{)} \\
 %% lex-dtr & \[\asort{stem}
 %%             head & \type{verb}\\
 %%             arg-st & \<\@1 $_{i}$\> $\oplus$ \@2\<np, \ldots\>
 %%           \]
 %% \]
 %% \end{avm}}
Passive lexical rule:\\
\avm{
[\type*{passive-verb}
 arg-st  & \2 ( \+ < ! PP[\type{by}]$_i$ ! > )\\
 lex-dtr & [\type*{stem}
            head   & verb\\
            arg-st & < \1$_i$ > \+ \2 < NP, \ldots > ]]  
}
% \avm{
% [\type*{passive-verb}\\
%  arg-st  & \2 ! \upshape ~( $\oplus$ \sliste{ PP[\type{by}]$_i$  } ) !\\
%  lex-dtr & [\type*{stem}
%             head   & verb\\
%             arg-st & \sliste{ \1$_i$ } \+ \2 \sliste{ NP, \ldots } ]]  
% }
\itdopt{Stefan: \ibox{1} is not used. Renumber?}
\z
% ALERT:  should the attribute be INPUT instead of IN?
% TD 28 Jan. 20: I'm fine with either
%JPK 1-29-20: I used "IN" because this is what Meurers (2001) uses and we cite in the lexicon chapter, i.e. for consistency across chapters. That said, if you both want INPUT, fine with me.
We will refer to this approach as the non-canonical linking analysis of passives. This kind of analysis invites at least three questions.\is{linking!and passives}
First, as already noted, the constraint linking the value of \feat{act} to the first element of \argst is violated.
If passives --- widespread and hardly exotic constructions --- violate canonical linking constraints, how strong an account of linking can be maintained?\itdopt{Stefan: Linking could be said to apply to basic lexemes only. Since passive takes over the \cont and the linked arguments in some order, the linking remains established. No need for relinking things. This is what I do in CoreGram.}
Second, what other predictions, such as changes in binding behavior, control constructions, and discourse availability, arise from the altered \argst of passives?
Third, what is the status of the \word{by}-phrase in long passives, and how is it represented on the \argst list?

Another approach maintains the \argst list of the active verb in its passive counterpart, thereby preserving linking constraints.
Passives differ from actives under this account in their non-canonical mapping from \argst to \changed{valence} lists; the subject is not the first element of the \argst list. This analysis bears some resemblance to the distinction between macro-roles and syntactic pivots in Role and Reference Grammar, with passives having a marked mapping from macro-roles to syntactic pivot \citep{VanValinandLapolla1997}.  
In this kind of approach, the passive subject might be the second element of the \argst list, as in a typical personal passive, or an expletive element, as in impersonal passives.
In a long passive, the first element of \argst is coindexed with a PP on the \feat{comps} list or an adjunct.
This analysis is reminiscent of the account of \ili{Balinese} objective voice presented in Section~\ref{ergativity} in that the account of both phenomena uses a non-canonical mapping between \argst and \changed{valence} lists.
% ALERT:  We argued in some detail that passive differs from OV and can't work this way.  cp. binding "They don't seem to mind being insulted by each other."    \ili{Balinese} would be:  "Each other insulted-OV them."
% TD 28 Jan. 20: Yes, I don't wish to imply that passive and OV are the same, just that the non-canonical mapping between arg-st abd valence in this treatment of passives resembles the account of OV presented earlier.  If that isn't clear, we should reword this so that it is.
%JPK 1-29-20 You are right, Steve, that our wording was unclear. I added something to the end of the sentence to make the parallel clearer. Hopefully, that clarifies things.
%SW 1-29-20.  Sorry, but it doesn't help. I'll email about it.
A version of this view is proposed \rep{in}{by} \citet[241]{Davis2001}, who \skp[2]\rep{proposes}{suggests} the representation in (\ref{pass-lxm}) for passive lexemes (as before, we substitute the attribute name \feat{lex-dtr} for \feat{in}).

\ea
\label{pass-lxm}
Passive lexical rule:\\
\avm{
[\type*{passive-verb}
subj   & \2  \\
comps  & \3 \\
arg-st & \1 \upshape !(! < XP > !)! \+ \2 \+ \3 \\\relax
lex-dtr & [\type*{trans-stem}
	   arg-st  & \1 \\
	   content & \4 ]\\
content & \4 ]
}
\itdopt{Stefan: Renumber top down?}
\z

% TD 15 Jan. 20: There are some incidental differences between (\ref{pass-lr}) and (\ref{pass-lxm}) that potentially obscure the comparison we wish to make.  The first is presented as a lexical rule with an arrow between input and output, while the second is shown as a morpholigically complex lexical entry.  But that difference is irrelevant here, so we should present them in parallel fashion.  I don't have a strong opinion about which format to use, though I would say that if we ever want to refer to information in the stem, we might find it simpler to use the morpholigically complex lexical entry option.  We can say that we've modified the originals to make them easily comparable, and perhaps try to be consistent with what we say about lexical rules in the Lexicon chapter (e.g., change \feat{stem} to \feat{in}, as I've done just above).
% TD 15 Jan. 20: Also, it looks like the brackets around the outer \argst in (\ref{pass-lxm}) are a bit messed up.

We will refer to this as the non-canonical argument realization analysis of passives. Again, at least three 
issues must be addressed.
%JPK 1-29-20 I changed the phrasing of this sentence to remove an implicature of "questions" that was not meant.
%questions immediately present themselves.
First, the standard mapping between the elements of \argst and those of the \changed{valence} lists is violated.
If passives violate these canonical mapping constraints, how strong an account of the relationship between \argst and \changed{valence} can be maintained?
Second, as with the non-canonical linking analysis, what predictions, such as changes in binding behavior, control constructions, and discourse availability, arise from the non-canonical valence values in passives?
Third, what is the status of the \word{by}-phrase in long passives, and how is it represented on the \argst list?  If the logical subject remains the first element of a passive verb's \argst list, does it appear as an additional oblique element on \argst as well? 

The implications of weakening canonical constraints under each of these analyses have not been thoroughly addressed, to our knowledge.
%ALERT.  Wayan Arka and I addressed it. maybe not thoroughly enough  
% TD 28 Jan. 20: Sorry; in that case this should be changed to reflect that.
%JPK 1-29-20: Agree with Tony, feel free to edit the way you want, Steve.
We are unaware, for example, of proposals that limit non-canonical linking in HPSG to only the kind observed in passives.
One might begin by stipulating that linking concerns only NP (i.e., ``direct') arguments on \argst, but the implications of this have not yet been well explored.
With respect to the non-canonical argument realization analysis, the required variation in \argst to \changed{valence} mappings has been investigated somewhat more (see Section~\ref{argst-valence-sec} for some details), especially in connection with ergativity and voice alternations, and also in analyses of pro-drop, cliticization, and extraction \citep{MillerandSag1997, Manning+Sag:1999, Boumaetal2001}.
Thus, there are some independent motivations for positing non-canonical mappings between \argst and \changed{valence} lists.
But we will leave matters here in regard to the general advantages and drawbacks of non-canonical linking versus non-canonical argument realization.

As for passives in particular, the two analyses make different predictions regarding binding and control by the ``logical subject'' (the subject of the corresponding actives).
Under the non-canonical linking analysis, it is not present on the \argst (and \changed{valence}) lists of short passives, so it is predicted to be unavailable to any syntactic process that depends on elements of \argst.
Binding and varieties of control that reference these elements therefore cannot involve the logical subject.
Under the non-canonical argument realization analysis, the logical subject is present on the \argst lists of short passives, so it is predicted to play much the same role in binding as it does in corresponding actives.
However, we can see that, at least when unexpressed, this is not the case, as in (\ref{no-bind}).
%ALERT:  Then why present this as an open question above?  oh ok... but still.
% TD 28 Jan. 20: We can reword that so it doesn't read like an open question; where is that, exactly?
\begin{exe}
\ex[*]{\label{no-bind}
The money was donated to himself. (\word{himself} intended to refer to the donor)
}
\end{exe}

Certain control constructions also illustrate this point.
While the unexpressed logical subject can control rationale clauses in \ili{English}, as exemplified above in (\ref{mkb}), not all cases of control exhibit parallel behavior.
The \ili{Italian} consecutive \word{da} + infinitive construction \citep{Perlmutter1984, Sanfilippo1998} appears to be controlled by the surface subject, as shown in (\ref{italian-da}).\il{Italian}

% TD 18 Jan. 20: these examples are from Sanfilippo (1998), (49a) and  (51)

\begin{exe}
\ex     \label{italian-da}
\begin{xlist}
\ex
\gll Gino ha  rimproverato Eva tante   volte da    arrabbiarsi.  \\
     Gino has scolded      Eva so.many times so.as to.get.angry  \\
\glt `Gino scolded Eva so many times that he/*she got angry.'
\ex
\gll Eva fu  rimproverata da Gino tante   volte da    arrabbiarsi.  \\
     Eva was scolded      by Gino so.many times so.as to.get.angry  \\
\glt `Eva was scolded by Gino so many times that *he/she got angry.'
\end{xlist}
\end{exe}

Although there are other factors involved in the choice of controller of consecutive \word{da} infinitive constructions, it is clear that the logical subject in the passivized main clause cannot control the infinitive.
Thus, even if it remains the initial element of the passive verb's \argst, it must be blocked as a controller.
Sanfilippo argues from these kinds of examples that the passive \word{by}-phrase should be regarded as a ``thematically bound'' (i.e., linked) adjunct that does
not appear on the passive verb's \argst list, but on the \feat{slash} list.
However, this would require some additional mechanism to explain the involvement of the \word{by}-phrase in binding, noted below, and possibly with respect to other evidence for including adjuncts on \argst, as discussed in the Section~\ref{sec:extended-arg-st}.

In addition, the implicit agent of short passives is ``inert'' in discourse, as discussed in \citet{KoenigandMauner1999}.
It cannot serve as an antecedent of cross"=sentential pronouns without additional inferences, as shown in (\ref{a-def}), where the referent of \word{he} cannot without additional inference be tied to the logical subject argument of \word{killed}, i.e., the killer.

\begin{exe}
\ex\label{a-def}
\# The president was killed. He$_{i}$ was from Iowa.
\end{exe}

Note that the discourse inertness of the implicit agent in (\ref{a-def}) does not follow from its being unexpressed, as shown by the indefinite use of the subject pronoun \word{on} in \ili{French} \citep[241--244]{Koenig1998c} or \ili{Hungarian} bare singular objects \citep[89--108]{FarkasandDeSwart2003}.
These, though syntactically expressed, do not introduce discourse referents either.
In such cases, as well as in passives under the non-canonical argument realization analysis, the first member of the \argst list must therefore be distinguished from indices that introduce discourse referents.
% TD 17 Jan. 20: Here's a bit that might be added to the paragraph above, with a brief mention of how ``inertness'' might be handled, and maybe Kathol's idea of two different kinds of indices (cross-ref to Agreement chapter for that as well).

These facts would seem to favor the non-canonical linking analysis.
However, there are options for representing the inertness of the logical subject under the non-canonical argument realization analysis.
One possibility is to introduce a special subtype of the type \type{index}, which we could call \type{inert} or \type{null}; by stipulation, it could not correspond to a discourse referent.
This is also one way to treat the inertness of expletive pronouns, so it has some plausible independent motivation.
Unlike expletive pronouns, the logical subject of passives is linked to an index in \feat{content}.
Its person and number features therefore cannot be assigned as defaults (e.g., third person singular \word{it} and \word{there} in \ili{English}), but must correspond to those of the entity playing the relevant semantic role in \feat{content}.
\citet[251--253]{Davis2001} offers a slightly different alternative using the dual indices \feat{index} and \feat{a-index}, following the distinction between \feat{agr} and \feat{index} used in \citet[240--250]{Kathol1999b} to model different varieties of agreement.
The \feat{a-index} of a passive verb's logical subject is of type \type{null}, which, by stipulation, can neither o-command other members of \argst nor appear on \changed{valence} lists.
%This would be the case for both personal and impersonal passives.
In both impersonal and short personal passives, the logical subject is coindexed with a role in \feat{content} representing an unspecified human (or animate).
These analyses of logical subject inertness have not been pursued, however.

% TD 17 Jan. 20: I'm not sure how this would work, actually.  For linking, the logical subject needs to be coindexed with the relevant role in \feat{content}.  So that index can't be \type{null}.  Attempting to use Wechsler's \feat{index} vs. \feat{concord} distinction, we'd then heed to say that the logical subject's \feat{concord} is the \type{null} element.  But my understanding of Wechsler's distinction is that a \type{null} value of \feat{concord} wouldn't block binding.  Binding is still stated in terms of \feat{index}.  I think Kathol's use of \feat{a-index} was a bit different, but I haven't checked.


% TD 16 Jan. 20: I suggest deleting the following, now paraphrased elsewhere.
%Such an analysis, though, seems to run afoul of data that suggest that unexpressed agents of passives can bind reflexives or control\is{controlled argument} the subject of rationale clauses\is{rationale clause} we presented above and already discussed by Manning and Sag. For such reasons as well as the view that voice phenomena of the kind we presented in Section~\ref{ergativity} for \ili{Balinese} are better treated as different kinds of relations between the \argst list and the valence lists, a frequent approach these days seems to treat passive as a non-canonical relation between the \argst list and the valence lists (see \crossrefchapteralt{binding} and \crossrefchapteralt{complex-predicates} for scholars who assume such a view within this handbook). One possible version of this view is proposed in \citet[241]{Davis2001} who proposes the representation in (\ref{pass-lxm}) for passive lexemes.
%
%\begin{exe}
%\ex\label{pass-lxm}
%{\avmoptions{center}
%\begin{avm}
%\[\asort{passive-verb}
%subj & \@2  \\
%comps & \@3 \\
%arg-st & \@1(\<XP\>$\oplus$\@2$\oplus$\@3 \\
%stem & \[\asort{trans-stem}
%			arg-st & \@1 \\
%			content & \@4 \]\\
%content & \@4
%\]
%\end{avm}}
%\end{exe}

%The first member of the \argst list is not mapped onto a valence list, but the rest of the \argst list is mapped onto the valence lists as per the Argument Realization Principle, i.e. the second member of the \argst list is the value of the \feat{subj} \isfeat{subj} attribute and the rest of the \argst list is the value of the \feat{comps} \isfeat{SUBJ} attribute.  \citep{Manning+Sag:1999} proposes a different version of this non-canonical alignment between \argst and valence lists approach to passives. Such differences are not critical in the context of this chapter; not is the representation of stems tentatively assumed by Davis. What matters is that, in contrast to the ``cycling'' argument structure approach to passives illustrated in (\ref{pass-lr}), the \argst list remains unchanged, what is changed is how that list relates to valence lists.

%Several issues remain to be addressed in the non-canonical alignment approach to passives. We simply mention them here and leave a full treatment to some other venue. 

% TD 17 Jan. 20: we need to tie what we say about by-phrases here to what we say above, which will require a bit of editing and maybe another example or two (the \ili{Russian} ones from Perlmutter are clear in showing that the by-phrase --- actually an instrumental case NP --- can bind another argument.

% TD 19 Jan. 20: Slightly revised and augmented text on by-phrases
Finally, we turn to \word{by}-phrases in long passives.
In languages like \ili{English}, a \word{by}-phrase can express the lexeme's logical subject.
Under both the non-canonical linking and non-canonical argument realization analyses, this might be represented as an optional oblique complement on \argst, as indicated in (\ref{pass-lr}) and (\ref{pass-lxm}), respectively.
As noted, the non-canonical argument realization analysis would then posit that the \argst of passives includes two members that correspond to the same argument, which again shows the need for an inert first element of \argst.
Another possibility is to treat such \word{by}-phrases as adjuncts (and therefore not part of the \argst list), see \citet[Chapter 7]{Hoehle78a} and \citet[292--294]{Mueller2003e} for \ili{German} and \citet[180]{Jackendoff1990} for \ili{English}. 
%or not targeting the logical subject per se \citep{Jackendoff1990}.
There is evidence, however, that \word{by}-phrases can serve as antecedents of anaphors in at least some languages.
\citet[111]{Collins2005} cites sentences like (\ref{by-bind}), which suggest that the complement of \word{by}-phrases can bind a reciprocal.

\begin{exe}
\ex\label{by-bind}The packages were sent by the children to each other.
\end{exe} 

\noindent
Acceptability judgements of this and similar examples vary, but they are certainly not outright unacceptable.  Likewise, \citet[10]{Perlmutter1984} furnishes \ili{Russian} examples in which the logical subject (realized as an instrumental case NP) binds a reflexive (note that the \ili{English} translation of it is also fairly acceptable).

\begin{exe}
\ex     \label{arg-st:russian-pass}
\gll Eta kniga byla kuplena Boris-om\added{$_{i}$} dlja sebja\added{$_{i}$}.  \\
     this book was bought Boris-\textsc{ins} for self  \\
\glt `This book was bought by Boris$_{i}$ for himself$_{i}$.'
\end{exe}

\noindent
Given that binding is a relation between members of the \argst list, such data would seem problematic for an approach that does not include \word{by}-phrases on the \argst list.
Interestingly, Perlmutter also argues that \ili{Russian} \word{sebja} is subject-oriented (see \crossrefchapteralt[Section~\ref{binding-sec-passive}]{binding}).
The intrumental NP \word{Borisom} can bind \word{sebja}, only because it corresponds to the subject of active \word{kupit'}, `buy'.
Assuming that is correct, an HPSG account of \ili{Russian} passives would need some means of representing the logical subjecthood of these instrumental NPs; this might involve some way of accessing their active counterpart's \subj value, or of referencing the first element of the passive verb's \argst list, despite its inertness.

The interaction of binding and control with passivization across languages appears to be varied, and as we have noted, we are not aware of systematic investigations into this variation and possible accounts of it within HPSG.
Here, we have surveyed these phenomena and two possible approaches, while noting that some key issues remain unresolved.
Notably, both of these approaches introduce non-canonical lexical items, violating either linking or argument realization constraints that otherwise have strong support. 
Further work is required to assure that these can be preserved in a meaningful way, as opposed to allowing non-canonical structures to appear freely in the lexicon.
% TD 16 Jan. 20: Reworded the following short paragraph to make it more of a summary of the entire section
%Several possibilities suggest themselves here, but, as our goal, was to point out some of the difficulties that remain with theories of passives in HPSG, we will not pursue the matter further in this chapter.

\subsection{Summary}

We have examined in this section several approaches to argument alternations in HPSG and their implications for \argst.
For alternations based on semantic differences, different alternants will have different \feat{content} values, and linking principles like those we outlined in the previous section account for their syntactic differences.
Even where such meaning differences are small, there are differing semantic entailments that can affect linking.
For some cases where there seems to be no discernible meaning difference between alternants, it is still possible for linking principles to yield syntactic differences, if the alternants select different \feat{key} predications in \feat{content}.
The active/passive alternation, however, cannot be accounted for in such a fashion, as it applies to verbs with widely varying \feat{content} values.
HPSG accounts of passives therefore resort to lexical items that are non-canonical, either in their linking or in their mapping between \argst and \changed{valence}.
Both of these are ways of modeling the demotion of the logical subject.
But there is as yet no consensus within the HPSG community on the correct analysis of passives.

%TD 14 Jan. 20: this paragraph will now need to be modified a bit and integrated with the new material
% TD 16 Jan. 20: At this point, I believes the following has already been covered, so I suggest deleting it
%One typical HPSG method for modeling valence alternations like passives is through lexical rules (see \crossrefchapteralt{lexicon}) with one alternant serving as input and the other as output; the main effect of the lexical rule in such an approach is to alter the \argst of the input, going from the \argst of \word{give} to that of \word{given} in (\ref{pasargst}). Critically, we must assume that the output cannot be subject to linking constraint (\ref{act-vb-linking}), since the actor argument is not linked to the first member of the \argst list.
%A simplified representation of what such a rule would look like is provided in (\ref{pass-lr}) where we assume that the input to the rule must be transitive ($\oplus$ indicates list concatenation).

%\begin{exe}
%\ex\label{pass-lr}
%{\avmoptions{center}
%\begin{avm}
%	\[\asort{$\textit{trans-vb}$}
%	head & $\textit{verb}$ \\
%		arg-st & \< \textsc{NP}$_{\@1}$, NP$_ {\@2}$\> $\oplus$ \@3$\textit{list}$
%	\] \,
%	$\mapsto$
%	\[head & \[vform & pass \] \\
%	arg-st & \<NP$_{\@2}$\> $\oplus$\@3 $\oplus$
%	\<PP[$_{by}$]$_{\@1}$\>
%	\]
%	\end{avm}
%}

%\end{exe}


%To sum up, in contrast to most meaning-driven alternations, valence alternations like the active/passive are  modeled through the use of lexical rules that alter the \argst of ``base'' entries. Which alternations pattern with active/passive and require positing lexical rules that alter a ``base'' entry's \argst list is, as of yet, not settled. 
%We turn now to roles that are putatively present semantically, but not realized syntactically at all. 

%JPK 1-18-20: I suggesting removin the above paragraph.
% TD: agreed

\is{passive)}
\is{alternation|(}
\section{Extended \texorpdfstring{\argst}{ARG-ST}}
\label{sec:extended-arg-st}
\is{feature!arg-st@\textsc{agst}!extended|(}

Most of this chapter focuses on cases where semantic roles linked to the \argst list are arguments of the verb's core meaning. But in quite a few cases, complements (or even subjects) of a verb are not part of this basic meaning; consequently, the \argst list must be extended to include elements beyond the basic meaning. We consider three cases here, illustrated in (\ref{fish})--(\ref{adj}).  

Resultatives,\is{resultative} illustrated in (\ref{fish}), express an effect, which is caused by an action of the type denoted by the basic meaning of the verb. The verb \textit{fischen} `to fish' is a simple intransitive verb (\ref{fish}a) that does not entail that any fish were caught, or any other specific effect of the fishing (see \citealt[219--220]{Mueller2002b}).  
\il{German}

\begin{exe}
\ex\label{fish}
\begin{xlist}
\ex
\gll dass er  fischt\\
     that he  fishes\\
\glt `that he is fishing'
\ex 
\gll dass er ihn leer fischt\\
     that he it empty fishes\\
\glt `that he is fishing it empty'
\ex 
\gll wegen der Leerfischung der Nordsee\footnotemark\\
     because.of the empty.fishing of.the.\textsc{gen} North.See \\
\footnotetext{die tageszeitung, 1996-06-20, p.\,6.
}
\glt `because of the North Sea being fished empty'
\end{xlist}
\end{exe}

\noindent
 In (\ref{fish}b) we see a resultative construction\del{,} with an object NP and a \changed{adjectival secondary predicate}.\itdobl{This is not an AP. It is an important part of the analysis that we have predicate complex formation.}  The meaning is that he is fishing, causing it (the body of water) to become empty of fish.  \citet[241]{Mueller2002b} posits a lexical rule \changed{for \ili{German}}\itdopt{Was: ``a German lexical rule''} applying to the verb that augments the \argst list with an NP and AP, and adds the causal semantics to the \content (see \citealt{Wechsler2005result} for a similar analysis of \ili{English} resultatives \added{and \citealt[Section~7.2.3]{MuellerLFGphrasal} for an updated lexical rule and interactions with benefactives}).     The existence of deverbal nouns like \textit{Leerfischung} `fishing empty', which takes the body of water as an argument in genitive case (see (\ref{fish}c)) confirms that the addition of the object is a lexical process, as noted by \citet{Mueller2002b}.  

%any effect or change of state due to the hammering. Since there is evidence th
\ili{Romance} clause-union structures as in (\ref{faire}) have long been analyzed as cases where the complements of the complements\itdopt{Why plural? Usually it is just one complement from which arguments are attracted. It should be: ``where the arguments of a complement of a \ldots{} are arguments of the clause-union verb''} of a clause-union verb (\word{faire} in (\ref{faire})) are complements of the clause-union verb itself \citep{Aissen1979}.
\il{French} 

\begin{exe}
\ex \label{faire}
\gll Johanna a fait manger les enfants. \\
Johanna has made eat the children \\
\glt `Johanna had the children eat.'
\end{exe}

\noindent
Within HPSG, the ``union'' of the two verbs' dependents is modeled via the composition of \argst lists of the \isi{clause union} verb, following \citet{HinrichsandNakazawa1994} (this is a slight simplification; see  \crossrefchapteralt{complex-predicates} for details). 



%JPK 1-14-2020: The following paragraph has been changed

\citet{AbeilleandGodard1997} have argued that many adverbs\is{adverbs as complements} including \word{souvent} in (\ref{adj}) and negative adverbs and negation\itdopt{Something seems to be wrong here. ``and negation particles''?} in \ili{French} are complements of the verb, and \citet{KimandSag2002} extended that view to some uses of negation in \ili{English}. Such analyses hypothesize that some semantic modifiers are realized as complements, and thus should be added as members of \argst (or members of the \deps list, if one countenances such an additional list; see below). In contrast to resultatives,\is{resultative} which affect the meaning of the verb, or to clause union, where one verb co-opts the argument structure of another verb, what is added to the \argst list in these cases is typically considered a semantic adjunct and a modifier in HPSG (thus it selects the verb or VP via the \feat{mod} attribute). 
\il{French}

\begin{exe}
\ex\label{adj}
 \gll
	Mes amis m’ ont souvent aidé. \\
	my friends me have often helped \\
	\glt `My friends often helped me.'
\end{exe}


Another case of an adjunct that behaves like a complement is found in (\ref{mourir}), taken from \citep[81]{KoenigandDavis2006}.  The clitic \word{en} expressing the cause of death   is not normally an argument of the verb \textit{mourir} `die', but rather an adjunct: % discuss examples such as (\ref{mourir}).

\ea
\label{mourir}
\gll Il en est mort. \\
     he of.it is dead.\textsc{pfv.pst} \\
\glt `He died of it.'
\z

\noindent
On the widespread assumption (at least within HPSG) that pronominal clitics are verbal affixes \citep{MillerandSag1997}, the adjunct cause of the verb \word{mourir} must be represented within the entry for \word{mourir}, so as to trigger affixation by \word{en}. \citet{Boumaetal2001} discuss cases where ``adverbials'', as they call them, can be part of a verb's lexical entry. To avoid mixing those adverbials with the argument structure list (and having to address their relative obliqueness with syntactic arguments of verbs), they introduce  an additional list, the dependents list (abbreviated as \deps) which includes the \argst list but also a list of adverbials. Each adverbial selects for the verb on whose \deps list it appears as a\rep{n argument}{ dependent}, as shown in (\ref{deps}). But, of course, not all verb modifiers can be part of the \deps list,\footnote{%
\added{See \citew[Section~20.4.1]{Mueller99a} and \crossrefchapterw[Section~\ref{sec-nominal-heads-as-binders}]{binding} for binding conflicts arising when adjuncts within the nominal domain are treated this way.}
} and Bouma, Malouf, and Sag discuss at length some of the differences between the two kinds of ``adverbials''.

\begin{exe}
\ex\label{deps}
\emph{verb} \impl
\avm{
	[cont|key & \2 \\
	head & \3 \\
	deps & \1 \+ list \upshape !(! [mod &	[head & \3 \\
											key & \2 ] ] !)! \\
	arg-st & \1 ]
}
\itdopt{CONT above HEAD? Renumber top down left right?}
\end{exe}

Although the three cases we have outlined result in an extended \argst, the ways in which this extension arises differ. In the case of resultatives,\is{resultative} the extension results partly or wholly from changing the meaning in a way similar to \citet{RappaportandLevin1998}: by adding a causal relation, the effect argument of this causal relation is added to the membership in the base \argst list (see Section~\ref{alternations} for a definition of the attributes \feat{key} and \feat{rels}; here it suffices to note that a \type{cause-rel} is added to the list of relations that are the input of the rule). 
%% ALERT:  is this about 51? if so then I don't quite follow how 51 works.  
%JPK 1-29-20: The causal relation is added to the list of RELS. It's flat (MRS), but isn't that what is needed for resultatives? 51 is oversimplified in many ways,  but it seems to me to get the idea across. Am I missing something.
%SW 1-29-20:   51 seems a little too vague.  e.g. it could represent resultative or causative. 
%JPK 1-31-20: I agree, but I am not sure it is worth going into more details. You think we should? 
The entries of the \isi{clause union} verbs are simply stipulated to include on their \argst lists the syntatic arguments of their (lexical) verbal arguments in (\ref{faire-ent}).
%TD 13 Jan. 20: reworded the definition of shuffle
%SW: jan 28. sorry, I rereworded it.  they dont have to be 'intermixed' do they?  ALERT
% TD 28 Jan. 20: Thanks; that's better
The symbol $\bigcirc$\is{$\bigcirc$} in this rule is known as ``shuffle''\is{shuffle}; it represents any list 
% in which the elements of the two lists are intermixed
containing the combined elements of the two lists, but with the relative ordering of elements on each list preserved.\footnote{
\added{See \crossrefchapterw[\pageref{order:rel-shuffle}]{order} for more on the shuffle operator.}}
In (\ref{faire-ent}), shuffling the members of the verbal complement's \argst list with those of the main verb's \argst list allows us \add{to} represent a possible reordering of them (for example in passivization -- depending on one's approach to passivization\itdopt{but didn't you just say that you assume a lexical rule for passive? I am afraid that the shuffling admits too much freedom.} in HPSG; see \crossrefchaptert{complex-predicates}).
%JPK 13 Jan 2020: Added a definition of the shuffle operation
Finally, (negative) adverbs that select for a verb (VP) are added to the \argst of the verb they select. A simplified representation of all three processes is provided in (\ref{res-lr})--(\ref{neg-lr}). 


\begin{exe}
	\ex\label{res-lr}
\avm{
	[key & \2 \\
	rels & \1 \changed{<\ldots, \2, \ldots >} ]
}
	$\mapsto$
\avm{
	[key & \3 cause-rel \\
	rels & \1 \+ \3 ]
}
\end{exe}

\begin{exe}
\ex\label{faire-ent}
\avm{
	[arg-st & <\ldots, [head   & verb \\
			    arg-st & \1 ] > \shuffle \1 ]
}
\end{exe} 

\begin{exe}
	\ex\label{neg-lr}
\avm{
	[arg-st & \1 ]
}
	$\mapsto$
\avm{
	[arg-st & \1 \shuffle < Adv$_{neg}$ > ]
}
\end{exe}
\is{feature!arg-st@\textsc{agst}!extended|)}
\itdopt{Can you please move the examples to where they are discussed? I am afraid of page breaks which makes stuff difficult to read. Examples should be where they are discussed. I added commas in (\ref{res-lr}).}
 

\section{Is  \texorpdfstring{\argst}{ARG-ST} universal?}

% TD 31 Jan. 20: I have not examined the new version of this section in detail, but I suggest something like the following approach to this material (which might enable us to make it more concise):
% TD 31 Jan. 20: start with somthing like the following (commented out at the moment:
% We have examined several functions of \argst thus far, motivated by evidence drawn from arange of languages.
% It is an ordered list of arguments, indexed both to roles in \feat{content} and to elements of \changed{valence} lists.
% Oredering on the \argst list is determined both by a predicator's subcategorization requirements and by the semantic properties (entailments) of its roles.
% But as extraction, pro-drop, null anaphora, and voice alternations show, the relationship between semantic roles and \changed{valence} lists is complex and \argst provides the needed interface.
% In articular, binding principles are stated with respect to ordering on \argst.
%
% If there are languages in which none of the foregoing kinds of evidence are to be found, the question of whether \argst should be posited in such language arise.
% This is in large part a methodological question, the answer to which is driven by one's stance on the relative importance of positing universal features across languages, even when there is no evidence supporting them in a particular language.
% Here, we review one case of this nature.

% TD 31 Jan. 20: Then, very briefly explain why there's no vidence supporting these functions of \argst in \ili{Oneida}.  I'd say it's less important to introduce IMFL-STR than to present the idea that \argst isn't motivated.  I don't know if this will shorten the presentation or not, and I'm suggesting this just to rethink the presentation a bit, so if it doesn't strike you as the right way to go, just forget what I've writen here.

HPSG's \argst attribute does not seem to be a universal property of natural
language grammars.  The \argst feature is the intermediary between, on
the one hand, a semantic representation of an event or state in which
participants fill specific roles, and on the other, their syntactic and
morphological expression.  \argst is defined as a list of \type{synsem}
objects in the entry for a verb lexeme, and is used to model the following grammatical regularities of particular predicators or sets of predicators:

\begin{itemize}
\item The verb selects grammatical features such as part of speech
category, case marking, and preposition forms of its dependent
phrases.
\item \argst list items:
\begin{enumerate}
\item are \rep{unified}{identified} with \changed{valence} list items
representing grammatical properties of phrasal dependents (subject and
complements),
\item determine verbal morphology, or
\item are left unexpressed.
\end{enumerate}
\item The different contexts of occurrence of a verb correspond to distinct \argst values.
\item The inflected verb can indicate agreement features on both phrasal
and affixal arguments.
\item Binding conditions on arguments are defined on \argst, making
crucial use of the ordering relation (obliqueness).
\itdopt{Here we have the problem with ``phrasal'' again. Maybe a footnote early on.}
\end{itemize}

\noindent
\citet{KoenigandMichelson2014,KoenigandMichelson2015a,KoenigandMichelson2015b} argue that the grammatical encoding of semantic arguments in \ili{Oneida} (Northern \ili{Iroquoian}) \il{Oneida} does not display any of these properties.  In fact, the only function of the
corresponding intermediate representation in \ili{Oneida} is to distinguish
the arguments of a verb for the purpose of determining verbal
prefixes indicating semantic person, number, and gender
features of animate arguments.  
For example,  the prefix \word{lak-} occurs if a third-person singular masculine proto-agent argument is acting on a first-person singular proto-patient argument as in \word{lak-hlo·lí-he\textipa{P}}  `he tells me’ (habitual aspect), whereas  the prefix \word{li-} occurs if a first singular proto-agent argument is acting on a third masculine singular argument, as in \word{li-hlo·lí-he\textipa{P}} `I tell him’ (habitual aspect).
 As there is no syntactic agreement, these verbal prefixes encode purely semantic features.
\changed{\type{synsem}} objects are therefore not appropriate for this intermediate
representation; all that is needed are semantic argument indices to
distinguish between (a maximum of two) animate co-arguments
distinguished for fixed argument roles for each verb.
Koenig \& Michelson use the attribute \feat{infl-str},\isfeat{infl-str} which is a list of referential indices for animate arguments, within the morphological information associated with each verb (within the value of \feat{morph}; see \crossrefchapteralt[Section~\ref{sec:IbM}]{morphology})
\itdopt{Do you mean \textsc{mph} or \textsc{infl}? I refer to the IbM section now. Section~\ref{sec:IbM}.} for this highly restricted
\ili{Oneida} feature that replaces \argst.\itdopt{The preceding sentence is difficult to parse. Especially with this insertion in the middle.} If Koenig \& Michelson are correct, the \argst list may thus not be a universal attribute of words, though present in the overwhelming majority of languages.
Linking, understood as constraints between semantic roles and members of the \argst list, is then but one possibility; constraints that relate semantic roles to an \feat{infl-str} list of semantic indices is also an option.
In languages that exclusively exploit that latter possibility, syntax is indeed simpler.



\section{The lexical approach to argument structure}
\label{lexicalapproach}

We end this chapter with a necessarily brief comparison between the approach to argument structure we have described here and other approaches to argument structure that have developed since the 1990s.
This chapter describes a \textit{lexical approach to argument structure}, which is typical of research in HPSG.  The basic tenet of such approaches is that lexical items include argument structures, which represent essential information about potential argument selection and expression, but
abstract away from the actual local phrasal structure.  In contrast, \emph{phrasal approaches}, which
are common both in Construction Grammar\is{Construction Grammar} and in transformational approaches such as Distributed Morphology,\is{Distributed Morphology} reject such lexical argument structures.   Let us briefly review the reasons for preferring a lexical approach. (This section is drawn from \citealt{MWArgSt}, which may be consulted for more detailed and extensive argumentation. \added{See also \crossrefchapteralt[Section~\ref{cxg:sec-valence-vs-phrasal-patterns}]{cxg}.}) 

In phrasal approaches\is{argument realization!phrasal approaches to|(} to argument structure, components of a verb's apparent meaning are actually ``constructional meaning'' contributed directly by the phrasal structure.  The linking constraints of the sort discussed above are then said to arise from the interaction of the verb meaning with the constructional meaning.  For example, agentive arguments tend to be realized as subjects, not objects, of transitive verbs.  On the theory presented above, that generalization is captured by the linking constraint (\ref{act-vb-linking}), which states that the \textsc{actor} argument of an \textit{act-und-rel} (actor-undergoer relation) is mapped to the initial item in the \argst list.  In a phrasal approach, the agentive semantics is directly associated with the subject position in the phrase structure.  In transformational theories, a silent ``light verb'' (usually called ``little \textit{v}'') heads a projection in the phrase structure and assigns the agent role to its specifier (the subject).  In constructional theories, the phrase structure itself assigns the agent role.  In either type of phrasal approach, the agentive component of the verb meaning is actually expressed by the phrasal structure into which the verb is inserted.  


The lexicalist’s approach to argument structure provides essential information for a verb's potential
combination with argument phrases.   If a given lexical entry could only  combine with the particular set of phrases specified in a single \changed{valence} feature,\itdopt{This probably needs some attention.} then the lexical and phrasal approaches would be difficult to distinguish: whatever information the lexicalist specifies for each \changed{valence} list item could, on the phrasal view, be specified instead for the phrases realizing those list items.  
But crucially, the verb need not immediately combine with its specified
arguments.  Alternatively, it can meet other fates: it can serve as the input to a lexical rule; it
can combine first with a modifier in an adjunction structure; it can be coordinated with another
word with the same predicate argument structure; instead of being realized locally, one or more of
its arguments can be effectively transferred to another head’s valence feature (raising or argument
transfer); or arguments can be saved for expression in some other syntactic position (partial
fronting).\footnote{
  \added{See \crossrefchapteralt[Section~\ref{cxg:sec-partial-verb-phrase-fronting}]{cxg} for discussion of partial verb phrase fronting.}
} Here we consider two of these, lexical rules and coordination.  
 
The lexically encoded argument structure is abstract: it does not directly encode the phrase structure or
precedence relations between this verb and its arguments. This abstraction captures the commonality
across different syntactic expressions of the arguments of a given root.

\begin{exe}
\ex \label{nibble}
\begin{xlist}
\ex  The rabbits were nibbling the carrots.  
\ex  The carrots were being nibbled (by the rabbits).
\ex  a large, partly nibbled, orange carrot 
\ex  the quiet, nibbling, old rabbits
\ex  the rabbit's nibbling of the carrots
\ex  The rabbit gave the carrot a nibble.  
\ex  The rabbit wants a nibble (on the carrot).  
\ex  The rabbit nibbled the carrot smooth.
\end{xlist}
\end{exe}

\noindent
Verbs undergo  morpholexical operations like passive
(\ref{nibble}b), as well as antipassive, causative, and applicative in other languages.  They have cognates in
other parts of speech such as adjectives  (\ref{nibble}c, d) and nouns  (\ref{nibble}e, f, g).  
Verbs have been argued to form complex predicates with resultative secondary predicates (\ref{nibble}h), and with serial verbs in other languages.\is{resultative}   

The same root lexical entry \emph{nibble}, with the same meaning, appears in all of these contexts.
The effects of lexical rules together with the rules of syntax dictate the proper argument
expression in each context.  For example, if we call the first two arguments in an \argst list
(such as the one in (\ref{nibble}) above) Arg1 and Arg2 (or \feat{act} or \feat{und}), respectively, then in an active
transitive sentence Arg1 is the subject and Arg2 the object; in the passive, Arg2 is the subject and
the referential index of Arg1 is optionally assigned to a \emph{by}-phrase.  The same rules of
syntax dictate the position of the subject, whether the verb is active or passive.  When adjectives
are derived from verbal participles, whether active (\emph{a nibbling rabbit}) or passive (\emph{a
  nibbled carrot}), the rule is that whichever role would have been expressed as the subject of the
verb is assigned by the participial adjective to the referent of the noun that it modifies; see \citet{Bresnan:1982passive}\addpages
and \citet[Chapter~3]{Bresnan+etal:2015}.  
The phrasal approach, in which the agent role is assigned to the subject position, is too rigid.  

Nor\itdopt{The ``nor'' comes out of the blue.} could this be solved by associating each syntactic environment with a different meaningful phrasal construction: an active construction with agent role in the subject position, a passive construction with agent in the \textit{by}-phrase position, etc.  The problem for that view is that one lexical rule can feed another.  In the example above, the output of the verbal passive rule (see (\ref{nibble}d)) feeds the adjective formation rule (see (\ref{nibble}e)).  
 
A verb can also be coordinated with another verb with the same valence requirements.  The two verbs then share their dependents.  This causes problems for the phrasal view, especially when a given dependent receives different semantic roles from the two verbs.  For example, in an influential phrasal analysis, \citet{hale+keyser:1993}
derived denominal verbs like \textit{to saddle} through noun incorporation out of a structure akin to
[PUT a saddle ON x].  Verbs with this putative derivation routinely coordinate and share
dependents with verbs of other types: 

\begin{exe}
\ex Realizing the dire results of such a capture and that he was the only one to prevent it, he quickly
[saddled and mounted] his trusted horse and with a grim determination began a journey that would
become legendary.\footnote{\url{http://www.jouetthouse.org/index.php?option=com_content&view=article&id=56&Itemid=63},
  21.07.2012}  
\end{exe}

\noindent
Under the phrasal analysis, the two verbs place contradictory demands on a single phrase structure.  But on the lexical analysis, this is simple V$^0$ coordination.\footnote{%
\added{See also \crossrefchapterw[Section~\ref{coordination:sec-lexical-coordination}]{coordination} on lexical coordination and for arguments why approaches assuming phrasal coordination for these cases fail.}
}
\is{argument realization!phrasal approaches to|)} 

\section{\added{Conclusion}}
 
To summarize, a lexical argument structure is an abstraction or generalization over various occurrences of the verb in syntactic contexts. To be sure, one key use of that argument structure is simply to indicate what sort of phrases\itdopt{phrases?} the verb must (or can) combine with, and the result of semantic composition\itdopt{I do not understand what ``and the result of semantic composition'' is supposed to mean here.}; if that were the whole story, then the phrasal theory would be viable. But it is not. As it turns out, lexically-encoded valence structure, once abstracted, can alternatively be used in other ways: among other possibilities, the verb (crucially including its valence structure) can be coordinated with other verbs that have a similar valence structure, or it can serve as the input to lexical rules specifying a new word \added{or lexeme} bearing a systematic relation to the input word.   The phrasal approach prematurely commits to a single phrasal position for the realization of 
a semantic argument.  In contrast, a lexical argument structure gives a word the appropriate flexibility to account for the full range of expressions found in natural language.   
 

%\section*{Abbreviations}
\section*{Acknowledgements}

\itdopt{add something}

%
{\sloppy
\printbibliography[heading=subbibliography,notkeyword=this] 
}
\end{document}


%      <!-- Local IspellDict: en_US-w_accents -->
