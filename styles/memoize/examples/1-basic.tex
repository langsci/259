\documentclass{article}
\usepackage{tikz}
\usetikzlibrary{positioning}
\usepackage{forest}

\usepackage{memoize}            % that's all folks!

\begin{document}

The use the package, just load it. By default, memoization is enabled and TikZ
pictures and Forest trees will be ``automemoized.''

\begin{figure}\centering
  \begin{tikzpicture}
    \node(atomic)[align=center]{atomically\\restricted};
    \node(strong)[right=4em of atomic,align=center]{strongly\\restricted};
    \node(weak)[right=3em of strong,align=center]{weakly\\restricted};
    \node(atomic frame)[gray,fit=(atomic),inner xsep=1.5em,xshift=1em,inner ysep=0.333em,draw]{};
    \node(strong frame)[gray,fit=(atomic frame)(strong),inner xsep=1.5em,inner ysep=1em,draw]{};
    \node(weak frame)[gray,fit=(strong frame)(weak),inner xsep=1.5em,inner ysep=1em,draw]{};
    \node(cons)[below=1em of weak frame]{conservative};
    \node(cons frame)[gray,fit=(weak frame)(cons),inner xsep=1.5em,inner ysep=1em,draw]{};
    \path (atomic.east) + (1em,0) node(atomic x)[circle,fill,inner sep=1pt]{};
    \path (atomic x |- cons) node(cons x)[circle,fill,inner sep=1pt]{};
    \draw [->] (cons x) -- node[right,pos=0.4]{$\sim$} (atomic x);
  \end{tikzpicture}
  \caption{Variants of restrictiveness}
  \label{fig:rest-and-cons}
\end{figure}

\begin{center}
  \begin{forest}
    [VP
      [DP]
      [V\rlap'
        [V]
        [DP]
      ]
    ]
  \end{forest}
\end{center}

If your {\tt -shell-escape} option (perhaps called {\tt--enable-write18}) is
enabled (if it operates in restricted mode, {\tt pdflatex} must be allowed),
things should ``just work.''

\begin{enumerate}
\item After the first compilation, you should get a document containing three
  pages: the tikzpicture, the forest tree and the document itself.
\item After the second compilation, the first two pages should disappear. You
  should find the \emph{memos} in files
  \begin{itemize}
  \item {\tt 1-basic.5809A894D3808C95F53EEE848F9C3BBA.memo.pdf} and
  \item {\tt 1-basic.B2E679FB208DD0D53C20B5017B4A8DAA.memo.pdf}.
  \end{itemize}

\end{enumerate}

The long numbers in memo names are the md5sums of the code that produced
them. Try changing the tikz or forest code and you will get a new memo file
after two passes.

\bigskip

\emph{If something goes wrong, try deleting the memos --- this is always safe
  to do --- and recompiling.}


\end{document}

%%% Local Variables:
%%% mode: latex
%%% TeX-master: t
%%% End:
